%\chapter{Inferential measurement}
 
%More often than not, we cannot measure a desired variable directly, but must instead \textit{infer} its value from some other measurement.  The measurement of fluid flow is a classic example of inferential measurement, where some related variable is the one actually being sensed, with flow rate being a calculated function of that primary variable.  To list a few examples:   \index{Inferred variable}

%\begin{itemize}
%\item Orifice plate flow measurement: \textit{Primary variable is actually differential pressure, which is related to flow rate by a quadratic function}
%\item Vortex shedding flow measurement: \textit{Primary variable is actually the frequency of the shed vortices, being directly proportional to fluid velocity and thus volumetric flow rate}
%\item Magnetic flow measurement: \textit{Primary variable is actually a voltage, being directly proportional to the product of magnetic flux density and fluid velocity}
%\item Thermal (mass) flow measurement: \textit{Primary variable is actually temperature, which is related to mass flow rate by a nonlinear function}
%\item Coriolis (mass) flow measurement: \textit{Primary variable is actually phase shift between two sinusoidal vibrations, which is a function of mass flow rate}
%\end{itemize}

%Perhaps the only \textit{non-}inferential form of flow measurement is positive displacement, where each cycle of the flowmeter mechanism directly represents a fixed volume (displacement) of fluid.  In fact, one could argue that there are additional levels of inference in each of the listed flowmeter technologies besides the one level described in each example.  Orifice plate flow measurement, for example, is based on the measurement of differential pressure, which in turn is often based on \textit{differential capacitance} or \textit{piezoresistive} sensor technology, sensing fluid pressure by changes in electrical quantities related to that pressure by the spring constant of a sensing diaphragm.  Very few quantities are measured \textit{directly} in the industrial world!

%It is important, therefore, to clearly define what we mean by an ``inferential'' measurement, if nearly everything measured in industry is done so indirectly.  For the purposes of this chapter, I will define the term ``inferential'' to refer to some calculated physical quantity in a process \textit{for which there is no dedicated sensing device}.  In other words, an ``inferential'' variable within the scope of this chapter is one for which no sensing device exists.

%Often, a process variable is inferred for the simple reason that no suitable sensing device exists to measure it.  An economic example of such a variable is the \textit{consumer confidence index} (CCI) in the United States, which is abstract at best.  Economists may directly measure retail sales, income levels, gross domestic product (GDP), and other such variables, but no technology (yet!) exists to directly measure how confident consumers are in an economy.  The CCI, therefore, is an example of an inferred variable.  \index{Consumer Confidence Index (CCI)}

%\vskip 10pt







%\filbreak
%\section{Simple examples}

% ADD: inferring flow rate through (unmeasured) pipe by subtracting other flows in vs. out
% ADD: inferring wet-bulb temperature in a dryer system from dry-bulb measurements ("Process Control Systems" by Shinskey, pp. 323-325)








%\filbreak
%\section{Complex examples}

% ADD: dead+lag time computation on powdered coal feeder system to infer flow rate of powdered coal to burners (SINST i01818.tex, "Energy Conservation Through Control" by Shinskey pg. 47)









%\filbreak
%\section*{References}

% In alphabetical order!
% \noindent
% Lastname, Firstname MiddleI., \textit{Book Title}, Publisher, City, State, Year.
% \vskip 10pt
% \noindent
% Lastname, Firstname MiddleI., \textit{Book Title}, Publisher, City, State, Year.
% etc . . .

%\noindent
%Lipt\'ak, B\'ela G. et al., \textit{Instrument Engineers' Handbook -- Process Control Volume II}, Third Edition, CRC Press, Boca Raton, FL, 1999.

%\vskip 10pt

%\noindent
%Shinskey, Francis G., \textit{Energy Conservation through Control}, Academic Press, New York, NY, 1978. 

%\vskip 10pt

%\noindent
%Shinskey, Francis G., \textit{Process-Control Systems -- Application / Design / Adjustment}, Second Edition, McGraw-Hill Book Company, New York, NY, 1979. 
















%%%%%%%%%%%%%%%%%%%%%%%%%%%%%%%%%%%%%%%%%%%%%%%%%%%%

