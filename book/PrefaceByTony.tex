\chapter*{Preface for original liii by Tony R. Kuphaldt}
\addcontentsline{toc}{chapter}{Preface}

I did not want to write this book . . . honestly.  

\vskip 10pt

\noindent
My first book project began in 1998, titled \textit{Lessons In Electric Circuits}, and I didn't call ``quit'' until six volumes and five years later.  Even then it was not complete, but being an open-source project it gained traction on the Internet to the point where other people took over its development and it grew fine without me.  The impetus for writing this first tome was a general dissatisfaction with available electronics textbooks.  Plenty of textbooks exist to describe things, but few really \textit{explain} things well for students, and the field of electronics is no exception.  I wanted my book(s) to be different, and so they were.  No one told me how time-consuming it was going to be to write them, though!

The next few years' worth of my spare time went to developing a set of question-and-answer worksheets designed to teach electronics theory in a Socratic, active-engagement style.  This project proved quite successful in my professional life as an instructor of electronics.  In the summer of 2006, my job changed from teaching electronics to teaching industrial instrumentation, and I decided to continue the Socratic mode of instruction with another set of question-and-answer worksheets.

However, the field of industrial instrumentation is not as well-represented as general electronics, and thus the array of available textbooks is not as vast.  I began to re-discover the drudgery of trying to teach with inadequate texts as source material.  The basis of my active teaching style was that students would spend time researching the material on their own, then engage in Socratic-style discussion with me on the subject matter when they arrived for class.  This teaching technique functions in direct proportion to the quality and quantity of the research sources at the students' disposal.  Despite much searching, I was unable to find a textbook adequately addressing my students' learning needs.  Many textbooks I found were written in a shallow, ``math-phobic'' style well below the level I intended to teach to.  Some reference books I found contained great information, but were often written for degreed engineers with lots of Laplace transforms and other mathematical techniques well above the level I intended to teach to.  Few on either side of the spectrum actually made an effort to explain certain concepts students generally struggle to understand.  I needed a text giving good, practical information and theoretical coverage at the same time.

In a futile effort to provide my students with enough information to study outside of class, I scoured the Internet for free tutorials written by others.  While some manufacturer's tutorials were nearly perfect for my needs, others were just as shallow as the textbooks I had found, and/or were little more than sales brochures.  I found myself starting to write my own tutorials on specific topics to ``plug the gaps,'' but then another problem arose: it became troublesome for students to navigate through dozens of tutorials in an effort to find the information they needed in their studies.  What my students really needed was a \textit{book}, not a smorgasbord of tutorials.

\vskip 10pt

So here I am again, writing another textbook.  This time around I have the advantage of wisdom gained from the first textbook project.  For this project, I will \textit{not}:

\begin{itemize}
\item . . . attempt to maintain a parallel book in HTML markup (for direct viewing on the Internet).  I had to go to the trouble of inventing my own quasi-XML markup language last time in an effort to generate multiple format versions of the book from the same source code.  Instead, this time I will use stock \LaTeX{} as the source code format and regular Adobe PDF format for the final output, which anyone may read thanks to its ubiquity.  If anyone else desires the book in a different format, I will gladly let them deal with issues of source code translation.  Not that this should be a terrible problem for anyone technically competent in markup languages, as \LaTeX{} source is rather easy to work with. 
\item . . . use a GNU GPL-style copyleft license.  Instead, I will use the Creative Commons Attribution-only license, which is far more permissive for anyone wishing to incorporate my work into derivative works.  My interest is maximum flexibility for those who may adapt my material to their own needs, not the imposition of certain philosophical ideals.
\item . . . start from a conceptual state of ``ground zero.''  I will assume the reader has certain familiarity with electronics and mathematics, which I will build on.  If a reader finds they need to learn more about electronics, they should go read \textit{Lessons In Electric Circuits}.
\item . . . avoid using calculus to help explain certain concepts.  Not all my readers will understand these parts, and so I will be sure to explain what I can without using calculus.  However, I want to give my more mathematically adept students an opportunity to see the power of calculus applied to instrumentation where appropriate.  By occasionally applying calculus and explaining my steps, I also hope this text will serve as a practical guide for students who might wish to learn calculus, so they can see its utility and function in a context that interests them.
\end{itemize}

There do exist many fine references on the subject of industrial instrumentation.  I only wish I could condense their best parts into a single volume for my students.  Being able to do so would certainly save me from having to write my own!  Listed here are some of the best books I can recommend for those wishing to explore instrumentation outside of my own presentation:

\begin{itemize}
\item \textit{Instrument Engineers' Handbook} series (Volumes I, II, and III), edited by B\'ela Lipt\'ak -- by far my favorite modern references on the subject.  Unfortunately, there is a fair amount of material within that lies well beyond my students' grasp (Laplace transforms, etc.), and the volumes are incredibly bulky and expensive (nearly 2000 pages, and at a cost of nearly \$200.00, \textit{apiece!}).  These texts also lack some of the basic content my students do need, and I don't have the heart to tell them to buy yet \textit{another} textbook to fill the gaps.  \index{Lipt\'ak, B\'ela}
\item \textit{Handbook of Instrumentation and Controls}, by Howard P. Kallen.  Perhaps the best-written textbook on general instrumentation I have ever encountered.  Too bad it is both long out of print -- my copy dates 1961 -- and technologically dated.  Like most American textbooks written during the years immediately following Sputnik, it is a masterpiece of practical content and conceptual clarity.  I consider books like this useful for their presentations of ``first principles,'' which of course are timeless.  \index{Kallen, Howard P.}
\item \textit{Industrial Instrumentation Fundamentals}, by Austin E. Fribance.  Another great post-Sputnik textbook -- my copy dates 1962.  \index{Fribance, Austin E.}
\item \textit{Instrumentation for Process Measurement and Control}, by Norman A. Anderson.  An inspiring effort by someone who knows the art of teaching as well as the craft of instrumentation.  Too bad the content doesn't seem to have been updated since 1980.  \index{Anderson, Norman A.}
\item \textit{Applied Instrumentation in the Process Industries} (Volume I), edited by William G. Andrew.  A very clear and fairly comprehensive overview of industrial instrumentation.  Sadly, this fine book is out of print, and much of the material is dated (second edition written in 1979). \index{Andrew, William G}
\item Practically anything written by Francis Greg Shinskey.  \index{Shinskey, Francis Greg}
\end{itemize}

Whether or not I achieve my goal of writing a better textbook is a judgment left for others to make.  One decided advantage my book will have over all the others is its \textit{openness}.  If you don't like anything you see in these pages, you have the right to modify it to your liking!  Delete content, add content, modify content -- it's all fair game thanks to the Creative Commons licensing.  My only conditions are declared in the license: you must give me credit for my original authorship, include a copyright notice, and also include the full text of the Creative Commons license informing readers of their rights (see Section 3(a)(1) of the License for more detail).  What you do with it beyond that is wholly up to you\footnote{This includes selling copies of it, either electronic or print.  Of course, you must include the Creative Commons license as part of the text you sell, which means anyone will be able to tell it is an open text and can probably figure out how to download an electronic copy off the Internet for free.  The only way you're going to make significant money selling this text is to add your own value to it, either in the form of expansions or bundled product (e.g. simulation software, learning exercises, etc.), which of course is perfectly fair -- you must profit from your \textit{own} labors.  All my work does for you is give you a starting point.}.  This way, perhaps I can spare someone else from having to write their own textbook from scratch!




