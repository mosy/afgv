
\section {Utførelse av arbeidsoppdrag}
%Måleteknikk 
\hrule

\vskip 1cm

I 3AUA utføres praktisk arbeid som arbeidsoppdrag, disse skal bestå av følgende:\begin{itemize}[noitemsep]
	\item Planlegging
	\item Gjennomføring
	\item Dokumentasjon
\end{itemize}

\subsection {Planlegging}

Planleggingen av arbeidsoppdragene har som hensikt å:

\begin{itemize}[noitemsep]
	\item Sørge for at vi har kunnskap nok til å starte opp arbeidet. 
	\item Tenke igjennom hva arbeidet består av og i hvilken rekkefølge det er hensiktmessig å utføre det. 
	\item Sørge for av vi har alt nøvendig utstyr og materiell. 
\end{itemize}



Alt arbeid skal risikovurderes før utførelse. Nødvendig sikringstiltak skal planlegges og utføres i henhold til bedriftens HMS-regelverk. Eksempelvis skal all energitilførsel stenges og sikres mot utilsiktet innkopling, hvis mulig. Hvis ikke, skal arbeidene planlegges og utføres under gjeldende sikringstiltak, eksempelvis med AUS verktøy isolerende matter, etc. Husk spesielt personlig verneutstyr, så som briller, hjelm, hansker, sko, klær, hørselvern, etc.

Planleggingsdlen skal innholde:

\textbf{Innledning:}
\vskip 10pt 
\vskip 10pt 
\textbf{Fremdriftsplan:}
\vskip 10pt 
Skal vise at du forstår hva som inngår av arbeid i de ulike oppdragene og at du forstår i hvilken rekkefølge de må utføres. Denne er også et greit utgangpunkt for å hvilke deler som sakl være med i en risikovurdering. 


\vskip 10pt 
SJA skal utføres om et arbeid ikke er kjent eller det finnes rutine for det. Som elev i et fag vil det si at i de flese arbeidsoppdrag må det utføres en SJA. 
\vskip 10pt 
\textbf{Verktøysliste:}
\vskip 10pt 
Skal vise at du vet hvilket verktøy som trengs til de ulike jobbene. Å bruke korrekt navn er en viktig del. 

\vskip 10pt 
\textbf{Utstyrsliste:}
\vskip 10pt 
Skal vise at du vet hvilket utstyr som trengs til de ulike jobbene. Å bruke korrekt navn er en viktig del. Utstyrslisten skal inneholde det som du ville tatt betalt for av en kunde. 

\vskip 10pt 
\textbf{HMS Risikovurdering:}
\vskip 10pt 
Skal vise at du kan vurdere farer med arbeidet og sette inn tiltakt for å unngå farene. FSE er sentral i forhold til farer med elektrisitet. 

\vskip 5pt 
\vskip 10pt 
\textbf{Teori, Instrumenter og utstyr:}
\vskip 10pt 
\vskip 5pt 
Her legger du inn forklaring for virkemåte og rolle i anlegget for valgte instrumenter og utstyr. Du forklarer også sentral teori for anlegget.

\vskip 5pt 
\textbf{Koblingstegninger:}

Her legger du ved skjema for alle oppkoblinger du planlegger. 

\vskip 10pt 
\vskip 10pt 
\textbf{Lover, forskrifter og normer:}
\vskip 10pt 
Her beskriver du hvilke lover, forskrifter og normer som er relevante for arbeidet. 
\subsection{Gjennomføring}

Enten arbeidsoppdraget skal utføres i praksis eller det bare er en del av en prøve skal gjennomføringenen beskrives. Dette er viktig del øvelsen til å besvare eksamensoppgaver. 
\vskip 5pt 
Om du har utført arbeidet kan det være lurt å beskrive hvordan du ville utført arbeidet om du skulle gjort det en gang til. 
\subsection{Dokumentasjon}

\vskip 5pt 
Alle arbeidsoppdrag skal inneholde relevant dokumentajson i forhold til arbeidet som er gjort. Det kan f.eks. være et prosjekt der du må levere komplett dokumentajson. Et annet eksempel kan være kalibrering av en transmitter der en kort arbeidsrapport og et kalibreringsskjema vil være relevant. 

\subsection{Oversikt over laboppgaver i 3AUA}

Laboppgaver kan utføres i tilfeldig rekkefølge gjennom året. Du må selv legge en plan for dette. Dette skal foregå parallell med at du jobber med arbeidsoppdrag. Det skal skrives rapport for hver laboppgave.  
\begin{center}
	\begin{tabular}{| m{5cm} |m{2cm} |m{2cm} |m{1cm} |} 
\hline
	\multicolumn{4}{|c|}{\textbf{\cellcolor[HTML]{D5D5D5}Oversikt over laboppgaver}} \\
\hline
\hline
\rowcolor [HTML]{D5D5D5}
Fil	&Stasjon&Emne&Utført\\ \hline
		Profinet tutorial & i04834.tex & & \\ \cline{1-4}
		Vision med robot & i04862.tex & & \\ \cline{1-4}
		Målesystermer for nivå & lStasjon09.tex & & \\ \cline{1-4}
		Målesystemer for trykk & i04842.tex &  &\\ \cline{1-4}
		Målesystemer for temperatur & i04841.tex &  &\\ \cline{1-4}
		Strømningsregulering & lStasjon13.tex &  &\\ \cline{1-4}
		Sjekk av kommunikasjonsnett & lStasjon14.tex &  &\\ \cline{1-4}
		Digitale målesystemer & lStasjon15.tex &  &\\ \cline{1-4}
		Sikkerhetsrele & lStasjon16.tex &  &\\ \cline{1-4}
		Sikkerhetsrele & lStasjon16.tex &  &\\ \cline{1-4}
		Servo med motion control blokker & i04858.tex &  &\\ \cline{1-4}
		
\end{tabular}
\end{center}
