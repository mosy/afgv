% !TEX root = /home/fred-olav/afgv/src/preamble.tex
% Start preamble
\documentclass[10pt,a5paper]{article}
\usepackage{geometry}
% \geometry{
% a4paper,
% total={170mm,257mm},
% left=20mm,
% top=20mm,
% }
\usepackage[utf8]{inputenc}
\usepackage[T1]{fontenc}
\usepackage[pdftex]{graphicx}
\graphicspath{{./}}
\usepackage{enumitem}
\usepackage{pdfpages}
\usepackage{hyperref}
\usepackage{tikz}
\usepackage{attachfile}
\usepackage[ngerman]{babel}
\usepackage{epstopdf}
\usepackage{array}
\usepackage{longtable}
\usepackage{amsmath}
\usepackage{makecell}
%\usepackage[table]{xcolor,colorbl}
%\setlength{\textwidth}{11cm}
%\setlength{\oddsidemargin}{-0.5cm}
\setlength{\evensidemargin}{-1.5cm}
%\setlenght{\headsep}{0cm}
\setlength\parindent{0pt}
%\setlength{\extrarowheight}{3pt}
\usepackage{listings}
\usepackage{microtype}
%\usepackage{xcolor}

\input{../src/arduinoLanguage.tex}

\include{def}
% End preamble

\begin{document}
\huge
\centerline \textbf{Håndbok for 3AUA Gand VGS} \bigskip
\normalsize
\vfil \eject
\section {Utførelse av arbeidsoppdrag}
%Måleteknikk 
\hrule

\vskip 1cm

I 3AUA utføres praktisk arbeid som "arbeidsoppdrag", dette skal bestå av følgende:\begin{itemize}[noitemsep]
	\item Planlegging
	\item Gjennomføring
	\item Dokumentasjon
\end{itemize}

\subsection {Planlegging}

Planleggingen av arbeidsoppdragene har som hovedhensikt å:

\begin{itemize}[noitemsep]
	\item Sørge for at vi har kunnskap nok til å starte opp arbeidet. 
	\item Tenke igjennom hva arbeidet består av og i hvilken rekkefølge det er hensiktmessig å utføre det. 
	\item Sørge for av vi har alt nøvendig utstyr og materiell. 
\end{itemize}



Alt arbeid skal HMS-vurderes før planlegging og utførelse. Nødvendig sikringstiltak skal planlegges og utføres i henhold til bedriftens HMS-regelverk. Eksempelvis skal all energitilførsel stenges og sikres mot utilsiktet innkopling, hvis mulig. Hvis ikke, skal arbeidene planlegges og utføres under gjeldende sikringstiltak, eksempelvis med AUS verktøy isolerende matter, etc. Husk spesielt personlig verneutstyr, så som briller, hjelm, hansker, sko, klær, hørselvern, etc.


\vskip 10pt 
Risikovurdering
\vskip 10pt 
SJA
\vskip 10pt 
\vskip 10pt 
\vskip 10pt 
\vskip 10pt 
\vskip 10pt 
\vfil \eject
\section{Kapittel 2 - Dokumentasjon}
\subsection{Instrument identifiserings tag} 

Frem til nå har vi sett på flere typer instrument skjema, i alle har de for skjellige instrumentene hatt en kode som forklarer funksjonen. TT er en Termperatur Transmitter, PDT er en Trykk Differanse Transmitter eller FV er en Flow Ventil. Disse bokstavene er definert i ISO 14617-6. 

Hvert instrument i et automatisert anlegg bør ha sin egen unike \textit{tag}, bestående av en bokstavkode som beskriver instrumentet sin funksjon i tillegg til et tall som beskriver hvilken sløyfe det tilhører. Det kan også være nummer som referer til et større om område i anlegget. Det samme instrumentet kan brukes flere plasser i en sløyfe, en kan da bruke en bokstavkode til de ulike delene.  

Som et eksempel, hvis vi ser et instrument med tag\texttt{FC-135}, så vet vi at det er en \textit{flow controller} (FC) for sløyfe 135. I et stort prosessanlegg med flere områder med ulike funksjoner, ville kanskje dette instrumentet vært merket 12-FC-135 (flow controller for sløyfe 135 i område 12. ) Om denne støyfen inneholdt flere regulatorer (controllere), kunne skilt mellom dem ved å bruke etterfølgende bokstaver på sløyfenummeret. (12-FC-135A, 12-FC-135B, 12-FC-135C). 

Alle instrumenter i en sløyfe blir identifisert med en bokstav som beskriver variabelen som sløyfen skal regulere, helt uavhengig av den fysike oppbygning av instrumentet. Vår tenkte flow controller FC-135, kan være fysisk lik nivå regulatoren i sløyfe \#72 (LC-72), eller til temperatur regulatoren i sløyfe \#288 (TC-288). Det som gjør FC-135 til en strømningsregulator er at den primære variabelen den skal regulere er strømning. Alle andre instrumenter i denne sløyfen vil også ha F som første bokstav\footnote{Exceptions do exist to this rule.  For example, in a cascade or feedforward loop where multiple transmitters feed into one or more controllers, each transmitter is identified by the type of process variable \textit{it} senses, and each controller's identifying tag follows suit.}. I en nivåreguleringssøyfe kan dette se slik ut: Transmitteren merkes "LT" selv om den måler trykk for å angi nivå, regulatoren merkes "LC" og reguleringsventilen merkes LV selv om den i prakses justerer strømning. Dette fordi deres primære funksjon er å bidra til nivåregulering. 

\filbreak

Bokstaver som kan brukes til å identifisere instrumenter angis i standarder. ISO14617-6 er en slik standard. Det finnes flere standarder i bruker rundt om i verden og gamle anlegg kan være basert på eldre standarder. Tabellen nedenfor er et utdrag av vanlige bokstaver som brukes. Legg merke til at bruken av signalomformer definerer en unik variabel. f.eks. en "PT" er en trykktransmitter som måler trykk på en plass, mens "PDT" er en måling av trykkdifferanse mellom to punkter. På samme måte kan "TC" være en temperaturregulator som regulerer temperaturen i en prosess, mens "TKC" er en regulator som regulerer temperatur forandring. 

% No blank lines allowed between lines of an \halign structure!
% I use comments (%) instead, so that TeX doesn't choke.
\small
\begin{center}
\begin{tabular}{ | m{1cm} | m{2.5cm}| m{2cm} | m{2.5cm} |} 
\hline
\multicolumn{4}{|c|}{Bokstavkode for indentifisering av instrumentfunksjoner} \\
\hline
	Bokst. & Variabel& Omformer & Funksjon \\ 
\hline
	A&&&Alarm\\
\hline
	B&&&Visning av diskre status\\
\hline
	C&&&Regulerende\\
\hline
	D&Densitet&Differanse&\\
\hline
	E&Elektrisk variabel&&Elemens (følende)\\
\hline
	F&Flow rate&Forhold&\\
\hline
	G&Posisjon, lengde&&Visning\\
\hline
	H&Håndbetjent&&\\
	\hline
	I&&&Indikerende\\
	\hline
	J&Effekt&Avsøke&\\
	\hline
	K&Tid&Forandring i tid&\\
	\hline
	L&Nivå&&\\
	\hline
	M&Fuktighet eller Relativ fuktighet&&\\
	\hline
	N&Etter brukers valg&&\\
	\hline
	N&Etter brukers valg&&\\
	\hline
	P&Trykk eller vakum&&Tilkobling til testpunkt\\
	\hline
	Q&\makecell{Egenskap\\f.eks:\\*Analyse\\*Konsentrasjon}&Integrere eller summere&Integrerende eller summerende\\
	\hline
	R&Radioaktiv stråling&&Skrivende\\
	\hline
	S&Hastighet&&Bryterfunksjon\\
	\hline
	T&Temperatur&&Overføring\\
	\hline
	U&Multivariabel&&Multifunksjon\\
	\hline
	V&Etter brukers valg& &Påvirkning på prosess med ventil eller pumpe, o.l.\\
	\hline
	W&Vekt, Kraft&Multipliserende&\\
	\hline
	X&Uspesifiserte variabler&&Uspesifisert\\
	\hline
	Y&Etter brukers valg&&Konvertering eller algoritme\\
	\hline
	Z&Antall hendelser, antall&&Nødbetjening eller sikkerhetsfunksjon\\
\hline
\end{tabular}
\end{center}
\normalsize
%Måleteknikk 
\vskip 5pt
\hrule

\vskip 5pt


\vfil \eject
\textbf{Kapittel 3 - Instrumentering}
%Måleteknikk

\begin{center}
\begin{longtable}{ | m{2cm} | m{7cm} | } 
\hline
\multicolumn{2}{|c|}{Definisjoner} \\
\hline
Begrep	& Beskrivelse \\ 
\hline
\hline
	Avvik&Forskjellen mellom den målte verdien og ønsket verdi (skalverdi) i ei reguleringssløyfe\\
	\hline
	Dødgang (dødsone)&For prosessinstrumenter er dødgang det området som et inngangssignal kanforandres innenfor, uten at det vises noen forandring i utgangssignalet. Med forandring menes her reversering av retningen på inngangssignalet\\
	\hline
	Forsterkning&Forholdet mellom forandring i inngangsverdien i forhold til den utgangsverdien som var årsak til forandringen\\
	\hline
	Hysterese&Samme måleverdi kan gi ulike utsignaler fra måleomformeren ved samme verdi, Hysterese oppstår når måleverdien veksler mellom å være stigende eller fallende.\\
	\hline
	Kalibrering&Sjekk av at måleinstrumentets verdier stemmer med et intrument med høyere nøyaktihet. Noen intrumenter kan justeres inn om de ikke stemmer overens.\\ 
	\hline
	Lesbarhet&Den minste inndeling på skalaen på et instrument.\\
	\hline
	Måleomfang&Differansen mellom øvre og nedre målegrense\\
	\hline
	Målegrense&URV: høyeste verdi som instrumentet kan vise. LRV: laveste verdi som instrumentet kan vise. \\
	\hline
	Målegrense&Den fulle skalaen for en måling, indikering eller avlesning. Eksempelvis -50 til 150°C\\
	\hline
	Nullpunktfeil&En forskyving av nede målegrense\\
	\hline
	Nøyaktighet& Forskjell mellom avlest og virkelig verdi for en målt variabel.\\
	\hline

	Reproduser- barhet& Evnen til å reprodusere samme resultat fra en rekke uavhengige målinger av den samme verdien selv om målingen utføres av forskjellige personer, på forskjellige steder og under ulike klimatiske forhold.\\
	\hline
	Repeterbarhet&Samsvar mellom en rekke på hverandre følgende målinger av utgangsverdi for den samme inngangsverdien. Målingene foretas under samme klimatiske forhold, av samme person og med korte intervaller.\\
	\hline
	Ulinearitet&En type feil der inngangsverdiene til et instrument ikke forholder seg til den ideelle rettlinjede sammenhengen mellom inngangs- og utgangs-verdiene (referansekarakteristikken).\\
	\hline

Regulator & En enhet som har til oppgave å påvirke prosessen slik at den oppnår en ønsket tilstand (f.eks. et ønsket nivå eller en ønsket temperatur).\\
	\hline
Prosess & Det anlegget eller systemet som inngår i reguleringen.\\
	\hline
Prosessvariabel & Den fysiske størrelsen i prosessen som skal reguleres (nivå, trykk, temperatur etc.)\\
	\hline
ProsessVerdi PV & Den verdien prosessvariabelen til enhver tid har.\\
	\hline
Settpunkt SP & Den verdien vi ønsker at prosessvariabelen skal ha.\\
	\hline
Manipulerende Variabel MV & Signalet som styrer pådragsorganet\\
	\hline
Forsyning & \\
	\hline
Pådrag & Det som er ment å variere prosessvariabelen. F.eks. væske inn i en nivåtank. \\
	\hline
Belastning & Det som tas ut av prosessen ved konstant PV. Vil ha samme verdi som pådraget. \\
	\hline
Forstyrrelse & Forandringer som påvirker verdien til prosessvariabelen. \\
	\hline
Avviket e & Forskjellen mellom PV og SP (Direkte virkning PV-SP, Reverserende virkning SP-PV)\\
	\hline
Pådragsorgan & Den komponenten som styrer pådraget (f.eks. motoren i bilen som påvirker hastigheten, eller ventilen som påvirker nivået i tanken).\\
	\hline
Forstillingsenhet & I vårt eksempel med regulering av bilens hastighet, er forstillingsenheten forgasseren. Motoren er pådragsorganet i reguleringssløyfen.\\
	\hline
Auto og Manuell modus (Lukket& eller åpensløyfe). Om pådraget styres av regulatoren eller en manuell innstilt verdi. \\
	\hline
LRV og URV& ( Lover Range Value og Upper Range value, Laveste og høyeste verdi målesignalet kan ha.)\\

	\hline


\end{longtable}
\end{center}
\vskip 5pt
\hrule

\vskip 5pt

\subsubsection{skallering}
$$\includegraphics{current63.eps}$$

Ut fra denne representasjonen kan vi sette opp følgende formel for konvertering mellom signaler. 
$$		\frac{y-y_{start}}{y_{range}}=\frac{x-x_{start}}{x_{range}}$$

$$\includegraphics{current59.eps}$$
\vfil \eject
\section{Reguleringsteknikk}
\subsection{Optimalisering}
\subsubsection{Tims rule of tumb}
\begin{enumerate}
	\item Bare jobb med en justering av et parameter om gangen. Når du justerer flere mister du kontrollen. 
	\item Proporsjonalleddet bestemmer hvor fort en prosess går mot settpunktet. Stor proporsjonalforsterkning vil få prosesse til å nå settpunktet raskt, men du vil få et oversving og mest sannsynlig oscillasjoner. Om du setter P-forsterkningen for lavt unngår du oscillasjoner men det vil ta lang tid å nå settpunktet. Start med I-leddet og D-leddet avslått og øk forsterkningen forsiktig  fra 0.25 (0.25, 0.5, 1, 2, 4, 8, 16) til du får oscillasjoner. Reduser så forsterkningen litt. En regulator med bare P-ledd vil ha et statisk avvik fra settpunktet. 
	\item I-leddet virker ved å fjerne feil. Det kan hjelpe til med å redusere oscillasjoner og statiske avvik, men feil justering vil føre til oversving og oscillasjoner. Reduser I-tiden i forsiktig til oscillasjoner og statisk avvik er elliminert. 
	\item D-leddet er ikke nødvendig i de fleste tilfeller om det er akseptabelt med oversving. Om du trenger D-leddet øk forsiktid d-tiden til du er fornød med responsen på variasjoner i prosessen  (SP forandringer).
\end{enumerate}





\subsubsection{Ziegler\_nichols svingemetode}
\begin{enumerate}
\item Regulatoren må stå i manuell modus.
\item Få prosessen til arbidpunktet ved å justere MV manuelt eller ved å bruke en ikke optimalisert regulering. Det er viktig at PV er tilnærmet lik SP da prosessen kan ha andre egenskaper ved andre verdier. 
\item Sett regulatoren til å bare bruke P-leddet. Sett I-ledd til uendelig eller så høyt den går. NB noen regulatorer slår av I-leddet om du setter I-tiden til 0, sjekk  manualen. Sett D-leddet til 0. 
\item Sett regulatoren i automatisk modus.
\item Øk $K_p$ (du kan starte med $K_p$ = 1) inntil det oppstår stående svingninger
i sløyfen etter et sprang i settpunktet. (Reguleringssystemet er da
på stabilitetsgrensen.) Spranget skal være lite, f.eks. 5\% av referansens
verdiområde, slik at prosessen holder seg nokså nær arbeidspunktet.
Men spranget må heller ikke være så lite at responsen ikke kan observeres.
Obs: Pass på at pådraget ikke når sine metningsgrenser (maks, min)
under eksperimentene. Hvis pådraget når en av disse grensene,
vil det kunne bli stående svingninger uansett hvor stor K p vi bruker.
F.eks. kan vi da ha funnet at K pk = 1000000 gir stående svingninger,
og ihht. formelen for K p i en PI-regulator skal da K p settes lik
450000, som temmelig sikkert gir et ustabilt reguleringssystem! Det
gjelder altså å finne den minste K p u som gir stående svingninger
uten at pådraget når metningsgrensene. Dette krever at du overvåker
pådraget under eksperimentene og passer på å ikke ha så store settpunktsendringer
at pådraget når maksimum- eller minimumsverdiene.
\item Noter $K_p$ -verdien som gir stående svingninger. Denne verdien kalles
	den kritiske forsterkning $K_{Pu}$. Noter også perioden $P_u$ for de stående
svingningene. Denne perioden kalles den kritiske perioden.
\item Beregn regulatorparametrene i henhold til tabellen  og legg dem
inn i regulatoren. Forhåpentligvis får da reguleringssystemet tilfredsstillende
ytelse. Er stabiliteten i reguleringssløyfen dårlig (store oversving
i responsene), er det enklest å prøve å redusere $K_p$ .
\end{enumerate}
\begin{tabular}{|c|c|c|c|}
\hline 
 &  &  & \tabularnewline
\hline 
P-regulator & 0.5$K_{pk}$ & $\infty$ & 0\tabularnewline
\hline 
PI-regulator & 0.45$K_{pk}$ & $\frac{P_{u}}{1.2}$ & 0\tabularnewline
\hline 
PID-regulator & 0.6$K_{pk}$ & $\frac{P_{u}}{2}$ & $\frac{P_{u}}{8}=\frac{T_{i}}{4}$\tabularnewline
\hline 
\end{tabular}




\subsubsection{Skogestads metode}

Skogestad har angitt PID-regulatorinnstilling for en rekke ulike prosesstyper. Vi bruker den på følgende prosesstyper:
\begin{itemize}
\item Selvstabiliserende med tidsforsinkelse. (Eksempel på
prosess med slik dynamikk er varmeveksler.)
\item Selvstabiliserende med neglisjerbar tidsforsinkelse. Ved denne type prossess vil Z\&N være vanskelig å få til å svinge
\item Integrerende med tidsforsinkelse 
(Eksempel: Tank med transportbånd, som flistanken.)
\item Integrerende uten tidsforsinkelse (Eksempel: Væsketank styrt av pumpe eller ventil på inn- eller utløp.)
\end{itemize}

\paragraph{Innstilling av PI-regulator for tidkonstant med tidsforsinkelse}

~

\includegraphics{Sprang_selvregulerende.eps}

\[
K_{P}=\dfrac{\tau}{K\left(T_{C}+\theta\right)}
\]

\[
T_{i}=min[\tau,c(T_{c}+\theta]
\]

Her bruker en den minste verdien av $\tau$ og c (c er 2 eller 4)

\includegraphics[width=1\textwidth]{Reg_step_response01}

Her er et eksempel på en prosess med tidsforsinkelse. Når vi leser
av garfen får vi at

\begin{eqnarray*}
\theta & = & 0.5s\\
\tau & = & 5s\\
K & = & 2
\end{eqnarray*}

Det gir oss:
\[
K_{P}=\dfrac{\tau}{K\left(T_{C}+\theta\right)}=\dfrac{5}{2(0.5+0.4)}=2.5
\]

c=2 er her mindre en T=5 som gir 
\[
T_{i}=c(T_{c}+\theta)=2(0.5+0.5)=2
\]

Det gir følgende innregulering på et sprang fra 50 til 55. 

\includegraphics[width=1\textwidth]{Reg_innreg_skogestad01}

\paragraph{Innstilling av PI-regulator for integrator med tidsforsinkelse}

~

\includegraphics{Sprang_integrerendeprosess.eps}

\[
K_{i}=\dfrac{\Delta PV}{\Delta MV\cdot\Delta t}
\]

\[
Kp=\dfrac{1}{K_{i}\left(T_{c}+\theta\right)}
\]

\[
T_{i}=c\left(T_{c}+\theta\right)
\]

Eksempel:

\includegraphics[width=1\textwidth]{Reg_step_integrator}

\[
K_{i}=\dfrac{\Delta PV}{\Delta MV\cdot\Delta t}=\dfrac{40\%}{5\%\cdot8s}=1/s
\]
\[
\tau=0.5s
\]

\[
Kp=\dfrac{1}{K_{i}\left(T_{c}+\theta\right)}=\dfrac{1}{1}
\]

\[
T_{i}=c\left(T_{c}+\theta\right)=2\left(0.5+0.5\right)=2
\]
Som gir følgende innregulering

\includegraphics[width=1\textwidth]{Reg_innreg_skogestad02}
\section{Formelsamling}

\subsection{Elektroteknikk}
\vskip 2.5pt
\subsubsection*{Ohms lov}
\vskip 2.5pt
$U=R\cdot I$.
\vskip 2.5pt
\subsubsection*{Kirchoffs lover}
\vskip 2.5pt  
Strøm $\Sigma I=0$\\
\vskip 2.5pt  
$I=I_{1}+I_{2}+\cdot\cdot\cdot+I_{n}$\\
\vskip 2.5pt  
$R=\dfrac{1}{\dfrac{1}{R_{1}}+\dfrac{1}{R_{2}}+\cdot\cdot\cdot+\dfrac{1}{R_{n}}}$
\vskip 2pt
 Spenning $\Sigma U=0$\\
\vskip 2.5pt  
$U=U_{1}+U_{2}+\cdot\cdot\cdot+U_{n}$\\
\vskip 2.5pt  
$R=R_{1}+R_{2}+\cdot\cdot\cdot+R_{n}$\\
\vskip 2pt
Lederresistans: $ R=\frac{\rho\cdot l}{A}$ 
\vskip 2pt
Kortsluttningstrøm $I_{k}=\frac{E}{R_{i}+R_{ytre}}$
\vskip 2pt
Elektrisk effekt: $P=\frac{W}{t}$
\vskip 2pt  
Effektloven: $P=U\cdot I\cdot \sqrt{3} \cdot \cos \varphi$
\vskip 2.5pt  
Virkningsgrad: $\eta=\frac{P_{avgitt}}{P_{tilf\\ort}}$
\vskip 2.5pt
Synkron hastihet: $n_1=\frac{2\cdot f}{P}$
\vskip 2.5pt 
Sakking: $S=\frac{n_1-n}{n_1}$
\vskip 2.5pt 
\subsection{Reguleringsteknikk}
\vskip 2.5pt 
\subsubsection*{PID-regulator}
$MV=$\\
$K_p(e+1/TN \int e dt + TV de/dt)+BIAS$\\
\vskip 2.5pt
$Direkte e=PV-SP$\\
$Reverserende e=SP-PV$\\
\subsubsection*{Ziegler \& Nichols svingemetode}
\subsubsection*{P-regulator}
$K_p = 0.5\cdot K_u$
\subsubsection*{PI-regulator}
$K_p = 0.45\cdot K_u$\\
$T_i = 0,83\cdot T_u$\\
\subsubsection*{PID-regulator}
$K_p = 0.6\cdot K_u$\\
$T_i = 0,5\cdot T_u$\\
$T_d = 0,125\cdot T_u$\\
\vskip 2.5pt 
\subsubsection*{Prosessdynamikk}
\vskip 2.5pt 
$\theta$=døtid
\vskip 2.5pt 
$ \text{Tidskonstant finnes når} \Delta PV_{63\%}$
\vskip 2.5pt 
$\tau$=tidskonstand
\vskip 2.5pt 
$T_C$ velges lik $\theta$
\vskip 2.5pt 
\subsubsection*{Selvstabiliserende}
\vskip 2.5pt 
$K=\dfrac{\Delta PV}{\Delta MV} $
\vskip 2.5pt 
$K_p=\dfrac{\tau}{K(T_C+\theta)}$
\vskip 2.5pt 
$T_i=2(T_C+\theta)$
\vskip 2.5pt 
\subsubsection*{Integrerende}
\vskip 2.5pt 
$K_i=\dfrac{\Delta PV}{\Delta MV \cdot {\Delta t}}$
\vskip 2.5pt 
$K_p=\dfrac{1}{K_i(T_C+\theta)}$
\vskip 2.5pt 
$T_i=2(T_C+\theta)$\\
\subsection{Måleteknikk}
\subsubsection*{Kalibrering}
Feilprosent av span:\\
$Error=\frac {IUT-Standard}{Span} \cdot 100\% $\\
\vskip 2.5pt 
Skaleringsformel:
\vskip 2.5pt 
$\dfrac{x-x_{start}}{x_{range}}=\dfrac{y-y_{start}}{y_{range}}$\\
Eksempel: $\dfrac{12mA-4mA}{16mA}=\dfrac{50\%-0\%}{100\%}$
\subsubsection*{Flowmåling}
\vskip 2.5pt 
\vskip 2.5pt 
\vskip 2.5pt 
Reynoldstall: $Re=\dfrac {D \cdot v \cdot \rho}{\mu}$\\
\vskip 2.5pt 
Bernoulli's formel:\\
$z_1 \rho g + {v_1^2 \rho \over 2} + P_1 = z_2 \rho g + {v_2^2 \rho \over 2} + P_2$\\
\vskip 2.5pt 
\subsubsection*{Law of Continuity}
$\rho_1 A_1 \overline{v_1} = \rho_2 A_2 \overline{v_2} = \cdots \rho_n A_n \overline{v_n}$\\
\\
$W = \rho A \overline{v}$\\\\
$Q=A_1 \overline{v_1} = A_2 \overline{v_2}$\\

Massestrøm: $W=\frac{m}{t}$\\
\vskip 2.5pt 
Volumstrøm: $Q=\frac{V}{t}$\\
\vskip 2.5pt 
Massetetthet: $\rho=\frac{m}{V}$\\
\vskip 2.5pt 
\subsubsection*{Trykkbaserte strømningsmålere}
Måleblende: $Q=k\cdot \sqrt{\Delta p}$\\
\vskip 2.5pt 
Måleblende med varierende $\rho$: $Q=k\cdot \sqrt{\frac{\Delta p}{\rho}}$\\
\vskip 2.5pt
Kvadratrot uttrekker:\\
s = signal i prosent (0-1)\\
$s_{ut}=\sqrt{s_{inn}}$\\
\subsubsection*{Hastighetsbaserte strømningsmålere}
Turbin og Vortex:\\
$Q=kf$\\
Elektromagnetisk:\\
$E=Blv$\\
Ultralyd:\\
$Q = k {t_{up} - t_{down} \over (t_{up})(t_{down})}$\\
 
\subsection{Nivåmåling}

$P = \rho \cdot g \cdot h $\\
\subsection{Temperaturmåling}
$R_T = R_{ref}[1 + \alpha(T - T_{ref})]$\\
European $\alpha=0.00385$\\
American $\alpha=0.00392$\\
\end {document}
