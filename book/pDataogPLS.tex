% ta med damping på DP-delle

\documentclass[aspectratio=169,xcolor=dvipsnames]{beamer}
%\usetheme{SimplePlus}

\usepackage{hyperref}
\usepackage{graphicx} % Allows including images
\usepackage{booktabs} % Allows the use of \toprule, \midrule and \bottomrule in tables

%----------------------------------------------------------------------------------------
%	TITLE PAGE
%----------------------------------------------------------------------------------------

\title[PLS]{PLS og datakommunikasjon for Gandsfjorden Gondol} % The short title appears at the bottom of every slide, the full title is only on the title page
%\subtitle{Subtitle}

\author[Fred-Olav] {Fred-Olav Mosdal}

\institute[Gand VGS] % Your institution as it will appear on the bottom of every slide, may be shorthand to save space
{
    Gand VGS \\
    VG3 Automasjon }
\date{\today} % Date, can be changed to a custom date


%----------------------------------------------------------------------------------------
%	PRESENTATION SLIDES
%----------------------------------------------------------------------------------------

\begin{document}
\begin{frame}
\titlepage
\end{frame}






\begin{frame}
	\frametitle{Blokkskjematisk oppbyggning av en PLS}
	$$\includegraphics[width=0.6\textwidth]{../output/noGPLimages/pls01.png}$$
\end{frame}
\begin{frame}
	\frametitle{Strømforsyningsenhet}
	\begin{columns}
		\begin{column}{0.5\textwidth}
			\begin{itemize}
				\item 24VDC eller 230VAC er vanlig
				\item Viktig å skille PLS forsyning fra IO-forsyning slik at feil ute i anlgget ikke gjøre at PLS mister strømforsyning. 
			\end{itemize}

			
		\end{column}

		\begin{column}{0.5\textwidth}
%	$$\includegraphics[width=1\textwidth]{../output/noGPLimages/pls01.png}$$
		\end{column}
	\end{columns}
\end{frame}

\begin{frame}
	\frametitle{Overføringskabler}
	\begin{columns}
		\begin{column}{0.5\textwidth}
			\begin{itemize}
				\item USB-kabel
				\item RS-232(Ofte med overgang fra USB)
				\item RS-485(Ofte med overgang fra USB)
				\item Ethernet (TCP/IP)
			\end{itemize}

			
		\end{column}

		\begin{column}{0.5\textwidth}
	$$\includegraphics[width=1\textwidth]{../output/noGPLimages/pls02.png}$$
		\end{column}
	\end{columns}
\end{frame}

\begin{frame}
	\frametitle{Inngangs og utgangsenheter}
	\begin{columns}
		\begin{column}{0.5\textwidth}
			Inngangs- og utgangsenheter i en PLS kalles for IO-er. Det kan være:
			\begin{itemize}
				\item Digitale IO-er. DI for innganger og DO for utganger
				\item Analoge IO-er AI for innganger og AO for utgangere
				\item Moduler for å avlese resolvere og enkodere. 
				\item Kommunikasjonsmoduler
			\end{itemize}


			
		\end{column}

		\begin{column}{0.5\textwidth}
%	$$\includegraphics[width=1\textwidth]{../output/noGPLimages/pls03.png}$$
		\end{column}
	\end{columns}
\end{frame}
\begin{frame}
	\frametitle{DO med rele}
	\begin{columns}
		\begin{column}{0.5\textwidth}
			Rele utganger :
			\begin{itemize}
				\item kan være potensialfrie
				\item Kan bryte forholdsvis store strømmer (6-10A)
				\item Kan brukes på AC og DC
			\end{itemize}

			
		\end{column}

		\begin{column}{0.5\textwidth}
	$$\includegraphics[width=1\textwidth]{../output/noGPLimages/pls03.png}$$
		\end{column}
	\end{columns}
\end{frame}

\begin{frame}
	\frametitle{DO med transistor (Transistorutgang}
	\begin{columns}
		\begin{column}{0.5\textwidth}
			Transistorutganger:
			\begin{itemize}
				\item bryter mindre strømmer (0.5 og 1 A er vanlig)
				\item bryter raskere en rele utganger
				\item Finnes i NPN eller PNP utgaver
				\item NPN kalles også low side switching
				\item PNP kalles også high side switching
				\item Kan brukes på DC
			\end{itemize}

			
		\end{column}

		\begin{column}{0.5\textwidth}
	$$\includegraphics[width=1\textwidth]{../output/noGPLimages/pls04.png}$$
		\end{column}
	\end{columns}
\end{frame}

\begin{frame}
	\frametitle{DO med triac}
	\begin{columns}
		\begin{column}{0.5\textwidth}
			\begin{itemize}
				\item Kan bryte mindre AC strømmer en releer
				\item Tåler flere bryte sykluser
			\end{itemize}

			
		\end{column}

		\begin{column}{0.5\textwidth}
	$$\includegraphics[width=0.8\textwidth]{../output/noGPLimages/pls05.png}$$
		\end{column}
	\end{columns}
\end{frame}

\begin{frame}
	\frametitle{Galvaniske skiller}
	\begin{columns}
		\begin{column}{0.5\textwidth}
			\begin{itemize}
				\item Isolerer PLS fra spenninger i felt. 
			\end{itemize}

			
		\end{column}

		\begin{column}{0.5\textwidth}
	$$\includegraphics[width=1\textwidth]{../output/noGPLimages/pls06.png}$$
		\end{column}
	\end{columns}
\end{frame}


\begin{frame}
	\frametitle{Analoge IO-er}
	\begin{columns}
		\begin{column}{0.5\textwidth}
			Analoge innganger (AI)
			\begin{itemize}
				\item 1-5 V
				\item 0-10 V
				\item 2-10 V
				\item 4-20 mA
				\item 0-20 mA
				\item RTD ($\Omega$)
				\item termoelement mV
			\end{itemize}

			
		\end{column}

		\begin{column}{0.5\textwidth}
			\href{https://www.contec.com/support/basic-knowledge/daq-control/analog-io/}{link}
			%	$$\includegraphics[width=1\textwidth]{../output/noGPLimages/pls06.png}$$
		\end{column}
	\end{columns}
\end{frame}

\begin{frame}
	\frametitle{Kompakte og  modulære PLS-er}
	\begin{columns}
		\begin{column}{0.5\textwidth}
			\begin{itemize}
				\item item
			\end{itemize}

			
		\end{column}

		\begin{column}{0.5\textwidth}
	$$\includegraphics[width=1\textwidth]{../output/noGPLimages/pls07.png}$$
		\end{column}
	\end{columns}
\end{frame}

\begin{frame}
	\frametitle{Blokkskjematisk oppbygging av PLS fra inngang til utgang}
	$$\includegraphics[width=1\textwidth]{../output/noGPLimages/pls08.png}$$
\end{frame}

\begin{frame}
	\frametitle{Oppbyggning av Module PLS}
	\begin{columns}
		\begin{column}{0.5\textwidth}
			\begin{itemize}
				\item item
			\end{itemize}

			
		\end{column}

		\begin{column}{0.5\textwidth}
	$$\includegraphics[width=1\textwidth]{../output/noGPLimages/pls09.png}$$
		\end{column}
	\end{columns}
\end{frame}

\begin{frame}
	\frametitle{Elektronisk skjema over Siemens DO rele modul}
	$$\includegraphics[width=1\textwidth]{../output/noGPLimages/pls10.png}$$
\end{frame}
\begin{frame}
	\frametitle{Datakommunikasjon}
Med datakommunikasjon menes utvekslig av data som datamaskiner behandler. 
\end{frame}




\begin{frame}
	\frametitle{Data og dataoverføring}
	\begin{columns}
		\begin{column}{0.5\textwidth}
I PLS systemer er ofte måledata som skal overføres. Det overføres da som et binærtall og konverteres til et annet tallsystem i PLS-en. 

			
		\end{column}

		\begin{column}{0.5\textwidth}
	$$\includegraphics[width=1\textwidth]{../output/noGPLimages/kap5x01}$$
		\end{column}
	\end{columns}
\end{frame}

\begin{frame}
	\frametitle{OSI-modellen}
	\begin{columns}
		\begin{column}{0.5\textwidth}

			\begin{itemize}
				\item      
			\end{itemize}

			
		\end{column}

		\begin{column}{0.5\textwidth}
	$$\includegraphics[width=1\textwidth]{../output/noGPLimages/kap5x02}$$
		\end{column}
	\end{columns}
\end{frame}
\begin{frame}
	\frametitle{Ethernet Type II Rammes}
	\begin{columns}
		\begin{column}{0.5\textwidth}

			\begin{itemize}
				\item      
			\end{itemize}

			
		\end{column}

		\begin{column}{0.5\textwidth}
	$$\includegraphics[width=1\textwidth]{../output/noGPLimages/kap5x03}$$
		\end{column}
	\end{columns}
\end{frame}
\begin{frame}
	\frametitle{Dataoverføringsmodus}
	\begin{columns}
		\begin{column}{0.5\textwidth}

Det er ofte nyttig å snakke om hvilke veier dataoverføringen går. Da snakker vi om simplex, halv duplex og duplex.

			
		\end{column}

		\begin{column}{0.5\textwidth}
	$$\includegraphics[width=1\textwidth]{../output/noGPLimages/kap5x04}$$
		\end{column}
	\end{columns}
\end{frame}
\begin{frame}
	\frametitle{Dataprotokoller}

For at 0 og 1 ere skal gi mening mellom senser og mottaker. Må de benytte en avtale om på hvilken måte dataene kommer. Dette kalles en protokoll. 
	
\end{frame}
\begin{frame}
	\frametitle{Dataoverføringsmetoder}
	\begin{itemize}
		\item Asynkron seriell dataoverføring
		\item Synkron seriell dataoverføring
		\item Parellell dataoverføring
	\end{itemize}
\end{frame}

\begin{frame}
	\frametitle{Asynkron seriel dataoverføring}
	\begin{columns}
		\begin{column}{0.5\textwidth}

			\begin{itemize}
				\item Bruker start og stopp bit
				\item kan bruke paritetsbit. 
				\item Veldig vanlig måte og overføre data på. 
			\end{itemize}

			
		\end{column}

		\begin{column}{0.5\textwidth}
	$$\includegraphics[width=1\textwidth]{../output/noGPLimages/kap5x08}$$
	$$\includegraphics[width=1\textwidth]{../output/noGPLimages/kap5x09}$$
		\end{column}
	\end{columns}
\end{frame}
\begin{frame}
	\frametitle{OSI-modellen}
	\begin{columns}
		\begin{column}{0.5\textwidth}

			\begin{itemize}
				\item 
			\end{itemize}

			
		\end{column}

		\begin{column}{0.5\textwidth}
	$$\includegraphics[width=1\textwidth]{../output/noGPLimages/kap5x21}$$
		\end{column}
	\end{columns}
\end{frame}
\begin{frame}
	\frametitle{Signalmodulering}
	\begin{columns}
		\begin{column}{0.5\textwidth}

			\begin{itemize}
				\item AM - Amplitudemodulering
				\item FM - Frekvensmodulering
			\end{itemize}

			
		\end{column}

		\begin{column}{0.5\textwidth}
	$$\includegraphics[width=1\textwidth]{../output/noGPLimages/kap5x22}$$
		\end{column}
	\end{columns}
\end{frame}
\begin{frame}
	\frametitle{Amplitudemodulering av dititalt signal}
	\begin{columns}
		\begin{column}{0.5\textwidth}

			\begin{itemize}
				\item      
			\end{itemize}

			
		\end{column}

		\begin{column}{0.5\textwidth}
	$$\includegraphics[width=1\textwidth]{../output/noGPLimages/kap5x23}$$
		\end{column}
	\end{columns}
\end{frame}
\begin{frame}
	\frametitle{Frekvensmodulering av dititalt signal}
	\begin{columns}
		\begin{column}{0.5\textwidth}

			\begin{itemize}
				\item      
			\end{itemize}

			
		\end{column}

		\begin{column}{0.5\textwidth}
	$$\includegraphics[width=1\textwidth]{../output/noGPLimages/kap5x24}$$
		\end{column}
	\end{columns}
\end{frame}
\begin{frame}
	\frametitle{Fasemodulering (PM) av dititalt signal}
	\begin{columns}
		\begin{column}{0.5\textwidth}

			\begin{itemize}
				\item      
			\end{itemize}

			
		\end{column}

		\begin{column}{0.5\textwidth}
	$$\includegraphics[width=1\textwidth]{../output/noGPLimages/kap5x25}$$
		\end{column}
	\end{columns}
\end{frame}
\begin{frame}
	\frametitle{Smarte instrumenter og HART}
	\begin{columns}
		\begin{column}{0.5\textwidth}

			\begin{itemize}
				\item      
			\end{itemize}

			
		\end{column}

		\begin{column}{0.5\textwidth}
	$$\includegraphics[width=1\textwidth]{../output/noGPLimages/kap5x27}$$
		\end{column}
	\end{columns}
\end{frame}
\begin{frame}
	\frametitle{HART-protokoll}
	\begin{columns}
		\begin{column}{0.5\textwidth}

			\begin{itemize}
				\item      
			\end{itemize}

			
		\end{column}

		\begin{column}{0.5\textwidth}
	$$\includegraphics[width=1\textwidth]{../output/noGPLimages/kap5x28}$$
		\end{column}
	\end{columns}
\end{frame}
\begin{frame}
	\frametitle{Håndterminaler og vedlikehold}
	\begin{columns}
		\begin{column}{0.5\textwidth}

	$$\includegraphics[width=1\textwidth]{../output/noGPLimages/kap5x30}$$

			
		\end{column}

		\begin{column}{0.5\textwidth}
	$$\includegraphics[width=1\textwidth]{../output/noGPLimages/kap5x29}$$
		\end{column}
	\end{columns}
\end{frame}

\begin{frame}
	\frametitle{Kommunikasjonsmoduler}
	\begin{columns}
		\begin{column}{0.5\textwidth}
			\begin{itemize}
				\item item
			\end{itemize}

			
		\end{column}

		\begin{column}{0.5\textwidth}
	$$\includegraphics[width=1\textwidth]{../output/noGPLimages/pls11.png}$$
		\end{column}
	\end{columns}
\end{frame}
\begin{frame}
	\frametitle{Modbus}
	\begin{columns}
		\begin{column}{0.5\textwidth}

			\begin{itemize}
				\item utgitt av modicon i 1979, elste kommnunikasjonsprotokoll innenfor automatisering
				\item er mye utbredt
				\item punkt til punkt og multidrop
				\item Modbus ASCII, Modbus Plus, Modbus RTU og Modubs TCP
			\end{itemize}

			
		\end{column}

		\begin{column}{0.5\textwidth}
	$$\includegraphics[width=1\textwidth]{../output/noGPLimages/kap5x79}$$
		\end{column}
	\end{columns}
\end{frame}
\begin{frame}
	\frametitle{Modbus dataoverføring med og uten feil}
	$$\includegraphics[width=1\textwidth]{../output/noGPLimages/kap5x80}$$
\end{frame}
\begin{frame}
	\frametitle{Modbus funksjonskoder}

	\url{https://ozeki.hu/p_5873-modbus-function-codes.html}

\end{frame}

\begin{frame}
	\frametitle{Logiske funksjoner}
	\begin{columns}
		\begin{column}{0.5\textwidth}
			\begin{itemize}
				\item item
			\end{itemize}

			
		\end{column}

		\begin{column}{0.5\textwidth}
	$$\includegraphics[width=1\textwidth]{../output/noGPLimages/pls12.png}$$
		\end{column}
	\end{columns}
\end{frame}

\begin{frame}
	\frametitle{Logiske funksjoner}
	\begin{columns}
		\begin{column}{0.5\textwidth}
			\begin{itemize}
				\item item
			\end{itemize}

			
		\end{column}

		\begin{column}{0.5\textwidth}
	$$\includegraphics[width=1\textwidth]{../output/noGPLimages/pls13.png}$$
		\end{column}
	\end{columns}
\end{frame}

\begin{frame}
	\frametitle{Data typer}
	\begin{columns}
		\begin{column}{0.5\textwidth}
			\begin{itemize}
				\item item
			\end{itemize}

			
		\end{column}

		\begin{column}{0.5\textwidth}
	$$\includegraphics[width=1\textwidth]{../output/noGPLimages/pls14.png}$$
		\end{column}
	\end{columns}
\end{frame}

\begin{frame}
	\frametitle{Frametitle}
	\begin{columns}
		\begin{column}{0.5\textwidth}
			\begin{itemize}
				\item item
			\end{itemize}

			
		\end{column}

		\begin{column}{0.5\textwidth}
	$$\includegraphics[width=1\textwidth]{../output/noGPLimages/pls15.png}$$
		\end{column}
	\end{columns}
\end{frame}


\end{document}
