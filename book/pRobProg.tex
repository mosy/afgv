
\documentclass[aspectratio=169,xcolor=dvipsnames]{beamer}
%\usetheme{SimplePlus}
\usepackage{hyperref}
\usepackage{graphicx} % Allows including images
\usepackage{booktabs} % Allows the use of \toprule, \midrule and \bottomrule in tables

%----------------------------------------------------------------------------------------
%	TITLE PAGE
%----------------------------------------------------------------------------------------

\title[COM]{Datakommunikasjon og elektroniske kommunikasjonsnett} % The short title appears at the bottom of every slide, the full title is only on the title page
%\subtitle{Subtitle}

\author[Fred-Olav] {Fred-Olav Mosdal}

\institute[Gand VGS] % Your institution as it will appear on the bottom of every slide, may be shorthand to save space
{
    Gand VGS \\
    VG3 Automasjon }
\date{\today} % Date, can be changed to a custom date


%----------------------------------------------------------------------------------------
%	PRESENTATION SLIDES
%----------------------------------------------------------------------------------------

\begin{document}
\begin{frame}
\titlepage
\end{frame}

\begin{frame}
	\frametitle{Datakommunikasjon}
Med datakommunikasjon menes utvekslig av data som datamaskiner behandler. 
\end{frame}




\begin{frame}
	\frametitle{Data og dataoverføring}
	\begin{columns}
		\begin{column}{0.5\textwidth}
I PLS systemer er ofte måledata som skal overføres. Det overføres da som et binærtall og konverteres til et annet tallsystem i PLS-en. 

			
		\end{column}

		\begin{column}{0.5\textwidth}
	$$\includegraphics[width=1\textwidth]{../output/noGPLimages/kap5x01}$$
		\end{column}
	\end{columns}
\end{frame}

\begin{frame}
	\frametitle{OSI-modellen}
	\begin{columns}
		\begin{column}{0.5\textwidth}

			\begin{itemize}
				\item      
			\end{itemize}

			
		\end{column}

		\begin{column}{0.5\textwidth}
	$$\includegraphics[width=1\textwidth]{../output/noGPLimages/kap5x02}$$
		\end{column}
	\end{columns}
\end{frame}
\begin{frame}
	\frametitle{Ethernet Type II Rammes}
	\begin{columns}
		\begin{column}{0.5\textwidth}

			\begin{itemize}
				\item      
			\end{itemize}

			
		\end{column}

		\begin{column}{0.5\textwidth}
	$$\includegraphics[width=1\textwidth]{../output/noGPLimages/kap5x03}$$
		\end{column}
	\end{columns}
\end{frame}
\begin{frame}
	\frametitle{Dataoverføringsmodus}
	\begin{columns}
		\begin{column}{0.5\textwidth}

Det er ofte nyttig å snakke om hvilke veier dataoverføringen går. Da snakker vi om simplex, halv duplex og duplex.

			
		\end{column}

		\begin{column}{0.5\textwidth}
	$$\includegraphics[width=1\textwidth]{../output/noGPLimages/kap5x04}$$
		\end{column}
	\end{columns}
\end{frame}
\begin{frame}
	\frametitle{Nettverkstopologier}
	$$\includegraphics[width=0.5\textwidth]{../output/noGPLimages/kap5x05}\includegraphics[width=0.5\textwidth]{../output/noGPLimages/kap5x06}$$
\end{frame}
\begin{frame}
	\frametitle{Krysset nettverkskabel}
	\begin{columns}
		\begin{column}{0.5\textwidth}

Nyere nettverksutstyr trenger vanlig vis ikke krysset kabel da den har auto sense
			
		\end{column}

		\begin{column}{0.5\textwidth}
	$$\includegraphics[width=1\textwidth]{../output/noGPLimages/kap5x07}$$
		\end{column}
	\end{columns}
\end{frame}
\begin{frame}
	\frametitle{Datanettverkstyper}
	\begin{itemize}
		\item LAN
		\item WAN
		\item WLAN
	\end{itemize}

	
\end{frame}
\begin{frame}
	\frametitle{Dataprotokoller}

For at 0 og 1 ere skal gi mening mellom senser og mottaker. Må de benytte en avtale om på hvilken måte dataene kommer. Dette kalles en protokoll. 
	
\end{frame}
\begin{frame}
	\frametitle{Dataoverføringsmetoder}
	\begin{itemize}
		\item Asynkron seriell dataoverføring
		\item Synkron seriell dataoverføring
		\item Parellell dataoverføring
	\end{itemize}
\end{frame}

\begin{frame}
	\frametitle{Asynkron seriel dataoverføring}
	\begin{columns}
		\begin{column}{0.5\textwidth}

			\begin{itemize}
				\item Bruker start og stopp bit
				\item kan bruke paritetsbit. 
				\item Veldig vanlig måte og overføre data på. 
			\end{itemize}

			
		\end{column}

		\begin{column}{0.5\textwidth}
	$$\includegraphics[width=1\textwidth]{../output/noGPLimages/kap5x08}$$
	$$\includegraphics[width=1\textwidth]{../output/noGPLimages/kap5x09}$$
		\end{column}
	\end{columns}
\end{frame}
\begin{frame}
	\frametitle{Parallell}
	\begin{columns}
		\begin{column}{0.5\textwidth}

			\begin{itemize}
				\item Data sendes samtidig på flere ledere
			\end{itemize}

			
		\end{column}

		\begin{column}{0.5\textwidth}
	$$\includegraphics[width=1\textwidth]{../output/noGPLimages/kap5x10}$$
		\end{column}
	\end{columns}
\end{frame}
\begin{frame}
	\frametitle{Ulike UBS-plugger}
	\begin{columns}
		\begin{column}{0.5\textwidth}

			\begin{itemize}
				\item      
			\end{itemize}

			
		\end{column}

		\begin{column}{0.5\textwidth}
	$$\includegraphics[width=1\textwidth]{../output/noGPLimages/kap5x11}$$
		\end{column}
	\end{columns}
\end{frame}
\begin{frame}
	\frametitle{Ethernet pluggen (RJ-45)}
	\begin{columns}
		\begin{column}{0.5\textwidth}

	$$\includegraphics[width=1\textwidth]{../output/noGPLimages/kap5x12}$$
			
		\end{column}

		\begin{column}{0.5\textwidth}
	$$\includegraphics[width=1\textwidth]{../output/noGPLimages/kap5x13}$$
		\end{column}
	\end{columns}
\end{frame}

\begin{frame}
	\frametitle{Tvinnede parkabler - TP-kabel}
	\begin{columns}
		\begin{column}{0.5\textwidth}

	$$\includegraphics[width=1\textwidth]{../output/noGPLimages/kap5x14}$$

			
		\end{column}

		\begin{column}{0.5\textwidth}
	$$\includegraphics[width=0.7\textwidth]{../output/noGPLimages/kap5x15}$$
	$$\includegraphics[width=1\textwidth]{../output/noGPLimages/kap5x16}$$
		\end{column}
	\end{columns}
\end{frame}
\begin{frame}
	\frametitle{HUB og Switch}
	\begin{columns}
		\begin{column}{0.5\textwidth}

			\begin{itemize}
				\item HUB sender alle data til alle     
				\item Switch kobler mellom enhetene som skal motta data-ene
			\end{itemize}

			
		\end{column}

		\begin{column}{0.5\textwidth}
	$$\includegraphics[width=1\textwidth]{../output/noGPLimages/kap5x17}$$
		\end{column}
	\end{columns}
\end{frame}
\begin{frame}
	\frametitle{Utstyr en finner i et nettverk}
	\begin{columns}
		\begin{column}{0.5\textwidth}

	\begin{itemize}
		\item Routere kobler sammen to ulike nettverk
		\item Et eksempel er lokalnettverket hjemme med internett
		\item Det er vanlig at routere har innebygget DHCP-server. Det vil si at alle PC-er som er satt opp til å motta IP-adresse vil koble seg opp mot denne og få en IP-adresse.
		\item routeren er gateway til internett. 
		\item En gateway kan være koble mellom mange ulike nett. 
	\end{itemize}

			
			
		\end{column}

		\begin{column}{0.5\textwidth}
			\begin{itemize}
				\item Enheter som tilbyr tjenester i et nettverk kalles servere
				\item Nettverk kan kobles sammen med ulike teknologier som f.eks. Ethernet (fast, gigabit, 10 Gigabit), wireless ethernet, fiber. 
			\end{itemize}

			
		\end{column}
	\end{columns}
\end{frame}
\begin{frame}
	\frametitle{Kategorier og klasser av kabler}
	\begin{columns}
		\begin{column}{0.5\textwidth}

			\begin{itemize}
				\item      
			\end{itemize}

			
		\end{column}

		\begin{column}{0.5\textwidth}
	$$\includegraphics[width=1\textwidth]{../output/noGPLimages/kap5x18}$$
	$$\includegraphics[width=1\textwidth]{../output/noGPLimages/kap5x19}$$
		\end{column}
	\end{columns}
\end{frame}
\begin{frame}
	\frametitle{OSI-modellen}
	\begin{columns}
		\begin{column}{0.5\textwidth}

	$$\includegraphics[width=1\textwidth]{../output/noGPLimages/kap5x20}$$

			
		\end{column}

		\begin{column}{0.5\textwidth}
		\end{column}
	\end{columns}
\end{frame}
\begin{frame}
	\frametitle{OSI-modellen}
	\begin{columns}
		\begin{column}{0.5\textwidth}

			\begin{itemize}
				\item 
			\end{itemize}

			
		\end{column}

		\begin{column}{0.5\textwidth}
	$$\includegraphics[width=1\textwidth]{../output/noGPLimages/kap5x21}$$
		\end{column}
	\end{columns}
\end{frame}
\begin{frame}
	\frametitle{Signalmodulering}
	\begin{columns}
		\begin{column}{0.5\textwidth}

			\begin{itemize}
				\item AM - Amplitudemodulering
				\item FM - Frekvensmodulering
			\end{itemize}

			
		\end{column}

		\begin{column}{0.5\textwidth}
	$$\includegraphics[width=1\textwidth]{../output/noGPLimages/kap5x22}$$
		\end{column}
	\end{columns}
\end{frame}
\begin{frame}
	\frametitle{Amplitudemodulering av dititalt signal}
	\begin{columns}
		\begin{column}{0.5\textwidth}

			\begin{itemize}
				\item      
			\end{itemize}

			
		\end{column}

		\begin{column}{0.5\textwidth}
	$$\includegraphics[width=1\textwidth]{../output/noGPLimages/kap5x23}$$
		\end{column}
	\end{columns}
\end{frame}
\begin{frame}
	\frametitle{Frekvensmodulering av dititalt signal}
	\begin{columns}
		\begin{column}{0.5\textwidth}

			\begin{itemize}
				\item      
			\end{itemize}

			
		\end{column}

		\begin{column}{0.5\textwidth}
	$$\includegraphics[width=1\textwidth]{../output/noGPLimages/kap5x24}$$
		\end{column}
	\end{columns}
\end{frame}
\begin{frame}
	\frametitle{Fasemodulering (PM) av dititalt signal}
	\begin{columns}
		\begin{column}{0.5\textwidth}

			\begin{itemize}
				\item      
			\end{itemize}

			
		\end{column}

		\begin{column}{0.5\textwidth}
	$$\includegraphics[width=1\textwidth]{../output/noGPLimages/kap5x25}$$
		\end{column}
	\end{columns}
\end{frame}
\begin{frame}
	\frametitle{Datasikkerhet}
	\begin{columns}
		\begin{column}{0.5\textwidth}

			\begin{itemize}
				\item Autentisering
				\item Digital dømmekraft
				\item Kryptering
				\item Brannmur
				\item VPN (Virtuelt privat nettverk)
				\item IT-nettverk (informasjonsteknologisk nettverk)
				\item OT-nettverk (opersasjonsteknologisk nettverk)
				\item DMZ
			\end{itemize}

			
		\end{column}

		\begin{column}{0.5\textwidth}
	$$\includegraphics[height=0.8\textheight]{../output/noGPLimages/kap5x26}$$
		\end{column}
	\end{columns}
\end{frame}
\begin{frame}
	\frametitle{Smarte instrumenter og HART}
	\begin{columns}
		\begin{column}{0.5\textwidth}

			\begin{itemize}
				\item      
			\end{itemize}

			
		\end{column}

		\begin{column}{0.5\textwidth}
	$$\includegraphics[width=1\textwidth]{../output/noGPLimages/kap5x27}$$
		\end{column}
	\end{columns}
\end{frame}
\begin{frame}
	\frametitle{HART-protokoll}
	\begin{columns}
		\begin{column}{0.5\textwidth}

			\begin{itemize}
				\item      
			\end{itemize}

			
		\end{column}

		\begin{column}{0.5\textwidth}
	$$\includegraphics[width=1\textwidth]{../output/noGPLimages/kap5x28}$$
		\end{column}
	\end{columns}
\end{frame}
\begin{frame}
	\frametitle{Håndterminaler og vedlikehold}
	\begin{columns}
		\begin{column}{0.5\textwidth}

	$$\includegraphics[width=1\textwidth]{../output/noGPLimages/kap5x30}$$

			
		\end{column}

		\begin{column}{0.5\textwidth}
	$$\includegraphics[width=1\textwidth]{../output/noGPLimages/kap5x29}$$
		\end{column}
	\end{columns}
\end{frame}
\begin{frame}
	\frametitle{ISA100 Wireless og WirelessHART}
	\begin{columns}
		\begin{column}{0.5\textwidth}

			\begin{itemize}
				\item Åpen universell IPv6 trådløs nettverksprotokoll.
				\item 2.4 GHz-båndet
				\item kryptert med AES128
				\item Maskenettverk
			\end{itemize}

			
		\end{column}

		\begin{column}{0.5\textwidth}
	$$\includegraphics[width=1\textwidth]{../output/noGPLimages/kap5x31}$$
		\end{column}
	\end{columns}
\end{frame}
\begin{frame}
	\frametitle{Alternativ rute i maskenett med WirelessHART}
	\begin{columns}
		\begin{column}{0.5\textwidth}

			\begin{itemize}
				\item      
			\end{itemize}

			
		\end{column}

		\begin{column}{0.5\textwidth}
	$$\includegraphics[width=1\textwidth]{../output/noGPLimages/kap5x32}$$
		\end{column}
	\end{columns}
\end{frame}
\begin{frame}
	\frametitle{Kontrollsystemet koblet opp mot WirelessHART}
	\begin{columns}
		\begin{column}{0.5\textwidth}

			\begin{itemize}
				\item      
			\end{itemize}

			
		\end{column}

		\begin{column}{0.5\textwidth}
	$$\includegraphics[height=0.8\textheight]{../output/noGPLimages/kap5x33}$$
		\end{column}
	\end{columns}
\end{frame}
\begin{frame}
	\frametitle{Radiokommunikasjon}
	\begin{columns}
		\begin{column}{0.5\textwidth}

			\begin{itemize}
				\item      
			\end{itemize}

			
		\end{column}

		\begin{column}{0.5\textwidth}
	$$\includegraphics[width=1\textwidth]{../output/noGPLimages/kap5x34}$$
		\end{column}
	\end{columns}
\end{frame}
\begin{frame}
	\frametitle{Frekvensbånd som brukes til radiokommunikasjon}
	\begin{columns}
		\begin{column}{0.5\textwidth}

			\begin{itemize}
				\item      
			\end{itemize}

			
		\end{column}

		\begin{column}{0.5\textwidth}
	$$\includegraphics[width=1\textwidth]{../output/noGPLimages/kap5x35}$$
		\end{column}
	\end{columns}
\end{frame}
\begin{frame}
	\frametitle{Antenner}
	\begin{columns}
		\begin{column}{0.5\textwidth}

			\begin{itemize}
				\item      
			\end{itemize}

			
		\end{column}

		\begin{column}{0.5\textwidth}
	$$\includegraphics[width=1\textwidth]{../output/noGPLimages/kap5x36}$$
		\end{column}
	\end{columns}
\end{frame}
\begin{frame}
	\frametitle{Mobilnett}
	\begin{columns}
		\begin{column}{0.5\textwidth}

			\begin{itemize}
				\item Mobilnett brukes mye for å fjernstyring av fjerntliggende anlegg
				\item G står for generasjon. 

			\end{itemize}

			
		\end{column}

		\begin{column}{0.5\textwidth}
	$$\includegraphics[width=1\textwidth]{../output/noGPLimages/kap5x37}$$
		\end{column}
	\end{columns}
\end{frame}
\begin{frame}
	\frametitle{LoRaWAN}
	\begin{columns}
		\begin{column}{0.5\textwidth}

			\begin{itemize}
				\item \url{https://www.youtube.com/watch?v=hMOwbNUpDQA}     
				\item Alternativ til mobilnet for å styre fjerntliggende anlegg. 
			\end{itemize}

			
		\end{column}

		\begin{column}{0.5\textwidth}
	$$\includegraphics[width=1\textwidth]{../output/noGPLimages/kap5x38}$$
		\end{column}
	\end{columns}
\end{frame}
\begin{frame}
	\frametitle{Dtabuss og busskabler med konnektor}
	\begin{columns}
		\begin{column}{0.5\textwidth}

			\begin{itemize}
				\item En databuss i automatiseringssystemer er en kabel med to eller flere ledninger osm transporterer data til og fra insturmenter og andre enheter. 
			
			\end{itemize}

			
		\end{column}

		\begin{column}{0.5\textwidth}
	$$\includegraphics[width=1\textwidth]{../output/noGPLimages/kap5x39}$$
		\end{column}
	\end{columns}
\end{frame}
\begin{frame}
	\frametitle{Proffibus kabel}
	$$\includegraphics[width=1\textwidth]{../output/noGPLimages/kap5x40}$$
\end{frame}
\begin{frame}
	\frametitle{9-pin D-sub (DB9-konnektor)}
	\begin{columns}
		\begin{column}{0.5\textwidth}

	$$\includegraphics[width=1\textwidth]{../output/noGPLimages/kap5x41}$$

			
		\end{column}

		\begin{column}{0.5\textwidth}
	$$\includegraphics[width=1\textwidth]{../output/noGPLimages/kap5x42}$$
		\end{column}
	\end{columns}
\end{frame}
\begin{frame}
	\frametitle{9-pin D-sub (DB9-konnektor)}
	$$\includegraphics[width=1\textwidth]{../output/noGPLimages/kap5x43}$$
\end{frame}
\begin{frame}
	\frametitle{Seriell kommunikasjonsstandard}
	\begin{columns}
		\begin{column}{0.5\textwidth}

			\begin{itemize}
				\item En RS (Recommended standard) for dataoverføring har standarisert fysiske, elektriske og logiske gransesnitt 
			\end{itemize}

			
		\end{column}

		\begin{column}{0.5\textwidth}
	$$\includegraphics[width=1\textwidth]{../output/noGPLimages/kap5x44}$$
		\end{column}
	\end{columns}
\end{frame}
\begin{frame}
	\frametitle{Ulike RS-standarder}
	$$\includegraphics[width=1\textwidth]{../output/noGPLimages/kap5x45}$$
\end{frame}
\begin{frame}
	\frametitle{USB til RS-232 omformer}
	$$\includegraphics[width=1\textwidth]{../output/noGPLimages/kap5x46}$$
\end{frame}
\begin{frame}
	\frametitle{Pinnekonfigurering med DB9 og DB25}
	$$\includegraphics[width=1\textwidth]{../output/noGPLimages/kap5x47}$$
\end{frame}
\begin{frame}
	\frametitle{Typisk RS-232 kobling i automatiserte anlegg}
	$$\includegraphics[width=0.5\textwidth]{../output/noGPLimages/kap5x48}$$
\end{frame}
\begin{frame}
	\frametitle{Koblingsskjema for en RS-232 oppkobling }
	$$\includegraphics[width=1\textwidth]{../output/noGPLimages/kap5x49}$$
\end{frame}
\begin{frame}
	\frametitle{Skjerming og jording av signalkobel}
	\begin{columns}
		\begin{column}{0.5\textwidth}

			\begin{itemize}
				\item RS-232 er beregnet på langsom dataoverføring
				\item Det stilles ikke høye krav til skjerming
				\item Skermin kan redusere problemer med EMI
				\item Ulike jord potesnial kan gi problemer med støy
			\end{itemize}

			
		\end{column}

		\begin{column}{0.5\textwidth}
		\end{column}
	\end{columns}
\end{frame}
\begin{frame}
	\frametitle{Baudrate}
	\begin{columns}
		\begin{column}{0.5\textwidth}

			\begin{itemize}
				\item antall linjeskift pr.sekund. 
				\item om et linjeskift tilsvarer 2 bit, vil en baudrate på 19200 gi en bitrate på 38400 bit/s
				\item 
			\end{itemize}

			
		\end{column}

		\begin{column}{0.5\textwidth}
		\end{column}
	\end{columns}
\end{frame}
\begin{frame}
	\frametitle{RS-422 og RS-485 balanserte strømsløyfer}
	\begin{columns}
		\begin{column}{0.5\textwidth}

			\begin{itemize}
				\item Industristandarder for seriekommunikajson
				\item Benytter termineringsmotstand i hver ende av kabelen
				\item mulig å benytte multidrop
			\end{itemize}

			
		\end{column}

		\begin{column}{0.5\textwidth}
	$$\includegraphics[width=1\textwidth]{../output/noGPLimages/kap5x50}$$
		\end{column}
	\end{columns}
\end{frame}
\begin{frame}
	\frametitle{DB9 plugg med termineringsmotstand som kan kobles av og på}
	$$\includegraphics[width=0.3\textwidth]{../output/noGPLimages/kap5x51}$$
\end{frame}
\begin{frame}
	\frametitle{RS-485 med multidrop}
	$$\includegraphics[width=1\textwidth]{../output/noGPLimages/kap5x52}$$
\end{frame}
\begin{frame}
	\frametitle{Flere regulatorer koblet til RS-485}
	$$\includegraphics[width=1\textwidth]{../output/noGPLimages/kap5x53}$$
\end{frame}
\begin{frame}
	\frametitle{Kommunikasjonsprotokoller i boligautomatisering}
	$$\includegraphics[width=1\textwidth]{../output/noGPLimages/kap5x54}$$
\end{frame}
\begin{frame}
	\frametitle{Z-wave}
	\begin{columns}
		\begin{column}{0.5\textwidth}

			\begin{itemize}
				\item Proporitær standard som alle kan kan utvikle produkter til. 
				\item 800-900 MHz
				\item Dataoverføringshastighet (9.6, 40 eller 100kbit/s
				\item mesh nettverkstopologi
			\end{itemize}

			
		\end{column}

		\begin{column}{0.5\textwidth}
	$$\includegraphics[width=1\textwidth]{../output/noGPLimages/kap5x55}$$
		\end{column}
	\end{columns}
\end{frame}
\begin{frame}
	\frametitle{Egenskaper ved ulike generasjoner av z-wave}
	$$\includegraphics[width=1\textwidth]{../output/noGPLimages/kap5x56}$$
\end{frame}
\begin{frame}
	\frametitle{ZigBee}
	$$\includegraphics[width=1\textwidth]{../output/noGPLimages/kap5x57}$$
\end{frame}
\begin{frame}
	\frametitle{Industrielle kommnunikasjonsprotokoller}
	\begin{columns}
		\begin{column}{0.5\textwidth}

			\begin{itemize}
				\item Opprineling laget for å spare kabling. 
				\item datainnsamling
				\item fjernkonfigurering
				\item fjern kalibrering
				\item diganose
				\item feilsøking
			\end{itemize}

			
		\end{column}

		\begin{column}{0.5\textwidth}
	$$\includegraphics[width=1\textwidth]{../output/noGPLimages/kap5x58}$$
		\end{column}
	\end{columns}
\end{frame}
\begin{frame}
	\frametitle{Fagrikkautomatisering og prosessautomatisering}
	$$\includegraphics[width=0.7\textwidth]{../output/noGPLimages/kap5x59}$$
\end{frame}
\begin{frame}
	\frametitle{ASi-buss}
	$$\includegraphics[width=1\textwidth]{../output/noGPLimages/kap5x60}$$
\end{frame}
\begin{frame}
	\frametitle{Tilkobling til  ASi-bussen}
	\begin{columns}
		\begin{column}{0.5\textwidth}

			\begin{itemize}
				\item En kan koble seg til hvor som helts på en ASi-buss. 
			\end{itemize}

			
		\end{column}

		\begin{column}{0.5\textwidth}
	$$\includegraphics[width=1\textwidth]{../output/noGPLimages/kap5x61}$$
		\end{column}
	\end{columns}
\end{frame}
\begin{frame}
	\frametitle{Tilkobling til  ASi-bussen}
	\begin{columns}
		\begin{column}{0.5\textwidth}

			\begin{itemize}
				\item To koblingspinner stikker inni kablen og skaper kontakt
				\item Om en kobling fjernes lukkes hullene nesten helt. 
			\end{itemize}

			
		\end{column}

		\begin{column}{0.5\textwidth}
	$$\includegraphics[width=1\textwidth]{../output/noGPLimages/kap5x62}$$
		\end{column}
	\end{columns}
\end{frame}
\begin{frame}
	\frametitle{IO-link}
	\begin{columns}
		\begin{column}{0.5\textwidth}


			
		\end{column}

		\begin{column}{0.5\textwidth}
	$$\includegraphics[width=1\textwidth]{../output/noGPLimages/kap5x63}$$
		\end{column}
	\end{columns}
\end{frame}
\begin{frame}
	\frametitle{Master/Slave system}
	$$\includegraphics[width=1\textwidth]{../output/noGPLimages/kap5x64}$$
\end{frame}
\begin{frame}
	\frametitle{Master/Slave system}
	\begin{columns}
		\begin{column}{0.5\textwidth}
\Huge Master kalling
\normalsize
	$$\includegraphics[width=1\textwidth]{../output/noGPLimages/kap5x65}$$
	$$\includegraphics[width=1\textwidth]{../output/noGPLimages/kap5x66}$$
			
		\end{column}

		\begin{column}{0.5\textwidth}
\Huge Slave svar
\normalsize
	$$\includegraphics[width=1\textwidth]{../output/noGPLimages/kap5x68}$$
	$$\includegraphics[width=1\textwidth]{../output/noGPLimages/kap5x67}$$
		\end{column}
	\end{columns}
\end{frame}
\begin{frame}
	\frametitle{Master/Slave system}
	$$\includegraphics[width=1\textwidth]{../output/noGPLimages/kap5x69}$$
\end{frame}
\begin{frame}
	\frametitle{Nettverkstopologier og avstander med ASi-buss}
	$$\includegraphics[width=1\textwidth]{../output/noGPLimages/kap5x70}$$
\end{frame}
\begin{frame}
	\frametitle{Distrubusjon av 24V DC til aktuatorer}
	\begin{columns}
		\begin{column}{0.5\textwidth}

			\begin{itemize}
				\item      
			\end{itemize}

			
		\end{column}

		\begin{column}{0.5\textwidth}
	$$\includegraphics[width=1\textwidth]{../output/noGPLimages/kap5x71}$$
		\end{column}
	\end{columns}
\end{frame}
\begin{frame}
	\frametitle{Strømforsyningsenhet}
	\begin{columns}
		\begin{column}{0.5\textwidth}

			\begin{itemize}
				\item      
			\end{itemize}

			
		\end{column}

		\begin{column}{0.5\textwidth}
	$$\includegraphics[width=1\textwidth]{../output/noGPLimages/kap5x72}$$
		\end{column}
	\end{columns}
\end{frame}
\begin{frame}
	\frametitle{Mastermodul}
	\begin{columns}
		\begin{column}{0.5\textwidth}

			\begin{itemize}
				\item      
			\end{itemize}

			
		\end{column}

		\begin{column}{0.5\textwidth}
	$$\includegraphics[width=1\textwidth]{../output/noGPLimages/kap5x73}$$
		\end{column}
	\end{columns}
\end{frame}
\begin{frame}
	\frametitle{Safety at work}
	\begin{columns}
		\begin{column}{0.5\textwidth}

			\begin{itemize}
				\item      
			\end{itemize}

			
		\end{column}

		\begin{column}{0.5\textwidth}
	$$\includegraphics[width=1\textwidth]{../output/noGPLimages/kap5x74}$$
		\end{column}
	\end{columns}
\end{frame}
\begin{frame}
	\frametitle{IO-link}
	\begin{columns}
		\begin{column}{0.5\textwidth}

			\begin{itemize}
				\item Standarisert i IEC 61131-3
				\item toveis kommunikasjon med sensorer
				\item IO-link sensorer og aktuatorer
				\item IO-link master eller PLS
				\item Maks lengde på kabel er 20m, krever ofte IO-link master i felt. 
			\end{itemize}

			
		\end{column}

		\begin{column}{0.5\textwidth}
	$$\includegraphics[width=1\textwidth]{../output/noGPLimages/kap5x75}$$
		\end{column}
	\end{columns}
\end{frame}
\begin{frame}
	\frametitle{Eksempel på akritektur for et IO-link system}
	\begin{columns}
		\begin{column}{0.5\textwidth}

			\begin{itemize}
				\item      
			\end{itemize}

			
		\end{column}

		\begin{column}{0.5\textwidth}
	$$\includegraphics[width=1\textwidth]{../output/noGPLimages/kap5x76}$$
		\end{column}
	\end{columns}
\end{frame}
\begin{frame}
	\frametitle{Konfigurering av IO-Link med nettbrett eller mobiltelefon}
	\begin{columns}
		\begin{column}{0.5\textwidth}

			\begin{itemize}
				\item      
			\end{itemize}

			
		\end{column}

		\begin{column}{0.5\textwidth}
	$$\includegraphics[width=1\textwidth]{../output/noGPLimages/kap5x77}$$
		\end{column}
	\end{columns}
\end{frame}
\begin{frame}
	\frametitle{Fondation fieldbus (FF)}

Dataoverføring med RS-485 eller FF High-speed Ethernet

			

	$$\includegraphics[width=1\textwidth]{../output/noGPLimages/kap5x78}$$
\end{frame}
\begin{frame}
	\frametitle{Modbus}
	\begin{columns}
		\begin{column}{0.5\textwidth}

			\begin{itemize}
				\item utgitt av modicon i 1979, elste kommnunikasjonsprotokoll innenfor automatisering
				\item er mye utbredt
				\item punkt til punkt og multidrop
				\item Modbus ASCII, Modbus Plus, Modbus RTU og Modubs TCP
			\end{itemize}

			
		\end{column}

		\begin{column}{0.5\textwidth}
	$$\includegraphics[width=1\textwidth]{../output/noGPLimages/kap5x79}$$
		\end{column}
	\end{columns}
\end{frame}
\begin{frame}
	\frametitle{Modbus dataoverføring med og uten feil}
	$$\includegraphics[width=1\textwidth]{../output/noGPLimages/kap5x80}$$
\end{frame}
\begin{frame}
	\frametitle{Profibus DP (decentralized periphery}
	$$\includegraphics[width=1\textwidth]{../output/noGPLimages/kap5x81}$$
\end{frame}
\begin{frame}
	\frametitle{Hastighet og distanse på Profibus DP}
	$$\includegraphics[height=0.8\textheight]{../output/noGPLimages/kap5x82}$$
\end{frame}
%\begin{frame}
%	\frametitle{Profinet}
%	$$\includegraphics[width=0.7\textwidth]{../output/noGPLimages/kap5x83}$$
%\end{frame}
\begin{frame}
	\frametitle{Profinet}
	$$\includegraphics[width=0.7\textwidth]{../output/noGPLimages/kap5x84}$$
\end{frame}
\begin{frame}
	\frametitle{EtherCAT}
	\begin{columns}
		\begin{column}{0.5\textwidth}

			\begin{itemize}
				\item Etnernetbasert tilpasset industriell automasjon. 
				\item baserer seg på det fysiske laget og en standard ramme som bekrevet i IEEE 802.3 Ethernet Standard. 
				\item Tar hensyn til responstid, minimalt datakrav for hver enhet og lave implementerinskostnader. 
				\item EtherCAT sender ut en dataramme som går gjennom alle nodene
			\end{itemize}

			
		\end{column}

		\begin{column}{0.5\textwidth}
			\href{https://www.youtube.com/watch?v=z2OagcHG-UU}{EtherCAT Video}
		\end{column}
	\end{columns}
\end{frame}
\begin{frame}
	\frametitle{OPC UA (Open Platform Communication Unified Arcitecture}
	\begin{columns}
		\begin{column}{0.5\textwidth}

				Åpenstandard for kommunikasjon mellom:
			\begin{itemize}
				\item PC
				\item PLS
				\item microkontrollere
				\item RIO
			\end{itemize}

			
		\end{column}

		\begin{column}{0.5\textwidth}
	$$\includegraphics[width=1\textwidth]{../output/noGPLimages/kap5x86}$$
		\end{column}
	\end{columns}
\end{frame}
\begin{frame}
	\frametitle{OPC UA (Open Platform Communication Unified Arcitecture}
	\begin{columns}
		\begin{column}{0.5\textwidth}

			OPC UA har:
			\begin{itemize}
				\item Punkt til punkt kommunikasjon
				\item En til mange kommunikasjon såkalt pub/sub
			\end{itemize}

			
		\end{column}

		\begin{column}{0.5\textwidth}
	$$\includegraphics[width=1\textwidth]{../output/noGPLimages/kap5x87}$$
		\end{column}
	\end{columns}
\end{frame}
\begin{frame}
	\frametitle{MQTT}
	\begin{columns}
		\begin{column}{0.5\textwidth}

			\begin{itemize}
				\item Laget for lett meldingstransport fra utgiver til abonnement PubSub. 
				\item Ideell for sammenkobling av fjerntliggende enheter med liten båndbredde
			\end{itemize}

			
		\end{column}

		\begin{column}{0.5\textwidth}
	$$\includegraphics[width=1\textwidth]{../output/noGPLimages/kap5x88}$$
		\end{column}
	\end{columns}
\end{frame}
\end{document}
