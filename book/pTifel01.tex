% ta med damping på DP-delle

\documentclass[aspectratio=169,xcolor=dvipsnames]{beamer}
%\usetheme{SimplePlus}

\usepackage{hyperref}
\usepackage{graphicx} % Allows including images
\usepackage{booktabs} % Allows the use of \toprule, \midrule and \bottomrule in tables

%----------------------------------------------------------------------------------------
%	TITLE PAGE
%----------------------------------------------------------------------------------------

\title[Industrielektronikk]{Industrielektronikk} % The short title appears at the bottom of every slide, the full title is only on the title page
%\subtitle{Subtitle}

\author[Fred-Olav] {Fred-Olav Mosdal}

\institute[Gand VGS] % Your institution as it will appear on the bottom of every slide, may be shorthand to save space
{
    Gand VGS \\
    VG1 TIF }
\date{\today} % Date, can be changed to a custom date


%----------------------------------------------------------------------------------------
%	PRESENTATION SLIDES
%----------------------------------------------------------------------------------------

\begin{document}
\begin{frame}
\titlepage
\end{frame}




\begin{frame}
	\frametitle{Elektrisitet}

	\begin{columns}
		\begin{column}{0.5\textwidth}
			Mål:
			\begin{itemize}
				\item vite hvordan elektrisitet oppfører seg
				\item kunne beregne vanlige elektriske størrelser
				\item kunne måle elektriske størrelser
				\item kunne stille inn et mulitmeter for måling av strøm, spenning og resistans. 
				\item kjenne til materialet i en leder
			\end{itemize}
		\end{column}

		\begin{column}{0.5\textwidth}
			$$\includegraphics[width=0.8\textwidth]{../output/noGPLimages/tifel01.png}$$
		\end{column}
	\end{columns}
\end{frame}

\begin{frame}
	\frametitle{Bohrs atommodell}

	\begin{columns}
		\begin{column}{0.5\textwidth}
			\begin{itemize}
				\item Alessandro Volta konstruerte første batteri
				\item Bohrs atommodell ga en bedre forståelse av elektrisitet
			\end{itemize}
		\end{column}

		\begin{column}{0.5\textwidth}
			$$\includegraphics[width=1\textwidth]{../output/noGPLimages/tifel02.png}$$
		\end{column}
	\end{columns}
\end{frame}
\begin{frame}
	\frametitle{Produksjons av elektrisk strøm}

	\begin{columns}
		\begin{column}{0.333\textwidth}
			\begin{itemize}
				\item Vannkraftverk
				\item Vindkraftverk
				\item Gasskraftverk
			\end{itemize}
		\end{column}

		\begin{column}{0.7\textwidth}
			$$\includegraphics[width=0.5\textwidth]{../output/noGPLimages/tyssedal-kfraftverk.jpg}\includegraphics[width=0.5\textwidth]{../output/noGPLimages/vindkraft.jpg}$$
			$$\includegraphics[width=0.5\textwidth]{../output/noGPLimages/naturkraft.jpg}$$
		\end{column}
	\end{columns}
\end{frame}

\begin{frame}
	\frametitle{Elektrisk strøm}

	\begin{columns}
		\begin{column}{0.5\textwidth}
			\begin{itemize}
				\item Elektrisk vekselstrøm
			\end{itemize}
		\end{column}

		\begin{column}{0.5\textwidth}
			$$\includegraphics[width=1\textwidth]{../output/noGPLimages/tifel03.png}$$
		\end{column}
	\end{columns}
\end{frame}

\begin{frame}
	\frametitle{Elektrisk strøm}

	\begin{columns}
		\begin{column}{0.5\textwidth}
			\begin{itemize}
				\item Elektrisk likestrøm
			\end{itemize}
		\end{column}

		\begin{column}{0.5\textwidth}
			$$\includegraphics[width=1\textwidth]{../output/noGPLimages/tifel04.png}$$
		\end{column}
	\end{columns}
\end{frame}

\begin{frame}
	\frametitle{Elektrisk Spenning}

	\begin{columns}
		\begin{column}{0.5\textwidth}
			\begin{itemize}
				\item Likespenning
			\end{itemize}
		\end{column}

		\begin{column}{0.5\textwidth}
			$$\includegraphics[width=1\textwidth]{../output/noGPLimages/tifel05.png}$$
		\end{column}
	\end{columns}
\end{frame}

\begin{frame}
	\frametitle{Elektrisk spenning}

	\begin{columns}
		\begin{column}{0.5\textwidth}
			\begin{itemize}
				\item Vekselspenning
			\end{itemize}
		\end{column}

		\begin{column}{0.5\textwidth}
			$$\includegraphics[width=1\textwidth]{../output/noGPLimages/tifel06.png}$$
		\end{column}
	\end{columns}
\end{frame}

\begin{frame}
	\frametitle{Vekselspenning og vekselstrøm}

	\begin{columns}
		\begin{column}{0.5\textwidth}
			\begin{itemize}
				\item Sinusformet spenning og strøm
			\end{itemize}
		\end{column}

		\begin{column}{0.5\textwidth}
			$$\includegraphics[width=1\textwidth]{../output/noGPLimages/tifel07.png}$$
		\end{column}
	\end{columns}
\end{frame}

\begin{frame}
	\frametitle{Generering av vekselspenning}

	\begin{columns}
		\begin{column}{0.2\textwidth}
			\begin{itemize}
				\item Spenning fra en magnet som roterer i nærheten av en spole 
			\end{itemize}
		\end{column}

		\begin{column}{0.8\textwidth}
			$$\includegraphics[width=1\textwidth]{../output/noGPLimages/tifel08.png}$$
		\end{column}
	\end{columns}

\end{frame}
\begin{frame}
	\frametitle{Måling av vekselspenning}

	\begin{columns}
		\begin{column}{0.5\textwidth}
			\begin{itemize}
				\item Multimeteret stilles på AC eller ~V
				\item Pinnene kobles til com og V
			\end{itemize}
		\end{column}

		\begin{column}{0.5\textwidth}
			$$\includegraphics[height=0.8\textheight]{../output/noGPLimages/tifel10.png}$$
		\end{column}
	\end{columns}
\end{frame}

\begin{frame}
	\frametitle{Likespenning og likestrøm}

			$$\includegraphics[height=0.7\textheight]{../output/noGPLimages/tifel09.png}$$
\end{frame}


\begin{frame}
	\frametitle{Likespenning og likestrøm}

	\begin{columns}
		\begin{column}{0.3\textwidth}
			\begin{itemize}
				\item Lampe koblet til batter med fysisk oppkobling og elektrisk skjema
			\end{itemize}
		\end{column}

		\begin{column}{0.7\textwidth}
			$$\includegraphics[width=1\textwidth]{../output/noGPLimages/tifel11.png}$$
		\end{column}
	\end{columns}
\end{frame}

\begin{frame}
	\frametitle{Resistans (motstand mot elektrisk strøm)}

	\begin{columns}
		\begin{column}{0.5\textwidth}
			\begin{itemize}
				\item Resistans er motstand mot elektrisk strøm, den gjør det vanskelig for spenningen å få strømmen igjennom kretsen
			\end{itemize}
			$$\includegraphics[width=1\textwidth]{../output/noGPLimages/ohmslow.png}$$
		\end{column}

		\begin{column}{0.5\textwidth}
			$$\includegraphics[width=1\textwidth]{../output/noGPLimages/tifel12.png}$$
		\end{column}
	\end{columns}
\end{frame}

\begin{frame}
	\frametitle{Resistans (motstand mot elektrisk strøm)}

	\begin{columns}
		\begin{column}{0.5\textwidth}
			\begin{itemize}
				\item Lydpærer er et eksempel på motstand mot elektrisk strøm
			\end{itemize}
		\end{column}

		\begin{column}{0.5\textwidth}
			$$\includegraphics[width=1\textwidth]{../output/noGPLimages/tifel13.png}$$
		\end{column}
	\end{columns}
\end{frame}

\begin{frame}
	\frametitle{Resistans (motstand mot elektrisk strøm)}

	\begin{columns}
		\begin{column}{0.5\textwidth}
			\begin{itemize}
				\item En glødepære var en vanlig komponent med resistans. 
				\item Nå er det snart ikke lov og selge den. 
				\item Mesteparten av energien i en glødepære blir til varme og ikke lys
			\end{itemize}
		\end{column}

		\begin{column}{0.5\textwidth}
			$$\includegraphics[width=0.6\textwidth]{../output/noGPLimages/tifel14.png}$$
		\end{column}
	\end{columns}
\end{frame}

\begin{frame}
	\frametitle{Måling av likespenning med multimeter}

	\begin{columns}
		\begin{column}{0.3\textwidth}
			\begin{itemize}
				\item multimeteret stilles på DC
				\item pinnene kobles til com og V på multimetereret 
				\item en måler over komponenten en ønsker å finne spenningen \textbf{\textit{over}}. 
			\end{itemize}
		\end{column}

		\begin{column}{0.7\textwidth}
			$$\includegraphics[width=1\textwidth]{../output/noGPLimages/tifel15.png}$$
		\end{column}
	\end{columns}
\end{frame}

\begin{frame}
	\frametitle{Måling av likespenning med mulitmeter}

	\begin{columns}
		\begin{column}{0.3\textwidth}
			\begin{itemize}
				\item mulitmeteres stilles på den strømstyrken en ønsker
				\item svart pinne kobles til com. Rød pinne kobles til A med rett strømstyrke
				\item pinnene kobles i serie med komponenten en ønsker å måle strømmen \textbf{\textit{igjennom}}
			\end{itemize}
		\end{column}

		\begin{column}{0.5\textwidth}
			$$\includegraphics[width=1\textwidth]{../output/noGPLimages/tifel16.png}$$
		\end{column}
	\end{columns}
\end{frame}

\begin{frame}
	\frametitle{Test av målepinnene}

	\begin{columns}
		\begin{column}{0.3\textwidth}
			\begin{itemize}
				\item Før du gjør noen målinger med et mulitimeter er det en god vane og sjekke om målepinnene virker.
			\end{itemize}
		\end{column}

		\begin{column}{0.7\textwidth}
			$$\includegraphics[width=1.1\textwidth]{../output/noGPLimages/tifel17.png}$$
		\end{column}
	\end{columns}
\end{frame}

\begin{frame}
	\frametitle{Måling av resistans med mulitimeter}

	\begin{columns}
		\begin{column}{0.4\textwidth}
			\begin{itemize}
				\item mulitimeteret settes på $\Omega$ elelr ohm
				\item svart pinne kobles til com og rød pinne til $\Omega$
				\item Komponenten en skal måle kobles ut av kretsen og pinnene settes over denne. 
				\item mulitimeteret sender en liten strøm for å måle resistans, derfor må vi koble fra komponenten. 
			\end{itemize}
		\end{column}

		\begin{column}{0.6\textwidth}
			$$\includegraphics[width=1\textwidth]{../output/noGPLimages/tifel18.png}$$
		\end{column}
	\end{columns}
\end{frame}

\begin{frame}
	\frametitle{Elektriske ledere}

	\begin{columns}
		\begin{column}{0.7\textwidth}
			\begin{itemize}
				\item Elektriske ledere en en form for resistans, men her ønsker vi at resistansen skal være minst mulig
				\item Metaller som leder godt og har en ok pris brukes til ledere i strømkabler. 
				\item Resistansen et metall har kalles resistivitet og er definert som reistansen 1 meter med 1mm² av materialet har. 
			\end{itemize}
		\end{column}

		\begin{column}{0.3\textwidth}
			$$\includegraphics[width=1\textwidth]{../output/noGPLimages/tifel19.png}$$
		\end{column}
	\end{columns}
\end{frame}

\begin{frame}
	\frametitle{Resistans i elektriske ledere}

	\begin{columns}
		\begin{column}{0.5\textwidth}
			\begin{itemize}
				\item $R=\rho \cdot \dfrac{l}{A}$
			\end{itemize}
		\end{column}

		\begin{column}{0.5\textwidth}
			$$\includegraphics[width=1\textwidth]{../output/noGPLimages/tifel20.png}$$
		\end{column}
	\end{columns}
\end{frame}

\begin{frame}
	\frametitle{Elektrisk kabel}

	\begin{columns}
		\begin{column}{0.5\textwidth}
			\begin{itemize}
				\item når vi samler en eller flere ledere i en felles kappe kalles det en kabel. 
			\end{itemize}
		\end{column}

		\begin{column}{0.5\textwidth}
			$$\includegraphics[width=1\textwidth]{../output/noGPLimages/tifel21.png}$$
		\end{column}
	\end{columns}
\end{frame}

\end{document}
