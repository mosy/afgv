
\section{Utstyr}
\subsection{Gandsfjorden Gondol RIO}

Kortet kobler seg opp mot MCI2 nettet og har IP-adressen som vises på kassen. 

\includegraphics[width=1\textwidth]{./GandsfjordenGondolRIO.jpg}
\small
\begin{tabular}{|c|l|l|c|}
\hline 
\multicolumn{4}{|c|}{Modbus Data Table}\tabularnewline
\hline 
\hline 
	IO & Tilkobling & Funksjonskode & Register \tabularnewline
\hline 
	DI1 & Sinking inngang & 2 (Read Discrete Input) & 0001H \tabularnewline
\hline 
	DI2 & Sinking inngang & 2 (Read Discrete Input) & 0002H \tabularnewline
\hline 
	DI3 & Sinking inngang & 2 (Read Discrete Input) & 0003H \tabularnewline
\hline 
	DI4 & Sinking inngang & 2 (Read Discrete Input) & 0004H \tabularnewline
\hline 
	DI5 & Sinking inngang & 2 (Read Discrete Input) & 0005H \tabularnewline
\hline 
	DI6 & Sinking inngang & 2 (Read Discrete Input) & 0006H \tabularnewline
\hline 
	DI7 & Sinking inngang & 2 (Read Discrete Input) & 0007H \tabularnewline
\hline 
	DI8 & Sinking inngang & 2 (Read Discrete Input) & 0008H \tabularnewline
\hline 
	DI9 & Sinking inngang & 2 (Read Discrete Input) & 0009H \tabularnewline
\hline 
	DI10 & Sinking inngang & 2 (Read Discrete Input) & 000AH0\tabularnewline
\hline 
	DO1 & Bryter mot COM & 5 (Write Single Coil) & 0001H \tabularnewline
\hline 
	DO2 & Bryter mot COM & 5 (Write Single Coil) & 0002H \tabularnewline
\hline 
	DO3 & Bryter mot COM & 5 (Write Single Coil) & 0003H \tabularnewline
\hline 
	DO4 & Bryter mot COM & 5 (Write Single Coil) & 0004H \tabularnewline
\hline 
	AI1 & Sinking 4-20mA & 4 (Read Input Register) & 0001H \tabularnewline
\hline 
	AI2 & Sinking 4-20mA & 4 (Read Input Register) & 0002H \tabularnewline
\hline 
\multicolumn{4}{|l|}{Motor: kun et av registrene kan ha verdi over 0}\tabularnewline
\hline 
	Motor &  Motor CW & 4 (Write Single Holding) & 0001H \tabularnewline
\hline 
	Motor &  Motor CCW & 4 (Write Single Holding) & 0002H \tabularnewline
\hline 
\end{tabular}
\normalsize
\vfil \eject
\subsection{Gand RIO-trainer}

Kortet kobles til PLS med en USB kabel og en trenger en USB til seriel
driver (CH340). 

\includegraphics[width=1\textwidth]{./GandRioTrainer.jpg}
\small
\begin{tabular}{|c|c|c|c|}
\hline 
\multicolumn{4}{|c|}{Modbus Data Table}\tabularnewline
\hline 
\hline 
IO & Tilkobling & Register & \tabularnewline
\hline 
Encoder &  & 0000H & Enkoder telleverdi\tabularnewline
\hline 
DO1 & åpen kollektor & 0001H & bit0\tabularnewline
\hline 
DO2 & åpen kollektor & 0001H & bit1\tabularnewline
\hline 
DO3 & åpen kollektor & 0001H & bit2\tabularnewline
\hline 
DO4 & åpen kollektor & 0001H & bit3\tabularnewline
\hline 
PWM1 & åpen kollektor & 0002H & AO 0 (PWM)\tabularnewline
\hline 
PWM2 & åpen kollektor & 0003H & AO 1 (PWM)\tabularnewline
\hline 
PWM3 & åpen kollektor & 0004H & AO 2 (PWM)\tabularnewline
\hline 
DI1 & 24VDC & 0005H & bit0\tabularnewline
\hline 
DI2 & 24VDC & 0005H & bit1\tabularnewline
\hline 
DI3 & 24VDC & 0005H & bit2\tabularnewline
\hline 
DI4 & 24VDC & 0005H & bit3\tabularnewline
\hline 
DI5 & 24VDC & 0005H & bit4\tabularnewline
\hline 
DI6 & 24VDC & 0005H & bit5\tabularnewline
\hline 
AI2 & 4-24mA/0-5V & 0006H & AI2\tabularnewline
\hline 
AI1 & 4-24mA/0-5V & 0007H & AI1\tabularnewline
\hline 
\end{tabular}
\normalsize
\vfil \eject
\subsection{Wago PFC200 på stasjon 3}

Stasjon 3 \\
IP 192.168.0.3\\
web grensesnitt innlogging:\\
user:admin\\
pass:wago\\
\\\\
Codesys innlogging:\\
user:Elev3AUA\\
pass:3AUAGand\\
\section{Instrumenter}
\subsection{multimeter}
\subsection{loop claibrator}
\subsection{nettverkstester}
\subsection{megger}
\subsection{kontinuitetstester}





