
%(BEGIN_QUESTION)
% Copyright 2006, Tony R. Kuphaldt, released under the Creative Commons Attribution License (v 1.0)
% This means you may do almost anything with this work of mine, so long as you give me proper credit

%Suppose we are measuring the flow rate of a liquid using a Coriolis flowmeter, and the volumetric flow rate of the liquid increases (with liquid density remaining the same).  Will the amplitude of the meter tubes' ``undulating'' motion increase, decrease, or remain the same given this change in flow?  Will the meter tubes' resonant frequency of vibration increase, decrease, or remain the same?  Explain your answers.

Vi m{\aa}ler den volumetriske str{\o}mningsraten med et coriolis meter og str{\o}mningen {\o}ker men massetettheten er lik. Vil vibrasjonsfrekvensen p{\aa} r{\o}rene {\o}ke, minke eller forbli den samme? Vil den flaprende bevegelsen til r{\o}rene {\o}ke, minke eller forbli den samme?

\vskip 10pt

%Now suppose we are using the same Coriolis flowmeter to measure liquid flow, but this time the liquid's density becomes greater (i.e. the liquid becomes denser) with no change in volumetric flow.  Again, qualitatively identify the change in undulation amplitude, and also in resonant frequency, for the flowmeter's metal tubes, and explain your answers.

Om den volumetriske str{\o}mningen er konstant og massetettheten til væsken {\o}kser, hva vil da skje med den vibrasjonsfekvensen og den flaprende bevegelsen?

\vskip 10pt

%Finally, suppose the flow through this Coriolis meter stops completely.  How will changes in fluid density affect the tubes' motion, given a condition of zero flow?  Again, explain your answers.

Om str{\o}mningen st{\aa}r stille, hva vil da skje med vibrasjonsfekvensen og den flaprende bevegelsen?

\vskip 20pt \vbox{\hrule \hbox{\strut \vrule{} {\bf Suggestions for Socratic discussion} \vrule} \hrule}

\medskip
\item{$\bullet$} A strong emphasis is placed on performing ``thought experiments'' in this course.  Explain why this is.  What practical benefits might students realize from regular mental exercises such as this?
\medskip

\underbar{file i00728}
%(END_QUESTION)





%(BEGIN_ANSWER)

{\bf Increased volumetric flow rate with constant density:} the undulating motion of the tubes will {\it increase} in amplitude due to the greater inertial forces, but the resonant frequency of the tubes will {\it remain the same} because the tubes' mass has not changed.

\vskip 10pt

{\bf Increased density with constant volumetric flow rate:} the undulating motion of the tubes will {\it increase} in amplitude due to the greater inertial forces resulting from an increased mass flow rate, and the resonant frequency of the tubes will {\it decrease} due to increased tube mass.

\vskip 10pt

{\bf Changes in fluid density at zero flow:} there will be no undulating motion, because there will be no Coriolis force with zero flow.  The tubes' resonant frequency, however, will vary inversely with fluid density.  One practical caveat is that there will need to be {\it some} flow in order to push a new fluid of different density into the flowmeter's vibrating tubes, in order to sense that new density.

%(END_ANSWER)





%(BEGIN_NOTES)

Students often misunderstand how a Coriolis flowmeter measures mass flow rate and fluid density.  Questions such as these help expose conceptual misunderstandings.










\vfil \eject

\noindent
{\bf Prep Quiz:}

Coriolis flowmeters infer the mass flow rate of fluid by actively monitoring:

\medskip
\item{$\bullet$} The power required to oscillate the tubes
\vskip 5pt 
\item{$\bullet$} The length of the oscillating tubes
\vskip 5pt 
\item{$\bullet$} The phase shift of the oscillating tubes
\vskip 5pt 
\item{$\bullet$} The temperature of the oscillating tubes
\vskip 5pt 
\item{$\bullet$} The frequency of the oscillating tubes
\vskip 5pt 
\item{$\bullet$} The turbulence within the oscillating tubes
\medskip











\vfil \eject

\noindent
{\bf Prep Quiz:}

Coriolis flowmeters infer the density of fluid by actively monitoring:

\medskip
\item{$\bullet$} The power required to oscillate the tubes
\vskip 5pt 
\item{$\bullet$} The length of the oscillating tubes
\vskip 5pt 
\item{$\bullet$} The phase shift of the oscillating tubes
\vskip 5pt 
\item{$\bullet$} The temperature of the oscillating tubes
\vskip 5pt 
\item{$\bullet$} The frequency of the oscillating tubes
\vskip 5pt 
\item{$\bullet$} The turbulence within the oscillating tubes
\medskip



%INDEX% Measurement, flow: Coriolis (mass)

%(END_NOTES)


