
% Copyright 2010, Tony R. Kuphaldt, released under the Creative Commons Attribution License (v 1.0)
% This means you may do almost anything with this work of mine, so long as you give me proper credit

%(BEGIN_FRONTMATTER)

\centerline{\bf Course Syllabus} \bigskip 
 
\noindent
{\bf INSTRUCTOR CONTACT INFORMATION:}

Tony Kuphaldt

(360)-752-8477 [office phone]

(360)-752-7277 [fax]

{\tt tony.kuphaldt@btc.ctc.edu}

\vskip 10pt

\noindent
{\bf DEPT/COURSE \#:} INST 200

\vskip 10pt

\noindent
{\bf CREDITS:} 2 \hskip 30pt {\bf Lecture Hours:} 10 \hskip 30pt {\bf Lab Hours:} 20 \hskip 30pt {\bf Work-based Hours:} 0

\vskip 10pt

\noindent
{\bf COURSE TITLE:} Introduction to Instrumentation

\vskip 10pt

\noindent
{\bf COURSE DESCRIPTION:} This course introduces you to the trade, terminology, and basic principles of instrumentation.  It is a preparatory course for any one of three sections within the second year of Instrumentation: {\it measurement}, {\it control}, and {\it systems}, enabling you to begin your second year of Instrumentation at the start of Fall, Winter, or Spring quarter.  {\bf Prerequisite course:} MATH\&141 (Precalculus 1) with a minimum grade of ``C'', or instructor permission

\vskip 10pt

\noindent
{\bf COURSE OUTCOMES:} Build and document a functioning control system, using industry-standard test equipment to measure and interpret signals within that system. 

\vskip 10pt

\noindent
{\bf COURSE OUTCOME ASSESSMENT:} Each student must demonstrate mastery (100\% competence) in the construction, documentation, and testing of a functional control loop in the lab.  Failure to meet all mastery standards by the next scheduled exam day will result in a failing grade for the course.

\vskip 10pt

%%%%%%%%%%%%%%%%%%%%%%%%%%%%%%%%%%%%

\noindent
{\bf STUDENT PERFORMANCE OBJECTIVES:}

\item{$\bullet$} In a team environment and with full access to references, notes, and instructor assistance, perform the following objectives with 100\% accuracy (mastery).  Multiple re-tries are allowed on mastery (100\% accuracy) objectives:
\item\item{$\rightarrow$} Communicate effectively with teammates to plan work, arrange for absences, and share responsibilities in completing all labwork
\item\item{$\rightarrow$} Construct and commission a working pressure control ``loop'' consisting of pressure transmitter, PID controller, and final control element (e.g. control valve)
\item\item{$\rightarrow$} Generate an accurate loop diagram compliant with ISA standards documenting your team's system, personally verified by the instructor
\item\item{$\rightarrow$} Demonstrate proper assembly of NPT pipe fittings, personally verified by the instructor
\item\item{$\rightarrow$} Demonstrate proper assembly of instrument tube fittings (e.g. Swagelok brand), personally verified by the instructor
\item\item{$\rightarrow$} Demonstrate proper use of safety equipment and application of safe procedures while using power tools, and working on live systems

%%%%%%%%%%%%%%%%%%%%%%%%%%%%%%%%%%%%

\vfil \eject

\noindent
{\bf COURSE OUTLINE:} A course calendar in electronic format (Excel spreadsheet) resides on the Y: network drive, and also in printed paper format in classroom DMC130, for convenient student access.  This calendar is updated to reflect schedule changes resulting from employer recruiting visits, interviews, and other impromptu events.  Course worksheets provide comprehensive lists of all course assignments and activities, with the first page outlining the schedule and sequencing of topics and assignment due dates.  These worksheets are available in PDF format at {\tt http://www.ibiblio.org/kuphaldt/socratic/sinst}

\vskip 5pt

\item{$\bullet$} INST200 Section 1: 5 days theory and labwork

\vskip 10pt

\noindent
{\bf METHODS OF INSTRUCTION:} Course structure and methods are intentionally designed to develop critical-thinking and life-long learning abilities, continually placing the student in an active rather than a passive role.  

\item{$\bullet$} {\bf Independent study:} daily worksheet questions specify {\it reading assignments}, {\it problems} to solve, and {\it experiments} to perform in preparation (before) classroom theory sessions.  Open-note quizzes and work inspections ensure accountability for this essential preparatory work.  The purpose of this is to convey information and basic concepts, so valuable class time isn't wasted transmitting bare facts, and also to foster the independent research ability necessary for self-directed learning in your career.
\item{$\bullet$} {\bf Classroom sessions:} a combination of {\it Socratic discussion}, short {\it lectures}, {\it small-group} problem-solving, and hands-on {\it demonstrations/experiments} review and illuminate concepts covered in the preparatory questions.  The purpose of this is to develop problem-solving skills, strengthen conceptual understanding, and practice both quantitative and qualitative analysis techniques.
\item{$\bullet$} {\bf Lab activities:} an emphasis on constructing and documenting {\it working projects} (real instrumentation and control systems) to illuminate theoretical knowledge with practical contexts.  Special projects off-campus or in different areas of campus (e.g. BTC's Fish Hatchery) are encouraged.  Hands-on {\it troubleshooting exercises} build diagnostic skills.
\item{$\bullet$} {\bf Feedback questions:} sets of {\it practice problems} at the end of each course section challenge your knowledge and problem-solving ability in current as as well as first year (Electronics) subjects.  These are optional assignments, counting neither for nor against your grade.  Their purpose is to provide you and your instructor with direct feedback on what you have learned.

\vskip 10pt

\noindent
{\bf STUDENT ASSIGNMENTS/REQUIREMENTS:} All assignments for this course are thoroughly documented in the following course worksheets located at:

\noindent
{\tt http://www.ibiblio.org/kuphaldt/socratic/sinst/index.html} 

\vskip 5pt

\item{$\bullet$} {\tt INST200\_sec1.pdf} 

%%%%%%%%%%%%%%%%%%%%%%%%%%%%%%%%%%%%

\vfil \eject

\noindent
{\bf EVALUATION AND GRADING STANDARDS:} 

\item{$\bullet$} Mastery lab objectives = 50\% of course grade
\item{$\bullet$} Lab questions = 25\% 
\item{$\bullet$} Daily quizzes = 25\%
\item{$\bullet$} Tardiness penalty = -1\% per incident (1 ``free'' tardy per course)
\item{$\bullet$} Extra credit = +5\% per project (assigned by instructor based on individual learning needs)

\vskip 10pt

\noindent
All grades are criterion-referenced (i.e. no grading on a ``curve'')

\medskip
\item{} 100\% $\geq$ {\bf A} $\geq$ 95\% \hskip 33pt 95\% $>$ {\bf A-} $\geq$ 90\%
\item{} 90\% $>$ {\bf B+} $\geq$ 86\% \hskip 30pt 86\% $>$ {\bf B} $\geq$ 83\% \hskip 30pt 83\% $>$ {\bf B-} $\geq$ 80\%
\item{} 80\% $>$ {\bf C+} $\geq$ 76\% \hskip 30pt 76\% $>$ {\bf C} $\geq$ 73\% \hskip 30pt 73\% $>$ {\bf C-} $\geq$ 70\% (minimum passing course grade)
\item{} 70\% $>$ {\bf D+} $\geq$ 66\% \hskip 30pt 66\% $>$ {\bf D} $\geq$ 63\% \hskip 30pt 63\% $>$ {\bf D-} $\geq$ 60\% \hskip 30pt 60\% $>$ {\bf F}
\medskip

\vskip 10pt

A graded ``preparatory'' quiz at the start of each classroom session gauges your independent learning prior to the session.  A graded ``summary'' quiz at the conclusion of each classroom session gauges your comprehension of important concepts covered during that session.  If absent during part or all of a classroom session, you may receive credit by passing comparable quizzes afterward or by having your preparatory work (reading outlines, work done answering questions) thoroughly reviewed prior to the absence.  

If any ``mastery'' objectives are not completed by their specified deadlines, your overall grade for the course will be capped at 70\% (C- grade), and you will have one more school day to complete the unfinished objectives.  Failure to complete those mastery objectives by the end of that extra day (except in the case of documented, unavoidable emergencies) will result in a failing grade (F) for the course.

``Lab questions'' are assessed by individual questioning, at any date after the respective lab objective (mastery) has been completed by your team.  These questions serve to guide your completion of each lab exercise and confirm participation of each individual student.  Grading is as follows: full credit for thorough, correct answers; half credit for partially correct answers; and zero credit for major conceptual errors.  All lab questions must be answered by the due date of the lab exercise.

%%%%%%%%%%%%%%%%%%%%%%%%%%%%%%%%%%%%

\vfil \eject

\noindent
{\bf REQUIRED STUDENT SUPPLIES AND MATERIALS:} 

\item{$\bullet$} Course worksheets available for download in PDF format
\item{$\bullet$} {\it Lessons in Industrial Instrumentation} textbook, available for download in PDF format
\itemitem{$\rightarrow$} Access worksheets and book at: {\tt http://www.ibiblio.org/kuphaldt/socratic/sinst}
\item{$\bullet$} Spiral-bound notebook for reading annotation, homework documentation, and note-taking.
\item{$\bullet$} Instrumentation reference CD-ROM (free, from instructor).  This disk contains many tutorials and datasheets in PDF format to supplement your textbook(s).
\item{$\bullet$} Tool kit (see detailed list)
\item{$\bullet$} Simple scientific calculator (non-programmable, non-graphing, no unit conversions, no numeration system conversions), TI-30Xa or TI-30XIIS recommended
\item{$\bullet$} Portable personal computer with Ethernet port and wireless.  Windows OS strongly preferred, tablets discouraged.

\vskip 10pt

\noindent
{\bf ADDITIONAL INSTRUCTIONAL RESOURCES:} 

\item{$\bullet$} The BTC Library hosts a substantial collection of textbooks and references on the subject of Instrumentation, as well as links in its online catalog to free Instrumentation e-book resources available on the Internet.
\item{$\bullet$} ``BTCInstrumentation'' channel on YouTube ({\tt http://www.youtube.com/BTCInstrumentation}), hosts a variety of short video tutorials and demonstrations on instrumentation.
\item{$\bullet$} Instrumentation student club meets regularly to set up industry tours, raise funds for scholarships, and serve as a general resource for Instrumentation students.
\item{$\bullet$} ISA website ({\tt http://www.isa.org}) provides all of its standards in electronic format, many of which are freely available to ISA members.
\item{$\bullet$} {\it Purdy's Instrument Handbook}, by Ralph Dewey.  ISBN-10: 1-880215-26-8.  A pocket-sized field reference on basic measurement and control.

\vskip 10pt

\noindent
{\bf CAMPUS EMERGENCIES:} If an emergency arises, your instructor may inform you of actions to follow.  You are responsible for knowing emergency evacuation routes from your classroom.  If police or university officials order you to evacuate, do so calmly and assist those needing help.  You may receive emergency information alerts via the building enunciation system, text message, email, or BTC's webpage ({\tt http://www.btc.ctc.edu}), Facebook or Twitter.  Refer to the emergency flipchart in the lab room (located on the main control panel) for more information on specific types of emergencies.

\vskip 10pt

\noindent
{\bf ACCOMMODATIONS:} If you think you could benefit from classroom accommodations for a disability (physical, mental, emotional, or learning), please contact our Accessibility Resources office.  Call (360)-752-8345, email {\tt ar@btc.ctc.edu}, or stop by the AR Office in the Admissions and Student Resource Center (ASRC), Room 106, College Services Building

\vskip 10pt




\vfil 

\underbar{file {\tt INST200syllabus}}
\eject
%(END_FRONTMATTER)


