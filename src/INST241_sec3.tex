 
\vskip 10pt

%%%%%%%%%%%%%%%
\hrule \vskip 5pt
\noindent
\underbar{Lab}

\vskip 5pt

\noindent Flow measurement loop: {\it Questions 91 and 92}, {\bf completed objectives due by the end of day 5, section 4}

\vskip 10pt

%%%%%%%%%%%%%%%
\hrule \vskip 5pt
\noindent
\underbar{Exam}

\vskip 5pt

\noindent {\it Day 5 of next section}

\vskip 2pt \noindent {\it Specific objectives for the ``mastery'' exam:}

\item{$\bullet$} Electricity Review: Calculate voltages, currents, and phase shifts in an AC reactive circuit
\item{$\bullet$} Calculate flow rate / pressure drop for a nonlinear flow element
\item{$\bullet$} Determine suitability of different flow-measuring technologies for a given process fluid type
\item{$\bullet$} Identify specific instrument calibration errors (zero, span, linearity, hysteresis) from data in an ``As-Found'' table
\item{$\bullet$} Solve for a specified variable in an algebraic formula (may contain exponents or logarithms)
\item{$\bullet$} Determine the possibility of suggested faults in a simple circuit given measured values (voltage, current), a schematic diagram, and reported symptoms
\item{$\bullet$} INST230 Review: Calculate voltages and currents within balanced three-phase AC electrical circuits
\item{$\bullet$} INST250 Review: Calculate split-ranged valve positions given signal value and valve calibration ranges
\item{$\bullet$} INST262 Review: Determine proper AI block parameters to range a Fieldbus transmitter for a given application

\vskip 10pt



%%%%%%%%%%%%%%%
\hrule \vskip 5pt

\centerline{\bf Recommended daily schedule} 

\vskip 5pt

%%%%%%%%%%%%%%%
\filbreak
\hrule \vskip 5pt
\noindent \underbar{Day 1}

\vskip 5pt

%INSTRUCTOR \noindent {\bf Problem-solving intro activity:} review INST241\_x1 exam.

\vskip 2pt \noindent {\bf Theory session topic:} Flow measurement technologies

\vskip 2pt \noindent Questions 1 through 20; \underbar{answer questions 1-10} in preparation for discussion (remainder for practice)

\vskip 10pt



%%%%%%%%%%%%%%%
\filbreak
\hrule \vskip 5pt
\noindent \underbar{Day 2}

\vskip 5pt

%INSTRUCTOR \noindent {\bf Problem-solving intro activity:} explore the ``curved arrow'' notation for voltage in DC circuits shown in the ``Electrical Sources and Loads'' section of the ``DC Electricity'' chapter of the LIII textbook, commenting on how this notation is analagous to force and displacement, helping to explain positive and negative quantities of mechanical work.

\vskip 2pt \noindent {\bf Theory session topic:} Fluid dynamics

\vskip 2pt \noindent Questions 21 through 40; \underbar{answer questions 21-30} in preparation for discussion (remainder for practice)

\vskip 10pt



%%%%%%%%%%%%%%%
\filbreak
\hrule \vskip 5pt
\noindent \underbar{Day 3}

\vskip 5pt

%INSTRUCTOR \noindent {\bf Problem-solving intro activity:} apply the critical reading strategy suggested in Question 0 where readers are encouraged to work through mathematical exercises.  A specific example of this would be to verify the square-root scales of indicator gauges shown in the textbook using a calculator, correlating equivalent values shown on the linear versus square-root scales.

\vskip 2pt \noindent {\bf Theory session topic:} Pressure-based flowmeters

\vskip 2pt \noindent Questions 41 through 60; \underbar{answer questions 41-50} in preparation for discussion (remainder for practice)

\vskip 10pt




%%%%%%%%%%%%%%%
\filbreak
\hrule \vskip 5pt
\noindent \underbar{Day 4}

\vskip 5pt

%INSTRUCTOR \noindent {\bf Problem-solving intro activity:} Research equipment manuals to sketch a complete circuit connecting a loop controller to either a 4-20 mA transmitter or a 4-20 mA final control element ({\tt i03773})

%INSTRUCTOR \noindent {\bf Problem-solving intro activity:} identifying possible ways in which an orifice-based flowmeter can give false readings.  Refer to the P\&ID in {\tt i03490} for examples, such as nitrogen flowmeter FT-29 or steam flowmeter FT-28.

\vskip 2pt \noindent {\bf Theory session topic:} High-accuracy pressure-based flow measurement

\vskip 2pt \noindent Questions 61 through 80; \underbar{answer questions 61-70} in preparation for discussion (remainder for practice)

\vskip 5pt

\noindent Feedback questions {\it (81 through 90)} are optional and may be submitted for review at the end of the day

\vskip 10pt



\vfil \eject

