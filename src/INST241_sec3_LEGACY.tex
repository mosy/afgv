
\centerline{\bf INST 241 (Temperature and Flow Measurement), section 3} \bigskip 
 
\vskip 10pt

\noindent
{\bf Recommended schedule}

\vskip 5pt

%%%%%%%%%%%%%%%
\hrule \vskip 5pt
\noindent
\underbar{Day 1}

\hskip 10pt Major topics: {\it Potential and kinetic energy}
 
\hskip 10pt Question 0: {\it Research sources}

\hskip 10pt Discussion questions: {\it 1 through 10}
 
\hskip 10pt Lab Exercise: {\it Flow measurement loop (question 61 + question 62)}

%INSTRUCTOR \hskip 10pt {\bf Prep Quiz: explain the difference between potential and kinetic energy}

%INSTRUCTOR \hskip 10pt {\bf Summary Quiz: work calculation}

\vskip 10pt
%%%%%%%%%%%%%%%
\hrule \vskip 5pt
\noindent
\underbar{Day 2}

\hskip 10pt Major topics: {\it Viscosity and Reynolds number}
 
\hskip 10pt Discussion questions: {\it 11 through 20}
 
\hskip 10pt Lab Exercise: {\it Flow measurement loop (question 61 + question 62)}
 
%INSTRUCTOR \hskip 10pt {\bf Prep Quiz: define ``laminar'' flow}

%INSTRUCTOR \hskip 10pt {\bf Summary Quiz: explain why it matters if a flow is ``laminar'' or ``turbulent'' when inferring flow rate from pressure drop}

%INSTRUCTOR \hskip 10pt {\bf Demo: show video of an extremely low Reynolds number flow, where three colored dyes do not mix}

\vskip 10pt
%%%%%%%%%%%%%%%
\hrule \vskip 5pt
\noindent
\underbar{Day 3}

\hskip 10pt Major topics: {\it Bernoulli's equation and Torricelli's theorem}
 
\hskip 10pt Discussion questions: {\it 21 through 30}
 
\hskip 10pt Lab Exercise: {\it Flow measurement loop (question 61 + question 62)}
 
%INSTRUCTOR \hskip 10pt {\bf Prep Quiz: identify points of min/max pressure in a venturi tube}

%INSTRUCTOR \hskip 10pt {\bf Summary Quiz: calculate fluid velocity at nozzle using Torricelli's theorem}

\vskip 10pt
%%%%%%%%%%%%%%%
\hrule \vskip 5pt
\noindent
\underbar{Day 4}

\hskip 10pt Major topics: {\it Venturis, orifice plates, and other head-generating flow elements}
 
\hskip 10pt Required reading: {\it Lessons In Industrial Instrumentation} ``Pressure-based flowmeters'' section of the ``Continuous flow measurement'' chapter
 
\hskip 10pt Discussion questions: {\it 31 through 40}
 
\hskip 10pt Lab Exercise: {\it Flow measurement loop (question 61 + question 62)}
 
%INSTRUCTOR \hskip 10pt {\bf Prep Quiz: explain how an orifice plate is used to measure fluid flow}

%INSTRUCTOR \hskip 10pt {\bf Summary Quiz: identify at least three PSE's other than orifice plates for measuring flow}

%INSTRUCTOR \hskip 10pt {\bf Demo: show an example of an orifice plate to the students}

\vskip 10pt
%%%%%%%%%%%%%%%
\hrule \vskip 5pt
\noindent
\underbar{Day 5}

\hskip 10pt Major topics: {\it Nonlinear flow signal conditioning (square root extraction)}
 
\hskip 10pt Discussion questions: {\it 41 through 50}
 
\hskip 10pt Lab Exercise: {\it Flow measurement loop (question 61 + question 62)}
 
\hskip 10pt Lab Exercise: {\it Mastery exam performance practice (question 63)}
 
%INSTRUCTOR \hskip 10pt {\bf Prep Quiz: explain why we must ``square-root'' the signal coming from a differential pressure transmitter when used as a flow-measuring instrument across a restriction such as an orifice plate}

%INSTRUCTOR \hskip 10pt {\bf Summary Quiz: square-root calculation}

%INSTRUCTOR \hskip 10pt {\bf Demo: show a pneumatic square root extractor to the students}

\vskip 10pt

\hskip 10pt Feedback questions: {\it 51 through 60}
 
\hskip 10pt {\bf Feedback questions due at the end of the day}
 
\vskip 10pt
%%%%%%%%%%%%%%%
\hrule \vskip 5pt
\noindent
\underbar{Practice problems}
 
\hskip 10pt {\it 64 through end}
 
\vskip 10pt
%%%%%%%%%%%%%%%
%\hrule \vskip 5pt
%\noindent
%\underbar{Impending deadlines}

%\hskip 10pt {\bf ???}
 
%\hskip 10pt Question ???: Sample grading criteria
 
%\vskip 10pt
%%%%%%%%%%%%%%%



%INSTRUCTOR \hrule \vskip 5pt

%INSTRUCTOR {\bf Interesting topics and applications to discuss:} 

%INSTRUCTOR (Open-channel flowmeter design).  {\it Discuss how to design a crude open-channel flowmeter using multiple ``dip'' electrodes in a stilling well next to a weir or flume.  Electrical output changes with the number of electrodes touching the water.  Topics include characterization of flow (voltage output linear with flow rate, not with liquid height), energization of electrodes with AC instead of DC to avoid electroplating in mineral-rich water, circuit design to convert probe continuity to voltage output}.

%INSTRUCTOR \vskip 10pt





\vfil \eject

