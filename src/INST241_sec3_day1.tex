% This file contains all the necessary TeX statements for specifying
% overall document format.  This is the file you would edit to set
% any global typesetting parameters.

\input epsf.tex

% This line effectively turns off "Underfull \vbox" error messages.
\vbadness=10000

\tolerance = 1000
\pretolerance = 10000

%%%%%%%%%%%%%%%%%%%%%%%%%%%%%%%%%%%%%%%%%%%%%%%%%%%%%%%%%%%%%%%%%%%%%%%%%%%%%
\vskip 5pt \hrule \vskip 5pt \noindent {\bf Question 01} -- LIII (skim continuous flow measurement) \vskip 10pt

Flow elements creating an acceleration of the fluid:

\medskip
\item{$\bullet$} Venturi tube
\item{$\bullet$} Orifice plate
\item{$\bullet$} Flow nozzle
\item{$\bullet$} Segmental wedge
\item{$\bullet$} V-cone
\item{$\bullet$} Pipe elbow ({\it radial} acceleration)
\medskip

\vskip 10pt

Flow elements creating a deceleration of the fluid:

\medskip
\item{$\bullet$} Pitot tube
\item{$\bullet$} Averaging pitot tube
\item{$\bullet$} Annubar
\item{$\bullet$} Target
\medskip

\vskip 10pt

In accelerating element types, the ``H'' port of the DP transmitter connects to the region of widest flow profile (slowest velocity), while the ``L'' port connects to the region of narrowest flow profile (fastest velocity).

\vskip 10pt

In decelerating element types, the ``H'' port of the DP transmitter connects to the region of flow stagnation (slowest velocity), while the ``L'' port connects to the region of full flow profile (normal velocity).



%%%%%%%%%%%%%%%%%%%%%%%%%%%%%%%%%%%%%%%%%%%%%%%%%%%%%%%%%%%%%%%%%%%%%%%%%%%%%
\vskip 5pt \hrule \vskip 5pt \noindent {\bf Question 02} -- LIII (skim continuous flow measurement) \vskip 10pt

Rotameters achieve a flow-varying area by using a tapered tube in which a plummet rises and falls.  As the plummet rises, the tube ``opens up'' around it to yield a greater area for fluid to pass by.

\vskip 10pt

Weirs and flumes are open-channel restrictions, where the area fluid flows through increases as fluid height increases in the throat of the device.

\vskip 10pt

Weirs are like dams over which liquid must spill.  They are analogous to orifice plates in pipes.

\vskip 10pt

Fluke are like narrow rapids in a river, through which liquid must rush.  They are analogous to venturi tubes.

%%%%%%%%%%%%%%%%%%%%%%%%%%%%%%%%%%%%%%%%%%%%%%%%%%%%%%%%%%%%%%%%%%%%%%%%%%%%%
\vskip 5pt \hrule \vskip 5pt \noindent {\bf Question 03} -- LIII (skim continuous flow measurement) \vskip 10pt

Turbine flowmeters use a free-spinning ``windmill'' placed in the fluid flow to measure that fluid's velocity.

\vskip 10pt

The von K\'arm\'an effect is the alternating series of vortices formed when a fluid moves past a blunt obstacle.  The frequency of this vortex shedding is the basis of a vortex flowmeter: the higher the fluid's velocity, the faster the oscillation ensues.

\vskip 10pt

Magnetic flowmters exploit the electrical conductivity of the process liquid to generate a small voltage based on fluid velocity.  The fluid moves through a magnetic field, generating an EMF perpendicular to both the field and the direction of flow.

\vskip 10pt

Ultrasonic flowmeters measure fluid velocity by passing high-frequency sound waves through the fluid path.  ``Doppler'' flowmeters bounce sound waves off of bubbles or solids in the liquid stream, inferring velocity from the pitch of the echo.  ``Transit time'' flowmeters send a sound wave upstream and downstream, comparing the times of flight for each pulse.  The difference in time indicates fluid velocity.


%%%%%%%%%%%%%%%%%%%%%%%%%%%%%%%%%%%%%%%%%%%%%%%%%%%%%%%%%%%%%%%%%%%%%%%%%%%%%
\vskip 5pt \hrule \vskip 5pt \noindent {\bf Question 04} -- LIII (skim continuous flow measurement) \vskip 10pt

Coriolis flowmeters pass liquid through a flexible, vibrating tube.  The resonant frequency of this vibrating tube varies with fluid density, while the inertia of the flowing fluid causes the tube to undulate (induces a phase shift).  By sensing the resonant frequency of the tubes' vibrations, the instrument infers fluid density.  By sensing the phase shift between one portion of the tube versus another, the instrument infers mass flow rate.

%%%%%%%%%%%%%%%%%%%%%%%%%%%%%%%%%%%%%%%%%%%%%%%%%%%%%%%%%%%%%%%%%%%%%%%%%%%%%
\vskip 5pt \hrule \vskip 5pt \noindent {\bf Question 05} -- LIII (skim continuous flow measurement) \vskip 10pt

Thermal flowmeters exploit the phenomenon of ``wind chill'' to infer mass flow rate.  A heated object is inserted into the flow path, and the rate of heat loss from this object (via convection) is measured to infer mass flow rate.

%%%%%%%%%%%%%%%%%%%%%%%%%%%%%%%%%%%%%%%%%%%%%%%%%%%%%%%%%%%%%%%%%%%%%%%%%%%%%
\vskip 5pt \hrule \vskip 5pt \noindent {\bf Question 06} -- LIII (skim continuous flow measurement) \vskip 10pt

Positive displacement flowmeters are constructed kind of like engines, where a moving mechanism transports a definite volume of fluid per rotation or cycle.  The frequency of this cycling (or the speed of rotation) is proportional to flow rate.  The total number of cycles or turns is proportional to the total volume of fluid passed through the flowmeter.

%%%%%%%%%%%%%%%%%%%%%%%%%%%%%%%%%%%%%%%%%%%%%%%%%%%%%%%%%%%%%%%%%%%%%%%%%%%%%
\vskip 5pt \hrule \vskip 5pt \noindent {\bf Question 07} -- LIII (skim continuous flow measurement) \vskip 10pt

A weighfeeder is a continuously moving belt equipped with weight sensors to measure the weight of the solid material conveyed by the belt.

%%%%%%%%%%%%%%%%%%%%%%%%%%%%%%%%%%%%%%%%%%%%%%%%%%%%%%%%%%%%%%%%%%%%%%%%%%%%%
\vskip 5pt \hrule \vskip 5pt \noindent {\bf Question 08} -- LIII (skim continuous flow measurement) \vskip 10pt

If we measure the weight of a storage vessel over time, we may calculate the rate of weight change and infer net flow in or out of the vessel this way.

\vskip 10pt

If we measured the weight of a propane tank fueling a BBQ, the rate of the tank's weight loss would be equal to the mass flow rate of the propane fuel.

%%%%%%%%%%%%%%%%%%%%%%%%%%%%%%%%%%%%%%%%%%%%%%%%%%%%%%%%%%%%%%%%%%%%%%%%%%%%%
\filbreak \vskip 5pt \hrule \vskip 5pt \noindent {\bf Question 09} -- LIII (energy, work, and power) \vskip 10pt

{\it Work} is defined as the exertion of energy by a force acting parallel to a motion.  It is typically expressed in a compound unit, of the force times the displacement (e.g. newton-meters or foot-pounds)

$$W = Fx$$

\noindent
Where,

$W$ = Work, in newton-meters (metric) or foot-pounds (British)

$F$ = Force doing the work, in newtons (metric) or pounds (British)

$x$ = Displacement over which the work was done, in meters (metric) or feet (British)

\vskip 10pt

If we pull a sled through snow, our pulling force multiplied by the distance pulled gives us the total energy expended in pulling the sled.  In the case of a sled, this energy gets transformed into heat at the runners, due to friction between the sled and the snow.

\vskip 10pt

{\it Potential energy} is energy existing in a stored state, having the potential to do useful work (e.g. a tensed spring, a charged capacitor, a suspended weight).  When a weight is lifted against gravity, the work done in lifting that weight gets stored in the weight's suspension:
 
$$E_p = W = Fx$$

$$E_p = mgh \hbox{ \hskip 30pt (if lifted vertically)}$$

\noindent
Where,

$E_p$ = Potential energy in newton-meters (metric) or foot-pounds (British)

$m$ = Mass of object in kilograms (metric) or slugs (British)

$g$ = Acceleration of gravity in meters per second squared (metric) or feet per second squared (British)

$h$ = Height of lift in meters (metric) or feet (British)

\vskip 10pt

If that suspended weight is then released, its stored energy will go into motion as the weight falls down.

\vskip 10pt

In industrial work settings, we must take measures to ensure dangerous potential energy does not release in a way that will cause harm.  This is why we have lock-out and tag-out (``Energy Control'') procedures, including locking strategies allowing a large number of people to ensure the safety locks cannot be removed until every last person's work is finished and it is once again safe to unleash the potential energy.

\vskip 10pt

{\it Kinetic energy} is energy in motion (e.g. a moving bullet, a light wave).  When applied to moving objects, kinetic energy is proportional to the {\it square} of the object's velocity:

$$E_k = {1 \over 2} mv^2$$

\noindent
Where,

$E_k$ = Kinetic energy in joules or newton-meters (metric), or foot-pounds (British)

$m$ = Mass of object in kilograms (metric) or slugs (British)

$v$ = Velocity of mass in meters per second (metric) or feet per second (British)

\vskip 10pt

All forms of energy have equivalent units of measurement, even though those units may be compounded and convoluted.  A ``joule'' of energy is equivalent to a ``newton-meter'' of energy, which is equivalent to a ``kilogram-meter-squared-per-second-squared'' of energy.


%%%%%%%%%%%%%%%%%%%%%%%%%%%%%%%%%%%%%%%%%%%%%%%%%%%%%%%%%%%%%%%%%%%%%%%%%%%%%
\filbreak \vskip 5pt \hrule \vskip 5pt \noindent {\bf Question 10} -- potential \& kinetic energy calculations \vskip 10pt

Elevator energy = 5700 ft-lb = 7728 joules

\vskip 10pt

Bullet energy = 2648.2 ft-lb = 3590.5 joules


%%%%%%%%%%%%%%%%%%%%%%%%%%%%%%%%%%%%%%%%%%%%%%%%%%%%%%%%%%%%%%%%%%%%%%%%%%%%%
\filbreak \vskip 5pt \hrule \vskip 5pt \noindent {\bf Summary questions and review of general principles} \vskip 10pt

\noindent
Identify any general principles you've learned today (i.e. principles spanning multiple applications).
\item{$\bullet$} Fluid accelerates as is passes through a narrow passageway, and decelerates as it either enters a wide area or stagnates in front of an obstacle.  This change in velocity generates a change in pressure which can be measured to infer flow rate.
\item{$\bullet$} Fluid passing by a blunt obstacle tend to ``oscillate'' around it, and the frequency of that oscillation is proportional to the velocity of the fluid.
\item{$\bullet$} Potential energy is energy in a static (stored) form; kinetic energy is energy in motion
\medskip

\medskip
\item{$(Q01)$} Summarize main points of the reading (accelerating/decelerating flow elements)
\item{$(Q02)$} Summarize main points of the reading (variable-area flow elements)
\item{$(Q03)$} Summarize main points of the reading (turbine, vortex, magnetic, ultrasonic flow elements)
\item{$(Q04)$} Summarize main points of the reading (coriolis)
\item{$(Q05)$} Summarize main points of the reading (thermal)
\item{$(Q06)$} Summarize main points of the reading (positive displacement)
\item{$(Q07)$} Summarize main points of the reading (weighfeeders)
\item{$(Q08)$} Summarize main points of the reading (change-of-weight)
\item{$(Q09)$} Summarize main points of the reading (work, energy, and power)
\item{$(Q10)$} Show mathematical work in calculating energy
\medskip

\medskip
\item{$(Q09)$} Give an example of potential energy that exists within a 100 foot radius of where you are sitting right now, and explain how that amount of energy could be precisely calculated
\item{$(Q09)$} Give an example of kinetic energy that exists within a 100 foot radius of where you are sitting right now, and explain how that amount of energy could be precisely calculated
\item{$(Q09)$} Give an example of energy (either potential or kinetic) applied to a type of flowmeter
\item{$(Q09)$} How does an {\it escape ramp} work to dissipate the kinetic energy of a run-away cargo truck going down a hill?
\item{$(Q09)$} How does a liquid-filled {\it crash drum} work to dissipate the kinetic energy of a speeding car?
\item{$(Q09)$} As a kid, shooting BB guns, I learned I could recapture and re-use by BB ammunition by setting up a hanging cloth behind my targets.  The cloth would gently stop the flying BB without deforming it, or being puncured by the BB.  Explain how this worked by appealing to kinetic and potential energy.
\item{$(Q09)$} Explain how conductive heat transfer works, knowing that the temperature of an object is an expression of its molecules' kinetic energy.
\item{$(Q09)$} Where does the kinetic energy of a bicycle go when the rider applies the brakes?
\item{$(Q09)$} Why do hybrid electric cars get such great mileage when driving in stop-and-go traffic, often {\it better} fuel mileage than when the car is driven at constant speed on a highway?
\item{$(Q09)$} Suppose someone planning an off-grid home needs to store energy collected by solar panels during the day, to be used at night.  What energy-storage options can you think of?  Be as creative as you can in your answer(s)!
\item{$(Q09)$} {\bf INST230 review question:} When a VFD uses {\it DC injection} to slow down an electric motor, where does the kinetic energy goe?
\item{$(Q09)$} {\bf INST230 review question:} When a VFD uses {\it dynamic braking} to slow down an electric motor, where does the kinetic energy go?
\item{$(Q09)$} {\bf INST230 review question:} When a VFD uses {\it regenerative braking} to slow down an electric motor, where does the kinetic energy go?
\item{$(Q09)$} {\bf INST230 review question:} When a VFD uses {\it plugging} to slow down an electric motor, where does the kinetic energy go?
\medskip


%%%%%%%%%%%%%%%%%%%%%%%%%%%%%%%%%%%%%%%%%%%%%%%%%%%%%%%%%%%%%%%%%%%%%%%%%%%%%
\filbreak \vskip 5pt \hrule \vskip 5pt \noindent {\bf Review INST241\_x1 exam} \vskip 10pt

\medskip
\item{$\bullet$} Mastery AM: 45\% first-attempt pass ; 65\% pass ; 35\% fail
\item{$\bullet$} Mastery PM: 35\% first-attempt pass ; ??\% pass ; ??\% fail
\medskip
\item{$\bullet$} MQ2: with a 4-wire RTD connection, both pairs of connections must terminate common at the pot! 
\item{$\bullet$} MQ2: trying to connect all three wires of the pot (going from memory to how pot was wired for thermocouple simulation in lab)
\item{$\bullet$} MQ3: temperature must be in degrees C, not degrees F, to use the formula with alpha = 0.00385
\item{$\bullet$} MQ3: people not checking their work (using RTD table versus formula)
\item{$\bullet$} MQ3: looking up resistance in table, then plugging THAT into formula as temperature (!)
\medskip
\item{$\bullet$} PQ1: many people missed the sheath color (should be brown)
\item{$\bullet$} PQ2: specific heat is not temperature and it is not heat! 
\item{$\bullet$} PQ3: l-atm/mol-K is a compound unit, not variables into which you must enter values!
\item{$\bullet$} PQ4: adding versus subtracting thermocouple millivoltage values (MANY MISTAKES HERE)
\item{$\bullet$} PQ4: no compensation here because the multimeter is not a thermocouple instrument
\item{$\bullet$} PQ5: proper substitution: note that the 32 degree offset disappears!
\item{$\bullet$} PQ5: proper substitution: 9/5 rather than 5/9!
\item{$\bullet$} PQ5: problem-solving technique -- test formula with known values (e.g. 32, 212 deg F)
\item{$\bullet$} PQ6: lots of decent explanations (basically good reasoning) but false conclusion
\item{$\bullet$} PQ6: some had no explanation given at all!
\item{$\bullet$} PQ8: $V_{out}$ should be same as $V_{RTD}$ (that is the purpose of the circuit!)
\item{$\bullet$} PQ8: $V_{CB}$ and $V_{CF}$ should both be 0.25 mV
\item{$\bullet$} PQ10: lots of decent explanations (basically good reasoning) but false conclusion
\item{$\bullet$} PQ10: problem-solving technique -- imagine an infinitely long heat exchanger!
\medskip


%$$\epsfxsize=2in \epsfbox{i00000x01.eps}$$


\bye



