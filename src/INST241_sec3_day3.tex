% This file contains all the necessary TeX statements for specifying
% overall document format.  This is the file you would edit to set
% any global typesetting parameters.

\input epsf.tex

% This line effectively turns off "Underfull \vbox" error messages.
\vbadness=10000

\tolerance = 1000
\pretolerance = 10000

%%%%%%%%%%%%%%%%%%%%%%%%%%%%%%%%%%%%%%%%%%%%%%%%%%%%%%%%%%%%%%%%%%%%%%%%%%%%%
\vskip 5pt \hrule \vskip 5pt \noindent {\bf Question 41} -- LIII (venturi tubes and basic principles) \vskip 10pt

When we manipulate Bernoulli's Equation to solve for flow rate based on differential pressure, we arrive at the following result:

$$Q = \sqrt{2} A_2 {1 \over \sqrt{1 - \left({A_2 \over A_1}\right)^2}} \sqrt{{P_1 - P_2} \over \rho}$$

Simplifying this to a proportionality:

$$Q = k \sqrt{{P_1 - P_2} \over \rho}$$


%%%%%%%%%%%%%%%%%%%%%%%%%%%%%%%%%%%%%%%%%%%%%%%%%%%%%%%%%%%%%%%%%%%%%%%%%%%%%
\filbreak \vskip 5pt \hrule \vskip 5pt \noindent {\bf Question 42} -- LIII (volumetric and mass flow calculations) \vskip 10pt

The constant of proportionality ($k$) in the basic volumetric flow calculation shown below not only accounts for the geometry of the flow element (e.g. venturi tube, orifice plate, pitot tube, etc.) but it may also account for all unit conversions:

$$Q = k \sqrt{{P_1 - P_2} \over \rho}$$

All we need to do is take a known amount of flow ($Q$) with its corresponding pressure drop ($P_1 - P_2$) and solve for $k$ to arrive at a formula we may use to solve for {\it any} combination of flow rates and pressure drops.

\vskip 10pt

{\it Mass} flow measurement applications are important where the mass of products must be carefully accounted, such as in custody transfer applications.  Substituting $W \over \rho$ for $Q$ in the volumetric flow formula yields a mass flow formula:

$$W = k \sqrt{\rho ({P_1 - P_2})}$$

The value of $k$ is determined from empirical values of $W$ and pressure drop for a particular flow element.






%%%%%%%%%%%%%%%%%%%%%%%%%%%%%%%%%%%%%%%%%%%%%%%%%%%%%%%%%%%%%%%%%%%%%%%%%%%%%
\filbreak \vskip 5pt \hrule \vskip 5pt \noindent {\bf Question 43} -- LIII (square-root characterization) \vskip 10pt

The pressure drop generated by an acceleration-style flow element is proportional to the {\it square} of the flow rate.  Flow rate doubles, and pressure quadruples; flow rate triples, and pressure increases by a factor of 9.

\vskip 10pt

In order to achieve a signal that is linear with flow (and not with pressure), we must somehow {\it square-root} the measured pressure signal.  This may be done internally to the transmitter or to the indicating instrument, or it may be done inside an independent ``square root'' relay:

% No blank lines allowed between lines of an \halign structure!
% I use comments (%) instead, so that TeX doesn't choke.

$$\vbox{\offinterlineskip
\halign{\strut
\vrule \quad\hfil # \ \hfil & 
\vrule \quad\hfil # \ \hfil \vrule \cr
\noalign{\hrule}
%
% First row
Input percentage & Output percentage \cr
%
\noalign{\hrule}
%
% Another row
0\% & $\sqrt{0} = 0\%$ \cr
%
\noalign{\hrule}
%
% Another row
25\% & $\sqrt{0.25} = 50\%$ \cr
%
\noalign{\hrule}
%
% Another row
50\% & $\sqrt{0.5} = 70.71\%$ \cr
%
\noalign{\hrule}
%
% Another row
75\% & $\sqrt{0.75} = 86.60\%$ \cr
%
\noalign{\hrule}
%
% Another row
100\% & $\sqrt{1} = 100\%$ \cr
%
\noalign{\hrule}
} % End of \halign 
}$$ % End of \vbox

\vskip 10pt

Indicators with nonlinear scales are also used to simply characterize the output signal of a DP transmitter.  A linear scale if often superimposed on the square-root scale for reference.  This does nothing to linearize the actual signal coming from the transmitter (for the benefit of any other instruments in the loop), but it at least linearizes the reading for any human operator looking at the indicator.

\vskip 10pt

The compression evident at the low end of a square-root scale is not just an artifact of the scale, but rather an important illustration of measurement uncertainty toward the low end of a DP flow transmitter's range.  At low percentages of flow, even tiny errors in pressure measurement translate into relatively large flow measurement errors.  This is why the practical turndown ratio for DP-based flowmeters is approximately 3:1 (i.e. the lowest measurement with trusted accuracy is only one-third of the flow's upper range value).


%%%%%%%%%%%%%%%%%%%%%%%%%%%%%%%%%%%%%%%%%%%%%%%%%%%%%%%%%%%%%%%%%%%%%%%%%%%%%
\filbreak \vskip 5pt \hrule \vskip 5pt \noindent {\bf Question 44} -- LIII (orifice plates) \vskip 10pt

An orifice plate is simply a metal plate with a hole machined in the middle for flow to pass through.  It is typically sandwiched between two pipe flanges.

\vskip 10pt

The point of maximum constriction in the flow stream is called the {\it vena contracta}.  The ratio of bore diameter to inside pipe diameter is called the {\it beta ratio} ($\beta = {d \over D}$).

\vskip 10pt

Square-edged concentric orifice plates have their bores located in the exact center of the plate.  The square-shaped bore edges allow for bidirectional flow measurement.  Text labeling stamped on the upstream face of an external tab identifies the upstream side of the plate.

\vskip 10pt

Thick orifice plates may have a beveled edge on the downstream face, to minimize contact surface area with the flow stream, and thus minimizing friction between the orifice and the flowing fluid.  These beveled orifice plates obviously are unidirectional.

\vskip 10pt

Eccentric orifice plates have their bore holes located off-center to allow passage of undesired flow components (e.g. bubbles in a liquid stream, where the hole is located high; liquid droplets in a gas stream, where the hole is located low).

\vskip 10pt

Quadrant-edge and conical-entrance orifice plates bevel the {\it upstream} edge of the bore in an attempt to improve measurement accuracy at low Reynolds number values.  The only sure way to tell which way flow should go through such an orifice plate is to pay attention to the text labeling on the tab (text facing upstream).  The increased pressure drop caused by viscous friction on this beveled edge tends to offset the decrease in pressure caused by reduced contraction at the vena contracta for high-viscosity fluids.

\vskip 10pt

Tap locations vary according to pipe size and convention.  Flange taps are very common in the United States.  The downstream tap should be clear of the highly turbulent region following the vena contracta.  Tap holes need to be flush and free of any burrs.

\vskip 10pt

Integral orifice plates mount directly to the DP transmitter, and are often used on small pipes for low flow rates.  Some integral orifice plate manifolds force the fluid to flow {\it through} the flanged body of the DP transmitter.  These are used in cases where the process line size is comparable to the pressure port size on the transmitter.

\vskip 10pt

Orifice plates are best sized by computer software, to account for all the variables related to plate style and accuracy.


%%%%%%%%%%%%%%%%%%%%%%%%%%%%%%%%%%%%%%%%%%%%%%%%%%%%%%%%%%%%%%%%%%%%%%%%%%%%%
\filbreak \vskip 5pt \hrule \vskip 5pt \noindent {\bf Question 45} -- LIII (other differential producers) \vskip 10pt

{\it Pitot tubes} generate a pressure based on fluid {\it stagnation} at the entrance of the upstream-facing tube.  Traditional Pitot tubes are sensitive to flow at only one point in the stream, so {\it averaging} Pitot tube assemblies are made with rows of holes to average out the effects of the flow profile across the pipe diameter.  The Dieterich-Standard ``Annubar'' is an example of an averaging Pitot tube element.

\vskip 10pt

{\it Paddle} or ``drag disk'' elements measure the raw force applied to an obstacle hanging in the flowstream, caused by fluid stagnation on the upstream face.

\vskip 10pt

{\it Flow tubes} are short venturi tubes.  {\it Flow nozzles} are venturi-like structures that clamp between pipe flanges kind of like an orifice plate.

\vskip 10pt

A {\it V-cone} element is kind of like an orifice plate in reverse: flow passes through an annular passage, around a ``donut hole'' in the middle of the pipe.

\vskip 10pt

{\it Segmental wedges} are nothing more than pipe segments with a wedge-shaped obstruction acting as a restriction.  They are most useful for measuring slurries.

\vskip 10pt

{\it Pipe elbows} may also be used as flow elements, albeit inaccurate ones.

\vskip 10pt

The amount of permanent pressure loss caused by a flow-sensing element may be a significant factor in the long-term cost of the device.  If such a loss constitutes an energy cost, then it may pay to invest in a more expensive low-loss element such as a venturi tube over a cheap orifice plate.


%%%%%%%%%%%%%%%%%%%%%%%%%%%%%%%%%%%%%%%%%%%%%%%%%%%%%%%%%%%%%%%%%%%%%%%%%%%%%
\filbreak \vskip 5pt \hrule \vskip 5pt \noindent {\bf Question 46} -- LIII (proper installation) \vskip 10pt

Poor installation will compromise the performance of a flowmeter.  We must ensure a well-developed flow profile entering flowmeters such as orifice plates, free of large-scale turbulence such as swirls and eddies.  Long lengths of straight pipe before and after the flow element help to achieve this goal.  If necessary, ``flow straighteners'' may be placed in the pipe to help condition the flow profile.  Flow elements with small beta ratios (more constriction) tend to fare better in poor installations than elements having large beta ratios (less constriction).

\vskip 10pt

DP transmitter location is also important.  DP sensors should be mounted above the pipe for gas applications (to avoid liquid capture in the impulse lines), and below the pipe for liquid applications (to avoid bubble capture in the impulse lines).


%%%%%%%%%%%%%%%%%%%%%%%%%%%%%%%%%%%%%%%%%%%%%%%%%%%%%%%%%%%%%%%%%%%%%%%%%%%%%
\filbreak \vskip 5pt \hrule \vskip 5pt \noindent {\bf Question 47} -- flow/DP calculations \vskip 10pt

$k$ = 120 if $P$ = 100 when $Q$ = 1200

\begin{itemize}
\item{$\bullet$} $Q$ = 500 GPM ; $\Delta$P = \underbar{\bf 17.36} inches water column
\item{$\bullet$} $Q$ = 245 GPM ; $\Delta$P = \underbar{\bf 4.168} inches water column
\item{$\bullet$} $Q$ = 630 GPM ; $\Delta$P = \underbar{\bf 27.56} inches water column
\item{$\bullet$} $Q$ = 950 GPM ; $\Delta$P = \underbar{\bf 62.67} inches water column
\medskip


%%%%%%%%%%%%%%%%%%%%%%%%%%%%%%%%%%%%%%%%%%%%%%%%%%%%%%%%%%%%%%%%%%%%%%%%%%%%%
\filbreak \vskip 5pt \hrule \vskip 5pt \noindent {\bf Question 48} -- flow/DP calculations \vskip 10pt

$$Q = 62.61 \sqrt{\Delta P}$$

\begin{itemize}
\item{$\bullet$} $Q$ range = 0 to 500 GPM ; $\Delta P$ range = \underbar{\bf 0 to 63.78 "W.C.}
\item{$\bullet$} $Q$ range = 0 to 600 GPM ; $\Delta P$ range = \underbar{\bf 0 to 91.84 "W.C.}
\item{$\bullet$} $Q$ range = 0 to 800 GPM ; $\Delta P$ range = \underbar{\bf 0 to 163.3 "W.C.}
\item{$\bullet$} $Q$ range = 0 to 1000 GPM ; $\Delta P$ range = \underbar{\bf 0 to 255.1 "W.C.}
\medskip


%%%%%%%%%%%%%%%%%%%%%%%%%%%%%%%%%%%%%%%%%%%%%%%%%%%%%%%%%%%%%%%%%%%%%%%%%%%%%
\filbreak \vskip 5pt \hrule \vskip 5pt \noindent {\bf Question 49} -- flow/DP calculations \vskip 10pt

$k$ = 1831.54 given 6000 GPM at 11 "Hg pressure and a density of 1.025 g/cm$^{3}$.

\vskip 10pt

$P$ = 5.438 "Hg at 4250 GPM and 1.01 g/cm$^{3}$ density

\vskip 10pt

$Q$ = 4,533.9 GPM flow rate at 3.1 PSID and 1.03 g/cm$^{3}$

\vskip 10pt

$Q$ = 3,413.7 GPM flow rate at 12 kPaD and 1.02 g/cm$^{3}$


%%%%%%%%%%%%%%%%%%%%%%%%%%%%%%%%%%%%%%%%%%%%%%%%%%%%%%%%%%%%%%%%%%%%%%%%%%%%%
\filbreak \vskip 5pt \hrule \vskip 5pt \noindent {\bf Question 50} -- square root extractor range calculations \vskip 10pt

% No blank lines allowed between lines of an \halign structure!
% I use comments (%) instead, so that TeX doesn't choke.

$$\vbox{\offinterlineskip
\halign{\strut
\vrule \quad\hfil # \ \hfil & 
\vrule \quad\hfil # \ \hfil & 
\vrule \quad\hfil # \ \hfil & 
\vrule \quad\hfil # \ \hfil \vrule \cr
\noalign{\hrule}
%
% First row
Input signal & Percent of input & Percent of output & Output signal \cr
%
% Another row
(PSI) & span (\%) & span (\%) & (PSI) \cr
%
\noalign{\hrule}
%
% Another row
5 & 16.67 & 40.82 & 7.899 \cr
%
\noalign{\hrule}
%
% Another row
13 & {\bf 83.33} & {\bf 91.29} & {\bf 13.95} \cr
%
\noalign{\hrule}
%
% Another row
9 & 50 & 70.71 & 11.49 \cr
%
\noalign{\hrule}
%
% Another row
6.6 & 30 & 54.77 & 9.573 \cr
%
\noalign{\hrule}
%
% Another row
10.68 & 64 & 80 & 12.6 \cr
%
\noalign{\hrule}
%
% Another row
{\bf 3.27} & {\bf 2.25} & 15 & {\bf 4.8} \cr
%
\noalign{\hrule}
%
% Another row
4.333 & 11.11 & 33.33 & 7 \cr
%
\noalign{\hrule}
%
% Another row
9.75 & 56.25 & 75 & 12 \cr
%
\noalign{\hrule}
} % End of \halign 
}$$ % End of \vbox

Values shown in bold-faced type are those given to students in the ``Answer'' section.


%%%%%%%%%%%%%%%%%%%%%%%%%%%%%%%%%%%%%%%%%%%%%%%%%%%%%%%%%%%%%%%%%%%%%%%%%%%%%
\filbreak \vskip 5pt \hrule \vskip 5pt \noindent {\bf Summary questions and review of general principles} \vskip 10pt

\noindent
Identify any general principles you've learned today (i.e. principles spanning multiple applications).
\item{$\bullet$} Use of proportionality constants to account for unit conversions
\item{$\bullet$} Necessity of square-root characterization to linearize DP-based flow measurements
\item{$\bullet$} Limited turndown of flow instruments (typically 3:1) based on nonlinear characteristic
\medskip

\begin{itemize}
\item{$(Q42)$} Summarize main points of the reading (volumetric versus mass flow measurement)
\item{$(Q42)$} Give a practical example of a {\it custody transfer} flow measurement application.
\item{$(Q42)$} When an automobile driver purchases gasoline, are they paying based on volumetric measurement or by mass measurement?
\item{$(Q42)$} When a bicyclist purchases fuel (food) at a bulk food store, are they paying based on volumetric measurement or by mass measurement?
\item{$(Q43)$} Summarize main points of the reading (square-root characterization)
\item{$(Q43)$} Explain what ``turndown'' means in the context of flowmeters
\item{$(Q44)$} How can we tell which way flow should go through an orifice plate?
\item{$(Q44)$} Explain the purpose of having a beveled edge on the downstream face of an orifice plate.
\item{$(Q45)$} Describe alternatives to venturi tubes and orifice plates
\item{$(Q47-49)$} Show mathematical work for flow/DP calculations!
\item{$(Q47)$} Why is it okay to use this general formula for {\it any} primary flow element based on differential pressure?  There are many different types of flow elements (venturis, orifices, nozzles, Pitot tubes, segmented wedge tubes, etc.), each with its own unique design.  What is common to all these elements that the same basic equation form may be used to describe the operation of them all?
\item{$(Q50)$} Show mathematical work for square-root extraction!
\medskip

\begin{itemize}
\item{$(Q43)$} Suppose a DP transmitter sensing pressure dropped across a venturi tube has a slight ``zero'' sensor trim error: it reads +1 inch of water column higher than it should at all applied pressures.  Where in the range of flow measurement will this error have the most effect on the accuracy of flow measurement, and why?  Does it matter where the square-root extraction is done in the loop?
\item{$(Q44)$} Identify which way to orient an eccentric or segmental orifice plate, for measuring natural gas where some water is present in the flow stream.
\item{$(Q44)$} Identify which way to orient an eccentric or segmental orifice plate, for measuring wastewater where some solids are present in the flow stream.
\item{$(Q44)$} Identify which way to orient an eccentric or segmental orifice plate, for measuring potable water where some air bubbles are present in the flow stream.
\item{$(Q44)$} Identify which way to orient an eccentric or segmental orifice plate, for measuring petroleum oil from an oil well where some sand is present in the flow stream.
\item{$(Q44)$} Identify which way to orient an eccentric or segmental orifice plate, for measuring natural gas from a gas well where some sand is present in the flow stream.
\item{$(Q46)$} Critique the installation of the orifice plate immediately downstream of the elbow
\item{$(Q46)$} Suppose a DP flow transmitter were located above the pipe in a liquid service application.  Describe in detail how this could compromise flow measurement accuracy.
\item{$(Q46)$} Suppose a DP flow transmitter were located below the pipe in a gas service application.  Describe in detail how this could compromise flow measurement accuracy.
\medskip


%%%%%%%%%%%%%%%%%%%%%%%%%%%%%%%%%%%%%%%%%%%%%%%%%%%%%%%%%%%%%%%%%%%%%%%%%%%%%
\filbreak \vskip 5pt \hrule \vskip 5pt \noindent {\bf Problem Solving question $(Q28)$} \vskip 10pt

Imagine the pressure safety valve PSV-355 stuck open!

\vskip 10pt

PG-316 is registering much less vacuum than is normally expected, despite FI-37 registering a normal amount of steam flow through the eductor.


%$$\epsfxsize=2in \includegraphics[width=15.5cm]{i00000x01.eps}$$


\bye



