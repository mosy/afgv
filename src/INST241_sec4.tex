
\centerline{\bf INST 241 (Temperature and Flow Measurement), section 4} \bigskip 
 
\vskip 10pt

%%%%%%%%%%%%%%%
\hrule \vskip 5pt
\noindent
\underbar{Lab}

\vskip 5pt

\noindent Flow measurement loop: {\it Questions 91 and 92}, {\bf completed objectives due by the end of day 5}

\vskip 10pt

%%%%%%%%%%%%%%%
\hrule \vskip 5pt
\noindent
\underbar{Exam}

\vskip 5pt

\noindent {\it Day 5} 

\vskip 2pt \noindent {\it Specific objectives for the ``mastery'' exam:}

\item{$\bullet$} Electricity Review: Calculate voltages, currents, and phase shifts in an AC reactive circuit
\item{$\bullet$} Calculate flow rate / pressure drop for a nonlinear flow element
\item{$\bullet$} Determine suitability of different flow-measuring technologies for a given process fluid type
\item{$\bullet$} Identify specific instrument calibration errors (zero, span, linearity, hysteresis) from data in an ``As-Found'' table
\item{$\bullet$} Solve for a specified variable in an algebraic formula (may contain exponents or logarithms)
\item{$\bullet$} Determine the possibility of suggested faults in a simple circuit given measured values (voltage, current), a schematic diagram, and reported symptoms
\item{$\bullet$} INST230 Review: Calculate voltages and currents within balanced three-phase AC electrical circuits
\item{$\bullet$} INST250 Review: Calculate split-ranged valve positions given signal value and valve calibration ranges
\item{$\bullet$} INST262 Review: Determine proper AI block parameters to range a Fieldbus transmitter for a given application

\vskip 10pt



%%%%%%%%%%%%%%%
\hrule \vskip 5pt

\centerline{\bf Recommended daily schedule} 

\vskip 5pt

%%%%%%%%%%%%%%%
\filbreak
\hrule \vskip 5pt
\noindent \underbar{Day 1} 

\vskip 5pt

%INSTRUCTOR \noindent {\bf Problem-solving intro activity:} identifying possible/impossible faults in a flow measurement loop using either a turbine, vortex, or positive displacement flowmeter, referencing a P\&ID (e.g. from the {\it Realistic Instrumentation Diagrams} practice worksheet)

\vskip 2pt \noindent {\bf Theory session topic:} Turbine, vortex, and positive-displacement flowmeters

\vskip 2pt \noindent Questions 1 through 20; \underbar{answer questions 1-9} in preparation for discussion (remainder for practice)

\vskip 10pt



%%%%%%%%%%%%%%%
\filbreak
\hrule \vskip 5pt
\noindent \underbar{Day 2}

\vskip 5pt

%INSTRUCTOR \noindent {\bf Problem-solving intro activity:} identifying valid/invalid diagnostic tests in a flow measurement system, referencing a loop diagram (e.g. from the {\it Realistic Instrumentation Diagrams} practice worksheet)

%INSTRUCTOR \noindent {\bf Problem-solving intro activity:} practice active reading and summarizing techniques referenced in Question 0 by summarizing the optical flowmeters subsection in the textbook.  Have students read this subsection of the textbook, then explain it to each other in their own terms, then share those explanations with the class at large.  After that, have students reflect on the similarities and differences between the different optical flowmeter technologies and other flowmeters they've studied thusfar.  Show your students some of the questions you may ask them (in the instructor's version of Question 0) in order to prompt them to apply these techniques to their study.

\vskip 2pt \noindent {\bf Theory session topic:} Magnetic and ultrasonic flowmeters

\vskip 2pt \noindent Questions 21 through 40; \underbar{answer questions 21-28} in preparation for discussion (remainder for practice)

\vskip 10pt



%%%%%%%%%%%%%%%
\filbreak
\hrule \vskip 5pt
\noindent \underbar{Day 3}

\vskip 5pt

%INSTRUCTOR \noindent {\bf Problem-solving intro activity:} brainstorm flow measusurement technologies suitable for refinery flare gas (file {\tt i00977})

%INSTRUCTOR \noindent {\bf Problem-solving intro activity:} brainstorm flow measusurement technologies suitable for natural gas (file {\tt i00978})

\vskip 2pt \noindent {\bf Theory session topic:} True mass flowmeters, weirs and flumes

\vskip 2pt \noindent Questions 41 through 60; \underbar{answer questions 41-49} in preparation for discussion (remainder for practice)

\vskip 10pt




%%%%%%%%%%%%%%%
\filbreak
\hrule \vskip 5pt
\noindent \underbar{Day 4}

\vskip 5pt

%INSTRUCTOR \noindent {\bf Problem-solving intro activity:} enthalpy calculation problem using a steam table (e.g. required steam mass flow rate for a heat exchanger)

\vskip 2pt \noindent {\bf Theory session topic:} Guest presentation on Flow / Review for exam

\vskip 2pt \noindent Questions 61 through 80; \underbar{answer questions 61-69} in preparation for discussion (remainder for practice)

\vskip 5pt

\noindent Feedback questions {\it (81 through 90)} are optional and may be submitted for review at the end of the day

\vskip 10pt



%%%%%%%%%%%%%%%
\filbreak
\hrule \vskip 5pt
\noindent \underbar{Day 5}

\vskip 5pt

\noindent {\bf Exam}

\vskip 10pt



\vfil \eject

