
\centerline{\bf INST 241 (Temperature and Flow Measurement), section 4} \bigskip 
 
\vskip 10pt

\noindent
{\bf Recommended schedule}

\vskip 5pt

%%%%%%%%%%%%%%%
\hrule \vskip 5pt
\noindent
\underbar{Day 1}

\hskip 10pt Major topics: {\it Vortex and turbine flow instruments}
 
\hskip 10pt Required reading: {\it Lessons In Industrial Instrumentation} ``Velocity-based flowmeters'' section of the ``Continuous flow measurement'' chapter
 
\hskip 10pt Question 0: {\it Research sources}

\hskip 10pt Discussion questions: {\it 1 through 10}
 
\hskip 10pt Lab Exercise: {\it Flow measurement loop (question 52 + question 53)}

%INSTRUCTOR \hskip 10pt {\bf Prep Quiz: explain how a vortex flow meter works}

%INSTRUCTOR \hskip 10pt {\bf Summary Quiz: calculate the K factor for a turbine meter given output frequency and known flow rate}

%INSTRUCTOR \hskip 10pt {\bf Demo: show an actual vortex flowmeter to students}

%INSTRUCTOR \hskip 10pt {\bf Demo: show an actual turbine flowmeter to students}

\vskip 10pt
%%%%%%%%%%%%%%%
\hrule \vskip 5pt
\noindent
\underbar{Day 2}

\hskip 10pt Major topics: {\it Magnetic and sonic flow measurement}
 
\hskip 10pt Discussion questions: {\it 11 through 20}
 
\hskip 10pt Lab Exercise: {\it Flow measurement loop (question 52 + question 53)}
 
%INSTRUCTOR \hskip 10pt {\bf Prep Quiz: explain how a magnetic flow meter works}

%INSTRUCTOR \hskip 10pt {\bf Summary Quiz: identify at least two problems that can affect the accuracy of a magflow meter}

%INSTRUCTOR \hskip 10pt {\bf Demo: show an actual magnetic flowmeter to students}

\vskip 10pt
%%%%%%%%%%%%%%%
\hrule \vskip 5pt
\noindent
\underbar{Day 3}

\hskip 10pt Major topics: {\it Mass flow measurement techniques}
 
\hskip 10pt Required reading: {\it Lessons In Industrial Instrumentation} ``Inertial-based (true mass) flowmeters'' and ``Thermal-based (mass) flowmeters'' sections of the ``Continuous flow measurement'' chapter
 
\hskip 10pt Discussion questions: {\it 21 through 30}
 
\hskip 10pt Lab Exercise: {\it Flow measurement loop (question 52 + question 53)}

%INSTRUCTOR \hskip 10pt {\bf Prep Quiz: explain what ``mass flow'' measurement is and how it differs from ``volumetric flow'' measurement}

%INSTRUCTOR \hskip 10pt {\bf Summary Quiz: explain what ``SCFM'' means}

%INSTRUCTOR \hskip 10pt {\bf Demo: show a working Coriolis flowmeter (cut-away) to students}

\vskip 10pt
%%%%%%%%%%%%%%%
\hrule \vskip 5pt
\noindent
\underbar{Day 4}

\hskip 10pt Major topics: {\it Positive displacement and other flow measurement technologies}
 
\hskip 10pt Required reading: {\it Lessons In Industrial Instrumentation} ``Positive displacement flowmeters'' section of the ``Continuous flow measurement'' chapter
 
\hskip 10pt Discussion questions: {\it 31 through 40}
 
\hskip 10pt Lab Exercise: {\it Lab questions (question 52)}

\hskip 10pt Lab Exercise: {\bf Flow measurement loop due at the end of the day (question 52 + question 53)}
 
%INSTRUCTOR \hskip 10pt {\bf Prep Quiz: explain how a ``positive displacement'' flow meter works}

%INSTRUCTOR \hskip 10pt {\bf Summary Quiz: identify two flow transmitter technologies that are bidirectional, and two that are not}

%INSTRUCTOR \hskip 10pt {\bf Demo: show one or more rotameters to students}

\vskip 10pt
%%%%%%%%%%%%%%%
\hrule \vskip 5pt
\noindent
\underbar{Day 5}

\hskip 10pt Feedback questions: {\it 41 through 50}
 
\hskip 10pt Labwork participation assessment: {\it 51}
 
\hskip 10pt {\bf Feedback questions due at the beginning of the day}
 
\vskip 10pt

\hskip 10pt Lab Exercise: {\it Mastery exam performance practice (question 54)}
 
\hskip 10pt Exam 2: {\it includes circuit-building performance exercise in Mastery portion}
 
\vskip 10pt
%%%%%%%%%%%%%%%
\hrule \vskip 5pt
\noindent
\underbar{Practice problems}
 
\hskip 10pt {\it 55 through end}
 
\vskip 10pt

%%%%%%%%%%%%%%%
%\hrule \vskip 5pt
%\noindent
%\underbar{Impending deadlines}

%\hskip 10pt {\bf ???}
 
%\hskip 10pt Question ???: Sample grading criteria
 
%\vskip 10pt
%%%%%%%%%%%%%%%



%INSTRUCTOR \hrule \vskip 5pt

%INSTRUCTOR {\bf Interesting topics and applications to discuss:} 

%INSTRUCTOR (Vortex flowmeter design).  {\it Discuss how to design a vortex-sensing digital circuit for making your own flowmeter.  Topics include how to detect vortices with electronic sensors, how to condition sensor signals to generate square-wave output, how to digitally count square-wave output, how to accurately time counts, how to latch digital output between counting cycles.  Note that the concept of a digital counting circuit to count events over a fixed time will also work to make a turbine flowmeter, a positive-displacement flowmeter, and even an ultrasonic flowmeter.}

%INSTRUCTOR \vskip 10pt





\vfil \eject

