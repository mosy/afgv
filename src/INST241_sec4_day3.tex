% This file contains all the necessary TeX statements for specifying
% overall document format.  This is the file you would edit to set
% any global typesetting parameters.

\input epsf.tex

% This line effectively turns off "Underfull \vbox" error messages.
\vbadness=10000

\tolerance = 1000
\pretolerance = 10000

%%%%%%%%%%%%%%%%%%%%%%%%%%%%%%%%%%%%%%%%%%%%%%%%%%%%%%%%%%%%%%%%%%%%%%%%%%%%%
\vskip 5pt \hrule \vskip 5pt \noindent {\bf Question 41} -- LIII (true mass intro) \vskip 10pt

Many flowmeters sense fluid {\it velocity} (e.g. turbine, ultrasonic, vortex) and as such are volumetric instruments.  However, in applications where we need to measure the molecular quantity of fluid moving through, a flowmeter sensing mass would be more appropriate.

\vskip 10pt

An application of mass flow measurement is chemical feed flow, where we need to know how many molecules of a substance are flowing into a reactor.  It does not matter how densely these molecules are packaged, and so volumetric measurements would be useless.

Another application of mass flow measurement is in steam boilers, to relate water flow in to steam flow out.  What really matters is an accounting of mass, since every pound of water boiled will make a pound of steam.  Volumetric flow measurement on a boiler would be meaningless.

Custody transfer is another application for mass flow measurement, where we are fundamentally concerned with the purchase and sale of molecules.  Since mass is an intrinsic property of matter, we know in advance how much mass each molecule (of a certain type) will have, and so gross measurements of mass tell us how many molecules have passed through.

\vskip 10pt

Mass flow ($W$) and volumetric flow ($Q$) are related to each other by the following formula:

$$W = \rho Q$$

\vskip 10pt

Volumetric flowmeter technologies may be compensated by pressure and temperature measurements to yield a computed mass flow measurement.  This is the principle of AGA3 (orifice plate), AGA7 (turbine), and AGA9 (ultrasonic) gas flow measurement standards for custody transfer.  In each of these cases, the signal output by the flow element must be ``compensated'' to yield a calculated value for mass flow rate, since the flow element itself is not inherently a mass-sensing device.

\vskip 10pt

True mass flowmeters, by contrast, naturally respond to the flow rate in mass units.  Coriolis flowmeters sense the inertia of a flowing fluid, and thermal flowmeters sense the heat transferred by a flowing fluid.

\vskip 10pt

500 CFM water = 31,200 lb/min = 520 lb/s


%%%%%%%%%%%%%%%%%%%%%%%%%%%%%%%%%%%%%%%%%%%%%%%%%%%%%%%%%%%%%%%%%%%%%%%%%%%%%
\filbreak \vskip 5pt \hrule \vskip 5pt \noindent {\bf Question 42} -- LIII (Coriolis flowmeters) \vskip 10pt

Coriolis flowmeters work by shaking a tube carrying a fluid.  The mass density of that fluid will affect the tubes' resonant frequency, and so frequency becomes an inversely proportional representation of fluid density.  As the fluid moves through the vibrating tubes, its inertia causes the tubes to undulate.  This undulation or twisting motion is directly proportional to mass flow rate.

\vskip 10pt

The Coriolis force is a ``pseudoforce'' in physics: a force that appears to exist when viewed from a non-inertial reference frame.  It arises when a mass moves perpendicular to an axis of rotation.  Rather than rotate, the mass tries to move in a straight line.  When viewed from the rotating reference frame, however, it appears as though the mass tries to follow a curved trajectory.

\vskip 10pt

An example of the Coriolis force is apparent if we were to modify a rotary lawn sprinkler to have the water shoot straight out rather than through angled nozzles.  In such a mechanism, the flowing water's inertia will act to oppose any rotation of the sprinkler tubes.  The more mass flow we have through those tubes, the more it will oppose any external attempt at rotation.

Another example is seen if we try to wave a garden hose back and forth while water flows through it: the intertia of the flowing water makes the hose's end lag, because the water's mass ``tries'' to move in a straight line, and opposes any attempts to change its direction.

\vskip 10pt

A practical Coriolis flowmeter design moves the fluid through a U-shaped tube, shaking the looped end of that tube back and forth.  Inertia of the flowing fluid causes that U-shaped tube to {\it twist}.  The more mass flow rate, the greater the intertial forces, and therefore the greater the amount of twist.  Usually, the flowmeter passes fluid through a pair of balanced tubes, those tubes vibrating 180$^{o}$ out of phase with each other, both to minimize vibration and also to minimize the effects of external vibration on the instrument's operation.  A force coil mounted between the tubes at the curved end shakes them back and forth relative to each other, while a pair of sensing coils measures the amount of twist.  

The frequency of the signals coming from the sensor coils indicate fluid density (higher frequency = less dense), while the phase shift between the two coils' signals represents mass flow rate (more phase shift = more mass flow rate).

\vskip 10pt

The vibrating tubes of a Coriolis flowmeter are matched spring elements, and as such their mechanical properties must be precisely known in order to achieve good measurement accuracy.  For this reason, tubes must be matched to the electronics package of a Coriolis flowmeter from the factory.  If the tubes are changed for whatever reason, the electronics must be re-programmed to match the new tubes.

Temperature affects the spring constant of metal, and therefore Coriolis flowmeters must be equipped with tube temperature sensors to allow the electronics to compensate for changes in tube stiffness resulting from changes in fluid temperature.  This sensing of temperature is also available as a variable (in addition to density and mass flow rate), making the Coriolis flowmeter a multi-variable instrument.

\vskip 10pt

Coriolis flowmeters may infer volumetric flow rate, by dividing the true mass flow ($W$) by the sensed liquid density ($\rho$).

\vskip 10pt

Coriolis flowmeters are linear and highly accurate, and are also completely immune to disturbances in the flow (i.e. no need for straight-runs of pipe upstream or downstream).  The American Gas Association has standardized the use of Coriolis flowmeters for custody transfer of natural gas in their Report \#11 (AGA11), just as they have for orifice plates (AGA3), turbine meters (AGA7), and ultrasonic meters (AGA9).

\vskip 10pt

Disadvantages of Corilios flowmeters include high cost, limited operating temperature range, poor sensitivity to light (non-dense) fluids, and difficulty of in-place cleaning (for sanitary applications).


%%%%%%%%%%%%%%%%%%%%%%%%%%%%%%%%%%%%%%%%%%%%%%%%%%%%%%%%%%%%%%%%%%%%%%%%%%%%%
\filbreak \vskip 5pt \hrule \vskip 5pt \noindent {\bf Question 43} -- Micro Motion ELITE manual \vskip 10pt

``Turndown'' refers to the ratio of lowest flow to highest flow measurable while remaining within a stated accuracy of measurement.

\vskip 10pt

Coriolis flowmeters have {\it far} better turndown performance than orifice plate flowmeters!  Turndown ratios of 500:1 are possible with a flow measurement accuracy of $\pm$1.25\%.  With $\pm$0.25\% accuracy, the turndown for this model is 100:1.

\vskip 10pt

Best model for 25 GPM liquid flow: {\bf CMF050} (up to 30 GPM), from table at the top of page 4.

\vskip 10pt

Best model for 400 SCFM natural gas flow: {\bf CMF050} (up to 970 SCFM), from table on page 5.  This assumes a pressure drop across the meter of about 50 PSID.


%%%%%%%%%%%%%%%%%%%%%%%%%%%%%%%%%%%%%%%%%%%%%%%%%%%%%%%%%%%%%%%%%%%%%%%%%%%%%
\filbreak \vskip 5pt \hrule \vskip 5pt \noindent {\bf Question 44} -- LIII (thermal flowmeters) \vskip 10pt

Thermal flowmeters exploit the phenomenon of {\it wind chill} (convective cooling of a heated object) to measure true mass flow rate.  The greater the mass flow rate of a cooler fluid past a heated object, the faster that heated object will lose heat due to convection.  A ``hot wire anemometer'' is a simple type of thermal flowmeter, measuring air flow by sensing the temperature of a hot wire, and modulating current through that wire to maintain its temperature steady.  The more air flow past the hot wire, the more current will be necessary to maintain its temperature -- therefore, current becomes proportional to mass air flow rate.

\vskip 10pt

Industrial thermal flowmeters use two temperature sensors: one bonded to the heating element, and the other used to sense the ``ambient'' temperature of the incoming fluid.

\vskip 10pt

{\it Specific heat} is a fluid parameter that affects the accuracy of thermal flowmeters.  Different fluid types have different heat capacities (i.e. different capacities to absorb heat for a given temperature change), and therefore will skew the response of a thermal flowmeter.  This means the chemical composition of the fluid must be stable and known in order to use a thermal mass flowmeter.


%%%%%%%%%%%%%%%%%%%%%%%%%%%%%%%%%%%%%%%%%%%%%%%%%%%%%%%%%%%%%%%%%%%%%%%%%%%%%
\filbreak \vskip 5pt \hrule \vskip 5pt \noindent {\bf Question 45} -- coriolis flowmeter response to fluid changes \vskip 10pt

{\bf Increased volumetric flow rate with constant density:} the undulating motion of the tubes will {\it increase} in amplitude due to the greater inertial forces, but the resonant frequency of the tubes will {\it stay the same} because the tubes' mass has not changed.

\vskip 10pt

{\bf Increased density with constant volumetric flow rate:} the undulating motion of the tubes will {\it increase} in amplitude due to the greater inertial forces, and the resonant frequency of the tubes will {\it decrease} due to increased tube mass.

\vskip 10pt

{\bf Changes in fluid density at zero flow:} there will be no undulating motion, because there will be no Coriolis force with zero flow.  The tubes' resonant frequency, however, will vary inversely with fluid density.  One practical caveat is that there will need to be {\it some} flow in order to push a new fluid of different density into the flowmeter's vibrating tubes, in order to sense that new density.


%%%%%%%%%%%%%%%%%%%%%%%%%%%%%%%%%%%%%%%%%%%%%%%%%%%%%%%%%%%%%%%%%%%%%%%%%%%%%
\filbreak \vskip 5pt \hrule \vskip 5pt \noindent {\bf Question 46} -- thermal flowmeter response to fluid changes \vskip 10pt

As flow increases, temperature decreases, because the increased convective cooling steal more heat away from the heated sensor.

\vskip 10pt

As incoming temperature increases, sensor temperature increases as well.  This is interpreted to be {\it less} flow (i.e. less convective heat transfer).  Since there is no ``reference'' temperature sensor, the flowmeter has no way of distinguishing between a change in fluid temperature and a change in fluid flow rate!

\vskip 10pt

In order to compensate for the fluid's temperature entering the flowmeter and thus cancel any effects resulting from temperature change, we must have an unheated sensor that detects the fluid's ``ambient'' temperature.  This way, the flowmeter will be able to compensate for changes in fluid temperature, discriminating between the effects of this temperature change on the sensor versus the effects of different mass flow rates on the sensor.


%%%%%%%%%%%%%%%%%%%%%%%%%%%%%%%%%%%%%%%%%%%%%%%%%%%%%%%%%%%%%%%%%%%%%%%%%%%%%
\filbreak \vskip 5pt \hrule \vskip 5pt \noindent {\bf Question 47} -- coriolis and thermal flowmeter responses (same pipe) \vskip 10pt

The two flowmeters will no longer agree with each other, because the thermal meter's calibration has been thrown off by the different specific heat value of hydrogen (by a ratio of 2.75:1 to be exact).

\vskip 10pt

The greater $c$ value for hydrogen gas means the thermal flowmeter registers {\it more} flow than there actually is.  This makes the thermal meter give the greater reading (i.e. the Coriolis flowmeter registers {\bf less} flow than the thermal flowmeter).

\vskip 10pt

The Coriolis flowmeter measures true mass flow regardless of fluid composition, because it operates on the inertial principle of the Coriolis effect.  Only the thermal mass flowmeter will be ``fooled'' by this change in process fluid.


%%%%%%%%%%%%%%%%%%%%%%%%%%%%%%%%%%%%%%%%%%%%%%%%%%%%%%%%%%%%%%%%%%%%%%%%%%%%%
\filbreak \vskip 5pt \hrule \vskip 5pt \noindent {\bf Question 48} -- mass flow rate calculation \vskip 10pt

$$W = \rho Q$$

$$\left({1100 \hbox{ gal} \over \hbox{min}}\right) \left(231 \hbox{ in}^3 \over 1 \hbox{ gal}\right) \left(1 \hbox{ ft}^3 \over 1728 \hbox{ in}^3 \right) \left(59.3 \hbox{ lbm} \over \hbox{ft}^3 \right) = 8719.98 \hbox{ lbm/min}$$

\vskip 10pt

Unit conversions from lbm/min to kg/s:

$$\left(8719.98 \hbox{ lbm} \over \hbox{min}\right) \left(1 \hbox{ min} \over 60 \hbox{ s}\right) \left(0.4535924 \hbox{ kg} \over 1 \hbox{ lbm}\right) = 65.922 \hbox{ kg/s}$$


%%%%%%%%%%%%%%%%%%%%%%%%%%%%%%%%%%%%%%%%%%%%%%%%%%%%%%%%%%%%%%%%%%%%%%%%%%%%%
\filbreak \vskip 5pt \hrule \vskip 5pt \noindent {\bf Question 49} -- LIII (weirs and flumes) \vskip 10pt


%%%%%%%%%%%%%%%%%%%%%%%%%%%%%%%%%%%%%%%%%%%%%%%%%%%%%%%%%%%%%%%%%%%%%%%%%%%%%
\filbreak \vskip 5pt \hrule \vskip 5pt \noindent {\bf Summary questions and review of general principles} \vskip 10pt

\noindent
Identify any general principles you've learned today (i.e. principles spanning multiple applications).
\item{$\bullet$} Coriolis force is a pseudoforce resulting from the perpendicular motion of a mass with respect to the axis of rotation.  As viewed from the rotating reference frame, the mass experiences a lateral force.
\item{$\bullet$} Wind chill is the effect of convective cooling on a heated object, exploited in thermal mass flowmeters.
\medskip

\medskip
\item{$(Q41)$} Summarize main points of the reading (true mass flowmeters)
\item{$(Q42)$} Summarize main points of the reading (Coriolis mass flowmeters)
\item{$(Q43)$} Show how to interpret graph, tables
\item{$(Q44)$} Summarize main points of the reading (thermal mass flowmeters)
\item{$(Q46)$} Explain why the single-sensor thermal flowmeter responds the way it does
\item{$(Q46)$} Explain why all practical thermal flowmeters have two temperature sensors
\item{$(Q47)$} Explain what happens to both flowmeter indications when gas type changes from helium to hydrogen
\item{$(Q48)$} Show calculations of mass flow rate
\item{$(Q48)$} Which unit of measurement do you think is best for {\it custody transfer} applications: GPM or lb/min?  Explain your reasoning.
\item{$(Q49)$} Summarize main points of the reading (weirs and flumes)
\medskip

\medskip
\item{$(Q42)$} Suppose the amount of phase shift measured between the two sensing coils of a Coriolis flowmeter suddenly increases, while the frequency remains unchanged.  What does this tell us about the fluid flowing through the flowmeter?
\item{$(Q42)$} Suppose the amount of phase shift measured between the two sensing coils of a Coriolis flowmeter suddenly decreases, while the frequency remains unchanged.  What does this tell us about the fluid flowing through the flowmeter?
\item{$(Q42)$} Suppose the frequency of the sensing coil signals in a Coriolis flowmeter suddenly decreases.  What does this tell us about the fluid flowing through the flowmeter?
\item{$(Q42)$} Suppose the frequency of the sensing coil signals in a Coriolis flowmeter suddenly increases.  What does this tell us about the fluid flowing through the flowmeter?
\item{$(Q44)$} Suppose a thermal flowmeter is measuring the mass flow rate of nitrogen gas, and then suddenly the nitrogen is replaced by hydrogen gas (having a much greater specific heat value).  All other factors being the same as before, will the flowmeter's indication increase, decrease, or remain unchanged?
\item{$(Q44)$} Suppose a thermal flowmeter is measuring the mass flow rate of hydrogen gas, and then suddenly the hydrogen is replaced by helium gas (having a lesser specific heat value).  All other factors being the same as before, will the flowmeter's indication increase, decrease, or remain unchanged?
\item{$(Q44)$} Suppose a thermal flowmeter is measuring the mass flow rate of oxygen gas, and then suddenly the incoming gas temperature rises.  All other factors being the same as before, will the flowmeter's indication increase, decrease, or remain unchanged?
\item{$(Q44)$} Suppose a thermal flowmeter is measuring the mass flow rate of air, and then suddenly the incoming gas temperature falls.  All other factors being the same as before, will the flowmeter's indication increase, decrease, or remain unchanged?
\item{$(Q49)$} Explain why the rangeability of a weir or of a flume is so good, compared to orifice plates
\medskip


%%%%%%%%%%%%%%%%%%%%%%%%%%%%%%%%%%%%%%%%%%%%%%%%%%%%%%%%%%%%%%%%%%%%%%%%%%%%%
\filbreak \vskip 5pt \hrule \vskip 5pt \noindent {\bf Problem Solving question $(Q55)$} \vskip 10pt

Identify and evaluate different flowmeter technologies for a flare-gas flow measurement application.


\bye



