
% Copyright 2010, Tony R. Kuphaldt, released under the Creative Commons Attribution License (v 1.0)
% This means you may do almost anything with this work of mine, so long as you give me proper credit

%(BEGIN_FRONTMATTER)

\centerline{\bf Course Syllabus} \bigskip 
 
\noindent
{\bf INSTRUCTOR CONTACT INFORMATION:}

Tony Kuphaldt

(360)-752-8477 [office phone]

(360)-752-7277 [fax]

{\tt tony.kuphaldt@btc.ctc.edu}

\vskip 10pt

\noindent
{\bf DEPT/COURSE \#:} INST 241

\vskip 10pt

\noindent
{\bf CREDITS:} 6 \hskip 30pt {\bf Lecture Hours:} 26 \hskip 30pt {\bf Lab Hours:} 82 \hskip 30pt {\bf Work-based Hours:} 0

\vskip 10pt

\noindent
{\bf COURSE TITLE:} Temperature and Flow Measurement

\vskip 10pt

\noindent
{\bf COURSE DESCRIPTION:} In this course you will learn how to precisely measure both temperature and fluid flow in a variety of applications, as well as accurately calibrate and efficiently troubleshoot temperature and flow measurement systems.   {\bf Pre/Corequisite course:} INST 240 (Pressure and Level Measurement) {\bf Prerequisite course:} MATH\&141 (Precalculus 1) with a minimum grade of ``C''

\vskip 10pt

\noindent
{\bf COURSE OUTCOMES:} Commission, analyze, and efficiently diagnose instrumented systems measuring temperature and fluid flow rate.

\vskip 10pt

\noindent
{\bf COURSE OUTCOME ASSESSMENT:} Temperature and flow system commissioning, analysis, and diagnosis outcomes are ensured by measuring student performance against mastery standards, as documented in the Student Performance Objectives.  Failure to meet all mastery standards by the next scheduled exam day will result in a failing grade for the course.

%%%%%%%%%%%%%%%%%%%%%%%%%%%%%%%%%%%%

\vfil \eject

\noindent
{\bf STUDENT PERFORMANCE OBJECTIVES:}

\item{$\bullet$} Without references or notes, within a limited time (3 hours total for each exam session), independently perform the following tasks.  Multiple re-tries are allowed on mastery (100\% accuracy) objectives, each with a different set of problems:
\item\item{$\rightarrow$} Calculate voltages and currents in an ideal AC transformer circuit, with 100\% accuracy (mastery) 
\item\item{$\rightarrow$} Calculate voltages, currents, and phase shifts in an AC reactive circuit, with 100\% accuracy (mastery)
\item\item{$\rightarrow$} Sketch proper wire connections showing how to simulate an RTD or thermocouple input to a temperature transmitter using simple electronic components, with 100\% accuracy (mastery) 
\item\item{$\rightarrow$} Calculate temperatures or voltages in thermocouple and RTD circuits given access to thermocouple and RTD tables, with 100\% accuracy (mastery)
\item\item{$\rightarrow$} Calculate flow rate and pressure drop for a nonlinear flow element given maximum flow specifications, with 100\% accuracy (mastery) 
\item\item{$\rightarrow$} Determine suitability of different flow-measuring technologies for a given process fluid type, with 100\% accuracy (mastery)
\item\item{$\rightarrow$} Calculate instrument input and output values given calibrated ranges, with 100\% accuracy (mastery)
\item\item{$\rightarrow$} Identify specific instrument calibration errors (zero, span, linearity, hysteresis) from data in an ``As-Found'' table, with 100\% accuracy (mastery)
\item\item{$\rightarrow$} Solve for specified variables in algebraic formulae, with 100\% accuracy (mastery)
\item\item{$\rightarrow$} Determine the possibility of suggested faults in simple circuits and Wheatstone bridge circuits given measured values (voltage, current), schematic diagrams, and reported symptoms, with 100\% accuracy (mastery) 
\item\item{$\rightarrow$} Predict the response of automatic temperature and flow control systems to component faults and changes in process conditions, given pictorial and/or schematic illustrations 
\item\item{$\rightarrow$} Sketch proper power and signal connections between individual instruments to fulfill specified control system functions, given pictorial and/or schematic illustrations of those instruments 
\vskip 5pt
\item{$\bullet$} In a team environment and with full access to references, notes, and instructor assistance, perform the following tasks:
\item\item{$\rightarrow$} Demonstrate proper use of safety equipment and application of safe procedures while using power tools, and working on live systems
\item\item{$\rightarrow$} Communicate effectively with teammates to plan work, arrange for absences, and share responsibilities in completing all labwork
\item\item{$\rightarrow$} Construct and commission a working temperature-measurement ``loop'' consisting of an electronic temperature transmitter, signal wiring, and indicator
\item\item{$\rightarrow$} Construct and commission a working flow-measurement ``loop'' consisting of an electronic flow transmitter, signal wiring, and flow computer programmed in a text-based programming language
\item\item{$\rightarrow$} Generate accurate loop diagrams compliant with ISA standards documenting your team's systems
\vskip 5pt
\item{$\bullet$} Independently perform the following tasks with 100\% accuracy (mastery).  Multiple re-tries are allowed with different specifications/conditions each time:
%\item\item{$\rightarrow$} Build a circuit to sense temperature using a thermocouple or RTD transmitter randomly selected by the instructor
\item\item{$\rightarrow$} Research equipment manuals to sketch a complete circuit connecting a loop controller to either a 4-20 mA transmitter or a 4-20 mA final control element, with all components randomly selected by the instructor
\item\item{$\rightarrow$} Calibrate an electronic RTD temperature transmitter to specified accuracy using industry-standard calibration equipment
\item\item{$\rightarrow$} Calibrate an electronic thermocouple temperature transmitter to specified accuracy using industry-standard calibration equipment
\item\item{$\rightarrow$} Accurately simulate a thermocouple signal using a millivoltage source
\item\item{$\rightarrow$} Calculate flow rate and pressure drop for a nonlinear flow element given maximum flow specifications
\item\item{$\rightarrow$} Edit text-based programming code for a flow computer
\item\item{$\rightarrow$} Utilize command-line instructions in a Unix computer operating system environment

%%%%%%%%%%%%%%%%%%%%%%%%%%%%%%%%%%%%

\vskip 10pt

\noindent
{\bf COURSE OUTLINE:} A course calendar in electronic format (Excel spreadsheet) resides on the Y: network drive, and also in printed paper format in classroom DMC130, for convenient student access.  This calendar is updated to reflect schedule changes resulting from employer recruiting visits, interviews, and other impromptu events.  Course worksheets provide comprehensive lists of all course assignments and activities, with the first page outlining the schedule and sequencing of topics and assignment due dates.  These worksheets are available in PDF format at {\tt http://www.ibiblio.org/kuphaldt/socratic/sinst}

\vskip 5pt

\item{$\bullet$} INST241 Section 1 (Heat, temperature, RTDs, and thermocouples): 4 days theory and labwork
\item{$\bullet$} INST241 Section 2 (Thermocouples, pyrometers): 4 days theory and labwork + 1 day for mastery/proportional Exams
\item{$\bullet$} INST241 Section 3 (Fluid dynamics, pressure-based flow technologies): 4 days theory and labwork
\item{$\bullet$} INST241 Section 4 (Turbine, magnetic, true mass, and open-channel flow measurement technologies): 4 days theory and labwork + 1 day for mastery/proportional Exams

\vskip 10pt

\noindent
{\bf METHODS OF INSTRUCTION:} Course structure and methods are intentionally designed to develop critical-thinking and life-long learning abilities, continually placing the student in an active rather than a passive role.  

\item{$\bullet$} {\bf Independent study:} daily worksheet questions specify {\it reading assignments}, {\it problems} to solve, and {\it experiments} to perform in preparation (before) classroom theory sessions.  Open-note quizzes and work inspections ensure accountability for this essential preparatory work.  The purpose of this is to convey information and basic concepts, so valuable class time isn't wasted transmitting bare facts, and also to foster the independent research ability necessary for self-directed learning in your career.
\item{$\bullet$} {\bf Classroom sessions:} a combination of {\it Socratic discussion}, short {\it lectures}, {\it small-group} problem-solving, and hands-on {\it demonstrations/experiments} review and illuminate concepts covered in the preparatory questions.  The purpose of this is to develop problem-solving skills, strengthen conceptual understanding, and practice both quantitative and qualitative analysis techniques.
\item{$\bullet$} {\bf Lab activities:} an emphasis on constructing and documenting {\it working projects} (real instrumentation and control systems) to illuminate theoretical knowledge with practical contexts.  Special projects off-campus or in different areas of campus (e.g. BTC's Fish Hatchery) are encouraged.  Hands-on {\it troubleshooting exercises} build diagnostic skills.
\item{$\bullet$} {\bf Feedback questions:} sets of {\it practice problems} at the end of each course section challenge your knowledge and problem-solving ability in current as as well as first year (Electronics) subjects.  These are optional assignments, counting neither for nor against your grade.  Their purpose is to provide you and your instructor with direct feedback on what you have learned.
\item{$\bullet$} {\bf Tours and guest speakers:} quarterly {\it tours} of local industry and {\it guest speakers} on technical topics add breadth and additional context to the learning experience.

\vskip 10pt

\noindent
{\bf STUDENT ASSIGNMENTS/REQUIREMENTS:} All assignments for this course are thoroughly documented in the following course worksheets located at:

\noindent
{\tt http://www.ibiblio.org/kuphaldt/socratic/sinst/index.html} 

\vskip 5pt

\item{$\bullet$} {\tt INST241\_sec1.pdf} 
\item{$\bullet$} {\tt INST241\_sec2.pdf} 
\item{$\bullet$} {\tt INST241\_sec3.pdf} 
\item{$\bullet$} {\tt INST241\_sec4.pdf} 

%%%%%%%%%%%%%%%%%%%%%%%%%%%%%%%%%%%%

\vfil \eject

\noindent
{\bf EVALUATION AND GRADING STANDARDS:} (out of 100\% for the course grade)

\item{$\bullet$} Completion of all mastery objectives = 50\% 
\item{$\bullet$} Mastery exam scores (first attempt) = 10\% (2 exams at 5\% each)
\item{$\bullet$} Proportional exam scores = 30\% (2 exams at 15\% each)
\item{$\bullet$} Lab questions = 10\% (2 question sets at 5\% each)
\item{$\bullet$} Quiz penalty = -1\% per failed quiz
\item{$\bullet$} Tardiness penalty = -1\% per incident (1 ``free'' tardy per course)
\item{$\bullet$} Attendance penalty = -1\% per hour (12 hours ``sick time'' per quarter)
\item{$\bullet$} Extra credit = +5\% per project (assigned by instructor based on individual learning needs)

\vskip 10pt

\noindent
All grades are criterion-referenced (i.e. no grading on a ``curve'')

\medskip
\item{} 100\% $\geq$ {\bf A} $\geq$ 95\% \hskip 33pt 95\% $>$ {\bf A-} $\geq$ 90\%
\item{} 90\% $>$ {\bf B+} $\geq$ 86\% \hskip 30pt 86\% $>$ {\bf B} $\geq$ 83\% \hskip 30pt 83\% $>$ {\bf B-} $\geq$ 80\%
\item{} 80\% $>$ {\bf C+} $\geq$ 76\% \hskip 30pt 76\% $>$ {\bf C} $\geq$ 73\% \hskip 30pt 73\% $>$ {\bf C-} $\geq$ 70\% (minimum passing course grade)
\item{} 70\% $>$ {\bf D+} $\geq$ 66\% \hskip 30pt 66\% $>$ {\bf D} $\geq$ 63\% \hskip 30pt 63\% $>$ {\bf D-} $\geq$ 60\% \hskip 30pt 60\% $>$ {\bf F}
\medskip

\vskip 10pt

A graded ``preparatory'' quiz at the start of each classroom session gauges your independent learning prior to the session.  A graded ``summary'' quiz at the conclusion of each classroom session gauges your comprehension of important concepts covered during that session.  If absent during part or all of a classroom session, you may receive credit by passing comparable quizzes afterward or by having your preparatory work (reading outlines, work done answering questions) thoroughly reviewed prior to the absence.  

\vskip 10pt

Absence on a scheduled exam day will result in a 0\% score for the proportional exam unless you provide documented evidence of an unavoidable emergency.  

\vskip 10pt

If you fail a mastery exam, you must re-take a different version of that mastery exam on a different day.  Multiple re-tries are allowed, on a different version of the exam each re-try.  There is no penalty levied on your course grade for re-taking mastery exams, but failure to successfully pass a mastery exam by the due date (i.e. by the date of the {\it next} exam in the course sequence) will result in a failing grade (F) for the course.  

\vskip 10pt

If any other ``mastery'' objectives are not completed by their specified deadlines, your overall grade for the course will be capped at 70\% (C- grade), and you will have one more school day to complete the unfinished objectives.  Failure to complete those mastery objectives by the end of that extra day (except in the case of documented, unavoidable emergencies) will result in a failing grade (F) for the course.

\vskip 10pt

``Lab questions'' are assessed by individual questioning, at any date after the respective lab objective (mastery) has been completed by your team.  These questions serve to guide your completion of each lab exercise and confirm participation of each individual student.  Grading is as follows: full credit for thorough, correct answers; half credit for partially correct answers; and zero credit for major conceptual errors.  All lab questions must be answered by the due date of the lab exercise.

\vskip 10pt

Extra credit opportunities exist for each course, and may be assigned to students upon request.  The student and the instructor will first review the student's performance on feedback questions, homework, exams, and any other relevant indicators in order to identify areas of conceptual or practical weakness.  Then, both will work together to select an appropriate extra credit activity focusing on those identified weaknesses, for the purpose of strengthening the student's competence.  A due date will be assigned (typically two weeks following the request), which must be honored in order for any credit to be earned from the activity.  Extra credit may be denied at the instructor's discretion if the student has not invested the necessary preparatory effort to perform well (e.g. lack of preparation for daily class sessions, poor attendance, no feedback questions submitted, etc.).

%%%%%%%%%%%%%%%%%%%%%%%%%%%%%%%%%%%%

\vfil \eject

\noindent
{\bf REQUIRED STUDENT SUPPLIES AND MATERIALS:} 

\item{$\bullet$} Course worksheets available for download in PDF format
\item{$\bullet$} {\it Lessons in Industrial Instrumentation} textbook, available for download in PDF format
\itemitem{$\rightarrow$} Access worksheets and book at: {\tt http://www.ibiblio.org/kuphaldt/socratic/sinst}
\item{$\bullet$} Spiral-bound notebook for reading annotation, homework documentation, and note-taking.
\item{$\bullet$} Instrumentation reference CD-ROM (free, from instructor).  This disk contains many tutorials and datasheets in PDF format to supplement your textbook(s).
\item{$\bullet$} Tool kit (see detailed list)
\item{$\bullet$} Simple scientific calculator (non-programmable, non-graphing, no unit conversions, no numeration system conversions), TI-30Xa or TI-30XIIS recommended
\item{$\bullet$} Portable personal computer with Ethernet port and wireless.  Windows OS strongly preferred, tablets discouraged.

\vskip 10pt

\noindent
{\bf ADDITIONAL INSTRUCTIONAL RESOURCES:} 

\item{$\bullet$} The BTC Library hosts a substantial collection of textbooks and references on the subject of Instrumentation, as well as links in its online catalog to free Instrumentation e-book resources available on the Internet.
\item{$\bullet$} ``BTCInstrumentation'' channel on YouTube ({\tt http://www.youtube.com/BTCInstrumentation}), hosts a variety of short video tutorials and demonstrations on instrumentation.
\item{$\bullet$} Instrumentation student club meets regularly to set up industry tours, raise funds for scholarships, and serve as a general resource for Instrumentation students.
\item{$\bullet$} ISA website ({\tt http://www.isa.org}) provides all of its standards in electronic format, many of which are freely available to ISA members.
\item{$\bullet$} {\it Instrument Engineer's Handbook, Volume 1: Process Measurement and Analysis}, edited by B\'ela Lipt\'ak, published by CRC Press.  4th edition ISBN-10: 0849310830 ; ISBN-13: 978-0849310836.
\item{$\bullet$} {\it Purdy's Instrument Handbook}, by Ralph Dewey.  ISBN-10: 1-880215-26-8.  A pocket-sized field reference on basic measurement and control.
\item{$\bullet$} {\it Cad Standard} (CadStd) or similar AutoCAD-like drafting software (useful for sketching loop and wiring diagrams).  Cad Standard is a simplified clone of AutoCAD, and is freely available at: {\tt http://www.cadstd.com}

\vskip 10pt

\noindent
{\bf CAMPUS EMERGENCIES:} If an emergency arises, your instructor may inform you of actions to follow.  You are responsible for knowing emergency evacuation routes from your classroom.  If police or university officials order you to evacuate, do so calmly and assist those needing help.  You may receive emergency information alerts via the building enunciation system, text message, email, or BTC's webpage ({\tt http://www.btc.ctc.edu}), Facebook or Twitter.  Refer to the emergency flipchart in the lab room (located on the main control panel) for more information on specific types of emergencies.

\vskip 10pt

\noindent
{\bf ACCOMMODATIONS:} If you think you could benefit from classroom accommodations for a disability (physical, mental, emotional, or learning), please contact our Accessibility Resources office.  Call (360)-752-8345, email {\tt ar@btc.ctc.edu}, or stop by the AR Office in the Admissions and Student Resource Center (ASRC), Room 106, College Services Building




\vskip 10pt




\vfil 

\underbar{file {\tt INST241syllabus}}
\eject
%(END_FRONTMATTER)


