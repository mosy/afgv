
% Copyright 2010, Tony R. Kuphaldt, released under the Creative Commons Attribution License (v 1.0)
% This means you may do almost anything with this work of mine, so long as you give me proper credit

%(BEGIN_FRONTMATTER)

\centerline{\bf Course Syllabus} \bigskip 
 
\noindent
{\bf INSTRUCTOR CONTACT INFORMATION:}

Tony Kuphaldt

(360)-752-8477 [office phone]

(360)-752-7277 [fax]

{\tt tkuphald@btc.ctc.edu}

\vskip 10pt

\noindent
{\bf DEPT/COURSE \#:} INST 241

\vskip 10pt

\noindent
{\bf CREDITS:} 6 \hskip 30pt {\bf Lecture Hours:} 26 \hskip 30pt {\bf Lab Hours:} 82 \hskip 30pt {\bf Work-based Hours:} 0

\vskip 10pt

\noindent
{\bf COURSE TITLE:} Temperature and Flow Measurement

\vskip 10pt

\noindent
{\bf COURSE OUTCOMES:} Construct, calibrate, analyze, document, and efficiently diagnose instrumented systems using industry-standard equipment to measure temperature and fluid flow.

\vskip 10pt

\noindent
{\bf COURSE DESCRIPTION:} In this course you will learn how to precisely measure both temperature and fluid flow in a variety of applications, as well as accurately calibrate and efficiently troubleshoot temperature and flow measurement systems.   {\bf Pre/Corequisite course:} INST 240 (Pressure and Level Measurement)

\vskip 10pt

\noindent
{\bf COURSE OUTLINE:} a course calendar in electronic format (Excel spreadsheet) resides on the Y: network drive, and also in printed paper format in classroom DMC130, for convenient student access.  This calendar is updated to reflect schedule changes resulting from employer recruiting visits, interviews, and other impromptu events.  Course worksheets provide comprehensive lists of all course assignments and activities, with the first page outlining the schedule and sequencing of topics and assignment due dates.  These worksheets are available in PDF format at {\tt http://openbookproject.net/books/socratic/sinst}

\vskip 5pt

\item{$\bullet$} INST241 Section 1 (Heat, temperature, RTDs, and thermocouples): 4 days theory and labwork
\item{$\bullet$} INST241 Section 2 (Thermocouples, pyrometers): 4 days theory and labwork + 1 day for mastery/proportional Exams
\item{$\bullet$} INST241 Section 3 (Fluid dynamics, pressure-based flow technologies): 4 days theory and labwork
\item{$\bullet$} INST241 Section 4 (Turbine, magnetic, true mass, and open-channel flow measurement technologies): 4 days theory and labwork + 1 day for mastery/proportional Exams

%%%%%%%%%%%%%%%%%%%%%%%%%%%%%%%%%%%%

\vfil \eject

\noindent
{\bf STUDENT PERFORMANCE OBJECTIVES:}

\item{$\bullet$} {\bf Mastery exams (two per course):} without references or notes, within a limited time (3 hours total for mastery and proportional exams), independently perform the following activities with no errors given a maximum of two attempts per exam sitting (up to three exam sittings allowed with a 10\% score deduction levied on the proportional exam score if not passed on first sitting).  At least 60\% of exam questions are ``Application'' level or higher according to Bloom's Taxonomy.
\item\item{$\rightarrow$} \underbar{Exam 1}: Build a circuit to sense temperature using a thermocouple or RTD temperature transmitter % Bloom's: COMPREHENSION
\item\item{$\rightarrow$} \underbar{Exam 1}: Sketch proper wire connections for RTD or thermocouple temperature transmitter circuits % Bloom's: APPLICATION
\item\item{$\rightarrow$} \underbar{Exam 1}: Calculate voltage in a simple circuit containing either an RTD or a thermocouple % Bloom's: APPLICATION (voltage divider configuration)
\item\item{$\rightarrow$} \underbar{Exam 1}: Calculate instrument calibration points given input and output ranges % Bloom's: COMPREHENSION
\item\item{$\rightarrow$} \underbar{Exam 1}: Circuit Fault Review: determine possibility of open/short faults in a simple circuit given measured values (voltage, current) % Bloom's: APPLICATION
\item\item{$\rightarrow$} \underbar{Exam 1}: INST251 Review: identify proper controller action (direct or reverse) for a process
\item\item{$\rightarrow$} \underbar{Exam 1}: INST261 Review: sketch an equivalent ladder logic diagram for a given truth table
\item\item{$\rightarrow$} \underbar{Exam 1}: {\bf Only a simple (non-programmable) scientific calculator will be allowed for this exam (and its proportional counterpart)}
\vskip 5pt
\item\item{$\rightarrow$} \underbar{Exam 2}: Build a circuit with a ``smart'' transmitter and use a HART communicator to re-range it % Bloom's: COMPREHENSION
\item\item{$\rightarrow$} \underbar{Exam 2}: Calculate flow rate / pressure drop for a nonlinear flow element % Bloom's: APPLICATION
\item\item{$\rightarrow$} \underbar{Exam 2}: Identify suitability of basic flow-measuring instruments to different processes % Bloom's: EVALUATION
\item\item{$\rightarrow$} \underbar{Exam 2}: Calculate turbine flowmeter calibration points given ranges % Bloom's: COMPREHENSION
\item\item{$\rightarrow$} \underbar{Exam 2}: Circuit Fault Review: determine possibility of open/short faults in a simple circuit given measured values (voltage, current) % Bloom's: APPLICATION
\item\item{$\rightarrow$} \underbar{Exam 2}: INST251 Review: Perform either numerical differentiation or numerical integration on a simple mathematical function (graphed)
\item\item{$\rightarrow$} \underbar{Exam 2}: INST262 Review: Determine the effect of a component fault or condition change in an automatically-controlled process
\vskip 5pt
\item{$\bullet$} {\bf Proportional exams (two per course):} without references or notes, independently solve several quantitative, conceptual, and diagnostic problems relating to the course subject.  At least 60\% of exam questions are ``Application'' level or higher according to Bloom's Taxonomy.
\vskip 5pt
\item{$\bullet$} {\bf Lab exercises (two per course):} in a team environment and with full access to references, notes, and instructor assistance; build and document functioning instrumentation systems as documented in the Lab Exercise questions found in all course worksheets.  Each lab exercise includes a set of qualitative and conceptual questions to be answered individually without references or notes, and also lists mastery objectives for the lab exercise (must be completed with no errors) including:
\item\item{$\rightarrow$} Generate an accurate loop diagram compliant with ISA standards documenting your team's system, personally verified by the instructor.
\item\item{$\rightarrow$} Calibrate all system instruments to specified accuracy using lab calibration equipment, personally verified by the instructor.
\item\item{$\rightarrow$} Diagnose a fault placed in another team's control system by the instructor within a limited time (5 minutes max.) using no test equipment except a multimeter, with each step logically justified in the instructor's direct presence.
\vskip 5pt
\item{$\bullet$} {\bf Quizzes (daily):} with access to notes, independently answer quantitative and conceptual questions relating to the day's assigned questions and reading.
\vskip 5pt
\item{$\bullet$} {\bf Feedback question sets (four per course):} ungraded exercises designed to review critical concepts and provide bidirectional student/instructor feedback on learning prior to exams.

%%%%%%%%%%%%%%%%%%%%%%%%%%%%%%%%%%%%

\vfil \eject

\noindent
{\bf METHODS OF INSTRUCTION:} Course structure and methods are intentionally designed to develop critical-thinking and life-long learning abilities, continually placing the student in an active rather than a passive role.  

\item{$\bullet$} {\bf Independent study:} daily worksheet questions specify {\it reading assignments}, {\it problems} to solve, and {\it experiments} to perform in preparation (before) classroom theory sessions.  Open-note quizzes and work inspections ensure accountability for this essential preparatory work.  The purpose of this is to convey information and basic concepts, so valuable class time isn't wasted transmitting bare facts, and also to foster the independent research ability necessary for self-directed learning in your career.
\item{$\bullet$} {\bf Classroom sessions:} a combination of {\it Socratic discussion}, short {\it lectures}, {\it small-group} problem-solving, and hands-on {\it demonstrations/experiments} review and illuminate concepts covered in the preparatory questions.  The purpose of this is to develop problem-solving skills, strengthen conceptual understanding, and practice both quantitative and qualitative analysis techniques.
\item{$\bullet$} {\bf Lab activities:} an emphasis on constructing and documenting {\it working projects} (real instrumentation and control systems) to illuminate theoretical knowledge with practical contexts.  Special projects off-campus or in different areas of campus (e.g. BTC's Fish Hatchery) are encouraged.  Hands-on {\it troubleshooting exercises} build diagnostic skills.
\item{$\bullet$} {\bf Tours and guest speakers:} quarterly {\it tours} of local industry and {\it guest speakers} on technical topics add breadth and additional context to the learning experience.

\vskip 10pt

\noindent
{\bf STUDENT ASSIGNMENTS/REQUIREMENTS:} All assignments for this course are thoroughly documented in the following course worksheets located at:

\noindent
{\tt http://openbookproject.net/books/socratic/sinst/index.html} 

\vskip 5pt

\item{$\bullet$} {\tt INST241\_sec1.pdf} 
\item{$\bullet$} {\tt INST241\_sec2.pdf} 
\item{$\bullet$} {\tt INST241\_sec3.pdf} 
\item{$\bullet$} {\tt INST241\_sec4.pdf} 

%%%%%%%%%%%%%%%%%%%%%%%%%%%%%%%%%%%%

\vfil \eject

\noindent
{\bf EVALUATION AND GRADING STANDARDS:} (out of 100\% for the course grade)

\item{$\bullet$} Mastery exams and mastery lab objectives = 50\% of course grade
\item{$\bullet$} Proportional exams = 40\% (2 exams at 20\% each)
\item{$\bullet$} Lab questions = 10\% (2 question sets at 5\% each)
\item{$\bullet$} Quiz penalty = -1\% per failed quiz
\item{$\bullet$} Tardiness penalty = -1\% per incident (1 ``free'' tardy per course)
\item{$\bullet$} Attendance penalty = -1\% per hour (12 hours ``sick time'' per quarter)
\item{$\bullet$} Repair bonus = +5\% per repaired instrument (instrument's broken and repaired statuses must be verified by the instructor)

\vskip 10pt

\noindent
All grades are criterion-referenced (i.e. no grading on a ``curve'')

\begin{itemize}
\item{} 100\% $\geq$ {\bf A} $\geq$ 95\% \hskip 33pt 95\% $>$ {\bf A-} $\geq$ 90\%
\item{} 90\% $>$ {\bf B+} $\geq$ 86\% \hskip 30pt 86\% $>$ {\bf B} $\geq$ 83\% \hskip 30pt 83\% $>$ {\bf B-} $\geq$ 80\%
\item{} 80\% $>$ {\bf C+} $\geq$ 76\% \hskip 30pt 76\% $>$ {\bf C} $\geq$ 73\% \hskip 30pt 73\% $>$ {\bf C-} $\geq$ 70\% (minimum passing course grade)
\item{} 70\% $>$ {\bf D+} $\geq$ 66\% \hskip 30pt 66\% $>$ {\bf D} $\geq$ 63\% \hskip 30pt 63\% $>$ {\bf D-} $\geq$ 60\% \hskip 30pt 60\% $>$ {\bf F}
\medskip

\vskip 10pt

Failing a mastery exam will result in a 10\% deduction from the proportional exam score, and you get a maximum of two re-takes (``sittings'') to pass new versions of the same mastery exam which must occur before the next exam date.  Failure to pass the mastery within three sittings will result in a failing grade (F) for the course.  Absence on a scheduled exam day will result in a 0\% score for the proportional exam unless you provide documented evidence of an unavoidable emergency.  

If any other ``mastery'' objectives are not completed by their specified deadlines, your overall grade for the course will be capped at 70\% (C- grade), and you will have one more school day to complete the unfinished objectives.  Failure to complete those mastery objectives by the end of that extra day (except in the case of documented, unavoidable emergencies) will result in a failing grade (F) for the course.

``Lab questions'' are assessed by individual questioning, at any date after the respective lab objective (mastery) has been completed by your team.  These questions serve to guide your completion of each lab exercise and confirm participation of each individual student.  Grading is as follows: full credit for thorough, correct answers; half credit for partially correct answers; and zero credit for major conceptual errors.  All lab questions must be answered by the due date of the lab exercise.

%%%%%%%%%%%%%%%%%%%%%%%%%%%%%%%%%%%%

\vfil \eject

\noindent
{\bf REQUIRED STUDENT SUPPLIES AND MATERIALS:} 

\item{$\bullet$} Course worksheets available for download in PDF format
\item{$\bullet$} {\it Lessons in Industrial Instrumentation} textbook, available for download in PDF format
\itemitem{$\rightarrow$} Access worksheets and book at: {\tt http://openbookproject.net/books/socratic/sinst}
\item{$\bullet$} Spiral-bound notebook for reading annotation, homework documentation, and note-taking.
\item{$\bullet$} Instrumentation reference CD-ROM (free, from instructor).  This disk contains many tutorials and datasheets in PDF format to supplement your textbook(s).
\item{$\bullet$} Tool kit (see detailed list)
\item{$\bullet$} Simple scientific calculator (non-programmable, non-graphing, no unit conversions, no numeration system conversions), TI-30Xa or TI-30XIIS recommended

\vskip 10pt

\noindent
{\bf ADDITIONAL INSTRUCTIONAL RESOURCES:} 

\item{$\bullet$} The BTC Library hosts a substantial collection of textbooks and references on the subject of Instrumentation, as well as links in its online catalog to free Instrumentation e-book resources available on the Internet.
\item{$\bullet$} ``BTCInstrumentation'' channel on YouTube ({\tt http://www.youtube.com/BTCInstrumentation}), hosts a variety of short video tutorials and demonstrations on instrumentation.
\item{$\bullet$} ISA Student Section at BTC meets regularly to set up industry tours, raise funds for scholarships, and serve as a general resource for Instrumentation students.  Membership in the ISA is \$10 per year, payable to the national ISA organization.  Membership includes a complementary subscription to {\it InTech} magazine.
\item{$\bullet$} ISA website ({\tt http://www.isa.org}) provides all of its standards in electronic format, many of which are freely available to ISA members.
\item{$\bullet$} {\it Instrument Engineer's Handbook, Volume 1: Process Measurement and Analysis}, edited by B\'ela Lipt\'ak, published by CRC Press.  4th edition ISBN-10: 0849310830 ; ISBN-13: 978-0849310836.
\item{$\bullet$} {\it Purdy's Instrument Handbook}, by Ralph Dewey.  ISBN-10: 1-880215-26-8.  A pocket-sized field reference on basic measurement and control.
\item{$\bullet$} {\it Cad Standard} (CadStd) or similar AutoCAD-like drafting software (useful for sketching loop and wiring diagrams).  Cad Standard is a simplified clone of AutoCAD, and is freely available at: {\tt http://www.cadstd.com}

\vskip 10pt




\vfil 

\underbar{file {\tt INST241syllabus}}
\eject
%(END_FRONTMATTER)


