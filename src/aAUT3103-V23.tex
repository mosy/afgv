% Start preamble
\documentclass[12pt,a4paper]{article}
\usepackage{geometry}
 \geometry{
 a4paper,
 total={170mm,257mm},
 left=20mm,
 top=20mm,
 }
\usepackage[utf8]{inputenc}
\usepackage[T1]{fontenc}
\usepackage[pdftex]{graphicx}
\graphicspath{{./}}
\usepackage{enumitem}
\usepackage{pdfpages}
\usepackage{hyperref}
\usepackage{tikz}
\usepackage{attachfile}
\usepackage{epstopdf}
\usepackage{array}
\usepackage{multirow}
\usepackage{multicol}
\usepackage{float}
%\usepackage[table]{xcolor,colorbl}
\setlength{\textwidth}{17cm}
\setlength{\oddsidemargin}{-0.5cm}
\setlength{\evensidemargin}{-0.5cm}
%\setlenght{\headsep}{0cm}
\setlength\parindent{0pt}
%\setlength{\extrarowheight}{3pt}
\usepackage{listings}
%\usepackage{xcolor}

\input{arduinoLanguage.tex}
%%%%%% Counting oppgaves %%%%%%
 \newcount\questnum \questnum=0
 \def\oppgave{
            \advance\questnum by 1
	    \ifthenelse{\questnum>0\AND \questnum<9}
	    {
                \vskip 1cm
		\textbf{Oppgave}\hskip 5pt\the\questnum \hfill \hfill
		\vskip 3pt
		\hrule
	\vskip 0.5cm}
	{
                \vskip 1cm
		\textbf{Oppgave}\hskip 5pt \the\questnum \hfill \hfill)
		\vskip 3pt \hrule \vskip 0.5cm }

		}

\begin{document}
\title{Eksamen AUT3101 V23}
\author{UDIR}
\maketitle
\href{https://rfka-my.sharepoint.com/:b:/g/personal/fred-olav_mosdal_skole_rogfk_no/EfPLJCqUpSRLqsAt8eUz9gABIPnCilD3gsXV23CZ_eATpQ?e=AQLd6l}{vedlegg}\\
Situasjonsbeskrivelse
\vskip  0.25cm
Du jobber i et team med automatikere, elektrikere og mekanikere som skal planlegge, gjennomføre og dokumentere årlig kontroll under vedlikeholdsstansen i 2 uker uten produksjon på utrustning tilkoplet akkumulator NDE10BB001, og bistå ved oppstart til normal drift. Arbeidsoppdragene skal dere løse i samarbeid med de lokale operatørene og bedriftens automasjonsleder.

\vskip  0.25cm
Hovedfokus for vedlikeholdsstansen er vedlikehold på pumper og ventiler. Akkumulatoren har vært i drift i fem år og nå skal alle analoge og digitale signaler kontrolleres og funksjonstestes. Prosess- og nødavstengnings løsningen skal spesielt testes knyttet til oppstart mot normal drift.

\vskip  0.25cm
Arbeidslederen tar en prat med deg før arbeidsoppdragene hvor han forteller at bedriften har fokus på kvalitetsarbeid og stiller krav til godt faglig utført arbeid som blant annet innebærer;
\begin{itemize}
	\item Sikker utførelse av både elektrisk og mekanisk arbeid
	\item Utarbeide nødvendig dokumentasjon i henhold til bedriftens standard som kan brukes av annet fagpersonell.
	\item Bruk av fagterminologi for å beskrive anlegget med tilhørende komponenter og viser faglig forståelse
	\item Velge egnet verktøy og instrumenter, og riktig bruk av disse
	\item Struktur og fremgangsmåte under feilsøking og reflektere over resultatene
\end{itemize}
Sikker utførelse av både elektrisk og mekanisk arbeid

Utarbeide nødvendig dokumentasjon i henhold til bedriftens standard som kan brukes av annet fagpersonell.

Bruk av fagterminologi for å beskrive anlegget med tilhørende komponenter og viser faglig forståelse

Velge egnet verktøy og instrumenter, og riktig bruk av disse

Struktur og fremgangsmåte under feilsøking og reflektere over resultatene

\vskip  0.25cm
 

Arbeidslederen din forteller at han har tatt fram et utvalg fra dokumentasjonen for anlegget til deg. Siden det er et utvalg, er sidenummereringen ikke komplett. Videre sier han at dokumentasjonen omtaler hele anlegget fra et systemperspektiv, og det kan hende han har tatt med mer enn du trenger. Du må derfor se gjennom hele omtalen av anlegget for å finne fram til det du mener er relevant. Arbeidslederen din har også erfart at originaldokumentasjon kan inneholde feil og mangler. Du må derfor huske å være faglig kritisk, og du må kanskje se flere deler av dokumentasjonen i sammenheng for å komme videre i arbeidet.
\oppgave{}%1
Du skal bytte aktuator/ventilstiller på ventil NDE20AA301-MS01 og sette denne i drift. 
\vskip 0.25cm

Planlegg, beskriv gjennomføring og dokumenter jobben. Tegn skisser til forklaringene og ta bilde av skissene. 
\vskip 0.25cm
\includegraphics[width=1\textwidth]{../output/noGPLimages/AUT3103V23x01.png}\\
\newpage
\oppgave{}%2
Du gjennomfører trykktesting og oppdager at kobberrøret med dimensjon 15mm fra aktuatoren er skadet. Du må derfor lage og montere et nytt rør. Se bildet under for plassering og utsende. 

\vskip 0.25cm
Planlegg, beskriv gjennomføring og dokumenter jobben. Tegn skisser til forklaringene og ta bilde av skissene.

\vskip 0.25cm
\includegraphics[width=1\textwidth]{../output/noGPLimages/AUT3103V23x01.png}\\
\newpage
\oppgave{}%3
Du skal skifte kabel fra servicebryter til en pumpe drevet av en asynkronmotor. Asynkronmotoren er tilkoblet en frekvensomformer. 

\vskip 0.25cm
Motorens data er: 30 kW, 230/400V, 103/60A, cos$\phi$=0,85. 

\vskip 0.25cm
Planlegg, beskriv gjennomføring og dokumenter jobben. Tegn skisser til forklaringene og ta bilde av skissene.

\vskip 0.25cm
\includegraphics[width=1\textwidth]{../output/noGPLimages/AUT3103V23x02.png}\\
\newpage
\oppgave{}%4
Under revisjonsstansen skal du kalibrere temperaturvakt NDD30CT001 på FV-vann fra dumpkondensor. 

\vskip 0.25cm
Planlegg, beskriv gjennomføring og dokumenter jobben. Tegn skisser til forklaringene og ta bilde av skissene.


\vskip 0.25cm
\includegraphics[width=1\textwidth]{../output/noGPLimages/AUT3103V23x04.png}\\
\newpage
\oppgave{}%5
Alarmgrensene til temperaturvakten blir konfigurert i signalomformer NDD30CT801, og utgangssignalene som går videre til sikkerhets-PLS Pluto K203/I0 skal funksjonstestes. 

\vskip 0.25cm
Planlegg, beskriv gjennomføring og dokumenter jobben. Tegn skisser til forklaringene og ta bilde av skissene.


\vskip 0.25cm
\includegraphics[width=1\textwidth]{../output/noGPLimages/AUT3103V23x05.png}\\
\newpage
\oppgave{}%6
I forbindelse med oppstart spør operatørene deg om hvordan «split range» er koplet og viser til punkt 4.2.1 i vedlegget som omhandler differansetrykk.
\vskip 0.25cm
4.2.1 Differansetrykk (hentet fra vedlegg og oversatt til norsk)

\vskip 0.25cm
Reguleringsventilene NDE30AA301/302 og pumpene NDE30AP001, 002 og 003 regulerer differansetrykk slik at vann ved høyt differansetrykk slippes tilbake til tanken og ved lavt differansetrykk, pumpes vann ut på nettet.

\vskip 0.25cm
Ventilene opererer i et split- range arrangement, og pumpene utføres med en lavbelastningspumpe 30 \% og 2x50 \% høybelastningspumper som opererer i rekkefølge.
\vskip 0.25cm
Pumpene er konstruert med fastkoblede sikkerhetsfunksjoner som beskyttelse for høyt og lavt trykk.
\vskip 0.25cm
Tegn og forklar kort en prinsipiell flerlinjet skisse som viser hvordan «split-range» kan være koplet, og vis prinsippet for kalibrering for et «split-range»- arrangement og ta bilde av skissen. 
\vskip 0.25cm


\vskip 0.25cm
\newpage
\end{document}
