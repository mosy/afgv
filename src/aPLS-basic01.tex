
% Start preamble
\documentclass[12pt,a4paper]{article}
\usepackage{geometry}
 \geometry{
 a4paper,
 total={170mm,257mm},
 left=20mm,
 top=20mm,
 }
\usepackage[utf8]{inputenc}
\usepackage[T1]{fontenc}
\usepackage[pdftex]{graphicx}
\graphicspath{{./}}
\usepackage{enumitem}
\usepackage{pdfpages}
\usepackage{hyperref}
\usepackage{tikz}
\usepackage{attachfile}
\usepackage{epstopdf}
\usepackage{array}
\usepackage{multirow}
\usepackage{multicol}
\usepackage{float}
%\usepackage[table]{xcolor,colorbl}
\setlength{\textwidth}{16cm}
\setlength{\oddsidemargin}{-0.5cm}
\setlength{\evensidemargin}{-0.5cm}
%\setlenght{\headsep}{0cm}
\setlength\parindent{0pt}
%\setlength{\extrarowheight}{3pt}
\usepackage{listings}
%\usepackage{xcolor}

\input{arduinoLanguage.tex}
%%%%%% Counting oppgaves %%%%%%
 \newcount\questnum \questnum=0
 \def\oppgave{
            \advance\questnum by 1
	    \ifthenelse{\questnum>0\AND \questnum<9}
	    {
                \vskip 1cm
		\textbf{Oppgave}\hskip 5pt\the\questnum \hfill \hfill(6p)
		\vskip 3pt
		\hrule
	\vskip 0.5cm}
	{
                \vskip 1cm
		\textbf{Oppgave}\hskip 5pt \the\questnum \hfill \hfill(12p)
		\vskip 3pt \hrule \vskip 0.5cm }

		}

% End preamble

\begin{document}
\title{Prøve i grunnleggende PLS-programmering}
\author{Faglærer: Fred-Olav Mosdal 90507684\\
Oppgave 1-8 gjøres på ark. Hjelpemidler: kalkulator. \\
Oppgave 9 gjøres på PC. Hjelpemidler: Alle ikke kommuniserende\\
Elektronisk del av prøve sendes på email til: \\
fred-olav.mosdal@skole.rogfk.no\\
}
\maketitle
\paragraph{Del I}
\vskip 2.5pt 
Oppgaven omhandler funksjoner rundt et transportbånd for metall deler.
Du skal utvide funksjonaliteten til styringen for hver oppgave. Det
starter enkelt og blir vanskligere og vanskligere. 
Det skal være skjermstyring som gør det lettere å teste funksjonen. 
\vskip 1cm
\begin{center}
\includegraphics[width=0.5\textwidth]{../output/transportbånd.png}
\end{center}

\oppgave{}

Lag et program for Start og Stopp av transportbåndet 
\oppgave{}

Legg til funksjonalitet slik at transportbåndet også stopper om
det mottar tilbakemelding om at nødstopp er aktivert. 
\oppgave{}

Sett opp en forsinket start på 3s av transportbåndet 
\oppgave{}


Når esken er full (10 deler), skal transportbåndet stoppe og det skal gis signal om at eske er full. Ved å trykke Ny Eske skal det være mulig å starte båndet på nytt. 
\oppgave{}


Om det ikke er registrert ny del på 15s skal transportbåndet stoppe og et varsellys aktiveres. Når en ser at det er nye deler klare, starer en bare båndet manuelt igjen. 
\oppgave{}

Legg til funksjonalitet slik at en lampe blinker 1Hz under forsinket start. 

\paragraph{Del II}

En kornsilo skal fyllest ved hjelp av en pumpe
\begin{center}
\includegraphics[width=0.5\textwidth]{i08012x01.png}
\end{center}
Der er to nivåvakter i hver silo (L og H) disse gir \textquotedblright{}
TRUE\textquotedblright{} når nivået ligger over giveren.

\vskip 10pt 
\begin{tabular}{|c|c|c|c|}
\hline 
Tilkoblet utstyr & IO på RIO & Variabel & Beskrivelse av tilkoblet utstyr\tabularnewline
\hline 
\hline 
Start drift & Bryter1 & Start & \tabularnewline
\hline 
Stopp drift & Bryter2 & Stopp & \tabularnewline
\hline 
H Sensor & Bryter3 & LevelHigh & \tabularnewline
\hline 
L Sensor & Bryter4 & LevelLow & \tabularnewline
\hline 
Drifts Lys & Lys1 & Drift & \tabularnewline
\hline 
Pumpe Lys & Lys2 & Pumpe & \tabularnewline
\hline 
Alarm Lys & Lys3 & AlarmLys & \tabularnewline
\hline 
Alarm Lyd & Lys4 & AlarmLyd & \tabularnewline
\hline 
\end{tabular}

\oppgave{}
Lag et proram som kan starte og stoppe drift av anlegget. Når anlegget er i drift skal pumpen startes når nivået kommer under L og pumpen skal gå til nivået kommer over H
\oppgave{}
Legg til Manuell styring av pumpen
\oppgave{}
Det skal aktiveres en alarm om det tar mer en 10min (10s) å komme over L nivå (etter at den har vært under). Alarmen skal ha bekreft og resett funksjon.

\end{document}
