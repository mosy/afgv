
%(BEGIN_QUESTION)
% Copyright 2012, Tony R. Kuphaldt, released under the Creative Commons Attribution License (v 1.0)
% This means you may do almost anything with this work of mine, so long as you give me proper credit
% Start preamble
\documentclass[12pt,a4paper]{article}
\usepackage{geometry}
 \geometry{
 a4paper,
 total={170mm,257mm},
 left=20mm,
 top=20mm,
 }
\usepackage[utf8]{inputenc}
\usepackage[table]{xcolor}
\usepackage[T1]{fontenc}
\usepackage[pdftex]{graphicx}
\graphicspath{{./}}
\usepackage{enumitem}
\usepackage{pdfpages}
\usepackage{hyperref}
\usepackage{tikz}
\usepackage{attachfile}
\usepackage{epstopdf}
\usepackage{array}
\usepackage{multirow}
\usepackage{multicol}
\usepackage{float}
\usepackage{comment}
%\usepackage[table]{xcolor,colorbl}
\setlength{\textwidth}{16cm}
\setlength{\oddsidemargin}{-0.5cm}
\setlength{\evensidemargin}{-0.5cm}
%\setlenght{\headsep}{0cm}
\setlength\parindent{0pt}
%\setlength{\extrarowheight}{3pt}
\usepackage{listings}
%\usepackage{xcolor}

\input{arduinoLanguage.tex}
%%%%%% Counting oppgaves %%%%%%
 \newcount\questnum \questnum=0
 \def\oppgave{
            \advance\questnum by 1
            \ifnum \questnum > 0
                 \hrule
                 \vskip 3pt
                 \leftline{Oppgave \the\questnum}
                 \vskip 3pt \fi}
 %%%%%%%%%%%%%%%%%%%%%%
%%%%%%%%%%%%%%%%%%%%%%


%%%%%% Counting answers %%%%%%
\newcount\answnum \answnum=0
\def\svar{
           \advance\answnum by 1
           \ifnum \answnum > 0
                \hrule
                \vskip 3pt
                \leftline{Svar \the\answnum}
                \vskip 3pt \fi}
%%%%%%%%%%%%%%%%%%%%%%


%%%%%% Counting notes %%%%%%
\newcount\explnum \explnum=0
\def\notes{
           \advance\explnum by 1
           \ifnum \explnum > 0
                \hrule
                \vskip 3pt
                \leftline{Notes \the\explnum}
                \vskip 3pt \fi}
%%%%%%%%%%%%%%%%%%%%%%

% End preamble

\begin{document}

\begin{centering}
\Huge{\textbf{Prøve 01 PLS programmering}}\\
\end{centering}
\vskip 2cm 
Følgende kompetansemål er relevante for prøven:
\begin{itemize}[noitemsep]


	\item programmere, montere og sette i drift programmerbare styresystemer for elektriske-, pneumatiske- og hydrauliske anlegg og gjøre rede for hvordan utstyret fungerer, og hvilke funksjoner det har

	\item programmere, montere og sette i drift programmerbare styresystemer for elektriske-, pneumatiske- og hydrauliske anlegg og gjøre rede for hvordan utstyret fungerer, og hvilke funksjoner det har
\end{itemize}
\vskip 2.5pt 
Hjelpemidler:\begin{itemize}[noitemsep]
	\item Alle ikke kommuniserende
\end{itemize}

\vskip 5pt 
\vskip 10pt 
\vskip 2.5pt 
\vskip 2.5pt 
Når oppgaven skal leveres kan elevene slå på trådløst nettverkt og sende oppgaven på mail til:
\vskip 2.5pt 
fred-olav.mosdal@skole.rogfk.no
\vskip 2.5pt 
I emnefeltet skal det stå: PLS prøve
\vskip 2.5pt
Oppgaven SKAL sendes fra skolemailen. 
\vskip 2cm   
Konaktinformasjon:
\begin{itemize}[noitemsep]
	\item Kontaktlærer: Fred-Olav Mosdal
	\item TLF: 90507684
\end{itemize}
\vfil\eject
\textbf{System beskrivelse}

$$\includegraphics[width=15.5cm]{./aProgrammering01x01.jpg}$$
Anlegget skal sortere ulike esker som kommer fra produksjonen, til et transportbånd for utkjøring. Det kommer 4 ulike typer esker fra anlegget, transportbåndene som disse kommer fra er navngitt Input 1-4. 

\vskip 5pt 
Hvert transportbånd har en ende sensor kalt InputXEndSensor. Denne registrerer at en eske er kommet til enden på båndet og er klar for å kjøres videre. På Input3- og 4 er det i tillegg en InputXBoxSensor som registrerer når en kan sleppe en ny ekse på båndet. Ny ekse slippes med DropBlue f.eks. på input3 båndet. 

\vskip 5pt 
Eskene skal transporteres videre fra Input båndene til Elevator båndene. Elevator 1 henter fra Input1- og 2 og Elevator2 henter fra Input3- og 4. Når Elevator båndene skal hente fra Input1- og 3 må det heves opp. Dette gjøres med kommandoen ElevatorXUp, og en får den ned med ElevatorXDown. 

\vskip 2.5pt 
Elevator transportbåndene startes med ElevatorXConv og ElevatorXEndSensor registrerer når eksene er kommet til enden på båndet. 

\vskip 5pt 
Sensorene ElevatorXUpSensor og ElevatorXDownSensor registrerer opp og nede posisjon på elevator båndene. 

\vskip 5pt 

Slider transportbåndet Kan veksle mellom Elevator1  og Elevator2. Dette gjøres med kommandoene SliderLeft(Elevator2) og SliderRight(Elevator1). Sensorene SliderLeftSensor og SliderRightSensor registrerer de respektive posisjonene. Når Slider transportbåndet skal levere til output må det stå i posisjonen som SliderMidSensor registrerer.  

\vskip 2.5pt 
Slider transportbåndet startes med SliderConv og SliderConvEndSensor registrerer når eksene er kommet til enden på båndet. 
\vskip 5pt 
Output transportbåndet startes med OutputConv og endeposisjon registreres med OutputConvEndSensor. 

\vskip 2.5pt 
Styrepanelet består av følgende:
\begin{itemize}[noitemsep]
	\item Knapp for å resette sekvensen SFCReset
	\item Knapper for å velge hvilket transportbånd en skal hente fra. 
	\item Et klar lys som skal vise at en kan velge å hente esker. 
\end{itemize}

$$\includegraphics[width=5cm]{./aProgrammering01x02.png}$$
\vskip 2.5pt 
  
\vskip 2.5pt 


Oppgave:
Din oppgave er å gjøre det mulig å hente esker ved output båndet. Du kan løse dette på ulike måter. 

Vurdering av oppgaven:
\begin{itemize}[noitemsep]
	\item Kan sette opp kommunikasjon med simumatik (2)
	\item Kan forflytte en eske (2)
	\item Kan forflytte en eske over og vise et eksempel med manuell styring av et av båndene. (3)
	\item Kan fylle opp transportbånd med esker og forflytte eskene fra ulike input over med valg mellom inputene (4)
	\item Kan forflytte fra alle input over med valg mellom inputene (5)
	\item Kan klargjøre den elevatoren som ikke skal transportere esker over med nye esker.(6)
\end{itemize}


%(END_NOTES)

\vfil\eject
$$\includegraphics[angle=90,height=25cm]{./aProgrammering01x01.jpg}$$
\end{document}
