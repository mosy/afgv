% Start preamble
\documentclass[12pt,a4paper]{article}
\usepackage{geometry}
 \geometry{
 a4paper,
 total={170mm,257mm},
 left=20mm,
 top=20mm,
 }
\usepackage[utf8]{inputenc}
\usepackage[T1]{fontenc}
\usepackage[pdftex]{graphicx}
\graphicspath{{./}}
\usepackage{enumitem}
\usepackage{pdfpages}
\usepackage{hyperref}
\usepackage{tikz}
\usepackage{attachfile}
\usepackage{epstopdf}
\usepackage{array}
%\usepackage[table]{xcolor,colorbl}
\setlength{\textwidth}{16cm}
\setlength{\oddsidemargin}{-0.5cm}
\setlength{\evensidemargin}{-0.5cm}
%\setlenght{\headsep}{0cm}
\setlength\parindent{0pt}
%\setlength{\extrarowheight}{3pt}
\usepackage{listings}
%\usepackage{xcolor}

 %%%%%%%%%%%%%%%%%%%%%%%%%%%%%%%%%%%%%%%%%%%%%%%%%%%%%%%%%%%%%%%%%%%%%%%%%%%%%%%% 
%%% ~ Arduino Language - Arduino IDE Colors ~                                  %%%
%%%                                                                            %%%
%%% Kyle Rocha-Brownell | 10/2/2017 | No Licence                               %%%
%%% -------------------------------------------------------------------------- %%%
%%%                                                                            %%%
%%% Place this file in your working directory (next to the latex file you're   %%%
%%% working on).  To add it to your project, place:                            %%%
%%%     %%%%%%%%%%%%%%%%%%%%%%%%%%%%%%%%%%%%%%%%%%%%%%%%%%%%%%%%%%%%%%%%%%%%%%%%%%%%%%%% 
%%% ~ Arduino Language - Arduino IDE Colors ~                                  %%%
%%%                                                                            %%%
%%% Kyle Rocha-Brownell | 10/2/2017 | No Licence                               %%%
%%% -------------------------------------------------------------------------- %%%
%%%                                                                            %%%
%%% Place this file in your working directory (next to the latex file you're   %%%
%%% working on).  To add it to your project, place:                            %%%
%%%     %%%%%%%%%%%%%%%%%%%%%%%%%%%%%%%%%%%%%%%%%%%%%%%%%%%%%%%%%%%%%%%%%%%%%%%%%%%%%%%% 
%%% ~ Arduino Language - Arduino IDE Colors ~                                  %%%
%%%                                                                            %%%
%%% Kyle Rocha-Brownell | 10/2/2017 | No Licence                               %%%
%%% -------------------------------------------------------------------------- %%%
%%%                                                                            %%%
%%% Place this file in your working directory (next to the latex file you're   %%%
%%% working on).  To add it to your project, place:                            %%%
%%%    \input{arduinoLanguage.tex}                                             %%%
%%% somewhere before \begin{document} in your latex file.                      %%%
%%%                                                                            %%%
%%% In your document, place your arduino code between:                         %%%
%%%   \begin{lstlisting}[language=Arduino]                                     %%%
%%% and:                                                                       %%%
%%%   \end{lstlisting}                                                         %%%
%%%                                                                            %%%
%%% Or create your own style to add non-built-in functions and variables.      %%%
%%%                                                                            %%%
 %%%%%%%%%%%%%%%%%%%%%%%%%%%%%%%%%%%%%%%%%%%%%%%%%%%%%%%%%%%%%%%%%%%%%%%%%%%%%%%% 

\usepackage{color}
\usepackage{listings}    
\usepackage{courier}

%%% Define Custom IDE Colors %%%
\definecolor{arduinoGreen}    {rgb} {0.17, 0.43, 0.01}
\definecolor{arduinoGrey}     {rgb} {0.47, 0.47, 0.33}
\definecolor{arduinoOrange}   {rgb} {0.8 , 0.4 , 0   }
\definecolor{arduinoBlue}     {rgb} {0.01, 0.61, 0.98}
\definecolor{arduinoDarkBlue} {rgb} {0.0 , 0.2 , 0.5 }

%%% Define Arduino Language %%%
\lstdefinelanguage{Arduino}{
  language=C++, % begin with default C++ settings 
%
%
  %%% Keyword Color Group 1 %%%  (called KEYWORD3 by arduino)
  keywordstyle=\color{arduinoGreen},   
  deletekeywords={  % remove all arduino keywords that might be in c++
                break, case, override, final, continue, default, do, else, for, 
                if, return, goto, switch, throw, try, while, setup, loop, export, 
                not, or, and, xor, include, define, elif, else, error, if, ifdef, 
                ifndef, pragma, warning,
                HIGH, LOW, INPUT, INPUT_PULLUP, OUTPUT, DEC, BIN, HEX, OCT, PI, 
                HALF_PI, TWO_PI, LSBFIRST, MSBFIRST, CHANGE, FALLING, RISING, 
                DEFAULT, EXTERNAL, INTERNAL, INTERNAL1V1, INTERNAL2V56, LED_BUILTIN, 
                LED_BUILTIN_RX, LED_BUILTIN_TX, DIGITAL_MESSAGE, FIRMATA_STRING, 
                ANALOG_MESSAGE, REPORT_DIGITAL, REPORT_ANALOG, SET_PIN_MODE, 
                SYSTEM_RESET, SYSEX_START, auto, int8_t, int16_t, int32_t, int64_t, 
                uint8_t, uint16_t, uint32_t, uint64_t, char16_t, char32_t, operator, 
                enum, delete, bool, boolean, byte, char, const, false, float, double, 
                null, NULL, int, long, new, private, protected, public, short, 
                signed, static, volatile, String, void, true, unsigned, word, array, 
                sizeof, dynamic_cast, typedef, const_cast, struct, static_cast, union, 
                friend, extern, class, reinterpret_cast, register, explicit, inline, 
                _Bool, complex, _Complex, _Imaginary, atomic_bool, atomic_char, 
                atomic_schar, atomic_uchar, atomic_short, atomic_ushort, atomic_int, 
                atomic_uint, atomic_long, atomic_ulong, atomic_llong, atomic_ullong, 
                virtual, PROGMEM,
                Serial, Serial1, Serial2, Serial3, SerialUSB, Keyboard, Mouse,
                abs, acos, asin, atan, atan2, ceil, constrain, cos, degrees, exp, 
                floor, log, map, max, min, radians, random, randomSeed, round, sin, 
                sq, sqrt, tan, pow, bitRead, bitWrite, bitSet, bitClear, bit, 
                highByte, lowByte, analogReference, analogRead, 
                analogReadResolution, analogWrite, analogWriteResolution, 
                attachInterrupt, detachInterrupt, digitalPinToInterrupt, delay, 
                delayMicroseconds, digitalWrite, digitalRead, interrupts, millis, 
                micros, noInterrupts, noTone, pinMode, pulseIn, pulseInLong, shiftIn, 
                shiftOut, tone, yield, Stream, begin, end, peek, read, print, 
                println, available, availableForWrite, flush, setTimeout, find, 
                findUntil, parseInt, parseFloat, readBytes, readBytesUntil, readString, 
                readStringUntil, trim, toUpperCase, toLowerCase, charAt, compareTo, 
                concat, endsWith, startsWith, equals, equalsIgnoreCase, getBytes, 
                indexOf, lastIndexOf, length, replace, setCharAt, substring, 
                toCharArray, toInt, press, release, releaseAll, accept, click, move, 
                isPressed, isAlphaNumeric, isAlpha, isAscii, isWhitespace, isControl, 
                isDigit, isGraph, isLowerCase, isPrintable, isPunct, isSpace, 
                isUpperCase, isHexadecimalDigit, 
                }, 
  morekeywords={   % add arduino structures to group 1
                break, case, override, final, continue, default, do, else, for, 
                if, return, goto, switch, throw, try, while, setup, loop, export, 
                not, or, and, xor, include, define, elif, else, error, if, ifdef, 
                ifndef, pragma, warning,
                }, 
% 
%
  %%% Keyword Color Group 2 %%%  (called LITERAL1 by arduino)
  keywordstyle=[2]\color{arduinoBlue},   
  keywords=[2]{   % add variables and dataTypes as 2nd group  
                HIGH, LOW, INPUT, INPUT_PULLUP, OUTPUT, DEC, BIN, HEX, OCT, PI, 
                HALF_PI, TWO_PI, LSBFIRST, MSBFIRST, CHANGE, FALLING, RISING, 
                DEFAULT, EXTERNAL, INTERNAL, INTERNAL1V1, INTERNAL2V56, LED_BUILTIN, 
                LED_BUILTIN_RX, LED_BUILTIN_TX, DIGITAL_MESSAGE, FIRMATA_STRING, 
                ANALOG_MESSAGE, REPORT_DIGITAL, REPORT_ANALOG, SET_PIN_MODE, 
                SYSTEM_RESET, SYSEX_START, auto, int8_t, int16_t, int32_t, int64_t, 
                uint8_t, uint16_t, uint32_t, uint64_t, char16_t, char32_t, operator, 
                enum, delete, bool, boolean, byte, char, const, false, float, double, 
                null, NULL, int, long, new, private, protected, public, short, 
                signed, static, volatile, String, void, true, unsigned, word, array, 
                sizeof, dynamic_cast, typedef, const_cast, struct, static_cast, union, 
                friend, extern, class, reinterpret_cast, register, explicit, inline, 
                _Bool, complex, _Complex, _Imaginary, atomic_bool, atomic_char, 
                atomic_schar, atomic_uchar, atomic_short, atomic_ushort, atomic_int, 
                atomic_uint, atomic_long, atomic_ulong, atomic_llong, atomic_ullong, 
                virtual, PROGMEM,
                },  
% 
%
  %%% Keyword Color Group 3 %%%  (called KEYWORD1 by arduino)
  keywordstyle=[3]\bfseries\color{arduinoOrange},
  keywords=[3]{  % add built-in functions as a 3rd group
                Serial, Serial1, Serial2, Serial3, SerialUSB, Keyboard, Mouse,
                },      
%
%
  %%% Keyword Color Group 4 %%%  (called KEYWORD2 by arduino)
  keywordstyle=[4]\color{arduinoOrange},
  keywords=[4]{  % add more built-in functions as a 4th group
                abs, acos, asin, atan, atan2, ceil, constrain, cos, degrees, exp, 
                floor, log, map, max, min, radians, random, randomSeed, round, sin, 
                sq, sqrt, tan, pow, bitRead, bitWrite, bitSet, bitClear, bit, 
                highByte, lowByte, analogReference, analogRead, 
                analogReadResolution, analogWrite, analogWriteResolution, 
                attachInterrupt, detachInterrupt, digitalPinToInterrupt, delay, 
                delayMicroseconds, digitalWrite, digitalRead, interrupts, millis, 
                micros, noInterrupts, noTone, pinMode, pulseIn, pulseInLong, shiftIn, 
                shiftOut, tone, yield, Stream, begin, end, peek, read, print, 
                println, available, availableForWrite, flush, setTimeout, find, 
                findUntil, parseInt, parseFloat, readBytes, readBytesUntil, readString, 
                readStringUntil, trim, toUpperCase, toLowerCase, charAt, compareTo, 
                concat, endsWith, startsWith, equals, equalsIgnoreCase, getBytes, 
                indexOf, lastIndexOf, length, replace, setCharAt, substring, 
                toCharArray, toInt, press, release, releaseAll, accept, click, move, 
                isPressed, isAlphaNumeric, isAlpha, isAscii, isWhitespace, isControl, 
                isDigit, isGraph, isLowerCase, isPrintable, isPunct, isSpace, 
                isUpperCase, isHexadecimalDigit, 
                },      
%
%
  %%% Set Other Colors %%%
  stringstyle=\color{arduinoDarkBlue},    
  commentstyle=\color{arduinoGrey},    
%          
%   
  %%%% Line Numbering %%%%
%  numbers=left,                    
%  numbersep=5pt,                   
%  numberstyle=\color{arduinoGrey},    
  %stepnumber=2,                      % show every 2 line numbers
%
%
  %%%% Code Box Style %%%%
  breaklines=true,                    % wordwrapping
  tabsize=8,         
  basicstyle=\ttfamily  
}
                                             %%%
%%% somewhere before \begin{document} in your latex file.                      %%%
%%%                                                                            %%%
%%% In your document, place your arduino code between:                         %%%
%%%   \begin{lstlisting}[language=Arduino]                                     %%%
%%% and:                                                                       %%%
%%%   \end{lstlisting}                                                         %%%
%%%                                                                            %%%
%%% Or create your own style to add non-built-in functions and variables.      %%%
%%%                                                                            %%%
 %%%%%%%%%%%%%%%%%%%%%%%%%%%%%%%%%%%%%%%%%%%%%%%%%%%%%%%%%%%%%%%%%%%%%%%%%%%%%%%% 

\usepackage{color}
\usepackage{listings}    
\usepackage{courier}

%%% Define Custom IDE Colors %%%
\definecolor{arduinoGreen}    {rgb} {0.17, 0.43, 0.01}
\definecolor{arduinoGrey}     {rgb} {0.47, 0.47, 0.33}
\definecolor{arduinoOrange}   {rgb} {0.8 , 0.4 , 0   }
\definecolor{arduinoBlue}     {rgb} {0.01, 0.61, 0.98}
\definecolor{arduinoDarkBlue} {rgb} {0.0 , 0.2 , 0.5 }

%%% Define Arduino Language %%%
\lstdefinelanguage{Arduino}{
  language=C++, % begin with default C++ settings 
%
%
  %%% Keyword Color Group 1 %%%  (called KEYWORD3 by arduino)
  keywordstyle=\color{arduinoGreen},   
  deletekeywords={  % remove all arduino keywords that might be in c++
                break, case, override, final, continue, default, do, else, for, 
                if, return, goto, switch, throw, try, while, setup, loop, export, 
                not, or, and, xor, include, define, elif, else, error, if, ifdef, 
                ifndef, pragma, warning,
                HIGH, LOW, INPUT, INPUT_PULLUP, OUTPUT, DEC, BIN, HEX, OCT, PI, 
                HALF_PI, TWO_PI, LSBFIRST, MSBFIRST, CHANGE, FALLING, RISING, 
                DEFAULT, EXTERNAL, INTERNAL, INTERNAL1V1, INTERNAL2V56, LED_BUILTIN, 
                LED_BUILTIN_RX, LED_BUILTIN_TX, DIGITAL_MESSAGE, FIRMATA_STRING, 
                ANALOG_MESSAGE, REPORT_DIGITAL, REPORT_ANALOG, SET_PIN_MODE, 
                SYSTEM_RESET, SYSEX_START, auto, int8_t, int16_t, int32_t, int64_t, 
                uint8_t, uint16_t, uint32_t, uint64_t, char16_t, char32_t, operator, 
                enum, delete, bool, boolean, byte, char, const, false, float, double, 
                null, NULL, int, long, new, private, protected, public, short, 
                signed, static, volatile, String, void, true, unsigned, word, array, 
                sizeof, dynamic_cast, typedef, const_cast, struct, static_cast, union, 
                friend, extern, class, reinterpret_cast, register, explicit, inline, 
                _Bool, complex, _Complex, _Imaginary, atomic_bool, atomic_char, 
                atomic_schar, atomic_uchar, atomic_short, atomic_ushort, atomic_int, 
                atomic_uint, atomic_long, atomic_ulong, atomic_llong, atomic_ullong, 
                virtual, PROGMEM,
                Serial, Serial1, Serial2, Serial3, SerialUSB, Keyboard, Mouse,
                abs, acos, asin, atan, atan2, ceil, constrain, cos, degrees, exp, 
                floor, log, map, max, min, radians, random, randomSeed, round, sin, 
                sq, sqrt, tan, pow, bitRead, bitWrite, bitSet, bitClear, bit, 
                highByte, lowByte, analogReference, analogRead, 
                analogReadResolution, analogWrite, analogWriteResolution, 
                attachInterrupt, detachInterrupt, digitalPinToInterrupt, delay, 
                delayMicroseconds, digitalWrite, digitalRead, interrupts, millis, 
                micros, noInterrupts, noTone, pinMode, pulseIn, pulseInLong, shiftIn, 
                shiftOut, tone, yield, Stream, begin, end, peek, read, print, 
                println, available, availableForWrite, flush, setTimeout, find, 
                findUntil, parseInt, parseFloat, readBytes, readBytesUntil, readString, 
                readStringUntil, trim, toUpperCase, toLowerCase, charAt, compareTo, 
                concat, endsWith, startsWith, equals, equalsIgnoreCase, getBytes, 
                indexOf, lastIndexOf, length, replace, setCharAt, substring, 
                toCharArray, toInt, press, release, releaseAll, accept, click, move, 
                isPressed, isAlphaNumeric, isAlpha, isAscii, isWhitespace, isControl, 
                isDigit, isGraph, isLowerCase, isPrintable, isPunct, isSpace, 
                isUpperCase, isHexadecimalDigit, 
                }, 
  morekeywords={   % add arduino structures to group 1
                break, case, override, final, continue, default, do, else, for, 
                if, return, goto, switch, throw, try, while, setup, loop, export, 
                not, or, and, xor, include, define, elif, else, error, if, ifdef, 
                ifndef, pragma, warning,
                }, 
% 
%
  %%% Keyword Color Group 2 %%%  (called LITERAL1 by arduino)
  keywordstyle=[2]\color{arduinoBlue},   
  keywords=[2]{   % add variables and dataTypes as 2nd group  
                HIGH, LOW, INPUT, INPUT_PULLUP, OUTPUT, DEC, BIN, HEX, OCT, PI, 
                HALF_PI, TWO_PI, LSBFIRST, MSBFIRST, CHANGE, FALLING, RISING, 
                DEFAULT, EXTERNAL, INTERNAL, INTERNAL1V1, INTERNAL2V56, LED_BUILTIN, 
                LED_BUILTIN_RX, LED_BUILTIN_TX, DIGITAL_MESSAGE, FIRMATA_STRING, 
                ANALOG_MESSAGE, REPORT_DIGITAL, REPORT_ANALOG, SET_PIN_MODE, 
                SYSTEM_RESET, SYSEX_START, auto, int8_t, int16_t, int32_t, int64_t, 
                uint8_t, uint16_t, uint32_t, uint64_t, char16_t, char32_t, operator, 
                enum, delete, bool, boolean, byte, char, const, false, float, double, 
                null, NULL, int, long, new, private, protected, public, short, 
                signed, static, volatile, String, void, true, unsigned, word, array, 
                sizeof, dynamic_cast, typedef, const_cast, struct, static_cast, union, 
                friend, extern, class, reinterpret_cast, register, explicit, inline, 
                _Bool, complex, _Complex, _Imaginary, atomic_bool, atomic_char, 
                atomic_schar, atomic_uchar, atomic_short, atomic_ushort, atomic_int, 
                atomic_uint, atomic_long, atomic_ulong, atomic_llong, atomic_ullong, 
                virtual, PROGMEM,
                },  
% 
%
  %%% Keyword Color Group 3 %%%  (called KEYWORD1 by arduino)
  keywordstyle=[3]\bfseries\color{arduinoOrange},
  keywords=[3]{  % add built-in functions as a 3rd group
                Serial, Serial1, Serial2, Serial3, SerialUSB, Keyboard, Mouse,
                },      
%
%
  %%% Keyword Color Group 4 %%%  (called KEYWORD2 by arduino)
  keywordstyle=[4]\color{arduinoOrange},
  keywords=[4]{  % add more built-in functions as a 4th group
                abs, acos, asin, atan, atan2, ceil, constrain, cos, degrees, exp, 
                floor, log, map, max, min, radians, random, randomSeed, round, sin, 
                sq, sqrt, tan, pow, bitRead, bitWrite, bitSet, bitClear, bit, 
                highByte, lowByte, analogReference, analogRead, 
                analogReadResolution, analogWrite, analogWriteResolution, 
                attachInterrupt, detachInterrupt, digitalPinToInterrupt, delay, 
                delayMicroseconds, digitalWrite, digitalRead, interrupts, millis, 
                micros, noInterrupts, noTone, pinMode, pulseIn, pulseInLong, shiftIn, 
                shiftOut, tone, yield, Stream, begin, end, peek, read, print, 
                println, available, availableForWrite, flush, setTimeout, find, 
                findUntil, parseInt, parseFloat, readBytes, readBytesUntil, readString, 
                readStringUntil, trim, toUpperCase, toLowerCase, charAt, compareTo, 
                concat, endsWith, startsWith, equals, equalsIgnoreCase, getBytes, 
                indexOf, lastIndexOf, length, replace, setCharAt, substring, 
                toCharArray, toInt, press, release, releaseAll, accept, click, move, 
                isPressed, isAlphaNumeric, isAlpha, isAscii, isWhitespace, isControl, 
                isDigit, isGraph, isLowerCase, isPrintable, isPunct, isSpace, 
                isUpperCase, isHexadecimalDigit, 
                },      
%
%
  %%% Set Other Colors %%%
  stringstyle=\color{arduinoDarkBlue},    
  commentstyle=\color{arduinoGrey},    
%          
%   
  %%%% Line Numbering %%%%
%  numbers=left,                    
%  numbersep=5pt,                   
%  numberstyle=\color{arduinoGrey},    
  %stepnumber=2,                      % show every 2 line numbers
%
%
  %%%% Code Box Style %%%%
  breaklines=true,                    % wordwrapping
  tabsize=8,         
  basicstyle=\ttfamily  
}
                                             %%%
%%% somewhere before \begin{document} in your latex file.                      %%%
%%%                                                                            %%%
%%% In your document, place your arduino code between:                         %%%
%%%   \begin{lstlisting}[language=Arduino]                                     %%%
%%% and:                                                                       %%%
%%%   \end{lstlisting}                                                         %%%
%%%                                                                            %%%
%%% Or create your own style to add non-built-in functions and variables.      %%%
%%%                                                                            %%%
 %%%%%%%%%%%%%%%%%%%%%%%%%%%%%%%%%%%%%%%%%%%%%%%%%%%%%%%%%%%%%%%%%%%%%%%%%%%%%%%% 

\usepackage{color}
\usepackage{listings}    
\usepackage{courier}

%%% Define Custom IDE Colors %%%
\definecolor{arduinoGreen}    {rgb} {0.17, 0.43, 0.01}
\definecolor{arduinoGrey}     {rgb} {0.47, 0.47, 0.33}
\definecolor{arduinoOrange}   {rgb} {0.8 , 0.4 , 0   }
\definecolor{arduinoBlue}     {rgb} {0.01, 0.61, 0.98}
\definecolor{arduinoDarkBlue} {rgb} {0.0 , 0.2 , 0.5 }

%%% Define Arduino Language %%%
\lstdefinelanguage{Arduino}{
  language=C++, % begin with default C++ settings 
%
%
  %%% Keyword Color Group 1 %%%  (called KEYWORD3 by arduino)
  keywordstyle=\color{arduinoGreen},   
  deletekeywords={  % remove all arduino keywords that might be in c++
                break, case, override, final, continue, default, do, else, for, 
                if, return, goto, switch, throw, try, while, setup, loop, export, 
                not, or, and, xor, include, define, elif, else, error, if, ifdef, 
                ifndef, pragma, warning,
                HIGH, LOW, INPUT, INPUT_PULLUP, OUTPUT, DEC, BIN, HEX, OCT, PI, 
                HALF_PI, TWO_PI, LSBFIRST, MSBFIRST, CHANGE, FALLING, RISING, 
                DEFAULT, EXTERNAL, INTERNAL, INTERNAL1V1, INTERNAL2V56, LED_BUILTIN, 
                LED_BUILTIN_RX, LED_BUILTIN_TX, DIGITAL_MESSAGE, FIRMATA_STRING, 
                ANALOG_MESSAGE, REPORT_DIGITAL, REPORT_ANALOG, SET_PIN_MODE, 
                SYSTEM_RESET, SYSEX_START, auto, int8_t, int16_t, int32_t, int64_t, 
                uint8_t, uint16_t, uint32_t, uint64_t, char16_t, char32_t, operator, 
                enum, delete, bool, boolean, byte, char, const, false, float, double, 
                null, NULL, int, long, new, private, protected, public, short, 
                signed, static, volatile, String, void, true, unsigned, word, array, 
                sizeof, dynamic_cast, typedef, const_cast, struct, static_cast, union, 
                friend, extern, class, reinterpret_cast, register, explicit, inline, 
                _Bool, complex, _Complex, _Imaginary, atomic_bool, atomic_char, 
                atomic_schar, atomic_uchar, atomic_short, atomic_ushort, atomic_int, 
                atomic_uint, atomic_long, atomic_ulong, atomic_llong, atomic_ullong, 
                virtual, PROGMEM,
                Serial, Serial1, Serial2, Serial3, SerialUSB, Keyboard, Mouse,
                abs, acos, asin, atan, atan2, ceil, constrain, cos, degrees, exp, 
                floor, log, map, max, min, radians, random, randomSeed, round, sin, 
                sq, sqrt, tan, pow, bitRead, bitWrite, bitSet, bitClear, bit, 
                highByte, lowByte, analogReference, analogRead, 
                analogReadResolution, analogWrite, analogWriteResolution, 
                attachInterrupt, detachInterrupt, digitalPinToInterrupt, delay, 
                delayMicroseconds, digitalWrite, digitalRead, interrupts, millis, 
                micros, noInterrupts, noTone, pinMode, pulseIn, pulseInLong, shiftIn, 
                shiftOut, tone, yield, Stream, begin, end, peek, read, print, 
                println, available, availableForWrite, flush, setTimeout, find, 
                findUntil, parseInt, parseFloat, readBytes, readBytesUntil, readString, 
                readStringUntil, trim, toUpperCase, toLowerCase, charAt, compareTo, 
                concat, endsWith, startsWith, equals, equalsIgnoreCase, getBytes, 
                indexOf, lastIndexOf, length, replace, setCharAt, substring, 
                toCharArray, toInt, press, release, releaseAll, accept, click, move, 
                isPressed, isAlphaNumeric, isAlpha, isAscii, isWhitespace, isControl, 
                isDigit, isGraph, isLowerCase, isPrintable, isPunct, isSpace, 
                isUpperCase, isHexadecimalDigit, 
                }, 
  morekeywords={   % add arduino structures to group 1
                break, case, override, final, continue, default, do, else, for, 
                if, return, goto, switch, throw, try, while, setup, loop, export, 
                not, or, and, xor, include, define, elif, else, error, if, ifdef, 
                ifndef, pragma, warning,
                }, 
% 
%
  %%% Keyword Color Group 2 %%%  (called LITERAL1 by arduino)
  keywordstyle=[2]\color{arduinoBlue},   
  keywords=[2]{   % add variables and dataTypes as 2nd group  
                HIGH, LOW, INPUT, INPUT_PULLUP, OUTPUT, DEC, BIN, HEX, OCT, PI, 
                HALF_PI, TWO_PI, LSBFIRST, MSBFIRST, CHANGE, FALLING, RISING, 
                DEFAULT, EXTERNAL, INTERNAL, INTERNAL1V1, INTERNAL2V56, LED_BUILTIN, 
                LED_BUILTIN_RX, LED_BUILTIN_TX, DIGITAL_MESSAGE, FIRMATA_STRING, 
                ANALOG_MESSAGE, REPORT_DIGITAL, REPORT_ANALOG, SET_PIN_MODE, 
                SYSTEM_RESET, SYSEX_START, auto, int8_t, int16_t, int32_t, int64_t, 
                uint8_t, uint16_t, uint32_t, uint64_t, char16_t, char32_t, operator, 
                enum, delete, bool, boolean, byte, char, const, false, float, double, 
                null, NULL, int, long, new, private, protected, public, short, 
                signed, static, volatile, String, void, true, unsigned, word, array, 
                sizeof, dynamic_cast, typedef, const_cast, struct, static_cast, union, 
                friend, extern, class, reinterpret_cast, register, explicit, inline, 
                _Bool, complex, _Complex, _Imaginary, atomic_bool, atomic_char, 
                atomic_schar, atomic_uchar, atomic_short, atomic_ushort, atomic_int, 
                atomic_uint, atomic_long, atomic_ulong, atomic_llong, atomic_ullong, 
                virtual, PROGMEM,
                },  
% 
%
  %%% Keyword Color Group 3 %%%  (called KEYWORD1 by arduino)
  keywordstyle=[3]\bfseries\color{arduinoOrange},
  keywords=[3]{  % add built-in functions as a 3rd group
                Serial, Serial1, Serial2, Serial3, SerialUSB, Keyboard, Mouse,
                },      
%
%
  %%% Keyword Color Group 4 %%%  (called KEYWORD2 by arduino)
  keywordstyle=[4]\color{arduinoOrange},
  keywords=[4]{  % add more built-in functions as a 4th group
                abs, acos, asin, atan, atan2, ceil, constrain, cos, degrees, exp, 
                floor, log, map, max, min, radians, random, randomSeed, round, sin, 
                sq, sqrt, tan, pow, bitRead, bitWrite, bitSet, bitClear, bit, 
                highByte, lowByte, analogReference, analogRead, 
                analogReadResolution, analogWrite, analogWriteResolution, 
                attachInterrupt, detachInterrupt, digitalPinToInterrupt, delay, 
                delayMicroseconds, digitalWrite, digitalRead, interrupts, millis, 
                micros, noInterrupts, noTone, pinMode, pulseIn, pulseInLong, shiftIn, 
                shiftOut, tone, yield, Stream, begin, end, peek, read, print, 
                println, available, availableForWrite, flush, setTimeout, find, 
                findUntil, parseInt, parseFloat, readBytes, readBytesUntil, readString, 
                readStringUntil, trim, toUpperCase, toLowerCase, charAt, compareTo, 
                concat, endsWith, startsWith, equals, equalsIgnoreCase, getBytes, 
                indexOf, lastIndexOf, length, replace, setCharAt, substring, 
                toCharArray, toInt, press, release, releaseAll, accept, click, move, 
                isPressed, isAlphaNumeric, isAlpha, isAscii, isWhitespace, isControl, 
                isDigit, isGraph, isLowerCase, isPrintable, isPunct, isSpace, 
                isUpperCase, isHexadecimalDigit, 
                },      
%
%
  %%% Set Other Colors %%%
  stringstyle=\color{arduinoDarkBlue},    
  commentstyle=\color{arduinoGrey},    
%          
%   
  %%%% Line Numbering %%%%
%  numbers=left,                    
%  numbersep=5pt,                   
%  numberstyle=\color{arduinoGrey},    
  %stepnumber=2,                      % show every 2 line numbers
%
%
  %%%% Code Box Style %%%%
  breaklines=true,                    % wordwrapping
  tabsize=8,         
  basicstyle=\ttfamily  
}

%%%%%% Counting oppgaves %%%%%%
 \newcount\questnum \questnum=0
 \def\oppgave{
            \advance\questnum by 1
            \ifnum \questnum > 0
                 \hrule
                 \vskip 3pt
                 \leftline{Oppgave \the\questnum}
                 \vskip 3pt \fi}
 %%%%%%%%%%%%%%%%%%%%%%
%%%%%%%%%%%%%%%%%%%%%%


%%%%%% Counting answers %%%%%%
\newcount\answnum \answnum=0
\def\svar{
           \advance\answnum by 1
           \ifnum \answnum > 0
                \hrule
                \vskip 3pt
                \leftline{Svar \the\answnum}
                \vskip 3pt \fi}
%%%%%%%%%%%%%%%%%%%%%%


%%%%%% Counting notes %%%%%%
\newcount\explnum \explnum=0
\def\notes{
           \advance\explnum by 1
           \ifnum \explnum > 0
                \hrule
                \vskip 3pt
                \leftline{Notes \the\explnum}
                \vskip 3pt \fi}
%%%%%%%%%%%%%%%%%%%%%%

% End preamble

\begin{document}
\begin{titlepage}
   \begin{center}
       \vspace*{1cm}

       \textbf{Infoskriv 3AUA Gand VGS}

       \vspace{0.5cm}
        Skoleåret 2021/2022
            
       \vspace{1.5cm}

       \textbf{Kontaktlærer: Fred-Olav Mosdal}

       \vfill
            
            
     
       \includegraphics[width=0.5\textwidth]{/home/fred-olav/Downloads/GandLogo.png}
    \vfill        
            
   \end{center}
\end{titlepage}

% Copyright 2015, Tony R. Kuphaldt, released under the Creative Commons Attribution License (v 1.0)
% This means you may do almost anything with this work of mine, so long as you give me proper credit

%(BEGIN_FRONTMATTER)
\centerline{\bf Hvordan . . .} \bigskip 

\noindent
{\bf Finne fagplan for VG3 Automatiseringsfaget:} \url{https://www.udir.no/kl06/AUT3-03}
\vskip 10pt

\noindent
{\bf Finner jeg skoleruten:} \url{https://www.gand.vgs.no/hovedmeny/for-elever/skolehverdag/skoleferier-og-fridager-skoleruta/}
\vskip 10pt

\noindent
{\bf Finner jeg informasjon om Skolen:} \url{https://www.gand.vgs.no/hovedmeny/for-elever/}
\vskip 10pt

\noindent
{\bf Finner jeg timeplanen:} Timeplanen din finner du i VIS. 
\vskip 10pt

\noindent
{\bf Få penger til utstyr som kreves:} Alle elever med ungdomsrett i videregående opplæring får utstyrsstipend. Utstyrsstipendet er ikke avhengig av hvor mye foreldrene tjener.
Søknaden skrives til lånekassen \url{http://www.lanekassen.no}
\vskip 10pt

\noindent
{\bf Få tak i oppgaver og lærebok} gå inn i kanalen Automatiseringssystemer i teamet Gand 3AUA på teams, under filer vil du finne oppgavene i mappen "Oppgaver" og læreboken vil du finne i mappen "Teori". Denne mappen anbefaler jeg at du synkroniserer mot PC-en din. 
\vskip 10pt

\noindent
{\bf Få tak i  software og biblioteker:} Software som vi bruker lagger jeg ut på Teams. Dere er med i et team som heter Gand 3AUA. I kanalen Automatiseringsystemer legger alle biblioteker vi skal bruke. I kanalen "Software vi bruker" Ligger all programware vi bruker. 
\vskip 10pt

\noindent
{\bf Hvordan lære mest mulig:} kom til skolen forberedt hver eneste dag -- dette betyr at du har gjort alle lekser gitt til/etter en leksjon. Fulgt alle tips som gis på oppgaveark og av lærer. Ikke spør andre om hjelp før du har gjort en rimlig innsats selv. Hjelp andre med å gjennomføre oppgave og å forstå, men ikke gjør jobben for DE. 
\vskip 10pt

\noindent
{\bf Holde orden på innleveringer og frister.} Følg med på beskjeder gitt av lærer. I arbeidslivet vil du få muntlige beskjeder som det forventes at du følger opp, slik er det her også. Er du vekke fra skolen må du orientere eg med medelever. 
\vskip 10pt


\noindent
{\bf Finne fagstoff og manualer fra produsenter gitt som leselekse:} I fagbibliotekmappen som er delt med deg på OneDrive/Teams vil du finne mappene "Fagstoff" og  "Manualer" 
\vskip 10pt



\noindent
{\bf Få seg læreplass:} Ca. i desember starter de første firmaene å legge annonser for nye lærlinger (Søk på så mange du klarer å håndtere ca. 50), når du er i utplassering må du vise deg som en attraktiv arbeidstager, sørg for å være faglig interessert/på, ha minst mulig fravær og følgmed på skolemailen din her videresender jeg forespørsler fra firmaer. 
\vskip 10pt


\vfil

\underbar{file {\tt hvordan}}
\eject
%(END_FRONTMATTER)


% Copyright 2015, Tony R. Kuphaldt, released under the Creative Commons Attribution License (v 1.0)
% This means you may do almost anything with this work of mine, so long as you give me proper credit

%(BEGIN_FRONTMATTER)
\centerline{\bf Verdier og forventninger. } \bigskip 
%\centerline{\bf General Values and Expectations} \bigskip 


\noindent
{\bf Gand Standarden:}    
\begin{itemize}
\item{} Alle møter hverandre med respekt
\item{} Alle møter godt forberedt og har med nødvendig utstyr
\item{} Alle møter presis til timene
\end{itemize}
\vskip 10pt


\noindent
{\bf Orden og adferd:} På Gand føres orden-og adferds anmerkninger. Når en lærer mener en elev har dårlig adverd deller orden settes en anmerkning. Det er ikke direkte sammenheng med antall anmerkninger og nedsatt karakter i orden eller adferd. Før det er aktuelt å med nedsatt karakter skal eleven ha en advarsel. 
\vskip 10pt
Noen eksempler på hva som forventes for å oppnå karakteren god i orden og adferd. 
\vskip 10pt
Orden:
\begin{itemize}
\vskip 10pt
\item{}Eleven skal rette seg etter lover, regler og instrukser som til enhver tid gjelder, og følge anvisninger fra skolens personale.
\item{}Eleven skal møte presis. Hvis en elev kommer mer enn 15 minutt for seint, regnes det som fravær.
\item{}Eleven skal gjøre pålagt skolearbeid til avtalt tid.
\item{}Eleven plikter å skaffe seg, og ha med til undervisning, det materiell og utstyr som opplæringa krever.
\item{}Eleven plikter å behandle skolens lokaler, bøker og utstyr på en forsvarlig måte.
\item{}Alle elever har ansvar for å holde det rent og ryddig på skolen, i nærområdet og på steder der opplæringa foregår.
\item{}Eleven plikter å bruke påbudt verneutstyr og ellers overholde gjeldende regler for helse, miljø og sikkerhet (HMS).
\item{}Eleven skal varsle lærer/kontaktlærer når han må forlate skolen i opplæringstida.

\item{} Eleven holder arbeidsplassen ren og oversiktlig. Dette inkluderer kaste boss, returnere verktøy etter bruk eller på slutten av dagen, delta i opprydding ved endt praksis. Kommer du over defekt verktøy eller utsyr, merk utstyret med en detaljert beskrivelse av problemet. 
\vskip 10pt
\end{itemize}
Atferd:
\vskip 10pt
\begin{itemize}
\item{}Elevene skal bidra til et godt arbeidsmiljø ved å opptre hensynsfullt og høflig, holde arbeidsro i timene og vise respekt for andre både i undervisningssituasjoner og i pauser.
\item{}Krenkende atferd (mobbing, diskriminering, rasistiske utsagn, seksuell trakassering, vold o.l.), det være seg fysisk, verbalt eller digitalt, aksepteres ikke.
\item{}Tobakksbruk er forbudt i videregående skolers lokaler og uteområder. Elever ved de videregående skoler skal være tobakksfrie i skoletiden. E-cigaretter er ikke tillatt brukt i skoletiden.
\item{}Kjøp, salg, besittelse og bruk av rusmidler på skolens område er forbudt. Det er heller ikke tillatt å være påvirket av rusmidler i skoletiden eller ved aktiviteter i skolens regi. Den enkelte skole har utfyllende regler for behandling av overtredelse av dette.
\item{}Det er ikke tillatt å ha med seg farlige gjenstander eller våpen på skolens område eller ved aktiviteter i skolens regi.
\item{}Fusk eller forsøk på fusk vil normalt føre til nedsatt karakter i atferd.
\end{itemize}
\vskip 10pt

%\noindent
%{\bf For å bli en god Automatikker kreves:} faglig integritet,   
%
%%{\bf Success in this career requires:} professional integrity, resourcefulness, persistence, close attention to detail, and intellectual curiosity.  Poor judgment spells disaster in this career, which is why employer background checks (including social media and criminal records) and drug testing are common.  The good news is that character and clear thinking are malleable traits: unlike intelligence, these qualities can be acquired and improved with effort.  {\it This is what you are in school to do} -- increase your ``human capital'' which is the sum of all knowledge, skills, and traits valuable in the marketplace.
%
%\vskip 10pt
%
%\noindent
%%\underbar{\bf Mastery:} You must master the fundamentals of your chosen profession.  ``Mastery'' assessments challenge you to demonstrate 100\% competence (with multiple opportunities to re-try).  Failure to complete any mastery objective(s) by the deadline date caps your grade at a C$-$.  Failure to complete by the end of the next school day results in a failing (F) grade.
%
%\vskip 10pt
%

\noindent
\underbar{\bf Bruk av tid:} Det er forventet at du prioriterer tiden en slik du må i jobb. Du er ansvarlig for at du har nok tid utenom skolen til å gjøre hjemmelekser og innleveringer. På skolen er det forventet at du bruker felles utstyr effektivt slik at alle får komme til. Forstyrrende aktiviteter, som spilling, sosiale medier, surfing på internett, er ikke akseptert i klasserommet når undervisning pågår. Handleturer må ikke forstyrre sammarbeid. 


\vskip 10pt



\noindent
\underbar{\bf Initiativ:} Personlig initiativ er blandt de viktigste egenskapene til en automatikker. Derfor vil lærere på i 3AUA kreve at du finner ut av problemer selv, istedenfor å hele tiden spør andre om hjelp. Eksempler kan være: Kom i gang med elevkonto i office365, sjekk manualer for du spør om hjelp, sjekk kalender på VIS for å holde orden på prøver og innleveringer. Om du finner ut at du ikke forstår en oppgave på grunn av at du ikke husker tidligere emner, les disse emnene om igjen. 
\vskip 10pt

\noindent
\underbar{\bf Sikkerhet:} Det forventes at du jobber sikkert i klasserommet. Dette innkluderer: at du har på deg arbeidsklær og vernesko (når risikovurderingen krever det,  skal du også ha på briller, hansker, og/eller hørselsværn), at du bruker lås og merk når du jobber på eller nær ved ledere med spenning på over 30 V, bruer gardintrapp for å komme til i høyden, bruke verktøy korrekt. Om du skal bruke et ukjent verktøy må du henvende deg til læreren for instruksjon. 
\vskip 10pt

\noindent
\underbar{\bf Gruppearbeid:} Dere vil jobbe i grupper må noen av arbeidsoppdragene i løpet av året. Som del av en gruppe må du holde de andre oppdatert på hvor du er. Elever som ødlegger for abeid i gruppen  ved å: ikke møte opp, ikke delta, manglende respekt eller annen forstyrrdende oppførsel vil fjernes fra gruppen settes ned i adferd og må utføre arbeidsoppdrag individuelt resten av terminen. 
\vskip 10pt

\noindent
\underbar{\bf Sammarbeid:} Måten vi jobber på i 3AUA gjør at det blir mye sammarbeid mellom elevene. Dette gjør det dere påvirker hverandres læring i stor grad. Det forventes at du tar denne rollen seriøst, og tilby hjelp som hjelper på læring. Det vil si å ikke løse problemer for andre elever, men hjelpe med problemløsning. Kom med tips for å løse problemer, stillsprøsmål som leder til en løsning istedenfor å gi en løsning. 
\vskip 10pt

%\noindent
%\underbar{\bf Academic Engagement:} Instrumentation is a challenging career requiring creative and critical thinking.  As industry advisors have said, ``Being an instrument technician is as close as you get to doing engineering without a four-year (Bachelor's) degree.''  The only way to prepare for the challenges of being an instrument technician is to exercise that same level of creative and critical thinking before stepping into the career, mastering {\it first principles} of science and {\it general problem-solving} strategies rather than focusing on simpler tasks such as memorization and procedures.  This also means personally involving yourself in every learning exercise, not being content to merely observe others.  Individual (unassisted) performance is the gold standard for learning: {\it unless and until \underbar{you} can consistently perform on your own, you haven't learned!}

%\vskip 10pt

%\noindent
%\underbar{\bf Grades:} Employers prize trustworthy, hard working, knowledgeable, resourceful problem-solvers.  The grade you receive in any course is but a {\it partial} measure of these traits.  What matters most are the traits themselves, which is why your instructor maintains detailed student records (including individual exam scores, attendance, tardiness, and behavioral comments) and will share these records with employers if you have signed the FERPA release form.  You are welcome to see your records at any time, and to compare calculated grades with your own records (i.e. the grade spreadsheet available to all students).  You should expect employers to scrutinize your records on attendance and character, and also challenge you with technical questions when considering you for employment.
%
%\vskip 10pt

%\noindent
%\underbar{\bf Representation:} You are an ambassador for this program.  Your actions, whether on tours, during a jobshadow or internship, or while employed, can open or shut doors of opportunity for other students.  Most of the job opportunities open to you as a BTC graduate were earned by the good work of previous graduates, and as such you owe them a debt of gratitude.  Future graduates depend on you to do the same.
%
%\vskip 10pt

\noindent
\underbar{\bf Ansvar for egne handlinger:} Dersom du mister eller ødelegger skolen sitt utstyr, forventes det at du finner, reparerer eller hjelper til med å erstatte det. 
\vskip 10pt
\vfil
\noindent
\underbar{\bf Praksis i 3AUA:} 
\begin{itemize}
	\item Uke 35-40 IO-Brett
	\item Uke 42-> Rulleringsuker og prosjekter
	\item Uke 45-46?? Motorstartere i 2ELA sitt klasserom
	\item Slutten av året. Mulig minifagprøve
\end{itemize}
\vskip 10pt
\vfil
\noindent
\underbar{\bf Utplassering:} Utplassering i 3AUA fåregår i uke 49-50 og uker 2-3. Lærer forsøker å skaffe plasser, men dere kan også finne plasser utenom de lærer skaffer. 
\vskip 10pt
\vfil


\noindent
\underbar{\bf Independent Study:} This career is marked by continuous technological development and ongoing change, which is why {\it self-directed learning} is ultimately more important to your future success than specific knowledge.  To acquire and hone this skill, all second-year Instrumentation courses follow an ``inverted'' model where lecture is replaced by independent study, and class time is devoted to addressing your questions and demonstrating your learning.  Most students require a {\it minimum} of 3 hours daily study time outside of school.  Arriving unprepared (e.g. homework incomplete) is unprofessional and counter-productive.  Question 0 of every worksheet lists practical study tips.
\vskip 10pt
\noindent
\underbar{\bf Independent Problem-Solving:} The best instrument technicians are versatile problem-solvers.  General problem-solving is arguably the most valuable skill you can possess for this career, and it can only be built through persistent effort.  This is why you must take every reasonable measure to {\it solve problems on your own} before seeking help.  It is okay to be perplexed by an assignment, but you are expected to apply problem-solving strategies given to you (see Question 0) and to precisely identify where you are confused so your instructor will be able to offer targeted help.  Asking classmates to solve problems for you is folly -- this includes having others break the problem down into simple steps.  The point is to learn how to {\it think on your own}.  When troubleshooting systems in lab you are expected to run diagnostic tests (e.g. using a multimeter instead of visually seeking circuit faults), as well as consult the equipment manual(s) before seeking help.  
\vskip 10pt
\underbar{file {\tt forventinger}}
\eject
%(END_FRONTMATTER)


% This means you may do almost anything with this work of mine, so long as you give me proper credit

%(BEGIN_FRONTMATTER)

\centerline{\bf Verktøy og utstyrsliste }

\vskip 10pt


\noindent
{\bf Verktøy}
\begin{itemize}[itemsep=1mm, parsep=0pt]
	\item Verktøykasse som passer i skap 
	\item Kjørner
	\item Rissenål
	\item Kniv
	\item Skyvelære
	\item Meterstokk
	\item Flattjern, 3, 4 og 6mm
	\item Phillips, \#1 and \#2
	\item Posidrive \#2
	\item Elektronikksett
	\item Multimeter, Fluke model 87-IV or better
	\item Avisoleringsverktøy
	\item Avbiter
	\item Hvit elektrikkertape og markeringspenn til å skrive på tapen. (til å lage midlertidige markeringer)
\medskip
\end{itemize}
\vskip 10pt

\noindent
{\bf Sikkerhetsutstyr}
\begin{itemize}[itemsep=1mm, parsep=0pt]
	\item Vernebriller
	\item Hørselsvern
	\item Vernesko
	\item Sveisehansker
	\item Monteringshansker
	\item Arbeidsklær i bomull
\medskip
\end{itemize}
\vskip 10pt

\noindent
{\bf Nødvendig utstyr til skoledagen}
\begin{itemize}[itemsep=1mm, parsep=0pt]
	
	\item Still ogdt forberedt! Ha med PC med windows 10, blyanter, visk, linjal, fargeblyanter, blyantspisser, kalkulator og skrivebøker. Ha med perm til å samle alle ark.
	\item Det er forventet at du holder orden på din PC, slik at denne fungerer i timene. (Kjør en reinstallasjon av PC-en i starten av skoleåret). 
	\item Lag et mappesystem fra dag en. (Da slipper du å tenke over hvor du skal lagre filene dine.)
\medskip
\end{itemize}
\vskip 10pt



\vfil

\underbar{file {\tt tools}}
\eject
%(END_FRONTMATTER)


\end{document}
