% Start preamble
\documentclass[12pt,a4paper]{article}
\usepackage{geometry}
 \geometry{
 a4paper,
 total={170mm,257mm},
 left=20mm,
 top=20mm,
 }
\usepackage[utf8]{inputenc}
\usepackage[table]{xcolor}
\usepackage[T1]{fontenc}
\usepackage[pdftex]{graphicx}
\graphicspath{{./}}
\usepackage{enumitem}
\usepackage{pdfpages}
\usepackage{hyperref}
\usepackage{tikz}
\usepackage{attachfile}
\usepackage{epstopdf}
\usepackage{array}
\usepackage{multirow}
\usepackage{multicol}
\usepackage{float}
\usepackage{comment}
%\usepackage[table]{xcolor,colorbl}
\setlength{\textwidth}{16cm}
\setlength{\oddsidemargin}{-0.5cm}
\setlength{\evensidemargin}{-0.5cm}
%\setlenght{\headsep}{0cm}
\setlength\parindent{0pt}
%\setlength{\extrarowheight}{3pt}
\usepackage{listings}
%\usepackage{xcolor}

 %%%%%%%%%%%%%%%%%%%%%%%%%%%%%%%%%%%%%%%%%%%%%%%%%%%%%%%%%%%%%%%%%%%%%%%%%%%%%%%% 
%%% ~ Arduino Language - Arduino IDE Colors ~                                  %%%
%%%                                                                            %%%
%%% Kyle Rocha-Brownell | 10/2/2017 | No Licence                               %%%
%%% -------------------------------------------------------------------------- %%%
%%%                                                                            %%%
%%% Place this file in your working directory (next to the latex file you're   %%%
%%% working on).  To add it to your project, place:                            %%%
%%%     %%%%%%%%%%%%%%%%%%%%%%%%%%%%%%%%%%%%%%%%%%%%%%%%%%%%%%%%%%%%%%%%%%%%%%%%%%%%%%%% 
%%% ~ Arduino Language - Arduino IDE Colors ~                                  %%%
%%%                                                                            %%%
%%% Kyle Rocha-Brownell | 10/2/2017 | No Licence                               %%%
%%% -------------------------------------------------------------------------- %%%
%%%                                                                            %%%
%%% Place this file in your working directory (next to the latex file you're   %%%
%%% working on).  To add it to your project, place:                            %%%
%%%     %%%%%%%%%%%%%%%%%%%%%%%%%%%%%%%%%%%%%%%%%%%%%%%%%%%%%%%%%%%%%%%%%%%%%%%%%%%%%%%% 
%%% ~ Arduino Language - Arduino IDE Colors ~                                  %%%
%%%                                                                            %%%
%%% Kyle Rocha-Brownell | 10/2/2017 | No Licence                               %%%
%%% -------------------------------------------------------------------------- %%%
%%%                                                                            %%%
%%% Place this file in your working directory (next to the latex file you're   %%%
%%% working on).  To add it to your project, place:                            %%%
%%%    \input{arduinoLanguage.tex}                                             %%%
%%% somewhere before \begin{document} in your latex file.                      %%%
%%%                                                                            %%%
%%% In your document, place your arduino code between:                         %%%
%%%   \begin{lstlisting}[language=Arduino]                                     %%%
%%% and:                                                                       %%%
%%%   \end{lstlisting}                                                         %%%
%%%                                                                            %%%
%%% Or create your own style to add non-built-in functions and variables.      %%%
%%%                                                                            %%%
 %%%%%%%%%%%%%%%%%%%%%%%%%%%%%%%%%%%%%%%%%%%%%%%%%%%%%%%%%%%%%%%%%%%%%%%%%%%%%%%% 

\usepackage{color}
\usepackage{listings}    
\usepackage{courier}

%%% Define Custom IDE Colors %%%
\definecolor{arduinoGreen}    {rgb} {0.17, 0.43, 0.01}
\definecolor{arduinoGrey}     {rgb} {0.47, 0.47, 0.33}
\definecolor{arduinoOrange}   {rgb} {0.8 , 0.4 , 0   }
\definecolor{arduinoBlue}     {rgb} {0.01, 0.61, 0.98}
\definecolor{arduinoDarkBlue} {rgb} {0.0 , 0.2 , 0.5 }

%%% Define Arduino Language %%%
\lstdefinelanguage{Arduino}{
  language=C++, % begin with default C++ settings 
%
%
  %%% Keyword Color Group 1 %%%  (called KEYWORD3 by arduino)
  keywordstyle=\color{arduinoGreen},   
  deletekeywords={  % remove all arduino keywords that might be in c++
                break, case, override, final, continue, default, do, else, for, 
                if, return, goto, switch, throw, try, while, setup, loop, export, 
                not, or, and, xor, include, define, elif, else, error, if, ifdef, 
                ifndef, pragma, warning,
                HIGH, LOW, INPUT, INPUT_PULLUP, OUTPUT, DEC, BIN, HEX, OCT, PI, 
                HALF_PI, TWO_PI, LSBFIRST, MSBFIRST, CHANGE, FALLING, RISING, 
                DEFAULT, EXTERNAL, INTERNAL, INTERNAL1V1, INTERNAL2V56, LED_BUILTIN, 
                LED_BUILTIN_RX, LED_BUILTIN_TX, DIGITAL_MESSAGE, FIRMATA_STRING, 
                ANALOG_MESSAGE, REPORT_DIGITAL, REPORT_ANALOG, SET_PIN_MODE, 
                SYSTEM_RESET, SYSEX_START, auto, int8_t, int16_t, int32_t, int64_t, 
                uint8_t, uint16_t, uint32_t, uint64_t, char16_t, char32_t, operator, 
                enum, delete, bool, boolean, byte, char, const, false, float, double, 
                null, NULL, int, long, new, private, protected, public, short, 
                signed, static, volatile, String, void, true, unsigned, word, array, 
                sizeof, dynamic_cast, typedef, const_cast, struct, static_cast, union, 
                friend, extern, class, reinterpret_cast, register, explicit, inline, 
                _Bool, complex, _Complex, _Imaginary, atomic_bool, atomic_char, 
                atomic_schar, atomic_uchar, atomic_short, atomic_ushort, atomic_int, 
                atomic_uint, atomic_long, atomic_ulong, atomic_llong, atomic_ullong, 
                virtual, PROGMEM,
                Serial, Serial1, Serial2, Serial3, SerialUSB, Keyboard, Mouse,
                abs, acos, asin, atan, atan2, ceil, constrain, cos, degrees, exp, 
                floor, log, map, max, min, radians, random, randomSeed, round, sin, 
                sq, sqrt, tan, pow, bitRead, bitWrite, bitSet, bitClear, bit, 
                highByte, lowByte, analogReference, analogRead, 
                analogReadResolution, analogWrite, analogWriteResolution, 
                attachInterrupt, detachInterrupt, digitalPinToInterrupt, delay, 
                delayMicroseconds, digitalWrite, digitalRead, interrupts, millis, 
                micros, noInterrupts, noTone, pinMode, pulseIn, pulseInLong, shiftIn, 
                shiftOut, tone, yield, Stream, begin, end, peek, read, print, 
                println, available, availableForWrite, flush, setTimeout, find, 
                findUntil, parseInt, parseFloat, readBytes, readBytesUntil, readString, 
                readStringUntil, trim, toUpperCase, toLowerCase, charAt, compareTo, 
                concat, endsWith, startsWith, equals, equalsIgnoreCase, getBytes, 
                indexOf, lastIndexOf, length, replace, setCharAt, substring, 
                toCharArray, toInt, press, release, releaseAll, accept, click, move, 
                isPressed, isAlphaNumeric, isAlpha, isAscii, isWhitespace, isControl, 
                isDigit, isGraph, isLowerCase, isPrintable, isPunct, isSpace, 
                isUpperCase, isHexadecimalDigit, 
                }, 
  morekeywords={   % add arduino structures to group 1
                break, case, override, final, continue, default, do, else, for, 
                if, return, goto, switch, throw, try, while, setup, loop, export, 
                not, or, and, xor, include, define, elif, else, error, if, ifdef, 
                ifndef, pragma, warning,
                }, 
% 
%
  %%% Keyword Color Group 2 %%%  (called LITERAL1 by arduino)
  keywordstyle=[2]\color{arduinoBlue},   
  keywords=[2]{   % add variables and dataTypes as 2nd group  
                HIGH, LOW, INPUT, INPUT_PULLUP, OUTPUT, DEC, BIN, HEX, OCT, PI, 
                HALF_PI, TWO_PI, LSBFIRST, MSBFIRST, CHANGE, FALLING, RISING, 
                DEFAULT, EXTERNAL, INTERNAL, INTERNAL1V1, INTERNAL2V56, LED_BUILTIN, 
                LED_BUILTIN_RX, LED_BUILTIN_TX, DIGITAL_MESSAGE, FIRMATA_STRING, 
                ANALOG_MESSAGE, REPORT_DIGITAL, REPORT_ANALOG, SET_PIN_MODE, 
                SYSTEM_RESET, SYSEX_START, auto, int8_t, int16_t, int32_t, int64_t, 
                uint8_t, uint16_t, uint32_t, uint64_t, char16_t, char32_t, operator, 
                enum, delete, bool, boolean, byte, char, const, false, float, double, 
                null, NULL, int, long, new, private, protected, public, short, 
                signed, static, volatile, String, void, true, unsigned, word, array, 
                sizeof, dynamic_cast, typedef, const_cast, struct, static_cast, union, 
                friend, extern, class, reinterpret_cast, register, explicit, inline, 
                _Bool, complex, _Complex, _Imaginary, atomic_bool, atomic_char, 
                atomic_schar, atomic_uchar, atomic_short, atomic_ushort, atomic_int, 
                atomic_uint, atomic_long, atomic_ulong, atomic_llong, atomic_ullong, 
                virtual, PROGMEM,
                },  
% 
%
  %%% Keyword Color Group 3 %%%  (called KEYWORD1 by arduino)
  keywordstyle=[3]\bfseries\color{arduinoOrange},
  keywords=[3]{  % add built-in functions as a 3rd group
                Serial, Serial1, Serial2, Serial3, SerialUSB, Keyboard, Mouse,
                },      
%
%
  %%% Keyword Color Group 4 %%%  (called KEYWORD2 by arduino)
  keywordstyle=[4]\color{arduinoOrange},
  keywords=[4]{  % add more built-in functions as a 4th group
                abs, acos, asin, atan, atan2, ceil, constrain, cos, degrees, exp, 
                floor, log, map, max, min, radians, random, randomSeed, round, sin, 
                sq, sqrt, tan, pow, bitRead, bitWrite, bitSet, bitClear, bit, 
                highByte, lowByte, analogReference, analogRead, 
                analogReadResolution, analogWrite, analogWriteResolution, 
                attachInterrupt, detachInterrupt, digitalPinToInterrupt, delay, 
                delayMicroseconds, digitalWrite, digitalRead, interrupts, millis, 
                micros, noInterrupts, noTone, pinMode, pulseIn, pulseInLong, shiftIn, 
                shiftOut, tone, yield, Stream, begin, end, peek, read, print, 
                println, available, availableForWrite, flush, setTimeout, find, 
                findUntil, parseInt, parseFloat, readBytes, readBytesUntil, readString, 
                readStringUntil, trim, toUpperCase, toLowerCase, charAt, compareTo, 
                concat, endsWith, startsWith, equals, equalsIgnoreCase, getBytes, 
                indexOf, lastIndexOf, length, replace, setCharAt, substring, 
                toCharArray, toInt, press, release, releaseAll, accept, click, move, 
                isPressed, isAlphaNumeric, isAlpha, isAscii, isWhitespace, isControl, 
                isDigit, isGraph, isLowerCase, isPrintable, isPunct, isSpace, 
                isUpperCase, isHexadecimalDigit, 
                },      
%
%
  %%% Set Other Colors %%%
  stringstyle=\color{arduinoDarkBlue},    
  commentstyle=\color{arduinoGrey},    
%          
%   
  %%%% Line Numbering %%%%
%  numbers=left,                    
%  numbersep=5pt,                   
%  numberstyle=\color{arduinoGrey},    
  %stepnumber=2,                      % show every 2 line numbers
%
%
  %%%% Code Box Style %%%%
  breaklines=true,                    % wordwrapping
  tabsize=8,         
  basicstyle=\ttfamily  
}
                                             %%%
%%% somewhere before \begin{document} in your latex file.                      %%%
%%%                                                                            %%%
%%% In your document, place your arduino code between:                         %%%
%%%   \begin{lstlisting}[language=Arduino]                                     %%%
%%% and:                                                                       %%%
%%%   \end{lstlisting}                                                         %%%
%%%                                                                            %%%
%%% Or create your own style to add non-built-in functions and variables.      %%%
%%%                                                                            %%%
 %%%%%%%%%%%%%%%%%%%%%%%%%%%%%%%%%%%%%%%%%%%%%%%%%%%%%%%%%%%%%%%%%%%%%%%%%%%%%%%% 

\usepackage{color}
\usepackage{listings}    
\usepackage{courier}

%%% Define Custom IDE Colors %%%
\definecolor{arduinoGreen}    {rgb} {0.17, 0.43, 0.01}
\definecolor{arduinoGrey}     {rgb} {0.47, 0.47, 0.33}
\definecolor{arduinoOrange}   {rgb} {0.8 , 0.4 , 0   }
\definecolor{arduinoBlue}     {rgb} {0.01, 0.61, 0.98}
\definecolor{arduinoDarkBlue} {rgb} {0.0 , 0.2 , 0.5 }

%%% Define Arduino Language %%%
\lstdefinelanguage{Arduino}{
  language=C++, % begin with default C++ settings 
%
%
  %%% Keyword Color Group 1 %%%  (called KEYWORD3 by arduino)
  keywordstyle=\color{arduinoGreen},   
  deletekeywords={  % remove all arduino keywords that might be in c++
                break, case, override, final, continue, default, do, else, for, 
                if, return, goto, switch, throw, try, while, setup, loop, export, 
                not, or, and, xor, include, define, elif, else, error, if, ifdef, 
                ifndef, pragma, warning,
                HIGH, LOW, INPUT, INPUT_PULLUP, OUTPUT, DEC, BIN, HEX, OCT, PI, 
                HALF_PI, TWO_PI, LSBFIRST, MSBFIRST, CHANGE, FALLING, RISING, 
                DEFAULT, EXTERNAL, INTERNAL, INTERNAL1V1, INTERNAL2V56, LED_BUILTIN, 
                LED_BUILTIN_RX, LED_BUILTIN_TX, DIGITAL_MESSAGE, FIRMATA_STRING, 
                ANALOG_MESSAGE, REPORT_DIGITAL, REPORT_ANALOG, SET_PIN_MODE, 
                SYSTEM_RESET, SYSEX_START, auto, int8_t, int16_t, int32_t, int64_t, 
                uint8_t, uint16_t, uint32_t, uint64_t, char16_t, char32_t, operator, 
                enum, delete, bool, boolean, byte, char, const, false, float, double, 
                null, NULL, int, long, new, private, protected, public, short, 
                signed, static, volatile, String, void, true, unsigned, word, array, 
                sizeof, dynamic_cast, typedef, const_cast, struct, static_cast, union, 
                friend, extern, class, reinterpret_cast, register, explicit, inline, 
                _Bool, complex, _Complex, _Imaginary, atomic_bool, atomic_char, 
                atomic_schar, atomic_uchar, atomic_short, atomic_ushort, atomic_int, 
                atomic_uint, atomic_long, atomic_ulong, atomic_llong, atomic_ullong, 
                virtual, PROGMEM,
                Serial, Serial1, Serial2, Serial3, SerialUSB, Keyboard, Mouse,
                abs, acos, asin, atan, atan2, ceil, constrain, cos, degrees, exp, 
                floor, log, map, max, min, radians, random, randomSeed, round, sin, 
                sq, sqrt, tan, pow, bitRead, bitWrite, bitSet, bitClear, bit, 
                highByte, lowByte, analogReference, analogRead, 
                analogReadResolution, analogWrite, analogWriteResolution, 
                attachInterrupt, detachInterrupt, digitalPinToInterrupt, delay, 
                delayMicroseconds, digitalWrite, digitalRead, interrupts, millis, 
                micros, noInterrupts, noTone, pinMode, pulseIn, pulseInLong, shiftIn, 
                shiftOut, tone, yield, Stream, begin, end, peek, read, print, 
                println, available, availableForWrite, flush, setTimeout, find, 
                findUntil, parseInt, parseFloat, readBytes, readBytesUntil, readString, 
                readStringUntil, trim, toUpperCase, toLowerCase, charAt, compareTo, 
                concat, endsWith, startsWith, equals, equalsIgnoreCase, getBytes, 
                indexOf, lastIndexOf, length, replace, setCharAt, substring, 
                toCharArray, toInt, press, release, releaseAll, accept, click, move, 
                isPressed, isAlphaNumeric, isAlpha, isAscii, isWhitespace, isControl, 
                isDigit, isGraph, isLowerCase, isPrintable, isPunct, isSpace, 
                isUpperCase, isHexadecimalDigit, 
                }, 
  morekeywords={   % add arduino structures to group 1
                break, case, override, final, continue, default, do, else, for, 
                if, return, goto, switch, throw, try, while, setup, loop, export, 
                not, or, and, xor, include, define, elif, else, error, if, ifdef, 
                ifndef, pragma, warning,
                }, 
% 
%
  %%% Keyword Color Group 2 %%%  (called LITERAL1 by arduino)
  keywordstyle=[2]\color{arduinoBlue},   
  keywords=[2]{   % add variables and dataTypes as 2nd group  
                HIGH, LOW, INPUT, INPUT_PULLUP, OUTPUT, DEC, BIN, HEX, OCT, PI, 
                HALF_PI, TWO_PI, LSBFIRST, MSBFIRST, CHANGE, FALLING, RISING, 
                DEFAULT, EXTERNAL, INTERNAL, INTERNAL1V1, INTERNAL2V56, LED_BUILTIN, 
                LED_BUILTIN_RX, LED_BUILTIN_TX, DIGITAL_MESSAGE, FIRMATA_STRING, 
                ANALOG_MESSAGE, REPORT_DIGITAL, REPORT_ANALOG, SET_PIN_MODE, 
                SYSTEM_RESET, SYSEX_START, auto, int8_t, int16_t, int32_t, int64_t, 
                uint8_t, uint16_t, uint32_t, uint64_t, char16_t, char32_t, operator, 
                enum, delete, bool, boolean, byte, char, const, false, float, double, 
                null, NULL, int, long, new, private, protected, public, short, 
                signed, static, volatile, String, void, true, unsigned, word, array, 
                sizeof, dynamic_cast, typedef, const_cast, struct, static_cast, union, 
                friend, extern, class, reinterpret_cast, register, explicit, inline, 
                _Bool, complex, _Complex, _Imaginary, atomic_bool, atomic_char, 
                atomic_schar, atomic_uchar, atomic_short, atomic_ushort, atomic_int, 
                atomic_uint, atomic_long, atomic_ulong, atomic_llong, atomic_ullong, 
                virtual, PROGMEM,
                },  
% 
%
  %%% Keyword Color Group 3 %%%  (called KEYWORD1 by arduino)
  keywordstyle=[3]\bfseries\color{arduinoOrange},
  keywords=[3]{  % add built-in functions as a 3rd group
                Serial, Serial1, Serial2, Serial3, SerialUSB, Keyboard, Mouse,
                },      
%
%
  %%% Keyword Color Group 4 %%%  (called KEYWORD2 by arduino)
  keywordstyle=[4]\color{arduinoOrange},
  keywords=[4]{  % add more built-in functions as a 4th group
                abs, acos, asin, atan, atan2, ceil, constrain, cos, degrees, exp, 
                floor, log, map, max, min, radians, random, randomSeed, round, sin, 
                sq, sqrt, tan, pow, bitRead, bitWrite, bitSet, bitClear, bit, 
                highByte, lowByte, analogReference, analogRead, 
                analogReadResolution, analogWrite, analogWriteResolution, 
                attachInterrupt, detachInterrupt, digitalPinToInterrupt, delay, 
                delayMicroseconds, digitalWrite, digitalRead, interrupts, millis, 
                micros, noInterrupts, noTone, pinMode, pulseIn, pulseInLong, shiftIn, 
                shiftOut, tone, yield, Stream, begin, end, peek, read, print, 
                println, available, availableForWrite, flush, setTimeout, find, 
                findUntil, parseInt, parseFloat, readBytes, readBytesUntil, readString, 
                readStringUntil, trim, toUpperCase, toLowerCase, charAt, compareTo, 
                concat, endsWith, startsWith, equals, equalsIgnoreCase, getBytes, 
                indexOf, lastIndexOf, length, replace, setCharAt, substring, 
                toCharArray, toInt, press, release, releaseAll, accept, click, move, 
                isPressed, isAlphaNumeric, isAlpha, isAscii, isWhitespace, isControl, 
                isDigit, isGraph, isLowerCase, isPrintable, isPunct, isSpace, 
                isUpperCase, isHexadecimalDigit, 
                },      
%
%
  %%% Set Other Colors %%%
  stringstyle=\color{arduinoDarkBlue},    
  commentstyle=\color{arduinoGrey},    
%          
%   
  %%%% Line Numbering %%%%
%  numbers=left,                    
%  numbersep=5pt,                   
%  numberstyle=\color{arduinoGrey},    
  %stepnumber=2,                      % show every 2 line numbers
%
%
  %%%% Code Box Style %%%%
  breaklines=true,                    % wordwrapping
  tabsize=8,         
  basicstyle=\ttfamily  
}
                                             %%%
%%% somewhere before \begin{document} in your latex file.                      %%%
%%%                                                                            %%%
%%% In your document, place your arduino code between:                         %%%
%%%   \begin{lstlisting}[language=Arduino]                                     %%%
%%% and:                                                                       %%%
%%%   \end{lstlisting}                                                         %%%
%%%                                                                            %%%
%%% Or create your own style to add non-built-in functions and variables.      %%%
%%%                                                                            %%%
 %%%%%%%%%%%%%%%%%%%%%%%%%%%%%%%%%%%%%%%%%%%%%%%%%%%%%%%%%%%%%%%%%%%%%%%%%%%%%%%% 

\usepackage{color}
\usepackage{listings}    
\usepackage{courier}

%%% Define Custom IDE Colors %%%
\definecolor{arduinoGreen}    {rgb} {0.17, 0.43, 0.01}
\definecolor{arduinoGrey}     {rgb} {0.47, 0.47, 0.33}
\definecolor{arduinoOrange}   {rgb} {0.8 , 0.4 , 0   }
\definecolor{arduinoBlue}     {rgb} {0.01, 0.61, 0.98}
\definecolor{arduinoDarkBlue} {rgb} {0.0 , 0.2 , 0.5 }

%%% Define Arduino Language %%%
\lstdefinelanguage{Arduino}{
  language=C++, % begin with default C++ settings 
%
%
  %%% Keyword Color Group 1 %%%  (called KEYWORD3 by arduino)
  keywordstyle=\color{arduinoGreen},   
  deletekeywords={  % remove all arduino keywords that might be in c++
                break, case, override, final, continue, default, do, else, for, 
                if, return, goto, switch, throw, try, while, setup, loop, export, 
                not, or, and, xor, include, define, elif, else, error, if, ifdef, 
                ifndef, pragma, warning,
                HIGH, LOW, INPUT, INPUT_PULLUP, OUTPUT, DEC, BIN, HEX, OCT, PI, 
                HALF_PI, TWO_PI, LSBFIRST, MSBFIRST, CHANGE, FALLING, RISING, 
                DEFAULT, EXTERNAL, INTERNAL, INTERNAL1V1, INTERNAL2V56, LED_BUILTIN, 
                LED_BUILTIN_RX, LED_BUILTIN_TX, DIGITAL_MESSAGE, FIRMATA_STRING, 
                ANALOG_MESSAGE, REPORT_DIGITAL, REPORT_ANALOG, SET_PIN_MODE, 
                SYSTEM_RESET, SYSEX_START, auto, int8_t, int16_t, int32_t, int64_t, 
                uint8_t, uint16_t, uint32_t, uint64_t, char16_t, char32_t, operator, 
                enum, delete, bool, boolean, byte, char, const, false, float, double, 
                null, NULL, int, long, new, private, protected, public, short, 
                signed, static, volatile, String, void, true, unsigned, word, array, 
                sizeof, dynamic_cast, typedef, const_cast, struct, static_cast, union, 
                friend, extern, class, reinterpret_cast, register, explicit, inline, 
                _Bool, complex, _Complex, _Imaginary, atomic_bool, atomic_char, 
                atomic_schar, atomic_uchar, atomic_short, atomic_ushort, atomic_int, 
                atomic_uint, atomic_long, atomic_ulong, atomic_llong, atomic_ullong, 
                virtual, PROGMEM,
                Serial, Serial1, Serial2, Serial3, SerialUSB, Keyboard, Mouse,
                abs, acos, asin, atan, atan2, ceil, constrain, cos, degrees, exp, 
                floor, log, map, max, min, radians, random, randomSeed, round, sin, 
                sq, sqrt, tan, pow, bitRead, bitWrite, bitSet, bitClear, bit, 
                highByte, lowByte, analogReference, analogRead, 
                analogReadResolution, analogWrite, analogWriteResolution, 
                attachInterrupt, detachInterrupt, digitalPinToInterrupt, delay, 
                delayMicroseconds, digitalWrite, digitalRead, interrupts, millis, 
                micros, noInterrupts, noTone, pinMode, pulseIn, pulseInLong, shiftIn, 
                shiftOut, tone, yield, Stream, begin, end, peek, read, print, 
                println, available, availableForWrite, flush, setTimeout, find, 
                findUntil, parseInt, parseFloat, readBytes, readBytesUntil, readString, 
                readStringUntil, trim, toUpperCase, toLowerCase, charAt, compareTo, 
                concat, endsWith, startsWith, equals, equalsIgnoreCase, getBytes, 
                indexOf, lastIndexOf, length, replace, setCharAt, substring, 
                toCharArray, toInt, press, release, releaseAll, accept, click, move, 
                isPressed, isAlphaNumeric, isAlpha, isAscii, isWhitespace, isControl, 
                isDigit, isGraph, isLowerCase, isPrintable, isPunct, isSpace, 
                isUpperCase, isHexadecimalDigit, 
                }, 
  morekeywords={   % add arduino structures to group 1
                break, case, override, final, continue, default, do, else, for, 
                if, return, goto, switch, throw, try, while, setup, loop, export, 
                not, or, and, xor, include, define, elif, else, error, if, ifdef, 
                ifndef, pragma, warning,
                }, 
% 
%
  %%% Keyword Color Group 2 %%%  (called LITERAL1 by arduino)
  keywordstyle=[2]\color{arduinoBlue},   
  keywords=[2]{   % add variables and dataTypes as 2nd group  
                HIGH, LOW, INPUT, INPUT_PULLUP, OUTPUT, DEC, BIN, HEX, OCT, PI, 
                HALF_PI, TWO_PI, LSBFIRST, MSBFIRST, CHANGE, FALLING, RISING, 
                DEFAULT, EXTERNAL, INTERNAL, INTERNAL1V1, INTERNAL2V56, LED_BUILTIN, 
                LED_BUILTIN_RX, LED_BUILTIN_TX, DIGITAL_MESSAGE, FIRMATA_STRING, 
                ANALOG_MESSAGE, REPORT_DIGITAL, REPORT_ANALOG, SET_PIN_MODE, 
                SYSTEM_RESET, SYSEX_START, auto, int8_t, int16_t, int32_t, int64_t, 
                uint8_t, uint16_t, uint32_t, uint64_t, char16_t, char32_t, operator, 
                enum, delete, bool, boolean, byte, char, const, false, float, double, 
                null, NULL, int, long, new, private, protected, public, short, 
                signed, static, volatile, String, void, true, unsigned, word, array, 
                sizeof, dynamic_cast, typedef, const_cast, struct, static_cast, union, 
                friend, extern, class, reinterpret_cast, register, explicit, inline, 
                _Bool, complex, _Complex, _Imaginary, atomic_bool, atomic_char, 
                atomic_schar, atomic_uchar, atomic_short, atomic_ushort, atomic_int, 
                atomic_uint, atomic_long, atomic_ulong, atomic_llong, atomic_ullong, 
                virtual, PROGMEM,
                },  
% 
%
  %%% Keyword Color Group 3 %%%  (called KEYWORD1 by arduino)
  keywordstyle=[3]\bfseries\color{arduinoOrange},
  keywords=[3]{  % add built-in functions as a 3rd group
                Serial, Serial1, Serial2, Serial3, SerialUSB, Keyboard, Mouse,
                },      
%
%
  %%% Keyword Color Group 4 %%%  (called KEYWORD2 by arduino)
  keywordstyle=[4]\color{arduinoOrange},
  keywords=[4]{  % add more built-in functions as a 4th group
                abs, acos, asin, atan, atan2, ceil, constrain, cos, degrees, exp, 
                floor, log, map, max, min, radians, random, randomSeed, round, sin, 
                sq, sqrt, tan, pow, bitRead, bitWrite, bitSet, bitClear, bit, 
                highByte, lowByte, analogReference, analogRead, 
                analogReadResolution, analogWrite, analogWriteResolution, 
                attachInterrupt, detachInterrupt, digitalPinToInterrupt, delay, 
                delayMicroseconds, digitalWrite, digitalRead, interrupts, millis, 
                micros, noInterrupts, noTone, pinMode, pulseIn, pulseInLong, shiftIn, 
                shiftOut, tone, yield, Stream, begin, end, peek, read, print, 
                println, available, availableForWrite, flush, setTimeout, find, 
                findUntil, parseInt, parseFloat, readBytes, readBytesUntil, readString, 
                readStringUntil, trim, toUpperCase, toLowerCase, charAt, compareTo, 
                concat, endsWith, startsWith, equals, equalsIgnoreCase, getBytes, 
                indexOf, lastIndexOf, length, replace, setCharAt, substring, 
                toCharArray, toInt, press, release, releaseAll, accept, click, move, 
                isPressed, isAlphaNumeric, isAlpha, isAscii, isWhitespace, isControl, 
                isDigit, isGraph, isLowerCase, isPrintable, isPunct, isSpace, 
                isUpperCase, isHexadecimalDigit, 
                },      
%
%
  %%% Set Other Colors %%%
  stringstyle=\color{arduinoDarkBlue},    
  commentstyle=\color{arduinoGrey},    
%          
%   
  %%%% Line Numbering %%%%
%  numbers=left,                    
%  numbersep=5pt,                   
%  numberstyle=\color{arduinoGrey},    
  %stepnumber=2,                      % show every 2 line numbers
%
%
  %%%% Code Box Style %%%%
  breaklines=true,                    % wordwrapping
  tabsize=8,         
  basicstyle=\ttfamily  
}

%%%%%% Counting oppgaves %%%%%%
 \newcount\questnum \questnum=0
 \def\oppgave{
            \advance\questnum by 1
            \ifnum \questnum > 0
                 \hrule
                 \vskip 3pt
                 \leftline{Oppgave \the\questnum}
                 \vskip 3pt \fi}
 %%%%%%%%%%%%%%%%%%%%%%
%%%%%%%%%%%%%%%%%%%%%%


%%%%%% Counting answers %%%%%%
\newcount\answnum \answnum=0
\def\svar{
           \advance\answnum by 1
           \ifnum \answnum > 0
                \hrule
                \vskip 3pt
                \leftline{Svar \the\answnum}
                \vskip 3pt \fi}
%%%%%%%%%%%%%%%%%%%%%%


%%%%%% Counting notes %%%%%%
\newcount\explnum \explnum=0
\def\notes{
           \advance\explnum by 1
           \ifnum \explnum > 0
                \hrule
                \vskip 3pt
                \leftline{Notes \the\explnum}
                \vskip 3pt \fi}
%%%%%%%%%%%%%%%%%%%%%%

% End preamble

\begin{document}
\begin{titlepage}
   \begin{center}
       \vspace*{1cm}

       \textbf{Infoskriv 3AUA Gand VGS}

       \vspace{0.5cm}
        Skoleåret 2024/2025
            
       \vspace{1.5cm}

       \textbf{Kontaktlærer: Fred-Olav Mosdal}

       \vfill
            
            
     
       \includegraphics[width=1\textwidth]{/home/fred-olav/Downloads/GandLogo.jpg}
    \vfill        
            
   \end{center}
\end{titlepage}

% Copyright 2015, Tony R. Kuphaldt, released under the Creative Commons Attribution License (v 1.0)
% This means you may do almost anything with this work of mine, so long as you give me proper credit

%(BEGIN_FRONTMATTER)
\centerline{\bf Hvordan . . .} \bigskip 

\noindent
{\bf Finne fagplan for VG3 Automatiseringsfaget:} \url{https://www.udir.no/lk20/aut03-04}
\vskip 10pt

\noindent
{\bf Finner jeg skoleruten:} \url{https://www.gand.vgs.no/hovedmeny/for-elever/skolehverdag/skoleferier-og-fridager-skoleruta/}
\vskip 10pt

\noindent
{\bf Finner jeg informasjon om Skolen:} \url{https://www.gand.vgs.no/hovedmeny/for-elever/}
\vskip 10pt

\noindent
{\bf Finner jeg timeplanen:} Timeplanen din finner du i VIS. 
\vskip 10pt

\noindent
{\bf Få penger til utstyr som kreves:} Alle elever med ungdomsrett i videregående opplæring får utstyrsstipend. Utstyrsstipendet er ikke avhengig av hvor mye foreldrene tjener.
Søknaden skrives til lånekassen \url{http://www.lanekassen.no}
\vskip 10pt

\noindent
{\bf Få tak i oppgaver og lærebok} På siden autofaget.no legges oppgaver og teori ut. Noe vil også bli levert på papir eller lagt ut på Teams. 
\vskip 10pt

\noindent
{\bf Få tak i  software og biblioteker:} Software som vi bruker legger jeg ut i ukesplanen på autofaget.no  
\vskip 10pt

\noindent
{\bf Hvordan lære mest mulig:} kom til skolen forberedt hver eneste dag -- dette betyr at du har gjort alle lekser gitt til/etter en leksjon. Fulgt alle tips som gis på oppgaveark og av lærer. Ikke spør andre om hjelp før du har gjort en rimlig innsats selv. Hjelp andre med å gjennomføre oppgave og å forstå, men ikke gjør jobben for DE. 
\vskip 10pt

\noindent
{\bf Holde orden på innleveringer og frister.} Følg med på beskjeder gitt av lærer. I arbeidslivet vil du få muntlige beskjeder som det forventes at du følger opp, slik er det her også. Er du vekke fra skolen må du orientere deg med medelever. Se også autofaget.no
\vskip 10pt


\noindent
{\bf Finne fagstoff og manualer fra produsenter } Det forventes at du kan søke dette opp på internet selv. 
\vskip 10pt



\noindent
{\bf Få seg læreplass:} Ca. i desember starter de første firmaene å legge annonser for nye lærlinger (Søk på så mange du klarer å håndtere ca. 50), når du er i utplassering må du vise deg som en attraktiv arbeidstager, sørg for å være faglig interessert/på, ha minst mulig fravær og følgmed på Teams der legger jeg forespørsler fra firmaer. 
\vskip 0.5 cm
1) Vær smart allerede når du søker utplassering\\
Hvis det er en spesiell bedrift du kunne tenke deg å være lærling hos, tenk på dette allerede når du skal være utplassert fra skolen. Det er ofte at vi tilbyr gode utplasseringselever lærlingplass.

\vskip 0.5 cm
På utdanning.no kan du selv søke opp aktuelle lærebedrifter innen ditt \href{https://utdanning.no/finnlarebedrift/}{fag}.
\vskip 0.5 cm
2) Vær smart under utplassering. \\
Når man er utplassert, er det ikke alltid man får så mye praktisk arbeid. Men ta det du får. Gjør oppgavene du får skikkelig, vær nysgjerrig, still spørsmål og gjør ditt beste. Møt opp i tide, og vær på tilbudssiden. Vi husker gode, proaktive ungdommer.
\vskip 0.5 cm


3) Lite fravær på skolen\\
Noe av det første vi ser på og spør deg om på intervju er hvor mye fravær du har. Har du mye, er det mindre sjanse for å få lærlingeplass. Så, det er viktig å møte opp på skolen! Hvis du har mye fravær, er det lurt å forklare hvorfor.
\vskip 0.5 cm

4) Forbered deg\\
Les om selskapet du vil være lærling hos på nettsiden deres, slik at du vet litt om dem før du skriver søknad. Det er alltid en fordel å vite litt om bedriften man søker jobb hos. De som mottar søknader, ser fort om det er en generell søknad som har gått ut til mange eller om kandidaten er ekte interessert i bedriften
\vskip 0.5 cm

5) Søk tidlig – og vis interesse\\
Lærebedriften bestemmer selv hvem de vil ha som lærling, så det er ingen garanti for å få plassen du helst vil ha. Derfor er det viktig å ta ansvar, vise interesse og kontakte bedriften direkte. Start jakten tidlig – gjerne et år før du planlegger å begynne.
\vskip 0.5 cm

6) Lever søknaden der du skal, og legg ved det du får beskjed om\\
Hvis bedriften ber om å få søknaden gjennom et skjema på nettsiden, send den der. Vil bedriften at du skal sende på epost, gjør det. Står det at du skal legge ved karakterutskrift og/eller CV. Send det med en gang, så slipper bedriften svare deg tilbake at de mangler informasjon. Dette viser at du er oppegående og følger med!
\vskip 0.5 cm

7) Vær nøye med søknaden og CVen\\
Bruk litt tid på å skrive en god søknad og fiks CVen din. All arbeidserfaring er positivt og kan listes opp i en CV, inkludert frivillige verv du måtte ha. Søknaden bør vise hvorfor du vil jobbe i den spesielle bedriften og hva du har å tilby. Med andre ord, vis at du har satt deg inn i bedriften og skriv hvordan du kan bidra til suksess. Det kan være nøkkelen til å få den lærlingeplassen du vil ha.
\vskip 0.5 cm


Et siste lite tips:
Hvis du har fått hjelp av AI med å skrive søknad, sørg for at du fjerner AI’ens kommentar, som: Selvfølgelig, jeg kan hjelpe deg med å lage et utkast til søknad for lærlingstilling… Det ser veldig slurvete ut. Les også gjennom teksten, og se om språket er ditt. Vi ser flere og flere AI-genererte søknader med nesten helt lik tekst. Da virker man ikke genuint interessert i jobben.


\vskip 10pt


\vfil

\underbar{file {\tt hvordan}}
\eject
%(END_FRONTMATTER)


% Copyright 2015, Tony R. Kuphaldt, released under the Creative Commons Attribution License (v 1.0)
% This means you may do almost anything with this work of mine, so long as you give me proper credit

%(BEGIN_FRONTMATTER)
\centerline{\bf Verdier og forventninger. } \bigskip 
%\centerline{\bf General Values and Expectations} \bigskip 


\noindent
{\bf Gand Standarden:}    
\begin{itemize}
\item{} Alle møter hverandre med respekt
\item{} Alle møter godt forberedt og har med nødvendig utstyr
\item{} Alle møter presis til timene
\end{itemize}
\vskip 10pt


\noindent
{\bf Orden og adferd:} I 3AUA føres orden-og adferds notater. Når en lærer mener en elev har dårlig adverd eller orden, noteres dette i VIS. I VIS kalles denne funkjonen for anmerkning, men det er bare et notat for oss lærer å huske hva som har skjedd.  
\vskip 10pt
Noen eksempler på hva som forventes for å oppnå karakteren god i orden og adferd. 
\vskip 10pt
Orden:
\begin{itemize}
\vskip 10pt
\item{}Eleven retter seg etter lover, regler og instrukser som til enhver tid gjelder, og følger anvisninger fra skolens personale.
\item{}Eleven møter presis. Hvis en elev kommer mer enn 10 minutt for seint til første time, regnes det som fravær. Om en kommer etter undervisningen har startet i timer resten av dagen regnes det som fravær. 
\item{}Eleven gjør pålagt skolearbeid til avtalt tid.
\item{}Eleven har med til undervisning, det materiell og utstyr som opplæringa krever.
\item{}Eleven behandler skolens lokaler, bøker og utstyr på en forsvarlig måte.
\item{}Eleven holder det rent og ryddig på skolen, i nærområdet og på steder der opplæringa foregår.
\item{}Eleven bruker påbudt verneutstyr og ellers overholde gjeldende regler for helse, miljø og sikkerhet (HMS).
\item{}Eleven varsler lærer/kontaktlærer når han må forlate skolen i opplæringstida.

\item{} Eleven holder arbeidsplassen ren og oversiktlig. Dette inkluderer kaste boss, returnere verktøy etter bruk eller på slutten av dagen, delta i opprydding ved endt praksis. Kommer du over defekt verktøy eller utsyr, merk utstyret med en detaljert beskrivelse av problemet. 
\vskip 10pt
\item{} Eleven rydder skrivepult og arbeidsplass for personlige eiendeler i slutten av dagen. Bøker og papierer skal i skap eller tas med hjem. Verktøk skal i verktøykasse og kassen plasseres på anvist plass. 
\end{itemize}
Atferd:
\vskip 10pt
\begin{itemize}
\item{}Eleven bidrar til et godt arbeidsmiljø ved å opptre hensynsfullt og høflig, holde arbeidsro i timene og vise respekt for andre både i undervisningssituasjoner og i pauser.
\item{}Krenkende atferd (mobbing, diskriminering, rasistiske utsagn, seksuell trakassering, vold o.l.), det være seg fysisk, verbalt eller digitalt, aksepteres ikke.
\item{}Tobakksbruk er forbudt i videregående skolers lokaler og uteområder. Elever ved de videregående skoler skal være tobakksfrie i skoletiden. E-cigaretter er ikke tillatt brukt i skoletiden.
\item{}Kjøp, salg, besittelse og bruk av rusmidler på skolens område er forbudt. Det er heller ikke tillatt å være påvirket av rusmidler i skoletiden eller ved aktiviteter i skolens regi. Den enkelte skole har utfyllende regler for behandling av overtredelse av dette.
\item{}Det er ikke tillatt å ha med seg farlige gjenstander eller våpen på skolens område eller ved aktiviteter i skolens regi.
\item{}Fusk eller forsøk på fusk vil normalt føre til nedsatt karakter i atferd.
\end{itemize}
\vskip 10pt

%\noindent
%{\bf For å bli en god Automatikker kreves:} faglig integritet,   
%
%%{\bf Success in this career requires:} professional integrity, resourcefulness, persistence, close attention to detail, and intellectual curiosity.  Poor judgment spells disaster in this career, which is why employer background checks (including social media and criminal records) and drug testing are common.  The good news is that character and clear thinking are malleable traits: unlike intelligence, these qualities can be acquired and improved with effort.  {\it This is what you are in school to do} -- increase your ``human capital'' which is the sum of all knowledge, skills, and traits valuable in the marketplace.
%
%\vskip 10pt
%
%\noindent
%%\underbar{\bf Mastery:} You must master the fundamentals of your chosen profession.  ``Mastery'' assessments challenge you to demonstrate 100\% competence (with multiple opportunities to re-try).  Failure to complete any mastery objective(s) by the deadline date caps your grade at a C$-$.  Failure to complete by the end of the next school day results in a failing (F) grade.
%
%\vskip 10pt
%

\noindent
\underbar{\bf Bruk av tid:} Det er forventet at du prioriterer tiden slik du må i jobb. Du er ansvarlig for at du har nok tid utenom skolen til å gjøre hjemmelekser og innleveringer. På skolen er det forventet at du bruker felles utstyr effektivt slik at alle får komme til. Forstyrrende aktiviteter, som spilling, sosiale medier, surfing på internett, er ikke akseptert i klasserommet når undervisning pågår. Mobiler er ikke lov og ha fremme i timen. 



\vskip 10pt



\noindent
\underbar{\bf Initiativ:} Personlig initiativ er blandt de viktigste egenskapene til en automatikker. Derfor vil lærere på i 3AUA kreve at du finner ut av problemer selv, istedenfor å hele tiden spør andre om hjelp. Eksempler kan være: Kom i gang med elevkonto i office365, sjekk manualer for du spør om hjelp, hold orden på prøver og innleveringer. Om du finner ut at du ikke forstår en oppgave på grunn av at du ikke husker tidligere emner, les disse emnene om igjen. 
\vskip 10pt

\noindent
\underbar{\bf Sikkerhet:} Det forventes at du jobber sikkert i klasserommet. Dette innkluderer: at du har på deg arbeidsklær og vernesko (når risikovurderingen krever det,  skal du også ha på briller, hansker, og/eller hørselsværn), at du bruker lås og merk når du jobber på eller nær ved ledere med spenning på over 30 V, bruker gardintrapp for å komme til i høyden, bruke verktøy korrekt. Om du skal bruke et ukjent verktøy må du henvende deg til læreren for instruksjon. 
\vskip 10pt

\noindent
\underbar{\bf Gruppearbeid:} Dere vil jobbe i grupper på noen av arbeidsoppdragene i løpet av året. Som del av en gruppe må du holde de andre oppdatert på hvor du er. Elever som ødelegger for abeid i gruppen  ved å: ikke møte opp, ikke delta, manglende respekt eller annen forstyrrdende oppførsel vil fjernes fra gruppen settes ned i adferd og må utføre arbeidsoppdrag individuelt resten av terminen. 
\vskip 10pt

\noindent
\underbar{\bf Samarbeid:} Måten vi jobber på i 3AUA gjør at det blir mye samarbeid mellom elevene. Dette gjør det dere påvirker hverandres læring i stor grad. Det forventes at du tar denne rollen seriøst, og tilby hjelp som hjelper på læring. Det vil si å ikke løse problemer for andre elever, men hjelpe med problemløsning. Kom med tips for å løse problemer, still spørsmål som leder til en løsning istedenfor å gi en løsning. 
\vskip 10pt

%\noindent
%\underbar{\bf Academic Engagement:} Instrumentation is a challenging career requiring creative and critical thinking.  As industry advisors have said, ``Being an instrument technician is as close as you get to doing engineering without a four-year (Bachelor's) degree.''  The only way to prepare for the challenges of being an instrument technician is to exercise that same level of creative and critical thinking before stepping into the career, mastering {\it first principles} of science and {\it general problem-solving} strategies rather than focusing on simpler tasks such as memorization and procedures.  This also means personally involving yourself in every learning exercise, not being content to merely observe others.  Individual (unassisted) performance is the gold standard for learning: {\it unless and until \underbar{you} can consistently perform on your own, you haven't learned!}

%\vskip 10pt

%\noindent
%\underbar{\bf Grades:} Employers prize trustworthy, hard working, knowledgeable, resourceful problem-solvers.  The grade you receive in any course is but a {\it partial} measure of these traits.  What matters most are the traits themselves, which is why your instructor maintains detailed student records (including individual exam scores, attendance, tardiness, and behavioral comments) and will share these records with employers if you have signed the FERPA release form.  You are welcome to see your records at any time, and to compare calculated grades with your own records (i.e. the grade spreadsheet available to all students).  You should expect employers to scrutinize your records on attendance and character, and also challenge you with technical questions when considering you for employment.
%
%\vskip 10pt

%\noindent
%\underbar{\bf Representation:} You are an ambassador for this program.  Your actions, whether on tours, during a jobshadow or internship, or while employed, can open or shut doors of opportunity for other students.  Most of the job opportunities open to you as a BTC graduate were earned by the good work of previous graduates, and as such you owe them a debt of gratitude.  Future graduates depend on you to do the same.
%
%\vskip 10pt

\noindent
\underbar{\bf Ansvar for egne handlinger:} Dersom du mister eller ødelegger skolen sitt utstyr, forventes det at du finner, reparerer eller hjelper til med å erstatte det. 
\vskip 10pt
\vfil
\noindent
\underbar{\bf Praksis automatiseringsfaget i 3AUA:} 
\begin{itemize}
	\item Uke 35-40 Gandsfjorden Gondol
	\item Uke 42-Jul Gand Reguleringsstasjon (skal i år monteres fast i klasserom
	\item Uke Nyttår-April prosjekt fabrikkautomasjon
	\item Mai - Juni Minifagprøve
	\item laboppgaver. 
\end{itemize}
\vskip 10pt
\vfil
\noindent
\underbar{\bf Utplassering:} Utplassering i 3AUA fåregår i uke 45-46 og uke 3-4. Lærer forsøker å skaffe plasser, men dere kan også finne plasser utenom de lærer skaffer. 
De som hønsker å være utplassert hos:
\begin {itemize}
	\item Aker Solutions ( Send mail med søknad til kenneth.holen@akersolutions.com)
\end {itemize}
\vskip 10pt
\vfil


\noindent
\underbar{\bf Selvstudie:} Automatiseringsfaget er preget av kontinuerlig teknologiutviling og pågående forandringer. Derfor er evne til selvstudie viktigere for din utvikling som automatikker en bestemte emner i faget. Vi har mye utstyr i kasserommet som dere må sette dere inn i virkemåten til. 

\vskip 10pt
\noindent
\underbar{\bf Selvstendig problemløsning:} Gode automatikkere kan løse ulike problemer som oppstår i komplekse anlegg. Dette er en egenskap som bygges med erfaring det er derfor viktig at du viser en rimlig innsats for å løse problemer med anleggene du jobber på.
\vskip 10pt
\underbar{file {\tt forventinger}}
\eject
%(END_FRONTMATTER)


% Copyright 2015, Tony R. Kuphaldt, released under the Creative Commons Attribution License (v 1.0)
% This means you may do almost anything with this work of mine, so long as you give me proper credit

%(BEGIN_FRONTMATTER)
\centerline{\bf IT hjelp} \bigskip 

\href{https://forms.office.com/Pages/DesignPageV2.aspx?origin=NeoPortalPage&subpage=design&id=H75sAsQBmEalZi_EPr7HMR3QcgyZmWZGpIMQU6JKCTZUM1dDVVM5WEZHU1A4N0dFRTJMSlpCRThVRi4u}{Hva har jeg gjort?}
\vskip 1cm
\href{https://rfka-my.sharepoint.com/:v:/g/personal/sverre_wilhelmsen_skole_rogfk_no/EfmqE3Pj3mNFlTp-p-owBP8Biqz8LDh1uGrZO5f0QOXu0g?e=z03gZU}{Filer og mappestruktor}
\vskip 1cm
\href{https://vimeo.com/850876389}{Konfiguere og bruke OneDrive}
\vskip 1cm
\href{https://vimeo.com/850863231}{Installere M365}

\vskip 1cm
\href{https://rfka.sharepoint.com/:b:/r/sites/GANDElevinformasjon/Delte%20dokumenter/IT/Infoskriv%20til%20%20VG1%20(2024).pdf?csf=1&web=1&e=NsSipR}{IT informasjon til Gand Elever}
\vskip 1cm
\vskip 1cm
\vskip 1cm


\vskip 1cm
\underbar{file {\tt it}}
\eject
%(END_FRONTMATTER)


% This means you may do almost anything with this work of mine, so long as you give me proper credit

%(BEGIN_FRONTMATTER)

\centerline{\bf Verktøy og utstyrsliste }

\vskip 10pt


\noindent
{\bf Verktøy}
\begin{itemize}[itemsep=1mm, parsep=0pt]
	\item Verktøykasse som passer i skap 
	\item Kjørner
	\item Rissenål
	\item Kniv
	\item Skyvelære
	\item Meterstokk
	\item Flattjern, 3, 4 og 6mm
	\item Phillips, \#1 and \#2
	\item presisjonsskrutrekkersett\\
       \includegraphics[width=0.3\textwidth]{/home/fred-olav/Downloads/presisjonstrekkersett.png}
	\item Posidrive \#2
	\item Trekker med bitssett
	\item Multimeter, CAT 3 600V 
	\item Avisoleringsverktøy
	\item Avbiter
	\item Hvit elektrikkertape og markeringspenn til å skrive på tapen. (til å lage midlertidige markeringer)
\medskip
\end{itemize}
\vskip 10pt

\noindent
{\bf Sikkerhetsutstyr}
\begin{itemize}[itemsep=1mm, parsep=0pt]
	\item Vernebriller
	\item Hørselsvern
	\item Vernesko
	\item Sveisehansker
	\item Monteringshansker
	\item Arbeidsklær i bomull
\medskip
\end{itemize}
\vskip 10pt

\noindent
{\bf Nødvendig utstyr til skoledagen}
\begin{itemize}[itemsep=1mm, parsep=0pt]
	
	\item Still godt forberedt! Ha med PC med windows 11, blyanter, visk, linjal, fargeblyanter, blyantspisser, kalkulator og skrivebøker. Ha med perm til å samle alle ark.
	\item Det er forventet at du holder orden på din PC, slik at denne fungerer i timene. (Kjør en reinstallasjon av PC-en i starten av skoleåret). 
	\item Lag et mappesystem fra dag en. (Da slipper du å tenke over hvor du skal lagre filene dine.)
\medskip
\end{itemize}
\vskip 10pt



\vfil

\underbar{file {\tt tools}}
\eject
%(END_FRONTMATTER)



% This means you may do almost anything with this work of mine, so long as you give me proper credit

%(BEGIN_FRONTMATTER)

\includepdf[pages=-,angle=90]{../output/nogpl/Timeplan.pdf}
\eject
%(END_FRONTMATTER)


\end{document}
