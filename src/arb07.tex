\input preamble.tex
\noindent
{\bf Arbeidsoppdrag 07 - Byggautomasjon}

\vskip 5pt

I dette arbeidsoppdraget skal du lære om komponenter som normalt brukes i et byggautomatiseringsanlegg. Du må også ta i bruk en ukjent PLS. Dette er noe du som automatikker ofte kan komme borti. Denne PLS-en følger standarden EN-60131-3 for programmering på samme måte som Codesys, slik at du vil etterhver finne likheter i oppbygningen, fordi om du ikke finner funksjonene i de samme menyene. 

Kompetansemål:

planlegge, programmere, montere og idriftsette programmerbare styresystemer

endre og tilpasse skjermbilder for grensesnitt mellom menneske og maskin

anvende ulike elektroniske kommunikasjonssystemer i automatiserte anlegg

montere, konfigurere, kalibrere og idriftsettelse digitale og analoge målesystemer

Dette arbeidsoppdraget består at følgende oppdrag:
\begin{enumerate}
	\item IO-sjekk av alle digitale- og analoge IO-er. 
	\item Lage PLS program for sekvensiell oppstart og nedstengning. 
	\item Lage skjermstyring av oppstart til oppstart.
	\item Lærer finne på en oppgave. 
\end{enumerate}
\begin{center}
\begin{tabular}{ | m{8cm} | m{1cm}| m{2cm} | } 
\hline
\multicolumn{3}{|c|}{Liste over oppgaver som skal utføres} \\
	\hline
	Oppgave	& Utført & Signatur \\ 
	\hline
	\hline
	& & \\ 
	\hline
	& & \\ 
	\hline
	& & \\ 
	\hline
	& & \\ 
	\hline
	& & \\ 
	\hline
	& & \\ 
	\hline
\end{tabular}
\end{center}
\underbar{file arb07.tex}

\end{document}

