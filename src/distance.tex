
% Copyright 2006, Tony R. Kuphaldt, released under the Creative Commons Attribution License (v 1.0)
% This means you may do almost anything with this work of mine, so long as you give me proper credit

%(BEGIN_FRONTMATTER)

\centerline{\bf Distance delivery methods}

\vskip 10pt


Sometimes the demands of life prevent students from attending college 6 hours per day.  In such cases, there exist alternatives to the normal 8:00 AM to 3:00 PM class/lab schedule, allowing students to complete coursework in non-traditional ways, at a ``distance'' from the college campus proper.

For such ``distance'' students, the same worksheets, lab activities, exams, and academic standards still apply.  Instead of working in small groups and in teams to complete theory and lab sections, though, students participating in an alternative fashion must do all the work themselves.  Participation via teleconferencing, video- or audio-recorded small-group sessions, and such is encouraged and supported.

There is no recording of hours attended or tardiness for students participating in this manner.  The pace of the course is likewise determined by the ``distance'' student.  Experience has shown that it is a benefit for ``distance'' students to maintain the same pace as their on-campus classmates whenever possible.

In lieu of small-group activities and class discussions, comprehension of the theory portion of each course will be ensured by completing and submitting detailed answers for {\it all} worksheet questions, not just passing daily quizzes as is the standard for conventional students.  The instructor will discuss any incomplete and/or incorrect worksheet answers with the student, and ask that those questions be re-answered by the student to correct any misunderstandings before moving on.

Labwork is perhaps the most difficult portion of the curriculum for a ``distance'' student to complete, since the equipment used in Instrumentation is typically too large and expensive to leave the school lab facility.  ``Distance'' students must find a way to complete the required lab activities, either by arranging time in the school lab facility and/or completing activities on equivalent equipment outside of school (e.g. at their place of employment, if applicable).  Labwork completed outside of school must be validated by a supervisor and/or documented via photograph or videorecording.

\vskip 10pt

Conventional students may opt to switch to ``distance'' mode at any time.  This has proven to be a benefit to students whose lives are disrupted by catastrophic events.  Likewise, ``distance'' students may switch back to conventional mode if and when their schedules permit.  Although the existence of alternative modes of student participation is a great benefit for students with challenging schedules, it requires a greater investment of time and a greater level of self-discipline than the traditional mode where the student attends school for 6 hours every day.  No student should consider the ``distance'' mode of learning a way to have more free time to themselves, because they will actually spend more time engaged in the coursework than if they attend school on a regular schedule.  It exists merely for the sake of those who cannot attend during regular school hours, as an alternative to course withdrawal.


\vfil

\underbar{file {\tt distance}}
\eject
%(END_FRONTMATTER)


