
% Copyright 2015, Tony R. Kuphaldt, released under the Creative Commons Attribution License (v 1.0)
% This means you may do almost anything with this work of mine, so long as you give me proper credit

%(BEGIN_FRONTMATTER)
\centerline{\bf Verdier og forventninger. } \bigskip 
%\centerline{\bf General Values and Expectations} \bigskip 

\noindent
{\bf For å bli en god Automatikker kreves:} faglig integritet,   

%{\bf Success in this career requires:} professional integrity, resourcefulness, persistence, close attention to detail, and intellectual curiosity.  Poor judgment spells disaster in this career, which is why employer background checks (including social media and criminal records) and drug testing are common.  The good news is that character and clear thinking are malleable traits: unlike intelligence, these qualities can be acquired and improved with effort.  {\it This is what you are in school to do} -- increase your ``human capital'' which is the sum of all knowledge, skills, and traits valuable in the marketplace.

\vskip 10pt

\noindent
%\underbar{\bf Mastery:} You must master the fundamentals of your chosen profession.  ``Mastery'' assessments challenge you to demonstrate 100\% competence (with multiple opportunities to re-try).  Failure to complete any mastery objective(s) by the deadline date caps your grade at a C$-$.  Failure to complete by the end of the next school day results in a failing (F) grade.

\vskip 10pt

\noindent
\underbar{\bf Punctuality and Attendance:} You are expected to arrive on time and be ``on-task'' all day just as you would for a job.  Each student has 12 hours of ``sick time'' per quarter applicable to absences not verifiably employment-related, school-related, weather-related, or required by law.  Each student must confer with the instructor to apply these hours to any missed time -- this is not done automatically.  Students may donate unused ``sick time'' to whomever they specifically choose.  You must contact your instructor and lab team members immediately if you know you will be late or absent or must leave early.  Absence on an exam day will result in a zero score for that exam, unless due to a documented emergency.

\vskip 10pt

\noindent
\underbar{\bf Time Management:} You are expected to budget and prioritize your time, just as you will be on the job.  You will need to reserve enough time outside of school to complete homework, and strategically apply your time during school hours toward limited resources (e.g. lab equipment).  Frivolous activities (e.g. games, social networking, internet surfing) are unacceptable when work is unfinished.  Trips to the cafeteria for food or coffee, smoke breaks, etc. must not interfere with team participation.

\vskip 10pt

\noindent
\underbar{\bf Independent Study:} This career is marked by continuous technological development and ongoing change, which is why {\it self-directed learning} is ultimately more important to your future success than specific knowledge.  To acquire and hone this skill, all second-year Instrumentation courses follow an ``inverted'' model where lecture is replaced by independent study, and class time is devoted to addressing your questions and demonstrating your learning.  Most students require a {\it minimum} of 3 hours daily study time outside of school.  Arriving unprepared (e.g. homework incomplete) is unprofessional and counter-productive.  Question 0 of every worksheet lists practical study tips.

\vskip 10pt

\noindent
\underbar{\bf Independent Problem-Solving:} The best instrument technicians are versatile problem-solvers.  General problem-solving is arguably the most valuable skill you can possess for this career, and it can only be built through persistent effort.  This is why you must take every reasonable measure to {\it solve problems on your own} before seeking help.  It is okay to be perplexed by an assignment, but you are expected to apply problem-solving strategies given to you (see Question 0) and to precisely identify where you are confused so your instructor will be able to offer targeted help.  Asking classmates to solve problems for you is folly -- this includes having others break the problem down into simple steps.  The point is to learn how to {\it think on your own}.  When troubleshooting systems in lab you are expected to run diagnostic tests (e.g. using a multimeter instead of visually seeking circuit faults), as well as consult the equipment manual(s) before seeking help.  

\vskip 10pt

\noindent
\underbar{\bf Initiative:} No single habit predicts your success or failure in this career better than personal initiative, which is why your instructor will demand {\it you do for yourself rather than rely on others to do for you.}  Examples include setting up and using your BTC email account to communicate with your instructor(s), consulting manuals for technical information before asking for help, regularly checking the course calendar and assignment deadlines, avoiding procrastination, fixing small problems before they become larger problems, etc.  If you find your performance compromised by poor understanding of prior course subjects, re-read those textbook sections and use the practice materials made available to you on the Socratic Instrumentation website -- don't wait for anyone else to diagnose your need and offer help.







%%%%%%%%%%%%%%% (NEW PAGE %%%%%%%%%%%%%%%
\vfil \eject

\centerline{{\bf General Values and Expectations} (continued)} 


\vskip 10pt

\noindent
\underbar{\bf Safety:} You are expected to work safely in the lab just as you will be on the job.  This includes wearing proper attire (safety glasses and closed-toed shoes in the lab at all times), implementing lock-out/tag-out procedures when working on circuits with exposed conductors over 30 volts, using ladders to access elevated locations, and correctly using all tools.  If you need to use an unfamiliar tool, see the instructor for directions.

\vskip 10pt

\noindent
\underbar{\bf Orderliness:} You are expected to keep your work area clean and orderly just as you will be on the job.  This includes discarding trash and returning tools at the end of every lab session, and participating in all scheduled lab clean-up sessions.  If you identify failed equipment in the lab, label that equipment with a detailed description of its symptoms. 

\vskip 10pt

\noindent
\underbar{\bf Teamwork:} You will work in instructor-assigned teams to complete lab assignments, just as you will work in teams to complete complex assignments on the job.  As part of a team, you must keep your teammates informed of your whereabouts in the event you must step away from the lab or will be absent for {\it any} reason.  Any student regularly compromising team performance through lack of participation, absence, tardiness, disrespect, or other disruptive behavior(s) will be removed from the team and required to complete all labwork individually for the remainder of the quarter.  The same is true for students found relying on teammates to do their work for them.

\vskip 10pt

\noindent
\underbar{\bf Cooperation:} The structure of these courses naturally lends itself to cooperation between students.  Working together, students significantly impact each others' learning.  You are expected to take this role seriously, offering real help when needed and not absolving classmates of their responsibility to think for themselves or to do their own work.  Solving problems for classmates and/or explaining to them what they can easily read on their own is unacceptable because these actions circumvent learning.  The best form of help you can give to your struggling classmates is to share with them your tips on independent learning and problem-solving, for example {\it asking questions} leading to solutions rather than simply providing solutions for them.

\vskip 10pt

%\noindent
%\underbar{\bf Academic Engagement:} Instrumentation is a challenging career requiring creative and critical thinking.  As industry advisors have said, ``Being an instrument technician is as close as you get to doing engineering without a four-year (Bachelor's) degree.''  The only way to prepare for the challenges of being an instrument technician is to exercise that same level of creative and critical thinking before stepping into the career, mastering {\it first principles} of science and {\it general problem-solving} strategies rather than focusing on simpler tasks such as memorization and procedures.  This also means personally involving yourself in every learning exercise, not being content to merely observe others.  Individual (unassisted) performance is the gold standard for learning: {\it unless and until \underbar{you} can consistently perform on your own, you haven't learned!}

%\vskip 10pt

\noindent
\underbar{\bf Grades:} Employers prize trustworthy, hard working, knowledgeable, resourceful problem-solvers.  The grade you receive in any course is but a {\it partial} measure of these traits.  What matters most are the traits themselves, which is why your instructor maintains detailed student records (including individual exam scores, attendance, tardiness, and behavioral comments) and will share these records with employers if you have signed the FERPA release form.  You are welcome to see your records at any time, and to compare calculated grades with your own records (i.e. the grade spreadsheet available to all students).  You should expect employers to scrutinize your records on attendance and character, and also challenge you with technical questions when considering you for employment.

\vskip 10pt

\noindent
\underbar{\bf Representation:} You are an ambassador for this program.  Your actions, whether on tours, during a jobshadow or internship, or while employed, can open or shut doors of opportunity for other students.  Most of the job opportunities open to you as a BTC graduate were earned by the good work of previous graduates, and as such you owe them a debt of gratitude.  Future graduates depend on you to do the same.

\vskip 10pt

\noindent
\underbar{\bf Responsibility For Actions:} If you lose or damage college property (e.g. lab equipment), you must find, repair, or help replace it.  If you represent BTC poorly to employers (e.g. during a tour or an internship), you must make amends.  The general rule here is this: {\it ``If you break it, you fix it!''}  

\vskip 10pt

\noindent
\underbar{\bf Non-negotiable terms:} disciplinary action, up to and including immediate failure of a course, will result from academic dishonesty (e.g. cheating, plagiarism), willful safety violations, theft, harassment, intoxication, destruction of property, or willful disruption of the learning (work) environment.  Such offenses are grounds for immediate termination in this career, and as such will not be tolerated here.

%\vskip 10pt

%\noindent
%\underbar{\bf Disciplinary action:} The Student Code of Conduct for students at Bellingham Technical College (Washington Administrative Codes {\tt WAC 495B-120}) explicitly authorizes disciplinary action against misconduct including: academic dishonesty (e.g. cheating, plagiarism), dangerous or lewd behavior, theft, harassment, intoxication, destruction of property, or disruption of the learning environment.  

\vfil

\underbar{file {\tt expectations}}
\eject
%(END_FRONTMATTER)

