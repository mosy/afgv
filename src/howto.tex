
% Copyright 2015, Tony R. Kuphaldt, released under the Creative Commons Attribution License (v 1.0)
% This means you may do almost anything with this work of mine, so long as you give me proper credit

%(BEGIN_FRONTMATTER)
\centerline{\bf Hvordan . . .} \bigskip 

\noindent
{\bf Få tak i oppgaver og lærebok} gå inn i fagbibliotek mappen som er delt med deg på OneDrive, der vil du finne oppgavene i mappen "Oppgaver" og læreboken vil du finne i mappen "Teori" 


\vskip 10pt

\noindent
{\bf Hvordan lære mest mulig:} kom til skolen forberedt hver eneste dag -- dette betyr at du har gjort alle lekser gitt til/etter en leksjon. Fulgt alle tips som gis på oppgaveark og av lærer. Ikke spør andre om hjelp før du har gjort en rimlig innsats selv. Hjelp andre med å gjennomføre oppgave og å forstå, men ikke gjør jobben for DE. 

%{\bf Maximize your learning:} come to school prepared each and every day -- this means completing all your homework {\it before} class starts.  Use every minute of class and lab time productively.  Follow all the tips outlined in ``Question 0'' (in every course worksheet) as well as your instructor's advice.  Don't ask anyone to help you solve a problem until you have made every reasonable effort to solve it on your own.

\vskip 10pt

\noindent
{\bf Holde orden på innleveringer og frister.} Følg med på beskjeder gitt av lærer. I arbeidslivet vil du få muntlige beskjeder som det forventes at du følger opp, slik er det her også. Er du vekke fra skolen må du orientere eg med medelever. 
%{\bf Identify upcoming assignments and deadlines:} read the first page of each course worksheet.

\vskip 10pt

\noindent
%{\bf Relate course days to calendar dates:} reference the calendar spreadsheet file ({\tt calendar.xlsx}), found on the BTC campus {\tt Y:} network drive.  A printed copy is posted in the Instrumentation classroom.

%\vskip 10pt

\noindent
{\bf Finne fagstoff og manualer fra produsenter gitt som leselekse:} I fagbibliotekmappen som er delt med deg på OneDrive/Teams vil du finne mappene "Fagstoff" og  "Manualer" 


%{\bf Locate industry documents assigned for reading:} use the Instrumentation Reference provided by your instructor (on CD-ROM and on the BTC campus {\tt Y:} network drive).  There you will find a file named {\tt 00\_index\_OPEN\_THIS\_FILE.html} readable with any internet browser.  Click on the ``Quick-Start Links'' to access assigned reading documents, organized per course, in the order they are assigned.

\vskip 10pt

\noindent
%{\bf Study for the exams:} \underbar{Mastery exams} assess specific skills critically important to your success, listed near the top of the front page of each course worksheet for your review.  Familiarize yourself with this list and pay close attention when those topics appear in homework and practice problems.  \underbar{Proportional exams} feature problems you haven't seen before that are solvable using general principles learned throughout the current and previous courses, for which the only adequate preparation is independent problem-solving practice every day.  Answer the ``feedback questions'' (practice exams) in each course section to hone your problem-solving skills, as these are similar in scope and complexity to proportional exams.  Answer these feedback independently (i.e. no help from classmates) in order to most accurately assess your readiness.

\vskip 10pt

\noindent
%{\bf Calculate course grades:} download the ``Course Grading Spreadsheet'' ({\tt grades\_template.xlsx}) from the Socratic Instrumentation website, or from the BTC campus {\tt Y:} network drive.  Enter your quiz scores, test scores, lab scores, and attendance data into this Excel spreadsheet and it will calculate your course grade.  You may compare your calculated grades against your instructors' records at any time.

\vskip 10pt

\noindent
%{\bf Identify courses to register for:} read the ``Sequence'' page found in each worksheet.

\vskip 10pt

\noindent
%{\bf Identify scholarship opportunities:} check your BTC email in-box daily.

\vskip 10pt

\noindent
{\bf Identify job openings:} regularly monitor job-search websites.  Set up informational interviews at workplaces you are interested in.  Participate in jobshadows and internships.  Apply to jobs long before graduation, as some employers take {\it months} to respond!  Check your BTC email account daily, because your instructor broadcast-emails job postings to all students as employers submit them to BTC.

\vskip 10pt

\noindent
{\bf Impress employers:} sign the FERPA release form granting your instructors permission to share academic records, then make sure your performance is worth sharing.  Document your project and problem-solving experiences for reference during interviews.  Honor \underbar{all} your commitments.

\vskip 10pt

\noindent
{\bf Begin your career:} participate in jobshadows and internships while in school to gain experience and references.  Take the first Instrumentation job that pays the bills, and give that employer at least two years of good work to pay them back for the investment they have made in you.  Employers look at delayed employment, as well as short employment spans, very negatively.  Failure to pass a drug test is an immediate disqualifier, as is falsifying any information.  Criminal records may also be a problem.








\vfil

\underbar{file {\tt howto}}
\eject
%(END_FRONTMATTER)

