
% Copyright 2015, Tony R. Kuphaldt, released under the Creative Commons Attribution License (v 1.0)
% This means you may do almost anything with this work of mine, so long as you give me proper credit

%(BEGIN_FRONTMATTER)
\centerline{\bf Hvordan . . .} \bigskip 

\noindent
{\bf Finne fagplan for VG3 Automatiseringsfaget:} \url{https://www.udir.no/lk20/aut03-04}
\vskip 10pt

\noindent
{\bf Finner jeg skoleruten:} \url{https://www.gand.vgs.no/hovedmeny/for-elever/skolehverdag/skoleferier-og-fridager-skoleruta/}
\vskip 10pt

\noindent
{\bf Finner jeg informasjon om Skolen:} \url{https://www.gand.vgs.no/hovedmeny/for-elever/}
\vskip 10pt

\noindent
{\bf Finner jeg timeplanen:} Timeplanen din finner du i VIS. 
\vskip 10pt

\noindent
{\bf Få penger til utstyr som kreves:} Alle elever med ungdomsrett i videregående opplæring får utstyrsstipend. Utstyrsstipendet er ikke avhengig av hvor mye foreldrene tjener.
Søknaden skrives til lånekassen \url{http://www.lanekassen.no}
\vskip 10pt

\noindent
{\bf Få tak i oppgaver og lærebok} På siden autofaget.no legges oppgaver og teori ut. Noe vil også bli levert på papir eller lagt ut på Teams. 
\vskip 10pt

\noindent
{\bf Få tak i  software og biblioteker:} Software som vi bruker legger jeg ut i ukesplanen på autofaget.no  
\vskip 10pt

\noindent
{\bf Hvordan lære mest mulig:} kom til skolen forberedt hver eneste dag -- dette betyr at du har gjort alle lekser gitt til/etter en leksjon. Fulgt alle tips som gis på oppgaveark og av lærer. Ikke spør andre om hjelp før du har gjort en rimlig innsats selv. Hjelp andre med å gjennomføre oppgave og å forstå, men ikke gjør jobben for DE. 
\vskip 10pt

\noindent
{\bf Holde orden på innleveringer og frister.} Følg med på beskjeder gitt av lærer. I arbeidslivet vil du få muntlige beskjeder som det forventes at du følger opp, slik er det her også. Er du vekke fra skolen må du orientere eg med medelever. Se også autofaget.no
\vskip 10pt


\noindent
{\bf Finne fagstoff og manualer fra produsenter } Det forventes at du kan søke dette opp på internet selv. 
\vskip 10pt



\noindent
{\bf Få seg læreplass:} Ca. i desember starter de første firmaene å legge annonser for nye lærlinger (Søk på så mange du klarer å håndtere ca. 50), når du er i utplassering må du vise deg som en attraktiv arbeidstager, sørg for å være faglig interessert/på, ha minst mulig fravær og følgmed på skolemailen din her videresender jeg forespørsler fra firmaer. 
\vskip 10pt


\vfil

\underbar{file {\tt hvordan}}
\eject
%(END_FRONTMATTER)

