%(BEGIN_QUESTION)
% Copyright 2015, Tony R. Kuphaldt, released under the Creative Commons Attribution License (v 1.0)
% This means you may do almost anything with this work of mine, so long as you give me proper credit

Use a computer to navigate to the ``Socratic Instrumentation'' website:

$$\hbox{\tt http://www.ibiblio.org/kuphaldt/socratic/sinst}$$

When you get there, click on the link for the quarter (Summer, Fall, Winter, or Spring) you are enrolled in, and download the INST200 ``Introduction to Instrumentation'' course worksheet.  Today's classroom session will cover Day 1 of this worksheet.

\vskip 10pt

Near the very beginning of this document, as is the case for {\it all} the 200-level Instrumentation course worksheets, you will find a page titled ``How To . . .''.  Locate this page and read it thoroughly, as you will be quizzed on its contents throughout the INST200 course.  The ``How to . . .'' tips make reference to a ``Question 0'' which is another page found in every course worksheet.  Read the points listed in Question 0 as well.

%\vskip 10pt

%When the class is finished with the reading, we will review and discuss items in the ``How To . . .'' and ``Question 0'' pages, relating them to the points brainstormed earlier regarding your own professional goals in becoming an instrument technician and the expectations of employers.  This exercise also serves as an introduction showing how all class work is handled in the second year of the Instrumentation program: you complete reading and research assignments prior to each class session, and then you spend nearly the entire class session discussing what you've learned while the instructor challenges you to explore the concepts deeper.  Later in this course you will be quizzed on various details contained in this ``How To . . .'' page.  Feel free to keep a printed copy of this page handy for future INST200 course class sessions, as the quizzes will be open-note!

\vskip 10pt

Your instructor will also hand out copies of a release form (``FERPA form'') which you may sign to grant permission to share your academic performance records with employers.  This is voluntary, not mandatory.  Without signed consent from student, federal law prohibits any instructor from sharing academic records with anyone but the student and appropriate college employees.

\vskip 10pt

Your instructor will also have electronic copies (e.g. flash drive and/or CD-ROM) of the ``Instrumentation Reference'' on hand for you to copy to your personal computer.  This is a collection of files, mostly obtained from various manufacturers' websites with their permission, of tutorials and reports and technical manuals which you will be assigned to read throughout the second-year courses.  The purpose of this Reference is to provide you with fast, off-line access so that you need not search the internet for these assigned documents.  There is a file in the root directory of this Reference named ``{\tt 00\_index\_OPEN\_THIS\_FILE.html}'' you should open using a web browser.  The hyperlinks within this HTML index file make it much easier to find the document(s) you're looking for than it would be scanning the various directories within the Reference to peruse filenames.


\vskip 20pt \vbox{\hrule \hbox{\strut \vrule{} {\bf Suggestions for Socratic discussion} \vrule} \hrule}

\begin{itemize}
\item{} One of the purposes of this exercise is to practice active reading strategies, where you interact with the text to identify and explore important principles.  An effective strategy is to write any thoughts that come to mind as you are reading the text.  Describe how this active reading strategy might be useful in daily homework assignments.
\item{} For each and every one of the points listed in the ``How To . . .'' and ``Question 0'' pages, identify why these points are important to your ultimate goal of becoming an instrument technician.
\item{} Identify how the INST200-level course design and expectations differ from what you have experienced in the past as students, and explain why these differences exist.
\end{itemize}

\underbar{file i00002}
%(END_QUESTION)




%(BEGIN_ANSWER)


%(END_ANSWER)





%(BEGIN_NOTES)

{\it You should print copies of the ``How To'' and ``Question 0'' pages (on one double-sided sheet of paper) and the ``General Values and Expectations'' on another sheet of paper for each and every student to have at their desk, in case they don't have a personal computer with them this day.}

\vskip 10pt

The ``Real examples'' showcase actual questions and scenarios posed to instructors in the Instrumentation program, which may serve as starting points for whole-class discussions on how to apply principles listed on the ``How To'' page.  While many of these statements are actually amalgams of various statements made by different students at different times, the general theme and tone of each is faithful to the original.

\vskip 10pt

Be prepared to augment these examples with live demonstrations on the lectern computer (e.g. have the grades spreadsheet ready to present, have the INSTREF ready to share, etc.).









\vfil \eject
\noindent
{\bf Real example:} 

The day before the exam, a student approaches the instructor and asks, ``What's going to be on tomorrow's exam?''







\vfil \eject
\noindent
{\bf Real example:} 

A student calls the instructor over for help.  They are trying to locate one of the instruction manuals for a particular instrument, that's been assigned for reading.  The instructor finds the student trying to navigate through the different file folders of the Instrumentation Reference, looking for a filename that is anything close to the brand and model of instrument cited in the homework.  ``It's so hard to find anything in this Instrumentation Reference,'' the student says.  ``Isn't there an easier way?''







\vfil \eject
\noindent
{\bf Real example:} 

At the start of a new quarter, one of the continuing second-year students is found by the instructor to have registered for a couple of incorrect courses.  When the instructor approaches this student to suggest they drop the incorrect courses, the student says ``The person at Registration told me these were the courses offered this quarter, so I signed up for all of them.''







\vfil \eject
\noindent
{\bf Real example:} 

A student approaches the instructor mid-way through the academic quarter.  ``I wish you would post grades to all the students electronically like my other instructors, showing us exactly where we stand.  I have no idea what my GPA is right now, and I'm worried about being able to qualify for an upcoming scholarship.''









\vfil \eject
\noindent
{\bf Real example:} 

A graduate who has been out of school for 6 months emails the instructor to ask about jobs.  ``I'm looking for some contract work preferably in southern California, and don't know where to begin my search.''  When asked by the instructor what jobs they have applied to so far, the response is ``None -- I'm just starting.''









\vfil \eject
\noindent
{\bf Real example:} 

During an interview a graduate is asked to describe a situation where they had to solve a set of complex problems to achieve a certain goal.  The graduate is caught off guard, not knowing what the interviewer wants to hear, and also feels a bit intimidated by the question because as of yet they have no industrial work experience.  Should he fabricate a story that sounds good?  Should he think of a hypothetical scenario where he can explain what he {\it would} do if that situation presented itself?  Should he relate an experience he had while working on a project in school, even though that wasn't an actual job?









\vfil \eject
\noindent
{\bf Real example:} 

A student comes to class one day having prepared for the wrong set of homework questions.  When they realize this, they tell the instructor about it.  ``I asked my lab teammates what questions we had to do in preparation for today and they gave me the wrong information!''










\vfil \eject
\noindent
{\bf Real example:} 

A student approaches the instructor with a question: ``My parents have a vacation scheduled the last week in October, and I was planning to go along with them.  Will I miss anything important if I'm gone for that week?''








%INDEX% Course organization, career advice
%INDEX% Course organization, course calendar spreadsheet
%INDEX% Course organization, course grading spreadsheet
%INDEX% Course organization, introduce ``How To...'' page
%INDEX% Course organization, Instrumentation Reference
%INDEX% Course organization, job searching
%INDEX% Course organization, second-year course sequence
%INDEX% Course organization, Socratic Instrumentation website
%INDEX% Course organization, student email

%(END_NOTES)


