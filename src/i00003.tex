
%(BEGIN_QUESTION)
% Copyright 2015, Tony R. Kuphaldt, released under the Creative Commons Attribution License (v 1.0)
% This means you may do almost anything with this work of mine, so long as you give me proper credit

Near the beginning of every course worksheet there are some pages titled ``General Values and Expectations''.  Your instructor will read through all these expectations with you and answer any questions you have about them.  Feel free to read this document in advance and bring questions with you to class for answering.  These expectations reference ``Question 0'' which is also found in every course worksheet, and which you will want to read through as well.

%\vskip 10pt

%We will practice a proven reading technique for boosting comprehension (listed in Question 0 for future reference), which entails {\bf summarizing the text in your own words at a suggested ratio of one sentence of your own thoughts per paragraph of text}.  This exercise will not only familiarize you with the basic expectations for the second-year Instrumentation courses, but also model a sound learning strategy you will find helpful throughout the rest of your education and career.

%After the class is finished summarizing the reading, we will discuss the rationale for these expectations: how they foster students' success in their chosen career of industrial instrumentation.  Later in this course you will be quizzed on various program policies referenced in these ``General Values and Expectations'' pages.  Feel free to keep a printed copy handy for future INST200 course class sessions, as the quizzes will be open-note!

\vskip 20pt \vbox{\hrule \hbox{\strut \vrule{} {\bf Suggestions for Socratic discussion} \vrule} \hrule}

\begin{itemize}
\item{} For each and every one of the points listed in the ``General Expectations'' pages, identify why these points are important to your ultimate goal of becoming an instrument technician.
\item{} Identify how the INST200-level course design and expectations differ from what you have experienced in the past as students, and explain why these differences exist.
\item{} One of the purposes of this exercise is to practice active reading strategies, where you interact with the text to identify and explore important principles.  An effective strategy is to write any thoughts that come to mind as you are reading the text.  Describe how this active reading strategy might be useful in daily homework assignments.
\end{itemize}

\underbar{file i00003}
%(END_QUESTION)





%(BEGIN_ANSWER)


%(END_ANSWER)





%(BEGIN_NOTES)

{\it You should print copies of the ``How To'' and ``Question 0'' pages (on one double-sided sheet of paper) and the ``General Values and Expectations'' on another sheet of paper for each and every student to have at their desk, in case they don't have a personal computer with them this day.}

\vskip 10pt

Begin the discussion of how to summarize paragraphs of text in your own words, by modeling this for your students on at least one paragraph from the ``General Values and Expectations'' pages.  Only after you do this with students, and there is time left, should you go on to posing ``real examples'' for them to apply the Expectations to.

\vskip 10pt

The ``Real examples'' showcase actual questions and scenarios posed to instructors in the Instrumentation program, which may serve as starting points for whole-class discussions on how to apply principles listed on the ``General Values and Expectations'' pages.  While many of these statements are actually amalgams of various statements made by different students at different times, the general theme and tone of each is faithful to the original.

\vskip 10pt

It is a good idea to share with your students some of the Socratic Questions in the instructor version of Question 0 which you will be regularly asking students to answer following their reading assignments throughout the program.










\vfil \eject
\noindent
{\bf Real example:} 

A student approaches the instructor, worried about the homework load.  ``My time for study outside of school is really limited.  What can I do to make more study time?''








\vfil \eject
\noindent
{\bf Real example:} 

A student approaches the instructor, concerned about the daily homework.  ``It's not often that I'm able to completely answer all of the questions assigned for the day.  I want to finish all the homework in order to be prepared for class, but what should I do if I get completely stuck on a homework problem?''








\vfil \eject
\noindent
{\bf Real examples:} 

A recent graduate of the program emails his instructor with a suggestion.  ``At my job I've had to rebuild over a half-dozen different {\it brands} of control valve, not just the Fisher brand we learned in school.  You really need to teach more control valve brands than this in order to prepare us for the job.''

\vskip 20pt

A guest speaker is invited to lead the class through some hands-on exercises with a particular piece of instrumentation equipment.  To ensure the exercise goes smoothly, this guest brings sets of printed instructions for each student to follow, guiding the students step-by-step through all the procedures.  At the end of this 3-hour session, a couple of students approach the instructor and say ``This is how {\it all} of the labs should be organized!  Everything was so easy to understand!  It wasn't as frustrating as building projects like we usually do.''









\vfil \eject
\noindent
{\bf Real examples:} 

A student arrives 5 minutes late, with a speeding ticket in his hand.  ``I got here as fast as I could, but there was construction on Bakerview road and when I tried to make up lost time a cop pulled me over and gave me this ticket.  Am I still tardy?''

\vskip 20pt

A student arrives 2 minutes late to class.  ``According to my watch, I'm still on time!''

\vskip 20pt

A student exhibits a pattern of showing up late for class or not showing up at all, with no contact whatsoever with his lab teammates or the instructor.  When asked about this the next day by the instructor, his reply is ``I had important business to take care of.''

\vskip 20pt

A student has a habit of immediately going to the cafeteria to get breakfast and coffee after the instructor concludes the morning introduction to lab at 8:15 AM.  The student's teammates must wait until he returns to progress on labwork together.










%\vskip 10pt

%The {\it Think Aloud} technique is described in the second edition of {\it Reading For Understanding -- How Reading Apprenticeship Improves Disciplinary Learning in Secondary and College Classrooms} by Ruth Schoenbach, Cynthia Greenleaf, and Lynn Murphy.  

%\vskip 10pt

%``Think Aloud'' is introduced in detail on pages 101-107 of this book.  A silent variant of this technique called {\it Talk To The Text} is described on page 106, where students write their metacognitive musings rather than speak their thoughts aloud.  Even more formalized variations on the theme of articulating one's own thoughts when reading are described on pages 110-118 (``Metacognitive Journaling'').





%INDEX% Course organization, expectations
%INDEX% Reading Apprenticeship technique, Think Aloud

%(END_NOTES)


