
%(BEGIN_QUESTION)
% Copyright 2010, Tony R. Kuphaldt, released under the Creative Commons Attribution License (v 1.0)
% This means you may do almost anything with this work of mine, so long as you give me proper credit

Read ``The Lecture System In Teaching Science'' by Robert T. Morrison, an article from the Journal {\it Undergraduate Education In Chemistry and Physics}, October 18-19, 1985, pages 50 through 58.  This article is available in electronic form from the BTC campus library, as well as on the Internet (easily found by performing a search).  In it, Morrison outlines a teaching method referred to as the ``Gutenberg Method.''

\vskip 20pt

How is the Gutenberg Method as described by Morrison similar to the classroom structure in these Instrumentation courses?

\vskip 100pt

Identify in your own words at least two advantages the Gutenberg Method enjoys over standard lectures.

\vskip 100pt

Explain how a person educated in this way might be better prepared for continuing education in the workplace, compared to those who learned by lecture while in school.

\vfil

\underbar{file i00008}
\eject
%(END_QUESTION)





%(BEGIN_ANSWER)

This is a graded question -- no answers or hints given!

%(END_ANSWER)





%(BEGIN_NOTES)


%INDEX% Course organization, philosophy: preparation for class sessions
%INDEX% Reading assignment: "The Lecture System In Teaching Science"

%(END_NOTES)


