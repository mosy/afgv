
%(BEGIN_QUESTION)
% Copyright 2006, Tony R. Kuphaldt, released under the Creative Commons Attribution License (v 1.0)
% This means you may do almost anything with this work of mine, so long as you give me proper credit

A group of mechanics are trying to figure out a solution to a problem.  They are trying to remove the lid from a large metal vessel, but the lid is stuck and will not come off.  Several of the mechanics try to use pry-bars to lift the lid, but to no avail.  Others try to heat the lid with oxygen-acetylene torches and then pry while it's hot, but this does not budge the lid either.

Finally, one of the mechanics decides to plug all the pipe holes exiting this vessel except for one, then connect a water supply hose to that last pipe hole and use water pressure to force the lid off.  After doing this, the lid comes off quite easily.

\vskip 10pt

Explain why the last mechanic's solution worked, addressing the following points in your explanation:

\begin{itemize}
\item{} How much force will a fluid such as water exert on a surface, given a certain fluid pressure?
\item{} Why was it prudent for the mechanic to use pressurized {\it water} and not compressed {\it air} to force the lid off?
\end{itemize}

\vskip 20pt \vbox{\hrule \hbox{\strut \vrule{} {\bf Suggestions for Socratic discussion} \vrule} \hrule}

\begin{itemize}
\item{} Identify which fundamental principles of science, technology, and/or math apply to the mechanics' solution to this problem.  In other words, be prepared to explain the reason(s) ``why'' rather than merely describing those steps.
\end{itemize}

\underbar{file i00018}
%(END_QUESTION)





%(BEGIN_ANSWER)

I'll let you figure this out on your own!

%(END_ANSWER)





%(BEGIN_NOTES)

Pressure is the distribution of a normal force per unit area.

\vskip 10pt

Force is equal to pressure times area: $F = PA$

\vskip 10pt

The use of air instead of water would be quite unwise due to the {\it compressibility} of a gas versus the (nearly) incompressible nature of a liquid.  For an illustration, try puncturing a balloon filled with air versus a balloon filled with water!  Air is ``springy'' and will expand once let loose.  Water has no such ``springiness'' and therefore does not explosively expand once the container ruptures.

\vskip 10pt

Water pressure sometimes comes from altitude, where the water source is higher than the point of use.  It often comes from (or is augmented by) pumps, though.

%INDEX% Physics, static fluids: water pressure used to lift lid off vessel

%(END_NOTES)


