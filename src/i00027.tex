
%(BEGIN_QUESTION)
% Copyright 2010, Tony R. Kuphaldt, released under the Creative Commons Attribution License (v 1.0)
% This means you may do almost anything with this work of mine, so long as you give me proper credit

\noindent
{\bf Lab Exercise}

\vskip 5pt

Your team's task is to build a complete, working process controlled by a networked digital controller or distributive control system.  Options include single-loop controllers connected to a workstation via network cabling, networked PLCs implementing PID control, DDC (building automation control) units, or a DCS.  When you operate your system, you must demonstrate operator control (e.g. setpoint changes, auto/manual mode, etc.) exercised remotely through a network computer connection to your process controller.

Each instrument in the loop should be labeled with a proper tag name (e.g. ``LT-59'' for a level transmitter), with all instruments in each loop sharing the same loop number.  Write on pieces of masking tape to make simple labels for all the instruments and signal lines.

After constructing the entire control loop, you will ``tune'' the P, I, and D settings of your controller so that it exhibits good control behavior (defined here as quarter-wave damping or better).

\vskip 10pt

\underbar{Objective completion table:}

% No blank lines allowed between lines of an \halign structure!
% I use comments (%) instead, so that TeX doesn't choke.

$$\vbox{\offinterlineskip
\halign{\strut
\vrule \quad\hfil # \ \hfil & 
\vrule \quad\hfil # \ \hfil & 
\vrule \quad\hfil # \ \hfil & 
\vrule \quad\hfil # \ \hfil & 
\vrule \quad\hfil # \ \hfil & 
\vrule \quad\hfil # \ \hfil & 
\vrule \quad\hfil # \ \hfil \vrule \cr
\noalign{\hrule}
%
% First row
{\bf Performance objective} & {\bf Grading} & {\bf 1} & {\bf 2} & {\bf 3} & {\bf 4} & {\bf Team} \cr
%
\noalign{\hrule}
%
% Another row
Choose process to build & mastery & -- & -- & -- & -- & \cr
%
\noalign{\hrule}
%
% Another row
Locate and navigate manual for controller & mastery & -- & -- & -- & -- & \cr
%
\noalign{\hrule}
%
% Another row
Loop diagram and process inspection & mastery & & & & & -- -- -- -- \cr
%
\noalign{\hrule}
%
% Another row
Process exhibits good control behavior & mastery & -- & -- & -- & -- &  \cr
%
\noalign{\hrule}
%
% Another row
Lab question: Selection/testing & proportional &  &  &  &  & -- -- -- -- \cr
%
\noalign{\hrule}
%
% Another row
Lab question: Commissioning & proportional &  &  &  &  & -- -- -- -- \cr
%
\noalign{\hrule}
%
% Another row
Lab question: Mental math & proportional &  &  &  &  & -- -- -- -- \cr
%
\noalign{\hrule}
%
% Another row
Lab question: Diagnostics & proportional &  &  &  &  & -- -- -- -- \cr
%
\noalign{\hrule}
} % End of \halign 
}$$ % End of \vbox

Each student will be asked to correctly answer a ``lab question'' from each of the four categories (examples shown on the next page).  These lab questions serve as a guide to knowledge and skills all team members should be learning as they progress through the lab exercise.  The instructor may quiz students on these questions at any appropriate time before the lab exercise is complete.

\vfil \eject

\noindent
\underbar{Lab Questions} 

\begin{itemize}
\item{} {\bf Selection and Initial Testing}
\item{} Identify all inputs and outputs on the field instruments (transmitter and FCE)
\item{} Explain the meanings of the various ratings specified on the instrument nameplate
\item{} Identify in the manufacturer documentation where to connect signal wires to the field instrument (transmitter or FCE)
\item{} Explain what types of test equipment were used to validate the operation of the field instrument (transmitter or FCE)
\item{} Explain how you could perform rudimentary tests of instrument function using simple test equipment (multimeter, air pumps, pressure gauges, resistors, batteries, etc.)
\end{itemize}

\filbreak

\begin{itemize}
\item{} {\bf Commissioning and Documentation}
\item{} Demonstrate how to isolate potentially hazardous energy in your system ({\it lock-out, tag-out}) and also how to safely verify the energy has been isolated prior to commencing work on the system
\item{} Demonstrate how to use a loop calibrator to measure/source/simulate signal current
\item{} Demonstrate proper wrench selection and use on nuts, bolts, and/or tube fittings
\item{} Identify the network connection(s) used to remotely access controller variables from a PC (e.g. IP addresses, Ethernet connections, Modbus networks and addresses, etc.)
\item{} Identify multiple locations (referencing a loop diagram) you may measure various 4-20 mA instrument signals in the system
\item{} Identify multiple locations (referencing a loop diagram) you may connect HART communicator in the system
\end{itemize}

\filbreak

\begin{itemize}
\item{} {\bf Mental math} (no calculator allowed!)
\item{} Determine allowable calibration error of instrument (e.g. +/- 0.5\% for an instrument ranged 200 to 500 degrees)
\item{} Convert 4-20 mA signal into a percentage of span (e.g. 13 mA = \underbar{\hskip 20pt}\%)
\item{} Convert percentage of span into a 4-20 mA signal value (e.g. 70\% = \underbar{\hskip 20pt} mA)
\item{} Convert 3-15 PSI signal into a percentage of span (e.g. 11 PSI = \underbar{\hskip 20pt}\%)
\item{} Convert percentage of span into a 3-15 PSI signal value (e.g. 40\% = \underbar{\hskip 20pt} PSI)
\end{itemize}

\filbreak

\begin{itemize}
\item{} {\bf Diagnostics}
\item{} Given a particular component or wiring fault ({\it instructor specifies type and location}), what symptoms would the loop exhibit and why?
\item{} Explain why breaking a 4-20 mA loop could cause serious problems in an actual instrument loop!
\item{} Explain what will happen (and why) in your control loop if the transmitter suddenly fails with a low (4 mA) signal.  Assume the controller is in automatic mode when this happens.
\item{} Explain what will happen (and why) in your control loop if the transmitter suddenly fails with a high (20 mA) signal.  Assume the controller is in automatic mode when this happens.
\item{} Explain what will happen (and why) in your control loop if the FCE suddenly fails with the equivalent of a low (4 mA) MV signal.
\item{} Explain what will happen (and why) in your control loop if the FCE suddenly fails with the equivalent of a high (20 mA) MV signal.
\item{} Explain how to confirm the presence of an {\it open} in a 4-20 mA signal cable using only a voltmeter (no resistance or current measurement allowed!). 
\item{} Explain how to confirm the presence of a {\it short} in a 4-20 mA signal cable using only a voltmeter (no resistance or current measurement allowed!).  Hint: you will need to break the circuit. 
\end{itemize}




\vfil \eject

\centerline{\bf A crude closed-loop PID tuning procedure}

\vskip 10pt

Tuning a PID controller is something of an art, and can be quite daunting to the novice.  What follows is a primitive (oversimplified for some situations!) procedure you can apply to many processes.

\vskip 10pt

\noindent
{\bf Step 1}

\underbar{Understand the process you are trying to control.}  If you do not have a fundamental grasp on the nature of the process you're controlling, it is pointless -- even dangerous -- to change controller settings.  Here is a simple checklist to cover before touching the controller:

\begin{itemize}
\item{} What is the process variable and how is it measured?
\item{} What is the final control element, and how does it exert control over the process variable?
\item{} What safety hazards exist in this process related to control (e.g. danger of explosion, solidification, production of dangerous byproducts, etc.)?  
\item{} What is the control scheme being used?  Is this a simple, stand-alone PID control algorithm, or does it connect to other control algorithms or special function blocks?  If this is not a simple algorithm, be sure you know how it is supposed to respond under all conceivable process conditions.
\item{} How far am I allowed to ``bump'' the process while I tune the controller and monitor the response?
\item{} How is the controller mode switched to ``manual,'' just in case I need to take over control?
\item{} In the event of a dangerous condition caused by the controller, how do you shut the process down?
\end{itemize}

\vskip 10pt

\noindent
{\bf Step 2}

\underbar{Understand what the settings on the controller do.}  Is your controller configured for gain or proportional band?  Minutes per repeat or repeats per minute?  Does it use reset windup limits?  Does rate respond to error or PV alone?  You had better understand what the PID values do to the controller's action if you are going to decide which way (and how much) to adjust them!  Back in the days of analog electronic and pneumatic controllers, I would recommend to technicians that they draw little arrow symbols next to each adjustment knob showing which way to turn for more aggressive action -- this way they wouldn't get mixed up figuring out gain vs PB, rep/min vs min/rep, etc.: all they had to think of is ``more'' or ``less'' of each action.

\vskip 10pt

\noindent
{\bf Step 3}

\underbar{Manually ``bump'' the manipulated variable (final control element) to learn how the process responds.}  Right now, {\it you} are the controller!  What you need to do is adjust the process to learn how it responds: is it an integrating process, a self-regulating process, or a runaway process?  Is there significant dead time or hysteresis?  Is the response linear and consistent?  Many process control problems are caused by factors other than the controller, and this ``manual test'' step is a key diagnostic technique for assessing these other factors.

\vskip 10pt

\noindent
{\bf Step 4}

\underbar{Set the PID constants to ``minimal'' settings and switch to automatic mode.}  This means gain less than 1, no integral action (0 rep/min or maximum min/rep), no derivative action, and no filtering.

\vskip 10pt

\noindent
{\bf Step 5}

\underbar{``Bump'' the setpoint and watch the controller's response.}  This tests the controller's ability to manage the process on its own.  What you want is a response that is reasonably fast without overshooting or undershooting too much, and without undue cycling.  The nature of the process and the constraints of quality standards will dictate what is ``too much'' response time, over/undershoot, and cycling.

\vskip 10pt

\noindent
{\bf Step 6}

\underbar{Increase or decrease the control action aggressiveness according to the results of Step 5.}

\vskip 10pt

\noindent
{\bf Step 7}

\underbar{Repeat steps 5 and 6 for P, I, and D, one at a time, in that order.}  In other words, tune the controller first to act as a P-only controller, then add integral (PI control), then derivative (PID), each as needed.

\filbreak

\centerline{\bf Caveats}

\vskip 10pt

The procedure described here is {\it very} crude, and should only be applied as a student's first foray into PID tuning, on a safe ``demonstration'' process.  It assumes that the process responds predominantly to proportional (P-only) action, which may not be true for some processes.  It also gives no specific advice for tuning based on the results of step 3, which is the mark of an experienced PID tuner.  With study, practice, and time, you will learn what types of processes respond best to P, I, and D actions, and then you will be able to intelligently choose what parameters to adjust, and what closed-loop behaviors to look for.

\vskip 10pt

\underbar{file i00027}
%(END_QUESTION)





%(BEGIN_ANSWER)


%(END_ANSWER)





%(BEGIN_NOTES)

\noindent
{\bf Loop diagrams / inspections:}

I strongly recommend checking off students' loop diagrams while you inspect their loop (checking for secure wiring, proper tubing, good conduit installation, etc.) with them.  Have all team members take you on a ``tour'' of their completed loop, with each team member explaining a different portion of the loop you select while using their own loop diagram as a guide.  While a student is explaining their section of the loop, you can check the other students' loop diagrams for accuracy.  This not only saves time by consolidating the tasks of loop inspection and loop diagram verification, but it also ensures students can actually relate their loop diagrams to the loop they have built and articulate that understanding to you.

%INDEX% Lab exercise, building a complete process control loop (with networked controller)

%(END_NOTES)


