
%(BEGIN_QUESTION)
% Copyright 2009, Tony R. Kuphaldt, released under the Creative Commons Attribution License (v 1.0)
% This means you may do almost anything with this work of mine, so long as you give me proper credit

Calculate the following values involved with heating a pot of water (2.1 pound aluminum pot, 5.6 pounds of water) from 58 degrees Fahrenheit to boiling:

\vskip 10pt

\begin{itemize}
\item{} Amount of heat necessary to achieve boiling temperature = \underbar{\hskip 50pt} BTU
\vskip 5pt
\item{} Amount of time to achieve boil (assuming 10,000 BTU/hour heat input) = \underbar{\hskip 50pt} minutes
\vskip 5pt
\item{} Amount of additional heat necessary to convert 2 pounds of water into steam = \underbar{\hskip 50pt} BTU
\end{itemize}

\vskip 10pt

\noindent
Be sure to show all your work!

\vfil 

\underbar{file i00039}
\eject
%(END_QUESTION)





%(BEGIN_ANSWER)

This is a graded question -- no answers or hints given!

%(END_ANSWER)





%(BEGIN_NOTES)

The amount of heat necessary to warm both the pot and the water to boiling temperature may be solved for using the specific heat formula:

$$Q = mc \Delta T$$

$$Q_{pot} = (2.1) (0.215) (212-58) = 69.531 \hbox{ BTU}$$

$$Q_{water} = (5.6) (1) (212-58) = 862.4 \hbox{ BTU}$$

$$Q_{total} = 69.531 + 862.4 = {\bf 931.9 \hbox{ BTU}}$$

\vskip 10pt

At a heat input rate of 10,000 BTU per hour (i.e. 166.67 BTU per minute), this will take {\bf 5.59 minutes} to reach boiling temperature.

\vskip 10pt

Once boiling has been reached, the amount of heat necessary to boil 2 pounds of water into steam may be calculate using the latent heat equation:

$$Q = mL$$

$$Q = (2)(970.3) = {\bf 1940.6 \hbox{ BTU}}$$


%INDEX% Physics, heat and temperature: calorimetry problem 

%(END_NOTES)


