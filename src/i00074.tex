
%(BEGIN_QUESTION)
% Copyright 2009, Tony R. Kuphaldt, released under the Creative Commons Attribution License (v 1.0)
% This means you may do almost anything with this work of mine, so long as you give me proper credit

Three instrument technicians are arguing over PID tuning, specifically the test procedure used to determine the lag time of a process before attempting to tune the controller.  Technician ``A'' claims you cannot determine the {\it lag time} (otherwise known as {\it time constant}) of a process by making setpoint step-changes with the controller in automatic mode, but rather that the test must be done by making {\it output} step-changes with the controller in {\it manual} mode.  Technician ``B'' claims just the opposite: that one cannot determine process lag time by performing open-loop (manual mode) tests, but that the tests must be made closed-loop (automatic mode).  Technician ``C'' claims either test method is adequate for measuring process lag time: that the process response as seen after a setpoint step-change (controller in auto mode) is equivalent to the process response as seen after an output step-change (controller in manual mode).

\vskip 10pt

Which technician do you agree with?  Be sure to explain why you think the other technicians are incorrect.

\vfil

\underbar{file i00074}
\eject
%(END_QUESTION)





%(BEGIN_ANSWER)

This is a graded question -- no answers or hints given!

%(END_ANSWER)





%(BEGIN_NOTES)

Technician ``A'' is correct, and the others are wrong.  In order to view just the response of the process -- without interference of any kind from the controller -- the test must be done open-loop (with the controller in manual mode).

This has shown itself to be a {\it very} non-intuitive concept for many students, as I find myself having to correct students over and over again as they try to determine process lag time by performing setpoint changes in automatic mode.

%INDEX% Control, PID tuning: closed-loop tuning procedure

%(END_NOTES)


