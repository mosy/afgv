
%(BEGIN_QUESTION)
% Copyright 2011, Tony R. Kuphaldt, released under the Creative Commons Attribution License (v 1.0)
% This means you may do almost anything with this work of mine, so long as you give me proper credit

Suppose you are asked to calibrate a pH transmitter, sensing the pH of water flowing through a pipe.  The water treatment process it is a part of must be kept running and not shut down while you do this task.  

The company's standard maintenance procedure for this loop tells you to place the pH transmitter in its ``Hold'' mode during the calibration so that the readings it gets from being ``standardized'' with 4 pH and 10 pH calibration standard solutions do not get sent to the control system and mess things up for the operating process.  The transmitter's ``Hold'' mode essentially freezes its 4-20 mA signal to the loop controller at the last value it was outputting while it was running normally.

\vskip 10pt

You happen to know the pH controller is a full PID unit (P, I, and D terms all active), and that this could cause a control problem if you engage the pH transmitter's ``Hold'' function during a long calibration procedure with the loop controller still in Automatic mode.  Describe what the potential problem is, why it is better to place the controller in Manual mode than to rely on the transmitter's ``Hold'' function to prevent process upset, and also what you would do if faced with a company-standard procedure that you knew could be improved.

\vfil

\underbar{file i00076}
\eject
%(END_QUESTION)





%(BEGIN_ANSWER)

This is a graded question -- no answers or hints given!

%(END_ANSWER)





%(BEGIN_NOTES)

The problem is if the pH transmitter happens to read a pH value that is off-setpoint when it is placed in the ``Hold'' mode.  The integral term in the controller will begin to ``wind'' as it seeks to eliminate an offset that has now become impossible to eliminate thanks to the pH transmitter's signal being frozen by the ``Hold'' mode.

However, the technician should not deviate from the written procedure of using the ``Hold'' mode just yet.  If something were to go wrong during this calibration and it was found the technician deviated from standard procedure, he or she could get in a lot of trouble even if their intention was right.  Do the procedure as specified in the procedure, then seek to have the procedure changed (through all proper channels, of course!).

%INDEX% Calibration, practice: taking transmitter out of service
%INDEX% Process: water pH neutralization (generic)

%(END_NOTES)


