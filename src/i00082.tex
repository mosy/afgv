
%(BEGIN_QUESTION)
% Copyright 2006, Tony R. Kuphaldt, released under the Creative Commons Attribution License (v 1.0)
% This means you may do almost anything with this work of mine, so long as you give me proper credit

An instrument technician working for a pharmaceutical processing company is given the task of calibrating a temperature recording device used to display and log the temperature of a critical batch vessel used to grow cultures of bacteria.  After removing the instrument from the vessel and bringing it to a workbench in the calibration lab, the technician connects it to a calibration standard which has the ability to simulate a wide range of temperatures.  This way, she will be able to test how the device responds to different temperatures and make adjustments if necessary.

Before making any adjustments, though, the technician first inputs the full range of temperatures to this instrument to see how it responds in its present condition.  Then, the instrument indications are recorded as {\it As-Found} data.  Only after this step is taken does the technician make corrections to the instrument's calibration.  Then, the instrument is put through one more full-range test and the indications recorded as {\it As-Left} data.

Explain why it is important that the technician make note of both ``As-Found'' and ``As-Left'' data?  Why not just immediately make adjustments as soon as an error is detected?  Why record any of this data at all?  Try to think of a practical scenario where this might matter.

\underbar{file i00082}
%(END_QUESTION)





%(BEGIN_ANSWER)

I'll answer the question with a scenario of my own: suppose it is discovered that some patients suffered complications after taking drugs manufactured by this company, and that the particular batch of suspect drugs were processed in this very same vessel about 6 months ago?  Now imagine that this temperature recording instrument gets routinely calibrated once a month.  See the problem?

%(END_ANSWER)





%(BEGIN_NOTES)

Doing both As-Found and As-Left calibration tests is important for long-term monitoring of the measurement device (usually the process {\it transmitter} tasked with measuring the process variable).  This ``paper trail'' created by As-Found and As-Left calibration tests allows us to measure how far the transmitter drifts over time.

If we never took ``As-Found'' readings, we would not know how far the transmitter drifted from its last calibration.  In other words, if all we ever saw in the documentation was the calibration data left {\it after} the technician calibrated the instrument to specifications, we would be led to think that all our transmitters were holding their calibrations perfectly well, whether or not they actually were.  Comparing the present ``As-Found'' data with the last ``As-Left'' data tells us how far the measuring device {\it drifted} over the calibration interval.

I know of one pharmaceuticals corporation which used this calibration data to judge the suitability of new transmitter models it was evaluating.  If the transmitter calibration drifted out of range (as evidenced by a significant difference between the last ``As-Left'' data and the present ``As-Found'' data), they would review the calibration tolerance and try it one more time over another maintenance interval.  If it failed the test once again, they got rid of the instrument and replaced it with another brand or model.

%INDEX% Calibration, practice: As-Found/As-Left

%(END_NOTES)


