
%(BEGIN_QUESTION)
% Copyright 2008, Tony R. Kuphaldt, released under the Creative Commons Attribution License (v 1.0)
% This means you may do almost anything with this work of mine, so long as you give me proper credit

A tachogenerator is used to measure the rotary speed of a machine.  Its calibrated range is 0 to 1500 RPM (revolutions per minute) and its corresponding signal output is 0 to 10 volts DC.  Given these range values, calculate the output voltages for the following input shaft speeds, and then describe how you were able to correlate the different speeds to output voltage values:

% No blank lines allowed between lines of an \halign structure!
% I use comments (%) instead, so that TeX doesn't choke.

$$\vbox{\offinterlineskip
\halign{\strut
\vrule \quad\hfil # \ \hfil & 
\vrule \quad\hfil # \ \hfil \vrule \cr
\noalign{\hrule}
%
% First row
Shaft speed & Output voltage \cr
%
% Another row
(RPM) & (volts DC) \cr
%
\noalign{\hrule}
%
% Another row
100 &  \cr
%
\noalign{\hrule}
%
% Another row
350 &  \cr
%
\noalign{\hrule}
%
% Another row
500 &  \cr
%
\noalign{\hrule}
%
% Another row
750 &  \cr
%
\noalign{\hrule}
%
% Another row
890 &  \cr
%
\noalign{\hrule}
%
% Another row
975 &  \cr
%
\noalign{\hrule}
%
% Another row
1230 &  \cr
%
\noalign{\hrule}
%
% Another row
1410 &  \cr
%
\noalign{\hrule}
%
% Another row
1500 &  \cr
%
\noalign{\hrule}
} % End of \halign 
}$$ % End of \vbox

\underbar{file i00085}
%(END_QUESTION)





%(BEGIN_ANSWER)

% No blank lines allowed between lines of an \halign structure!
% I use comments (%) instead, so that TeX doesn't choke.

$$\vbox{\offinterlineskip
\halign{\strut
\vrule \quad\hfil # \ \hfil & 
\vrule \quad\hfil # \ \hfil \vrule \cr
\noalign{\hrule}
%
% First row
Shaft speed & Output voltage \cr
%
% Another row
(RPM) & (volts DC) \cr
%
\noalign{\hrule}
%
% Another row
100 & 0.67 \cr
%
\noalign{\hrule}
%
% Another row
350 & 2.3 \cr
%
\noalign{\hrule}
%
% Another row
500 & 3.3 \cr
%
\noalign{\hrule}
%
% Another row
750 & 5.0 \cr
%
\noalign{\hrule}
%
% Another row
890 & 5.9 \cr
%
\noalign{\hrule}
%
% Another row
975 & 6.5 \cr
%
\noalign{\hrule}
%
% Another row
1230 & 8.2 \cr
%
\noalign{\hrule}
%
% Another row
1410 & 9.4 \cr
%
\noalign{\hrule}
%
% Another row
1500 & 10.00 \cr
%
\noalign{\hrule}
} % End of \halign 
}$$ % End of \vbox


%(END_ANSWER)





%(BEGIN_NOTES)

This sort of problem may be solved using simple proportions.  We know that the voltage output by a tachogenerator will be directly proportional to shaft speed.  In other words, doubling the shaft speed doubles the voltage.  This makes the math very simple.

All we have to do is determine what percentage of full speed (1500 RPM) our given speed is, and then multiply the full-speed voltage (10 volts) by that percentage.

We may express the relationship between shaft speed ($S$) and voltage ($V$) as a proportionality:

$$S \propto V$$

We may express this relationship a little more precisely by inserting a {\it constant of proportionality} ($k$) and changing the proportional symbol to an equality:

$$S = kV$$

Now, all we need to do is solve for $k$ using a pair of simultaneous values for $S$ and $V$.  Since we were told the output of the tachogenerator is 10 volts at 1500 RPM:

$$1500 = k 10$$

$$k = 150$$

Knowing that $k$ = 150 for this tachogenerator (with $S$ in units of RPM and $V$ in units of volts) allows us to solve for any speed given any voltage by way of simple multiplication:

$$S = 150 V$$

If the tachogenerator drives a voltmeter, we could even post this $k$ value on the face of the meter so that an operator could multiply the voltage measurement by 150 to obtain shaft speed in RPM.

\vskip 10pt

Another way to think about this sort of problem is to use a technique I call ``unity fractions.''  Since we know tachogenerator speed and voltage to be proportional to one another, and we know they share the same zero point (0 RPM = 0 volts), we may form a fraction out of the equality 1500 RPM = 10 volts and use it in such a way as to cancel the unit of RPM and replace it with volts.  

We know that any fraction where the numerator and denominator are the same quantity ($5 \over 5$ for example) has a value of 1, or unity.  We also know that multiplying any value by 1 does not change that value (for example, $570 \times 1 = 570$).  So, if we form a fraction out of two {\it physically equal} quantities -- in this case, the fraction 10 volts divided by 1500 RPM, since those two quantities are physically equal to one another in the context of the tachogenerator -- this ``unity fraction'' has a {\it physical} value of 1, and we may safely multiply it by any other value without changing that other value's quantity.  The point of doing this is to cancel out the unwanted unit and replace it with another unit.

To see how this works, let's use this technique to calculate tachogenerator voltage at 1230 RPM:

$$\left({1230 \hbox{ RPM} \over 1} \right)  \left( {10 \hbox{ volts} \over 1500 \hbox{ RPM}} \right) = 8.2 \hbox{ volts}$$

The unit of ``RPM'' cancels from the top of the left-hand fraction and the bottom of the right-hand fraction, leaving the unit of ``volts'' as the only one remaining.  Since the unity fraction of $10 \hbox{ volts} \over 1500 \hbox{ RPM}$ has a physical value of 1, multiplying it by the fraction $1230 \hbox{ RPM} \over 1$ does not alter that value.  We only cancel out the unit of RPM, to tell us how many volts the tachogenerator will output at that speed.

We can use this same ``unity fraction'' to translate from volts to RPM, too!  Imagine a situation where the tachogenerator was spinning fast enough to output a voltage of 7.1 volts.  How fast is this?  We can use this same unit-cancellation technique to find out:

$$\left({7.1 \hbox{ volts} \over 1} \right)  \left( 1500 \hbox{ RPM} \over {10 \hbox{ volts}} \right) = 1065 \hbox{ RPM}$$

Since the unity fraction of $10 \hbox{ volts} \over 1500 \hbox{ RPM}$ has a physical value of 1, we may flip it upside-down without changing its value ($5 \over 5$ is still 1, whether you flip the fraction upside-down or not!).  In this case, we needed to flip the unity fraction upside-down in order to make the unit of ``volts'' cancel out and leave us with the unit of ``RPM'' as the only one left standing.

\vskip 10pt

While the technique of ``unity fractions'' may seem over-complicated, especially for a simple proportion problem like this one, it is very powerful.  We will find lots of application for this technique later on in our studies!


%INDEX% Basics, tachogenerator: input and output ranges
%INDEX% Calibration, table: tachogenerator

%(END_NOTES)


