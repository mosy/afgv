
%(BEGIN_QUESTION)
% Copyright 2006, Tony R. Kuphaldt, released under the Creative Commons Attribution License (v 1.0)
% This means you may do almost anything with this work of mine, so long as you give me proper credit

An important part of performing instrument calibration is determining the extent of an instrument's error.  Error is usually measured in {\it percent of span}.  Calculate the percent of span error for each of the following examples, and be sure to note the sign of the error (positive or negative):

\begin{itemize}
\item{} {\bf Pressure gauge}
\item{} LRV = 0 PSI
\item{} URV = 100 PSI 
\item{} Test pressure = 65 PSI 
\item{} Instrument indication = 67 PSI
\item{} Error = \underbar{\hskip 50pt} \% of span
\end{itemize}

\vskip 10pt

\begin{itemize}
\item{} {\bf Weigh scale}
\item{} LRV = 0 pounds
\item{} URV = 40,000 pounds
\item{} Test weight = 10,000 pounds
\item{} Instrument indication = 9,995 pounds
\item{} Error = \underbar{\hskip 50pt} \% of span
\end{itemize}

\vskip 10pt

\begin{itemize}
\item{} {\bf Thermometer}
\item{} LRV = -40$^{o}$F
\item{} URV = 250$^{o}$F
\item{} Test temperature = 70$^{o}$F
\item{} Instrument indication = 68$^{o}$F
\item{} Error = \underbar{\hskip 50pt} \% of span
\end{itemize}

\vskip 10pt

\begin{itemize}
\item{} {\bf pH analyzer}
\item{} LRV = 4 pH
\item{} URV = 10 pH
\item{} Test buffer solution = 7.04 pH
\item{} Instrument indication = 7.13 pH
\item{} Error = \underbar{\hskip 50pt} \% of span
\end{itemize}

\vskip 10pt

Also, show the math you used to calculate each of the error percentages.

\vskip 10pt

Challenge: build a computer spreadsheet that calculates error in percent of span, given the LRV, URV, test value, and actual indicated value for each instrument.

\underbar{file i00089}
%(END_QUESTION)





%(BEGIN_ANSWER)

\begin{itemize}
\item{} {\bf Pressure gauge}
\item{} LRV = 0 PSI
\item{} URV = 100 PSI 
\item{} Test pressure = 65 PSI 
\item{} Instrument indication = 67 PSI
\item{} Error = {\it +2} \% of span
\end{itemize}

\vskip 10pt

\begin{itemize}
\item{} {\bf Weigh scale}
\item{} LRV = 0 pounds
\item{} URV = 40,000 pounds
\item{} Test weight = 10,000 pounds
\item{} Instrument indication = 9,995 pounds
\item{} Error = {\it -0.0125} \% of span
\end{itemize}

\vskip 10pt

\begin{itemize}
\item{} {\bf Thermometer}
\item{} LRV = -40$^{o}$F
\item{} URV = 250$^{o}$F
\item{} Test temperature = 70$^{o}$F
\item{} Instrument indication = 68$^{o}$F
\item{} Error = {\it -0.69} \% of span
\end{itemize}

\vskip 10pt

\begin{itemize}
\item{} {\bf pH analyzer}
\item{} LRV = 4 pH
\item{} URV = 10 pH
\item{} Test buffer solution = 7.04 pH
\item{} Instrument indication = 7.13 pH
\item{} Error = {\it +1.5} \% of span
\end{itemize}

%(END_ANSWER)





%(BEGIN_NOTES)

Here is the equation I used to calculate percentage error in each case:

$$\hbox{\% error} = \left({\hbox{Actual} - \hbox{Ideal} \over \hbox{Span}}\right) (100 \%)$$

Remember that the mathematical {\it sign} of the error is very important to note!  Both the weigh scale and the thermometer have {\it negative} error values because their indications fell below the test (ideal) values.

A positive error value means the instrument registers too much, while a negative error value means the instrument registers too little.

%INDEX% Calibration, tolerance: error in percent of span

%(END_NOTES)


