
%(BEGIN_QUESTION)
% Copyright 2006, Tony R. Kuphaldt, released under the Creative Commons Attribution License (v 1.0)
% This means you may do almost anything with this work of mine, so long as you give me proper credit

%An important part of performing instrument calibration is determining the extent of an instrument's error.  Error is usually measured in {\it percent of span}.  Calculate the percent of span error for each of the following examples, and be sure to note the sign of the error (positive or negative):

En viktig del av instrumentkalibrering er {\aa} avgj{\o}re instrumentets avvik. Avvik blir vanligvis m{\aa}lt i \% av m{\aa}leomfanget. Regn ut avviket i \% for eksmeplene nedenfor. 
\medskip
Vis utregningene

\medskip
\item{$\bullet$} {\bf Pressure gauge}
\item{$\bullet$} LRV = 0 PSI
\item{$\bullet$} URV = 100 PSI 
\item{$\bullet$} Test pressure = 65 PSI 
\item{$\bullet$} Instrument indication = 67 PSI
\item{$\bullet$} Error = \underbar{\hskip 50pt} \% of span
\medskip

\vskip 10pt

\medskip
\item{$\bullet$} {\bf Weigh scale}
\item{$\bullet$} LRV = 0 pounds
\item{$\bullet$} URV = 40,000 pounds
\item{$\bullet$} Test weight = 10,000 pounds
\item{$\bullet$} Instrument indication = 9,995 pounds
\item{$\bullet$} Error = \underbar{\hskip 50pt} \% of span
\medskip

\vskip 10pt

\medskip
\item{$\bullet$} {\bf Thermometer}
\item{$\bullet$} LRV = -40$^{o}$F
\item{$\bullet$} URV = 250$^{o}$F
\item{$\bullet$} Test temperature = 70$^{o}$F
\item{$\bullet$} Instrument indication = 68$^{o}$F
\item{$\bullet$} Error = \underbar{\hskip 50pt} \% of span
\medskip

\vskip 10pt

\medskip
\item{$\bullet$} {\bf pH analyzer}
\item{$\bullet$} LRV = 4 pH
\item{$\bullet$} URV = 10 pH
\item{$\bullet$} Test buffer solution = 7.04 pH
\item{$\bullet$} Instrument indication = 7.13 pH
\item{$\bullet$} Error = \underbar{\hskip 50pt} \% of span
\medskip

\vskip 10pt

%Also, show the math you used to calculate each of the error percentages.

\vskip 10pt

%Challenge: build a computer spreadsheet that calculates error in percent of span, given the LRV, URV, test value, and actual indicated value for each instrument.

Utfordring: Lag et regneark som kalkulerer avviket i \% av gitte data. 

\underbar{file i00089}
%(END_QUESTION)





%(BEGIN_ANSWER)

\medskip
\item{$\bullet$} {\bf Pressure gauge}
\item{$\bullet$} LRV = 0 PSI
\item{$\bullet$} URV = 100 PSI 
\item{$\bullet$} Test pressure = 65 PSI 
\item{$\bullet$} Instrument indication = 67 PSI
\item{$\bullet$} Error = {\it +2} \% of span
\medskip

\vskip 10pt

\medskip
\item{$\bullet$} {\bf Weigh scale}
\item{$\bullet$} LRV = 0 pounds
\item{$\bullet$} URV = 40,000 pounds
\item{$\bullet$} Test weight = 10,000 pounds
\item{$\bullet$} Instrument indication = 9,995 pounds
\item{$\bullet$} Error = {\it -0.0125} \% of span
\medskip

\vskip 10pt

\medskip
\item{$\bullet$} {\bf Thermometer}
\item{$\bullet$} LRV = -40$^{o}$F
\item{$\bullet$} URV = 250$^{o}$F
\item{$\bullet$} Test temperature = 70$^{o}$F
\item{$\bullet$} Instrument indication = 68$^{o}$F
\item{$\bullet$} Error = {\it -0.69} \% of span
\medskip

\vskip 10pt

\medskip
\item{$\bullet$} {\bf pH analyzer}
\item{$\bullet$} LRV = 4 pH
\item{$\bullet$} URV = 10 pH
\item{$\bullet$} Test buffer solution = 7.04 pH
\item{$\bullet$} Instrument indication = 7.13 pH
\item{$\bullet$} Error = {\it +1.5} \% of span
\medskip

%(END_ANSWER)





%(BEGIN_NOTES)

Here is the equation I used to calculate percentage error in each case:

$$\hbox{\% error} = \left({\hbox{Actual} - \hbox{Ideal} \over \hbox{Span}}\right) (100 \%)$$

Remember that the mathematical {\it sign} of the error is very important to note!  Both the weigh scale and the thermometer have {\it negative} error values because their indications fell below the test (ideal) values.

A positive error value means the instrument registers too much, while a negative error value means the instrument registers too little.

%INDEX% Calibration, tolerance: error in percent of span

%(END_NOTES)


