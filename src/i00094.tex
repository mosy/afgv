
%(BEGIN_QUESTION)
% Copyright 2006, Tony R. Kuphaldt, released under the Creative Commons Attribution License (v 1.0)
% This means you may do almost anything with this work of mine, so long as you give me proper credit

Suppose your recipe book specified 462 quarts of cooking oil to deep-fry a camel, but your local wholesale food outlet only sold cooking oil by the 55-gallon drum.  Realizing the need to convert units, you decide to convert the 462 quarts into gallons by dividing by 4, then convert gallons into drums by dividing by 55.

A friend of yours decides to help by showing you a simple method for organizing conversion calculations.  On a piece of paper, he writes the following product of fractions:

$${462 \hbox{ quarts} \over 1}  \times  {1 \hbox{ gallon} \over 4 \hbox{ quarts}}  \times  {1 \hbox{ drum} \over 55 \hbox{ gallons}}$$

Then, he proceeds to show you how to cancel units as though they were equal numbers, leaving only the unit of ``drums'' at the end (462 quarts equals 2.1 drums).

Explain how and why this technique for unit conversions works.

\vskip 10pt

Challenge question: explain why the following example is an {\it improper} use of this unit conversion technique.  Converting a temperature rate-of-change from 80$^{o}$ F per minute to degrees C per hour:

$${80^{o} \hbox{ F} \over \hbox{min}} \times {60 \hbox{ min} \over 1 \hbox{ hour}} \times {0^{o} \hbox{ C} \over 32^{o} \hbox{ F}}$$

\underbar{file i00094}
%(END_QUESTION)





%(BEGIN_ANSWER)

Hint: treat the units of measurement as you would algebraic {\it variables}:

$${462a \over 1}  \times  {1b \over 4a}  \times  {1c \over 55b}$$

\vskip 10pt

Answer to challenge question: this technique of multiplying by fractions comprised of equal physical quantities (numerator and denominator) is perfectly fine so long as those unit scales share the same zero point.  When different units of measurement for a common variable are {\it shifted} from each other by a constant (e.g. 32 degrees of shift in the case of $^{o}$C and $^{o}$F), this technique will not work.

%(END_ANSWER)





%(BEGIN_NOTES)

Each of the fractions relating one unit quantity to another (1 gallon per 4 quarts ; 1 drum per 55 gallons) is called a {\it unity fraction}, because the {\it physical} value of each is exactly 1 (unity).  This is why it is ``legal'' to just multiply the original quantity by these fractions, because multiplying any quantity by unity does not change that quantity's value.  All we are doing here is changing the unit of measurement used to express the final answer, not the physical quantity.

This technique has obvious value because the procedure of unit cancellation ensures the multiplication will always be done correctly.

\vskip 10pt

Regarding the challenge question, most scales of measurement share the same zero.  0 minutes is the same thing as 0 hours, 0 days, 0 years, etc.  0 inches is the same as 0 feet, 0 centimeters, 0 yards, 0 meters, 0 miles, and 0 kilometers.  However, 0 degrees F is {\it not} the same as 0 degrees C, nor is 0 PSIA the same as 0 PSIG.  Since the ``unity fraction'' technique is based on ratios and proportions, it only works when all scales share the same zero, and therefore it fails when shifted-zero scales are involved.

%INDEX% Physics, units and conversions: unit conversions using unity fractions

%(END_NOTES)


