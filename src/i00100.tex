
%(BEGIN_QUESTION)
% Copyright 2006, Tony R. Kuphaldt, released under the Creative Commons Attribution License (v 1.0)
% This means you may do almost anything with this work of mine, so long as you give me proper credit

A Foxboro pneumatic square root extractor has a calibrated range of 3 to 15 PSI for both input and output.  Complete the following table of values for this relay, assuming perfect calibration (no error).  Be sure to show your work!

% No blank lines allowed between lines of an \halign structure!
% I use comments (%) instead, so that TeX doesn't choke.

$$\vbox{\offinterlineskip
\halign{\strut
\vrule \quad\hfil # \ \hfil & 
\vrule \quad\hfil # \ \hfil & 
\vrule \quad\hfil # \ \hfil & 
\vrule \quad\hfil # \ \hfil \vrule \cr
\noalign{\hrule}
%
% First row
Input signal & Percent of input & Percent of output & Output signal \cr
%
% Another row
(PSI) & span (\%) & span (\%) & (PSI) \cr
%
\noalign{\hrule}
%
% Another row
5 &  &  &  \cr
%
\noalign{\hrule}
%
% Another row
13 &  &  &  \cr
%
\noalign{\hrule}
%
% Another row
 & 50 &  &  \cr
%
\noalign{\hrule}
%
% Another row
 & 30 &  &  \cr
%
\noalign{\hrule}
%
% Another row
 &  & 80 &  \cr
%
\noalign{\hrule}
%
% Another row
 &  & 15 &  \cr
%
\noalign{\hrule}
%
% Another row
 &  &  & 7 \cr
%
\noalign{\hrule}
%
% Another row
 &  &  & 12 \cr
%
\noalign{\hrule}
} % End of \halign 
}$$ % End of \vbox

\vskip 20pt \vbox{\hrule \hbox{\strut \vrule{} {\bf Suggestions for Socratic discussion} \vrule} \hrule}

\begin{itemize}
\item{} Why are pneumatic square-root extractors all but obsolete in modern industry?  What has replaced their functionality?
\item{} Share problem-solving techniques for obtaining answers to this problem.
\end{itemize}

\underbar{file i00100}
%(END_QUESTION)





%(BEGIN_ANSWER)

$$\vbox{\offinterlineskip
\halign{\strut
\vrule \quad\hfil # \ \hfil & 
\vrule \quad\hfil # \ \hfil & 
\vrule \quad\hfil # \ \hfil & 
\vrule \quad\hfil # \ \hfil \vrule \cr
\noalign{\hrule}
%
% First row
Input signal & Percent of input & Percent of output & Output signal \cr
%
% Another row
(PSI) & span (\%) & span (\%) & (PSI) \cr
%
\noalign{\hrule}
%
% Another row
5 & 16.67 & 40.82 & 7.899 \cr
%
\noalign{\hrule}
%
% Another row
13 & {\bf 83.33} & {\bf 91.29} & {\bf 13.95} \cr
%
\noalign{\hrule}
%
% Another row
9 & 50 & 70.71 & 11.49 \cr
%
\noalign{\hrule}
%
% Another row
6.6 & 30 & 54.77 & 9.573 \cr
%
\noalign{\hrule}
%
% Another row
10.68 & 64 & 80 & 12.6 \cr
%
\noalign{\hrule}
%
% Another row
{\bf 3.27} & {\bf 2.25} & 15 & {\bf 4.8} \cr
%
\noalign{\hrule}
%
% Another row
4.333 & 11.11 & 33.33 & 7 \cr
%
\noalign{\hrule}
%
% Another row
9.75 & 56.25 & 75 & 12 \cr
%
\noalign{\hrule}
} % End of \halign 
}$$ % End of \vbox

Values shown in bold-faced type are those given to students in the ``Answer'' section.
\noindent


%(END_ANSWER)





%(BEGIN_NOTES)

% No blank lines allowed between lines of an \halign structure!
% I use comments (%) instead, so that TeX doesn't choke.

$$\vbox{\offinterlineskip
\halign{\strut
\vrule \quad\hfil # \ \hfil & 
\vrule \quad\hfil # \ \hfil & 
\vrule \quad\hfil # \ \hfil & 
\vrule \quad\hfil # \ \hfil \vrule \cr
\noalign{\hrule}
%
% First row
Input signal & Percent of input & Percent of output & Output signal \cr
%
% Another row
(PSI) & span (\%) & span (\%) & (PSI) \cr
%
\noalign{\hrule}
%
% Another row
5 & 16.67 & 40.82 & 7.899 \cr
%
\noalign{\hrule}
%
% Another row
13 & {\bf 83.33} & {\bf 91.29} & {\bf 13.95} \cr
%
\noalign{\hrule}
%
% Another row
9 & 50 & 70.71 & 11.49 \cr
%
\noalign{\hrule}
%
% Another row
6.6 & 30 & 54.77 & 9.573 \cr
%
\noalign{\hrule}
%
% Another row
10.68 & 64 & 80 & 12.6 \cr
%
\noalign{\hrule}
%
% Another row
{\bf 3.27} & {\bf 2.25} & 15 & {\bf 4.8} \cr
%
\noalign{\hrule}
%
% Another row
4.333 & 11.11 & 33.33 & 7 \cr
%
\noalign{\hrule}
%
% Another row
9.75 & 56.25 & 75 & 12 \cr
%
\noalign{\hrule}
} % End of \halign 
}$$ % End of \vbox

Values shown in bold-faced type are those given to students in the ``Answer'' section.

%INDEX% Calibration: table, pneumatic square root extractor
%INDEX% Relay, square root: calibration table

%(END_NOTES)


