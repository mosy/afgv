
%(BEGIN_QUESTION)
% Copyright 2006, Tony R. Kuphaldt, released under the Creative Commons Attribution License (v 1.0)
% This means you may do almost anything with this work of mine, so long as you give me proper credit

A turbine flowmeter measuring cooling water for a large power generator uses an electronic circuit to convert its pickup coil pulses into a 4-20 mA analog current signal.  The ``K factor'' for the turbine element is 99 pulses per gallon, and the 4-20 mA analog output is ranged from 0 to 500 GPM flow.  Complete the following table of values for this transmitter, assuming perfect calibration (no error).  Be sure to show your work!

% No blank lines allowed between lines of an \halign structure!
% I use comments (%) instead, so that TeX doesn't choke.

$$\vbox{\offinterlineskip
\halign{\strut
\vrule \quad\hfil # \ \hfil & 
\vrule \quad\hfil # \ \hfil & 
\vrule \quad\hfil # \ \hfil & 
\vrule \quad\hfil # \ \hfil \vrule \cr
\noalign{\hrule}
%
% First row
Measured flow & Pickup signal & Percent of output & Output signal \cr
%
% Another row
(GPM) & frequency (Hz) & span (\%) & (mA) \cr
%
\noalign{\hrule}
%
% Another row
250 &  &  &  \cr
%
\noalign{\hrule}
%
% Another row
412 &  &  &  \cr
%
\noalign{\hrule}
%
% Another row
 & 305 &  &  \cr
%
\noalign{\hrule}
%
% Another row
 & 780 &  &  \cr
%
\noalign{\hrule}
%
% Another row
 &  & 63 &  \cr
%
\noalign{\hrule}
%
% Another row
 &  & 49 &  \cr
%
\noalign{\hrule}
%
% Another row
 &  &  & 10 \cr
%
\noalign{\hrule}
%
% Another row
 &  &  & 16 \cr
%
\noalign{\hrule}
} % End of \halign 
}$$ % End of \vbox

\vskip 20pt \vbox{\hrule \hbox{\strut \vrule{} {\bf Suggestions for Socratic discussion} \vrule} \hrule}

\begin{itemize}
\item{} Demonstrate how to {\it estimate} numerical answers for this problem without using a calculator.
\item{} Suppose you were asked to check the accuracy of the frequency-to-current converter circuit for this flowmeter.  What sort of test equipment would you use, and how could you perform the test with the flowmeter still installed in the cooling water pipe?
\item{} Could the pulse output of the pickup coil be used directly as a flow signal, or is the converter circuit absolutely necessary?
\item{} Explain how a PLC could be used to {\it totalize} the water flow through this flowmeter, to provide total usage values at the end of each day.
\end{itemize}

\underbar{file i00101}
%(END_QUESTION)





%(BEGIN_ANSWER)

% No blank lines allowed between lines of an \halign structure!
% I use comments (%) instead, so that TeX doesn't choke.

$$\vbox{\offinterlineskip
\halign{\strut
\vrule \quad\hfil # \ \hfil & 
\vrule \quad\hfil # \ \hfil & 
\vrule \quad\hfil # \ \hfil & 
\vrule \quad\hfil # \ \hfil \vrule \cr
\noalign{\hrule}
%
% First row
Measured flow & Pickup signal & Percent of output & Output signal \cr
%
% Another row
(GPM) & frequency (Hz) & span (\%) & (mA) \cr
%
\noalign{\hrule}
%
% Another row
250 & 412.5 & 50 & 12 \cr
%
\noalign{\hrule}
%
% Another row
412 & 679.8 & 82.4 & 17.18 \cr
%
\noalign{\hrule}
%
% Another row
184.8 & 305 & 36.97 & 9.915 \cr
%
\noalign{\hrule}
%
% Another row
472.7 & 780 & 94.55 & 19.13 \cr
%
\noalign{\hrule}
%
% Another row
315 & 519.8 & 63 & 14.08 \cr
%
\noalign{\hrule}
%
% Another row
245 & 404.3 & 49 & 11.84 \cr
%
\noalign{\hrule}
%
% Another row
187.5 & 309.4 & 37.5 & 10 \cr
%
\noalign{\hrule}
%
% Another row
375 & 618.8 & 75 & 16 \cr
%
\noalign{\hrule}
} % End of \halign 
}$$ % End of \vbox

$$f = kQ$$

\noindent
Where,

$f$ = Frequency in Hertz (pulses per second)

$k$ = Calibration factor in pulses per gallon

$Q$ = Volumetric flow rate in gallons per second

\vskip 10pt

$$f = {kQ \over 60}$$

\noindent
Where,

$f$ = Frequency in Hertz (pulses per second)

$k$ = Calibration factor in pulses per gallon

$Q$ = Volumetric flow rate in gallons per minute


%(END_ANSWER)





%(BEGIN_NOTES)


%INDEX% Calibration: table, turbine flowmeter
%INDEX% Measurement, flow: calibration table
%INDEX% Measurement, flow: turbine

%(END_NOTES)


