
%(BEGIN_QUESTION)
% Copyright 2006, Tony R. Kuphaldt, released under the Creative Commons Attribution License (v 1.0)
% This means you may do almost anything with this work of mine, so long as you give me proper credit

John has a bicycle with 26 inch diameter wheels.  How fast are the wheels spinning (in revolutions per minute, or ``RPM'') when John rides the bike at a velocity of 15 miles per hour?  Remember that the circumference of a circle is equal to the diameter multiplied by $\pi$.  This figure will be the distance traveled for each revolution of the wheel.

\underbar{file i00105}
%(END_QUESTION)





%(BEGIN_ANSWER)

15 MPH = 193.924 RPM

\vskip 10pt

At first this might not even seem like a unit conversion problem, but it is.  Obviously, John's bicycle wheels will not be spinning when the velocity is 0 miles per hour, and the wheels' rotational speed will directly follow the bicycle's velocity in a linear fashion.  What we are dealing with here is a direct proportion, and this is what unit conversions are really about: direct proportions.

The key to solving this problem is coming up with a unity fraction that relates one revolution of the bicycle wheel to the amount of distance traveled.  Keeping everything in units of inches, we have a distance of 81.68 inches per revolution.  These figures are what we will use to build the unity fraction to convert velocity into rotational speed (the fourth fraction in this long calculation):

$${15 \hbox{ mi} \over 1 \hbox{ hr}} \times {5280 \hbox{ ft} \over 1 \hbox{ mi}} \times {12 \hbox{ in} \over 1 \hbox{ ft}} \times {1 \hbox{ rev} \over 81.68 \hbox{ in}} \times {1 \hbox{ hr} \over 60 \hbox{ min}}$$

%(END_ANSWER)





%(BEGIN_NOTES)


%INDEX% Physics, units and conversions: unit conversions using unity fractions

%(END_NOTES)


