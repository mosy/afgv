
%(BEGIN_QUESTION)
% Copyright 2006, Tony R. Kuphaldt, released under the Creative Commons Attribution License (v 1.0)
% This means you may do almost anything with this work of mine, so long as you give me proper credit

Suppose you needed to convert the distance of 35 feet into inches.  The calculation is trivial, but you decide to practice the use of ``unity fractions'' to solve for the answer:

$${{35 \hbox{ ft}} \over {1}} \times {{12 \hbox{ in}} \over {1 \hbox{ ft}}}$$

Units of ``feet'' (ft) cancel to leave you with an answer of 420 inches.  So far, so good.

\vskip 10pt

Now suppose you must convert the area of 70 square feet (ft$^{2}$) into square inches (in$^{2}$).  There should be a way that you can use the ``unity fraction'' concept to do this, but it will not work if we set it up identically to the last problem:

$${{70 \hbox{ ft}^2} \over {1}} \times {{12 \hbox{ in}} \over {1 \hbox{ ft}}} = \hbox{ Wrong Answer!}$$

The reason this will not work as a final solution for the area problem is because square feet do not cancel with lineal feet.  The quantities are still one-dimensional: feet, not square feet; inches, not square inches.

But what if we did this?

$${{70 \hbox{ ft}^2} \over {1}} \times {{12^2 \hbox{ in}^2} \over {1^2 \hbox{ ft}^2}}$$

. . . which is the same as doing this . . .

$${{70 \hbox{ ft}^2} \over {1}} \times {{12 \hbox{ in}} \over {1 \hbox{ ft}}} \times {{12 \hbox{ in}} \over {1 \hbox{ ft}}}$$

Will this give us the correct answer?  Why or why not?  If not, explain how we {\it can} get the correct answer using the ``unity fraction'' method of units conversion.

Finally, once you are decided on a proper way to do convert units of area, generalize this rule to cover conversions involving units for volume ({\it cubic} feet and inches -- ft$^{3}$ and in$^{3}$).

\underbar{file i00107}
%(END_QUESTION)





%(BEGIN_ANSWER)

70 ft$^{2}$ = 10,080 in$^{2}$

\vskip 10pt

It is possible to take any one-dimensional unity fraction and turn it into an equivalent two- or three-dimensional unity fraction simply by applying a power (exponent) to the entire fraction.  Since the physical value of a unity fraction is always 1, squaring or cubing such a fraction won't change its physical value any more than squaring or cubing the fraction $1 \over 1$ will change its value.

To give some examples:

$$\left({{12 \hbox{ in}} \over {1 \hbox{ ft}}}\right)^2 = {{144 \hbox{ in}^2} \over {1 \hbox{ ft}^2}}$$

\vskip 10pt

$$\left({{12 \hbox{ in}} \over {1 \hbox{ ft}}}\right)^3 = {{1728 \hbox{ in}^3} \over {1 \hbox{ ft}^3}}$$

\vskip 10pt

$$\left({{1 \hbox{ in}} \over {2.54 \hbox{ cm}}}\right)^2 = {{1 \hbox{ in}^2} \over {6.4516 \hbox{ cm}^2}}$$

\vskip 10pt

$$\left({{1 \hbox{ in}} \over {2.54 \hbox{ cm}}}\right)^3 = {{1 \hbox{ in}^3} \over {16.387 \hbox{ cm}^3}}$$

This is nice to know, as it reduces the number of conversion factors you need to memorize.  Rather than memorizing 12 inches to 1 foot {\it and} 144 square inches to 1 square foot {\it and} 1728 cubic inches to 1 cubic foot, all you need to commit to memory is 12 inches = 1 foot and then you can square or cube when necessary to derive the other unity fractions.

%(END_ANSWER)





%(BEGIN_NOTES)


%INDEX% Physics, units and conversions: unit conversions using unity fractions

%(END_NOTES)


