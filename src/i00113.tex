
%(BEGIN_QUESTION)
% Copyright 2006, Tony R. Kuphaldt, released under the Creative Commons Attribution License (v 1.0)
% This means you may do almost anything with this work of mine, so long as you give me proper credit

Explain the difference between a {\it mastery} assessment and a {\it proportional-graded} assessment.  Given examples of each in the course(s) you are taking.

\underbar{file i00113}
%(END_QUESTION)





%(BEGIN_ANSWER)

A {\it mastery} assessment is one that must be passed with a 100\% score (no errors).  Mastery assessments are usually given with multiple opportunities to pass.  The basic idea is, you try and try until you get it perfect.  This ensures mastery of the concept, hence the name.

By contrast, a {\it proportional-graded} assessment is one where you do not have to achieve perfection to pass.  Most of the tests and assignments you have completed in your life are of this type.  A grade (percentage, ranking, and/or letter) is given based on how well you answer the question(s).

\vskip 10pt

In all the Instrumentation courses, all exams have both mastery and proportional sections.  Lab exercises likewise have both mastery and proportional sections as well.  Preparation and feedback grades are strictly proportional, with no mastery component.

\vskip 10pt

Follow-up question: what happens if you fail to fulfill a mastery assessment within the allotted time?

%(END_ANSWER)





%(BEGIN_NOTES)

It should be noted that passing {\it only} the mastery assessments in these courses is not enough to achieve the 70\% minimum (C- grade) score necessary to pass each course itself, and that failing to pass any mastery assessment within the allotted time or maximum number of attempts will result in a failing grade for the course.

%INDEX% Course organization, assessment: mastery versus proportional assessments

%(END_NOTES)


