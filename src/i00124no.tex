
%(BEGIN_QUESTION)
% Copyright 2003, Tony R. Kuphaldt, released under the Creative Commons Attribution License (v 1.0)
% This means you may do almost anything with this work of mine, so long as you give me proper credit

Identifiser retningen på endringen (enten {\it økning} eller {\it reduksjon}) for hver av følgende variabler i dette nivåreguleringssystemet (flottørtransmitter, P-regulator, I/P-omformer og reguleringsventil), forutsatt at regulatoren er i automatisk modus, er reversvirkende, og at innstrømmingen (flow in) plutselig øker:

\vskip 10pt

$$\includegraphics{i00124x01.eps}$$

\vskip 10pt

\begin{itemize}
\item{} Flow out (Utstrømming):
\item{} Float position (Flottørposisjon):
\item{} LT output signal (LT utgangssignal):
\item{} Controller error (LT - SP) (Regulatoravvik):
\item{} Controller output signal (Regulatorutgang/Pådrag):
\item{} I/P output signal (I/P utgangssignal):
\item{} Valve position (Ventilposisjon):
\end{itemize}

\underbar{file i00124}
%(END_QUESTION)





%(BEGIN_ANSWER)

\begin{itemize}
\item{} Flow out (Utstrømming): {\it øker}
\item{} Float position (Flottørposisjon): {\it øker}
\item{} LT output signal (LT utgangssignal): {\it øker}
\item{} Controller error (LT - SP) (Regulatoravvik): {\it øker}
\item{} Controller output signal (Regulatorutgang/Pådrag): {\it reduseres}
\item{} I/P output signal (I/P utgangssignal): {\it reduseres}
\item{} Valve position (Ventilposisjon): {\it reduseres} (stenger mer)
\end{itemize}

\vskip 10pt

Det er selvsagt mer enn ett riktig svar for "Flow out", avhengig av tidsperspektivet. "Flow out" vil ikke endres umiddelbart etter at "Flow in" endres, men vil til slutt endres som et resultat av at regulatoren flytter ventilen. Svarene gitt ovenfor gjelder for konsekvensene av en P-regulators handling.

%(END_ANSWER)





%(BEGIN_NOTES)

Dette spørsmålet kan brukes som et utgangspunkt for å diskutere begrepet "belastning" (load) i et kontrollsystem. Be elevene dine definere hva "belastningen" er i dette prosessystemet.

Et viktig poeng å diskutere er at "Flow out" til slutt {\it må} bli lik "Flow in" hvis nivået skal stabilisere seg. Dette betyr at til tross for at ventilen strupes igjen (reduserer åpning), må den faktiske utstrømmingen til slutt øke til den er lik den økte innstrømmingen. Hvordan er dette mulig, at en ventil beveger seg mot stenget stilling, men likevel slipper gjennom mer væske? Be elevene forklare dette tilsynelatende paradokset.

%INDEX% Control, basics: signal changes in an automatic control loop

%(END_NOTES)
