
%(BEGIN_QUESTION)
% Copyright 2006, Tony R. Kuphaldt, released under the Creative Commons Attribution License (v 1.0)
% This means you may do almost anything with this work of mine, so long as you give me proper credit

The very simplest style of automatic control is known as {\it on-off} or more whimsically, {\it bang-bang} control.  This is where the automatic controller only has two output signal modes: fully on and fully off.  Your home's heating system is most likely of this sort, where a thermostat can either tell the furnace to turn on or to turn off.

Describe the advantages and disadvantages of ``on-off'' control, as contrasted against more sophisticated control schemes where a final control element such as a control valve may be proportionately positioned anywhere between fully open and fully closed according to the demands of the process.

\vskip 20pt \vbox{\hrule \hbox{\strut \vrule{} {\bf Suggestions for Socratic discussion} \vrule} \hrule}

\begin{itemize}
\item{} What other control systems in common experience use the ``bang-bang'' strategy?
\end{itemize}

\underbar{file i00125}
%(END_QUESTION)





%(BEGIN_ANSWER)


%(END_ANSWER)





%(BEGIN_NOTES)

The main advantages of on/off control schemes is that they are simple and cheap.  The disadvantages include ``cycling'' of the final control element and also ``rippling'' of the process variable over time.  For rather obvious reasons, an on/off control system can never maintain a process variable at a steady value, but must always incur some up/down oscillation over time.

A common concept in on/off control systems is {\it hysteresis}, where the controller device turns on and off at different process variable points.  For example, consider a room heating system that starts the heater when the temperature falls below 70 degrees F, and does not turn the heater off until the temperature rises above 73 degrees F.  The three-degree difference between the on and off points is known as the {\it differential gap}, or {\it deadband} of the control system.  Normally we try to avoid having deadband within a process transmitter or control valve (since it makes steady control in a proportioning system nearly impossible).  However, some amount of deadband is usually a good thing to have in an on/off control system because it reduces the number of times the final control element must cycle over any given period of time.  Larger differential gaps mean fewer on/off cycles over time, but at the expense of greater up-and-down oscillations in process variable.

%INDEX% Control, basics: on/off control

%(END_NOTES)


