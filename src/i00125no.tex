
%(BEGIN_QUESTION)
% Copyright 2003, Tony R. Kuphaldt, released under the Creative Commons Attribution License (v 1.0)
% This means you may do almost anything with this work of mine, so long as you give me proper credit

En enkel form for automatisert regulering kalles {\it av/på-regulering}. Termostaten i et enkel oppvarmingssystem for hus er et godt eksempel på en slik regulator: enten slås varmeovnen på for fullt, eller så slås den helt av.

Tenk deg en væsketank der væskenivået opprettholdes av en slik nivåsjalter (level switch) og magnetventil, som enten fyller tanken (ventil helt åpen) når nivået blir for lavt, eller stopper fyllingen (ventil helt stengt) når nivået blir høyt nok:


Utvikle en graf som viser nivået i denne tanken over tid, forutsatt at utstrømmingen fra tanken er konstant og mindre enn innstrømmingsraten når ventilen er åpen.

\underbar{file i00125}
%(END_QUESTION)





%(BEGIN_ANSWER)

Plottene bør se omtrentlig ut som en sagtannbølge:


%(END_ANSWER)





%(BEGIN_NOTES)

Be studentene beskrive hva som skjer med "sagtann"-bølgeformen hvis utstrømmingen (belastningen) plutselig øker. Endres frekvensen? Endres amplituden (høyden)? Endres gjennomsnittsverdien (DC-bias)?

Hva skjer hvis hysteresen i nivåbryteren økes? Hva tilsvarer dette i et P-regulatorsystem?

%INDEX% Control, basics: on/off control

%(END_NOTES)
