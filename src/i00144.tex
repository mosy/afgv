
%(BEGIN_QUESTION)
% Copyright 2006, Tony R. Kuphaldt, released under the Creative Commons Attribution License (v 1.0)
% This means you may do almost anything with this work of mine, so long as you give me proper credit

Identify and distinguish between {\it absolute} pressure, {\it gauge} pressure, and {\it differential} pressure.  Give at least one example of each kind of pressure.
Forklar forskjellen mellom absolutt trykk, relativt trykk og differanse trykk, gi også et eksempel på hver av dem. 

\underbar{file i00144}
%(END_QUESTION)





%(BEGIN_ANSWER)

{\it Absolute} pressure is the measurement of a pressure as compared to a pure vacuum.  Atmospheric (``barometric'') pressure, like the pressure figures reported by meteorologists, is an example of absolute pressure measurement.

{\it Gauge} pressure is the measurement of a pressure as compared to the pressure of Earth's atmosphere.  The pressure indicated by a pressure gauge (like an oil pressure gauge for a car engine, or a tire pressure gauge) is an example of gauge pressure.  When vented, such a gauge will register zero, even though there is still absolute pressure all around us due to Earth's atmosphere.

{\it Differential} pressure is the measurement of a difference between two different pressures.  In essence, all pressure measurements are differential in nature: notice how {\it absolute} and {\it gauge} pressures are defined in terms of a comparison of one pressure against another!

Suffixes are sometimes appended to pressure units to distinguish between absolute (A), gauge (G), and differential (D) pressures.  For example, you might see an absolute pressure represented as ``150 PSIA'', a gauge pressure as ``35 PSIG'', or a differential pressure as ``86.5 PSID''.  If no such suffix is given, the pressure unit is assumed to be {\it gauge}.

Some units of pressure measurement are {\it always} absolute, never gauge or differential.  These units include the {\it atmosphere} (14.7 PSIA), the {\it bar} (very close to 1 atmosphere -- think of it as a ``metric'' atmosphere), and the {\it torr}, which is absolute millimeters of mercury column.

%(END_ANSWER)





%(BEGIN_NOTES)


%INDEX% Physics, static fluids: absolute, gauge, and differential pressures

%(END_NOTES)


