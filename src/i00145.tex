%(BEGIN_QUESTION)
En dykker sin lufttak har et trykk på 300 bar før han går i vannet. Dykkeren dykkern den til en dybde på 20m. Her er trykket som som forårsakes av vekten til vannet 2bar, også kalt hydrostatisk trykk. Anta at mengden luft dykkern bruke på vei ned er å liten at den er uten betydning. Regn ut følgende trykk for tanken:
\begin{itemize}
	\item absolutt trykk
	\item relativt trykk
	\item differansetrykk (i forhold til vannet utenfor tanken)
\end{itemize}

\underbar{file i00145}
%(END_QUESTION)





%(BEGIN_ANSWER)

Absolute pressure = 2,014.7 PSIA.  Gauge pressure = 2,000 PSIG.  Differential pressure (between tank and water) = 1,978 PSID.

\vskip 10pt

Gauge pressure is simple: it is the figure initially measured by the pressure gauge (2,000 PSIG).  Again, we are assuming that the diver has not significantly decreased the tank's air pressure by consuming air from it as he or she descended to the specified depth.  In reality, the pressure in the tank would have decreased a bit in supplying the diver with air to breathe during the descent time.

Absolute pressure is simply gauge pressure added to the pressure of Earth's atmosphere.  Since the gauge pressure measured at the water's surface was (obviously) at sea level, and atmospheric pressure at sea level is approximately 14.7 PSIG, absolute air pressure inside the tank is 2,000 PSI + 14.7 PSI = 2,014.7 PSIG.

Differential pressure is simply the difference (subtraction) between the tank's gauge pressure of 2,000 PSI and the water's hydrostatic pressure (gauge) of 22 PSI.  This is equal to 1,978 PSID.  The same differential figure will be found even if atmospheric pressure is taken into consideration: the tank's absolute air pressure is 2,014.7 PSIA and the water's hydrostatic pressure is 36.7 PSIA (22 PSI + 14.7 PSI), resulting in a difference that is still 1,978 PSID.  The key here in figuring differential pressure is to always keep pressure units the same: don't mix gauge and absolute pressures!

%(END_ANSWER)





%(BEGIN_NOTES)

%INDEX% Physics, static fluids: absolute, gauge, and differential pressures

%(END_NOTES)


