
%(BEGIN_QUESTION)
% Copyright 2009, Tony R. Kuphaldt, released under the Creative Commons Attribution License (v 1.0)
% This means you may do almost anything with this work of mine, so long as you give me proper credit

Read and outline the ``Systems of Pressure Measurement'' subsection of the ``Fluid Mechanics'' section of the ``Physics'' chapter in your {\it Lessons In Industrial Instrumentation} textbook, paying close attention to the use of ``unity fractions'' for cancellation of units, and how to manage conversions between units of pressure measurement that do not share the same zero point.  Then, use that same mathematical technique to convert between the following units of pressure using the following gauge pressure equivalencies:

\vskip 10pt

\noindent
{\it 1.000 pound per square inch (PSI) = 2.036 inches of mercury (in. Hg) = 27.68 inches of water (in. W.C.) = 6.895 kilo-pascals (kPa) = 0.06895 bar}

$$\left(25 \hbox{ PSI} \over 1 \right) \left(\hbox{\hskip 50pt} \over \hbox{\hskip 50pt}\right) = \hbox{ ??? kPa} $$

$$\left(40 \hbox{ "WC} \over 1 \right) \left(\hbox{\hskip 50pt} \over \hbox{\hskip 50pt}\right) = \hbox{ ??? PSI} $$

$$\left(5.6 \hbox{ bar} \over 1 \right) \left(\hbox{\hskip 50pt} \over \hbox{\hskip 50pt}\right) = \hbox{ ??? PSI} $$

$$\left(1200 \hbox{ "Hg} \over 1 \right) \left(\hbox{\hskip 50pt} \over \hbox{\hskip 50pt}\right) = \hbox{ ??? "WC} $$

$$\left(12 \hbox{ "WC} \over 1 \right) \left(\hbox{\hskip 50pt} \over \hbox{\hskip 50pt}\right) = \hbox{ ??? bar} $$

$$\left(110 \hbox{ kPa} \over 1 \right) \left(\hbox{\hskip 50pt} \over \hbox{\hskip 50pt}\right) = \hbox{ ??? "WC} $$

$$\left(982 \hbox{ "Hg} \over 1 \right) \left(\hbox{\hskip 50pt} \over \hbox{\hskip 50pt}\right) = \hbox{ ??? kPa} $$

$$\left(50 \hbox{ PSI} \over 1 \right) \left(\hbox{\hskip 50pt} \over \hbox{\hskip 50pt}\right) = \hbox{ ??? bar} $$

$$\left(250 \hbox{ kPa} \over 1 \right) \left(\hbox{\hskip 50pt} \over \hbox{\hskip 50pt}\right) = \hbox{ ??? "Hg} $$

$$\left(31 \hbox{ bar} \over 1 \right) \left(\hbox{\hskip 50pt} \over \hbox{\hskip 50pt}\right) = \hbox{ ??? "Hg} $$

\vskip 10pt

For more information on the ``unity fraction'' method of unit conversion, refer to the ``Unity Fractions" subsection of the ``Unit Conversions and Physical Constants'' section of the ``Physics'' chapter in your {\it Lessons In Industrial Instrumentation} textbook.

\vskip 20pt \vbox{\hrule \hbox{\strut \vrule{} {\bf Suggestions for Socratic discussion} \vrule} \hrule}

\begin{itemize}
\item{} Identify some pressure units that are always absolute and never gauge.
\item{} Demonstrate how to {\it estimate} numerical answers for these conversion problems without using a calculator.
\item{} Explain why your ears ``pop'' when ascending or descending through large changes in altitude.
\item{} Explain why a SCUBA diver is not crushed by the pressure of water as he or she descends.
\end{itemize}

\underbar{file i00146}
%(END_QUESTION)





%(BEGIN_ANSWER)

\noindent
{\bf Partial answer:}

\vskip 10pt

Note how each and every ``unity fraction'' is comprised of {\it physically equal} pressures, in order to have a physical value of one (i.e. unity).  We purposely arrange the units in the numerator and denominator of each unity fraction in such a way that the original unit gets canceled out and replaced by the desired unit:

\vskip 20pt

\noindent
{\it Here, the unity fraction is made from the equivalence 6.895 kPa = 1 PSI:}

$$\left(25 \hbox{ PSI} \over 1 \right) \left(6.895 \hbox{ kPa} \over 1 \hbox{ PSI}\right) = 172.4 \hbox{ kPa} $$

\vskip 20pt

\noindent
{\it Here, the unity fraction is made from the equivalence 0.06895 bar = 27.68 "WC:}

$$\left(12 \hbox{ "WC} \over 1 \right) \left(0.06895 \hbox{ bar} \over 27.68 \hbox{ "WC}\right) = 0.02989 \hbox{ bar} $$

\vskip 20pt

\noindent
{\it Here, the unity fraction is made from the equivalence 6.895 kPa = 2.036 "Hg:}

$$\left(982 \hbox{ "Hg} \over 1 \right) \left(6.895 \hbox{ kPa} \over 2.036 \hbox{ "Hg}\right) = 3326 \hbox{ kPa} $$

\vskip 20pt

\noindent
{\it Here, the unity fraction is made from the equivalence 2.036 "Hg = 0.06895 bar:}

$$\left(31 \hbox{ bar} \over 1 \right) \left(2.036 \hbox{ "Hg} \over 0.06895 \hbox{ bar}\right) = 915.4 \hbox{ "Hg} $$

%(END_ANSWER)





%(BEGIN_NOTES)

Pressure may be expressed in ``gauge'' units: as the difference between the fluid in question and the surrounding environment.  Pressure may alternatively be expressed in ``absolute'' units: compared to a perfect vacuum.  The difference between gauge and absolute pressure is 14.7 PSI (the pressure of air at sea level).

\vskip 10pt

Using unity fractions for unit conversions: form a fraction of physically equal quantities, one expressed in the desired unit and the other expressed in the given unit, arranged such that when multiplied by the given quantity the undesired unit cancels and the desired unit remains.  If an offset is involved (e.g. PSIA versus PSIG), the addition/subtraction must be dealt with separately from the unity fraction (multiplication/division).

\vskip 10pt

Sometimes the distinction between gauge and absolute pressure is explicit (e.g. PSIA, PSIG), while other times it is implicit (e.g. atmospheres, torr are both always absolute).  The unit of the bar (100 kPa) may be gauge or absolute, and it is customary to {\it not} append a ``g'' or an ``a'' suffix!  With units such as PSI, the lack of a suffix is generally taken to mean gauge pressure.








$$\left(25 \hbox{ PSI} \over 1 \right) \left(6.895 \hbox{ kPa} \over 1 \hbox{ PSI}\right) = 172.4 \hbox{ kPa} $$

$$\left(40 \hbox{ "WC} \over 1 \right) \left(1 \hbox{ PSI} \over 27.68 \hbox{ "WC}\right) = 1.445 \hbox{ PSI} $$

$$\left(5.6 \hbox{ bar} \over 1 \right) \left(1 \hbox{ PSI} \over 0.06895 \hbox{ bar}\right) = 81.22 \hbox{ PSI} $$

$$\left(1200 \hbox{ "Hg} \over 1 \right) \left(27.68 \hbox{ "WC} \over 2.036 \hbox{ "Hg}\right) = 16314 \hbox{ "WC} $$

$$\left(12 \hbox{ "WC} \over 1 \right) \left(0.06895 \hbox{ bar} \over 27.68 \hbox{ "WC}\right) = 0.02989 \hbox{ bar} $$

$$\left(110 \hbox{ kPa} \over 1 \right) \left(27.68 \hbox{ "WC} \over 6.895 \hbox{ kPa}\right) = 441.6 \hbox{ "WC} $$

$$\left(982 \hbox{ "Hg} \over 1 \right) \left(6.895 \hbox{ kPa} \over 2.036 \hbox{ "Hg}\right) = 3326 \hbox{ kPa} $$

$$\left(50 \hbox{ PSI} \over 1 \right) \left(0.06895 \hbox{ bar} \over 1 \hbox{ PSI}\right) = 3.448 \hbox{ bar} $$

$$\left(250 \hbox{ kPa} \over 1 \right) \left(2.036 \hbox{ "Hg} \over 6.895 \hbox{ kPa}\right) = 73.82 \hbox{ "Hg} $$

$$\left(31 \hbox{ bar} \over 1 \right) \left(2.036 \hbox{ "Hg} \over 0.06895 \hbox{ bar}\right) = 915.4 \hbox{ "Hg} $$









\vfil \eject

\noindent
{\bf Prep Quiz:}

Suppose two pressure gauges are connected to an inflated automobile tire.  The first gauge registers in units PSIA, while the second gauge registers in units of PSIG.  Which statement is most accurate in describing how these two different gauges will behave?

\begin{itemize}
\item{} The PSIA gauge will register a greater value than the PSIG gauge
\vskip 5pt 
\item{} The PSIA gauge will register zero 
\vskip 5pt 
\item{} The PSIG gauge will register a greater value than the PSIA gauge 
\vskip 5pt 
\item{} The PSIG gauge will register zero 
\vskip 5pt 
\item{} Both gauges will register the same value
\vskip 5pt 
\item{} The PSIA gauge will register exactly twice the value of the PSIG gauge
\vskip 5pt 
\item{} The PSIG gauge will register exactly twice the value of the PSIA gauge
\end{itemize}









\vfil \eject

\noindent
{\bf Prep Quiz:}

Explain why we cannot convert between certain pressure units (such as torr and PSIG) {\it only} using unity fractions.


%INDEX% Physics, units and conversions: pressure

%(END_NOTES)


