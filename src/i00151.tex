
%(BEGIN_QUESTION)
% Copyright 2006, Tony R. Kuphaldt, released under the Creative Commons Attribution License (v 1.0)
% This means you may do almost anything with this work of mine, so long as you give me proper credit

A hydraulic system has two cylinders linked together with high-pressure tubing.  The piston diameter of cylinder \#1 is 3 inches, while the piston diameter of cylinder \#2 is 4.5 inches.  How much force will cylinder \#1's piston exert if cylinder \#2's piston is pushed with 200 pounds of force?  How much fluid pressure will be within the hydraulic tube with 200 pounds of force applied to the piston of cylinder \#2?

\vskip 10pt

Also, determine which of the two pistons will travel furthest, and explain why this is so.

\vskip 20pt \vbox{\hrule \hbox{\strut \vrule{} {\bf Suggestions for Socratic discussion} \vrule} \hrule}

\begin{itemize}
\item{} A useful problem-solving technique is to sketch a simple diagram of the system you are asked to analyze.  This is useful even when you already have some graphical representation of the problem given to you, as a simple sketch often reduces the complexity of the problem so that you can solve it more easily.  Draw your own sketch showing how the given information in this problem inter-relates, and use this sketch to explain your solution.
\end{itemize}

\underbar{file i00151}
%(END_QUESTION)





%(BEGIN_ANSWER)

The fluid pressure will be 39.5 PSI, and cylinder \#1's piston force will be 88.889 pounds. 

\vskip 10pt

Piston \#1 will travel further than piston \#2.

%(END_ANSWER)





%(BEGIN_NOTES)







\vfil \eject

\noindent
{\bf Prep Quiz:}

Suppose 800 PSI of hydraulic fluid pressure pushes against a round piston with a diameter of 2 inches.  Calculate the amount of force generated by this fluid pressure on the piston.

\begin{itemize}
\item{} 2,513.3 pounds
\vskip 5pt 
\item{} 254.6 pounds
\vskip 5pt 
\item{} 5,026.5 pounds
\vskip 5pt 
\item{} 800 pounds
\vskip 5pt 
\item{} 10,053.1 pounds
\vskip 5pt 
\item{} 1600 pounds
\end{itemize}


%INDEX% Physics, fluids: pressure, force, and area
%INDEX% Physics, static fluids: Pascal's Principle

%(END_NOTES)


