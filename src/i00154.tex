
%(BEGIN_QUESTION)
% Copyright 2006, Tony R. Kuphaldt, released under the Creative Commons Attribution License (v 1.0)
% This means you may do almost anything with this work of mine, so long as you give me proper credit

Answer the following four questions about deadweight testers:


\vskip 10pt {\narrower \noindent \baselineskip5pt

\noindent
(1) What is it about the nature of a deadweight tester that makes it so accurate and repeatable?  To phrase this question in the negative, what would have to change in order to affect the accuracy of a deadweight tester's output pressure?

\par} \vskip 10pt




\vskip 10pt {\narrower \noindent \baselineskip5pt

\noindent
(2) Why is it important for a deadweight tester to be {\it level} while it is being used to calibrate a pressure instrument?

\par} \vskip 10pt



\vskip 10pt {\narrower \noindent \baselineskip5pt

\noindent
(3) What effect will trapped air have inside a deadweight tester?

\par} \vskip 10pt




\vskip 10pt {\narrower \noindent \baselineskip5pt

\noindent
(4) Why is it advisable to {\it gently} spin the primary piston and weights while the piston is suspended by oil pressure?

\par} \vskip 10pt



\underbar{file i00154}
%(END_QUESTION)





%(BEGIN_ANSWER)

\vskip 10pt {\narrower \noindent \baselineskip5pt

\noindent
(1) The accuracy of a deadweight tester is fixed by three fundamental variables, all of which are quite constant, two of which can be manufactured to highly accurate specifications, and the third being a constant of nature:

\begin{itemize}
\item{} The mass of the calibration weights
\item{} The area of the primary piston
\item{} The gravity of the Earth
\end{itemize}

\par} \vskip 10pt




\vskip 10pt {\narrower \noindent \baselineskip5pt

\noindent
(2) If a deadweight is not level, the force generated by the precision weights will not be parallel to the primary piston's axis of travel, meaning that the piston will not support their full weight.

\par} \vskip 10pt




\vskip 10pt {\narrower \noindent \baselineskip5pt

\noindent
(3) Entrapped air will make the piston's motion ``springy'' rather than solid and secure.

\par} \vskip 10pt




\vskip 10pt {\narrower \noindent \baselineskip5pt

\noindent
(4) Spinning the primary piston eliminates static friction, leaving only dynamic friction (which is much less) to interfere with gravity's force on the primary piston.

\par} \vskip 10pt



%(END_ANSWER)





%(BEGIN_NOTES)


\vskip 10pt {\narrower \noindent \baselineskip5pt

\noindent
(2) If a deadweight is not level, the force generated by the precision weights will not be parallel to the primary piston's axis of travel, meaning that the piston will ``see'' less force than it should.  Imagine an object parked on an incline, and note how the force perpendicular to the incline's face is {\it less} than the vertical force (weight) of the object.  In the case of the deadweight tester, this will result in a pressure that is erroneously low.

\vskip 10pt

\noindent
Also, this will result in some amount of lateral force being applied to the piston rod bushing, resulting in friction that may similarly diminish the pressure generated.

\par} \vskip 10pt






\vskip 10pt {\narrower \noindent \baselineskip5pt

\noindent
(4) It should be mentioned that {\it gentle} spinning is all that is needed for the primary piston to overcome static friction.  Some students have the tendency to spin the weights as though they were a potter's wheel!

\par} \vskip 10pt

%INDEX% Calibration, deadweight tester: basic operation

%(END_NOTES)


