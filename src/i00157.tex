
%(BEGIN_QUESTION)
% Copyright 2008, Tony R. Kuphaldt, released under the Creative Commons Attribution License (v 1.0)
% This means you may do almost anything with this work of mine, so long as you give me proper credit

Convert between the following units of pressure.  Remember that any pressure unit not explicitly specified as either absolute (A) or differential (D) is to be considered {\it gauge}.  Also, remember those units which {\it always} represent absolute pressure, and have no need for a letter ``A'' suffix!

\medskip
{\item{} 25 PSIA = ??? atm
\vskip 5pt
{\item{} 340 "W.C. = ??? PSIA
\vskip 5pt
{\item{} 0.73 bar (gauge) = ??? "Hg
\vskip 5pt
{\item{} 5.5 atm = ??? torr
\vskip 5pt
{\item{} 2,300 cm Hg = ??? "W.C.A
\vskip 5pt
{\item{} 500 m torr = ??? PSIA
\vskip 5pt
{\item{} 91.2 cm W.C. = ??? kPa
\vskip 5pt
{\item{} 110 kPa = ??? "W.C.
\vskip 5pt
{\item{} 620 mm HgA = ??? torr
\vskip 5pt
{\item{} 77 Pa = ??? PSIA
\vskip 5pt
{\item{} 1 atm = ??? "W.C.A
\vskip 5pt
{\item{} 270 PSIA = ??? atm
\medskip

\vskip 10pt

There is a technique for converting between different units of measurement called ``unity fractions'' which is imperative for students of Instrumentation to master.  For more information on the ``unity fraction'' method of unit conversion, refer to the ``Unity Fractions" subsection of the ``Unit Conversions and Physical Constants'' section of the ``Physics'' chapter in your {\it Lessons In Industrial Instrumentation} textbook.

\vskip 20pt \vbox{\hrule \hbox{\strut \vrule{} {\bf Suggestions for Socratic discussion} \vrule} \hrule}

\begin{itemize}
\item{} Which of these conversions require an additive or subtractive offset, and which of these may be performed using multiplication and division alone?
\item{} Demonstrate how to {\it estimate} numerical answers for these conversion problems without using a calculator.
\item{} Suppose a novice tries to convert 3.5 atmospheres into PSIG, and arrives at a result of 51.45 PSIG.  Identify the mistake made here, and also the proper conversion to go from units of atmospheres to PSIG.
\end{itemize}

\underbar{file i00157}
%(END_QUESTION)





%(BEGIN_ANSWER)

\medskip
{\item{} 25 PSIA = 1.701 atm
\vskip 5pt
{\item{} 340 "W.C. = 26.983 PSIA
\vskip 5pt
{\item{} 0.73 bar (gauge) = 21.557 "Hg
\vskip 5pt
{\item{} 5.5 atm = 4,180 torr
\vskip 5pt
{\item{} 2,300 cm Hg = 12,717.72 "W.C.A
\vskip 5pt
{\item{} 500 m torr = 0.0096683 PSIA
\vskip 5pt
{\item{} 91.2 cm W.C. = 8.9434 kPa
\vskip 5pt
{\item{} 110 kPa = 441.62 "W.C.
\vskip 5pt
{\item{} 620 mm HgA = 620 torr {\it (A ``trick'' question . . .)}
\vskip 5pt
{\item{} 77 Pa = 14.711168 PSIA
\vskip 5pt
{\item{} 1 atm = 406.91 "W.C.A
\vskip 5pt
{\item{} 270 PSIA = 18.367 atm
\medskip

%(END_ANSWER)





%(BEGIN_NOTES)

Conversion equivalencies used to obtain answers:

\vskip 10pt

\vskip 10pt {\narrower \noindent \baselineskip5pt

1 pound per square inch (PSI) = 2.03603 inches of mercury (in. Hg) = 27.6807 inches of water (in. W.C.) = 6894.757 Pascals (Pa) = 6.894757 kilopascals (kPa) = 0.0680460 atmospheres (Atm) = 0.0689476 bar (bar)

\par} \vskip 10pt

\vskip 10pt

Approximate answers may be obtained by rounding the above factors to four significant figures:

\vskip 10pt

\vskip 10pt {\narrower \noindent \baselineskip5pt

1 pound per square inch (PSI) = 2.036 inches of mercury (in. Hg) = 27.68 inches of water (in. W.C.) = 6895 Pascals (Pa) = 6.895 kilopascals (kPa) = 0.068 atmospheres (Atm) = 0.069 bar (bar)

\par} \vskip 10pt

\vskip 10pt

(25 PSIA)/(1 atm / 14.7 PSIA) = {\bf 1.701 atm}

\vskip 10pt

(340 "W.C)(1 PSI / 27.6807 "W.C) = 12.283 PSI 

12.283 PSI + 14.7 PSI = {\bf 26.983 PSIA}

\vskip 10pt

(0.73 bar)(100 kPa / 1 bar)(2.03603 "Hg / 6.894757 kPa) = {\bf 21.557 "Hg}

\vskip 10pt

(5.5 atm)(760 torr / 1 atm) = {\bf 4,180 torr}

\vskip 10pt

(2,300 cm Hg)(1 inch / 2.54 cm)(1 PSI / 2.03603 "Hg) = 444.74 PSI

444.74 PSI + 14.7 PSI = 459.44 PSIA

(459.44 PSIA)(27.6807 "W.C. / 1 PSI) = {\bf 12,717.72 "W.C.A}

\vskip 10pt

500 m torr = 0.5 mm HgA

(0.5 mm HgA)(1 inch / 25.4 mm)(1 PSI / 2.03603 "Hg) = {\bf 0.0096683 PSIA}

\vskip 10pt

(91.2 cm W.C.)(1 inch / 2.54 cm)(6.894757 kPa / 27.6807 "W.C.) = {\bf 8.9434 kPa}

\vskip 10pt

(110 kPa)(27.6807 "W.C. / 6.894757 kPa) = {\bf 441.62 "W.C.}

\vskip 10pt

620 mm HgA = {\bf 620 torr}

{\it Ha!  And you thought this was tougher than it looked!}

\vskip 10pt

(77 Pa)(1 PSI / 6,894.757 Pa) = 0.011168 PSI

0.011168 PSI + 14.7 PSI = {\bf 14.711168 PSIA}

\vskip 10pt

(1 atm)(14.7 PSIA / 1 atm)(27.6807 "W.C. / 1 PSI) = {\bf 406.91 "W.C.A}

\vskip 10pt

(270 PSIA)(1 atm / 14.7 PSIA) = {\bf 18.367 atm}











\vfil \eject

\noindent
{\bf Prep Quiz:}

Convert a pressure of 233 inches water column ("W.C.) into inches mercury column ("Hg):

\begin{itemize}
\item{} 8.4176 "Hg
\vskip 5pt 
\item{} 114.44 "Hg
\vskip 5pt 
\item{} 3167.7 "Hg
\vskip 5pt 
\item{} 33.792 "Hg
\vskip 5pt 
\item{} 17.138 "Hg
\vskip 5pt 
\item{} 1606.5 "Hg
\end{itemize}



\vfil \eject

\noindent
{\bf Prep Quiz:}

Convert a pressure of 24 inches mercury ("Hg) into inches water column ("W.C.):

\begin{itemize}
\item{} 326.3 "W.C.
\vskip 5pt 
\item{} 2.00 "W.C.
\vskip 5pt 
\item{} 1.765 "W.C.
\vskip 5pt 
\item{} 11.79 "W.C.
\vskip 5pt 
\item{} 288 "W.C.
\vskip 5pt 
\item{} 0.867 "W.C.
\end{itemize}


%INDEX% Physics, units and conversions: pressure

%(END_NOTES)


