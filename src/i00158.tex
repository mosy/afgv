
%(BEGIN_QUESTION)
% Copyright 2015, Tony R. Kuphaldt, released under the Creative Commons Attribution License (v 1.0)
% This means you may do almost anything with this work of mine, so long as you give me proper credit

Convert between the following units of pressure.  Remember that any pressure unit not explicitly specified as either absolute (A) or differential (D) is to be considered {\it gauge}.  Also, remember those units which {\it always} represent absolute pressure, and have no need for a letter ``A'' suffix!

\medskip
{\item{} 5 PSI vacuum = ??? PSIA
\vskip 5pt
{\item{} 25 "Hg vacuum = ??? PSIA
\vskip 5pt
{\item{} 2,800 $\mu$ torr = ??? PaA
\vskip 5pt
{\item{} $-59$ "W.C. = ??? torr
\vskip 5pt
{\item{} 4,630 PaA = ??? PSI
\vskip 5pt
{\item{} 0.05 atm = ??? "W.C.
\vskip 5pt
{\item{} $-3$ kPa = ??? atm
\vskip 5pt
{\item{} 10 feet W.C. vacuum = ??? "HgA
\vskip 5pt
{\item{} 300 cm Hg = ??? atm
\vskip 5pt
{\item{} $-2$ mm W.C. = ??? bar (absolute)
\vskip 5pt
{\item{} 4 atm = ??? "W.C.A
\medskip

\vskip 10pt

There is a technique for converting between different units of measurement called ``unity fractions'' which is imperative for students of Instrumentation to master.  For more information on the ``unity fraction'' method of unit conversion, refer to the ``Unity Fractions" subsection of the ``Unit Conversions and Physical Constants'' section of the ``Physics'' chapter in your {\it Lessons In Industrial Instrumentation} textbook.


\vskip 20pt \vbox{\hrule \hbox{\strut \vrule{} {\bf Suggestions for Socratic discussion} \vrule} \hrule}

\begin{itemize}
\item{} Which of these conversions require an additive or subtractive offset, and which of these may be performed using multiplication and division alone?
\item{} Demonstrate how to {\it estimate} numerical answers for these conversion problems without using a calculator.
\end{itemize}


\underbar{file i00158}
%(END_QUESTION)





%(BEGIN_ANSWER)

\medskip
{\item{} 5 PSI vacuum = 9.7 PSIA
\vskip 5pt
{\item{} 25 "Hg vacuum = 2.421 PSIA
\vskip 5pt
{\item{} 2,800 $\mu$ torr = 0.3733 PaA
\vskip 5pt
{\item{} $-59$ "W.C. = 649.98 torr
\vskip 5pt
{\item{} 4,630 PaA = $-14.028$ PSI
\vskip 5pt
{\item{} 0.05 atm = $-386.56$ "W.C.
\vskip 5pt
{\item{} $-3$ kPa = 0.9704 atm
\vskip 5pt
{\item{} 10 feet W.C. vacuum = 21.103 "HgA
\vskip 5pt
{\item{} 300 cm Hg = 4.946 atm
\vskip 5pt
{\item{} $-2$ mm W.C. = 1.0133 bar (absolute)
\vskip 5pt
{\item{} 4 atm = 1,627.63 "W.C.A
\medskip

%(END_ANSWER)





%(BEGIN_NOTES)


5 PSI vacuum = $-5$ PSI

$-5$ PSI + 14.7 PSI = {\bf 9.7 PSIA}


\vskip 10pt


25 "Hg vacuum = $-25$ "Hg

($-25$ "Hg)(1 PSI / 2.03603 "Hg) = $-12.279$ PSI

$-12.279$ PSI + 14.7 PSI = {\bf 2.421 PSIA}


\vskip 10pt


2,800 $\mu$ torr = 2800 $\times$ $10^{-6}$ mm HgA = 0.0028 mm HgA

(0.0028 mm HgA)(1 inch / 25.4 mm)(6,894.757 Pa / 2.03603 "Hg) = {\bf 0.3733 PaA}


\vskip 10pt


($-59$ "W.C.)(1 PSI / 27.6807 "W.C.) = $-2.131$ PSI

$-2.131$ PSI + 14.7 PSI = 12.569 PSIA

(12.569 PSIA)(2.03603 "Hg / 1 PSI)(25.4 mm / 1 inch) = 649.98 mm HgA = {\bf 649.98 torr}


\vskip 10pt


(4,630 PaA)(1 PSI / 6894.757 Pa) = 0.6715 PSIA

0.6715 PSIA $-$ 14.7 PSI = {\bf $-$14.028 PSIG}, or {\bf 14.028 PSI vacuum}


\vskip 10pt


(0.05 atm)(14.7 PSIA / 1 atm) = 0.735 PSIA

0.735 PSIA $-$ 14.7 PSI = $-13.965$ PSIG, or 13.965 PSI vacuum

($-13.965$ PSI)(27.6807 "W.C. / 1 PSI) = {\bf $-386.56$ "W.C.}


\vskip 10pt


($-3$ kPa)(1 PSI / 6.894757 kPa) = $-0.4351$ PSI

$-0.4351$ PSI + 14.7 PSI = 14.2649 PSIA

(14.2649 PSIA)(1 atm / 14.7 PSIA) = {\bf 0.9704 atm}


\vskip 10pt


10 feet W.C. vacuum = $-10$ feet W.C.

($-10$ feet W.C.)(12 inches / 1 foot)(1 PSI / 27.6807 "W.C.) = $-4.3352$ PSI

$-4.3352$ PSI + 14.7 PSI = 10.3648 PSIA

(10.3648 PSIA)(2.03603 "Hg / 1 PSI) = {\bf 21.103 "HgA}


\vskip 10pt


(300 cm Hg)(1 inch / 2.54 cm)(1 PSI / 2.03603 "Hg) = 58.01 PSI

58.01 PSI + 14.7 PSI = 72.71 PSIA

(72.71 PSIA)(1 atm / 14.7 PSIA) = {\bf 4.946 atm}


\vskip 10pt


($-2$ mm W.C.)(1 inch / 25.4 mm)(1 PSI / 27.6807 "W.C.) = $-0.0028446$ PSI

$-0.0028446$ PSI + 14.7 PSI = 14.69715 PSIA

(14.69715 PSIA)(6.894757 kPa / 1 PSI)(1 bar / 100 kPa) = {\bf 1.0133 kPaA}


\vskip 10pt


(4 atm)(14.7 PSIA / 1 atm)(27.6807 "W.C. / 1 PSI) = {\bf 1,627.63 "W.C.A}



%INDEX% Physics, units and conversions: pressure

%(END_NOTES)


