
%(BEGIN_QUESTION)
% Copyright 2011, Tony R. Kuphaldt, released under the Creative Commons Attribution License (v 1.0)
% This means you may do almost anything with this work of mine, so long as you give me proper credit

The manufacturing company you work for installs a PLC control system on its assembly line, counting the number of components produced every shift.  For quite a while, the system works without any problems whatsoever, and then one day management decides to scrap a run of product mid-shift and start over.  This is when they discover the system integrator they contracted to build and program the PLC system provided no way to reset the shift production counter except to wait until the shift is over.

\vskip 10pt

An operations manager summons you to reset the counter for them.  Identify at least two different ways you could reset the counter to meet their needs, as quickly as possible.

\underbar{file i00182}
%(END_QUESTION)





%(BEGIN_ANSWER)


%(END_ANSWER)





%(BEGIN_NOTES)

You may connect to the PLC with a laptop PC and do the following:

\begin{itemize}
\item{} Force the counter's live value to zero, {\it or}
\item{} Force the counter's reset bit high, {\it or}
\item{} Program an input contact instruction to reset the counter whenever that input is jumpered with power
\end{itemize}

%INDEX% PLC, I/O: forcing bits

%(END_NOTES)


