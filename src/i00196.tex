
%(BEGIN_QUESTION)
% Copyright 2011, Tony R. Kuphaldt, released under the Creative Commons Attribution License (v 1.0)
% This means you may do almost anything with this work of mine, so long as you give me proper credit

It is customary when using a rupture disk in conjunction with a pressure safety valve to locate the disk between the PSV and the process vessel.  Explain the purpose for locating the rupture disk there, and why it might be a very bad idea to locate the PSV between the rupture disk and the vessel.

\underbar{file i00196}
%(END_QUESTION)





%(BEGIN_ANSWER)

Rupture disks are sometimes inserted between the PSV and the process vessel in order to stop fugitive emissions (because all valves will leak some finite amount during normal operation, whereas an intact rupture disk cannot leak).  This same leaking problem explains why one should never place a rupture disk on the outlet of a pressure safety valve.

\vskip 10pt

To illustrate the problem, imagine a PSV set to lift at 800 PSI, and a rupture disk with the same burst pressure rating placed at the outlet of the PSV.  Now imagine that PSV leaking a bit over time, slowly pressurizing the pipe between the PSV and the rupture disk until that enclosed space holds 500 PSI of vapor.  Now, how much process pressure will it take to lift the PSV?  Remember that PSVs, as self-actuated devices, respond to the pressure {\it difference} between inlet and outlet!

%(END_ANSWER)





%(BEGIN_NOTES)


%INDEX% Safety, overpressure protection: rupture disk
%INDEX% Safety, overpressure protection: safety valve

%(END_NOTES)


