
%(BEGIN_QUESTION)
% Copyright 2006, Tony R. Kuphaldt, released under the Creative Commons Attribution License (v 1.0)
% This means you may do almost anything with this work of mine, so long as you give me proper credit

Why is {\it temperature} an important factor in pressure transmitters equipped with remote seals (sometimes called ``chemical'' seals)?  Specifically, what temperature condition(s) could cause a differential pressure instrument with remote seals to experience measurement errors?

\vskip 10pt

Also, describe what would happen to a pressure instrument equipped with remote seals if ever a leak developed in one of the capillary tubes.

\underbar{file i00217}
%(END_QUESTION)





%(BEGIN_ANSWER)

Increased temperature causes the fill liquid in the capillary tube to expand, thus creating a pressure at the transmitter's sensing element independent of the process fluid pressure being measured.

If the seal-equipped transmitter is differential in nature, and the temperatures of both seal units and capillary tubes are equal, then any temperature-induced pressures will cancel out at the transmitter, and there will be no problem.  However, if the temperature of one seal or capillary is greater than the other, there will be a {\it differential} pressure induced by temperature that the transmitter will detect and interpret as process fluid differential pressure.

\vskip 10pt

Remote seal/capillary systems must be completely gas-free (nothing but liquid inside) in order to work.  If even a small air or other gas bubble works its way in to the fill fluid between the remote seal diaphragm and the instrument sensing element, the fill fluid will be compressible, meaning that the seal diaphragm will displace a greater volume of fill fluid than the instrument's sensing element.  Thus, the instrument may not ``see'' the entire amount of process pressure change applied to the seal diaphragm, and its measurement accuracy will be compromised.

In summary, leaks in a remote seal system have {\it very detrimental effects} on instrument accuracy.  If ever such a system develops a leak, the whole instrument must be replaced.  Even if you could patch the leak, you would have to ``pack'' the seal unit, capillary tubing, and transmitter pressure housing with new fill liquid under a strong vacuum to ``pull'' any dissolved gas bubbles out of the liquid.  Needless to say, the facilities required for such an operation are typically not available in a plant instrument shop, thus rendering the instrument unrepairable by you.

%(END_ANSWER)





%(BEGIN_NOTES)


%INDEX% Measurement, pressure: chemical seals
%INDEX% Measurement, pressure: remote seals

%(END_NOTES)


