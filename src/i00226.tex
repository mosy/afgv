
%(BEGIN_QUESTION)
% Copyright 2006, Tony R. Kuphaldt, released under the Creative Commons Attribution License (v 1.0)
% This means you may do almost anything with this work of mine, so long as you give me proper credit

Convert between the following units of pressure:

\begin{itemize}
\item{} 22 PSI = ??? PSIA
\vskip 5pt
\item{} 13 kPa = ??? "W.C.
\vskip 5pt
\item{} 81 kPa = ??? PSI
\vskip 5pt
\item{} 5 atm = ??? PSIA
\vskip 5pt
\item{} 200 "Hg = ??? "W.C.
\vskip 5pt
\item{} 17 feet W.C. = ??? "Hg
\vskip 5pt
\item{} 8 PSI vacuum = ??? PSIA
\vskip 5pt
\item{} 900 Torr = ??? "W.C.A
\vskip 5pt
\item{} 300 mm Hg = ??? PSI
\vskip 5pt
\item{} 250 "W.C. = ??? bar (gauge)
\vskip 5pt
\item{} 70 "W.C. = ??? "Hg
\vskip 5pt
\item{} 300 PSIG = ??? atm
\end{itemize}

\vskip 10pt

There is a technique for converting between different units of measurement called ``unity fractions'' which is imperative for students of Instrumentation to master.  For more information on the ``unity fraction'' method of unit conversion, refer to the ``Unity Fractions" subsection of the ``Unit Conversions and Physical Constants'' section of the ``Physics'' chapter in your {\it Lessons In Industrial Instrumentation} textbook.

\vskip 20pt \vbox{\hrule \hbox{\strut \vrule{} {\bf Suggestions for Socratic discussion} \vrule} \hrule}

\begin{itemize}
\item{} Demonstrate how to {\it estimate} numerical answers for these conversion problems without using a calculator.
\end{itemize}

\underbar{file i00226}
%(END_QUESTION)





%(BEGIN_ANSWER)

\begin{itemize}
\item{} 22 PSIG = 36.7 PSIA
\vskip 5pt
\item{} 13 kPa = 52.19 "W.C.
\vskip 5pt
\item{} 81 kPa = 11.75 PSI
\vskip 5pt
\item{} 5 atm = 73.5 PSIA
\vskip 5pt
\item{} 200 "Hg = 2719 "W.C.
\vskip 5pt
\item{} 17 feet W.C. = 15.01 "Hg
\vskip 5pt
\item{} 8 PSI vacuum = 6.7 PSIA
\vskip 5pt
\item{} 900 Torr = 481.9 "W.C.A
\vskip 5pt
\item{} 300 mm Hg = 5.801 PSI
\vskip 5pt
\item{} 250 "W.C. = 0.6227 bar (gauge)
\vskip 5pt
\item{} 70 "W.C. = 5.149 "Hg
\vskip 5pt
\item{} 300 PSIG = 21.41 atm
\end{itemize}

%(END_ANSWER)





%(BEGIN_NOTES)


%INDEX% Physics, units and conversions: pressure

%(END_NOTES)


