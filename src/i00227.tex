
%(BEGIN_QUESTION)
% Copyright 2011, Tony R. Kuphaldt, released under the Creative Commons Attribution License (v 1.0)
% This means you may do almost anything with this work of mine, so long as you give me proper credit

Suppose you wish to calibrate a pneumatic pressure transmitter to an input range of 0 to 200 inches of water, with an output range of 3 to 15 PSI.  Complete the following calibration table showing the test pressures to use and the allowable low/high output signals for a calibrated tolerance of +/- 0.5\% (of span):

% No blank lines allowed between lines of an \halign structure!
% I use comments (%) instead, so that TeX doesn't choke.

$$\vbox{\offinterlineskip
\halign{\strut
\vrule \quad\hfil # \ \hfil & 
\vrule \quad\hfil # \ \hfil & 
\vrule \quad\hfil # \ \hfil & 
\vrule \quad\hfil # \ \hfil & 
\vrule \quad\hfil # \ \hfil \vrule \cr
\noalign{\hrule}
%
% First row
Input pressure & Percent of span & Output signal & Output signal & Output signal \cr
%
% Another row
applied (" W.C.) & (\%) & {\it ideal} (PSI) & {\it low} (PSI) & {\it high} (PSI) \cr
%
\noalign{\hrule}
%
% Another row
 & 0 & & & \cr
%
\noalign{\hrule}
%
% Another row
 & 25 & & & \cr
%
\noalign{\hrule}
%
% Another row
 & 50 & & & \cr
%
\noalign{\hrule}
%
% Another row
 & 75 & & & \cr
%
\noalign{\hrule}
%
% Another row
 & 100 & & & \cr
%
\noalign{\hrule}
} % End of \halign 
}$$ % End of \vbox

Suppose this transmitter is installed as part of a complete pressure measurement system (transmitter plus remote indicator and associated components), and the entire measurement system has been calibrated within the specified tolerance ($\pm$ 0.5\%) from beginning to end.  If the operator happens to read a process pressure of 153 inches W.C. at the indicator, how far off might the actual process pressure be from this indicated value?

\vskip 20pt \vbox{\hrule \hbox{\strut \vrule{} {\bf Suggestions for Socratic discussion} \vrule} \hrule}

\begin{itemize}
\item{} Demonstrate how to {\it estimate} numerical answers for this problem without using a calculator.
\end{itemize}

\underbar{file i00227}
%(END_QUESTION)





%(BEGIN_ANSWER)

% No blank lines allowed between lines of an \halign structure!
% I use comments (%) instead, so that TeX doesn't choke.

$$\vbox{\offinterlineskip
\halign{\strut
\vrule \quad\hfil # \ \hfil & 
\vrule \quad\hfil # \ \hfil & 
\vrule \quad\hfil # \ \hfil & 
\vrule \quad\hfil # \ \hfil & 
\vrule \quad\hfil # \ \hfil \vrule \cr
\noalign{\hrule}
%
% First row
Input pressure & Percent of span & Output signal & Output signal & Output signal \cr
%
% Another row
applied (PSI) & (\%) & {\it ideal} (PSI) & {\it low} (PSI) & {\it high} (PSI) \cr
%
\noalign{\hrule}
%
% Another row
0 & 0 & 3 & 2.94 & 3.06 \cr
%
\noalign{\hrule}
%
% Another row
50 & 25 & 6 & 5.94 & 6.06 \cr
%
\noalign{\hrule}
%
% Another row
100 & 50 & 9 & 8.94 & 9.06 \cr
%
\noalign{\hrule}
%
% Another row
150 & 75 & 12 & 11.94 & 12.06 \cr
%
\noalign{\hrule}
%
% Another row
200 & 100 & 15 & 14.94 & 15.06 \cr
%
\noalign{\hrule}
} % End of \halign 
}$$ % End of \vbox

Given the tolerance of $\pm$ 0.5\% of the 200" span ($\pm$ 1"), the actual process pressure could be as low as 152 "W.C. or as high as 154 "W.C.

%(END_ANSWER)





%(BEGIN_NOTES)


%INDEX% Calibration: table, pressure transmitter
%INDEX% Measurement, pressure: calibration table

%(END_NOTES)


