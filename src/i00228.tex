
%(BEGIN_QUESTION)
% Copyright 2006, Tony R. Kuphaldt, released under the Creative Commons Attribution License (v 1.0)
% This means you may do almost anything with this work of mine, so long as you give me proper credit

Suppose you wish to calibrate an electronic pressure transmitter to an input range of -50 to 300 inches of water, with an output range of 4 to 20 mA.  Complete the following calibration table showing the test pressures to use and the allowable low/high output signals for a calibrated tolerance of +/- 0.1\% (of span).  Assume you can only use positive test pressures (no vacuum), and be sure to designate which side the test pressure should be applied to (H = high ; L = low):

% No blank lines allowed between lines of an \halign structure!
% I use comments (%) instead, so that TeX doesn't choke.

$$\vbox{\offinterlineskip
\halign{\strut
\vrule \quad\hfil # \ \hfil & 
\vrule \quad\hfil # \ \hfil & 
\vrule \quad\hfil # \ \hfil & 
\vrule \quad\hfil # \ \hfil & 
\vrule \quad\hfil # \ \hfil \vrule \cr
\noalign{\hrule}
%
% First row
Input pressure & Percent of span & Output signal & Output signal & Output signal \cr
%
% Another row
applied (" W.C.) & (\%) & {\it ideal} (mA) & {\it low} (mA) & {\it high} (mA) \cr
%
\noalign{\hrule}
%
% Another row
 & 0 & & & \cr
%
\noalign{\hrule}
%
% Another row
 & 25 & & & \cr
%
\noalign{\hrule}
%
% Another row
 & 50 & & & \cr
%
\noalign{\hrule}
%
% Another row
 & 75 & & & \cr
%
\noalign{\hrule}
%
% Another row
 & 100 & & & \cr
%
\noalign{\hrule}
} % End of \halign 
}$$ % End of \vbox

Suppose this transmitter is installed as part of a complete pressure measurement system (transmitter plus remote indicator and associated components), and the entire measurement system has been calibrated within the specified tolerance ($\pm$ 0.1\%) from beginning to end.  If the operator happens to read a process pressure of 210 inches W.C. at the indicator, how far off might the actual process pressure be from this indicated value?

\vskip 20pt \vbox{\hrule \hbox{\strut \vrule{} {\bf Suggestions for Socratic discussion} \vrule} \hrule}

\begin{itemize}
\item{} Demonstrate how to {\it estimate} numerical answers for this problem without using a calculator.
\end{itemize}

\underbar{file i00228}
%(END_QUESTION)





%(BEGIN_ANSWER)

% No blank lines allowed between lines of an \halign structure!
% I use comments (%) instead, so that TeX doesn't choke.

$$\vbox{\offinterlineskip
\halign{\strut
\vrule \quad\hfil # \ \hfil & 
\vrule \quad\hfil # \ \hfil & 
\vrule \quad\hfil # \ \hfil & 
\vrule \quad\hfil # \ \hfil & 
\vrule \quad\hfil # \ \hfil \vrule \cr
\noalign{\hrule}
%
% First row
Input pressure & Percent of span & Output signal & Output signal & Output signal \cr
%
% Another row
applied (" W.C.) & (\%) & {\it ideal} (mA) & {\it low} (mA) & {\it high} (mA) \cr
%
\noalign{\hrule}
%
% Another row
50 L & 0 & 4 & 3.984 & 4.016 \cr
%
\noalign{\hrule}
%
% Another row
37.5 H & 25 & 8 & 7.984 & 8.016 \cr
%
\noalign{\hrule}
%
% Another row
125 H & 50 & 12 & 11.984 & 12.016 \cr
%
\noalign{\hrule}
%
% Another row
212.5 H & 75 & 16 & 15.984 & 16.016 \cr
%
\noalign{\hrule}
%
% Another row
350 H & 100 & 20 & 19.984 & 20.016 \cr
%
\noalign{\hrule}
} % End of \halign 
}$$ % End of \vbox


Given the tolerance of $\pm$ 0.1\% of the 350" span ($\pm$ 0.35"), the actual process pressure could be as low as 209.65 "W.C. or as high as 210.35 "W.C.

%(END_ANSWER)





%(BEGIN_NOTES)








\vfil \eject

\noindent
{\bf Summary Quiz:}

Suppose a pressure indicator has a range of 50 to 250 PSI.  If it has been calibrated to a tolerance of $\pm$ 1\% of span, and it happens to register a process pressure of 137 PSI, what is the possible range of actual process pressures?  In other words, how much might the actual process pressure differ compared to the indicator's reading of 137 PSI?

\begin{itemize}
\item{} 50 PSI to 250 PSI
\vskip 5pt 
\item{} 137 PSI to 138 PSI
\vskip 5pt 
\item{} 135 PSI to 139 PSI
\vskip 5pt 
\item{} 136 PSI to 137 PSI
\vskip 5pt 
\item{} 136 PSI to 138 PSI
\vskip 5pt 
\item{} 87 PSI to 187 PSI
\end{itemize}

%INDEX% Calibration: table, pressure transmitter
%INDEX% Measurement, pressure: calibration table

%(END_NOTES)


