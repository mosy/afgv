
%(BEGIN_QUESTION)
% Copyright 2010, Tony R. Kuphaldt, released under the Creative Commons Attribution License (v 1.0)
% This means you may do almost anything with this work of mine, so long as you give me proper credit

Read and outline the ``Species Identification'', ``Chromatograph Detectors'', and ``Measuring Species Concentration'' subsections of the ``Chromatography'' section in the ``Continuous Analytical Measurement'' chapter in your {\it Lessons In Industrial Instrumentation} textbook.  Note the page numbers where important illustrations, photographs, equations, tables, and other relevant details are found.  Prepare to thoughtfully discuss with your instructor and classmates the concepts and examples explored in this reading.

\underbar{file i00230}
%(END_QUESTION)





%(BEGIN_ANSWER)


%(END_ANSWER)





%(BEGIN_NOTES)

Chromatograph detectors need not be selective to chemical species in the process stream -- they only need to be able to differentiate between carrier and non-carrier compounds.  Since we have the freedom to choose the carrier gas in a GC, this makes detector technology rather simple.

\vskip 10pt

Generally, light molecules have shorter retention times than heavy molecules.  See the flip-book animation in the Appendix!

\vskip 10pt

Chromatographs are calibrated by introducing a mixture of known chemical composition, and then recording the times and detector peaks on the chromatograph to gauge when each species exits the column and how strongly the detector responds to each species.

\vskip 10pt

Flame Ionization Detectors (FIDs) and Thermal Conductivity Detectors (TCDs) are the two most common detector types used in process GCs.  FIDs detect ionization in a gas flame, the idea being that neither the burner's fuel (hydrogen gas) nor the carrier (typically helium or nitrogen) generate significant ions in a flame, but hydrocarbon compounds do.  For this reason, FIDs are often called ``carbon counters''.  Therefore, FID chromatographs are widely used in the hydrocarbon industries.  TCDs work like thermal mass flowmeters, sensing changes in gas composition based on differences in specific heat.  Here, we tend to use carrier gases having much higher specific heat values (e.g. hydrogen, helium) than the species of interest.

\vskip 10pt

Species concentration is measured by integrating the peak on the chromatogram.  The area underneath each peak is proportional to the total quantity of species passed through the detector.  A major caveat here is that this integration value is not comparable from one species to another, because chromatograph detectors usually respond to different species with different levels of sensitivity.  A FID detector, for example, responds much more strongly to larger hydrocarbon molecules than to lighter hydrocarbon molecules, simply due to the relative difference in carbon concentration between species.  TCDs suffer the same problem because not all species have the same specific heat value.

\vskip 10pt

This problem is overcome by programming the chromatograph with ``response factors'' for each chemical species, telling it ahead of time how to interpret the signal coming from the detector.  These response factors are calculated by passing a calibration gas sample of known composition through the GC and letting it see just how strongly the detector responds for each species.








\vskip 20pt \vbox{\hrule \hbox{\strut \vrule{} {\bf Suggestions for Socratic discussion} \vrule} \hrule}

\begin{itemize}
\item{} How may we {\it calibrate} a GC, to test its ability to accurately identify and quantify certain chemical species?
\item{} Identify the operating principle of at least two different types of GC detectors.
\item{} {\bf Watch the animation of a GC shown in the textbook and narrate the processes illustrated therein.}
\item{} Why is the maximum peak height on a chromatogram not sufficient for us to calculate species concentration?
\item{} Would it be possible, in principle at least, to use a conductivity probe as the detector in a liquid chromatograph?  Why or why not?
\item{} Would it be possible, in principle at least, to use a pH probe as the detector in a liquid chromatograph?  Why or why not?
\item{} Explain why hydrogen and helium are especially well-suited as carrier gases for a GC using a thermal conductivity detector (TCD).
\item{} Explain what a ``response factor'' is for a chromatograph, and why this factor is relevant for thermal conductivity detectors (TCDs) and flame ionization detectors (FIDs) alike.
\item{} Explain how automated chromatographs are able to accurate quantify different chemical species when those species all stimulate the detector to different degrees.
\item{} Explain what would happen if the carrier gas were not pure, and contained one or more gases of interest (i.e. gas species present in the sample stream that we wished to measure).  Sketch chromatograms to show the GC's response with pure as well as impure carrier gases.
\item{} Suppose you were setting up a brand-new GC, and decided to test it using a calibration gas mixture consisting of 5 different compounds (CO, NH$_{3}$, CO$_{2}$, CH$_{4}$, and NO).  Naturally, you get 5 peaks showing up on the chromatogram.  Not knowing the order in which these five species will exit the column, can you think of a way to identify which peak represents which species for this chromatograph?
\end{itemize}

%INDEX% Reading assignment: Lessons In Industrial Instrumentation, Analytical (chromatography -- species ID and concentration)
%INDEX% Reading assignment: Lessons In Industrial Instrumentation, Analytical (chromatography -- detectors)

%(END_NOTES)

