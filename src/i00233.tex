
%(BEGIN_QUESTION)
% Copyright 2006, Tony R. Kuphaldt, released under the Creative Commons Attribution License (v 1.0)
% This means you may do almost anything with this work of mine, so long as you give me proper credit

Toluene has a density of 0.8669 g/cm$^{3}$ at 20$^{o}$ C.  Calculate its density in units of pounds per cubic feet and its specific gravity (unitless).

\underbar{file i00233}
%(END_QUESTION)





%(BEGIN_ANSWER)

Specific gravity = 0.8669 (same as density in units of g/cm$^{3}$)

\vskip 10pt

The units of grams and cubic centimeters are {\it defined} in such a way that their density quotient in relation to pure water is 1.  This is similar to the Celsius temperature scale, similarly {\it defined} at the 0$^{o}$ and 100$^{o}$ points by the freezing and boiling points of pure water, respectively.

To calculate the density of toluene in units of pounds per cubic feet, simply multiply the density of water (62.428 lb/ft$^{3}$) by the specific gravity of toluene (0.8669):

\vskip 10pt

(62.428 lb/ft$^{3}$)(0.8669) = 54.12 lb/ft$^{3}$

%(END_ANSWER)





%(BEGIN_NOTES)


%INDEX% Physics, static fluids: density and specific gravity

%(END_NOTES)


