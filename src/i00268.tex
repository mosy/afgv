
%(BEGIN_QUESTION)
% Copyright 2011, Tony R. Kuphaldt, released under the Creative Commons Attribution License (v 1.0)
% This means you may do almost anything with this work of mine, so long as you give me proper credit

{\it Ebony} wood is a very dense -- so dense that it does not float in water.  How much would a piece of ebony wood of D=75 lb/ft$^{3}$, 200 cubic inches in volume, weight when dry and when submerged in water?

\vskip 10pt

What would the density of wood have to be in order for it to float in water?

\vskip 10pt

Be sure to show all your mathematical work so that your instructor will be able to check the conceptual validity of your technique(s).  A good way to check to see if you're solving the problem correctly is to check that each and every one of your intermediate calculations (i.e. the results you get mid-way during the process to arrive at the final answer) has real physical meaning.  {\bf If you truly understand what you are doing, you will be able to identify the correct unit of measurement for every intermediate result and also be able to show where that number applies to the scenario at hand}.

\vskip 10pt

\underbar{file i00268}
%(END_QUESTION)





%(BEGIN_ANSWER)

Dry weight = 8.681 pounds.  Submerged weight = 1.455 pounds.

\vskip 10pt

For {\it any} object to float in water, its density must be less than that of water (62.428 lb/ft$^{3}$).

%(END_ANSWER)





%(BEGIN_NOTES)

Weight = weight density $\times$ volume

\vskip 10pt

$$F = \gamma V$$

\vskip 10pt

$F_{dry} = (75 \hbox{ lb/ft}^3) (200 \hbox{ in}^3) \left(1 \hbox{ ft}^3 \over 1728 \hbox{ in}^3\right) = 8.681 \hbox{ lb}$

\vskip 10pt

$F_{buoyant} = (62.428 \hbox{ lb/ft}^3) (200 \hbox{ in}^3) \left(1 \hbox{ ft}^3 \over 1728 \hbox{ in}^3\right) = 7.225 \hbox{ lb}$

\vskip 10pt

$F_{apparent} = F_{dry} - F_{buoyant} = 8.681 - 7.222 = 1.455 \hbox{ lb}$

%INDEX% Physics, static fluids: buoyancy

%(END_NOTES)


