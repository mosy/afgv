
%(BEGIN_QUESTION)
% Copyright 2011, Tony R. Kuphaldt, released under the Creative Commons Attribution License (v 1.0)
% This means you may do almost anything with this work of mine, so long as you give me proper credit

A block of metal of unknown density weighs 500 pounds when dry.  When completely submerged in water, its measured weight is only 400 pounds.  Calculate the volume of the block, in units of cubic feet.

\vskip 10pt

Also, calculate the density of this block, both in units of lb/ft$^{3}$ and g/cm$^{3}$.

\vskip 10pt

Finally, write an equation solving for the density of the block ($D_{block}$) in terms of the block's dry weight ($W_{dry}$), the block's submerged weight ($W_{wet}$), and the known density of the liquid ($D_{liquid}$).

\vskip 20pt \vbox{\hrule \hbox{\strut \vrule{} {\bf Suggestions for Socratic discussion} \vrule} \hrule}

\begin{itemize}
\item{} A useful problem-solving technique to apply when formulating a general equation is to think of an extremely simply numerical example problem, one where the number values are so simple that the solution to that problem is obvious and does not need an equation.  Then, take that obvious problem/solution and see what form of equation would make the variables relate properly to each other.  Try to apply this technique to the problem given here, in order to write a general equation useful for solving any similar problem.
\end{itemize}

\underbar{file i00272}
%(END_QUESTION)





%(BEGIN_ANSWER)

The block's volume is 1.602 ft$^{3}$.  Its density is 312.14 lb/ft$^{3}$ or 5 g/cm$^{3}$.

$$D_{block} = D_{liquid}\left({W_{dry} \over {W_{dry} - W_{wet}}}\right)$$

%(END_ANSWER)





%(BEGIN_NOTES)

The difference in dry and apparent weights (100 lb) is the weight of the water displaced by the block, as per Archimedes' Principle.  Therefore, the block's volume is equal to the volume of water weighing 100 pounds:

$$F = \gamma V$$

$$V = {F \over \gamma} = {100 \hbox{ lb} \over 62.4 \hbox{ lb/ft}^3} = 1.603 \hbox{ ft}^3$$

Now that we know the block's volume, we may calculate its density based on the dry weight of the block (given to be 500 pounds) and the now-known volume of the block:

$$F = \gamma V$$

$$\gamma = {F \over V} = {500 \hbox{ lb} \over 1.603 \hbox{ ft}^3} = 312 \hbox{ lb/ft}^3$$

\vskip 10pt

Writing a formula where the order of operations reflects the exact sequence of steps shown above:

$$D_{block} = {W_{dry} \over \left({{W_{dry} - W_{wet}} \over D_{liquid}}\right)}$$

\vskip 10pt

Simplifying this formula by eliminating compound fractions:

$$D_{block} = D_{liquid}\left({W_{dry} \over {W_{dry} - W_{wet}}}\right)$$

%INDEX% Physics, static fluids: buoyancy

%(END_NOTES)


