
%(BEGIN_QUESTION)
% Copyright 2006, Tony R. Kuphaldt, released under the Creative Commons Attribution License (v 1.0)
% This means you may do almost anything with this work of mine, so long as you give me proper credit

A king is given a shiny crown as a gift.  The person giving the crown claims that it is made of pure, solid gold.  It looks like gold, but the king -- being wise to the ways of the world -- knows that it might just be gold-plated tin or some other cheaper metal.

He hands his new crown over to a famous scientist to have it analyzed, with the command that the crown is not to be damaged in any way by the testing.  The scientist ponders the task of determining the crown's composition nondestructively, and decides that a density measurement performed by weighing the crown both dry and submerged in water would suffice, since gold is substantially heavier than any cheap metal.

The crown weighs 5 pounds dry.  When completely submerged in water, it weighs 4.444 pounds.  Is it really made of solid gold?

\underbar{file i00273}
%(END_QUESTION)





%(BEGIN_ANSWER)

First, determine the specific gravity of the crown.  If it weighs 5 pounds dry and 4.444 pounds submerged, then the weight of the displaced water must be 0.556 pounds.  A crown weighing 5 pounds, and having a volume equivalent to 0.556 pounds of water, must have a density approximately 9 times that of water (5 / 0.556 $\approx$ 9).  We can say it has a specific gravity of 9, or say that it has a density of 9 g/cm$^{3}$.

Pure gold has a specific gravity of almost 19, so this crown cannot be made of pure, solid gold.  It is most likely made of copper (D = 8.96 g/cm$^{3}$) or tin (D = 7.31 g/cm$^{3}$), plated with gold.

\vskip 10pt

According to legend, this is how Archimedes came up with his principle relating displacement to buoyant force.

%(END_ANSWER)





%(BEGIN_NOTES)


%INDEX% Physics, static fluids: Archimedes' principle

%(END_NOTES)


