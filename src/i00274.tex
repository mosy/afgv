
%(BEGIN_QUESTION)
% Copyright 2006, Tony R. Kuphaldt, released under the Creative Commons Attribution License (v 1.0)
% This means you may do almost anything with this work of mine, so long as you give me proper credit

A pure lead crown weighs 5 pounds dry.  It's a crummy crown, I know, but fit for a lesser king.  How much will it weigh when completely submerged in water?  Assume a density of 11.35 g/cm$^{3}$ for pure lead.

\underbar{file i00274}
%(END_QUESTION)





%(BEGIN_ANSWER)

We know that an object's dry weight divided by the difference between dry and submerged weights gives us the specific gravity, or density in units of g/cm$^{3}$.  If the lead crown's dry weight is 5 pounds and its specific gravity is 11.35, then the weight of the water it will displace when submerged is:

\vskip 10pt

(5 lb)/(11.35) = 0.441 lb

\vskip 10pt

Subtracting this figure of 0.441 pounds from the dry weight of 5 pounds should give us the submerged weight:

\vskip 10pt

5 lb - 0.441 lb = 4.559 lb

%(END_ANSWER)





%(BEGIN_NOTES)


%INDEX% Physics, static fluids: buoyancy

%(END_NOTES)


