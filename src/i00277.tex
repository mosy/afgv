
%(BEGIN_QUESTION)
% Copyright 2011, Tony R. Kuphaldt, released under the Creative Commons Attribution License (v 1.0)
% This means you may do almost anything with this work of mine, so long as you give me proper credit

Read selected portions of the National Transportation Safety Board's safety study, {\it Supervisory Control and Data Acquisition (SCADA) in Liquid Pipelines} (Document NTSB/SS-05/02 ; PB2005-917005), and answer the following questions:

\vskip 10pt

Question \#22 of the Safety Study in Appendix E lists the major features provided by pipeline SCADA systems.  Identify the most common features of these SCADA systems.  Question \#38 lists the number of data points handled by each respondent's SCADA system.  Just how data-rich are some of these systems?  

\vskip 10pt

Question \#42 of the Safety Study in Appendix E lists the various communication media used by SCADA systems to relay data between RTU and MTU points.  Explain what each of these terms refers to.  How many SCADA systems do not use redundant (backup) communication channels between RTU and MTU locations (hint: see question \#43)?

\vskip 10pt

Questions \#19 and \#21 of the Safety Study in Appendix E list the rationale given by respondents as to why they are considering an upgrade (or are currently implementing an upgrade) for their SCADA system.  What are some of the reasons given?  Do any of them surprise you?

\vskip 10pt

The ``SCADA Screens and Graphics'' section of chapter 4 showcases several examples of graphic displays in use in liquid pipeline control systems.  How much blank space should there be on any one screen in order to avoid ``clutter''?  How should color schemes be chosen to maximize effectiveness?

\vskip 10pt

The Safety Study results shown in Appendix E list a number of different manufacturers (``vendors'') for SCADA systems used in pipeline control (see question \#15).  Based on the vendor names and number of installations, what is your impression of pipeline SCADA systems in the United States: is there much standardization, or is there a wide diversity of system types in use?  Is there a clear leader among the manufacturers represented in the survey?


\vskip 20pt \vbox{\hrule \hbox{\strut \vrule{} {\bf Suggestions for Socratic discussion} \vrule} \hrule}

\begin{itemize}
\item{} An interesting section of this report to read is the one entitled ``Alarm Philosophy''.  In this section, the report describes how alarm conditions are reported to human operators by the SCADA system, and some of the challenges associated with the management of this information.  Read this section and discuss your findings with classmates!
\item{} Another interesting section of this report to read is the one entitled ``Training and Selection'', in which the report describes how pipeline operators are trained for their jobs.  It is interesting to compare the different methods and modalities of operator training.  Read this section and discuss your findings with classmates!
\item{} There is a typographical error in the caption of Figure 4.10 -- can you identify it?
\end{itemize}

\underbar{file i00277}
%(END_QUESTION)





%(BEGIN_ANSWER)


%(END_ANSWER)





%(BEGIN_NOTES)

In the responses to question \#22, we see features listed such as: 

\begin{itemize}
\item{} Remote valve operation
\item{} Data acquisition
\item{} Trend graphing
\item{} Pipeline leak detection
\end{itemize}

\vskip 10pt

In the responses to question \#38, we see that most pipeline SCADA systems have {\it 5000 or more} telemetered points of data!

\vskip 10pt

An RTU is a Remote Terminal Unit, a field-mounted SCADA node where sensors and valve controls connect.  An MTU is centrally located, polling data from multiple RTUs.  The most popular mode of data communication at the time of this survey was leased lines (telephone lines), with satellite and 2-way radio being close second- and third-place alternatives.  Approximately 25\% (25.9\% to be exact) of all pipeline SCADA systems do not employ any form of network redundancy.  Among those that do, the most popular option for redundant networking is ``dial-up'' (telephone modem).  The reason the percentages in this question add up in excess of 100\% is due to multiple communication methods used within the same system.  \#46 is interesting, indicating that 15\% of these systems have experienced total communication failure at some point in time.

\vskip 10pt

Hardware obsolescence is the dominant reason for SCADA system upgrading, in this survey!  Improved security ranked high among those respondents considering an upgrade.  Only about half of the respondents indicated they were replacing or upgrading their SCADA system at all.

\vskip 10pt

An FAA (Federal Aviation Administration) standard calls for at least 40\% blank space on each screen in order to avoid clutter.  The ISA's more liberal standard recommends 25\% to 40\% blank space.  Colors should be consistently applied (e.g. don't use red both for alarms and for general labels), and sufficiently contrasted for good visibility (e.g. don't use yellow characters on a light background as shown in figure 4.7, or light-blue text on a grey background as shown in figure 4.8).  The ISA recommends using between 5 and 7 colors.  A fascinating detail highlighted in this section was the existence of color-blind operators who had trouble distinguishing different colors from each other, and the fact that several of the companies surveyed did not test for color-blindness! 

\vskip 10pt

A wide variety of SCADA names are represented in this survey: 41 to be exact, out of 98 survey respondents (!).  The incidence numbers for each are rather low, suggesting a real lack of standardization in the industry.  The most popular (with 13 installations) is Valmet Automation.  Wonderware and Telvent are the second-most popular (at eight installations each).









\vskip 20pt \vbox{\hrule \hbox{\strut \vrule{} {\bf Suggestions for Socratic discussion} \vrule} \hrule}

\begin{itemize}
\item{} Provide examples of poor color choices in SCADA display screens.
\item{} Provide examples of clutter in SCADA display screens.
\item{} Explain why you think so many SCADA display designs are so poor.  Why all these bad color choices and cluttered displays?
\end{itemize}










\vfil \eject

\noindent
{\bf Prep Quiz:}

Based on the survey of SCADA systems used for liquid pipeline control in the United States performed by the NSTB in 2002, how standardized are these SCADA systems?

\begin{itemize}
\item{} Highly standardized; most pipelines controlled by Allen-Bradley SCADA systems
\vskip 5pt 
\item{} Highly standardized; most pipelines controlled by Siemens SCADA systems 
\vskip 5pt 
\item{} Exactly ten different manufacturer represented among the SCADA systems surveyed
\vskip 5pt 
\item{} Not very standardized; a wide variety of manufacturers' products represented
\vskip 5pt 
\item{} No standardization whatsoever; no manufacturer represented more than once in the survey
\end{itemize}


%INDEX% Reading assignment: NTSB Report, SCADA in Liquid Pipelines

%(END_NOTES)


