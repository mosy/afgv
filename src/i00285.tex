
%(BEGIN_QUESTION)
% Copyright 2011, Tony R. Kuphaldt, released under the Creative Commons Attribution License (v 1.0)
% This means you may do almost anything with this work of mine, so long as you give me proper credit

Suppose a radio transmitter unit outputs 327 watts of RF power at a frequency of 1070 kHz, into a half-wave ``whip'' antenna.  The radio energy radiates outward from the transmitting antenna, and is received by another antenna to send a signal of 9.5 microwatts into a radio receiver.

Calculate the transmitter's output power ($P_{tx}$) in both dBm and dBW, and also the receiver's input signal power ($P_{rx}$) in dBm and dBW.  Then, calculate the total power loss between transmitter and receiver ($P_{loss}$), in dB.

\vskip 10pt

$P_{tx}$ = \underbar{\hskip 50pt} dBm = \underbar{\hskip 50pt} dBW 

\vskip 10pt

$P_{rx}$ = \underbar{\hskip 50pt} dBm = \underbar{\hskip 50pt} dBW 

\vskip 10pt

$P_{loss}$ = \underbar{\hskip 50pt} dB

\vskip 20pt \vbox{\hrule \hbox{\strut \vrule{} {\bf Suggestions for Socratic discussion} \vrule} \hrule}

\begin{itemize}
\item{} Explain why the ``decibel'' is such a commonly used unit for power, and also for power gains and losses, in RF communications work.
\item{} Estimate these decibel values by first rounding the power ratio to the nearest product of 10 and/or 2, then applying the equivalence of 10 dB to a 10-fold ratio and 3 dB to a 2-fold ratio.
\end{itemize}

\underbar{file i00285}
%(END_QUESTION)





%(BEGIN_ANSWER)

$P_{tx}$ = \underbar{55.15} dBm = \underbar{25.15} dBW 

\vskip 10pt

$P_{rx}$ = \underbar{$-20.22$} dBm = \underbar{$-50.22$} dBW 

\vskip 10pt

$P_{loss}$ = \underbar{$-75.37$} dB

%(END_ANSWER)





%(BEGIN_NOTES)


%INDEX% Electronics review: decibel power calculations

%(END_NOTES)

