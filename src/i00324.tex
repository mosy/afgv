
%(BEGIN_QUESTION)
% Copyright 2011, Tony R. Kuphaldt, released under the Creative Commons Attribution License (v 1.0)
% This means you may do almost anything with this work of mine, so long as you give me proper credit

Enter this {\it strapping table} into a computer spreadsheet, then use the spreadsheet program's graphing capabilities to plot the volume-vs.-height relationship for this liquid storage vessel:

% No blank lines allowed between lines of an \halign structure!
% I use comments (%) instead, so that TeX doesn't choke.

$$\vbox{\offinterlineskip
\halign{\strut
\vrule \quad\hfil # \ \hfil & 
\vrule \quad\hfil # \ \hfil \vrule \cr
\noalign{\hrule}
%
% First row
Height (ft) & Volume (gallons) \cr
%
\noalign{\hrule}
%
% Another row
0 & 0 \cr
%
\noalign{\hrule}
%
% Another row
1 & 498 \cr
%
\noalign{\hrule}
%
% Another row
2 & 1165 \cr
%
\noalign{\hrule}
%
% Another row
3 & 1699 \cr
%
\noalign{\hrule}
%
% Another row
4 & 2321 \cr
%
\noalign{\hrule}
%
% Another row
5 & 2910 \cr
%
\noalign{\hrule}
%
% Another row
6 & 3520 \cr
%
\noalign{\hrule}
%
% Another row
7 & 4105 \cr
%
\noalign{\hrule}
%
% Another row
8 & 4743 \cr
%
\noalign{\hrule}
%
% Another row
9 & 5304 \cr
%
\noalign{\hrule}
%
% Another row
10 & 5899 \cr
%
\noalign{\hrule}
%
% Another row
11 & 6522 \cr
%
\noalign{\hrule}
%
% Another row
12 & 7077 \cr
%
\noalign{\hrule}
%
% Another row
13 & 7650 \cr
%
\noalign{\hrule}
%
% Another row
14 & 8285 \cr
%
\noalign{\hrule}
%
% Another row
15 & 8879 \cr
%
\noalign{\hrule}
} % End of \halign 
}$$ % End of \vbox

Finally, use the spreadsheet program to generate a {\it best fit equation} for the strapping table data (plotting a ``trendline'' and displaying the formula for it).  It is recommended to try both a linear function as well as a polynomial function to see which type gives the best fit.

\vskip 20pt \vbox{\hrule \hbox{\strut \vrule{} {\bf Suggestions for Socratic discussion} \vrule} \hrule}

\begin{itemize}
\item{} Explain how the ``best fit'' feature of spreadsheet software might be useful if you need to linearize a level measurement signal in a programmable system such as a PLC.
\end{itemize}

\underbar{file i00324}
%(END_QUESTION)





%(BEGIN_ANSWER)


%(END_ANSWER)





%(BEGIN_NOTES)


%INDEX% Computer spreadsheet exercise: introduction to spreadsheet usage
%INDEX% Measurement, level: strapping table

%(END_NOTES)


