
%(BEGIN_QUESTION)
% Copyright 2016, Tony R. Kuphaldt, released under the Creative Commons Attribution License (v 1.0)
% This means you may do almost anything with this work of mine, so long as you give me proper credit

The scientific paper ``Cavitation Pressure in Water'' written by Eric Herbert, S\'ebastien Balibar, and Fr\'ed\'eric Caupin, and published in the October 16 (2006) edition of {\it Physical Review} opens with the following statement about material phases:

\vskip 10pt {\narrower \noindent \baselineskip5pt

The liquid and vapor phases of a pure substance can coexist at equilibrium only on a well defined line relating pressure and temperature.  Away from this coexistence line, one of the phases is more stable than the other.  However, because of the existence of a liquid-vapor surface tension, if one phase is brought in the stability region of the other, it can be observed for a finite time in a metastable state; the lifetime of this metastable state decreases as one goes away from the coexistence line.

\par} \vskip 10pt

After reading this passage, reflect on it and try to answer the following questions based on what it says as well as what you have learned about phases and phase changes:

\begin{itemize}
\item{} What is meant by the phrase {\it coexistence line}?  Does this refer to something you have already learned about, or is this a brand-new concept?
\vskip 5pt
\item{} What is meant by the phrase {\it stability region}?  Again, does this refer to something you already know, or is the reference new to you?
\vskip 5pt
\item{} Describe an experiment by which we might produce the type of ``metastable state'' referred to in this passage, and measure its lifespan.
\end{itemize}

\underbar{file i00336}
%(END_QUESTION)





%(BEGIN_ANSWER)

The ``coexistence line'' is simply the curve found in any phase diagram: the boundary(ies) delineating stable states of solid, liquid, and/or vapor.

\vskip 10pt

The ``stability region'' is that area within a phase diagram where the substance in question exists in one phase only.  For example, the ``liquid'' region of a phase diagram would be one of the ``stability regions'' for that substance.

\vskip 10pt

Practical examples of metastable states are supercooled water, and superheated water.  In each case, the temperature of a substance is varied under controlled conditions until the sample exists at a temperature where it ``should'' have changed states, but has not yet.

%(END_ANSWER)





%(BEGIN_NOTES)


%INDEX% Physics, heat and temperature: metastable states
%INDEX% Reading assignment: scientific paper on cavitation in water

%(END_NOTES)


