
%(BEGIN_QUESTION)
% Copyright 2011, Tony R. Kuphaldt, released under the Creative Commons Attribution License (v 1.0)
% This means you may do almost anything with this work of mine, so long as you give me proper credit

Program a computer spreadsheet (e.g. Microsoft Excel) application to automatically convert between the temperature units of Fahrehneit and Celcius:

$$\includegraphics[width=15.5cm]{i00339x01.eps}$$

The yellow and blue cell shading (color fill) shown in this example is entirely optional, but helps to distinguish number-entry fields from calculated-value fields.

I recommend placing the entered temperatures into cells R2C1 and R2C3, while the calculated values go into cells R5C1 and R5C3 (respectively).  Then, test your spreadsheet program on the following conversions to see how well it works:

\begin{itemize}
\item{} 300$^{o}$ C = ???$^{o}$ F 
\vskip 5pt
\item{} 50$^{o}$ F = ???$^{o}$ C 
\vskip 5pt
\item{} 4$^{o}$ C = ???$^{o}$ F 
\vskip 5pt
\item{} 894$^{o}$ F = ???$^{o}$ C 
\vskip 5pt
\item{} -250$^{o}$ F = ???$^{o}$ C 
\vskip 5pt
\item{} -312$^{o}$ F = ???$^{o}$ C 
\vskip 5pt
\item{} -150$^{o}$ C = ???$^{o}$ F 
\vskip 5pt
\item{} -230$^{o}$ C = ???$^{o}$ F 
\vskip 5pt
\item{} 2600$^{o}$ F = ???$^{o}$ C 
\vskip 5pt
\item{} 3000$^{o}$ C = ???$^{o}$ F 
\end{itemize}

\vskip 20pt \vbox{\hrule \hbox{\strut \vrule{} {\bf Suggestions for Socratic discussion} \vrule} \hrule}

\begin{itemize}
\item{} How could you {\it test} your spreadsheet temperature conversion calculator for accuracy (to verify you haven't made any mistakes) once you've entered all your equations?
\item{} Did you have to use parentheses in either formula (in cell R5C1 or in cell R5C3)?  Why or why not?
\item{} Explain how a spreadsheet is such a powerful mathematical tool for performing ``tedious'' calculations such as instrument input/output responses.  Can you think of any other practical uses for a spreadsheet?
\end{itemize}

\underbar{file i00339}
%(END_QUESTION)





%(BEGIN_ANSWER)

\begin{itemize}
\item{} 300$^{o}$ C = 572$^{o}$ F 
\vskip 5pt
\item{} 50$^{o}$ F = 10$^{o}$ C 
\vskip 5pt
\item{} 4$^{o}$ C = 39.2$^{o}$ F 
\vskip 5pt
\item{} 894$^{o}$ F = 478.89$^{o}$ C 
\vskip 5pt
\item{} -250$^{o}$ F = -156.67$^{o}$ C 
\vskip 5pt
\item{} -312$^{o}$ F = -191.11$^{o}$ C 
\vskip 5pt
\item{} -150$^{o}$ C = -238$^{o}$ F 
\vskip 5pt
\item{} -230$^{o}$ C = -382$^{o}$ F 
\vskip 5pt
\item{} 2600$^{o}$ F = 1426.67$^{o}$ C 
\vskip 5pt
\item{} 3000$^{o}$ C = 5432$^{o}$ F 
\end{itemize}

%(END_ANSWER)





%(BEGIN_NOTES)


\begin{itemize}
\item{} {\bf Cell R5C1:} {\tt = (R2C1 - 32) * 5 / 9}
\item{} {\bf Cell R5C3:} {\tt = R2C1 * 9 / 5 + 32}
\end{itemize}


\vfil \eject

\noindent
{\bf Prep Quiz:}

Convert 175 degrees Celsius into degrees Fahrenheit:

\begin{itemize}
\item{} 315 $^{o}$F 
\vskip 5pt 
\item{} 207 $^{o}$F 
\vskip 5pt 
\item{} 129.2 $^{o}$F 
\vskip 5pt 
\item{} 79.44 $^{o}$F 
\vskip 5pt 
\item{} 283 $^{o}$F
\vskip 5pt 
\item{} 347 $^{o}$F
\end{itemize}


%INDEX% Computer spreadsheet exercise: temperature unit conversions
%INDEX% Physics, units and conversions: temperature
%INDEX% Physics, heat and temperature: unit conversions

%(END_NOTES)


