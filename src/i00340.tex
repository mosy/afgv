
%(BEGIN_QUESTION)
% Copyright 2006, Tony R. Kuphaldt, released under the Creative Commons Attribution License (v 1.0)
% This means you may do almost anything with this work of mine, so long as you give me proper credit

Convert between the following units of temperature:

\begin{itemize}
\item{} 350 K = ???$^{o}$ C 
\vskip 5pt
\item{} 575$^{o}$ F = ???$^{o}$ R 
\vskip 5pt
\item{} -210$^{o}$ C = ??? K 
\vskip 5pt
\item{} 900$^{o}$ R = ???$^{o}$ F 
\vskip 5pt
\item{} -366$^{o}$ F = ???$^{o}$ R 
\vskip 5pt
\item{} 100 K = ???$^{o}$ C 
\vskip 5pt
\item{} 2888$^{o}$ C = ??? K 
\vskip 5pt
\item{} 4502$^{o}$ R = ???$^{o}$ F 
\vskip 5pt
\item{} 1000 K = ???$^{o}$ R 
\vskip 5pt
\item{} 3000$^{o}$ R = ??? K 
\vskip 5pt
\medskip

\vskip 20pt \vbox{\hrule \hbox{\strut \vrule{} {\bf Suggestions for Socratic discussion} \vrule} \hrule}

\begin{itemize}
\item{} Demonstrate how to {\it estimate} numerical answers for these conversion problems without using a calculator.
\item{} What pracitcal purpose does a temperature scale such as Kelvin or Rankine serve, especially since the scales of Celsius and Fahrenheit are so well-known and commonly used?
\end{itemize}

\underbar{file i00340}
%(END_QUESTION)





%(BEGIN_ANSWER)

\begin{itemize}
\item{} 350 K = 76.85$^{o}$ C 
\vskip 5pt
\item{} 575$^{o}$ F = 1034.67$^{o}$ R 
\vskip 5pt
\item{} -210$^{o}$ C = 63.15 K 
\vskip 5pt
\item{} 900$^{o}$ R = 440.33$^{o}$ F 
\vskip 5pt
\item{} -366$^{o}$ F = 93.67$^{o}$ R 
\vskip 5pt
\item{} 100 K = -173.15$^{o}$ C 
\vskip 5pt
\item{} 2888$^{o}$ C = 3161.15 K 
\vskip 5pt
\item{} 4502$^{o}$ R = 4042.33$^{o}$ F 
\vskip 5pt
\item{} 1000 K = 1800$^{o}$ R 
\vskip 5pt
\item{} 3000$^{o}$ R = 1666.67 K 
\end{itemize}

%(END_ANSWER)





%(BEGIN_NOTES)

Remember, the offset between $^{o}$F and $^{o}$R is {\bf 459.67}.  The offset between $^{o}$C and K is {\bf 273.15}.

%INDEX% Physics, units and conversions: temperature
%INDEX% Physics, heat and temperature: unit conversions

%(END_NOTES)


