
%(BEGIN_QUESTION)
% Copyright 2006, Tony R. Kuphaldt, released under the Creative Commons Attribution License (v 1.0)
% This means you may do almost anything with this work of mine, so long as you give me proper credit

A very useful principle in physics is the {\it Ideal Gas Law}, so called because it relates pressure, volume, molecular quantity, and temperature of an ideal gas together in one neat mathematical expression:

$$PV = nRT$$

\noindent
Where,

$P$ = Absolute pressure (atmospheres)

$V$ = Volume (liters)

$n$ = Gas quantity (moles)

$R$ = Universal gas constant (0.0821 L $\cdot$ atm / mol $\cdot$ K)

$T$ = Absolute temperature (K)

\vskip 10pt

Note that temperature $T$ in this equation must be in {\it absolute} units (Kelvin).  Modify the Ideal Gas Law equation to accept a value for $T$ in units of $^{o}$C.

Then, modify the equation once more to accept a value for $T$ in units of $^{o}$F.

\underbar{file i00342}
%(END_QUESTION)





%(BEGIN_ANSWER)

$$PV = nR(T + 273.15) \hbox{\hskip 30pt Temperature in degrees C}$$

$$PV = nR\left({5 \over 9}(T - 32) + 273.15\right) \hbox{\hskip 30pt Temperature in degrees F}$$

It should be noted that the behavior of real gases departs significantly from the Ideal Gas Law model at temperatures near absolute zero, especially when phase changes (liquefaction and/or solidification) take place.  Still, it is possible through intuition to tell the intended unit of temperature measurement for $T$ in this equation must be absolute and not elevated, and this is the thrust of the challenge question.

%(END_ANSWER)





%(BEGIN_NOTES)


%INDEX% Physics, heat and temperature: absolute temperature
%INDEX% Physics, static fluids: ideal gas law

%(END_NOTES)


