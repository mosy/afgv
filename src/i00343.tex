
%(BEGIN_QUESTION)
% Copyright 2006, Tony R. Kuphaldt, released under the Creative Commons Attribution License (v 1.0)
% This means you may do almost anything with this work of mine, so long as you give me proper credit

The rate of heat transfer through {\it radiation} from a warm body may be expressed by the Stefan-Boltzmann equation:

$${dQ \over dt} = e \sigma A T^4$$

\noindent
Where,

$dQ \over dt$ = Rate of heat flow 

$e$ = Emissivity factor 

$\sigma$ = Stefan-Boltzmann constant (5.67 $\times$ $10^{-8}$ W / m$^{2}$ $\cdot$ K$^{4}$)

$A$ = Area of radiating surface

$T$ = Absolute temperature 

\vskip 10pt

Based on the unit of measurement given for the Stefan-Boltzmann constant, determine the proper units of measurement for heat flow, emissivity, area, and temperature.

\underbar{file i00343}
%(END_QUESTION)





%(BEGIN_ANSWER)

$dQ \over dt$ = Rate of heat flow (Watts)

$e$ = Emissivity factor ({\it unitless})

$A$ = Area of radiating surface (square meters)

$T$ = Absolute temperature (Kelvin)

\vskip 10pt

Challenge question: a more complete expression of the Stefan-Boltzmann equation takes into account the temperature of the warm object's surroundings:

$${dQ \over dt} = e \sigma A (T_1^4 - T_2^4)$$

\noindent
Where,

$T_1$ = Temperature of the object

$T_2$ = Ambient temperature

\vskip 10pt

Explain why this second $T$ term is necessary for the equation to make sense.

%(END_ANSWER)





%(BEGIN_NOTES)

Answer to challenge question: if we take the first (simpler) equation literally, we would have to conclude either that all warm objects are eternal sources of energy (constantly radiating energy without cooling down), or that all warm objects will cool themselves to absolute zero in a vacuum (where there is no conduction or convection)!  In other words, the simple equation tells us that objects can only emit radiant energy, not receive any.

%INDEX% Physics, heat and temperature: Stefan-Boltzmann equation

%(END_NOTES)


