
%(BEGIN_QUESTION)
% Copyright 2006, Tony R. Kuphaldt, released under the Creative Commons Attribution License (v 1.0)
% This means you may do almost anything with this work of mine, so long as you give me proper credit

Solids tend to expand when heated.  The amount a solid sample will expand with increased temperature depends on the size of the sample and the material it is made of.  A formula expressing linear expansion in relation to temperature is as follows:

$$l = l_0 (1 + \alpha \Delta T)$$

\noindent
Where,

$l$ = Length of material after heating

$l_0$ = Original length of material

$\alpha$ = Coefficient of linear expansion 

$\Delta T$ = Change in temperature

\vskip 10pt

Here are some typical values of $\alpha$ for common metals:

\begin{itemize}
\item{} Aluminum = 25 $\times$ $10^{-6}$ per degree C
\item{} Copper = 16.6 $\times$ $10^{-6}$ per degree C
\item{} Iron = 12 $\times$ $10^{-6}$ per degree C
\item{} Tin = 20 $\times$ $10^{-6}$ per degree C
\item{} Titanium = 8.5 $\times$ $10^{-6}$ per degree C
\end{itemize}

We may also express the tendency for the {\it area} and the {\it volume} of a solid to expand when heated, not just its linear dimensions.  If we imagine a square with original length $l_0$ and original width $l_0$, the original area of the square must be ${l_0}^2$, which means the new area of the square after heating will be:

$$A = [l_0 (1 + \alpha \Delta T)]^2$$

$$A = l_0^2 (1 + \alpha \Delta T)^2$$

$$A = l_0^2 (1 + \alpha \Delta T)(1 + \alpha \Delta T)$$

$$A = l_0^2 [1 + 2 \alpha \Delta T + (\alpha \Delta T)^2]$$

$$\hbox{or}$$

$$A = A_0 [1 + 2 \alpha \Delta T + (\alpha \Delta T)^2]$$

\vskip 10pt

This equation may be simplified by approximation -- a mathematical principle commonly applied in electrical engineering known as {\it swamping}:

$$A \approx A_0 (1 + 2 \alpha \Delta T)$$

Explain why it is okay to make this simplification, and extrapolate the principle to calculating the new {\it volume} of a solid material after heating.

\underbar{file i00346}
%(END_QUESTION)





%(BEGIN_ANSWER)

$$V \approx V_0 (1 + 3 \alpha \Delta T)$$

%(END_ANSWER)





%(BEGIN_NOTES)

Please note that the coefficient of volumetric expansion is approximately three times that of the coefficient of linear expansion, just as the coefficient of area expansion was approximately two times that of the coefficient of linear expansion.  This is good to know, for if we know one of these coefficients, we may approximate all three!

Also note that this approximation only holds true for materials which are {\it isotropic}: that is, they expand with the same proportion in all directions due to warming.

\vskip 10pt

Values for $\alpha$ were taken from the {\it CRC Handbook of Chemistry and Physics}, 64$^{th}$ edition.

%INDEX% Physics, heat and temperature: thermal expansion of solids

%(END_NOTES)


