
%(BEGIN_QUESTION)
% Copyright 2006, Tony R. Kuphaldt, released under the Creative Commons Attribution License (v 1.0)
% This means you may do almost anything with this work of mine, so long as you give me proper credit

A propane tank holds both liquid propane and propane vapor at high pressure:

$$\includegraphics[width=15.5cm]{i00349x01.eps}$$

How may the pressure in the tank be altered?  What physical variable must be changed in order to increase or decrease the vapor pressure inside the tank?

\underbar{file i00349}
%(END_QUESTION)





%(BEGIN_ANSWER)

The only factor able to alter the saturated vapor pressure inside the tank is {\it temperature}.  Increasing the tank's temperature will cause the pressure to likewise increase.

This principle is put to use in Class II filled systems for measuring temperature: the vapor pressure of the volatile fill fluid indicates its temperature.  Since this pressure does not depend on volume, any changes in volume resulting from expansion or contraction of the liquid or the vapor at the indicator end of the system will be absorbed by either condensation or evaporation (respectively), until the pressure again stabilizes at the value determined by the liquid/vapor interface's temperature.

%(END_ANSWER)





%(BEGIN_NOTES)


%INDEX% Physics, heat and temperature: saturated vapor pressure

%(END_NOTES)


