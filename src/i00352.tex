
%(BEGIN_QUESTION)
% Copyright 2015, Tony R. Kuphaldt, released under the Creative Commons Attribution License (v 1.0)
% This means you may do almost anything with this work of mine, so long as you give me proper credit

Search the Rosemount model 648 WirelessHART temperature transmitter Reference Manual (document 00809-0100-4648, revision BA) to answer the following questions:

\vskip 10pt

Identify the necessary parameters within the {\sl Wireless}HART device enabling it to communicate with a particular network gateway device.  Where is this information stored at in a {\sl Wireless}HART system?

\vskip 10pt

How does one access and edit the parameters of a {\sl Wireless}HART field instrument such as the model 648 temperature transmitter {\it before} it has been detected and incorporated into an active {\sl Wireless}HART network?

\vskip 10pt

How fast does a typical {\sl Wireless}HART field instrument update its information to the network gateway?  Does the total number of field devices in any one mesh network affect this update rate?

\vskip 10pt

How should the antenna be oriented for each {\sl Wireless}HART field instrument?  Explain the reason why.

\vskip 20pt \vbox{\hrule \hbox{\strut \vrule{} {\bf Suggestions for Socratic discussion} \vrule} \hrule}

\begin{itemize}
\item{} What would happen if someone mounted a {\sl Wireless}HART device with the antenna oriented incorrectly?  Would the instrument be able to function at all?  Would it be able to function in a limited capacity?  Explain your reasoning thoroughly.
\item{} What diagnostic data is available from the gateway to help you determine the integrity of the radio link for a particular {\sl Wireless}HART device?
\item{} Appendix page B-3 in this manual describes the device's radio transmit power as limited to ``10 mW e.i.r.p.''.  Explain what this means.
\end{itemize}

\underbar{file i00352}
%(END_QUESTION)





%(BEGIN_ANSWER)


%(END_ANSWER)





%(BEGIN_NOTES)

The {\sl Wireless}HART device must be programmed with a {\it network ID} and a {\it join key} in order to begin communicating with a network gateway device (pages 2-7 and 4-3).  Both the network ID and join key are stored in the gateway device itself, as well as within each active field device in the mesh network.

\vskip 10pt

{\sl Wireless}HART devices are first and foremost HART devices, and so they have connections to which you may attach a hand-held HART communicator (such as the Emerson 375 or 475) and access those HART parameters just like any other HART instrument.  The ``COMM'' terminals are shown in Figure 2-6 on page 2-7.

\vskip 10pt

The recommended update rate should be between 15 seconds and 60 minutes, with 15 seconds being the fastest update rate possible for networks containing 50 or fewer devices.  For networks containing more than 50 devices (up to 100 devices), the fastest update rate is 60 seconds.  The default update rate is 5 minutes (page 2-7).

\vskip 10pt

These instruments have ``whip'' antennas, and as such should have their antennas oriented vertically for maximum omnidirectional coverage (pages 3-3, 3-4).  Placement close to parallel metal objects should be avoided (page 1-3).

%INDEX% Networking, WirelessHART
%INDEX% Reading assignment: Rosemount 648 WirelessHART temperature transmitter reference manual

%(END_NOTES)

