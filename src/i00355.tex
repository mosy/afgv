
%(BEGIN_QUESTION)
% Copyright 2006, Tony R. Kuphaldt, released under the Creative Commons Attribution License (v 1.0)
% This means you may do almost anything with this work of mine, so long as you give me proper credit

An instrument technician wants to create a temperature reference for a thermocouple transmitter by freezing some water, knowing that the freezing point of water at sea level is 32$^{o}$ F (or 0$^{o}$ C).  He inserts the thermocouple into a cup of water, then sets the cup and thermocouple inside a freezer until the water is frozen solid.  He then takes the cup out of the freezer and connects the thermocouple to the temperature transmitter for calibration.

What is wrong with the technician's procedure?  What must be done differently to ensure a reference temperature of 32$^{o}$ F (0$^{o}$ C) at the thermocouple tip?

\vskip 20pt \vbox{\hrule \hbox{\strut \vrule{} {\bf Suggestions for Socratic discussion} \vrule} \hrule}

\begin{itemize}
\item{} Refer to a {\it phase diagram} to identify points where H$_{2}$O maintains a very stable temperature despite changes in surrounding pressure. 
\item{} Explain how the {\it triple point} of water is used in metrology laboratories to maintain both stable temperature and stable pressure for instrument calibration purposes. 
\end{itemize}

\underbar{file i00355}
%(END_QUESTION)





%(BEGIN_ANSWER)

%(END_ANSWER)





%(BEGIN_NOTES)

What the technician needs is an ice-water {\it mixture} in order to guarantee stability at the freezing temperature.

\vskip 10pt

In lieu of setting a cup of water inside a freezer, I recommend dropping ice cubes into a cup of water, and stirring the mixture to achieve a uniform temperature.  Keep adding ice to make up for melting, and the mixture will remain at a constant 32$^{o}$ F (0$^{o}$ C) no matter what the ambient temperature of the room.

%INDEX% Calibration, standard: using an ice bath properly

%(END_NOTES)


