
%(BEGIN_QUESTION)
% Copyright 2006, Tony R. Kuphaldt, released under the Creative Commons Attribution License (v 1.0)
% This means you may do almost anything with this work of mine, so long as you give me proper credit

In a filled-system type of temperature instrument, why is it important to use a small-diameter (``capillary'') tube to connect the bulb to the bellows?  Why not use regular, large-diameter tubing instead?

\underbar{file i00358}
%(END_QUESTION)





%(BEGIN_ANSWER)

Think about what would happen if the connecting tubing were the same inside diameter as the bulb itself.  Would this not act like one, long bulb?  If this were the case, the entire length of tubing would act as a temperature sensor, rather than just the bulb.  In other words, the measurement provided by the instrument would be an aggregate of all temperatures between the process and the instrument itself, rather than a single measurement of process temperature!

%(END_ANSWER)





%(BEGIN_NOTES)


%INDEX% Measurement, temperature: filled-bulb system

%(END_NOTES)


