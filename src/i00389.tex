
%(BEGIN_QUESTION)
% Copyright 2009, Tony R. Kuphaldt, released under the Creative Commons Attribution License (v 1.0)
% This means you may do almost anything with this work of mine, so long as you give me proper credit

Determine the following parameters for type J, type K, type E, type T, type S, and type R thermocouples.  This includes the American color codes for individual wires as well as cable jackets:

\begin{itemize}
\item{} {\bf Type J}
\item{} Positive wire metal type: \underbar{\hskip 50pt} -- Color: \underbar{\hskip 50pt} 
\item{} Negative wire metal type: \underbar{\hskip 50pt} -- Color: \underbar{\hskip 50pt} 
\item{} Thermocouple-grade cable jacket color: \underbar{\hskip 50pt} 
\item{} Extension-grade cable jacket color: \underbar{\hskip 50pt} 
%\item{} Thermocouple temperature range -- Maximum: \underbar{\hskip 50pt} -- Minimum: \underbar{\hskip 50pt} 
\end{itemize}

\begin{itemize}
\item{} {\bf Type K}
\item{} Positive wire metal type: \underbar{\hskip 50pt} -- Color: \underbar{\hskip 50pt} 
\item{} Negative wire metal type: \underbar{\hskip 50pt} -- Color: \underbar{\hskip 50pt} 
\item{} Thermocouple-grade cable jacket color: \underbar{\hskip 50pt} 
\item{} Extension-grade cable jacket color: \underbar{\hskip 50pt} 
%\item{} Thermocouple temperature range -- Maximum: \underbar{\hskip 50pt} -- Minimum: \underbar{\hskip 50pt} 
\end{itemize}

\begin{itemize}
\item{} {\bf Type E}
\item{} Positive wire metal type: \underbar{\hskip 50pt} -- Color: \underbar{\hskip 50pt} 
\item{} Negative wire metal type: \underbar{\hskip 50pt} -- Color: \underbar{\hskip 50pt} 
\item{} Thermocouple-grade cable jacket color: \underbar{\hskip 50pt} 
\item{} Extension-grade cable jacket color: \underbar{\hskip 50pt} 
%\item{} Thermocouple temperature range -- Maximum: \underbar{\hskip 50pt} -- Minimum: \underbar{\hskip 50pt} 
\end{itemize}

\begin{itemize}
\item{} {\bf Type T}
\item{} Positive wire metal type: \underbar{\hskip 50pt} -- Color: \underbar{\hskip 50pt} 
\item{} Negative wire metal type: \underbar{\hskip 50pt} -- Color: \underbar{\hskip 50pt} 
\item{} Thermocouple-grade cable jacket color: \underbar{\hskip 50pt} 
\item{} Extension-grade cable jacket color: \underbar{\hskip 50pt} 
%\item{} Thermocouple temperature range -- Maximum: \underbar{\hskip 50pt} -- Minimum: \underbar{\hskip 50pt} 
\end{itemize}

\begin{itemize}
\item{} {\bf Type S}
\item{} Positive wire metal type: \underbar{\hskip 50pt} -- Color: \underbar{\hskip 50pt} 
\item{} Negative wire metal type: \underbar{\hskip 50pt} -- Color: \underbar{\hskip 50pt} 
\item{} Thermocouple-grade cable jacket color: \underbar{\hskip 50pt} 
\item{} Extension-grade cable jacket color: \underbar{\hskip 50pt} 
%\item{} Thermocouple temperature range -- Maximum: \underbar{\hskip 50pt} -- Minimum: \underbar{\hskip 50pt} 
\end{itemize}

\begin{itemize}
\item{} {\bf Type R}
\item{} Positive wire metal type: \underbar{\hskip 50pt} -- Color: \underbar{\hskip 50pt} 
\item{} Negative wire metal type: \underbar{\hskip 50pt} -- Color: \underbar{\hskip 50pt} 
\item{} Thermocouple-grade cable jacket color: \underbar{\hskip 50pt} 
\item{} Extension-grade cable jacket color: \underbar{\hskip 50pt} 
%\item{} Thermocouple temperature range -- Maximum: \underbar{\hskip 50pt} -- Minimum: \underbar{\hskip 50pt} 
\end{itemize}

\vskip 10pt

Additionally, rank these thermocouples in order of maximum temperature, from lowest to highest.

\vskip 20pt \vbox{\hrule \hbox{\strut \vrule{} {\bf Suggestions for Socratic discussion} \vrule} \hrule}

\begin{itemize}
\item{} Why use all these different types of thermocouples?  Why not just choose the one type with the highest temperature rating and one with the lowest temperature rating, and just use either of those two for any application we happen to encounter in industry?
\end{itemize}

\underbar{file i00389}
%(END_QUESTION)





%(BEGIN_ANSWER)

%(END_ANSWER)





%(BEGIN_NOTES)

\begin{itemize}
\item{} {\bf Type J} 
\item{} Positive wire metal type: Iron -- Color: White 
\item{} Negative wire metal type: Constantan -- Color: Red 
\item{} Thermocouple-grade cable jacket color: Brown 
\item{} Extension-grade cable jacket color: Black 
%\item{} Thermocouple temperature range -- Maximum: 1382$^{o}$ F -- Minimum: 32$^{o}$ F 
\end{itemize}

\vskip 10pt

\begin{itemize}
\item{} {\bf Type K} 
\item{} Positive wire metal type: Chromel -- Color: Yellow 
\item{} Negative wire metal type: Alumel -- Color: Red 
\item{} Thermocouple-grade cable jacket color: Brown 
\item{} Extension-grade cable jacket color: Yellow 
%\item{} Thermocouple temperature range -- Maximum: 2282$^{o}$ F -- Minimum: -328$^{o}$ F 
\end{itemize}

\vskip 10pt

\begin{itemize}
\item{} {\bf Type E} 
\item{} Positive wire metal type: Chromel -- Color: Violet 
\item{} Negative wire metal type: Constantan -- Color: Red 
\item{} Thermocouple-grade cable jacket color: Brown 
\item{} Extension-grade cable jacket color: Violet 
%\item{} Thermocouple temperature range -- Maximum: 1652$^{o}$ F -- Minimum: -328$^{o}$ F 
\end{itemize}

\vskip 10pt

\begin{itemize}
\item{} {\bf Type T} 
\item{} Positive wire metal type: Copper -- Color: Blue 
\item{} Negative wire metal type: Constantan -- Color: Red 
\item{} Thermocouple-grade cable jacket color: Brown 
\item{} Extension-grade cable jacket color: Blue 
%\item{} Thermocouple temperature range -- Maximum: 662$^{o}$ F -- Minimum: -328$^{o}$ F 
\end{itemize}

\vskip 10pt

\begin{itemize}
\item{} {\bf Type S} 
\item{} Positive wire metal type: Platinum 90\%, Rhodium 10\%-- Color: Black 
\item{} Negative wire metal type: Platinum -- Color: Red 
\item{} Thermocouple-grade cable jacket color: Brown 
\item{} Extension-grade cable jacket color: Green (not standardized, though)
%\item{} Thermocouple temperature range -- Maximum: 2642$^{o}$ F -- Minimum: 32$^{o}$ F 
\end{itemize}

\vskip 10pt

\begin{itemize}
\item{} {\bf Type R} 
\item{} Positive wire metal type: Platinum 87\%, Rhodium 13\%-- Color: Black 
\item{} Negative wire metal type: Platinum -- Color: Red 
\item{} Thermocouple-grade cable jacket color: Brown 
\item{} Extension-grade cable jacket color: Green (not standardized, though) 
%\item{} Thermocouple temperature range -- Maximum: 2642$^{o}$ F -- Minimum: 32$^{o}$ F 
\end{itemize}

\vskip 10pt

The order of maximum service temperature from lowest to highest is (S and R), K, E, J, and T.

\vskip 10pt

Note the common points between all thermocouple types: the negative wire is always colored {\it red}, and the thermocouple-grade cable jacket color is always {\it brown}.  Also, the positive wire color usually (but not always!) matches the extension-grade cable jacket color.

Over the years, I've developed (and copied!) some memory ``tricks'' to help me associate these colors and properties with their respective thermocouples:

\vskip 10pt

{\bf Type J} has a {\it jet} black extension-grade cable jacket color.  For some odd reason, though, the positive lead on this thermocouple is white, not black.

\vskip 10pt

{\bf Type K} has a {\it kanary} yellow extension-grade cable jacket color.  Its positive wire color is yellow as well.

\vskip 10pt

{\bf Type E} has a violet extension-grade cable jacket color (and positive wire color), just like the color purple was considered royal in {\it European} monarchies.  When I see the letter ``E'' I think of voltage, and type E just happens to have the greatest amount of mV/degree voltage/temperature relationship of any thermocouple type.

\vskip 10pt

{\bf Type T} is for low-{\it Temperatures}, and has an ice-cold {\it blue} cable jacket and positive wire color to match.

\vskip 10pt

{\bf Type S}: When I see the letter ``S'' I think of a dollar sign (\$).  This type of thermocouple is made of platinum, and is therefore {\it expensive}.  In accordance with its expense, its extension-grade cable jacket color is green, like American money.  Like type J, though, the positive wire color does not match its extension-grade jacket color: in this case, the positive wire is black.

\vskip 10pt

{\bf Type R} is nearly identical to type S, sharing the same colors all around.  Because R is a little ``lower'' in the alphabet than S, I think of type R thermocouples as being ``a step below'' type S in quality.  Accordingly, the percentages of Platinum and Rhodium in the positive wire alloy show less platinum and more rhodium than type S: a lowering of purity. 

\vfil \eject

\noindent
{\bf Prep Quiz:}

Identify the wire colors and metals for a {\it type K} thermocouple:

\begin{itemize}
\item{} Yellow and Red ; chromel and alumel
\vskip 5pt 
\item{} White and Red ; chromel and alumel
\vskip 5pt 
\item{} Blue and Red ; copper and constantan
\vskip 5pt 
\item{} Blue and Red ; iron and constantan
\vskip 5pt 
\item{} White and Red ; iron and constantan
\vskip 5pt 
\item{} Yellow and Red ; chromel and constantan
\end{itemize}

\vfil \eject

\noindent
{\bf Prep Quiz:}

Identify the wire colors and metals for a {\it type J} thermocouple:

\begin{itemize}
\item{} Yellow and Red ; chromel and alumel
\vskip 5pt 
\item{} White and Red ; chromel and alumel
\vskip 5pt 
\item{} Blue and Red ; copper and constantan
\vskip 5pt 
\item{} Blue and Red ; iron and constantan
\vskip 5pt 
\item{} White and Red ; iron and constantan
\vskip 5pt 
\item{} Yellow and Red ; chromel and constantan
\end{itemize}

\vfil \eject

\noindent
{\bf Prep Quiz:}

Identify the wire colors and metals for a {\it type T} thermocouple:

\begin{itemize}
\item{} Yellow and Red ; chromel and alumel
\vskip 5pt 
\item{} White and Red ; chromel and alumel
\vskip 5pt 
\item{} Blue and Red ; copper and constantan
\vskip 5pt 
\item{} Blue and Red ; iron and constantan
\vskip 5pt 
\item{} White and Red ; iron and constantan
\vskip 5pt 
\item{} Yellow and Red ; chromel and constantan
\end{itemize}

%INDEX% Measurement, temperature: thermocouple types (colors, metals, etc.)

%(END_NOTES)


