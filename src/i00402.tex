
%(BEGIN_QUESTION)
% Copyright 2006, Tony R. Kuphaldt, released under the Creative Commons Attribution License (v 1.0)
% This means you may do almost anything with this work of mine, so long as you give me proper credit

Terminal blocks used in thermocouple circuits are designed to be {\it isothermal}.  For example, the dual-terminal connection block found inside a thermocouple ``head'' box is a prime example of an isothermal block.  

Explain what this term ``isothermal'' means, how it is physically designed, and also why it is an important feature.

\underbar{file i00402}
%(END_QUESTION)





%(BEGIN_ANSWER)

The metal blocks into which thermocouple wires go to connect are usually made of heavy brass, and they are physically secured to a thick ceramic (electrically insulating) base.  This helps ensure the two connection points are held to the same temperature.

%(END_ANSWER)





%(BEGIN_NOTES)

Remember that a pair of thermocouple wire-to-copper wire junctions will act as a single equivalent junction between the thermocouple wire metal types {\it if and only if} both junctions are at the same temperature.  If the two terminals are at different temperatures, the Law of Intermediate metals does not apply!

%INDEX% Measurement, temperature: thermocouple

%(END_NOTES)


