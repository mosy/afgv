
%(BEGIN_QUESTION)
% Copyright 2006, Tony R. Kuphaldt, released under the Creative Commons Attribution License (v 1.0)
% This means you may do almost anything with this work of mine, so long as you give me proper credit

A platinum RTD with an $R_0$ of 1000 $\Omega$ and an $\alpha$ = 0.00392 $\Omega$/$\Omega \cdot ^{o}$C is heated to a temperature of 120$^{o}$ C.  Calculate its resistance at that temperature.

\vskip 10pt

Also, calculate the temperature of the same RTD if its resistance measures 1043.8 $\Omega$.

\vskip 20pt \vbox{\hrule \hbox{\strut \vrule{} {\bf Suggestions for Socratic discussion} \vrule} \hrule}

\begin{itemize}
\item{} Explain how an RTD may be constructed to have a different ``base'' resistance value (e.g. 1000 ohms versus 100 ohms).  Specifically, what is it about the RTD's physical construction that determines this resistance value?
\item{} Explain why RTDs do not all have the same ``alpha'' ($\alpha$) value.
\end{itemize}

\underbar{file i00406}
%(END_QUESTION)





%(BEGIN_ANSWER)

\noindent
{\bf Partial answer:}

\vskip 10pt

$R_T$ = 1470.4 $\Omega$ at 120$^{o}$ C

%(END_ANSWER)





%(BEGIN_NOTES)

$T$ = 11.173$^{o}$ C at 1043.8 $\Omega$

\vfil \eject

\noindent
{\bf Summary Quiz:}

Calculate the resistance of an RTD with an alpha value of 0.00385 $\Omega$/$\Omega ^{o}$C and a base resistance of 100 ohms (at the freezing point of water), when it is at a temperature of 450 degrees Fahrenheit.

\begin{itemize}
\item{} 189.4 $\Omega$
\vskip 5pt 
\item{} 100.0 $\Omega$
\vskip 5pt 
\item{} 260.9 $\Omega$ 
\vskip 5pt 
\item{} 89.41 $\Omega$ 
\vskip 5pt 
\item{} 196.3 $\Omega$ 
\vskip 5pt 
\item{} 273.3 $\Omega$ 
\end{itemize}


%INDEX% Measurement, temperature: RTD

%(END_NOTES)


