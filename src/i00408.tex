
%(BEGIN_QUESTION)
% Copyright 2012, Tony R. Kuphaldt, released under the Creative Commons Attribution License (v 1.0)
% This means you may do almost anything with this work of mine, so long as you give me proper credit

While on a camping trip, Sally needs to collect enough firewood to fuel a campfire that will bring 5 gallons of water to a boil.  Assuming a heat of combustion value of 9000 BTU per pound of dry wood, how much wood must Sally collect for the fire that will heat this water?  Assume that the ambient air temperature is 51 degrees F, and that the boiling point of water at Sally's altitude is 205 degrees F.

\underbar{file i00408}
%(END_QUESTION)





%(BEGIN_ANSWER)

This is a calorimetry problem, where we must use the specific heat of water (1 BTU per pound-degree F) to calculate the necessary heat for raising its temperature a specified amount (from 51 $^{o}$F to 205 $^{o}$F):

$$Q = mc \Delta T$$

Before we may use this formula, however, we need to figure out the mass (in pounds) for 5 gallons of water:

$$\left(5 \hbox{ gal} \over 1 \right) \left( 231 \hbox{ in}^3 \over 1 \hbox{ gal} \right) \left(1 \hbox{ ft}^3 \over 1728 \hbox{ in}^3 \right) \left( 62.4 \hbox{ lb} \over \hbox{ft}^3 \right) = 41.71 \hbox{ lb}$$

Now we are ready to plug all the values into our specific heat formula:

$$Q = mc \Delta T$$

$$Q = (41.71 \hbox{ lb}) (1 \hbox{ BTU/lb}\cdot^o\hbox{F}) (205^o\hbox{F} - 51^o\hbox{F}) = 6423 \hbox{ BTU}$$

If dry wood has a fuel value of 9000 BTU per pound, Sally should only (theoretically) need 0.713 pounds (11.4 ounces) of dry wood for this fire.

\vskip 10pt

However . . . we know that not all the heat from an open campfire gets tranferred to the water kettle.  Given the many forms of heat loss (radiation away from the kettle, convection up into the open air, evaporation of the heating water, etc., etc.), we can count on only a small fraction of the wood fire's heat going into useful heating of the 5 gallons of water.  Therefore, Sally will probably need to gather at least several pounds of wood to do the job.

%(END_ANSWER)





%(BEGIN_NOTES)


%INDEX% Physics, heat and temperature: calorimetry problem 

%(END_NOTES)


