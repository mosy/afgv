
%(BEGIN_QUESTION)
% Copyright 2006, Tony R. Kuphaldt, released under the Creative Commons Attribution License (v 1.0)
% This means you may do almost anything with this work of mine, so long as you give me proper credit

A truck weighing 39,000 newtons (N) is traveling at 10 meters per second when its clutch blows out, disconnecting the engine from the drivetrain (transmission, axle, wheels, etc.).  This failure occurs exactly at the base of a steep incline.  How many meters (vertical) will the truck coast up the hill with no engine power before it stops, neglecting friction of any kind?

\vskip 10pt

How high would the truck have coasted if it had been traveling twice as fast?

\underbar{file i00431}
%(END_QUESTION)





%(BEGIN_ANSWER)

5.1 meters (measured vertically) at an initial velocity of 10 m/s.  At 20 m/s, the truck would have gained {\it four times} as much altitude (20.4 meters)!

\vskip 10pt

Knowing that potential energy when the truck reaches its stopping point on the hill should be equal to kinetic energy when the clutch fails (assuming zero energy loss due to friction), we may solve for $h$ quite easily:

$$E_p = mgh \hbox{\hskip 100pt} E_k = {1 \over 2}mv^2$$

$$mgh = {1 \over 2}mv^2$$

$$gh = {1 \over 2}v^2$$

$$h = {{1 \over 2}v^2 \over g}$$

$$h = {v^2 \over 2g}$$

Note that the slope of the hill is unspecified, because it is irrelevant to the answer of how much vertical height the truck gains by coasting.  What {\it would} the slope of the hill affect, though?

%(END_ANSWER)





%(BEGIN_NOTES)


%INDEX% Physics, energy, work, power: kinetic energy transformed to potential

%(END_NOTES)


