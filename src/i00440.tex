
%(BEGIN_QUESTION)
% Copyright 2006, Tony R. Kuphaldt, released under the Creative Commons Attribution License (v 1.0)
% This means you may do almost anything with this work of mine, so long as you give me proper credit

As best as possible, define {\it Reynolds number} for a fluid flow using your own words.

\vskip 10pt

Then, calculate the Reynolds number for 500 gallons per minute of water (at 20$^{o}$ C) flowing through a pipe with an inside diameter of 8.5 inches.

\underbar{file i00440}
%(END_QUESTION)





%(BEGIN_ANSWER)

The {\it Reynolds number} for a fluid flow is the ratio of a fluid's inertial (motion) forces as compared to its friction (viscous) forces.

\vskip 30pt
 
To calculate Reynolds number given metric units:

$$\hbox{Re} = {{D \overline{V} \rho} \over \mu}$$

\noindent
Where,

Re = Reynolds number (unitless)

$D$ = Diameter of pipe, in meters (m)

$\overline{V}$ = Average velocity of fluid, in meters per second (m/s)

$\rho$ = Mass density of fluid, in kilograms per cubic meter (kg/m$^{3}$)

$\mu$ = Absolute viscosity of fluid, in Pascal-seconds (Pa $\cdot$ s)

\vskip 60pt \goodbreak

To calculate Reynolds number given English units (liquid flow):

$$\hbox{Re} = {{(50.7) \rho Q} \over {D \mu}}$$

\noindent
Where,

Re = Reynolds number (unitless)

$\rho$ = Mass density of liquid, in pounds (mass) per cubic foot (lbm/ft$^{3}$)

$Q$ = Flow rate, gallons per minute (GPM)

$D$ = Diameter of pipe, in inches (in)

$\mu$ = Absolute viscosity of fluid, in centipoise (cP)

\vskip 60pt \goodbreak

To calculate Reynolds number given English units (gas flow):

$$\hbox{Re} = {{(6.32) \rho Q} \over {D \mu}}$$

\noindent
Where,

Re = Reynolds number (unitless)

$\rho$ = Mass density of gas, in pounds (mass) per cubic foot (lbm/ft$^{3}$)

$Q$ = Flow rate, standard cubic feet per hour (SCFH)

$D$ = Diameter of pipe, in inches (in)

$\mu$ = Absolute viscosity of fluid, in centipoise (cP)

\vskip 60pt \goodbreak

To calculate Reynolds number given specific gravity instead of density (liquid flow):

$$\hbox{Re} = {{(3160) G_f Q} \over {D \mu}}$$

\noindent
Where,

Re = Reynolds number (unitless)

$G_f$ = Specific gravity of liquid (unitless)

$Q$ = Flow rate, gallons per minute (GPM)

$D$ = Diameter of pipe, in inches (in)

$\mu$ = Absolute viscosity of fluid, in centipoise (cP)

\vskip 60pt \goodbreak

Re = 186,000 for 500 gallons per minute of water flowing through an 8.5 inch pipe.

%(END_ANSWER)





%(BEGIN_NOTES)

Reynolds number is an expression of a flow stream's kinetic action versus dissipative resistance.  It is a unitless number, and a very important one as far as turbulence is concerned.

%INDEX% Physics, dynamic fluids: Reynolds number

%(END_NOTES)


