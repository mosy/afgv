
%(BEGIN_QUESTION)
% Copyright 2006, Tony R. Kuphaldt, released under the Creative Commons Attribution License (v 1.0)
% This means you may do almost anything with this work of mine, so long as you give me proper credit

Jeg kan fylle et kar på 4 liter kjøkken springen på 30 sekunder. Røret til kjøkkenspringen har en ID=12.5mm. Regn ut Reynols tall for denne strømningen. Vil det være turbulent eller laminær stømning?


\underbar{file i00442}
%(END_QUESTION)





%(BEGIN_ANSWER)

Re $\approx$ 13581 = {\it turbulent}

\vskip 10pt

Reynolds numbers less than 2,000 usually correspond to laminar flows, while Reynolds numbers above 10,000 usually correspond to turbulent flows.  Reynolds numbers between 2,000 and 10,000 usually represent conditions of mild turbulence called ``transitional flow.''  Bear in mind these cutoff points are {\it very approximate}, and depend on many factors including pipe geometry and wall smoothness.

Examples of Reynolds number thresholds for laminar vs. turbulent flows are given here, from different sources:

\begin{itemize}
\goodbreak
\item{} Re $<$ 2,000 = ``Laminar''
\item{} 2,000 $<$ Re $<$ 10,000 = ``Transitional''
\item{} Re $>$ 10,000 = ``Fully developed turbulent''
\item{} Source: R. Siev, J.B. Arant, B.G. Lipt\'ak; \underbar{Chapter 2.8: Laminar Flowmeters}; {\it Instrument Engineer's Handbook, Process Measurement and Analysis, Third Edition}; pg. 105
\end{itemize}

\begin{itemize}
\goodbreak
\item{} Re $>$ 10,000 = ``Definitely turbulent''
\item{} Source: W.H. Howe, J.B. Arant, B.G. Lipt\'ak; \underbar{Chapter 2.14: Orifices}; {\it Instrument Engineer's Handbook, Process Measurement and Analysis, Third Edition}; pg. 153
\end{itemize}

\begin{itemize}
\goodbreak
\item{} Re $<$ 2,000 = ``Laminar''
\item{} 2,000 $<$ Re $<$ 4,000 = ``Transitional''
\item{} Re $>$ 4,000 = ``Turbulent''
\item{} Source: Instrument Society of America; \underbar{Chapter 2: Fluid Properties -- Part II}; {\it ISA Industrial Measurement Series -- Flow}; pg. 11
\end{itemize}

\begin{itemize}
\goodbreak
\item{} Re $<$ 2,100 = ``Laminar''
\item{} Re $>$ 3,000 = ``Turbulent''
\item{} Source: Tyler G. Hicks, P.E.; \underbar{Laminar Flow in a Pipe}; {\it Standard Handbook of Engineering Calculations}; pg. 1-202
\end{itemize}

\begin{itemize}
\goodbreak
\item{} Re $<$ 1,200 = ``Laminar''
\item{} Re $>$ 2,500 = ``Turbulent''
\item{} Source: Tyler G. Hicks, P.E.; \underbar{Piping and Fluid Flow}; {\it Standard Handbook of Engineering Calculations}; pg. 3-384
\end{itemize}

You've got to laugh when you see such vastly different threshold values given in the exact same reference book!

\begin{itemize}
\goodbreak
\item{} Re $<$ (about) 2,000 = ``Laminar''
\item{} Re $>$ 2,000 = ``Turbulent''
\item{} Source: Douglas C. Giancoli; \underbar{Chapter 10: Fluids}; {\it Physics (Third Edition)}; pg. 11
\end{itemize}

\begin{itemize}
\goodbreak
\item{} Re $<$ (about) 2,000 = ``Laminar''
\item{} 2,000 $<$ Re $<$ 4,000 = ``Transitional''
\item{} Re $>$ 4,000 = ``Turbulent''
\item{} Source: Schoolcraft Publishing; \underbar{Chapter 20: Properties of Fluid Flow}; {\it Process Instrumentation -- Volume I}; pg. 258
\end{itemize}

\goodbreak

Another source, laughable in its attempt to precisely demarcate the threshold of turbulence, gives these figures:

\begin{itemize}
\item{} Re $<$ 2,320 = ``Laminar''
\item{} Re $>$ 2,320 = ``Turbulent''
\item{} Source: Website ({\tt http://flow.netfirms.com/reynolds/theory.htm})
\end{itemize}

It should be noted that laminar flow can be sustained at Reynolds numbers significantly in excess of 10,000 under very special circumstances.  For example, in certain coiled capillary tubes, laminar flow may be sustained all the way up to Re = 15,000, due to something known as the {\it Dean effect}!

%(END_ANSWER)





%(BEGIN_NOTES)


%INDEX% Physics, dynamic fluids: Reynolds number

%(END_NOTES)


