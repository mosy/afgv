
%(BEGIN_QUESTION)
% Copyright 2015, Tony R. Kuphaldt, released under the Creative Commons Attribution License (v 1.0)
% This means you may do almost anything with this work of mine, so long as you give me proper credit

Read and outline the introduction and ``PLC Examples'' sections of the ``Programmable Logic Controllers'' chapter in your {\it Lessons In Industrial Instrumentation} textbook.  Note the page numbers where important illustrations, photographs, equations, tables, and other relevant details are found.  Prepare to thoughtfully discuss with your instructor and classmates the concepts and examples explored in this reading.

\underbar{file i00460}
%(END_QUESTION)





%(BEGIN_ANSWER)


%(END_ANSWER)





%(BEGIN_NOTES)

PLCs are programmable controllers, useful for many different control applications.  Originally designed to replace networks of electromechanical relays.  Analog I/O and control capability now common for PLCs in addition to discrete I/O and control.  Ladder Diagram programming resembles electrical schematic diagrams, to ease the digital transition for electricians who would install and use PLCs.  

\vskip 10pt

Some PLCs have modular construction, with circuit ``cards'' plugging into a rack, while others are ``monolithic'' having all I/O built-in to the main body of the unit.  Modular construction lends itself to easy replacement and upgrade of I/O.  Semi-modular construction has I/O built into the main unit, plus the ability to expand I/O by plugging in additional I/O cards.  Both processor and I/O typically have visual indications of mode and status.

PLC hardware is very rugged and reliable, which means it is common to find obsolete PLCs still in use controlling industrial processes!












\vskip 20pt \vbox{\hrule \hbox{\strut \vrule{} {\bf Suggestions for Socratic discussion} \vrule} \hrule}

\begin{itemize}
\item{} Explain what a PLC is, and what one might be good for.
\item{} Describe the difference(s) between a ``monolithic'' PLC and a ``modular'' PLC.
\item{} Identify some of the practical PLC control applications mentioned in the book.
\itemitem{} Wastewater treatment trash rack control (Siemens 505)
\itemitem{} Gas compressor control (Allen-Bradley PLC-5)
\itemitem{} High-purity water treatment system (Allen-Bradley SLC 500)
\itemitem{} Cereal processing system (Allen-Bradley ControlLogix 5000)
\itemitem{} Hydro power generator control (General Electric Series One)
\end{itemize}

%INDEX% Reading assignment: Lessons In Industrial Instrumentation, Programmable Logic Controllers (intro, PLC examples)

%(END_NOTES)

