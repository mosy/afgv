
%(BEGIN_QUESTION)
% Copyright 2006, Tony R. Kuphaldt, released under the Creative Commons Attribution License (v 1.0)
% This means you may do almost anything with this work of mine, so long as you give me proper credit

The rate of volumetric flow through {\it any} head-generating flow element is proportional to the square root of the differential pressure measured across it, so long as the flow regime is ``fully-developed'' turbulent:

$$Q \propto \sqrt{P}$$

Re-write this proportionality in the form of an equation, then solve for the new constant of proportionality ($k$) given these full-flow ratings for an orifice plate:

\begin{itemize}
\item{} Full flow $Q$ = 270 m³/h
\item{} $\Delta$P at full flow = 25 kPa
\end{itemize}

Now that you have a value for $k$, solve for the differential pressure across the orifice plate at these flow rates:

\begin{itemize}
\item{} $Q$ = 110 m³/h ; $\Delta$P = \underbar{\hskip 50pt} kPa
\vskip 5pt
\item{} $Q$ = 55 m³/h ; $\Delta$P = \underbar{\hskip 50pt} kPa
\vskip 5pt
\item{} $Q$ = 140 m³/h ; $\Delta$P = \underbar{\hskip 50pt} kPa
\vskip 5pt
\item{} $Q$ = 215 m³/h ; $\Delta$P = \underbar{\hskip 50pt} kPa
\end{itemize}

%\vskip 20pt \vbox{\hrule \hbox{\strut \vrule{} {\bf Suggestions for Socratic discussion} \vrule} \hrule}
%
%\begin{itemize}
%\item{} Explain why we need not pay attention to maintaining compatible units of measurement for flow and pressure when solving this type of problem the way we did when using Bernoulli's equation directly. 
%\item{} Why is it okay to use this general formula for {\it any} primary flow element based on differential pressure?  There are many different types of flow elements (venturis, orifices, nozzles, Pitot tubes, segmented wedge tubes, etc.), each with its own unique design.  What is common to all these elements that the same basic equation form may be used to describe the operation of them all?
%\end{itemize}

\underbar{file i00474}
%(END_QUESTION)





%(BEGIN_ANSWER)

\noindent
{\bf Partial answer:}

\begin{itemize}
\item{} $Q$ = 110 m³/h ; $\Delta$P = 4.15 kPa
\vskip 5pt
\item{} $Q$ = 55 m³/h ; $\Delta$P = 1.04 kPa
\vskip 5pt
\item{} $Q$ = 140 m³/h ; $\Delta$P = 6.75 kPa
\vskip 5pt
\item{} $Q$ = 215 m³/h ; $\Delta$P = 15.85 kPa
\end{itemize}

%(END_ANSWER)





%(BEGIN_NOTES)

$$Q = k \sqrt{P}$$

$k$ = 120 if $P$ = 100 when $Q$ = 1200

\begin{itemize}
\item{} $Q$ = 500 GPM ; $\Delta$P = \underbar{\bf 17.36} inches water column
\item{} $Q$ = 245 GPM ; $\Delta$P = \underbar{\bf 4.168} inches water column
\item{} $Q$ = 630 GPM ; $\Delta$P = \underbar{\bf 27.56} inches water column
\item{} $Q$ = 950 GPM ; $\Delta$P = \underbar{\bf 62.67} inches water column
\end{itemize}


%INDEX% Mathematics, proportionalities converted into equalities
%INDEX% Measurement, flow: simple ``k'' factor equation for flow/pressure correlation

%(END_NOTES)


