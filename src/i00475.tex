
%(BEGIN_QUESTION)
% Copyright 2006, Tony R. Kuphaldt, released under the Creative Commons Attribution License (v 1.0)
% This means you may do almost anything with this work of mine, so long as you give me proper credit

Suppose a 6 inch V-cone flow element is sized to generate a $\Delta$P of 30 kPa at a flow rate of 160 m³/h.  Determine the new differential pressure instrument calibration ranges if this same flow element will now be used to measure the following water flow ranges:

\begin{itemize}
\item{} $Q$ range = 0 to 110 m³/h ; $\Delta P$ range = \underbar{\hskip 50pt}
\vskip 5pt
\item{} $Q$ range = 0 to 140 m³/h ; $\Delta P$ range = \underbar{\hskip 50pt}
\vskip 5pt
\item{} $Q$ range = 0 to 180 m³/h ; $\Delta P$ range = \underbar{\hskip 50pt}
\vskip 5pt
\item{} $Q$ range = 0 to 230 m³/h ; $\Delta P$ range = \underbar{\hskip 50pt}
\end{itemize}

%\vskip 20pt \vbox{\hrule \hbox{\strut \vrule{} {\bf Suggestions for Socratic discussion} \vrule} \hrule}
%
%\begin{itemize}
%\item{} If the density of the fluid being measured by this flowmeter were to suddenly change, would it affect the {\it zero}, the {\it span}, or the {\it linearity} of the flowmeter's calibration?
%\end{itemize}

\underbar{file i00475}
%(END_QUESTION)





%(BEGIN_ANSWER)

\noindent
{\bf Partial answer:}

\begin{itemize}
\item{} $Q$ range = 0 to 110 m³/h ; $\Delta P$ range = 0-14.18 kPa
\vskip 5pt
\item{} $Q$ range = 0 to 140 m³/h ; $\Delta P$ range = 0-22.97 kPa
\vskip 5pt
\item{} $Q$ range = 0 to 180 m³/h ; $\Delta P$ range = 0-37.97 kPa
\vskip 5pt
\item{} $Q$ range = 0 to 230 m³/h ; $\Delta P$ range = 0-62.00 kPa
\end{itemize}

%(END_ANSWER)





%(BEGIN_NOTES)

$$Q = 62.61 \sqrt{\Delta P}$$

\begin{itemize}
\item{} $Q$ range = 0 to 500 GPM ; $\Delta P$ range = \underbar{\bf 0 to 63.78 "W.C.}
\item{} $Q$ range = 0 to 600 GPM ; $\Delta P$ range = \underbar{\bf 0 to 91.84 "W.C.}
\item{} $Q$ range = 0 to 800 GPM ; $\Delta P$ range = \underbar{\bf 0 to 163.3 "W.C.}
\item{} $Q$ range = 0 to 1000 GPM ; $\Delta P$ range = \underbar{\bf 0 to 255.1 "W.C.}
\end{itemize}

\vskip 10pt

The pipe size (6 inch) of the V-cone element is extraneous information, included for the purpose of challenging students to identify whether or not information is relevant to solving a particular problem.


%INDEX% Measurement, flow: re-ranging orifice plates

%(END_NOTES)


