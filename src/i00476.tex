
%(BEGIN_QUESTION)
% Copyright 2011, Tony R. Kuphaldt, released under the Creative Commons Attribution License (v 1.0)
% This means you may do almost anything with this work of mine, so long as you give me proper credit

Yagi antennas are sometimes used in radio SCADA systems at remote terminal units (RTUs) for stronger signal reception and transmission.  Explain why Yagi antennas are so effective, and also identify the proper antenna type to use at the master terminal unit (MTU), and the proper antenna orientation to use at both points.

\vskip 10pt

Do you think Yagi antennas would be good to use in a mesh-network such as {\sl Wireless}HART?  Why or why not?

\underbar{file i00476}
%(END_QUESTION)





%(BEGIN_ANSWER)

Yagi antennas are highly directional, giving them high gain.  Used in conjunction with a vertical whip antenna at the MTU, the elements of the RTU Yagi antennas should likewise be oriented vertically.

\vskip 10pt

Yagi antennas are not good for mesh-networking systems such as {\sl Wireless}HART because they do not facilitate communication with neighboring devices out of the line-of-site path with the gateway.  Omnidirectional antennas are best here, ideally a 1/2 wave whip or dipole for maximum gain.

%(END_ANSWER)





%(BEGIN_NOTES)


%INDEX% Electronics review: antenna mounting orientation
%INDEX% Electronics review: antenna type

%(END_NOTES)


