
%(BEGIN_QUESTION)
% Copyright 2006, Tony R. Kuphaldt, released under the Creative Commons Attribution License (v 1.0)
% This means you may do almost anything with this work of mine, so long as you give me proper credit

Explain what a {\it weir} is, and how it works as a primary sensing element for flow.

\underbar{file i00480}
%(END_QUESTION)





%(BEGIN_ANSWER)

They say a picture is worth a thousand words . . .

$$\includegraphics[width=15.5cm]{i00480x01.eps}$$

A {\it weir} is nothing more than a dam that impedes the flow of liquid in an open channel.  In order for liquid to flow past this dam, its upstream level must rise high enough so that it spills over the top of the dam.  The more volumetric flow there is, the higher the upstream liquid level will rise.  So, by measuring level, the flow rate may be inferred.

Most instrumentation weirs have notches cut into them to create exact flow characteristics.  Among the more popular weir types are {\it rectangular}, {\it Cippoletti} ({\it trapezoidal}), and {\it V-notch}.

To measure the upstream liquid level, a displacer or ultrasonic type of level instrument is usually used, inside of a {\it stilling well}: a length of vertical pipe inserted into the water to provide a ``still'' column of liquid to measure.  If there were no stilling well, the turbulent surface of the liquid might produce measurement errors.

%(END_ANSWER)





%(BEGIN_NOTES)


%INDEX% Measurement, flow: weirs

%(END_NOTES)


