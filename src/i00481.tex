
%(BEGIN_QUESTION)
% Copyright 2006, Tony R. Kuphaldt, released under the Creative Commons Attribution License (v 1.0)
% This means you may do almost anything with this work of mine, so long as you give me proper credit

Explain what a {\it flume} is, and how it works as a primary sensing element for flow.

\underbar{file i00481}
%(END_QUESTION)





%(BEGIN_ANSWER)

Ooooh . . . aahhhhh . . . a {\it three-dimensional} picture is worth 1000$^{3}$ words!

$$\includegraphics[width=15.5cm]{i00481x01.eps}$$

A {\it flume} is a constricted channel that produces different upstream liquid levels for different flow rates through it.  Whereas a weir is a dam installed in the path of the liquid flow, a flume is a special channel section with no abrupt obstructions in it.

While flumes are more difficult to construct than weirs, their advantages over weirs are significant: flumes tend to be self-cleaning of silt and other debris that would otherwise plug up a weir, and flumes do not lose as much liquid head from entry to exit as weirs do (less permanent head loss).

%(END_ANSWER)





%(BEGIN_NOTES)


%INDEX% Measurement, flow: flumes

%(END_NOTES)


