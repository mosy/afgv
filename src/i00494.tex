
%(BEGIN_QUESTION)
% Copyright 2008, Tony R. Kuphaldt, released under the Creative Commons Attribution License (v 1.0)
% This means you may do almost anything with this work of mine, so long as you give me proper credit

Compare and contrast the vortex-shedding flow meter against the standard orifice plate flow meter.  What are some of the advantages of vortex meters over orifice plates?  Are there any significant disadvantages?  

Also, compare signal linearity between the two flow measurement technologies: we know that orifice plates require square-root characterization to obtain a linear response to flow rate.  Is the same true for vortex meters?  Why or why not?

\underbar{file i00494}
%(END_QUESTION)





%(BEGIN_ANSWER)

\begin{itemize}
\item{} {\bf Advantages of vortex meters over orifice plates}
\item{} Immune to changes in fluid density (and therefore temperature and pressure as well)
\item{} Linear output requires no square-root characterization
\item{} Better rangeability due to linear flow response (at least down to the ``cut off'' point)
\end{itemize}

\begin{itemize}
\item{} {\bf Advantages of orifice plates over vortex meters}
\item{} Cheaper for very large pipe sizes
\item{} Orifice plates may be more tolerant of low-frequency pipe vibrations
\item{} Some orifice plates may measure bidirectional flow
\item{} Able to sense flow down to zero (vortex flowmeters will ``cut off'' at some low flow rate)
\end{itemize}

\vskip 10pt

Low-flow cutoff is a problem unique to vortex flowmeters.  At low flow rates, the Reynolds number drops below the turbulent threshold, at which point fluid viscosity prevents vortices from shedding.  The vortex street simply ceases to exist at any flow rate below this critical point, meaning the flowmeter's output goes to zero at any flow rate below the cutoff point.

%(END_ANSWER)





%(BEGIN_NOTES)


%INDEX% Measurement, flow: vortex shedding vs. orifice plate

%(END_NOTES)


