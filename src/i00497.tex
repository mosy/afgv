
%(BEGIN_QUESTION)
% Copyright 2008, Tony R. Kuphaldt, released under the Creative Commons Attribution License (v 1.0)
% This means you may do almost anything with this work of mine, so long as you give me proper credit

{\it Turbine} flow meters are almost self-explanatory in their operation.  Compare and contrast the turbine flow meter against the standard orifice plate flow meter as a flow-measuring device.  What are some of the advantages of turbine meters over orifice plates?  Are there any significant disadvantages?  

Also, compare signal linearity between the two flow measurement technologies: we know that orifice plates require square-root characterization to obtain a linear response to flow rate.  Is the same true for turbine meters?  Why or why not?

\underbar{file i00497}
%(END_QUESTION)





%(BEGIN_ANSWER)

\begin{itemize}
\item{} {\bf Advantages of turbine meters over orifice plates}
\item{} Very high accuracy
\item{} Linear output requires no square-root characterization
\item{} Better rangeability due to linear response to flow
\end{itemize}

\begin{itemize}
\item{} {\bf Advantages of orifice plates over turbine meters}
\item{} Typically cheaper
\item{} Cleanliness of flow stream not as critical
\item{} Turbine may become bound if viscous or fibrous solids are present in the flow stream
\item{} Less wear over time (no bearings to wear out)
\end{itemize}

%(END_ANSWER)





%(BEGIN_NOTES)


%INDEX% Measurement, flow: turbine vs. orifice plate

%(END_NOTES)


