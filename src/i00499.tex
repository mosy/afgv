
%(BEGIN_QUESTION)
% Copyright 2006, Tony R. Kuphaldt, released under the Creative Commons Attribution License (v 1.0)
% This means you may do almost anything with this work of mine, so long as you give me proper credit

What is meant by the phrase ``mass flow measurement,'' especially as it compares to ``volumetric flow measurement?''  In other words, what is the practical difference between measuring flow rate in units of mass per time versus units of volume per time?

\underbar{file i00499}
%(END_QUESTION)





%(BEGIN_ANSWER)

Mass flow measurement entails detecting units of mass (pounds, kilograms, etc.) passing by a specific point in a pipe or tube.  Volumetric flow measurement entails detecting units of volume (cubic feet, gallons, liters, etc.) passing by a specific point in a pipe or tube.

Mass flow measurement will give true mass figures for the fluid flow rate.  Volumetric flow measurement must be corrected for fluid density in order to obtain real figures for mass.

%(END_ANSWER)





%(BEGIN_NOTES)

Some measurement technologies, such as positive displacement metering, are inherently volumetric, and provide no intrinsic correction for density.  Other measurement technologies, such as those based on the Coriolis effect, are inherently mass-related, and need no correction for fluid density to provide accurate figures in pounds (mass) or kilograms per unit time.

%INDEX% Measurement, flow: mass flow versus volumetric flow
%INDEX% Physics, dynamic fluids: mass flow versus volumetric flow

%(END_NOTES)


