
%(BEGIN_QUESTION)
% Copyright 2011, Tony R. Kuphaldt, released under the Creative Commons Attribution License (v 1.0)
% This means you may do almost anything with this work of mine, so long as you give me proper credit

``Whip''-style radio antennas are typically mounted in a vertical orientation, either pointing up or pointing down.  Explain why this is the preferred orientatin for this style of antenna.

\vskip 10pt

Now suppose someone breaks this rule when installing two whip antennas on a radio transceiver pair, separated by 1000 feet of distance: the antennas are pointing {\it horizontal} rather than vertical.  How should the horizontal whip antennas be pointed in order to maximize their effectiveness?

\vskip 20pt \vbox{\hrule \hbox{\strut \vrule{} {\bf Suggestions for Socratic discussion} \vrule} \hrule}

\begin{itemize}
\item{} In which direction(s) will the horizontal whip antennas be {\it minimally} effective?  How do you know this to be the case?
\end{itemize}

\underbar{file i00519}
%(END_QUESTION)





%(BEGIN_ANSWER)

The two horizontal antennas should be parallel to each other, and perpendicular to the path-line between them.

%(END_ANSWER)





%(BEGIN_NOTES)

The radiation pattern of a whip antenna shows clearly why the two need to be oriented as described in the answer!

%INDEX% Electronics review: antenna mounting orientation

%(END_NOTES)


