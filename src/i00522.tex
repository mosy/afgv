
%(BEGIN_QUESTION)
% Copyright 2006, Tony R. Kuphaldt, released under the Creative Commons Attribution License (v 1.0)
% This means you may do almost anything with this work of mine, so long as you give me proper credit

Michael Faraday, the famous English physicist, once attempted to {\it electrically} measure the flow rate of the Thames river.  His technique, unsuccessful as it was, was based on the idea of a fluid conductor moving perpendicular to a magnetic field.  Explain what Michael Faraday did and how we may apply this technique (successfully!) to the measurement of fluids in pipes.

\underbar{file i00522}
%(END_QUESTION)





%(BEGIN_ANSWER)

I'll let you figure out the answer to this question!

%(END_ANSWER)





%(BEGIN_NOTES)


%INDEX% Measurement, flow: magnetic

%(END_NOTES)


