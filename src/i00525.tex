
%(BEGIN_QUESTION)
% Copyright 2006, Tony R. Kuphaldt, released under the Creative Commons Attribution License (v 1.0)
% This means you may do almost anything with this work of mine, so long as you give me proper credit

Magnetic flowmeters exhibit special advantages and disadvantages when compared to other flow-measuring technologies.  For each of the following strengths and weaknesses, explain {\it why} it is this way for a magnetic flowmeter:

\vskip 10pt

{\bf Strengths:}

\begin{itemize}
\item{} Short upstream/downstream straight-pipe requirements: 5 up and 3 down (typically)
\item{} Output is linearly related to volumetric flow rate -- no square root characterization required
\item{} Good rangeability
\item{} Bidirectional measurement possible
\end{itemize}

\vskip 10pt

{\bf Weaknesses:}

\begin{itemize}
\item{} Does not work with nonconducting fluids
\item{} Excellent electrical grounding of the flowmeter is {\it essential}
\item{} Coating of electrodes may affect performance
\item{} Needs to be installed in pipe with electrodes horizontal, never vertical
\end{itemize}

\vskip 10pt

Suppose a magflow meter is operating with a partially-filled pipe, with both electrodes still fully contacting the liquid.  Will this operating condition cause a {\it zero shift}, a {\it span shift}, a {\it linearity error}, or a {\it hysteresis error}?  Explain your reasoning.

\vskip 10pt

Suppose the flowstream through a magflow meter contains some non-conductive solids in addition to conductive liquid.  Will this affect the accuracy or reliability of the flowmeter?  Explain why or why not.

\vskip 10pt

\underbar{file i00525}
%(END_QUESTION)





%(BEGIN_ANSWER)


%(END_ANSWER)





%(BEGIN_NOTES)

{\bf Strengths:}

\begin{itemize}
\item{} Short upstream/downstream straight-pipe requirements: 5 up and 3 down -- {\it induced EMF is a good representation of average flow rate; directions of flow not perpendicular to magnetic field and to electrodes don't contribute much to the EMF}
\item{} Output is linearly related to volumetric flow rate -- no square root characterization required -- {\it electromagnetic induction is a linear function of velocity}
\item{} Good rangeability -- {\it linearity makes this possible; limited by ambient electrical noise, though}
\item{} Bidirectional measurement possible -- {\it polarity (phase) simply reverses}
\end{itemize}

\vskip 10pt

{\bf Weaknesses:}

\begin{itemize}
\item{} Does not work with nonconducting fluids -- {\it induction requires a conductor}
\item{} Excellent electrical grounding of the flowmeter is {\it essential} -- {\it to minimize effects of electrical noise on measurement}
\item{} Coating of electrodes may affect performance -- {\it adds electrical resistance to measurement circuit}
\item{} Needs to be installed in pipe with electrodes horizontal, never vertical -- {\it so air bubble will have a harder time breaking the measurement circuit}
\end{itemize}

\vskip 10pt

A partially-filled pipe will cause the velocity to be greater for any given volumetric flow rate (less effective cross-sectional area).  This multiplies the flowmeter's indication, causing it to exhibit a positive span error.

\vskip 10pt

The presence of non-conductive matter in the flowstream is of no consequence, so long as the electrodes do not become fouled with a non-conductive coating.  Otherwise, the presence of conductive liquid around the non-conductive matter will still ensure conductivity across the diameter of the magnetic flowmeter, and thus will continue to ensure its proper operation.

Students commonly think that the presence of non-conducting material in an otherwise conductive stream of liquid is somehow a deal-breaker for magflow meters.  Practical examples to the contrary include wastewater treatment and wood pulp flow measurement.


%INDEX% Measurement, flow: magnetic

%(END_NOTES)


