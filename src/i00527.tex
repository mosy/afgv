
%(BEGIN_QUESTION)
% Copyright 2006, Tony R. Kuphaldt, released under the Creative Commons Attribution License (v 1.0)
% This means you may do almost anything with this work of mine, so long as you give me proper credit

Describe the operational principles of two types of {\it ultrasonic} flowmeter technologies: {\it Doppler} and {\it transit-time}.  What physical properties of the fluid stream affect an ultrasonic flowmeter's calibration?

\underbar{file i00527}
%(END_QUESTION)





%(BEGIN_ANSWER)

Flow stream velocity may be measured via the use of sound waves transmitted and received through the liquid.  One sonic technology, called {\it Doppler}, infers velocity by the change in sound frequency between the transmitted sound wave and the received sound wave.

Another sonic flowmeter technology, called {\it transit-time}, measures liquid velocity by measuring the difference between upstream and downstream velocities of sound waves transmitted through the fluid.

Doppler flowmeter calibration depends on the speed of sound through the process fluid.  Transit-time flowmeter calibration does not.  Ultrasonic flowmeters are not suitable for multiphase (vapor/liquid mixed) flows, and thus the pipe must be completely full of liquid (no gas pockets) or completely full of gas (no puddles or streams of liquid) in order to function properly.  

%(END_ANSWER)





%(BEGIN_NOTES)

Velocity profile in the liquid stream is another important factor: since the sound waves traverse the entire pipe diameter, they will encounter velocities across all points of the flow.  Thus, the Reynolds number, which largely determines the velocity profile of the liquid flow, will impact calibration of an ultrasonic flowmeter.

%INDEX% Measurement, flow: ultrasonic

%(END_NOTES)


