
%(BEGIN_QUESTION)
% Copyright 2006, Tony R. Kuphaldt, released under the Creative Commons Attribution License (v 1.0)
% This means you may do almost anything with this work of mine, so long as you give me proper credit

One way to measure the mass flow rate of a fluid stream is to use an impeller spun by a constant-speed motor (usually an AC synchronous motor) and a turbine downstream of the impeller to measure the angular momentum of the fluid.  This arrangement is commonly called an {\it impeller-turbine flowmeter}.  

Explain how the synchronous motor maintains a constant speed regardless of mechanical load, and also explain how the angular momentum of the fluid directly relates to flow.

\underbar{file i00532}
%(END_QUESTION)





%(BEGIN_ANSWER)

{\it Synchronous} AC motors spin at a speed dictated by the frequency of the power line voltage (60 Hz in the United States) regardless of loading.  The rotor of a synchronous AC motor is magnetically ``locked'' on to the rotating magnetic field produced by the stator windings.

As for explaining how the fluid's angular momentum relates to mass flow, I'll let you explore that concept by setting up a few ``thought experiments'' of different fluids and different flow rates.

%(END_ANSWER)





%(BEGIN_NOTES)


%INDEX% Measurement, flow: turbine/impeller (mass)

%(END_NOTES)


