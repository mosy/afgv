
%(BEGIN_QUESTION)
% Copyright 2006, Tony R. Kuphaldt, released under the Creative Commons Attribution License (v 1.0)
% This means you may do almost anything with this work of mine, so long as you give me proper credit

On cold days, the perception of ambient temperature becomes skewed with the presence of any substantial wind speed.  Although a thermometer will not register a colder temperature if the wind blows, it certainly {\it feels} colder when the wind blows!  This effect is commonly known as {\it wind chill}.

This same effect may be exploited to measure mass flow.  Flowmeters based on this principle are called {\it thermal} mass flowmeters.  Explain how a thermal mass flowmeter is constructed, and identify any physical properties of the fluid stream that will affect its calibration.

\underbar{file i00534}
%(END_QUESTION)





%(BEGIN_ANSWER)

Thermal mass flowmeters use an electric heating element and at least two temperature sensors to detect difference in temperature related to convection.  

The calibration of a thermal flowmeter depends on the fluid's thermal conductivity, as well as its specific heat (the amount of heat energy it absorbs per mass unit per temperature rise), similar to how an orifice plate's calibration depends on the fluid's density.  If either or both of these factors are variable in the process flow stream, a thermal mass flowmeter will give erratic indications, just as an orifice plate will give erratic readings of flow if the process fluid's density changes randomly.

\vskip 10pt

An example of a substance with a very high specific heat value is {\it hydrogen} gas.  If a thermal mass flowmeter is calibrated to accurately read the mass flow rate of air, for example, and then it is subjected to a stream of hydrogen gas, it will falsely register an excessive flow rate for the hydrogen due to that gas's extremely large specific heat value.

%(END_ANSWER)





%(BEGIN_NOTES)

%INDEX% Measurement, flow: thermal (mass)

%(END_NOTES)


