
%(BEGIN_QUESTION)
% Copyright 2006, Tony R. Kuphaldt, released under the Creative Commons Attribution License (v 1.0)
% This means you may do almost anything with this work of mine, so long as you give me proper credit

%Coriolis-effect mass flowmeters have several advantages over other mass-flow technologies, which make them worth their high price in some applications.  Identify what some of these advantages are.  Also, identify some of their outstanding disadvantages (besides relatively high cost).

Coriolis-effekt flowmeter har mange egenskaper som gjør at de brukes mye til tross for den høye prisen. List opp noen av disse, list ogs{aa} opp noen ulemper med instrumentet.

\underbar{file i00539}
%(END_QUESTION)





%(BEGIN_ANSWER)

{\bf Advantages:}

\begin{itemize}
\item{$\bullet$} Very high accuracy
\item{$\bullet$} Immunity to upstream/downstream piping disturbances
\item{$\bullet$} Provides real measurement of mass flow, fluid density, and fluid temperature
\item{$\bullet$} Excellent rangeability
\item{$\bullet$} Immunity to changes in density -- this makes Coriolis flowmeters particularly well-suited for measuring non-Newtonian fluids
\item{$\bullet$} Bidirectional
\medskip

\vskip 10pt

{\bf Disadvantages:}

\begin{itemize}
\item{$\bullet$} Relatively low operating temperature limit ($<$ 800$^{o}$ F)
\item{$\bullet$} Difficulty measuring multi-phase flows (e.g. gas + liquid)
\item{$\bullet$} Prohibitively expensive for large pipe sizes
\item{$\bullet$} Cannot measure low-pressure gases very well (Coriolis forces too small)
\item{$\bullet$} May suffer errors from external vibrations
\medskip

\vskip 10pt

Mass flow measurement is obtained by measuring the phase shift of the tube's oscillation between the two ends.  

Density measurement is obtained by measuring the resonant frequency of the tubes.  The basic equation for a mass-and-spring mechanical system is as follows:

$$f_r = {1 \over 2 \pi} \sqrt{k \over m}$$

\noindent
Where,

$f_r$ = Resonant frequency

$k$ = Spring constant

$m$ = Mass 

\vskip 10pt

Given a known tube mass and a known tube volume, knowing the resonant frequency of the tubes makes it quite easy to calculate the mass of the fluid filling the tubes, and thus the fluid density.  

Temperature measurement comes from an RTD sensing fluid temperature as it enters the tube assembly.  

%(END_ANSWER)





%(BEGIN_NOTES)


%INDEX% Measurement, flow: Coriolis (mass)

%(END_NOTES)


