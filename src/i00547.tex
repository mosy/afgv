
%(BEGIN_QUESTION)
% Copyright 2011, Tony R. Kuphaldt, released under the Creative Commons Attribution License (v 1.0)
% This means you may do almost anything with this work of mine, so long as you give me proper credit

Read and outline the ``RF Link Budget'' subsection of the ``Radio Systems'' section of the ``Wireless Instrumentation'' chapter in your {\it Lessons In Industrial Instrumentation} textbook.  Note the page numbers where important illustrations, photographs, equations, tables, and other relevant details are found.  Prepare to thoughtfully discuss with your instructor and classmates the concepts and examples explored in this reading.

\underbar{file i00547}
%(END_QUESTION)





%(BEGIN_ANSWER)


%(END_ANSWER)





%(BEGIN_NOTES)

The amount of RF power received by a radio receiver must be significant enough that it does not become lost in ambient noise.  We must ``budget'' all the gains and losses in a radio system (from transmitter to receiver, including all forms of power loss) in order to ensure good data integrity from end to end.

$$P_{rx} = P_{tx} + G_{total} + L_{total}$$

$$P_{tx} = N_f + N_{rx} + S - (G_{total} + L_{total})$$

\noindent
Where,

$P_{rx}$ = Signal power available at receiver input, in dBm

$P_{tx}$ = Transmitter output signal power, in dBm

$G_{total}$ = Sum of all gains (amplifiers, antenna directionality, etc.), a positive dB value

$L_{total}$ = Sum of all losses (cables, filters, path loss, fade, etc.), a negative dB value

$N_f$ = Noise floor, in dBm

$N_{rx}$ = Noise figure of receiver, in dBm

$S$ = Desired signal-to-noise ratio margin, in dB

\vskip 10pt

Radio manufacturers usually encapsulate the noise floor, noise figure, and signal-to-noise ratio into one figure they call {\it receiver sensitivity} (the minimum amount of signal power needed at the receiver for good data integrity).

$$P_{tx} = P_{rx} - (G_{total} + L_{total})$$

\noindent
Where,

$P_{tx}$ = Minimum transmitter output signal power, in dBm

$P_{rx}$ = Receiver sensitivity (minimum received signal power), in dBm

$G_{total}$ = Sum of all gains (amplifiers, antenna directionality, etc.), a positive dB value

$L_{total}$ = Sum of all losses (cables, filters, path loss, fade, etc.), a negative dB value

\vskip 10pt

Signal power received directly correlates to data errors (Bit Error Rate, or BER).  The more signal power received, the less often bits get corrupted.

\vskip 10pt

Path loss is the inevitable loss in signal power resulting from the inverse-square law as RF energy spreads out in space as it leaves the transmitting antenna.  Calculating path loss in clear, empty space:

$$L_p = -20 \log \left(4 \pi D \over \lambda\right)$$

\noindent
Where,

$L_p$ = Path loss, a negative dB value

$D$ = Distance between transmitting and receiving antennas

$\lambda$ = Wavelength of transmitted RF field, in same physical unit as $D$

\vskip 10pt

Fade is where RF waves deconstructively interfere to weaken received power.  RF energy reflected from conductive objects to arrive out-of-phase at the receiver will act to weaken the received signal.  Fade is difficult to quantitatively predict, and so we often simply build a 20 dB to 30 dB ``fade margin'' into our RF link budgets to account for its effects.

\vskip 10pt

Link budget calculations are estimates only.  A site test is usually recommended to determine the exact amount of transmitter power necessary for the job.










\vskip 10pt

\filbreak

\vskip 20pt \vbox{\hrule \hbox{\strut \vrule{} {\bf Suggestions for Socratic discussion} \vrule} \hrule}

\begin{itemize}
\item{} Explain what {\it receiver sensitivity} is, and how this is incorporated into link budget calculations.
\item{} Explain the mechanism of {\it path loss} for an electromagnetic wave.
\item{} Explain the mechanism of {\it fade loss} for an electromagnetic wave.
\item{} Does path loss increase, decrease, or remain the same as distance between antennas increases?
\item{} Does path loss increase, decrease, or remain the same as distance between antennas decreases?
\item{} Does path loss increase, decrease, or remain the same as signal frequency increases?
\item{} Does path loss increase, decrease, or remain the same as signal frequency decreases?
\end{itemize}


%INDEX% Reading assignment: Lessons In Industrial Instrumentation, Wireless instrumentation (RF link budget)

%(END_NOTES)


