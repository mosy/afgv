
%(BEGIN_QUESTION)
% Copyright 2006, Tony R. Kuphaldt, released under the Creative Commons Attribution License (v 1.0)
% This means you may do almost anything with this work of mine, so long as you give me proper credit

Classify the following substances as {\it particle}, {\it atom}, {\it element}, {\it molecule}, {\it compound}, and/or {\it mixture}.  Note that some of these substances might be properly described by more than one classification:

\begin{itemize}
\item{} One gram of perfectly pure water
\item{} A proton
\item{} Oxygen
\item{} Carbon dioxide (CO$_{2}$)
\item{} A quark
\item{} Gasoline
\item{} Air
\item{} Plutonium
\end{itemize} 

\vskip 20pt \vbox{\hrule \hbox{\strut \vrule{} {\bf Suggestions for Socratic discussion} \vrule} \hrule}

\begin{itemize}
\item{} Why do we care about these labels?  What practical importance is there, for example, in knowing whether carbon dioxide is an element or a compound?
\end{itemize}

\underbar{file i00551}
%(END_QUESTION)





%(BEGIN_ANSWER)

Note: the ``Terms and Definitions'' section of the ``Chemistry'' chapter in the {\it Lessons In Industrial Instrumentation} textbook is a good resource for helping to answer these questions.

\vskip 10pt

A {\bf particle} is a part of an atom, separable from the other portions only by levels of energy far in excess of chemical reactions.

\vskip 10pt

An {\bf atom} is the smallest unit of matter that may be isolated by chemical means. 

\vskip 10pt

An {\bf element} is a substance composed of atoms all sharing the same number of protons in their nuclei (e.g. hydrogen, helium, nitrogen, iron, cesium, fluorine).

\vskip 10pt

A {\bf molecule} is the smallest unit of matter composed of two or more atoms joined by electron interaction in a fixed ratio (e.g. water: H$_{2}$O) -- the smallest unit of a {\it compound}.

\vskip 10pt

A {\bf compound} is a substance composed of identical molecules (e.g. pure water).

\vskip 10pt

A {\bf mixture} is a substance composed of different atoms or molecules {\it not} electronically bonded to each other.

\vskip 10pt

\begin{itemize}
\item{} One gram of perfectly pure water: {\bf Compound}
\item{} A proton: {\bf Particle}
\item{} Oxygen: {\bf Atom}, {\bf Molecule}, or {\bf Element}, depending on context.
\item{} Carbon dioxide (CO$_{2}$): {\bf Molecule} or {\bf Compound}, depending on context.
\item{} A quark: {\bf Particle}
\item{} Gasoline: {\bf Mixture}
\item{} Air: {\bf Mixture}
\item{} Plutonium: {\bf Atom} or {\bf Element}, depending on context.
\end{itemize} 

\vskip 10pt

``Oxygen'' may refer to a single atom, in which case it would be classified as an atom.  However, oxygen is usually found in molecular form (pairs of oxygen atoms bound together: O$_{2}$), in which case a single atomic pair would be classified as a molecule.  If the reference to ``Oxygen'' is the {\it type} of substance, it would be classified as an element.

\vskip 10pt

Similar reasoning holds with Carbon dioxide and Plutonium.  A single atom of Plutonium, of course, would be an atom.  A single grouping of one Carbon atom (C) and two Oxygen atoms (O$_{2}$) would constitute one CO$_{2}$ molecule.  If the reference to either Plutonium or Carbon dioxide were in regard to type, they would be classified as ``element'' and ``compound,'' respectively.

\vskip 10pt

Gasoline and Air are both mixtures, each substance composed of multiple types of molecules.

%(END_ANSWER)





%(BEGIN_NOTES)


%INDEX% Chemistry, basic principles: atom, particle, element, molecule, compound, ion, mixture

%(END_NOTES)


