
%(BEGIN_QUESTION)
% Copyright 2006, Tony R. Kuphaldt, released under the Creative Commons Attribution License (v 1.0)
% This means you may do almost anything with this work of mine, so long as you give me proper credit

The three ``elementary'' constituent particles of atoms are called {\it protons}, {\it neutrons}, and {\it electrons}.  Classify these three particle types according to their locations within an atom, their electrical charge, and their relative masses.

Furthermore, describe the effects that the number of protons, neutrons, and electrons have on the characteristics of an atom.

\underbar{file i00552}
%(END_QUESTION)





%(BEGIN_ANSWER)

{\it Protons} and {\it neutrons} reside in the center (or {\it nucleus}) of an atom, a very dense core.  {\it Electrons} reside in patterns surrounding the nucleus (many textbooks depict these patterns as orbits, like satellites orbiting a planet, but this model is not entirely accurate).

Protons and electrons are both electrically charged particles.  Each proton has a positive electrical charge of 1 unit, while each electron has a negative electrical charge of 1 unit.  Neutrons are electrically neutral, possessing no charge at all.

Shown here is the ranking of elementary particles according to mass, in order from lightest to heaviest:

\medskip 
\item{} Electron (lightest)
\item{} Proton
\item{} Neutron (heaviest)
\end{itemize} 

\vskip 10pt

The number of protons in the nucleus of an atom determines that atom's {\it atomic number}, which determines its identity (what type of element it is).

The number of neutrons in the nucleus, added to the atomic number (the number of protons in the nucleus) constitutes the atom's {\it atomic weight}, also called {\it atomic mass}.  This determines the gross mass of the atom, and may also determine some nuclear properties such as radioactivity.

The number of electrons surrounding the nucleus of an atom is ideally equal to the number of protons in the nucleus (atomic number), to balance the opposing electric charges.  Extra electrons may surround an atom, though, giving it an overall negative electric charge.  Electrons may be ``robbed'' from an atom as well, giving it an overall positive electric charge.  The number and arrangement of electrons in an atom determine its chemical properties (reactivity with other atoms).

Since the base number of electrons is established by the number of protons, it is the atomic number that fundamentally determines an element's physical properties (reactivity, boiling point, freezing point, etc.).  Adding or subtracting electrons to/from an atom also affects certain chemical properties, but not to the extent that the atomic number does.

%(END_ANSWER)





%(BEGIN_NOTES)


%INDEX% Chemistry, basic principles: proton, neutron, electron

%(END_NOTES)


