
%(BEGIN_QUESTION)
% Copyright 2006, Tony R. Kuphaldt, released under the Creative Commons Attribution License (v 1.0)
% This means you may do almost anything with this work of mine, so long as you give me proper credit

Hundreds of years ago, people known by the title of {\it alchemists} attempted to convert one type of element into another, most notably {\it lead} (Pb) into {\it gold} (Au).  Their efforts failed miserably.  Explain why!

\vskip 10pt

What, specifically, {\it would} be necessary to convert an atom of lead into an atom of gold?

\underbar{file i00554}
%(END_QUESTION)





%(BEGIN_ANSWER)

Physical forces, energy sources, and reactions common to everyday experience are insufficient to alter the {\it atomic number} (number of protons in the nucleus) of any atom, because of the extremely high binding force between nuclear particles.  Thus, the chemical identity of elements remains extremely stable.

Chemical reactions are merely exchanges and interactions involving {\it electrons} between atoms, not protons or neutrons.  Electrons are rather loosely bound around atoms compared to protons and neutrons, which explains why chemical changes may occur from simple physical processes such as heating and cooling, or exposure to light.

\vskip 10pt

Three protons would have to be removed from an atom of lead ($_{82}$Pb) into gold ($_{79}$Au), and this would require the use of equipment such as particle accelerators.  Although this is possible, it is not practical due to the extremely high cost (it takes more money to convert lead into gold than the gold is worth!).

%(END_ANSWER)





%(BEGIN_NOTES)


%INDEX% Chemistry, basic principles: alchemy

%(END_NOTES)


