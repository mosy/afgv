
%(BEGIN_QUESTION)
% Copyright 2006, Tony R. Kuphaldt, released under the Creative Commons Attribution License (v 1.0)
% This means you may do almost anything with this work of mine, so long as you give me proper credit

In 1808, an English schoolteacher named John Dalton published several postulates regarding substances, representing the state-of-the-art of atomic theory at that time:

\begin{itemize}
\item{} An element is composed of extremely small indivisible particles called atoms.
\item{} All atoms of a given element have identical properties, which differ from those of other elements.
\item{} Atoms cannot be created, destroyed, or transformed into atoms of another element.
\item{} Compounds are formed when atoms of different elements combine with each other in small whole-number ratios.
\item{} The relative numbers and kinds of atoms are constant in a given compound.
\end{itemize} 

Determine whether or not these postulates are still in agreement with modern atomic theory, and state the nature of any deviations you discern.

\underbar{file i00556}
%(END_QUESTION)





%(BEGIN_ANSWER)

{\bf An element is composed of extremely small indivisible particles called atoms.}  This is in agreement with modern theory, except that we now know atoms {\it can} be divided into smaller subcomponents (elementary particles), all of which are further divisible into yet smaller particles (quarks, etc.).

\vskip 10pt

{\bf All atoms of a given element have identical properties, which differ from those of other elements.}  Isotopes have {\it slightly} different within the same element definition.  For some elements such as hydrogen (H), the isotopic mass ratios are substantial (Tritium:Deuterium:Hydrogen mass ratios = 3:2:1).  For most isotopes, though, the difference in mass is slight, and requires precision equipment to differentiate.  Certainly in Dalton's time there was no equipment capable of detecting much less measuring these differences in isotope mass.

\vskip 10pt

{\bf Atoms cannot be created, destroyed, or transformed into atoms of another element.}  Modern scientists would agree that {\it matter} cannot be created or destroyed, but may be converted between different forms (including energy!).  Atoms most certainly can be {\it transmutated} into different elements.  In nature, though, this happens on such a small scale as to be virtually unmeasurable, and it may only be synthesized on a large scale by apparatus unavailable to scientists in Dalton's time.

\vskip 10pt

{\bf Compounds are formed when atoms of different elements combine with each other in small whole-number ratios.}  This is in perfect agreement with modern atomic theory, and in fact is an excellent definition of what a ``compound'' is.

\vskip 10pt

{\bf The relative numbers and kinds of atoms are constant in a given compound.}  Again, this is in perfect agreement with modern theory.

%(END_ANSWER)





%(BEGIN_NOTES)


%INDEX% Chemistry, basic principles: Dalton's Laws

%(END_NOTES)


