
%(BEGIN_QUESTION)
% Copyright 2006, Tony R. Kuphaldt, released under the Creative Commons Attribution License (v 1.0)
% This means you may do almost anything with this work of mine, so long as you give me proper credit

Define the following chemical terms: {\it solution}, {\it solute}, and {\it solvent}.

\underbar{file i00557}
%(END_QUESTION)





%(BEGIN_ANSWER)

A chemical {\it solution} is a stable and homogeneous mixture (at the molecular level) of two or more substances.  {\it Solute} refers to the minor substance(s) and {\it solvent} refers to the major substance in which the solute(s) is/are dissolved.

\vskip 10pt

In many cases, the identity of solute and solvent is arbitrary, as in the case of a 50/50 alcohol/water solution.  Often, the solute is normally a gas or a solid, and the solvent is a liquid into which the gas(es) and/or solid(s) dissolve.  However, other combinations are possible.

\vskip 10pt

An example where both the solute(s) and solvent are gaseous is air.  Here, nitrogen would be classified as the solvent, since it is the predominant substance.  Oxygen, carbon dioxide, helium, and the rest of the ``lesser'' gases would be considered solutes.

\vskip 10pt

Examples where the solute is a gas and the solvent is a liquid include ammonia in water, acetylene in acetone, and nitrogen in blood.

\vskip 10pt

An example where the solute is a gas and the solvent is a solid is hydrogen dissolved in palladium metal.

\vskip 10pt

Examples where both the solute(s) and the solvent are liquids include alcohol in water and many metal alloys in their molten states.

\vskip 10pt

A class of solutions where the solute is a liquid and the solvent is a solid are called {\it amalgams}.  One kind of amalgam is mercury (solute) dissolved in silver (solvent), used in dental fillings.

\vskip 10pt

Examples where the solute is a solid and the solvent is a liquid include sugar in water and potassium chloride (KCl) in water.

\vskip 10pt

Examples where both the solute(s) and solvent are solid are brass and bronze in their solid states.  In either of these specific alloys, copper is the solvent (being the predominant substance) and either zinc or tin are the solutes, respectively.

%(END_ANSWER)





%(BEGIN_NOTES)

%INDEX% Chemistry, basic principles: solution, solute, and solvent

%(END_NOTES)


