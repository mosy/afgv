
%(BEGIN_QUESTION)
% Copyright 2006, Tony R. Kuphaldt, released under the Creative Commons Attribution License (v 1.0)
% This means you may do almost anything with this work of mine, so long as you give me proper credit

When ``table'' sugar is added to water, at first the sugar crystals seem to disappear as they dissolve.  But, if enough sugar is added to the water, eventually sugar crystals will begin to form at the bottom of the container.  In the context of this example, define the terms {\it solute}, {\it solvent}, {\it solution}, {\it saturation}, {\it supersaturation}, {\it precipitate}, and {\it supernatant}.

\underbar{file i00558}
%(END_QUESTION)





%(BEGIN_ANSWER)

\begin{itemize}
\item{} Solute: table sugar
\item{} Solvent: water
\item{} Solution: the liquid sugar/water mixture
\item{} Saturation: when the solution can accommodate no more sugar
\item{} Supersaturation: a condition beyond saturation, where any additional solute (sugar) forces a larger quantity of solute to immediately fall out of the solution
\item{} Precipitate: the sugar left in solid form on the bottom of the container
\item{} Supernatant: the saturated solution above the precipitate
\end{itemize} 

In this example, the sugar is the {\it solute}, the water the {\it solvent}, and the sugar/water mixture the {\it solution}.  When enough solute has been added to solvent that the solvent can dissolve no more, the solution is said to be {\it saturated}.

If additional solute is added to a saturated solution, the solution may become {\it supersaturated}.  This is a ``metastable'' state that may exist for substantial periods of time only if the solution is undisturbed.  

%(END_ANSWER)





%(BEGIN_NOTES)


%INDEX% Chemistry, basic principles: solution, solute, and solvent
%INDEX% Chemistry, basic principles: saturation, supersaturation, precipitate, and supernatant

%(END_NOTES)


