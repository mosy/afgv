
%(BEGIN_QUESTION)
% Copyright 2006, Tony R. Kuphaldt, released under the Creative Commons Attribution License (v 1.0)
% This means you may do almost anything with this work of mine, so long as you give me proper credit

Contrast the following chemical terms: {\it solution}, {\it suspension}, and {\it colloid}.  Furthermore, determine whether each of these mixtures is a solution, a suspension, or a colloid:

\begin{itemize}
\item{} Muddy water
\item{} Household ammonia
\item{} Milk
\item{} Rubbing alcohol
\item{} Tobacco smoke in air
\item{} Battery acid
\item{} Whipped cream
\item{} Dusty air
\item{} Sugar water
\item{} Coffee
\end{itemize} 

Colloids are often subdivided into the following types: {\it aerosols}, {\it foams}, {\it emulsions}, and {\it sols}.  Explain what each of these different colloids are, and classify all colloids in the list above as one of these sub-categories.

\underbar{file i00559}
%(END_QUESTION)





%(BEGIN_ANSWER)

A {\it solution} is a stable and molecularly homogeneous mixture of two or more substances.  A {\it suspension} is a heterogeneous mixture where separation of the components occurs over time.  A {\it colloid} is a heterogeneous mixture where separation is negligible over time.

\vskip 10pt

An {\it aerosol} is either a solid or a liquid dispersed in a gaseous medium.  {\it Foams} are gaseous substances dispersed in either liquid or solid media.  {\it Emulsions} are liquids dispersed in either liquid or solid media.  Finally, {\it sols} are solids dispersed in either liquid or solid media.

\vskip 10pt

Colloidal mixtures may be viewed as in-between suspensions and true solutions.  One way to distinguish a liquid colloidal mixture (often referred to as a {\it colloidal dispersion}) from a true solution is to use the {\it Tyndall effect} of light scattering.  When light passes though a clear solution, there will be no scattering, because solutions are homogeneous on a molecular level.  With either a suspension or a colloid, though, the particles are large enough to scatter light, and will become visibly apparent when a beam of light is passed through the mixture.  Once the particle size has been determined to be more than that in a true solution, the distinction between colloid and suspension may be made by detection of settling.

Examples of colloids are smoke in air and calcium in water.  Steel, being a mixture of iron, carbon, and other elements, is technically considered a solid colloid (``sol'') rather than a solid solution, due to particle size.  In other words, the granularity of the carbon particles is not fine enough to be ``molecularly homogeneous.''

\vskip 10pt

There are two easy tests used to divide gas-based and liquid-based mixtures into solutions, suspensions, and colloids: {\it settling} and {\it light dispersal}.  If there is substantial settling over time, then the mixture is a suspension.  If there is negligible settling, it may be either be a solution or a colloid.  The latter distinction is made on the basis of light scattering.

\vskip 10pt

Muddy water is a {\it suspension} because the dirt particles will settle over time.

\vskip 10pt

Household ammonia is a {\it solution} because settling never occurs and it does not scatter light.

\vskip 10pt 

Milk is a {\it colloid} emulsion (liquid/liquid) because settling never occurs and it scatters light.

\vskip 10pt 

Rubbing alcohol is a {\it solution} because settling never occurs and it does not scatter light.

\vskip 10pt 

Tobacco smoke in air is a {\it colloid} aerosol (solid/gas) because settling never occurs and it scatters light.

\vskip 10pt 

Battery acid is a {\it solution} because settling never occurs and it does not scatter light.

\vskip 10pt 

Whipped cream is a {\it colloid} foam (gas/liquid) because settling never occurs and it scatters light.

\vskip 10pt 

Dusty air is a {\it suspension} because the dirt particles will settle over time.

\vskip 10pt 

Sugar water is a {\it solution} because settling never occurs and it does not scatter light.

\vskip 10pt 

Coffee is a {\it colloid} sol (solid/liquid) because settling never occurs and it scatters light.

%(END_ANSWER)





%(BEGIN_NOTES)


%INDEX% Chemistry, basic principles: suspension, solution, and colloid

%(END_NOTES)


