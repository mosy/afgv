
%(BEGIN_QUESTION)
% Copyright 2006, Tony R. Kuphaldt, released under the Creative Commons Attribution License (v 1.0)
% This means you may do almost anything with this work of mine, so long as you give me proper credit

Physicists and chemists alike often refer to atomic structures called electron {\it shells}.  What is an electron shell within an atom, and why is it an important concept in chemistry?

\underbar{file i00560}
%(END_QUESTION)





%(BEGIN_ANSWER)

When electrons cluster around the nucleus of an atom, they do not do so with equal status.  Rather, they must locate themselves in different energy ``levels'' known as {\it shells}, kind of like people sitting in different levels of seats in an amphitheater.  Since the outermost level of electrons (the outermost ``shell'') is the one most likely to interact with electrons from other atoms to form chemical bonds, these electrons predominantly determine the chemical nature of the atom.

\vskip 10pt

Many students find the analogy of an {\it amphitheater} particularly insightful.  Just as people desire to sit as close to the action in an amphitheater as they can (the lowest-level seating), electrons seek the lowest {\it energy} levels that they can.  

We can even extend the analogy further, where money paid for amphitheater seating is analogous to energy exchanged in electron shell filling: just as people must give up more money to sit closer to the action in an amphitheater, electrons must lose more energy to occupy a lower-level shell.  Conversely, just as you must pay money to get someone to move to a higher-level seat in an amphitheater, you must infuse more energy in an electron to get it to ``jump'' to a higher-level shell.

%(END_ANSWER)





%(BEGIN_NOTES)

Electron shells are responsible more so than any other factor for the periodicity seen in the periodic table.  Since each shell can only hold a finite number of electrons, the sequence of ``shell filling'' follows a quasi-repeating pattern as we examine elements from the lightest (hydrogen, atomic number = 1) to the heaviest (increasing atomic numbers).

%INDEX% Physics, atomic: electron shells
%INDEX% Chemistry, basic principles: electron shells

%(END_NOTES)


