
%(BEGIN_QUESTION)
% Copyright 2011, Tony R. Kuphaldt, released under the Creative Commons Attribution License (v 1.0)
% This means you may do almost anything with this work of mine, so long as you give me proper credit

Suppose a radio transmitter unit outputs 25 watts of RF power at a frequency of 900 MHz, into a half-wave dipole antenna.  The radio energy radiates outward from the transmitting antenna, and is received by another antenna to send a signal of 1.8 microwatts into a radio receiver.

Calculate the transmitter's output power ($P_{tx}$) in both dBm and dBW, and also the receiver's input signal power ($P_{rx}$) in dBm and dBW.  Then, calculate the total power loss between transmitter and receiver ($P_{loss}$), in dB.

\vskip 10pt

$P_{tx}$ = \underbar{\hskip 50pt} dBm = \underbar{\hskip 50pt} dBW 

\vskip 10pt

$P_{rx}$ = \underbar{\hskip 50pt} dBm = \underbar{\hskip 50pt} dBW 

\vskip 10pt

$P_{loss}$ = \underbar{\hskip 50pt} dB

\vskip 20pt \vbox{\hrule \hbox{\strut \vrule{} {\bf Suggestions for Socratic discussion} \vrule} \hrule}

\begin{itemize}
\item{} Explain why the ``decibel'' is such a commonly used unit for power, and also for power gains and losses, in RF communications work.
\item{} Does it matter if $P_{loss}$ is calculated from the transmit and receive powers in units of watts, dBm, or dBW?  Why or why not?
\item{} If we wished to reduce the power loss from transmitter and receiver, how could we do so?  Note: there are many different solutions to this problem!
\item{} If the transmitting and receiving antennas were spaced farther apart, would $P_{loss}$ increase, decrease, or remain the same?
\item{} If the transmitting and receiving antennas were both located closer to ground level (i.e. not as high off the ground), would $P_{loss}$ increase, decrease, or remain the same?
\item{} Estimate these decibel values by first rounding the power ratio to the nearest product of 10 and/or 2, then applying the equivalence of 10 dB to a 10-fold ratio and 3 dB to a 2-fold ratio.
\end{itemize}

\underbar{file i00561}
%(END_QUESTION)





%(BEGIN_ANSWER)

\noindent
{\bf Partial answer:}

\vskip 10pt

$P_{tx}$ = \underbar{13.98} dBW 

\vskip 10pt

$P_{rx}$ = \underbar{$-57.45$} dBW 

\vskip 10pt

$P_{loss}$ = \underbar{$-71.43$} dB


%(END_ANSWER)





%(BEGIN_NOTES)

$P_{tx}$ = $10 \log \left( 25 \hbox { W} \over 1 \hbox{ mW} \right)$ = \underbar{43.98} dBm  

\vskip 10pt

$P_{tx}$ = $10 \log \left( 25 \hbox { W} \over 1 \hbox{ W} \right)$ = \underbar{13.98} dBW 

\vskip 10pt

$P_{rx}$ = $10 \log \left( 1.8 \mu \hbox {W} \over 1 \hbox{ mW} \right)$ = \underbar{-27.45} dBm 

\vskip 10pt

$P_{rx}$ = $10 \log \left( 1.8 \mu \hbox {W} \over 1 \hbox{ W} \right)$ = \underbar{-57.45} dBW 

\vskip 10pt

$P_{loss}$ = $10 \log \left( 1.8 \mu \hbox {W} \over 25 \hbox{ W} \right)$ = \underbar{-71.43} dB


%INDEX% Electronics review: decibel power calculations

%(END_NOTES)

