
%(BEGIN_QUESTION)
% Copyright 2006, Tony R. Kuphaldt, released under the Creative Commons Attribution License (v 1.0)
% This means you may do almost anything with this work of mine, so long as you give me proper credit

An interesting way to denote the spectroscopic electron shell structure of elements is shown here:

% No blank lines allowed between lines of an \halign structure!
% I use comments (%) instead, so that TeX doesn't choke.

$$\vbox{\offinterlineskip
\halign{\strut
\vrule \quad\hfil # \ \hfil & 
\vrule \quad\hfil # \ \hfil \vrule \cr
\noalign{\hrule}
%
% First row
Element & Electron configuration \cr
%
\noalign{\hrule}
%
% Another row
Hydrogen & 1s$^{1}$ \cr
%
\noalign{\hrule}
%
% Another row
Helium & 1s$^{2}$ \cr
%
\noalign{\hrule}
%
% Another row
Lithium & [He]2s$^{1}$ \cr
%
\noalign{\hrule}
%
% Another row
Beryllium & [He]2s$^{2}$ \cr
%
\noalign{\hrule}
%
% Another row
Boron & [He]2s$^{2}$2p$^{1}$ \cr
%
\noalign{\hrule}
%
% Another row
Carbon & [He]2s$^{2}$2p$^{2}$ \cr
%
\noalign{\hrule}
%
% Another row
Nitrogen & [He]2s$^{2}$2p$^{3}$ \cr
%
\noalign{\hrule}
%
% Another row
Oxygen & [He]2s$^{2}$2p$^{4}$ \cr
%
\noalign{\hrule}
%
% Another row
Fluorine & [He]2s$^{2}$2p$^{5}$ \cr
%
\noalign{\hrule}
%
% Another row
Neon & [He]2s$^{2}$2p$^{6}$ \cr
%
\noalign{\hrule}
%
% Another row
Sodium & [Ne]3s$^{1}$ \cr
%
\noalign{\hrule}
%
% Another row
Magnesium & [Ne]3s$^{2}$ \cr
%
\noalign{\hrule}
%
% Another row
Aluminum & [Ne]3s$^{2}$3p$^{1}$ \cr
%
\noalign{\hrule}
%
% Another row
Silicon & [Ne]3s$^{2}$3p$^{2}$ \cr
%
\noalign{\hrule}
%
% Another row
Phosphorus & [Ne]3s$^{2}$3p$^{3}$ \cr
%
\noalign{\hrule}
%
% Another row
Sulfur & [Ne]3s$^{2}$3p$^{4}$ \cr
%
\noalign{\hrule}
%
% Another row
Chlorine & [Ne]3s$^{2}$3p$^{5}$ \cr
%
\noalign{\hrule}
%
% Another row
Argon & [Ne]3s$^{2}$3p$^{6}$ \cr
%
\noalign{\hrule}
%
% Another row
Potassium & [Ar]4s$^{1}$ \cr
%
\noalign{\hrule}
} % End of \halign 
}$$ % End of \vbox

Explain how this notation condenses what is otherwise a very lengthy and unwieldy format (showing {\it all} shells in each atom), without omitting any crucial information.

\vskip 20pt \vbox{\hrule \hbox{\strut \vrule{} {\bf Suggestions for Socratic discussion} \vrule} \hrule}

\begin{itemize}
\item{} The bracketed element name is sometimes referred to as an ``electron core'' in spectroscopic notation.  Explain why this label makes sense.
\item{} Examine this table of electron notations, and identify the ``periods'' (intervals) comprising each row in a periodic table.
\item{} Identify elements in this table which are the easiest to positively ionize (i.e. remove one electron from).
\item{} Identify elements in this table which are the easiest to negatively ionize (i.e. add one electron to).
\end{itemize}

\underbar{file i00562}
%(END_QUESTION)





%(BEGIN_ANSWER)


%(END_ANSWER)





%(BEGIN_NOTES)

This (shortened) spectroscopic notation is simpler than the full-length form because the repetitive portions are ``condensed'' into the next lowest elemental symbol with completely filled electron shells.  For example, the electron configuration of Neon (1s$^{2}$2s$^{2}$2p$^{6}$) is simply repeated in every element with an atomic number in excess of 10, so instead of showing the full ordering of electrons for all those higher elements, the 1 and 2 shell fillings are represented by the symbol for Neon [Ne] instead.

\vskip 10pt

A chemist might say that the electron configuration for the element sodium, for example, is a single ``s'' subshell electron in the 3rd shell (3s$^{1}$), over a neon {\it core}.

The {\it really} short notation found in most periodic tables simply lists the outermost shell of the atom, with an implied ``core'' configuration of lower-level filled shells that is not shown.

%INDEX% Physics, atomic: electron shells (spectroscopic notation)
%INDEX% Chemistry, basic principles: electron shells (spectroscopic notation)

%(END_NOTES)


