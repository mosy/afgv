
%(BEGIN_QUESTION)
% Copyright 2006, Tony R. Kuphaldt, released under the Creative Commons Attribution License (v 1.0)
% This means you may do almost anything with this work of mine, so long as you give me proper credit

What do you notice about the electron configurations for the elements in each of the vertical columns of the periodic table?  What does this tell you about the nature of those elements? 

\underbar{file i00564}
%(END_QUESTION)





%(BEGIN_ANSWER)

In each vertical column of the periodic table (each column called a {\it group}), the number of electrons in the outermost (unfilled) shells is the same.  This indicates that elements sharing a common table column have similar chemical properties.

At first glance, it may appear as though the outermost shells for elements in each group of the periodic table do {\it not} share the same number of electrons, but one must realize the equality of electron numbers applies to the total number of electrons in {\it all unfilled shells} and not just the last shell.  Tungsten and molybdenum, for example, with the respective electron configurations of 5d$^{4}$6s$^{2}$ and 4d$^{5}$5s$^{1}$, both have a total of 6 electrons in their two (unfilled) shells.

%(END_ANSWER)





%(BEGIN_NOTES)


%INDEX% Physics, atomic: valence number
%INDEX% Chemistry, basic principles: valence number

%(END_NOTES)


