
%(BEGIN_QUESTION)
% Copyright 2006, Tony R. Kuphaldt, released under the Creative Commons Attribution License (v 1.0)
% This means you may do almost anything with this work of mine, so long as you give me proper credit

Calculate the mass (in grams) for each of these substance quantities:

\begin{itemize}
\item{} 1 mole of pure $^{12}$C
\vskip 5pt
\item{} 1 mole of carbon (naturally occurring)
\vskip 5pt
\item{} 1 mole of pure $^{56}$Fe
\vskip 5pt
\item{} 5.5 moles of mercury (naturally occurring)
\vskip 5pt
\item{} 0.002 moles of helium (naturally occurring)
\end{itemize} 

Hint: you will find the Periodic Table of the Elements extremely helpful here!

\underbar{file i00566}
%(END_QUESTION)





%(BEGIN_ANSWER)

\begin{itemize} 
\item{} 1 mole of pure $^{12}$C = 12 g
\vskip 5pt
\item{} 1 mole of carbon (naturally occurring) = 12.011 g
\vskip 5pt
\item{} 1 mole of pure $^{56}$Fe = 56 g
\vskip 5pt
\item{} 5.5 moles of mercury (naturally occurring) = 1.1032 kg
\vskip 5pt
\item{} 0.002 moles of helium (naturally occurring) = 8.0052 mg
\end{itemize} 

\vskip 10pt

To calculate each of these masses, simply multiply the number of moles by the atomic mass given or found in the periodic table.

%(END_ANSWER)





%(BEGIN_NOTES)


%INDEX% Chemistry, stoichiometry: moles

%(END_NOTES)


