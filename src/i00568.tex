
%(BEGIN_QUESTION)
% Copyright 2006, Tony R. Kuphaldt, released under the Creative Commons Attribution License (v 1.0)
% This means you may do almost anything with this work of mine, so long as you give me proper credit

A sample of ``table'' salt has a mass of 1 kg.  How many moles of salt is this equal to?

\vskip 10pt

Now, suppose this exact quantity of salt were completely dissolved in water, resulting in a saltwater solution of 850 liters' volume.  Calculate the {\it molarity} of this saltwater solution.

\vskip 20pt \vbox{\hrule \hbox{\strut \vrule{} {\bf Suggestions for Socratic discussion} \vrule} \hrule}

\begin{itemize}
\item{} Identify a particular analyzer technology that might readily estimate the molarity of a salt-water solution, and identify different versions of this technology.
\end{itemize}

\underbar{file i00568}
%(END_QUESTION)





%(BEGIN_ANSWER)

``Table'' salt is sodium chloride (NaCl), with 1 atom of sodium bound to 1 atom of chlorine.  Together, the number of atomic mass units (amu) for each sodium chloride molecule is the sum of the individual atoms' atomic masses:

\vskip 10pt

\medskip 
\item{} {\it Each molecule of NaCl contains:}
\item{} 1 atom of Na at 22.99 amu each
\item{} 1 atom of Cl at 35.45 amu each
\end{itemize} 

[(1 atom)(22.99 amu/atom) + (1 atom)(35.45 amu/atom)] = 58.44 g per mole of NaCl

\vskip 10pt

Since we now know the number of grams per mole for NaCl, we may calculate the number of moles needed to make 1 kg (1000 g) of salt:

\vskip 10pt

$$\left( 1000 \hbox{ g} \over 1 \right) \left(1 \hbox{ mol NaCl} \over 58.44 \hbox{ g} \right) = 17.11 \hbox{ mol NaCl}$$

\vskip 10pt

Therefore, 1 kg of ``table'' salt is equal to 17.11 moles.

\vskip 10pt

{\it Molarity}, or {\it molar concentration}, is the number of moles of solute per liter of solution.  Given 17.11 moles of solute and 850 liters of solution, the molarity will be equal to:

$${{17.11 \hbox{ mol NaCl}} \over {850 \hbox{ L}}} = 0.0201 M$$

Molarity for a substance is usually represented by placing the chemical symbol for that substance inside square brackets.  Molarity as a unit of measurement is expressed as an italicized $M$.  Therefore, in this question, [NaCl] = 0.0201 $M$.

%(END_ANSWER)





%(BEGIN_NOTES)


%INDEX% Chemistry, stoichiometry: moles
%INDEX% Chemistry: molarity

%(END_NOTES)


