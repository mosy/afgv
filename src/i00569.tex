
%(BEGIN_QUESTION)
% Copyright 2006, Tony R. Kuphaldt, released under the Creative Commons Attribution License (v 1.0)
% This means you may do almost anything with this work of mine, so long as you give me proper credit

When hydrogen and oxygen combine under sufficient pressure and/or temperature, they react to form water (H$_{2}$O) in a simple example of {\it combustion}.  If these are the only elements involved, and the samples are pure, water is the only byproduct of the reaction (in addition to heat energy, of course).

Hydrogen, as a gas at room temperature and atmospheric pressure, exists as pairs of atoms (a {\it diatomic} molecule) denoted as H$_{2}$.  Likewise, oxygen as a gas at room temperature and atmospheric pressure exists as pairs of atoms (O$_{2}$) as well.

If I have a quantity of pure hydrogen gas equal to 250 moles, how many moles of pure oxygen gas will be needed to completely react with (burn) the hydrogen gas sample?  Assume a gas pressure of 830 mm HgA.

\vskip 20pt \vbox{\hrule \hbox{\strut \vrule{} {\bf Suggestions for Socratic discussion} \vrule} \hrule}

\begin{itemize}
\item{} Should the $\Delta H$ value for the combustion reaction have a positive or a negative value?
\item{} The main engines on the Space Shuttle use hydrogen as the fuel and oxygen as the oxidizer.  Thus, the only emission in the exhaust of these engines is water vapor.  Does that mean these engines are completely non-polluting?
\end{itemize}

\underbar{file i00569}
%(END_QUESTION)





%(BEGIN_ANSWER)

125 moles of oxygen gas will be required to completely burn 250 moles of hydrogen gas, because there are an equal number of hydrogen and oxygen atoms in one mole of each, and only one atom of oxygen is needed for every 2 atoms of hydrogen.

\vskip 10pt

The 830 mm HgA pressure is extraneous information, included for the purpose of challenging you to identify whether or not information is relevant to solving a particular problem.

\vskip 10pt

The question of whether or not these rocket engines are polluting requires some thinking beyond the chemistry of combustion in the engine.  While the engines themselves only release water vapor (along with some unreacted hydrogen and oxygen), the process by which hydrogen fuel is produced is indeed polluting.  Large quantities of pure hydrogen are typically produced using a {\it reforming} reaction whereby a plentiful hydrocarbon compound such as natural gas (CH$_{4}$) is reformed into hydrogen (H$_{2}$) and carbon dioxide (CO$_{2}$), the latter being a ``greenhouse'' gas pollutant.

%(END_ANSWER)





%(BEGIN_NOTES)


%INDEX% Chemistry, stoichiometry: reaction quantities

%(END_NOTES)


