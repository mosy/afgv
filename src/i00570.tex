
%(BEGIN_QUESTION)
% Copyright 2006, Tony R. Kuphaldt, released under the Creative Commons Attribution License (v 1.0)
% This means you may do almost anything with this work of mine, so long as you give me proper credit

When methane (CH$_{4}$) and oxygen (O$_{2}$) gas combine under sufficient pressure and/or temperature, they react to form carbon dioxide (CO$_{2}$) gas and water (H$_{2}$O).  The following chemical equation shows the combustion of methane and oxygen to form carbon dioxide and water:

$$\hbox{CH}_4 + \hbox{O}_2 \to \hbox{CO}_2 + \hbox{H}_2\hbox{O}$$

Unfortunately, this equation is incomplete.  Although it does indicate the {\it identities} of the reaction products (carbon dioxide and water), it does not indicate their {\it relative quantities}.  A simple head-count of atoms on each side of the equation confirms this: 

% No blank lines allowed between lines of an \halign structure!
% I use comments (%) instead, so that TeX doesn't choke.

$$\vbox{\offinterlineskip
\halign{\strut
\vrule \quad\hfil # \ \hfil & 
\vrule \quad\hfil # \ \hfil \vrule \cr
\noalign{\hrule}
%
% First row
{\bf Reactants} & {\bf Reaction products} \cr
%
\noalign{\hrule}
%
% Another row
Carbon = 1 & Carbon = 1 \cr
%
\noalign{\hrule}
%
% Another row
Hydrogen = 4 & Hydrogen = 2 \cr
%
\noalign{\hrule}
%
% Another row
Oxygen = 2 & Oxygen = 3 \cr
%
\noalign{\hrule}
} % End of \halign 
}$$ % End of \vbox

Since we know it would violate the Law of Mass Conservation (one of the fundamental laws of physics) to have fewer atoms of each element coming out of a reaction than going into it, we know this equation must be incomplete.

To make this a {\it balanced} equation, we must determine how many molecules of methane, oxygen, carbon dioxide, and water are involved in this reaction so that the numbers of atoms for each element are the same on both sides of the equation.  

\vskip 10pt

Re-write this chemical equation so that it is balanced.

\vskip 10pt

Furthermore, calculate the number of moles of oxygen gas needed to completely react (burn) 1400 moles of methane gas under ideal conditions.

\underbar{file i00570}
%(END_QUESTION)





%(BEGIN_ANSWER)

For each molecule of methane, there is a single atom of carbon (C) and four atoms of hydrogen (H$_{4}$).  In order to convert one atom of carbon into 1 molecule of carbon dioxide (CO$_{2}$), one molecule of oxygen (O$_{2}$) is required:

$$\hbox{C} + \hbox{O}_2 \to \hbox{CO}_2$$

In order to convert four atoms of hydrogen into water, two more atoms of oxygen are required:

$$\hbox{H}_4 + \hbox{O}_2 \to 2\hbox{H}_2\hbox{O}$$

Thus, two molecules of O$_{2}$ are needed to completely burn one molecule of CH$_{4}$.  This means, by extension, that two moles of oxygen gas will be required for every one mole of methane.  So, if our methane quantity is 1400 moles, our oxygen quantity must be 2800 moles.

%(END_ANSWER)





%(BEGIN_NOTES)


%INDEX% Chemistry, stoichiometry: reaction quantities

%(END_NOTES)


