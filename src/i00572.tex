
%(BEGIN_QUESTION)
% Copyright 2006, Tony R. Kuphaldt, released under the Creative Commons Attribution License (v 1.0)
% This means you may do almost anything with this work of mine, so long as you give me proper credit

When the sugar {\it glucose} (C$_{6}$H$_{12}$O$_{6}$) ferments, the result is the production of carbon dioxide (CO$_{2}$) and ethanol (C$_{2}$H$_{5}$OH).  This is the chemical reaction on which all alcohol beverage industries are founded: the conversion of sugar into ethanol.  The following chemical equation shows the conversion of glucose into carbon dioxide and ethanol:

$$\hbox{C}_6 \hbox{H}_{12} \hbox{O}_6 \to \hbox{CO}_2 + \hbox{C}_2 \hbox{H}_5 \hbox{OH}$$

Unfortunately, this equation is incomplete.  Although it does indicate the {\it identities} of the reaction products (carbon dioxide and ethanol), it does not indicate their {\it relative quantities}.  Re-write this chemical equation so that it is balanced.

\vskip 20pt \vbox{\hrule \hbox{\strut \vrule{} {\bf Suggestions for Socratic discussion} \vrule} \hrule}

\begin{itemize}
\item{} Fermentation is a process strongly influenced by pH.  Is this evident from the balanced equation, or not?
\item{} Is fermentation an endothermic or exothermic process?  How may we tell by this chemical equation?
\end{itemize}

\underbar{file i00572}
%(END_QUESTION)





%(BEGIN_ANSWER)


%(END_ANSWER)





%(BEGIN_NOTES)

Balancing this reaction using simultaneous linear equations:

% No blank lines allowed between lines of an \halign structure!
% I use comments (%) instead, so Tex doesn't choke.

$$\vbox{\offinterlineskip
\halign{\strut
\vrule \quad\hfil # \ \hfil & 
\vrule \quad\hfil # \ \hfil & 
\vrule \quad\hfil # \ \hfil & 
\vrule \quad\hfil # \ \hfil \vrule \cr
\noalign{\hrule}
%
% First row
1 & = & $x$ & $y$ \cr
%
\noalign{\hrule}
%
% Another row
C$_{6}$H$_{12}$O$_{6}$ & $\to$ & CO$_{2}$ & C$_{2}$H$_{5}$OH \cr
%
\noalign{\hrule}
} % End of \halign 
}$$ % End of \vbox

% No blank lines allowed between lines of an \halign structure!
% I use comments (%) instead, so Tex doesn't choke.

$$\vbox{\offinterlineskip
\halign{\strut
\vrule \quad\hfil # \ \hfil & 
\vrule \quad\hfil # \ \hfil \vrule \cr
\noalign{\hrule}
%
% First row
{\bf Element} & {\bf Balance equation} \cr
%
\noalign{\hrule}
%
% Another row
Hydrogen & $12 = 0x + 6y$ \cr
%
\noalign{\hrule}
%
% Another row
Oxygen & $6 = 2x + y$ \cr
%
\noalign{\hrule}
%
% Another row
Carbon & $6 = x + 2y$ \cr
%
\noalign{\hrule}
} % End of \halign 
}$$ % End of \vbox

From the balance equation for hydrogen we can tell that $y$ must be equal to 2.  Substituting the number 2 for $y$ in either of the other balance equations allows us to solve for $x$, which is also equal to 2.  Therefore, the balanced chemical equation for the fermentation of glucose is:

$$\hbox{C}_6 \hbox{H}_{12} \hbox{O}_6 \to 2 \hbox{CO}_2 + 2 \hbox{C}_2 \hbox{H}_5 \hbox{OH}$$

%INDEX% Chemistry, stoichiometry: balancing a chemical equation
%INDEX% Process: glucose fermentation

%(END_NOTES)


