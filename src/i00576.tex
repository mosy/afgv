
%(BEGIN_QUESTION)
% Copyright 2006, Tony R. Kuphaldt, released under the Creative Commons Attribution License (v 1.0)
% This means you may do almost anything with this work of mine, so long as you give me proper credit

What binds different atoms together to form molecules?  What particles or forces are involved to hold the constituent atoms within molecules together?

\underbar{file i00576}
%(END_QUESTION)





%(BEGIN_ANSWER)

Chemical bonds are always the result of interactions between the electrons of different atoms.

\vskip 10pt

Chemical bonds are classified into two major categories: {\it ionic} and {\it covalent}.  Ionic bonds are where the constituent atoms have the tendency to dissociate into oppositely-charged ions (for example, sodium and chlorine forming table salt: NaCl), and thus are held together by electrostatic attraction.  Covalent bonds are formed by the sharing of electron pairs between atoms.  Molecular hydrogen (H$_{2}$) is an example of a covalent bond.

The truth is, most chemical bonds cannot be classified simply as purely ionic or purely covalent.  Chemical bonds may be ionic to some degree and covalent to some degree (indeed, if we expand the definition of ``some degree'' to be inclusive of 0\% and 100\%, we may safely say that {\it all} chemical bonds are ionic to some degree and covalent to some degree). 

A good example of this is a {\it polar covalent bond}, where atoms bonded by the sharing of electron pairs do not share each electron in the pair equally.  In other words, a covalently-bonded molecule may possess a {\it dipole}, where one side of it is more negative and the other side of it is more positive.  Water is an example of a polar covalent bond, and its dipole character is what makes microwave cooking possible.  Microwaves force the polar water molecules to oscillate along with the oscillating electric field.  If water molecules were completely nonpolarized, they would not be affected by any external electric field, and thus they would not be moved by the oscillating electric field of a microwave beam.

%(END_ANSWER)





%(BEGIN_NOTES)


%INDEX% Chemistry, basic: ionic versus covalent bonds
%INDEX% Chemistry, basic: molecular bonds

%(END_NOTES)


