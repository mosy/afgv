
%(BEGIN_QUESTION)
% Copyright 2006, Tony R. Kuphaldt, released under the Creative Commons Attribution License (v 1.0)
% This means you may do almost anything with this work of mine, so long as you give me proper credit

A popular concept for automotive fuel is {\it hydrogen}.  One of the main attractions of hydrogen as a fuel is that it is clean: no pollutants are formed (ideally) when hydrogen is burned.  If ``burned'' in a fuel cell instead of by combustion in air, pure water is the only chemical byproduct of the reaction.

However, there are some serious public {\it misconceptions} about hydrogen as a fuel source.  One of the most common misconceptions is the availability of hydrogen fuel.  Since most people know that water is comprised of hydrogen and oxygen (H$_{2}$O), a common public belief is that we have unlimited reserves of hydrogen fuel stored in the oceans.  All we need to do, so goes the logic, is separate the hydrogen from the oxygen and then we'll have a limitless supply of fuel and all our energy problems will be solved.

Based on what you know of chemical reactions and energy exchange, explain why this belief is fundamentally flawed.

\underbar{file i00578}
%(END_QUESTION)





%(BEGIN_ANSWER)

Hydrogen, by itself, is indeed a fuel.  By ``fuel,'' I mean a substance that releases energy when oxidized (burned).  Hydrogen combining with oxygen to form water is a process of chemical bond-making, and like all bond-forging processes, it occurs with a corresponding release of energy.

However, this is the very bond we must {\it break} to release hydrogen from water.  And we know that bond-breaking requires an {\it input} of energy.  Not surprisingly, it takes exactly the same amount of energy (under ideal conditions) to break this bond as will be released by the re-forging of the same bond during combustion.  Practically speaking, it will take a bit more energy to break the bond than we will get back during combustion, so we don't even get as much back as we put in.  While hydrogen is indeed a clean-burning fuel, it ``costs'' slightly more energy to obtain from water than the energy we get out of it, making the whole process a net loss of energy, not a net gain.

Once you understand the exchange of energy in the ``hydrogen economy,'' you realize that hydrogen from the oceans is not a fuel at all, but rather a medium for energy exchange.  To electrolyze water, we need energy from some other source (solar, nuclear, etc.), then that energy is ``stored'' in the hydrogen gas we extract from the water, to be released later on when burned or consumed in a fuel cell.  Far from creating energy, we have merely transported energy.

This is not to suggest that the concept of a hydrogen economy is flawed, but merely to underscore the need for a scientifically accurate understanding of it: all hydrogen does is serve as a transport medium, transporting energy from whatever source is used to electrolyze water to the end-point of energy use (automobiles, space heating, industry, etc.).

%(END_ANSWER)





%(BEGIN_NOTES)


%INDEX% Chemistry, basic: molecular bonds and energy exchange

%(END_NOTES)


