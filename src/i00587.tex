
%(BEGIN_QUESTION)
% Copyright 2011, Tony R. Kuphaldt, released under the Creative Commons Attribution License (v 1.0)
% This means you may do almost anything with this work of mine, so long as you give me proper credit

Dilute chemical concentrations are often measured in the unit of {\it parts per million}, abbreviated {\it ppm}.  For extremely dilute solutions, the unit {\it parts per billion} (ppb) is used.  These are nothing more than ratios, much like {\it percentage}.  In fact, the unit of ``percent'' may be thought of as nothing more than ``parts per hundred'' although it is never conventionally expressed as such.

\vskip 10pt

In light of this definition for {\it ppm}, express the tolerance of a $\pm$5\% carbon-composition resistor in ppm instead of percent.

\vskip 10pt

Also calculate the following volumetric and mass concentrations in units of ppm:

\begin{itemize}
\item{} 3.6 milliliters of methyl alcohol mixed into 10.5 liters of pure water
\vskip 10pt
\item{} 55 cubic inches of natural gas released into a room of air 10 feet by 15 feet by 8 feet
\vskip 10pt
\item{} 10 grams of hydrofluoric acid added to 560 kg of water
\vskip 10pt
\item{} 140 grams of H$_{2}$S gas released into open air
\end{itemize}

\vskip 10pt

\underbar{file i00587}
%(END_QUESTION)





%(BEGIN_ANSWER)

\noindent
{\bf Partial answer:}

\vskip 10pt

$\pm$5\% (gold color code) is {\bf 50,000 ppm}.

\begin{itemize}
\item{} 3.6 milliliters of methyl alcohol mixed into 10.5 liters of pure water = {\bf 342.74 ppm}  {\it If your calculated answer was 342.86 ppm, you made a minor error: you took 10.5 liters to be the total volume of liquid after adding the alcohol.  10.5 liters is just the water's volume, not the total solution (mixed) volume!}
\vskip 10pt
\item{} 10 grams of hydrofluoric acid added to 560 kg of water = {\bf 17.86 ppm}
\vskip 10pt
\end{itemize}

%(END_ANSWER)





%(BEGIN_NOTES)

\begin{itemize}
\item{} 55 cubic inches of natural gas released into a room of air 10 feet by 15 feet by 8 feet = {\bf 26.523 ppm}
\vskip 10pt
\item{} 140 grams of H$_{2}$S gas released into open air = {\it not enough information given to solve!}
\end{itemize}

%INDEX% Chemistry, safety: parts per billion concentration
%INDEX% Chemistry, safety: parts per million concentration

%(END_NOTES)


