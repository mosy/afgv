
%(BEGIN_QUESTION)
% Copyright 2015, Tony R. Kuphaldt, released under the Creative Commons Attribution License (v 1.0)
% This means you may do almost anything with this work of mine, so long as you give me proper credit

Read and outline the ``Antennas'', ``Decibels'', ``Antenna Radiation Patterns'', ``Antenna Gain Calculations'', and ``Effective Radiated Power'' subsections of the ``Radio Systems'' section of the ``Wireless Instrumentation'' chapter in your {\it Lessons In Industrial Instrumentation} textbook.  Note the page numbers where important illustrations, photographs, equations, tables, and other relevant details are found.  Prepare to thoughtfully discuss with your instructor and classmates the concepts and examples explored in this reading.

\vskip 20pt \vbox{\hrule \hbox{\strut \vrule{} {\bf Suggestions for Socratic discussion} \vrule} \hrule}

\begin{itemize}
\item{} A helpful ``active reading'' technique for technical texts is to work through every mathematical example presented, to ensure you understand the math as you read along.  Apply this technique here, demonstrating how to work through at least one of the calculation examples presented in the textbook.
\item{} A critically important step when learning any new mathematical procedure is the {\it meaning} of all the mathematical terms and concepts.  Here, for example, we must identify what {\it decibels} really means in real-life terms.  Do your best to define ``dB'' so that it has practical meaning.
\end{itemize}

\underbar{file i00588}
%(END_QUESTION)





%(BEGIN_ANSWER)


%(END_ANSWER)





%(BEGIN_NOTES)

Decibels are a unit useful for describing ratios of power.  Instead of expressing a power ratio as a simple fraction like this . . .

$$A \hbox{ (ratio)} = {P_{out} \over P_{in}}$$

. . . we may express it as the {\it logarithm} of the ratio like this:

$$A \hbox{ (decibels)} = 10 \log \left(P_{out} \over P_{in}\right)$$

Decibels may be used to express increases in power, in which case the decibel figure will be positive.  Decibels may also be used to express losses in power, in which case the decibel figure will be negative.  When components are connected together in such a way that the RF power must pass through one, then the other, and so on, power (ratios) multiply to give an overall gain ratio, but decibel gains {\it add} to give an overall decibel gain value.

A difference of 10 dB represents a 10-fold ratio of powers.  A difference of 3 dB is approximately equal to a 2-fold ratio of powers.  These two dB-to-ratio relationships are useful for estimating decibel values without a calculator.

\vskip 10pt

Decibels may also be used to express absolute powers, referenced to power levels of 1 milliwatt, 1 watt, etc.:

\begin{itemize}
\item{} dBm = power expressed in reference to 1 milliwatt
\item{} dBW = power expressed in reference to 1 watt
\item{} Power levels greater than the reference level will be positive (+dBx)
\item{} Power levels less than the reference level will be negative (-dBx)
\end{itemize}

RF power losses in cables are primarily caused by the radio frequency electric field exciting molecules in the cable's dielectric, dissipating the energy in the form of heat.

\vskip 10pt

Antennas radiation electromagnetic energy as shown by the graphical radiation pattern illustrations.  The principle of {\it reciprocity} means that any good transmitting antenna will also be a good receiving antenna (in the same direction from the antenna).  Transmitting and receiving antenna conductors should always be parallel to each other for maximum signal coupling efficiency.  Orient whip antennas vertically.  Orient Yagi antennas either (both) horizontally or (both) vertically.  When using Yagi antennas with whip antennas, orient (all) vertically.

\vskip 10pt

Antenna performance may be quantified by relating its transmitting and receiving effectiveness against that of a perfectly omnidirectional (isotropic) antenna.  The ``gain'' of an antenna is an expression of how much better it is at transmitting and receiving RF energy in a particular direction than an isotropic antenna (dBi).  Note that this is not a true amplification of power, but rather an expression of an antenna's ability to {\it focus} energy in a particular direction.  Half-wave whip antennas usually around 6 dBi.  Yagi and other highly directional antennas upwards of 20 dBi.  If antenna gain is not enough, a real power amplifying device may be added to the system to truly boost power.

\vskip 10pt

We may rate the gain of an entire transmitting system in terms of how it performs compared to a 1 mW transmitter signal into a lossless dipole antenna (ERP).  We may alternatively compare a system's performance against a 1 mW transmitter signal into a lossless isotropic antenna (EIRP).  An example of how this is used practically is how the FCC limits radio transmitter power: not by raw power output at the transmitter terminals, but rather by its effective radiated power when all components (cables, antenna) are considered together.  This is why it is illegal to take a low-power transmitter such as a walkie-talkie and connect a huge antenna to it.

\vskip 10pt







\filbreak

\vskip 20pt \vbox{\hrule \hbox{\strut \vrule{} {\bf Suggestions for Socratic discussion} \vrule} \hrule}

\begin{itemize}
\item{} Explain the purpose of using decibels to represent power gains and losses, as opposed to simple ratios.
\item{} What does dBm mean?
\item{} What does dBW mean?
\item{} What does dBi mean?
\item{} What does dBd mean?
\item{} What does ERP or EIRP mean?
\item{} Why must all amplifiers be powered by some energy source?
\item{} How does the principle of a microwave oven relate to losses in RF cables?
\item{} Explain what the principle of {\it reciprocity} means for an antenna, and why this is important to us.
\item{} Explain why it would be incorrect to point {\sl Wireless}HART devices antennas toward the Smart Wireless Gateway unit.  Which way should all the field devices antennas point?
\item{} What is an {\it isotropic} antenna, and why do we use this imaginary construct as a reference point for gauging the effectiveness of real antennas?
\item{} How should transmitting and receiving antennas be oriented with respect to one another?
\item{} Explain why Yagi antennas are almost always oriented with their elements in the {\it vertical} plane.
\item{} Could two Yagi antennas work when pointed toward each other and oriented with their elements in the {\it horizontal} plane?  Why or why not?
\item{} Why is the MTU in a radio SCADA system equipped with a whip antenna rather than a Yagi antenna?
\item{} Why is it important to place an antenna as far away as practically possible from the ground (earth) and from metal objects?
\item{} {\bf Given the fact that the Law of Energy Conservation tells us we can never get more power out of a system than what we put into that system, how can an antenna have {\it gain}?}
\item{} Why does the FCC limit the EIRP of license-free radio transmitters, rather than just limit the raw power output of the transmitter?
\end{itemize}












\vfil \eject

\noindent
{\bf Prep Quiz:}

Suppose an RF amplifier takes in a weak radio signal of 0.2 milliwatts and outputs a much stronger radio signal of 950 milliwatts.  Calculate the {\it gain} of this amplifier in decibels.

\begin{itemize}
\item{} 84.66 dB
\vskip 5pt 
\item{} 32.65 dB
\vskip 5pt 
\item{} 4750 dB
\vskip 5pt 
\item{} 36.77 dB
\vskip 5pt 
\item{} 475 dB
\vskip 5pt 
\item{} 8.466 dB
\end{itemize}



%INDEX% Reading assignment: Lessons In Industrial Instrumentation, Wireless instrumentation (antenna gain and ERP)

%(END_NOTES)


