
%(BEGIN_QUESTION)
% Copyright 2006, Tony R. Kuphaldt, released under the Creative Commons Attribution License (v 1.0)
% This means you may do almost anything with this work of mine, so long as you give me proper credit

Any substance that enhances electrical conductivity when dissolved in water is called an {\it electrolyte}.  When ``table'' salt is mixed with water to become a solution, an interesting thing happens called {\it dissociation}.  A similar process called {\it ionization} occurs when hydrogen chloride (HCl) enters a solution.  Explain what these processes are.

Then, identify how the following electrolytes will separate into cations and anions when dissolved in water.  In other words, identify which part of each electrolyte molecule will become an anion, and which part will become a cation:

\begin{itemize}
\item{} Sodium Chloride (NaCl) $\to$
\vskip 5pt
\item{} Hydrogen Chloride (HCl) $\to$
\vskip 5pt
\item{} Sodium Hydroxide (NaOH) $\to$
\vskip 5pt
\item{} Calcium Sulfate (CaSO$_{4}$) $\to$
\vskip 5pt
\item{} Potassium Sulfate (K$_{2}$SO$_{4}$) $\to$
\vskip 5pt
\item{} Sulfuric Acid (H$_{2}$SO$_{4}$) $\to$
\end{itemize}

\vskip 10pt

Hint: you might find a {\it Periodic Table of the Ions} helpful in determining how these compounds dissociate.

\underbar{file i00604}
%(END_QUESTION)





%(BEGIN_ANSWER)

Both {\it dissociation} and {\it ionization} refer to the separation of formerly joined atoms upon entering a solution.  The difference between these terms is the type of substance that splits: ``dissociation'' refers to the division of ionic compounds (such as table salt), while ``ionization'' refers to covalent-bonded (molecular) compounds such as HCl which are not ionic in their pure state.

\begin{itemize}
\item{} Sodium Chloride (NaCl) $\to$ Na$^{+}$ + Cl$^{-}$
\vskip 5pt
\item{} Hydrogen Chloride (HCl) $\to$ H$^{+}$ + Cl$^{-}$
\vskip 5pt
\item{} Sodium Hydroxide (NaOH) $\to$ Na$^{+}$ + OH$^{-}$
\vskip 5pt
\item{} Calcium Sulfate (CaSO$_{4}$) $\to$ Ca$^{2+}$ + SO$_{4}^{2-}$
\vskip 5pt
\item{} Potassium Sulfate (K$_{2}$SO$_{4}$) $\to$ 2K$^{+}$ + SO$_{4}^{2-}$
\vskip 5pt
\item{} Sulfuric Acid (H$_{2}$SO$_{4}$) $\to$ 2H$^{+}$ + SO$_{4}^{2-}$
\end{itemize}

\vskip 10pt

The distinguishing characteristic of an ionic compound is that it is a conductor of electricity in its pure, liquid state.  That is, it readily separates into anions and cations all by itself.  Even in its solid form, an ionic compound is already ionized, with its constituent atoms held together by an imbalance of electric charge.  Being in a liquid state simply gives those atoms the physical mobility needed to dissociate.

\vskip 10pt

Molecular (covalent) compounds, in contrast, do not readily separate into anions and cations in their pure, liquid states.  Covalent bonds are formed by {\it shared} electrons, not electrostatic attraction between ionized atoms.  An electron must be exchanged from one atom to another in order for a covalent compound to break into ionized parts, and so this process is called ionization.  Water is an excellent example of a covalent compound.  Although some molecules in water do ionize with no impurities added, it is a {\it very} small proportion and so we treat pure (deionized) water as an electrical insulator.

Other examples of covalently-bonded molecular compounds exist, such as hydrogen chloride.  HCl ionizes into hydrogen cations and chlorine anions when dissolved in water, making HCl an electrolyte even though HCl is not conductive in its pure, liquid state.

\vskip 10pt

In the electrolysis of saltwater, oxygen and chlorine gas will both be liberated at the anode (+), while hydrogen gas will be liberated at the cathode ($-$).  Sodium accumulates on the cathode.  The proportion of chlorine to oxygen liberated at the anode depends on the concentration of the saltwater.  If you use enough current, you will dissolve copper ions into the water, turning it bluish-green in color.

%(END_ANSWER)





%(BEGIN_NOTES)


%INDEX% Chemistry, basic: ionic versus covalent bonds
%INDEX% Chemistry, ion: dissociation

%(END_NOTES)


