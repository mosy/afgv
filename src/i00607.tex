
%(BEGIN_QUESTION)
% Copyright 2006, Tony R. Kuphaldt, released under the Creative Commons Attribution License (v 1.0)
% This means you may do almost anything with this work of mine, so long as you give me proper credit

Conductivity cells are rated in terms of a {\it cell constant}, often denoted by the Greek letter $\theta$ (``theta'') in conductivity equations.  The proper unit of measurement for this cell constant is inverse centimeters, or cm$^{-1}$.

Take the following two equations and combine them using algebraic substitution to form a new equation solving for the specific conductivity of a liquid ($k$) given raw conductance ($G$) and the cell constant ($\theta$):

$$G = k{A \over d} \hbox{\hskip 100pt} \theta = {d \over A}$$

\noindent
Where,

$G$ = Conductance, in Siemens (S)

$\theta$ = Conductivity cell constant, inverse centimeters (cm$^{-1}$)

$k$ = Specific conductivity of liquid, in Siemens per centimeter (S/cm)

$A$ = Electrode area, in square centimeters (cm$^{2}$)

$d$ = Electrode separation distance, in centimeters (cm)

\vskip 10pt

\underbar{file i00607}
%(END_QUESTION)





%(BEGIN_ANSWER)

$$k = G \theta$$

%(END_ANSWER)





%(BEGIN_NOTES)


%INDEX% Measurement, analytical: conductivity
%INDEX% Mathematics review: manipulating and combining equations to form a new equation

%(END_NOTES)


