
%(BEGIN_QUESTION)
% Copyright 2006, Tony R. Kuphaldt, released under the Creative Commons Attribution License (v 1.0)
% This means you may do almost anything with this work of mine, so long as you give me proper credit

What is the difference between an {\it acid}, a {\it base}, and a {\it salt}?

\vskip 10pt

Determine whether the following compounds are acids, bases, or salts, based on their formulae:

\medskip 
\item{} NaOH
\item{} KCl
\item{} H$_{2}$SO$_{4}$
\item{} HNO$_{3}$
\item{} KOH
\item{} HCN
\item{} NaCl
\item{} ZnSO$_{4}$
\end{itemize} 

\vskip 20pt \vbox{\hrule \hbox{\strut \vrule{} {\bf Suggestions for Socratic discussion} \vrule} \hrule}

\begin{itemize}
\item{} A helpful reference to consult is a {\it periodic table of the ions}, showing the ``preferred'' ionic states of elements and common compounds.
\item{} Explain why the addition of O$^{2-}$ ions to a solution increases pH just as the addition of OH$^{-}$ ions will.
\item{} All other factors being equal, does the electrical conductivity of any electrolyte vary depending on if it is an acid, a base, or a salt?
\end{itemize}

\underbar{file i00612}
%(END_QUESTION)





%(BEGIN_ANSWER)

\noindent
{\bf Partial answer:}

\vskip 10pt

An {\it acid} is a substance that produces positive hydrogen ions (hydrogen cations, H$^{+}$) when mixed with water.  A {\it base} (often called a {\it caustic} or an {\it alkaline}) is a substance that produces negative hydroxyl ions (hydroxyl anions, OH$^{-}$) when mixed with water.  A {\it salt} is a substance that contains a cation other than H$^{+}$ and an anion other than OH$^{-}$ or O$^{2-}$.

%(END_ANSWER)





%(BEGIN_NOTES)

When acids mix with bases, neutralize one another to produce water and salt(s).

\vskip 10pt

{\bf Sodium hydroxide} is a {\it base} (produces OH$^{-}$ in solution)

NaOH $\to$ Na$^{+}$ + OH$^{-}$

\vskip 10pt

{\bf Potassium chloride} is a {\it salt} (produces neither H$^{+ }$ nor OH$^{-}$ nor O$^{2-}$ in solution)

KCl $\to$ K$^{+}$ + Cl$^{-}$

\vskip 10pt

{\bf Sulfuric acid} is an {\it acid} (produces H$^{+}$ in solution)

H$_{2}$SO$_{4}$ $\to$ 2H$^{+}$ + SO$_{4}$$^{2-}$

\vskip 10pt

{\bf Nitric acid} is an {\it acid} (produces H$^{+}$ in solution)

HNO$_{3}$ $\to$ H$^{+}$ + NO$_{3}$$^{-}$

\vskip 10pt

{\bf Potassium hydroxide} is a {\it base} (produces OH$^{-}$ in solution)

KOH $\to$ K$^{+}$ + OH$^{-}$

\vskip 10pt

{\bf Hydrocyanic acid} is an {\it acid} (produces H$^{+}$ in solution)

HCN $\to$ H$^{+}$ + CN$^{-}$

\vskip 10pt

{\bf Sodium chloride} is a {\it salt} (produces neither H$^{+ }$ nor OH$^{-}$ nor O$^{2-}$ in solution)

NaCl $\to$ Na$^{+}$ + Cl$^{-}$

\vskip 10pt

{\bf Zinc sulfate} is a {\it salt} (produces neither H$^{+ }$ nor OH$^{-}$ nor O$^{2-}$ in solution)

ZnSO$_{4}$ $\to$ Zn$^{+}$ + SO$_{4}$$^{-}$


%INDEX% Chemistry, pH: acid (defined)
%INDEX% Chemistry, pH: alkaline (defined)
%INDEX% Chemistry, pH: base (defined)
%INDEX% Chemistry, pH: caustic (defined)
%INDEX% Chemistry, pH: salt (defined)

%(END_NOTES)


