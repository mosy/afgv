
%(BEGIN_QUESTION)
% Copyright 2006, Tony R. Kuphaldt, released under the Creative Commons Attribution License (v 1.0)
% This means you may do almost anything with this work of mine, so long as you give me proper credit

Many liquid substances undergo a process whereby their constituent molecules split into positively and negatively charged ion pairs.  Liquid {\it ionic} compounds split into ions completely or nearly completely, while only a small percentage of the molecules in a liquid {\it covalent} compound split into ions.  The process of neutral molecules separating into ion pairs is called {\it dissociation} when it happens in ionic compounds, and {\it ionization} when it happens to covalent compounds.


Molten salt (NaCl) is an example of the former, while pure water is an example of the latter.  The large presence of ions in molten salt explains why it is a good conductor of electricity, while the comparative lack of ions in pure water explains why it is often considered an insulator.

However, while the ionization process in water is very slight, it is not zero.  Research the molarity (how many moles per liter of volume -- $M$) for both anions and cations in pure water at 25$^{o}$ C.

\vskip 10pt

Hint: the best sources to research for this question may be found in tutorials on {\it pH} and {\it pH measurement}.

\underbar{file i00613}
%(END_QUESTION)





%(BEGIN_ANSWER)

\noindent
In pure water at 25$^{o}$ C:

\vskip 10pt

[H$^{+}$] = 1.00 $\times$ $10^{-7}$ $M$ 

\vskip 10pt

[OH$^{-}$] = 1.00 $\times$ $10^{-7}$ $M$ 

\vskip 10pt

For pure water and also for dilute solutions of water at 25$^{o}$ C, the product of these two molarities is always (very nearly) equal to 1.00 $\times$ $10^{-14}$.  This is known as the {\it ionization constant of water}, or $K_W$:

$$K_W = [\hbox{H}^{+}] [\hbox{OH}^{-}] = 1.00 \times 10^{-14}$$

%(END_ANSWER)





%(BEGIN_NOTES)

Indeed, the electrical conductivity of a liquid substance is the definitive test of whether it is an ionic or a covalent (``molecular'') substance.

%INDEX% Chemistry, ion: dissociation (defined)
%INDEX% Chemistry, ion: ionization (defined)

%(END_NOTES)


