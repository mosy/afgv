
%(BEGIN_QUESTION)
% Copyright 2006, Tony R. Kuphaldt, released under the Creative Commons Attribution License (v 1.0)
% This means you may do almost anything with this work of mine, so long as you give me proper credit

Dilute (or pure) water at 25$^{o}$ C has an ionization constant of 1.00 $\times$ $10^{-14}$, meaning that the product of the hydrogen ion molarity and the hydroxyl ion molarity is constantly equal to this value:

$$K_W = [\hbox{H}^{+}] [\hbox{OH}^{-}] = 1.00 \times 10^{-14}$$

The {\it pH} value of an aqueous (water-based) solution is a way to express the molarity of hydrogen ions [H$^{+}$] in solution.  Specifically, it is the negative logarithm of hydrogen ion molarity:

$$\hbox{pH} = - \log [\hbox{H}^{+}]$$

Given this definition of pH, calculate the pH value of absolutely pure water, where the number of hydrogen ions [H$^{+}$] exactly equals the number of hydroxyl ions [OH$^{-}$].

\vskip 10pt

Challenge question: while pH is often defined as the negative logarithm of hydrogen ion activity (nearly equal to the negative log of hydrogen ion molarity $- \log [\hbox{H}^{+}]$ in dilute solutions), it is more properly defined as the negative log of {\it hydronium ion} activity.  Explain what this distinction means.

\underbar{file i00614}
%(END_QUESTION)





%(BEGIN_ANSWER)

Pure water at 25$^{o}$ C = 7.0 pH

\vskip 10pt

Free hydrogen ions rarely exist in solution, but more commonly are found attached to water molecules (H$_{2}$O) where they form the positive ion H$_{3}$O$^{+}$ called hydronium.

%(END_ANSWER)





%(BEGIN_NOTES)


%INDEX% Chemistry, pH: (defined)
%INDEX% Chemistry, pH: molarity calculation
%INDEX% Chemistry, ion: ionization constant of water (Kw)

%(END_NOTES)


