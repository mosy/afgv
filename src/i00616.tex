
%(BEGIN_QUESTION)
% Copyright 2006, Tony R. Kuphaldt, released under the Creative Commons Attribution License (v 1.0)
% This means you may do almost anything with this work of mine, so long as you give me proper credit

It is commonly said that the pH value of pure, or ``neutral'' water is always 7.0.  This is not completely true.  While the pH of pure water is indeed 7.0 at 25$^{o}$ C, it is not at other temperatures!

What follows is a table of ionization constants for pure water at different temperatures:

% No blank lines allowed between lines of an \halign structure!
% I use comments (%) instead, so that TeX doesn't choke.

$$\vbox{\offinterlineskip
\halign{\strut
\vrule \quad\hfil # \ \hfil & 
\vrule \quad\hfil # \ \hfil \vrule \cr
\noalign{\hrule}
%
% First row
Temperature & $K_W$ \cr
%
\noalign{\hrule}
%
% Another row
0$^{o}$ C & 1.139 $\times$ $10^{-15}$ \cr
%
\noalign{\hrule}
%
% Another row
5$^{o}$ C & 1.846 $\times$ $10^{-15}$ \cr
%
\noalign{\hrule}
%
% Another row
10$^{o}$ C & 2.920 $\times$ $10^{-15}$ \cr
%
\noalign{\hrule}
%
% Another row
15$^{o}$ C & 4.505 $\times$ $10^{-15}$ \cr
%
\noalign{\hrule}
%
% Another row
20$^{o}$ C & 6.809 $\times$ $10^{-15}$ \cr
%
\noalign{\hrule}
%
% Another row
25$^{o}$ C & 1.008 $\times$ $10^{-14}$ \cr
%
\noalign{\hrule}
%
% Another row
30$^{o}$ C & 1.469 $\times$ $10^{-14}$ \cr
%
\noalign{\hrule}
%
% Another row
35$^{o}$ C & 2.089 $\times$ $10^{-14}$ \cr
%
\noalign{\hrule}
%
% Another row
40$^{o}$ C & 2.919 $\times$ $10^{-14}$ \cr
%
\noalign{\hrule}
%
% Another row
45$^{o}$ C & 4.018 $\times$ $10^{-14}$ \cr
%
\noalign{\hrule}
%
% Another row
50$^{o}$ C & 5.474 $\times$ $10^{-14}$ \cr
%
\noalign{\hrule}
%
% Another row
55$^{o}$ C & 7.296 $\times$ $10^{-14}$ \cr
%
\noalign{\hrule}
%
% Another row
60$^{o}$ C & 9.614 $\times$ $10^{-14}$ \cr
%
\noalign{\hrule}
} % End of \halign 
}$$ % End of \vbox

Calculate the pH values of pure water at some of these temperatures, noting which direction pH changes as temperature increases and as temperature decreases. 

\vskip 20pt \vbox{\hrule \hbox{\strut \vrule{} {\bf Suggestions for Socratic discussion} \vrule} \hrule}

\begin{itemize}
\item{} High-accuracy pH probes often have {\it temperature compensating sensors} (RTDs) installed to measure the temperature of the probe as it is immersed in the solution.  Contrary to popular belief, this temperature probe does not compensate for changes in the water's pH value, but rather it compensates for something else.  Identify what this ``something else'' is, and why it needs to be compensated in high-accuracy applications.
\end{itemize}

\underbar{file i00616}
%(END_QUESTION)





%(BEGIN_ANSWER)

pH = 7.47 at 0$^{o}$ C and pH = 6.51 at 60$^{o}$ C, so pH decreases with increasing temperature.

\vskip 10pt

The definition of ``neutral'' is worth exploring a bit.  For all the different scenarios represented by the table of $K_W$ values, the water is still neutral.  The definition of a ``neutral'' solution in the context of pH is that hydrogen and hydroxyl ion activity is exactly equal ([H$^{+}$] = [OH$^{-}$]).  Given the changing ionization constant of water at different temperatures, it is entirely possible to have a neutral solution that does {\it not} have a pH value of 7, as your calculations should show here!

\vskip 10pt

This trend may be explained in terms of Conservation of Energy.  Remember that it takes energy to break a chemical bond.  Ionization of a compound is an example of this, separating atoms that were formerly joined by either covalent bonding or electrostatic attraction.  As heat energy is added to a sample of water, more water molecules are able to separate.  A very cold sample of water has minimal energy, and so we find the molecules in a low energy state (hydrogen bonded with oxygen, rather than separating into ions).



%(END_ANSWER)





%(BEGIN_NOTES)

Data for the table was derived from $- \log K_W$ constants given in the {\it CRC Handbook of Chemistry and Physics}, 64th edition, top of page D-170.

%INDEX% Chemistry, pH: molarity calculation
%INDEX% Chemistry, pH: pure water at different temperatures

%(END_NOTES)


