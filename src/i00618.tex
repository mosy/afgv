
%(BEGIN_QUESTION)
% Copyright 2006, Tony R. Kuphaldt, released under the Creative Commons Attribution License (v 1.0)
% This means you may do almost anything with this work of mine, so long as you give me proper credit

The {\it Nernst equation} is the primary formula used to predict voltage generated across {\it any} ion-selective membrane, pH measurement electrodes included.  The general form of the equation is as follows:

$$V = {{R T} \over {nF}} \ln \left({C_1 \over C_2}\right)$$

\noindent
Where,

$V$ = Voltage produced across membrane due to ion exchange, in volts (V)

$R$ = Universal gas constant (8.315 J/mol$\cdot$K)

$T$ = Absolute temperature, in Kelvin (K)

$n$ = Number of electrons transferred per ion exchanged (unitless)

$F$ = Faraday constant, in coulombs per mole (96,485 C/mol e$^{-}$)

$C_1$ = Concentration of ion in measured solution, in moles per liter of solution ($M$)

$C_2$ = Concentration of ion in reference solution (on other side of membrane), in moles per liter of solution ($M$)

\vskip 10pt

We may also write the Nernst equation using of common logarithms instead of natural logarithms, which is usually how we see it written in the context of pH measurement:

$$V = {{2.303 R T} \over {nF}} \log \left({C_1 \over C_2}\right)$$

Calculate the pH values for the following solution concentrations, assuming a temperature of 25$^{o}$ C, and also use the Nernst equation to calculate the amount of voltage produced across the thickness of a glass pH measurement electrode.  Keep in mind that the measurement electrode's internal potassium chloride (KCl) reference solution is buffered to maintain a hydrogen ion molarity of [H$^{+}$] = 1.00 $\times$ $10^{-7}$ $M$ under all conditions.  The proper value for $n$ is 1, because we are dealing with hydrogen ions:

% No blank lines allowed between lines of an \halign structure!
% I use comments (%) instead, so that TeX doesn't choke.

$$\vbox{\offinterlineskip
\halign{\strut
\vrule \quad\hfil # \ \hfil & 
\vrule \quad\hfil # \ \hfil & 
\vrule \quad\hfil # \ \hfil \vrule \cr
\noalign{\hrule}
%
% First row
[H$^{+}$] in solution & pH & V \cr
%
\noalign{\hrule}
%
% Another row
5.83 $\times$ $10^{-5}$ $M$ &  &  \cr
%
\noalign{\hrule}
%
% Another row
1 $\times$ $10^{-5}$ $M$ &  &  \cr
%
\noalign{\hrule}
%
% Another row
1 $\times$ $10^{-6}$ $M$ &  &  \cr
%
\noalign{\hrule}
%
% Another row
1 $\times$ $10^{-7}$ $M$ &  &  \cr
%
\noalign{\hrule}
%
% Another row
1 $\times$ $10^{-8}$ $M$ &  &  \cr
%
\noalign{\hrule}
%
% Another row
7.02 $\times$ $10^{-9}$ $M$ &  &  \cr
%
\noalign{\hrule}
%
% Another row
1 $\times$ $10^{-9}$ $M$ &  &  \cr
%
\noalign{\hrule}
%
% Another row
1 $\times$ $10^{-10}$ $M$ &  &  \cr
%
\noalign{\hrule}
} % End of \halign 
}$$ % End of \vbox

After calculating some voltage values, you will begin to notice a pattern.  There is a proportionality between voltage generated and the pH value of the measured solution.  See if you can detect this pattern, and express the relationship in simple terms.

\underbar{file i00618}
%(END_QUESTION)





%(BEGIN_ANSWER)

% No blank lines allowed between lines of an \halign structure!
% I use comments (%) instead, so that TeX doesn't choke.

$$\vbox{\offinterlineskip
\halign{\strut
\vrule \quad\hfil # \ \hfil & 
\vrule \quad\hfil # \ \hfil & 
\vrule \quad\hfil # \ \hfil \vrule \cr
\noalign{\hrule}
%
% First row
[H$^{+}$] in solution & pH & V \cr
%
\noalign{\hrule}
%
% Another row
5.83 $\times$ $10^{-5}$ $M$ & 4.23 & 163.6 mV \cr
%
\noalign{\hrule}
%
% Another row
1 $\times$ $10^{-5}$ $M$ & 5 & 118.3 mV \cr
%
\noalign{\hrule}
%
% Another row
1 $\times$ $10^{-6}$ $M$ & 6 & 59.16 mV \cr
%
\noalign{\hrule}
%
% Another row
1 $\times$ $10^{-7}$ $M$ & 7 & 0 mV \cr
%
\noalign{\hrule}
%
% Another row
1 $\times$ $10^{-8}$ $M$ & 8 & $-59.16$ mV \cr
%
\noalign{\hrule}
%
% Another row
7.02 $\times$ $10^{-9}$ $M$ & 8.15 & $-68.25$ mV \cr
%
\noalign{\hrule}
%
% Another row
1 $\times$ $10^{-9}$ $M$ & 9 & $-118.3$ mV \cr
%
\noalign{\hrule}
%
% Another row
1 $\times$ $10^{-10}$ $M$ & 10 & $-177.5$ mV \cr
%
\noalign{\hrule}
} % End of \halign 
}$$ % End of \vbox

Note that the difference in pH between the measured solution and the internal reference solution is indicated by a simple DC voltage!

\vskip 10pt

Slope = 59.17 mV per pH unit deviation from 7.0 pH

\vskip 10pt

Temperature compensation is required for high-accuracy pH measurement, to compensate for the temperature term in the Nernst equation.  This temperature term strictly relates to the ion-permeable membrane, and not the solution.  In other words, using an RTD to sense and correct for temperature merely compensates for changes in the pH/voltage proportionality of the measurement electrode, and does not (indeed, it {\it cannot}) compensate for actual ion activity changes in the solution.  In order to do the latter (i.e. provide a temperature-corrected measurement of pH telling you what the pH of the solution {\it would} be at a fixed reference temperature rather than what it actually is at the real temperature), one would have to know the solution's unique temperature/[H$^{+}$] relationship, which will be unique for every different solution.

\vskip 10pt

Incidentally, the true version of the Nernst equation is as follows:

$$V = {{R T} \over {nF}} \ln \left({a_1 \over a_2}\right)$$

\noindent
Where,

$a_1$ = Activity of ion in measured solution

$a_2$ = Activity of ion in reference solution (on other side of membrane)

\vskip 10pt

In the absence of chemical reactions that ``tie up'' ions, activity ($a$) and concentration ($C$) are directly proportional to one another ($a = \gamma C$).  However, the activity coefficient ($\gamma$) may be subject to change in some situations, in which case the Nernst equation becomes even uglier.

%(END_ANSWER)





%(BEGIN_NOTES)


%INDEX% Chemistry, electro-: Nernst equation
%INDEX% Chemistry, pH: molarity calculation
%INDEX% Measurement, analytical: pH

%(END_NOTES)


