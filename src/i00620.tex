
%(BEGIN_QUESTION)
% Copyright 2006, Tony R. Kuphaldt, released under the Creative Commons Attribution License (v 1.0)
% This means you may do almost anything with this work of mine, so long as you give me proper credit

Explain why the glass measurement electrode if a pH sensor must always be kept wet, ideally with a potassium chloride (KCl) solution, when in storage.

\underbar{file i00620}
%(END_QUESTION)





%(BEGIN_ANSWER)

The function of this electrode depends on the glass having internal and external {\it hydrated} layers.  If the electrode becomes dried on the outside, ions cannot permeate it, and it becomes useless.

The hydrated layers of a glass electrode slowly dissolve over time, exposing new layers of (dry) glass underneath which hydrate and function anew.  The maintenance of this hydrated layer is critical to proper measurement probe operation.

This also explains why pH measurement probe life is finite, and why it shortens with elevated temperature.  As solution temperature increases, the speed of glass dissolution increases, eating away the electrode at a faster rate.

%(END_ANSWER)





%(BEGIN_NOTES)


%INDEX% Measurement, analytical: pH

%(END_NOTES)


