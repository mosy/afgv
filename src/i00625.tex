
%(BEGIN_QUESTION)
% Copyright 2010, Tony R. Kuphaldt, released under the Creative Commons Attribution License (v 1.0)
% This means you may do almost anything with this work of mine, so long as you give me proper credit

Suppose you are tasked with calibrating a pneumatic temperature transmitter having a {\it filled bulb} as the sensing element and a 3 to 15 PSI output signal range.  The only way to calibrate such a transmitter, of course, is to make adjustments to the ``zero'' and ``span'' screws with the sensing bulb stabilized at accurately known temperatures.  Unfortunately for you, there are no dry-block, sand bath, or oil bath calibrators available for use, nor are there any calibration-grade thermometers or other temperature-sensing test instruments to compare this transmitter against.  Instead, you must use water as your {\it only} temperature standard.

\vskip 10pt

The desired LRV of the transmitter is 10 $^{o}$F, and the desired URV for the transmitter is 250 $^{o}$F.  Explain in detail how you would go about calibrating this transmitter to the desired range using only water as your temperature standard.  Assume your geographical elevation is sea level.

\vskip 50pt

\underbar{file i00625}
%(END_QUESTION)





%(BEGIN_ANSWER)

An ice-water mix will provide a standard of 32 $^{o}$F, while a boiling water pot will provide a standard of 212 $^{o}$F.  Since these temperatures do not match the LRV and URV values given, you will have to perform your zero and span adjustments at these output signal pressures:

\begin{itemize}
\item{} At 32 $^{o}$F, adjust to 4.1 PSI
\vskip 5pt
\item{} At 212 $^{o}$F, adjust to 13.1 PSI
\end{itemize}

Only award a full 10 points if these numerical values are given as part of the answer.  If a general procedure is described without these values, award only 5 points.

\vskip 10pt

Placing the water in a vessel where pressure may be adjusted (in order to skew the phase-change temperatures) won't work at the freezing point, only at the boiling point.  Pressure gauges are available to you, as is necessitated by the pneumatic nature of this transmitter.  If the answer is conceptually and numerically correct for only one calibration point but not the other (e.g. using pressurized, boiling water for the URV but attempting to use pressure to skew the freezing point), award half-credit.

\vskip 10pt

If the answer is conceptually correct on all counts, but a simple math error was made, subtract 2 points for each erroneous value.

%(END_ANSWER)





%(BEGIN_NOTES)

{\bf This question is intended for exams only and not worksheets!}.

%(END_NOTES)

