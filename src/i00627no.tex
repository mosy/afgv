
%(BEGIN_QUESTION)
% Copyright 2010, Tony R. Kuphaldt, released under the Creative Commons Attribution License (v 1.0)
% This means you may do almost anything with this work of mine, so long as you give me proper credit

%In this circuit, an electronic differential pressure transmitter with a 4-20 mA output signal connects to a local pressure indicator and to a remote pressure indicator.  Your task is to figure out how to connect the loop calibrator to force the remote indicator (only) to register a pressure of 300 PSI, without sending any signal to the local indicator:
I denne kretsen er en DP-celle med et 4-20 mA utgangssignal tilkoblet en lokal trykk indikator og en ekstern trykk indikator. Din oppgave er {\aa} koble loop-kalibratoren slik at de eksterne trykk indikatoren viser 300 PSI, uten {\aa} sende noe signal til den lokale indikatoren. 

$$\epsfbox{i00627x01.eps}$$

%Sketch your solution, showing both the loop calibrator's test lead connections and the mode it must be set to ({\it source}, {\it read}, or {\it simulate}).  Assume a transmitter calibration of -10 PSI to +440 PSI.
Skisser din l{\o}sning, som viser hvor ledningen tilkobles og hvordan loop-kalibratoren innstilles ({\it source}, {\it read}, or {\it simulate}). Anta at transmitteren er kalibrert fra -10 PSI til +440 PSI.  


\vfil

% Note: there is more than one correct answer to this question!
Obs, det er mange rette svar p{\aa} dette sp{\o}rsm{\aa}let. 

\underbar{file i00627no}
\eject
%(END_QUESTION)





%(BEGIN_ANSWER)

This is a graded question -- no answers or hints given!

%(END_ANSWER)





%(BEGIN_NOTES)

First, we will calculate the amount of current necessary from the loop calibrator in order to simulate a 300 PSI condition.  We know that the transmitter's range is -10 PSI to +440 PSI.  Therefore, a pressure of +300 PSI will be 310 PSI above the ``zero'' point over a span of 450 PSI, and therefore will represent a percentage of range equal to ${310 \over 450} = 68.66\%$.  Converting a 68.66\% signal into milliamps over a 4-20 mA range, we get {\bf 15.02 mA}.

\vskip 10pt

Solution using the loop calibrator to {\it source} current:

$$\epsfbox{i00627x02.eps}$$

\vfil \eject

Solution using the loop calibrator to {\it simulate} current:

$$\epsfbox{i00627x03.eps}$$

%INDEX% Electronics review: 4-20 mA loop calibrator (test equipment)

%(END_NOTES)


