
%(BEGIN_QUESTION)
% Copyright 2016, Tony R. Kuphaldt, released under the Creative Commons Attribution License (v 1.0)
% This means you may do almost anything with this work of mine, so long as you give me proper credit

A relatively new technology for measuring the concentration of oxygen gas dissolved in water is called {\it dynamic luminescence quenching}.  This is an optical technology, relying on the fluorescent properties of a substance containing ``lumiphore'' molecules.  The basic concept is this: the lumiphore molecules readily fluoresce when exposed to a particular wavelength of light, but this fluorescence is affected by the presence of oxygen molecules.  Oxygen near the lumiphore molecules tends to ``quench'' the excited state of those molecules after being exposed to the incident light, thereby prohibiting fluorescence.  By repeatedly exciting the lumiphore molecules with pulses of light and measuring the fluorescent light returned by those molecules, the concentration of dissolved oxygen may be inferred.

\vskip 10pt

Different colors of light are used by this type of analyzer: one color injected into the lumiphore molecules by a light source, and a different color of light emitted by the lumiphore molecules to be detected by a light sensor.  One of these colors is {\it blue} and the other is {\it red}.

\vskip 10pt

Based on what you know about the interaction of light with molecules, identify which color is emitted by the light source and which color is emitted by the lumiphore molecules.  Also, determine whether the intensity of the light emitted by the lumiphore molecules is {\it directly proportional} to or {\it inversely proportional} to the concentration of dissolved oxygen in the liquid sample.

\underbar{file i00635}
%(END_QUESTION)





%(BEGIN_ANSWER)

The light source outputs blue light, while the lumiphore molecules fluoresce red light.  We know this because in all cases of fluorescence the incident wavelength is shorter (higher-frequency) than the fluoresced wavelength.  Blue light has a shorter wavelength (higher frequency) than red light, and therefore it must be the blue color that is emitted by the light source and the red color that is returned by the lumiphore molecules as they fluoresce.

\vskip 10pt

The proportionality between the intensity of this fluoresced light and the concentration of dissolved oxygen is {\it inverse} based on the description given of dynamic luminescence quenching at the start of this question.  The more dissolved oxygen in the presence of the lumiphore molecules, the more their fluorescence will be quenched, and therefore the less red light they will emit when excited by the blue light coming from the light source.

%(END_ANSWER)





%(BEGIN_NOTES)


%INDEX% Measurement, analytical: dissolved oxygen
%INDEX% Physics, optics: dynamic luminescence quenching
%INDEX% Physics, optics: fluorescence

%(END_NOTES)


