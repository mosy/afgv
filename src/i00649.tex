
%(BEGIN_QUESTION)
% Copyright 2009, Tony R. Kuphaldt, released under the Creative Commons Attribution License (v 1.0)
% This means you may do almost anything with this work of mine, so long as you give me proper credit

A radiation-style pyrometer measuring the temperature of a piece of glowing-hot steel coming out of a furnace at a steel mill produces a millivolt signal (at the thermopile sensing element) of 12.69 mV when the steel is at a temperature of 1400 $^{o}$F.  Calculate the approximate millivolt output signals at the following target temperatures:

\begin{itemize}
\item{} 1600 $^{o}$F ; Output = \underbar{\hskip 50pt} mV
\vskip 5pt
\item{} 800 $^{o}$C ; Output = \underbar{\hskip 50pt} mV
\vskip 5pt
\item{} 1000 K ; Output = \underbar{\hskip 50pt} mV
\end{itemize}

\underbar{file i00649}
%(END_QUESTION)





%(BEGIN_ANSWER)

\begin{itemize}
\item{} 1600 $^{o}$F ; Output = \underbar{\bf 19.09} mV
\vskip 5pt
\item{} 800 $^{o}$C ; Output = \underbar{\bf 14.77} mV
\vskip 5pt
\item{} 1000 K ; Output = \underbar{\bf 11.14} mV
\end{itemize}


%(END_ANSWER)





%(BEGIN_NOTES)


%INDEX% Measurement, temperature: optical

%(END_NOTES)


