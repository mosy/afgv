
%(BEGIN_QUESTION)
% Copyright 2006, Tony R. Kuphaldt, released under the Creative Commons Attribution License (v 1.0)
% This means you may do almost anything with this work of mine, so long as you give me proper credit

In both gas chromatography and liquid chromatography, there must always be two ``phases'' at work: a {\it mobile phase} and a {\it stationary phase}.  Explain what each of these phases is, and give examples for both gas and liquid chromatography.

\underbar{file i00666}
%(END_QUESTION)





%(BEGIN_ANSWER)

\noindent
{\bf Gas chromatography}:

\vskip 10pt

Stationary phase = a solid material, and/or a liquid held stationary by a solid material, all held in a chamber known as the ``column.''

\vskip 10pt

Mobile phase = sample (gas) mixed with an inert gas that ``carries'' the sample through and past the stationary phase.

\vskip 40pt

\noindent
{\bf Liquid chromatography}:

\vskip 10pt

Stationary phase = a solid material, and/or a liquid held stationary by a solid material, all held in a chamber known as the ``column.''

\vskip 10pt

Mobile phase = sample (liquid) mixed with an inert liquid that ``carries'' the sample through and past the stationary phase.

%(END_ANSWER)





%(BEGIN_NOTES)


%INDEX% Measurement, analytical: chromatography

%(END_NOTES)


