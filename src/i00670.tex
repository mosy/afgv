
%(BEGIN_QUESTION)
% Copyright 2006, Tony R. Kuphaldt, released under the Creative Commons Attribution License (v 1.0)
% This means you may do almost anything with this work of mine, so long as you give me proper credit

Chromatographs work by passing a fluid through a narrow length of tubing called a {\it column} and measuring what comes out the other end.  This fluid consists of a {\it carrier} mixed with the sample to be analyzed.  

Explain (in more detail) what happens inside the ``column'' of a chromatograph, and how this relates to the analysis of chemicals in the sample stream.

\underbar{file i00670}
%(END_QUESTION)





%(BEGIN_ANSWER)

As the mix of carrier and sample fluids travels through the column, the packing material inside the column delays the travel of some sample components, resulting in different chemical components exiting the column at different times.  The basic principle of a chromatograph is that a mix of flowing chemicals becomes separated over time, allowing for individual, quantitative measurement of each chemical by a single detector device.

\vskip 10pt

The column itself is usually just a long metal tube with a very small inside diameter.  Sometimes the length to diameter ratio of the column tubing is enormous: take for example a packed column used for the separation of components in gasoline, with a column length of 100 meters and an inside diameter of only 250 $\mu$m (micro-meters!).

As a side note, this spectacularly long column example comes from Raymond P.W. Scott's excellent book, {\it Principles and Practice of Chromatography}, available for free download on the Internet.  In this particular example of chromatographic gasoline analysis, the packing material was Petrocol DH with a thickness of 0.5 $\mu$m on the inside of the column tube wall, and the sample volume was 0.1 $\mu$l.  The complete analysis took about 100 minutes, the long time due chiefly to the long length of the column.  The carrier gas was helium, and the detector was an FID (flame ionization) type.  Temperature programming was used to achieve good separation of the heavier components, starting at 35$^{o}$ C near the beginning of the analysis and ending at 200$^{o}$ C.  This being gas chromatography, higher temperatures mean higher gas viscosity (the very opposite temperature relation of liquid viscosity!), so the higher temperatures at the end resulted in greater retardation of the heavier gas components at the end of the run, allowing better separation.

%(END_ANSWER)





%(BEGIN_NOTES)


%INDEX% Measurement, analytical: chromatography

%(END_NOTES)


