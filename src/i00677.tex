
%(BEGIN_QUESTION)
% Copyright 2010, Tony R. Kuphaldt, released under the Creative Commons Attribution License (v 1.0)
% This means you may do almost anything with this work of mine, so long as you give me proper credit

When a hydrocarbon such as methane (CH$_{4}$) is burned with oxygen (O$_{2}$), the {\it ideal} reaction products are simply water vapor (H$_{2}$O) and carbon dioxide gas (CO$_{2}$):

$$\hbox{CH}_4 + \hbox{O}_2 \to \hbox{H}_2\hbox{O} + \hbox{CO}_2$$

However, in practice we often find some carbon monoxide gas (CO) formed in addition to carbon dioxide gas (CO$_{2}$):

$$\hbox{CH}_4 + \hbox{O}_2 \to \hbox{H}_2\hbox{O} + \hbox{CO}_2 + \hbox{CO}$$

If you try to find a single set of proportions to ``balance'' the above equation, you will find the task impossible.  Explain why.

\vskip 10pt

Now, try balancing the following two equations, each one supplied with definite proportions of methane to oxygen:

$$40\hbox{CH}_4 + 77\hbox{O}_2 \to \hbox{H}_2\hbox{O} + \hbox{CO}_2 + \hbox{CO}$$

$$40\hbox{CH}_4 + 79\hbox{O}_2 \to \hbox{H}_2\hbox{O} + \hbox{CO}_2 + \hbox{CO}$$

Explain why these chemical equations are easier to balance.  Also, identify whether it is a {\it rich} fuel/air mixture or a {\it lean} fuel/air mixture that produces more dangerous carbon monoxide in the exhaust of an engine burning natural gas.

\underbar{file i00667}
%(END_QUESTION)





%(BEGIN_ANSWER)

The first equation has an infinite number of balanced solutions.  Mathematically, there are more unknowns (variables) than we have linear equations to write.  In practical terms, this means the combustion of methane may produce {\it any} ratio of CO$_{2}$:CO depending on how rich or lean the flame burns.  This tells us that air/fuel mixture is a critical parameter for ensuring complete combustion (minimal remaining CO gas).

\vskip 10pt

The two equations with given fuel/air ratios are balanced as such:

$$40\hbox{CH}_4 + 77\hbox{O}_2 \to 80\hbox{H}_2\hbox{O} + 34\hbox{CO}_2 + 6\hbox{CO}$$

$$40\hbox{CH}_4 + 79\hbox{O}_2 \to 80\hbox{H}_2\hbox{O} + 38\hbox{CO}_2 + 2\hbox{CO}$$

Having been given a fixed ratio between methane to oxygen, we only have three unknowns (variables) to solve for, and three linear equations to write (one for carbon, one for hydrogen, and one for oxygen), making it a solvable linear system of equations.  As you can see, a leaner fuel/air mixture results in less carbon monoxide being produced.

%(END_ANSWER)





%(BEGIN_NOTES)


%INDEX% Chemistry, stoichiometry: balancing a chemical equation

%(END_NOTES)


