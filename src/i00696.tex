
%(BEGIN_QUESTION)
% Copyright 2016, Tony R. Kuphaldt, released under the Creative Commons Attribution License (v 1.0)
% This means you may do almost anything with this work of mine, so long as you give me proper credit

Suppose two valves are used to control the amount of gas pressure in a vessel, one valve admitting gas into the vessel from a supply header with a constant pressure of 95 PSIG, and the other valve venting gas out of the vessel to atmosphere.  The desired pressure in the vessel is 8 PSIG.

\vskip 10pt

Assuming no other substantial flows of gas into or out of this vessel but these two control valves, and also assuming we wish these two valves to operate at approximately 50\% open while maintaining the setpoint pressure of 8 PSIG, which of the two control valves should have the larger $C_v$ rating?  Explain your answer in detail.

\underbar{file i00696}
%(END_QUESTION)





%(BEGIN_ANSWER)

The inlet control valve sees a pressure drop of 95 PSIG $-$ 8 PSIG = 87 PSIG, while the outlet control valve only sees a pressure drop of 8 PSIG (as it vents gas to atmosphere at 0 PSIG).  However, both valves will be flowing the same rate of gas assuming no other paths for gas flow into or out of the vessel.  This means the outlet valve must flow the same amount of gas with less than one-tenth the pressure drop across it.  

\vskip 10pt

Therefore, the outlet valve must have a much greater $C_v$ than the inlet valve.  The calculated ratio is thus:

$$\sqrt{87 \over 8} = 3.2977$$

%(END_ANSWER)





%(BEGIN_NOTES)


%INDEX% Final Control Elements, valve: sizing

%(END_NOTES)


