
%(BEGIN_QUESTION)
% Copyright 2015, Tony R. Kuphaldt, released under the Creative Commons Attribution License (v 1.0)
% This means you may do almost anything with this work of mine, so long as you give me proper credit

Read the ``LabVIEW exercise \#1'' tutorial document in preparation for implementing it in the computer lab, in conjunction with a USB-based data acquisition unit.  Here, you will learn how to configure a simple ``virtual instrument'' in National Instruments LabVIEW software to display an analog signal value on the computer screen.

\vskip 10pt

Note: the tutorial you need to do this exercise is found on your ``Instrumentation Reference'' (a set of digital document files) your instructor has prepared for you.

\underbar{file i00705}
%(END_QUESTION)





%(BEGIN_ANSWER)


%(END_ANSWER)





%(BEGIN_NOTES)

In this exercise we configure LabVIEW to acquire an analog voltage signal and display it on the Front Panel.  When complete, we should have a simple voltmeter display on the PC. 










\vfil \eject

\noindent
{\bf Prep Quiz:}

LabVIEW programming software uses two display windows in its development environment: the {\it Block Diagram} and the {\it Front Panel}.  The difference between these two windows is:

\begin{itemize}
\item{} The Block Diagram shows a "live" view of all the signals, while the Front Panel is just a static display with no interactive capability
\vskip 10pt
\item{} The Block Diagram shows user controls and displays, while the Front Panel shows how the mathematical calculations will be done
\vskip 10pt
\item{} The Block Diagram documents all electrical wiring on the DAQ unit, while the Front Panel documents all the software functions
\vskip 10pt
\item{} The Block Diagram shows the signal processing functions, while the Front Panel contains the user controls and displays
\end{itemize}












\vfil \eject

\noindent
{\bf Prep Quiz:}

Describe just one of the many activities you'll be doing with LabVIEW and a DAQ today.  Please be as detailed as you can.

%INDEX% Reading assignment: LabVIEW exercise #1

%(END_NOTES)

