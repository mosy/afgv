
%(BEGIN_QUESTION)
% Copyright 2006, Tony R. Kuphaldt, released under the Creative Commons Attribution License (v 1.0)
% This means you may do almost anything with this work of mine, so long as you give me proper credit

\vskip 10pt
\hrule
\vskip 10pt

Bernoulli's equation expressed in terms of mass density ($\rho$) and in terms of weight density ($\gamma$):

$$z_1 \rho g + {v_1^2 \rho \over 2} + P_1 = z_2 \rho g + {v_2^2 \rho \over 2} + P_2$$

$$z_1 + {v_1^2 \over {2 g}} + {P_1 \over \gamma} = z_2 + {v_2^2 \over {2 g}} + {P_2 \over \gamma}$$

\noindent
Where,

$z$ = Height of fluid, in feet (ft)

$\rho$ = Mass density of fluid, in slugs per cubic foot (slug/ft$^{3}$)

$\gamma$ = Weight density of fluid ($\gamma = \rho g$), in pounds per cubic foot (lb/ft$^{3}$)

$g$ = Acceleration of gravity, in feet per second squared (ft/s$^{2}$)

$v$ = Velocity of fluid, in feet per second (ft/s)

$P$ = Pressure of fluid, in pounds per square foot (lb/ft$^{2}$)

\vskip 10pt
\hrule
\vskip 10pt

Volumetric flow:

$$Q = A \overline{v}$$

\noindent
Where,

$Q$ = Volumetric flow in cubic feet per second (ft$^{3}$/s)

$A$ = Cross-sectional flowing area in square feet (ft$^{2}$)

$\overline{v}$ = Average flowing velocity in feet per second (ft/s)
\vskip 10pt
\hrule
\vskip 10pt

Calculating Reynolds number:

$$\hbox{Re} = {{(3160) G_f Q} \over {D \mu}}$$

\noindent
Where,

Re = Reynolds number (unitless)

$G_f$ = Specific gravity of liquid (unitless)

$Q$ = Flow rate, gallons per minute (GPM)

$D$ = Diameter of pipe, in inches (in)

$\mu$ = Absolute viscosity of fluid, in centipoise (cP)

\vskip 10pt
\hrule
\vskip 10pt

\noindent
Mass density of water ($\rho_{\hbox{water}}$) = 1.951 slugs/ft$^{3}$ \hskip 50pt Weight density of water ($\gamma_{\hbox{water}}$) = 62.4 lb/ft$^{3}$

\vskip 10pt

\noindent
Absolute viscosity of water at 20$^{o}$C: = 1.0019 centipoise (cp) = 0.0010019 Pascal-seconds (Pas)

\vskip 10pt

\noindent
Acceleration of gravity ($g$) on Earth = 32.2 ft/s$^{2}$ = 9.81 m/s$^{2}$

\vskip 10pt

\noindent
1 gallon (gal) = 231.0 cubic inches (in$^{3}$) = 4 quarts (qt) = 8 pints (pt) = 128 fluid ounces (fl. oz.) = 3.7854 liters (l)

\vskip 10pt

\noindent
1 pound per square inch (PSI) = 2.03603 inches of mercury (in. Hg) = 27.6807 inches of water (in. W.C.) = 6.894757 kilo-pascals (kPa) 

\underbar{file i00706}
%(END_QUESTION)





%(BEGIN_ANSWER)


%(END_ANSWER)





%(BEGIN_NOTES)

{\bf This question is intended for exams only and not worksheets!}.

%(END_NOTES)


