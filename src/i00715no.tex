
%(BEGIN_QUESTION)
% Copyright 2011, Tony R. Kuphaldt, released under the Creative Commons Attribution License (v 1.0)
% This means you may do almost anything with this work of mine, so long as you give me proper credit

Et uunnværlig verktøy for både prosessoperatører og instrumentteknikere er {\it trendgrafen}, som viser variabler fra kontrollsløyfen som PV, SP og regulatorutgang (Output) overlagret på samme tidsakse. Følgende eksempel viser prosessvariabel, settpunkt og utgang for en P-regulator (kun proporsjonal) som reagerer på endringer i en kontrollsløyfes PV mens settpunktet forblir på en konstant verdi av 40\%:

$$\includegraphics[width=15.5cm]{i00715x01.eps}$$

Basert på en undersøkelse av denne trendgrafen, bestem {\it bias}-verdien til regulatoren og {\it forsterknings}-verdien (gain) til regulatoren, samt dens virkemåte ({\it direkte} eller {\it revers}).

\vskip 10pt

En nyttig analyseteknikk når man relaterer trendgrafer til regulatorligninger, er å tegne en vertikal linje på grafen for å identifisere et bestemt tidspunkt, og deretter identifisere verdiene for PV, SP og Utgang (Output) ved det tidspunktet. En korrekt ligning for regulatoren vil vellykket forutsi Utgangsverdien ut fra PV- og SP-verdiene ved {\it ethvert} tidspunkt vist på trenden.

\vskip 20pt \vbox{\hrule \hbox{\strut \vrule{} {\bf Forslag til sokratisk diskusjon} \vrule} \hrule}

\begin{itemize}
\item{} Når du har beregnet forsterkningen (gain) for denne sløyferegulatoren, beregn også dens {\it proporsjonalbånd}-verdi.
\item{} Lag et regnearkprogram for datamaskin for å modellere oppførselen til P-regulatoren i dette scenariet. Du vet du har lykkes når det er i stand til å duplisere enhver Utgangsverdi vist på trendgrafen ved ethvert gitt tidspunkt, tilsvarende PV- og SP-verdiene ved det samme tidspunktet.
\item{} Hvordan ville denne trenden sett ut hvis regulatoren ble stående i {\it manuell} modus i stedet for {\it automatisk} modus?
\end{itemize}

\underbar{file i00715}
%(END_QUESTION)





%(BEGIN_ANSWER)

Gain = 0,5 og bias = 30\%

%(END_ANSWER)





%(BEGIN_NOTES)

Forsterkningen kan beregnes ved å sammenligne endring i utgang ($\Delta m$) med endring i inngang ($\Delta$PV) mellom hvilke som helst to punkter i tid på trenden. Bias bestemmes enklest ved å notere utgangsverdien når avviket er lik null (PV = SP).

%INDEX% Control, proportional: graphing controller response

%(END_NOTES)
