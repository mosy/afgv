
%(BEGIN_QUESTION)
% Copyright 2005, Tony R. Kuphaldt, released under the Creative Commons Attribution License (v 1.0)
% This means you may do almost anything with this work of mine, so long as you give me proper credit

Plott responsen for følgende P-regulator (reversvirkende, gain = 2, bias = 50\%), forutsatt en trinnvis endring i skal-verdien (SP):

$$\includegraphics{i00715x01.eps}$$

\underbar{file i00715}
%(END_QUESTION)





%(BEGIN_ANSWER)


%(END_ANSWER)





%(BEGIN_NOTES)

En vanlig feil blant studentene er å glemme bias-verdien. De vil ofte plotte utgangssignalet som starter på 0\% i stedet for 50\%.

De som husker bias-verdien vil kanskje fortsatt gjøre feil med endringsretningen. "Skal den gå opp eller ned?" spør de seg selv. Min metode for å besvare dette spørsmålet er å se for meg at PV øker. Hvis regulatoren er reversvirkende, skal utgangen gå ned. Dette betyr at PV og Output må bevege seg i motsatte retninger. Derfor, hvis PV går {\it opp}, skal Output gå {\it ned}. Men vent! Vi snakker om en endring i {\it skal-verdien} (SP), ikke prosessvariabelen (PV).

Dette bringer oss til et grunnleggende prinsipp for regulering: skal-verdi og prosessvariabel har alltid motsatt effekt på utgangen. Påstanden "reversvirkende" refererer til PV'ens effekt på utgangen. Derfor vil en økning i PV drive utgangen {\it ned}. Dette innebærer at en økning i SP vil drive utgangen {\it opp}.

%INDEX% Control, proportional: graphing controller response

%(END_NOTES)
