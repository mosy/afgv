
%(BEGIN_QUESTION)
% Copyright 2011, Tony R. Kuphaldt, released under the Creative Commons Attribution License (v 1.0)
% This means you may do almost anything with this work of mine, so long as you give me proper credit

One day you are handed a Fisher model 546 (or comparable) I/P transducer to calibrate.  You have the following tools and equipment to use:

\begin{itemize}
\item{} Adjustable DC power supply (0 to 30 volts)
\item{} Jumper leads (alligator-clip ends)
\item{} Multimeter with a blown fuse (cannot measure current)
\item{} Precision pressure gauges
\item{} Precision pressure regulators
\item{} Lots of plastic tubes and tube fittings
\item{} Regulated instrument air supply (25 PSI)
\item{} Bin full of electronic components (resistors, switches, transistors, diodes, etc.)
\item{} Selection of basic hand tools (wrenches, pliers, screwdrivers, etc.)
\end{itemize}

First, describe the challenge of performing this calibration -- what factor makes it more difficult than usual?  Next, explain how you could overcome this challenge using the parts on hand, sketching a simple diagram to illustrate your solution.

\underbar{file i00719}
%(END_QUESTION)





%(BEGIN_ANSWER)

I recommend half-credit for identifying the challenge and half-credit for the solution to overcome it.

\vskip 10pt

The challenge is that your only meter cannot measure current (4-20 mA).  One way overcome this challenge is to wire a resistor in series with the power supply, using the power supply's adjustability to provide variable current, and using Ohm's Law (I = V/R) to determine current from a voltage measurement across the known resistance.

%(END_ANSWER)





%(BEGIN_NOTES)

{\bf This question is intended for exams only and not worksheets!}.

%(END_NOTES)


