
%(BEGIN_QUESTION)
% Copyright 2010, Tony R. Kuphaldt, released under the Creative Commons Attribution License (v 1.0)
% This means you may do almost anything with this work of mine, so long as you give me proper credit

A general rule-of-thumb for graduates of an Instrumentation program is to remain at your first job for at least a year or two before considering other employment, even if better-paying prospects come to your attention.  The most important rationale for this attitude is to guarantee a ``return on the investment'' your first employer makes in you.  

\vskip 10pt

Identify some of the significant ``investments'' that an employer makes in an unexperienced instrument technician.  

\vskip 50pt

\underbar{file i00721}
%(END_QUESTION)





%(BEGIN_ANSWER)

Here are a couple to get you started:

\begin{itemize}
\item{} Cost of pay and benefits during non-productive initiation time
\item{} Cost of mistakes made due to inexperience
\end{itemize}

%(END_ANSWER)





%(BEGIN_NOTES)

\begin{itemize}
\item{} Cost of advertising the position
\item{} Cost of performing interviews
\item{} Cost of background check (in some cases)
\item{} Cost of pay and benefits during non-productive initiation time
\item{} Cost of mistakes made due to inexperience
\item{} Cost of tools (in some cases)
\item{} Cost of uniform (in some cases)
\item{} Cost of on-the-job training (in some cases)
\end{itemize}

Many employers provide tools, uniforms, training, and other benefits with tangible worth to their new employees.  Additionally, there are tangible costs associated with the task of hiring someone new, especially if a background check is required for the position.  Some employers also provide relocation benefits to new hires.  

Given the fact that the complexity of the job renders it unlikely for any new employee to be truly productive (earn the company money) sooner than a few months means those first few months of pay, benefits, training, and other perks are an investment made with the hope of future returns.  In most cases, the employee will have to work for quite a while longer before their benefit to the company offsets the initial investment.

Remember, this is not a job you can master in a matter of days, weeks, or even months.  It will take time before you are truly good at what you do, and the time it takes for you to get there is a cost borne by your first employer.

Relocation benefits offered by some employers are distributed over time.  In other words, you don't get the full financial assistance up-front, but rather are reimbursed over time as you continue to work for that employer.  A common policy is to reimburse in full only after one year of continuous employment: a fact underscoring my advice to stay at your first job for a year or two.

\vskip 10pt

An interesting exception to this is contract work, where employer investment in new hires is minimal, and where transitory employment is the norm.  It may be difficult, in fact, to find a contract employer who is able to guarantee steady work in one geographic area for a two-year period.
 
%INDEX% Career, first job: tenure

%(END_NOTES)


