
%(BEGIN_QUESTION)
% Copyright 2006, Tony R. Kuphaldt, released under the Creative Commons Attribution License (v 1.0)
% This means you may do almost anything with this work of mine, so long as you give me proper credit

Suppose we are measuring the flow rate of a weak acid solution using a magnetic flowmeter.  The conductivity of the acid is well within the acceptable range for this meter, and so it works just fine.

Now suppose the acid solution grows in strength (greater acid concentration).  This will increase the conductivity of the solution, because there are now more ions available to carry an electric current.  What effect will this have on the magnetic flowmeter's calibration?  Will someone have to re-calibrate the flowmeter in order for it to properly measure the acid flow again?  If so, will this be a zero or a span shift?  Which way will the zero and/or span shift, higher or lower?  Explain your answer(s)!

\underbar{file i00729}
%(END_QUESTION)





%(BEGIN_ANSWER)

There is negligible effect on the flowmeter's calibration with changes in liquid conductivity.

\vskip 10pt

A common misunderstanding with magnetic flowmeters is the relationship between liquid conductivity and magnetic flowmeter calibration.  So long as the conductivity stays within the acceptable range for the meter, changes in conductivity have negligible effect on calibration.  The flowmeter's voltage-measuring circuitry has such vastly greater impedance than the electrical path through the liquid, that any changes in liquid conductivity are ``swamped'' by the much greater input impedance of the meter.

%(END_ANSWER)





%(BEGIN_NOTES)


%INDEX% Measurement, flow: magnetic

%(END_NOTES)


