
%(BEGIN_QUESTION)
% Copyright 2006, Tony R. Kuphaldt, released under the Creative Commons Attribution License (v 1.0)
% This means you may do almost anything with this work of mine, so long as you give me proper credit

Suppose we are measuring the flow rate of a gas using a turbine flowmeter.  This is a simple turbine flowmeter, with one turbine spinning freely, generating electronic pulses via a ``pick-up'' coil sensing the passing of turbine blades.  

\vskip 10pt

If the density of this gas suddenly increases with no change in volumetric flow, will the turbine speed increase, decrease, or stay the same?

\underbar{file i00730}
%(END_QUESTION)





%(BEGIN_ANSWER)


%(END_ANSWER)





%(BEGIN_NOTES)

The turbine will keep spinning at the same speed, being a direct function of fluid {\it velocity} past the turbine, and not fluid density.

\vskip 10pt

The only time fluid density will make a difference for a turbine is if the turbine cannot spin freely.  Fluid density directly impacts fluid momentum, which will affect the {\it impulse} of the fluid striking a resisting turbine blade.  For normal turbine meter operation, however, the turbine spins very freely and so it does not ``resist'' the fluid passing by at all.  Thus, density does not come into play.

A notable exception to this rule is the {\it twin-turbine} mass flowmeter design, which intentionally pits two turbines against one another by giving them different blade pitches, and coupling them through a spring to the same shaft.  Here, differential torque is used as an indicator of fluid momentum, which is a direct function of {\it mass} flow rate.

%INDEX% Measurement, flow: turbine

%(END_NOTES)


