
%(BEGIN_QUESTION)
% Copyright 2012, Tony R. Kuphaldt, released under the Creative Commons Attribution License (v 1.0)
% This means you may do almost anything with this work of mine, so long as you give me proper credit

Two flow-indicating instruments employ a common orifice plate to measure the flow of water through a pipe.  The full differential pressure generated by this orifice plate at its rated flow of 800 GPM is 120 inches water column (120 "WC):

\vskip 10pt

\begin{itemize}
\item{} Receiver gauge (3-15 PSI input) connected to the output of a pneumatic DP transmitter connected across the orifice, registering 385 GPM on a 0-800 GPM square-root scale
\vskip 10pt
\item{} Panel-mounted indicator (3-15 PSI) connected to the output of the same pneumatic DP transmitter, registering 403 GPM on a 0-800 GPM square-root scale
\end{itemize}

\vskip 10pt

Based on this information, where do you think the calibration error is located?  If there isn't enough information yet to pinpoint the location of the error, devise a test to reveal where the error is.

\underbar{file i00733}
%(END_QUESTION)





%(BEGIN_ANSWER)

We cannot tell exactly where the problem is, but we know it must be either in the receiver gauge or in the panel-mounted indicator (assuming only one fault in the system).

\vskip 10pt

One test would be to block and equalize the DP transmitter's manifold, to see which indicator goes closest to zero.  Chances are, the error is (at least) a zero shift, and as such should reveal itself in this test.  Whichever indicator goes exactly to zero during this test is good; whichever one reads some non-zero value during this test is in error.

\vskip 10pt

Another test would be to use a pressure gauge to measure the 3-15 PSI pneumatic signal coming from the transmitter.  If the pressure is 5.78 PSI, the receiver gauge is good and the panel-mounted indicator must be in error.  If the pressure is 6.05 PSI, the receiver gauge is in error and the panel-mounted indicator is good.

%(END_ANSWER)





%(BEGIN_NOTES)


%INDEX% Calibration errors, identifying

%(END_NOTES)


