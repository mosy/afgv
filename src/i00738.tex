
%(BEGIN_QUESTION)
% Copyright 2006, Tony R. Kuphaldt, released under the Creative Commons Attribution License (v 1.0)
% This means you may do almost anything with this work of mine, so long as you give me proper credit

Suppose an interviewer asks you a technical question that you have no idea how to answer.  Perhaps the anxiety of the moment makes it too difficult for you to recall the answer, or perhaps you {\it never} knew the answer to this question.  Either way, you are stumped.  Identify a good way to respond to this scenario, and explain why it is preferable to some alternatives.

\vskip 50pt

\underbar{file i00738}
%(END_QUESTION)





%(BEGIN_ANSWER)

Honesty is always the best policy: say ``I don't know.''  However, you probably do not want to end on that note.  You can redeem yourself by explaining how you would begin to work through a solution, or where you would go to find an answer to that question.  It is also appropriate to tell your interviewer(s) that you can get back to them later with an answer.  If they accept, and you do get back with a correct answer, it will demonstrate perseverance and the ability to learn new things.

%(END_ANSWER)





%(BEGIN_NOTES)

It should go without saying that any time you promise an employer or potential employer to get back to them on some issue, you had better follow through with it.  {\it Not} following through is worse than simply leaving your initial answer at ``I don't know.''

%INDEX% Career, interviewing: ``I don't know''

%(END_NOTES)


