
%(BEGIN_QUESTION)
% Copyright 2006, Tony R. Kuphaldt, released under the Creative Commons Attribution License (v 1.0)
% This means you may do almost anything with this work of mine, so long as you give me proper credit

It is not unheard of for an interviewer to ask the interviewee an impossible question.  This may be a technical question that has no solution, or a soft-skill sort of question that has no (good) solution.  Why would an interviewer do this, and what do you think your best response would be?

\vskip 50pt

\underbar{file i00739}
%(END_QUESTION)





%(BEGIN_ANSWER)

Impossible questions are not asked for the sake of obtaining a concrete answer.  Rather, they are asked with the intent to expose the interviewee's thought processes.  In other words, they test the interviewee's {\it problem-solving} skills, assumptions, biases, and (in some cases) values.

%(END_ANSWER)





%(BEGIN_NOTES)

The best way to answer an impossible question?  Easy: give them what they want to see.  Show them your problem-solving ability by narrating your thoughts as you attempt a solution.

%INDEX% Career, interviewing: impossible questions

%(END_NOTES)


