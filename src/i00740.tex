
%(BEGIN_QUESTION)
% Copyright 2010, Tony R. Kuphaldt, released under the Creative Commons Attribution License (v 1.0)
% This means you may do almost anything with this work of mine, so long as you give me proper credit

A DDC (Direct Digital Control) system used for building automation sends a 4-20 mA control signal to a steam valve with an electronic positioner.  This particular loop has a problem, for the valve remains in the full-closed (0\%) position regardless of what the DDC tries to tell it to do.  A technician begins diagnosing the problem by taking a DC voltage measurement at terminal block TB-10 in this loop circuit:

$$\includegraphics[width=15.5cm]{i00740x01.eps}$$

The technician knows a reading of 0 volts could indicate either an ``open'' fault or a ``shorted'' fault in the wiring.  Based on the location of the measured voltage (0.00 VDC), determine where in the wiring a single ``open'' fault would be located (if that is the culprit), and also where in the wiring a ``short'' fault would be located (if that is the culprit).

\vskip 10pt

For the next diagnostic test, the technician disconnects the red wire of cable 41 where it attaches to the screw terminal on TB-10, and re-measures voltage at TB-10.  After disconnecting the wire, the new voltage measurement at TB-10 reads 24.9 volts.  Determine what this result tells us about the nature and location of the fault.

\vskip 20pt \vbox{\hrule \hbox{\strut \vrule{} {\bf Suggestions for Socratic discussion} \vrule} \hrule}

\begin{itemize}
\item{} Explain why it is critically important to determine the identities of the valve and DDC card as being either electrical {\it sources} or electrical {\it loads} when interpreting the diagnostic voltage measurements.
\item{} Identify some of the pros and cons of this style of testing (measuring voltage at a set of points before and after a purposeful wiring break) compared to other forms of multimeter testing when looking for either an ``open'' or a ``shorted'' wiring fault.
\item{} Identify a fault other than open or shorted cables which could account for all the symptoms and measurements we see in this troubleshooting scenario.
\end{itemize}


\underbar{file i00740}
%(END_QUESTION)





%(BEGIN_ANSWER)

Based on the first measurement (only), we could conclude the wiring fault {\it may} be an ``open'' in cable 26 or cable 30, or a ``short'' in {\it any} cable. 

%(END_ANSWER)





%(BEGIN_NOTES)

After taking the second measurement, we must conclude the fault is a ``short'' (not an ``open''), and that it lies somewhere between TB-10 and the control valve (most likely in cable 41).

%INDEX% Troubleshooting review: electric circuits

%(END_NOTES)


