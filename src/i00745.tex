
%(BEGIN_QUESTION)
% Copyright 2006, Tony R. Kuphaldt, released under the Creative Commons Attribution License (v 1.0)
% This means you may do almost anything with this work of mine, so long as you give me proper credit

A challenging question sometimes encountered in interviews goes along the lines of this:

\vskip 10pt {\narrower \noindent \baselineskip5pt

{\it ``Tell me about an incident on the job where you made a mistake, and also describe what you did to correct it.''}

\par} \vskip 10pt

A common mistake many inexperienced interviewees make is to not refer to an actual experience that took place in their lives when answering a question like this.  Instead, interviewees often answer such questions in the hypothetical, telling the interviewer what they {\it might} do {\it if} something like this {\it were to happen} to them.

Explain why it is important to answer questions like these with {\it real life experiences} and not hypothetically.  Specifically, how you would answer this question about making mistakes, and what positive attribute(s) would be revealed about yourself in your answer?

\vskip 50pt

\underbar{file i00745}
%(END_QUESTION)





%(BEGIN_ANSWER)

I'll let you discuss this with your classmates!

%(END_ANSWER)





%(BEGIN_NOTES)

Again, honesty is the best policy: do not fabricate a scenario to appease the interviewer.  Instead, relate a true experience that highlights your ability to self-correct and improve.  Your anecdote should end on a positive note, illustrating your desire and capacity for professional growth.

%INDEX% Career, interviewing: questions about past mistakes

%(END_NOTES)


