
%(BEGIN_QUESTION)
% Copyright 2010, Tony R. Kuphaldt, released under the Creative Commons Attribution License (v 1.0)
% This means you may do almost anything with this work of mine, so long as you give me proper credit

Read and outline the ``Industrial Applications of Chromatographs'' subsection of the ``Chromatography'' section in the ``Continuous Analytical Measurement'' chapter in your {\it Lessons In Industrial Instrumentation} textbook.  Note the page numbers where important illustrations, photographs, equations, tables, and other relevant details are found.  Prepare to thoughtfully discuss with your instructor and classmates the concepts and examples explored in this reading.

\underbar{file i00777}
%(END_QUESTION)





%(BEGIN_ANSWER)


%(END_ANSWER)





%(BEGIN_NOTES)

Chromatographs are multi-variable instruments, and so are often equipped with multiple 4-20 mA analog output channels and/or digital communication network ability.

\vskip 10pt

The computational power inherent in a GC makes it suitable for higher-level analysis, such as gasoline octane value, or natural gas heating value.  In the case of natural gas heating value, this figure is used to correct the pricing per SCF of gas, so that customers are paying for the true energy content of the gas and not just volume or even mass.

\vskip 10pt

GCs may be used to analyze liquid samples, if those liquid samples are first boiled into vapor before entering the heated GC column.

\vskip 10pt

Gas chromatograph columns are very long, skinny tubes.  The material used to ``pack'' a column is chosen based on its ability to retard the diffusion of certain chemical species, and is a choice best left to GC specialists.








\vskip 20pt \vbox{\hrule \hbox{\strut \vrule{} {\bf Suggestions for Socratic discussion} \vrule} \hrule}

\begin{itemize}
\item{} Describe at least one industrial application for an automated chromatograph, explaining why chromatographs are particularly well-suited for that application.
\item{} Can you design an analyzer to measure the heating value of natural gas, that's simpler than a GC?
\item{} One GC application highlighted in the textbook is the determination of natural gas heating value.  An alternative to chromatography in this application is a simple thermal instrument whereby a continuously regulated flow of natural gas is burnt and the resulting heat is measured.  Such an instrument would be much simpler in construction and calibration than a gas chromatograph!  Explain then why a natural gas handling company would go to all the trouble of purchasing and maintaining a GC when a much simpler instrument could do that one task better.
\end{itemize}














\vfil \eject

\noindent
{\bf Summary Quiz}

What will happen if the sample valve in a chromatograph stays in the ``sample'' position just a little bit longer than usual at the beginning of an analysis cycle?

\begin{itemize}
\item{} A larger-than-normal sample quantity will enter the column.
\vskip 5pt
\item{} The chromatograph will register too little quantity for each component. 
\vskip 5pt
\item{} The flow of carrier gas to the column will be partially blocked.
\vskip 5pt
\item{} A smaller-than-normal sample quantity will enter the column.
\vskip 5pt
\item{} Nothing -- the chromatograph will continue to function normally.
\end{itemize}

%INDEX% Reading assignment: Lessons In Industrial Instrumentation, Analytical (chromatography -- industrial applications of)

%(END_NOTES)

