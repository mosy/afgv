
%(BEGIN_QUESTION)
% Copyright 2014, Tony R. Kuphaldt, released under the Creative Commons Attribution License (v 1.0)
% This means you may do almost anything with this work of mine, so long as you give me proper credit

Read and outline the ``Chromatograph Sample Valves'' and ``Improving Chromatograph Analysis Time'' subsections of the ``Chromatography'' section in the ``Continuous Analytical Measurement'' chapter in your {\it Lessons In Industrial Instrumentation} textbook.  Note the page numbers where important illustrations, photographs, equations, tables, and other relevant details are found.  Prepare to thoughtfully discuss with your instructor and classmates the concepts and examples explored in this reading.

\underbar{file i00778}
%(END_QUESTION)





%(BEGIN_ANSWER)


%(END_ANSWER)





%(BEGIN_NOTES)

Sample valves direct a continuous flow of sample to waste.  In the ``loading'' position, this sample fills a small section of tubing called a {\it sample loop}.  In the ``sampling'' position, the valve switches so that carrier gas flushes out this sample loop and empties its contents into the column where it will be separated.  The known interior volume of the sample loop provides a constant volume of sample for each cycle of the GC, regardless of the sample valve's precision of timing.

\vskip 10pt

When analyzing mixtures with widely disparate species elution times, the total analysis time is fixed by the slowest-moving species.  In order to speed up the analysis, GCs use some tricks.  One such trick is to alter the temperature of the column over time to artificially ``rig the race'' and speed up those slower-moving molecules.  Another trick is to use multiple columns to give the slower-moving molecules a shorter ``race'' to run.  Timing of the different valves is key in a multi-column GC, with the switching times based on known species elution times.






\vskip 20pt \vbox{\hrule \hbox{\strut \vrule{} {\bf Suggestions for Socratic discussion} \vrule} \hrule}

\begin{itemize}
\item{} {\bf In what ways may a gas chromatograph be ``fooled'' to report a false chemical concentration measurement?}
\item{} Identify how to increase the sample volume in a GC.
\item{} Explain what would happen if the sample quantity injected into a GC column suddenly increased beyond its usual amount, and no other parameters in the GC were adjusted accordingly.
\item{} Explain how the unique design of a GC sample valve and sample loop guarantee precise metering of the injected quantity even without precise sample valve timing.
\item{} Explain why it is necessary for all GC sample valves to send sample to ``waste'' during the loading portion of the valve's cycle.
\item{} Explain why chromatographs tend to be so slow in their analyses.
\item{} Explain how {\it temperature programming} works for gas chromatographs.
\item{} Explain step-by-step how the dual-column GC works as shown in the textbook.
\item{} Explain how a GC chromatogram would be affected by an alteration in sample tube volume (either increase or decrease).
\item{} Explain how to increase the injected sample volume in a gas chromatograph.  Specifically, what would have to be altered?
\item{} Identify what might happen in a dual-column GC if the bypass valve is not programmed to switch states at the proper times in the analysis cycle.
\end{itemize}


%INDEX% Reading assignment: Lessons In Industrial Instrumentation, Analytical (chromatography -- sample valves)
%INDEX% Reading assignment: Lessons In Industrial Instrumentation, Analytical (chromatography -- improving analysis time)

%(END_NOTES)

