
%(BEGIN_QUESTION)
% Copyright 2006, Tony R. Kuphaldt, released under the Creative Commons Attribution License (v 1.0)
% This means you may do almost anything with this work of mine, so long as you give me proper credit

Explain the operating principles of these valve actuator types, then answer the questions that follow:

\begin{itemize}
\item{}Spring-and-diaphragm (pneumatic)
\item{}Piston (pneumatic)
\item{}Electromechanical (electric motor)
%\item{}Electromagnetic (electric solenoid)
%\item{}Electrohydraulic
\end{itemize}

\vskip 10pt

Excluding the solenoid style of valve actuator, which is not used for proportioning control, which actuator type listed here is the simplest?

\vskip 10pt

Which of the actuators listed here inherently hold their position under a loss of electric power or compressed air supply?

\vskip 10pt

Which of the actuators listed here inherently ``fail'' in one direction (either full-open or full-closed) under a loss-of-supply condition, or else may be modified to do so?

\underbar{file i00781}
%(END_QUESTION)





%(BEGIN_ANSWER)

Spring-and-diaphragm pneumatic actuators work by allowing pressurized air to push against a large, flexible diaphragm.  The force created by the air pressure against the diaphragm's surface area works in opposition to the force produced by a compressed spring.  The spring's compression must change to balance out new levels of air pressure, and this results in a change in position which is then used to move a valve mechanism.

\vskip 10pt

Pneumatic piston actuators use a piston instead of a flexible diaphragm for compressed air to push against.  Some piston actuators have springs to provide a ``return'' force (air pushes the piston one way, and the spring pushes it back), while other piston actuators are double-acting (compressed air on both sides of the piston).

\vskip 10pt

Electromechanical actuators use an electric motor coupled to the valve stem or shaft via a gear mechanism.  As the motor turns, the valve changes position.  Often used for on-off control of valves (from full-open to full-closed and back), electric actuators may also be equipped with position sensors and servo drive electronics for proportioning control.

\vskip 10pt

Solenoid (electromagnetic) valve actuators are simple open-closed devices, not used for throttling.  They utilize an electromagnet coil to magnetically attract an armature, sometimes in opposition to a spring force.

\vskip 10pt

Electrohydraulic actuators usually use an electric motor to turn a hydraulic pump, the fluid output of which moves a piston actuator.

\vskip 10pt

Electromechanical actuators (where the valve is moved by the turning of an electric motor) inherently hold their positions under a loss-of-supply condition, as well as electrohydraulic actuators (where the valve is moved by a piston actuated by hydraulic pressure from a pump turned by an electric motor).

\vskip 10pt

Any actuator with a return spring will inherently ``fail'' to a consistent position under a loss-of-supply condition.  This includes pneumatic diaphragm and piston actuators, and solenoid actuators.

%(END_ANSWER)





%(BEGIN_NOTES)

Pneumatic spring-and-diaphragm actuators tend to be much simpler than any sort of actuator using electric motors.  Pneumatic piston actuators often require a special device called a {\it positioner} to convert an air pressure signal into a definite piston position, because of friction effects of the piston inside the cylinder.  With a diaphragm, however, actuator motion is a direct function of applied air pressure.

This is not to say that pneumatic piston actuators {\it cannot} function without a positioner.  However, the fact that many cannot places the diaphragm class of pneumatic actuator above the piston in terms of design simplicity.

\vskip 10pt

Pneumatic actuators, if of the spring-return variety, tend to move to one extreme position (``fail-closed'' or ``fail-open'') if air supply is lost.  Double-acting piston actuators, though, will stop moving under a loss of air pressure, but they may not rigidly hold their positions against force from the valve trim.  Spring-return actuators may be modified for ``hold-position'' behavior with the addition of a ``locking valve'' that isolates the diaphragm chamber when supply air pressure is lost.

\vskip 10pt

Electrohydraulic actuators may be modified to fail in a consistent direction if power is lost, by providing a return spring in the hydraulic cylinder and a fail-open solenoid valve to equalize the cylinder under loss of power.

%INDEX% Final Control Elements, valve: actuator types

%(END_NOTES)


