
%(BEGIN_QUESTION)
% Copyright 2010, Tony R. Kuphaldt, released under the Creative Commons Attribution License (v 1.0)
% This means you may do almost anything with this work of mine, so long as you give me proper credit

Read the National Transportation Safety Board's pipeline accident brief (DCA-97-FP-002) of the 1996 diesel pipeline rupture in Murfreesboro, Tennessee, and answer the following questions.

\vskip 10pt

Identify the physical cause(s) of the pipeline rupture: what exactly caused the pipeline pressure to rise to the point where it ruptured the line?

\vskip 10pt

One of the factors contributing to this accident was an incorrect schematic on the SCADA system display screen for the Murfreesboro pumping station.  Describe what was incorrect about this schematic, and how this error mis-led the human operator monitoring it.

\vskip 10pt

Why did the electric block valve at the Murfreesboro station refuse to open when the operator commanded it to open at 9:35:02 AM?  Based on your knowledge of different control valve types, explain why a valve such as this would be affected in this manner.

\vskip 10pt

A technician called to inspect the electric block valve and its associated control equipment at the Murfreesboro pumping station reported there ``were no problems.''  What do you think of this assessment, given what you know about the valve and why it was not opening?  Did the technician thoroughly check the valve, in your estimation?

\vskip 20pt \vbox{\hrule \hbox{\strut \vrule{} {\bf Suggestions for Socratic discussion} \vrule} \hrule}

\begin{itemize}
\item{} Page 3 of the report mentions a ``depressurization wave'' taking 74 seconds to travel through the pipeline from the rupture site to Coalmont.  What do you think a ``depressurization wave'' is, and how does this relate to Pascal's Principle?
\item{} Identify ways an instrument technician could have helped to prevent this pipeline rupture.
\end{itemize}

\underbar{file i00785}
%(END_QUESTION)





%(BEGIN_ANSWER)


%(END_ANSWER)





%(BEGIN_NOTES)

On November 5 1996, Colonial pipeline company was preparing to ``pig'' a diesel fuel pipeline.  Pumps at the Atlanta facility (upstream) were shut off, and then a block valve at Murfreesboro (downstream) was shut.  The pigging was to be done on the pipeline between Murfreesboro and a further downstream station in Nashville.  (Page 1)

\vskip 10pt

Shortly after the line block valve had been shut in Murfreesboro, the plan was changed and pumps restarted upstream of the blocked valve (contrary to Colonial's procedures).  This caused the pipeline between the pumping station (in Coalmont, the first station upstream of Murfreesboro) and Murfreesboro to overpressure.  This excessive pressure was registered at the pumping station in Coalmont, but improper overpressure trip points in the control equipment allowed the pipeline pressure to exceed safe limits.  (Page 2)

The pressure transmitter at Murfreesboro did not register this condition because it was installed downstream of the shut line block valve.  Unfortunately, the SCADA system graphic showed this transmitter installed on the upstream side of the valve, as it was at most other stations operating on this line.  (Page 3)  Stated more accurately, the difference between the Murfreesboro station and other stations was actually the location of the electric line block valve rather than the transmitter.  In most stations, flow goes through a manual block valve, then past the transmitter, then through a check valve, then through the electric block valve.  At the Murfreesboro station, flow first encounters the electric block valve, then goes past the transmitter, then through the check valve, then through the manual block valve.  (Page 4)

\vskip 10pt

Sixteen minutes after starting the pumps into the blocked line, the operator first attempted to open the shut block valve at Murfreesboro (without shutting off the pumps, again contrary to Colonial's procedures).  This valve did not open because it had too much differential pressure across it (1700 PSI, when the actuator was only rated to overcome 1440 PSI).  The operator called a technician at Murfreesboro to check this valve, but the technician reported it was okay.  (Page 3)

\vskip 10pt

Four minutes later the controller shut off some of the pumps and diverted flow at another station to lower pipeline pressure so that the Murfreesboro valve could open.  It finally did on the fourth attempt, but this approximately three minutes after the pipeline had already ruptured due to overpressuring.  A sudden drop in pressure ($\Delta P$ = 416 PSI) registered at the Coalmont station following the rupture, but no alarms were generated by the SCADA system.  The estimated pipeline pressure at the ruptured weld was 1820 PSI.  The ``Maximum Operating Pressure'' (MOP) for this pipeline was 1318 PSI.

The operator continued to pump diesel fuel into the ruptured pipeline for approximately an hour before realizing there was a problem, failing to see the expected pressure rise after all that time.
















\vfil \eject

\noindent
{\bf Prep Quiz:}

One of the factors contributing to this accident was an incorrect schematic on the SCADA system display screen for the Murfreesboro pumping station.  Describe what was incorrect about this schematic, and how this error mis-led the human operator monitoring it.

\begin{itemize}
\item{} The pressure was incorrectly displayed in PSI, rather than in units of ``bar''
\vskip 5pt 
\item{} The pressure transmitter was incorrectly shown upstream of the block valve
\vskip 5pt 
\item{} The composition display incorrectly showed gasoline instead of diesel in the line
\vskip 5pt 
\item{} The block valve was incorrectly shown to be air-to-open, instead of air-to-close
\vskip 5pt 
\item{} The pump was incorrectly shown pumping in the wrong direction
\vskip 5pt 
\item{} The block valve was incorrectly shown to be air-to-close, instead of air-to-open
\end{itemize}



%INDEX% Reading assignment: NTSB report on the 1996 diesel pipeline rupture in Murfreesboro, Tennessee

%(END_NOTES)


