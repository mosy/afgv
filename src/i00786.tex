
%(BEGIN_QUESTION)
% Copyright 2006, Tony R. Kuphaldt, released under the Creative Commons Attribution License (v 1.0)
% This means you may do almost anything with this work of mine, so long as you give me proper credit

Like all mechanical devices, control valves experience some mechanical friction in their movement.  What effect will it have on a process if the control valve experiences significant friction?

\vskip 10pt

Hint: if you are having difficulty visualizing the effects of valve friction on {\it control system performance}, imagine trying to drive a car with a lot of friction in the steering mechanism (making it very difficult to turn the steering wheel to new positions).

\vskip 20pt \vbox{\hrule \hbox{\strut \vrule{} {\bf Suggestions for Socratic discussion} \vrule} \hrule}

\begin{itemize}
\item{} Explain how you could empirically determine the amount of friction on the stem of a sliding-stem valve without disassembling the valve in any way.
\item{} Note the suggested use of an {\it analogy} to help understand the effects of valve friction on control performance.  While often helpful in explaining concepts, reasoning by analogy may lead to misunderstandings.  Explain why this is so, and identify what situations analogies may be trusted for conceptual comprehension.
\end{itemize}

\underbar{file i00786}
%(END_QUESTION)





%(BEGIN_ANSWER)

Excessive valve friction leads to unstable control.  If you think this answer is minimal, you're right!  I want you to think deeply about this and present your own thoughts on {\it how} and {\it why} valve friction leads to instability.

%(END_ANSWER)





%(BEGIN_NOTES)

Friction in a control valve is detrimental to stable control, because it makes it more difficult to quickly and precisely position the valve.  Friction, in dissipating mechanical energy, makes valve motion slower.  It also impedes precise positioning, especially when pneumatic actuators are used.

\vskip 10pt

Valve friction may be empirically measured by applying variable air pressure to the actuating diaphragm and noting the difference in applied pressure necessary to make the valve move open versus making it move closed.  That difference in pressure ($\Delta P$), when multiplied by the diaphragm area will yield a difference in applied force $\Delta F = (\Delta P) (A)$.  This $\Delta F$ value represents the static friction for upward movement {\it plus} the static friction for downward movement.  If we assume the two friction values to be equal to each other, then the stem's static friction value will be $\Delta F \div 2$.

\vskip 10pt

In answer to the Socratic question, reasoning by analogy may be logically unsound if the analogy is inappropriate (in accurate for the application).  Analogies are excellent tools in the hands of an experienced and knowledgeable instructor who knows when and when not to apply them.  Self-made analogies can be misleading for the new student, however, who does not necessarily understand the concept well enough (yet) to know whether or not a chosen analogy may be relied upon as a conceptual guide.  To sum this up in a phrase, {\it analogies are great for explaining things, but they prove nothing.}

\vfil \eject

\noindent
{\bf Summary Quiz:}

Practical methods to minimize control valve {\it stiction} include:

\begin{itemize}
\item{} Maintaining a minimum bench set pressure at 0\% position
\vskip 5pt 
\item{} Proper calibration of the I/P signal transducer
\vskip 5pt 
\item{} Leaving the loop controller in manual mode whenever possible
\vskip 5pt 
\item{} Increasing the instrument supply air pressure to the I/P
\vskip 5pt 
\item{} Periodic lubrication of the packing with grease
\vskip 5pt 
\item{} Selecting the best graphite-based packing for the application
\end{itemize}


%INDEX% Final Control Elements, valve: stiction

%(END_NOTES)


