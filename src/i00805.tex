
%(BEGIN_QUESTION)
% Copyright 2006, Tony R. Kuphaldt, released under the Creative Commons Attribution License (v 1.0)
% This means you may do almost anything with this work of mine, so long as you give me proper credit

A student needs to convert a temperature value from degrees Fahrenheit to degrees Celsius.  Unfortunately, they are not sure of the correct formula.  They think the formula goes like this, but they are unsure:

$$^{o}C = \left({5 \over 9}\right) {^{o}F} - 32$$

Devise a simple way for this student to {\it test} whether or not the formula is correct, without seeking help from a reference document of any kind.

\vskip 20pt \vbox{\hrule \hbox{\strut \vrule{} {\bf Suggestions for Socratic discussion} \vrule} \hrule}

\begin{itemize}
\item{} Why do you suppose it is helpful to develop the habit and skill of ``testing'' hypotheses in this way, as opposed to simply consulting a reference to see if the formula is written correctly?
\end{itemize}

\underbar{file i00805}
%(END_QUESTION)





%(BEGIN_ANSWER)

If the student can remember the freezing and/or boiling points of water in both degrees F and degrees C, it is a trivial matter to test the formula for correctness!

%(END_ANSWER)





%(BEGIN_NOTES)

Time after time I have encountered students who cannot remember the conversion formulae (for sure) and come to me (the instructor) asking to verify what they think they remember.  The simple test of 0$^{o}$ C = 32$^{o}$ F or 100$^{o}$C = 212$^{o}$ F never occurs to them, and even after I mention it there is usually hesitance to invoke it.  Yet it is precisely this sort of ``experimental'' attitude that is crucial to solving complex problems.

%INDEX% Physics, units and conversions: temperature

%(END_NOTES)


