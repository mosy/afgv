
%(BEGIN_QUESTION)
% Copyright 2006, Tony R. Kuphaldt, released under the Creative Commons Attribution License (v 1.0)
% This means you may do almost anything with this work of mine, so long as you give me proper credit

One of the potential problems with using an RTD to measure temperature is something called {\it self-heating}.  This problem affects all temperature-sensing elements that are externally powered.  By contrast, thermocouples do not suffer this problem.  Explain what the problem of self-heating is, how it may be mitigated for RTDs, and why thermocouples do not suffer from it.

\underbar{file i00806}
%(END_QUESTION)





%(BEGIN_ANSWER)

The root of the answer is expressed in this equation:

$$P = I^2 R$$

Power dissipated in the RTD by its own resistance leads to increased temperature.  We may mitigate the self-heating of RTDs and thermistors by limiting the excitation current to a bare minimum.  The problem with doing this is that it necessitates greater electronic amplification, and with that you have the problem of greater noise.  One solution to this problem is to {\it pulse} power to the RTD, so that it spends most of its time unpowered.  This means we cannot measure temperature between the sample times, but that is not a problem in most cases because temperature is not typically a fast-changing process variable.

%(END_ANSWER)





%(BEGIN_NOTES)


%INDEX% Measurement, temperature: RTD

%(END_NOTES)


