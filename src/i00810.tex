
%(BEGIN_QUESTION)
% Copyright 2012, Tony R. Kuphaldt, released under the Creative Commons Attribution License (v 1.0)
% This means you may do almost anything with this work of mine, so long as you give me proper credit

100 ohm industrial RTDs are constructed from platinum wire.  The cables used to connect platinum RTD sensing elements to remote temperature indicators or transmitters are invariably made of copper wire.  When these copper conductors attach to the RTD, they form dissimilar-metal junctions with the platinum metal of the RTD sensor.

\vskip 10pt

Despite the existence of these dissimilar-metal junctions (copper-to-platinum), we never have to use ``reference junction compensation'' in RTD measurement systems or use special extension wire as we do in thermocouple circuits.  Explain why the dissimilar-metal junctions in an RTD circuit are of no consequence.

\underbar{file i00810}
%(END_QUESTION)





%(BEGIN_ANSWER)

The two copper-platinum junctions formed when copper wire joins with the platinum wire of an RTD sensor are both at the same location (at the RTD sensor) and therefore share the same temperature.  Their polarities are directly opposed to one another, so their voltages cancel.

\vskip 10pt

Another way of explaining this is to invoke the Law of Intermediate Metals: with the two copper-platinum junctions at the same temperature, the platinum wire of the RTD sensing element becomes an intermediate metal between two copper wires.  According to this Law, we may treat this pair of copper-platinum junctions as equivalent to a single copper-copper junction at that same temperature.  As we know, a similar-metal junction such as copper-copper generates no voltage at any temperature, and so does not pose any measurement problem in RTD circuits the way the reference junction poses a problem in thermocouple circuits.

%(END_ANSWER)





%(BEGIN_NOTES)

{\bf This question is intended for exams only and not worksheets!}.

%(END_NOTES)


