
%(BEGIN_QUESTION)
% Copyright 2014, Tony R. Kuphaldt, released under the Creative Commons Attribution License (v 1.0)
% This means you may do almost anything with this work of mine, so long as you give me proper credit

Read Exercise 5 (``Creating a New Module (LI-101) from Scratch'') in Chapter 4 (``Creating and Downloading the Control Strategy'') of the ``Getting Started with your DeltaV Digital Automation System'' manual (document D800002X122, March 2006) and answer the following questions:

\vskip 10pt

Which function block type is used in the LI-101 module, and what does it do?

\vskip 10pt

The output of the main function block in this module is ``wired'' to another object called an {\it Output Connector}.  What does an ``output connector'' do in the DeltaV system?

\vskip 10pt

One of the configuration steps described has you add ``History Collection'' to the function block.  What exactly does this feature do?

\vskip 10pt

After this module's function block has been ``wired'' to the Output Connector, there are a few final steps required to finish the module.  Identify these steps and explain their purpose.



\vskip 20pt \vbox{\hrule \hbox{\strut \vrule{} {\bf Suggestions for Socratic discussion} \vrule} \hrule}

\begin{itemize}
\item{} Access a DeltaV workstation PC and try opening a module using Control Studio.  Find some of the {\tt AI} block parameters and options discussed in this exercise.  {\it Do not ``download'' or ``save'' anything, which will alter the configuration of the DCS -- just explore and observe!}
\item{} Compare the setting of the process variable's ``engineering units'' to the MINSCALE and MAXSCALE parameters of the {\tt AI} function block in a Siemens 353 loop controller.  How are these tasks similar, and how are they different?
\end{itemize}

\underbar{file i00812}
%(END_QUESTION)





%(BEGIN_ANSWER)


%(END_ANSWER)





%(BEGIN_NOTES)

This module uses an {\tt AI} (analog input) function block, which takes the signal in from the analog input channel and scales it into engineering units.  In this example, the output scale of the {\tt AI} block is 0 to 10000 gallons.

\vskip 10pt

The output connector (page 4-24) provides a point which you can ``wire'' to other function blocks if this module ends up being referenced within another module in the DeltaV system.

\vskip 10pt

The History Collection feature enables the DeltaV system's Continuous Historian subsystem to archive data on this particular tag (the PV value within the {\tt AI} function block) at specified intervals for retrieval at some later time (page 4-25). 

\vskip 10pt

The finishing steps are (listed on page 4-27):

\begin{itemize}
\item{} Set the primary control picture (graphic) to TANK101
\item{} Assign this control module to a particular controller for execution
\item{} Save the control module under the name LI-101
\end{itemize}

%INDEX% Reading assignment: Emerson DeltaV "Getting Started" manual (Chapter 4, Exercise 5)

%(END_NOTES)


