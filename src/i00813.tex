
%(BEGIN_QUESTION)
% Copyright 2015, Tony R. Kuphaldt, released under the Creative Commons Attribution License (v 1.0)
% This means you may do almost anything with this work of mine, so long as you give me proper credit

Read and outline the ``Distributed Control Systems (DCS)'' subsection of the ``Digital PID Controllers''section of the ``Closed-Loop Control'' chapter in your {\it Lessons In Industrial Instrumentation} textbook.  Note the page numbers where important illustrations, photographs, equations, tables, and other relevant details are found.  Prepare to thoughtfully discuss with your instructor and classmates the concepts and examples explored in this reading.

\underbar{file i00813}
%(END_QUESTION)





%(BEGIN_ANSWER)


%(END_ANSWER)





%(BEGIN_NOTES)

A Distributed Control System (DCS) is a collection of dedicated digital loop controllers networked together to share information with each other and display information on operator consoles.  If any one controller fails, only a handful of loops fail.  This stands in contrast with DDC control systems where one large computer controls all the loops in a facility.

\vskip 10pt

Redundancy is a hallmark of a DCS, with redundant network cables, redundant processors, and even redundant I/O cards.

\vskip 10pt

Honeywell's TDC 2000 was the first American DCS, introduced in 1975.  It featured a ``Data Hiway'' communications network and a one-to-eight backup controller ratio.  Yokogawa of Japan released their CENTUM DCS in the same year.  Modern DCS systems typically use conventional PCs as operator stations.

\vskip 10pt

PLCs can be used to do distributed control, but only if the end-user invests a lot of time and resources programming the PLCs with all the code necessary to emulate the out-of-the-box functionality of a DCS.  PLCs are thus initially cheap, but their lack of advanced features makes them expensive to emulate the same functionality.









\vskip 20pt \vbox{\hrule \hbox{\strut \vrule{} {\bf Suggestions for Socratic discussion} \vrule} \hrule}

\begin{itemize}
\item{} A very common question asked by new instrument engineers is, ``What's the difference between a PLC and a DCS?''  Give your answer to this question based on what you have read in this textbook section.
\item{} Explain the difference between a {\it DCS} and a {\it SCADA} (or {\it telemetry}) system.  Feel free to reference previous sections in your textbook to help answer this question.
\item{} Identify a practical application where you would {\it not} want to use a DCS, but instead you might prefer to use a PLC or SCADA system to do the job.  Explain the rationale for your answer.
\item{} The term {\it redundant} is typically used in a pejorative sense, but this is not the case with high-reliability instrumentation such as distributed control systems.  Explain what {\it redundancy} means in this context, and why this is an important feature in a DCS.
\item{} Identify some of the specific ways in which a DCS is {\it redundant}.  Give a practical example whereby one component of a DCS fails but the system as a whole keeps doing its job.
\end{itemize}







\vfil \eject

\noindent
{\bf Prep Quiz:}

The term ``distributed'' in the acronym ``DCS'' refers to what unique feature of this type of control system:

\begin{itemize}
\item{} Multiple control computers handling all tasks as opposed to one central processor unit
\vskip 5pt 
\item{} Digital control messages sent along a broad network reaching the far corners of the facility
\vskip 5pt 
\item{} The packaging and delivery system used to transport DCS components to the end user
\vskip 5pt 
\item{} The ability of this type of control system to control more than one loop in a facility
\vskip 5pt 
\item{} Long lengths of 4-20 mA analog cabling required between the control room and field instruments
\vskip 5pt 
\item{} It uses the exact same type of programming language as most PLCs (i.e. relay ladder logic)
\vskip 5pt 
\item{} Every detail of the control algorithms must be designed and programmed by the end-user
\end{itemize}


%INDEX% Reading assignment: Lessons In Industrial Instrumentation, closed-loop control (basic principles)

%(END_NOTES)


