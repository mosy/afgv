
%(BEGIN_QUESTION)
% Copyright 2014, Tony R. Kuphaldt, released under the Creative Commons Attribution License (v 1.0)
% This means you may do almost anything with this work of mine, so long as you give me proper credit

Read Exercise 6 (``Creating a PID Control Loop (FIC-101) from Scratch'') in Chapter 4 (``Creating and Downloading the Control Strategy'') of the ``Getting Started with your DeltaV Digital Automation System'' manual (document D800002X122, March 2006) and answer the following questions:

\vskip 10pt

In this exercise the user is shown how to begin configuring a control module without starting from scratch as in Exercise 5.  How is this done?

\vskip 10pt

Although you will often find {\tt PID} function blocks ``wired'' to analog input ({\tt AI}) function blocks and analog output ({\tt AO}) function blocks to make a complete working loop module, here in this exercise the {\tt PID} function block stands alone.  In lieu of {\tt AI} and {\tt AO} function blocks to route the signals to and from real-world I/O channels, how does this {\tt PID} function block ``know'' where to get its PV input and where to send its MV output signals?



\vskip 20pt \vbox{\hrule \hbox{\strut \vrule{} {\bf Suggestions for Socratic discussion} \vrule} \hrule}

\begin{itemize}
\item{} Access a DeltaV workstation PC and try opening a module using Control Studio.  Find some of the {\tt PID} block parameters and options discussed in this exercise.  {\it Do not ``download'' or ``save'' anything, which will alter the configuration of the DCS -- just explore and observe!}
\item{} Note where the controller's direction of action (i.e. ``direct'' or ``reverse'' action) is selected in the {\tt PID} function block.  How does one determine the correct direction of control action for any specific process?
\item{} Immediately following the instruction on how to set controller's direction of action (i.e. ``direct'' or ``reverse'' action), this exercise specifies how to set a similar parameter in the IO\_OPTS collection of parameters called {\it Increase to Close}.  This is used when the control valve happens to be air-to-close (fail-open), to make the controller faceplate's output bargraph match the valve stem position (so that a displayed output of 0\% represents a shut valve and a displayed output of 100\% represents a wide-open valve).  Explain why this parameter is an important one to set in processes where the control valve is air-to-close.  One way of explaining the importance of this parameter is to describe what would happen if it were {\it not} set correctly for a particular control loop.  
\item{} Compare the setting of the process variable's ``engineering units'' to the MINSCALE and MAXSCALE parameters of the {\tt AI} function block in a Siemens 353 loop controller.  How are these tasks similar, and how are they different?
\end{itemize}


\underbar{file i00814}
%(END_QUESTION)





%(BEGIN_ANSWER)


%(END_ANSWER)





%(BEGIN_NOTES)

On page 4-27 the instructions have the user copy a template from the DeltaV Library, rename that copied module, and begin editing it from there.

\vskip 10pt

On page 4-28 the reader is instructed to set the {\tt IO\_IN} and {\tt IO\_OUT} parameters on the {\tt PID} function block, telling that block where to get its input (PV) signal from and where to send its output (MV) signal to.

%INDEX% Reading assignment: Emerson DeltaV "Getting Started" manual (Chapter 4, Exercise 6)

%(END_NOTES)


