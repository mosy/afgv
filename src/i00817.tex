
%(BEGIN_QUESTION)
% Copyright 2014, Tony R. Kuphaldt, released under the Creative Commons Attribution License (v 1.0)
% This means you may do almost anything with this work of mine, so long as you give me proper credit

Search through Chapter 5 (``Creating Operator Pictures'') of the ``Getting Started with your DeltaV Digital Automation System'' manual (document D800002X122, March 2006) to answer the following questions:

\vskip 10pt

Identify a fast way to switch between the ``Run'' and ``Configure'' modes of {\it DeltaV Operate}.

\vskip 10pt

This tutorial advises against creating a new ``picture'' file from scratch.  Instead, what is the recommended method for creating a graphical screen for operators to use?

\vskip 10pt

Describe the purpose of the {\it DeltaV\_Toolbox} in {\it DeltaV Operate} (Configure mode), and how this ``toolbox'' may be used to create graphic objects on the screen for operator displays.

\vskip 10pt

Explain what a ``link'' is in {\it DeltaV Operate}.  Specifically, what is a ``datalink'' and what might one be used for?

\vskip 10pt

Explain what a ``parameter reference'' is in {\it DeltaV Operate}.

\vskip 10pt

Explain what a ``dynamo'' is in {\it DeltaV Operate}.

\vskip 10pt

Page 5-30 begins a discussion on Trend Links, which is another form of link supported by {\it DeltaV\_Run}.  In this discussion the tutorial mentions that trend links are able to display any ``floating point'' data (i.e. any parameter reference ending in {\tt .F\_}.  Explain what ``floating point'' data is and how it differs from integer or boolean data.



\vskip 20pt \vbox{\hrule \hbox{\strut \vrule{} {\bf Suggestions for Socratic discussion} \vrule} \hrule}

\begin{itemize}
\item{} Access a DeltaV workstation PC and try opening an operator picture using DeltaV Operate Configure.  Explore the parameters associated with the datalinks and dynamos in this picture.  {\it Do not ``download'' or ``save'' anything, which will alter the configuration of the DCS -- just explore and observe!}
\item{} For those students who have studied HMI programming in their PLC coursework, identify aspects of {\it DeltaV Operate} that are similar to other HMIs you have worked with.
\item{} Identify practical applications where you might wish to show {\it integer} data on an operator display.
\item{} Identify practical applications where you might wish to show {\it boolean} data on an operator display.
\end{itemize}

\underbar{file i00817}
%(END_QUESTION)





%(BEGIN_ANSWER)


%(END_ANSWER)





%(BEGIN_NOTES)

The $<$Ctrl-W$>$ hotkey combination switches between ``Run'' and ``Configure'' modes (page 5-6).  An alternative method to switch from ``Run'' to ``Configure'' mode is to right-click on the display and select ``Quick Edit''.

\vskip 10pt

Rather than create a new picture file from scratch, the preferred method is to copy the ``Main'' picture template file and then modify it to your specifications.  The ``Important'' note on page 5-3 explains this.

\vskip 10pt

The {\it DeltaV\_Toolbox} is a palette of graphical objects that may be selected and placed on the operator picture (page 5-11).  These range from simple geometric figures (e.g. rectangles, circles, triangles) to trend recorder displays and PID controller faceplates.

\vskip 10pt

A ``link'' is any association between an on-screen object and real-time and/or system data in the DeltaV control system (page 5-14).  This tutorial focuses on datalinks and trend links.  A ``datalink'' is a numerical display on the operator screen, such as a process variable readout or setpoint register.  A ``trend link'' provides a real-time trend graph display for operator viewing.

The {\it DeltaV\_Toolbox} button for placing a datalink on the display is the ``Datalink Stamper'' button, marked {\bf ABC100} (page 5-17). 

\vskip 10pt

A ``parameter reference'' is the tagname of the data as it is addressed in the database of the DeltaV system (page 5-14).  All HMI systems have some form of tagname database, whereby real-time and system data is associated with easily-referenced names which may be linked to on-screen graphic objects such as numerical displays and bargraphs.

\vskip 10pt

A ``dynamo'' is a pre-built graphic object in {\it DeltaV Operate}, often resembling a real-life piece of equipment such as a pump or a tank (pages 5-24 to 5-25).  You can create your own dynamos and save them to a library for future use!

\vskip 10pt

Floating-point data is numerical data in which the placement of the decimal point may be moved (or ``floated'') across a wide range.  Floating-point data is typically used to express non-integer numerical data, especially where a high degree of precision is required.


%INDEX% Reading assignment: Emerson DeltaV "Getting Started" manual (Chapter 4, Exercise 9)

%(END_NOTES)


