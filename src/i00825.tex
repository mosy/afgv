
%(BEGIN_QUESTION)
% Copyright 2015, Tony R. Kuphaldt, released under the Creative Commons Attribution License (v 1.0)
% This means you may do almost anything with this work of mine, so long as you give me proper credit

Read selected portions of the ``SEL-RS Rotary Switch'' instruction manual (document SEL-RS Rotary Switch Family Instruction Manual, May 2013) and answer the following questions:

\vskip 10pt

A typical SEL-RS rotary hand switch is comprised of different modules stacked on a common rotary shaft.  Table 1 on page 5 lists four different module types.  Identify what these modules are called and explain what each of them do:

\begin{itemize}
\item{} 
\item{} 
\item{} 
\item{} 
\end{itemize}

\vskip 10pt

Identify what the ``A'' wire terminals are used for in this switch (e.g. A1+, A1$-$, A2+, A2$-$, etc.).

\vskip 10pt

Figure 13 on page 8 shows common connections for the optional {\it tripping module}.  In the right-hand diagram we see a pair of LEDs connected along with the tripping coil.  From this diagram identify the purpose of LED1 and LED3 -- under what condition(s) will they illuminate?

\vskip 10pt

SEL sells three different types of switch contact modules for the SEL-RS86 rotary switch (``lockout'' applications).  Identify these different module types, and explain where each one might be used.

\vskip 10pt

SEL-RS rotary switches are typically used for three major types of application, reflected in the model numbers following ``RS'': SEL-RS86, SEL-RS52, and SEL-RS43.  Identify these applications, and determine what the model numbers refer to.

\vskip 20pt \vbox{\hrule \hbox{\strut \vrule{} {\bf Suggestions for Socratic discussion} \vrule} \hrule}

\begin{itemize}
\item{} This design of panel-mounted switch is called {\it modular} or {\it stackable}.  Explain what these terms refer to, and why this design feature is very useful for industrial control applications.
\item{} A feature available on some models of this switch is a {\it target}.  Explain what a ``target'' is and what purpose it might serve in an electrical protection system.
\end{itemize}

\underbar{file i00825}
%(END_QUESTION)





%(BEGIN_ANSWER)

\noindent
{\bf Partial answer:}

\vskip 10pt

The model numbers refer to the ANSI/LIII code for the application:

\begin{itemize}
\item{} SEL-RS86 = auxiliary lockout relay
\item{} SEL-RS52 = circuit breaker manual trip/close
\item{} SEL-RS43 = manual transfer or selector
\end{itemize}


%(END_ANSWER)





%(BEGIN_NOTES)

Available module types for the SEL-RS family of switches:

\begin{itemize}
\item{} LED module (front panel lights)
\item{} Trip module (tripping solenoid, for 86 lockout applications)
\item{} Mechanism module (front handle)
\item{} Contact module (switch contacts actuated by the rotary shaft)
\end{itemize}

\vskip 10pt

The ``A' terminals connect to anode and cathode sides of each front-panel LED in the LED module.

\vskip 10pt

LED1 illuminates whenever the NC contact is in the closed state (switch in the ``reset'' position) and there is continuity through the tripping coil.  LED3 illuminates whenever some device asserts a ``trip'' signal to the switch's trip solenoid.

\vskip 10pt

The three contact module types available for the SEL-RS86 switch all have four contacts apiece, but the different is in the ``normal'' statuses of those contacts.  One type has two NC contacts and two NO contacts.  Another type has four NO contacts.  The last type has four NC contacts.

\vskip 10pt

The model numbers refer to the ANSI/LIII code for the application:

\begin{itemize}
\item{} SEL-RS86 = auxiliary lockout relay
\item{} SEL-RS52 = circuit breaker manual trip/close
\item{} SEL-RS43 = manual transfer or selector
\end{itemize}


%INDEX% Reading assignment: SEL model RS rotary hand switch

%(END_NOTES)


