
%(BEGIN_QUESTION)
% Copyright 2014, Tony R. Kuphaldt, released under the Creative Commons Attribution License (v 1.0)
% This means you may do almost anything with this work of mine, so long as you give me proper credit

Every protective relay presents a {\it burden} to the instrument transformer(s) driving signal into it.  The greater a relay's burden, the more power is demanded from the instrument transformer and the less accurate the system will tend to be.  For this reason, burden is a very important parameter to consider when wiring a protective relay system.

\vskip 10pt

The datasheet for a General Electric model CEB52 ``distance'' protective relay (ANSI/IEEE code 21) shows the burden for its ``polarizing coil'' PT input to be $1540 - j162$ ohms.  Calculate the current through this coil when the PT's output signal is 103 volts $\angle$ $-6^o$.

\underbar{file i00836}
%(END_QUESTION)





%(BEGIN_ANSWER)

$$I = {V \over Z}$$

$$I = {103 \hbox{ V} \angle -6^o \over 1540 - j162 \> \Omega}$$

$$I = 66.52 \hbox{ mA} \angle 0.0051^o$$

%(END_ANSWER)





%(BEGIN_NOTES)


%(END_NOTES)

