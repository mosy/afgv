
%(BEGIN_QUESTION)
% Copyright 2015, Tony R. Kuphaldt, released under the Creative Commons Attribution License (v 1.0)
% This means you may do almost anything with this work of mine, so long as you give me proper credit

Complex number arithmetic makes possible the analysis of AC circuits using (almost) the exact same Laws that were learned for DC circuit analysis.  The only bad part about this is that doing complex-number arithmetic by hand can be very tedious.  Some calculators, though, are able to add, subtract, multiply, divide, and invert complex quantities as easy as they do scalar quantities, making this method of AC circuit analysis relatively easy.

This question is really a series of practice problems in complex number arithmetic, the purpose being to give you lots of practice using the complex number facilities of your calculator (or to give you a {\it lot} of practice doing trigonometry calculations, if your calculator does not have the ability to manipulate complex numbers!).  

\vskip 10pt

\noindent
{\bf Addition and subtraction:}

\vskip 5pt

\settabs 3 \columns
\+ $(5 + j6) + (2 - j1) =$  &
$(10 - j8) + (4 - j3) =$  &
$(-3 + j0) + (9 - j12) =$ \cr

\vskip 20pt

\+ $(3 + j5) - (0 - j9) =$  &
$(25 - j84) - (4 - j3) =$  &  
$(-1500 + j40) + (299 - j128) =$ \cr

\vskip 20pt

\+ $(25 \angle 15^o) + (10 \angle 74^o) =$  &
$(1000 \angle 43^o) + (1200 \angle -20^o) =$  &  
$(522 \angle 71^o) - (85 \angle 30^o) =$  \cr

\vskip 20pt

\noindent
{\bf Multiplication and division:}

\vskip 5pt

\settabs 3 \columns
\+ $(25 \angle 15^o) \times (12 \angle 10^o) =$  &
$(1 \angle 25^o) \times (500 \angle -30^o) =$  &  
$(522 \angle 71^o) \times (33 \angle 9^o) =$  \cr

\vskip 20pt

\+ ${{10 \angle -80^o} \over {1 \angle 0^o}} =$  &
${{25 \angle 120^o} \over {3.5 \angle -55^o}} =$  &  
${{-66 \angle 67^o} \over {8 \angle -42^o}} =$ \cr

\vskip 20pt

\+ $(3 + j5) \times (2 - j1) =$  &  
$(10 - j8) \times (4 - j3) =$  &  
${(3 + j4) \over (12 - j2)} =$ \cr

\vskip 20pt

\noindent
{\bf Reciprocation:}

\vskip 5pt

\settabs 3 \columns
\+ ${1 \over (15 \angle 60^o)} =$  &
${1 \over (750 \angle -38^o)} =$  &  
${1 \over (10 + j3)} =$  \cr

\vskip 20pt

\+ ${1 \over {1 \over {15 \angle 45^o}} + {1 \over {92 \angle -25^o}}} =$  &
${1 \over {1 \over {1200 \angle 73^o}} + {1 \over {574 \angle 21^o}}} =$  &
${1 \over {1 \over {23k \angle -67^o}} + {1 \over {10k \angle -81^o}}} =$  \cr

\vskip 20pt

\+ ${1 \over {1 \over {110 \angle -34^o}} + {1 \over {80 \angle 19^o}} + {1 \over {70 \angle 10}}} =$  &
${1 \over {1 \over {89k \angle -5^o}} + {1 \over {15k \angle 33^o}} + {1 \over {9.35k \angle 45}}} =$  &
${1 \over {1 \over {512 \angle 34^o}} + {1 \over {1k \angle -25^o}} + {1 \over {942 \angle -20}} + {1 \over {2.2k \angle 44^o}}} =$  \cr


\vskip 20pt \vbox{\hrule \hbox{\strut \vrule{} {\bf Suggestions for Socratic discussion} \vrule} \hrule}

\begin{itemize}
\item{} Your calculator's manual will be an excellent reference for learning how to enter and interpret complex numbers.  Show where in the manual you were able to find instructions on entering complex numbers, displaying them in different forms (e.g. polar vs. rectangular), and performing basic arithmetic operations on complex numbers.
\end{itemize}

\underbar{file i00846}
%(END_QUESTION)





%(BEGIN_ANSWER)

\noindent
{\bf Addition and subtraction:}

\vskip 5pt

\settabs 3 \columns
\+ $(5 + j6) + (2 - j1) =$  &  
$(10 - j8) + (4 - j3) =$  &  
$(-3 + j0) + (9 - j12) =$ \cr
\+ ${\bf 7 + j5}$  &  
${\bf 14 - j11}$  &  
${\bf 6 - j12}$ \cr

\vskip 20pt

\+ $(3 + j5) - (0 - j9) =$  &
$(25 - j84) - (4 - j3) =$  &  
$(-1500 + j40) + (299 - j128) =$ \cr
\+ ${\bf 3 + j14}$  &  
${\bf 21 - j81}$  &  
${\bf -1201 - j88}$ \cr

\vskip 20pt

\+ $(25 \angle 15^o) + (10 \angle 74^o) =$  &
$(1000 \angle 43^o) + (1200 \angle -20^o) =$  &  
$(522 \angle 71^o) - (85 \angle 30^o) =$  \cr
\+ ${\bf 31.35 \angle 30.87^o}$  &  
${\bf 1878.7 \angle 8.311^o}$  &  
${\bf 461.23 \angle 77.94^o}$  \cr  

\vskip 20pt

\noindent
{\bf Multiplication and division:}

\vskip 5pt

\settabs 3 \columns
\+ $(25 \angle 15^o) \times (12 \angle 10^o) =$  &
$(1 \angle 25^o) \times (500 \angle -30^o) =$  &  
$(522 \angle 71^o) \times (33 \angle 9^o) =$  \cr
\+ ${\bf 300 \angle 25^o}$  &  
${\bf 500 \angle -5^o}$  &  
${\bf 17226 \angle 80^o}$  \cr  

\vskip 20pt

\+ ${{10 \angle -80^o} \over {1 \angle 0^o}} =$  &
${{25 \angle 120^o} \over {3.5 \angle -55^o}} =$  &  
${{-66 \angle 67^o} \over {8 \angle -42^o}} =$ \cr
\+ ${\bf 10 \angle -80^o}$  &  
${\bf 7.142 \angle 175^o}$  &  
${\bf 8.25 \angle -71^o}$  \cr  

\vskip 20pt

\+ $(3 + j5) \times (2 - j1) =$  &  
$(10 - j8) \times (4 - j3) =$  &  
${(3 + j4) \over (12 - j2)} =$ \cr
\+ ${\bf 11 + j7}$  &  
${\bf 16 - j62}$  &  
${\bf 0.1892 + j0.3649}$ \cr

\vskip 20pt

\noindent
{\bf Reciprocation:}

\vskip 5pt

\settabs 3 \columns
\+ ${1 \over (15 \angle 60^o)} =$  &
${1 \over (750 \angle -38^o)} =$  &  
${1 \over (10 + j3)} =$  \cr
\+ ${\bf 0.0667 \angle -60^o}$  &  
${\bf 0.00133 \angle 38^o}$  &  
${\bf 0.0917 -j0.0275}$  \cr  

\vskip 20pt

\+ ${1 \over {1 \over {15 \angle 45^o}} + {1 \over {92 \angle -25^o}}} =$  &
${1 \over {1 \over {1200 \angle 73^o}} + {1 \over {574 \angle 21^o}}} =$  &
${1 \over {1 \over {23k \angle -67^o}} + {1 \over {10k \angle -81^o}}} =$  \cr
\+ ${\bf 14.06 \angle 36.74^o}$  &  
${\bf 425.7 \angle 37.23^o}$  &  
${\bf 7.013k \angle -76.77^o}$  \cr 

\vskip 20pt

\+ ${1 \over {1 \over {110 \angle -34^o}} + {1 \over {80 \angle 19^o}} + {1 \over {70 \angle 10}}} =$  &
${1 \over {1 \over {89k \angle -5^o}} + {1 \over {15k \angle 33^o}} + {1 \over {9.35k \angle 45}}} =$  &
${1 \over {1 \over {512 \angle 34^o}} + {1 \over {1k \angle -25^o}} + {1 \over {942 \angle -20}} + {1 \over {2.2k \angle 44^o}}} =$  \cr
\+ ${\bf 29.89 \angle 2.513^o}$  &  
${\bf 5.531k \angle 37.86^o}$  &  
${\bf 256.4 \angle 9.181^o}$  \cr 

\vskip 10pt

\filbreak

Some Texas Instruments brand calculators such as the TI-84 offer an exponential key and imaginary ($i$) key which allows you to enter numbers in complex exponential form (i.e. $e^{i \theta}$).  With the TI-84, for example, the complex number $10 - j8$ may be entered in either of the two following forms:

\vskip 10pt

{\tt (10 - i8)} \hskip 10pt or \hskip 10pt {\tt (10 - 8i)} 

\vskip 10pt

The result may be displayed in either rectangular or polar forms according to the complex-number display {\it mode} the TI-84 calculator has been set to.  In rectangular mode the displayed result for $10 - j8$ will be {\tt 10 - 8i}, whereas in polar mode the displayed result will be {\tt 12.806 e$^{-38.66i}$}.  Note how the TI-84 uses exponential notation for polar display, where the angle ($-38.66$ degrees, in this example) is an imaginary power of $e$.

\vskip 10pt

If you wish to enter a complex number in polar form on a TI-84, you must unfortunately express the angle in units of {\it radians} (even though the calculator is able to display the result in {\it degrees}).  For example, to enter the number $25 \angle 15^o$ into a TI-84 calculator, you must type:

\vskip 10pt

{\tt 25 e$^{i15 \pi / 180}$}

\vskip 10pt

The fraction $\pi / 180$ is the conversion factor from degrees to radians, since there are $2\pi$ radians to a full circle, or $\pi$ radians to every 180 degrees.  Thus, writing $15 \pi / 180$ multiplies the desired angle (15 degrees) by the conversion factor $\pi / 180$ to yield a power in radians.  The obligatory $i$ simply makes this power an imaginary quantity, which is mathematically necessary with exponential notation for describing a complex number.  It should be noted that the order of entry for the power matters little.  $i15 \pi / 180$ works just as well as $15 i \pi / 180$ or $15 \pi / 180 i$.

\vskip 10pt

A time-saving step some students find useful is to save the imaginary quantity $i \pi / 180$ to a memory location in the TI-84 such as {\tt Z}.  That way, they can recall that imaginary factor from memory instead of typing the whole thing by hand every time they wish to enter a polar-form complex number.  Supposing the memory location {\tt Z} contains $i \pi / 180$, entering the number $25 \angle 15^o$ becomes as simple as:

\vskip 10pt

{\tt 25 e$^{15Z}$} \hskip 10pt or \hskip 10pt {\tt 25 e$^{Z15}$}

\vskip 10pt

It should be understood that {\it any} memory location in your calculator is suitable for storing $i \pi / 180$, not just {\tt Z}.  The TI-84 calculator even provides a $\Theta$ memory location ($<$Alpha$>$-$<$3$>$) that you may use and find easy to remember because of its common association with angles.  It should also be understood that this imaginary quantity is not the same as $i$ or $j$, which the calculator already provides a dedicated function for.  The imaginary quantity we're storing in memory for the purpose of entering polar-notation angles contains not only $i$ but also the $\pi / 180$ conversion factor necessary for translating your {\it degree} entry into {\it radians}.  

%(END_ANSWER)





%(BEGIN_NOTES)


%INDEX% Mathematics review: complex numbers

%(END_NOTES)

