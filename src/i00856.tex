
%(BEGIN_QUESTION)
% Copyright 2014, Tony R. Kuphaldt, released under the Creative Commons Attribution License (v 1.0)
% This means you may do almost anything with this work of mine, so long as you give me proper credit

An interesting method of electrical {\it heat tracing} is to pass a very large electric current through the walls of the pipe itself, using the metal pipe as a large resistor.  The heat dissipated by the ``resistor'' will maintain the pipe's temperature high enough to avoid solidification of the liquid inside it.

\vskip 10pt

In order to achieve such a high current, a step-down power transformer is used to convert industrial line power (typically 240 VAC or 480 VAC) into low-voltage, high-current power used to resistively heat the pipe.  Assuming the use of 480 VAC power (single-phase), determine the step-down ratio necessary to deliver 1500 watts of heating power to a pipe exhibiting an end-to-end electrical resistance of 0.75 ohms.

\vfil 

\underbar{file i00856}
\eject
%(END_QUESTION)





%(BEGIN_ANSWER)

This is a graded question -- no answers or hints given!

%(END_ANSWER)





%(BEGIN_NOTES)

Power dissipated by a resistance is proportional to the square of the voltage dropped across that resistance:

$$P = {V^2 \over R}$$

Since we know the desired power dissipation (1500 watts) and the resistance of the pipe (0.75 ohms), we may use this formula to solve for voltage dropped across the length of the pipe:

$$V = \sqrt{P R} = \sqrt{(1500 \hbox { W})(0.75 \> \Omega)} = 33.54 \hbox{ V}$$
 
The transformer needs to step 480 volts down to 33.54 volts, and therefore requires a turns ratio of:

$$\hbox{Turns ratio} = {480 \over 33.54} = 14.31$$

%INDEX% Electronics review: AC transformer circuit

%(END_NOTES)

