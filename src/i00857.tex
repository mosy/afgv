
%(BEGIN_QUESTION)
% Copyright 2014, Tony R. Kuphaldt, released under the Creative Commons Attribution License (v 1.0)
% This means you may do almost anything with this work of mine, so long as you give me proper credit

Suppose an agricultural feed supply company needs to measure the rate at which grain is lifted to the top of a storage silo by a motor-driven conveyor belt.  One solution to this measurement problem is to install a {\it weighfeeder} on the belt, sensing the weight of grain per unit length and multiplying this figure by belt speed to calculate mass flow rate (e.g. pounds per minute).  However, if good accuracy is not essential, a crude determination of mass flow rate may be obtained by measuring the true power ($P$) of the AC motor driving the conveyor. 

\vskip 10pt

Attaching a power monitor to the motor leads, you collect the following data:

\begin{itemize}
\item{} $V$ = 238 volts AC
\item{} $I$ = 17.4 amps AC
\item{} Phase angle ($\theta$) = 35$^{o}$
\end{itemize}

Assuming this motor is 93\% efficient, calculate the true mechanical power output by this electric motor.

\vskip 10pt

Next, explain what will be necessary to do to convert the motor's power output value in units of {\it horsepower} into a mass flow rate in units of {\it pounds of grain per second}.

\vfil 

\underbar{file i00857}
\eject
%(END_QUESTION)





%(BEGIN_ANSWER)

This is a graded question -- no answers or hints given!

%(END_ANSWER)





%(BEGIN_NOTES)

In any AC circuit where there is a non-zero phase shift between voltage and current, true power is given by the following formula:

$$P = I V \cos \theta$$

Calculating the true electrical power input to this motor:

$$P = (17.4 \hbox{ A}) (238 \hbox{ V}) (\cos 35^o) = 3392.3 \hbox{ W}$$

The motor is not 100\% efficient, which means some of this energy is being converted into heat and not into usable mechanical work.  The actual amount of mechanical power output by this motor will be 93\% of the input power figure:

$$P_{out} = (93\%) (P_{in}) = (0.93) (3392.3 \hbox{ W}) = 3154.8 \hbox{ W}$$

Converting this power value into units of horsepower:

$$\left(3392.3 \hbox{ W} \over 1 \right)  \left(1 \hbox{ HP} \over 746 \hbox{ W}\right) = 4.229 \hbox{ HP}$$

\vskip 10pt

Now, we know that one horsepower is defined as 550 foot-pounds of work done per second of time.  Looking at the units of measurement defining one horsepower, it should be evident that ``pounds per second'' are already part of that definition.  All we need to do is divide the power (in units of foot-pounds per second) by feet to get pounds per second.

But which physical measurement in units of {\it feet} is most appropriate?  Seeing as how the conveyor's task is to lift grain to the top of a storage silo, the work being done by the motor is the work of lifting the grain against gravity.  Therefore, it is the {\it vertical lift height} that matters in the calculation of work, and consequently that same lift height which we will divide into power (ft-lbs/s) to obtain a mass flow rate of pounds per second:

$$\left(4.229 \hbox{ HP} \over 1 \right)  \left( 550 \hbox{ ft-lb/s} \over 1 \hbox{ HP} \right) = 2325.9 \hbox{ ft-lb/s}$$

We were not given the vertical lift height in this problem, so we cannot calculate grain mass flow rate, but we know all we would need to do is divide this power value (2325.9 ft-lb/s) by the vertical lift height in feet to obtain mass flow rate in pounds of grain per second.

%INDEX% Electronics review: AC motor horsepower calculation
%INDEX% Final Control Elements, motor: weighfeeder

%(END_NOTES)

