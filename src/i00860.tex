
%(BEGIN_QUESTION)
% Copyright 2014, Tony R. Kuphaldt, released under the Creative Commons Attribution License (v 1.0)
% This means you may do almost anything with this work of mine, so long as you give me proper credit

Read and outline the ``Electrical Voltage'' section of the ``DC Electricity'' chapter in your {\it Lessons In Industrial Instrumentation} textbook.  Note the page numbers where important illustrations, photographs, equations, tables, and other relevant details are found.  Prepare to thoughtfully discuss with your instructor and classmates the concepts and examples explored in this reading.

\vskip 20pt \vbox{\hrule \hbox{\strut \vrule{} {\bf Suggestions for Socratic discussion} \vrule} \hrule}

\begin{itemize}
\item{} Explain what {\it potential} energy is, and identify some of the different forms it may take.
\item{} Explain what it means to say that voltage is always {\it relative} between two different physical locations rather than being an {\it absolute} quantity at any one point.
\item{} Sketch a diagram showing 1.5 volt batteries connected in such a way as to produce 6 volts.
\item{} Sketch a diagram showing four 1.5 volt batteries connected in such a way as to produce 3 volts.
\end{itemize}

\underbar{file i00860}
%(END_QUESTION)





%(BEGIN_ANSWER)


%(END_ANSWER)





%(BEGIN_NOTES)

%INDEX% Reading assignment: Lessons In Industrial Instrumentation, DC Electricity (electrical voltage)

%(END_NOTES)

