
%(BEGIN_QUESTION)
% Copyright 2014, Tony R. Kuphaldt, released under the Creative Commons Attribution License (v 1.0)
% This means you may do almost anything with this work of mine, so long as you give me proper credit

Read and outline the ``Electrical Current'' section of the ``DC Electricity'' chapter in your {\it Lessons In Industrial Instrumentation} textbook.  Note the page numbers where important illustrations, photographs, equations, tables, and other relevant details are found.  Prepare to thoughtfully discuss with your instructor and classmates the concepts and examples explored in this reading.

\vskip 20pt \vbox{\hrule \hbox{\strut \vrule{} {\bf Suggestions for Socratic discussion} \vrule} \hrule}

\begin{itemize}
\item{} We know there is no such thing as voltage at a single point, but is there such a thing as current at a single point?  Explain why or why not.
\item{} Explain what happens when a voltage is applied across the length of an insulating material, versus when applied across the length of a conducting material.
\item{} Examining the diagram in the textbook showing an electric generator connected to an electric motor, explain what would happen if one of those wires were cut in half.
\item{} Examining the diagram in the textbook showing an electric generator connected to an electric motor, explain what would happen if a {\it short circuit} were to occur (i.e. a connection made from one wire to the other so as to bypass the electric motor).
\item{} Compare and contrast the flow of electric charges with the flow of a fluid in a piping system, especially how those flows relate to {\it voltage} and {\it pressure}, respectively.
\end{itemize}

\underbar{file i00861}
%(END_QUESTION)





%(BEGIN_ANSWER)


%(END_ANSWER)





%(BEGIN_NOTES)

%INDEX% Reading assignment: Lessons In Industrial Instrumentation, DC Electricity (electrical current)

%(END_NOTES)

