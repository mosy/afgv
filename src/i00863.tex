
%(BEGIN_QUESTION)
% Copyright 2014, Tony R. Kuphaldt, released under the Creative Commons Attribution License (v 1.0)
% This means you may do almost anything with this work of mine, so long as you give me proper credit

Read and outline the ``Electrical Resistance and Ohm's Law'' section of the ``DC Electricity'' chapter in your {\it Lessons In Industrial Instrumentation} textbook.  Note the page numbers where important illustrations, photographs, equations, tables, and other relevant details are found.  Prepare to thoughtfully discuss with your instructor and classmates the concepts and examples explored in this reading.

\vskip 20pt \vbox{\hrule \hbox{\strut \vrule{} {\bf Suggestions for Socratic discussion} \vrule} \hrule}

\begin{itemize}
\item{} A very important concept when applying Ohm's Law is to keep all variables in context.  Explain what this means, and then give an example of how this rule could be violated (i.e. show a {\it wrong} use of Ohm's Law) in one of the circuit analysis examples shown in the textbook.
\item{} Explore the use of ``Engineering'' mode in your scientific calculator to display values using power-of-ten notation that fit well with common metric prefixes such as {\it milli} (one-thousandth, or $1 \times 10^{-3}$).
\end{itemize}

\underbar{file i00863}
%(END_QUESTION)





%(BEGIN_ANSWER)


%(END_ANSWER)





%(BEGIN_NOTES)

%INDEX% Reading assignment: Lessons In Industrial Instrumentation, DC Electricity (electrical resistance and ohm's law)

%(END_NOTES)

