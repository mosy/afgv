
%(BEGIN_QUESTION)
% Copyright 2011, Tony R. Kuphaldt, released under the Creative Commons Attribution License (v 1.0)
% This means you may do almost anything with this work of mine, so long as you give me proper credit

Suppose you were asked to write instructions for an inexperienced instrument technician to test a pneumatic loop controller on a workbench rather than installed and working in a process, with standard (3-15 PSI) signal ranges for its input as well as its output.  The goal of this test is to prove that the controller exhibits a gain of 2 when its proportional band dial is set to 50\%.

\vskip 10pt

Describe a set of easy-to-follow steps to test this controller's gain, giving numerical examples of pressures to apply at the controller's input and pressures to look for at the controller's output.  Assume {\it direct} action.

\underbar{file i00885}
%(END_QUESTION)





%(BEGIN_ANSWER)

Full credit for answering to look for a pressure {\it change} at the output that is precisely twice the pressure {\it change} applied to the input by a pressure source.  Only half-credit if the answer implies a fixed (bias-less) relationship between input and output (e.g. $P_{out}$ will be twice $P_{in}$, or 6 PSI in = 9 PSI out, or any other answer ignoring the effects of the controller's bias while getting the out/in ratio right).

%(END_ANSWER)





%(BEGIN_NOTES)

{\bf This question is intended for exams only and not worksheets!}.

%(END_NOTES)


