
%(BEGIN_QUESTION)
% Copyright 2006, Tony R. Kuphaldt, released under the Creative Commons Attribution License (v 1.0)
% This means you may do almost anything with this work of mine, so long as you give me proper credit

Orthophosphoric acid (H$_{3}$PO$_{4}$) is formed by combining tetraphosphorus decoxide (P$_{4}$O$_{10}$) with water (H$_{2}$O).  Write a balanced equation showing all reactants and all reaction products in the proper proportions.

\vskip 10pt

Also, determine how many {\it moles} of tetraphosphorus decoxide need to be added to twenty moles of water to completely react, and how many {\it moles} of acid will be produced as a result.

\vskip 20pt \vbox{\hrule \hbox{\strut \vrule{} {\bf Suggestions for Socratic discussion} \vrule} \hrule}

\begin{itemize}
\item{} Explain how to check your work to make sure the final equation is properly balanced.
\end{itemize}

\underbar{file i00901}
%(END_QUESTION)





%(BEGIN_ANSWER)

\noindent
{\bf Partial answer:}

\vskip 10pt

3.33 moles of $\hbox{P}_4\hbox{O}_{10}$ needed to completely react with 20 moles of water, because the ratio of water to tetraphosphorus decoxide in the balanced equation is 6:1.

%(END_ANSWER)





%(BEGIN_NOTES)

Balancing this reaction using simultaneous linear equations:

% No blank lines allowed between lines of an \halign structure!
% I use comments (%) instead, so Tex doesn't choke.

$$\vbox{\offinterlineskip
\halign{\strut
\vrule \quad\hfil # \ \hfil & 
\vrule \quad\hfil # \ \hfil & 
\vrule \quad\hfil # \ \hfil & 
\vrule \quad\hfil # \ \hfil \vrule \cr
\noalign{\hrule}
%
% First row
1 & $x$ & = & $y$ \cr
%
\noalign{\hrule}
%
% Another row
P$_{4}$O$_{10}$ & H$_{2}$O & $\to$ & H$_{3}$PO$_{4}$ \cr
%
\noalign{\hrule}
} % End of \halign 
}$$ % End of \vbox

% No blank lines allowed between lines of an \halign structure!
% I use comments (%) instead, so Tex doesn't choke.

$$\vbox{\offinterlineskip
\halign{\strut
\vrule \quad\hfil # \ \hfil & 
\vrule \quad\hfil # \ \hfil \vrule \cr
\noalign{\hrule}
%
% First row
{\bf Element} & {\bf Balance equation} \cr
%
\noalign{\hrule}
%
% Another row
Phosphorus & $4 + 0x = 1y$ \cr
%
\noalign{\hrule}
%
% Another row
Hydrogen & $0 + 2x = 3y$ \cr
%
\noalign{\hrule}
%
% Another row
Oxygen & $10 + x = 4y$ \cr
%
\noalign{\hrule}
} % End of \halign 
}$$ % End of \vbox

From the phosphorus balance equation, we see that $y = 4$.  Plugging this value of $y$ into the hydrogen balance equation ($2x = 12$) we see that $x = 6$.  Plugging this value of $x$ into the oxygen balance equation ($10 + 6 = 4y$) we see that $y = 4$ which confirms our first solution.  Therefore:

$$\hbox{P}_4\hbox{O}_{10} + \hbox{6H}_2\hbox{O} \to \hbox{4H}_3\hbox{PO}_4$$

\vskip 10pt

3.33 moles of $\hbox{P}_4\hbox{O}_{10}$ needed to completely react with 20 moles of water, since according to the balanced equation we have a 6:1 molecular ratio of water to decoxide.  20/6 = 3.33

\vskip 10pt

13.33 moles of acid produced after twenty moles of water react, since according to the balanced equation we have a 4:6 ratio of acid to water.  (20)(4/6) = 13.33



%INDEX% Chemistry, stoichiometry: balancing a chemical equation

%(END_NOTES)


