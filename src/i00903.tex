
%(BEGIN_QUESTION)
% Copyright 2006, Tony R. Kuphaldt, released under the Creative Commons Attribution License (v 1.0)
% This means you may do almost anything with this work of mine, so long as you give me proper credit

German submarines in World War II used highly purified hydrogen peroxide (H$_{2}$O$_{2}$) as a monopropellant for their torpedo propulsion systems: when placed in contact with a catalyst, hydrogen peroxide spontaneously decomposes to produce water (H$_{2}$O), oxygen (O$_{2}$), and a lot of heat.  Write a balanced equation showing all reactants and all reaction products in the proper proportions.

\vskip 10pt

Also, determine how many {\it moles} of water and how many {\it moles} of O$_{2}$ will be produced when exactly three moles of hydrogen peroxide decomposes.

\vskip 20pt \vbox{\hrule \hbox{\strut \vrule{} {\bf Suggestions for Socratic discussion} \vrule} \hrule}

\begin{itemize}
\item{} Explain how to check your work to make sure the final equation is properly balanced.
\item{} Should the peroxide decomposition reaction have a {\it positive} or {\it negative} $\Delta H$ value written next to it?
\item{} Explain {\it why} this reaction is exothermic, based on an analysis of the reaction equation: where is energy being absorbed, and where is energy being released?  What can we tell about the relative bond strengths of oxygen to H$_{2}$O, versus oxygen to itself?
\item{} Explain the purpose of using a {\it catalyst} to initiate the decomposition reaction of hydrogen peroxide in such a torpedo.
\item{} Will the catalyst be consumed in the decomposition reaction?
\end{itemize}

\underbar{file i00903}
%(END_QUESTION)





%(BEGIN_ANSWER)

\noindent
{\bf Partial answer:}

$$\hbox{2H}_2\hbox{O}_2 \to \hbox{2H}_2\hbox{O} + \hbox{O}_2$$

%(END_ANSWER)





%(BEGIN_NOTES)

Balancing this reaction using simultaneous linear equations:

% No blank lines allowed between lines of an \halign structure!
% I use comments (%) instead, so Tex doesn't choke.

$$\vbox{\offinterlineskip
\halign{\strut
\vrule \quad\hfil # \ \hfil & 
\vrule \quad\hfil # \ \hfil & 
\vrule \quad\hfil # \ \hfil & 
\vrule \quad\hfil # \ \hfil \vrule \cr
\noalign{\hrule}
%
% First row
1 & = & $x$ & $y$ \cr
%
\noalign{\hrule}
%
% Another row
H$_{2}$O$_{2}$ & $\to$ & H$_{2}$O & O$_{2}$ \cr
%
\noalign{\hrule}
} % End of \halign 
}$$ % End of \vbox

% No blank lines allowed between lines of an \halign structure!
% I use comments (%) instead, so Tex doesn't choke.

$$\vbox{\offinterlineskip
\halign{\strut
\vrule \quad\hfil # \ \hfil & 
\vrule \quad\hfil # \ \hfil \vrule \cr
\noalign{\hrule}
%
% First row
{\bf Element} & {\bf Balance equation} \cr
%
\noalign{\hrule}
%
% Another row
Hydrogen & $2 = 2x + 0y$ \cr
%
\noalign{\hrule}
%
% Another row
Oxygen & $2 = 1x + 2y$ \cr
%
\noalign{\hrule}
} % End of \halign 
}$$ % End of \vbox

From the hydrogen balance equation ($2 = 2x$) we see that $x = 1$.  Plugging that value of $x$ into the oxygen balance equation ($2 = 1 + 2y$) we see that $y = {1 \over 2}$.  In the interest of having whole-number molecular multipliers we will express the solutions as $x = 2$ and $y = 1$, with the original constant multiplier of peroxide molecules bumped up from 1 to 2 as well:

$$\hbox{2H}_2\hbox{O}_2 \to \hbox{2H}_2\hbox{O} + \hbox{O}_2$$

\vskip 10pt

As we can see from the balanced equation, the molar ratio of hydrogen peroxide to water is 1:1, so three moles of peroxide will generate three moles of water.  Also, we can see that the ratio of peroxide to oxygen molecules is 2:1, so three moles of peroxide will generate 1.5 moles of oxygen gas.

\vskip 10pt

Although energy is required to liberate the oxygen atom from the hydrogen peroxide molecule, more energy is released in the subsequent bonding of free oxygen atoms to form oxygen molecules (O$_{2}$).  This makes the catalytic decomposition of hydrogen peroxide {\it exothermic}.

%INDEX% Chemistry, catalyst
%INDEX% Chemistry, stoichiometry: balancing a chemical equation
%INDEX% Process: torpedo propulsion (hydrogen peroxide)

%(END_NOTES)


