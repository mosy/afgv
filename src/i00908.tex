
%(BEGIN_QUESTION)
% Copyright 2010, Tony R. Kuphaldt, released under the Creative Commons Attribution License (v 1.0)
% This means you may do almost anything with this work of mine, so long as you give me proper credit

\vskip 5pt
\hrule
\vskip 5pt

\noindent
Avogadro's constant = 6.022 $\times$ 10$^{23}$ entities per mole

\vskip 5pt
\hrule
\vskip 5pt

$$Q = mc \Delta T \hskip 50 pt Q = mL \hskip 50pt R_T = R_{ref} \left[1 + \alpha (T - T_{ref}) \right]$$

$${dQ \over dt} = {{A \Delta T} \over R} \hskip 50pt {dQ \over dt} = {{k A \Delta T} \over l} \hskip 50pt {dQ \over dt} = e \sigma A T^4$$

\vskip 5pt
\hrule
\vskip 5pt


\noindent
Ideal Gas Law: 

$$PV = nRT$$

\noindent
Where,

$P$ = Absolute pressure (atmospheres)

$V$ = Volume (liters)

$n$ = Gas quantity (moles)

$R$ = Universal gas constant (0.0821 L $\cdot$ atm / mol $\cdot$ K)

$T$ = Absolute temperature (K)


\vskip 5pt
\hrule
\vskip 5pt

\noindent
Electrical conductivity of a fluid, measured between two electrodes:

$$G = k{A \over d}$$

\noindent
Where,

$G$ = Conductance, in Siemens (S)

$k$ = Specific conductivity of liquid, in Siemens per centimeter (S/cm)

$A$ = Electrode area (each), in square centimeters (cm$^{2}$)

$d$ = Electrode separation distance, in centimeters (cm)

\vskip 5pt
\hrule
\vskip 5pt

\noindent
Ionization constant of pure water at 25$^{o}$ C:

$$K_W = [\hbox{H}^{+}] [\hbox{OH}^{-}] = 1.00 \times 10^{-14}$$ 

\vskip 5pt
\hrule
\vskip 5pt

\noindent
Definition of pH:

$$\hbox{pH} = - \log [\hbox{H}^{+}]$$

\vskip 5pt
\hrule
\vskip 5pt

\noindent
Nernst equation:

$$V = {{R T} \over {nF}} \ln \left({C_1 \over C_2}\right) \hskip 30pt V = {{2.303 R T} \over {nF}} \log \left({C_1 \over C_2}\right) \hskip 30pt V = {{2.303 R T} \over {nF}} (7 - \hbox{pH})$$

\noindent
Where,

$V$ = Voltage produced across membrane due to ion exchange, in volts (V)

$R$ = Universal gas constant (8.315 J/mol$\cdot$K)

$T$ = Absolute temperature, in Kelvin (K)

$n$ = Number of electrons transferred per ion exchanged (unitless)

$F$ = Faraday constant, in coulombs per mole (96,485 C/mol e$^{-}$)

$C_1$ = Concentration of measured solution, in moles per liter of solution ($M$)

$C_2$ = Concentration of reference solution, in moles per liter of solution ($M$)

\vskip 5pt
\hrule
\vskip 5pt


\underbar{file i00908}
%(END_QUESTION)





%(BEGIN_ANSWER)


%(END_ANSWER)





%(BEGIN_NOTES)

{\bf This question is intended for exams only and not worksheets!}.

%(END_NOTES)


