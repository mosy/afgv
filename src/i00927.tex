
%(BEGIN_QUESTION)
% Copyright 2006, Tony R. Kuphaldt, released under the Creative Commons Attribution License (v 1.0)
% This means you may do almost anything with this work of mine, so long as you give me proper credit

Suppose we need to measure the concentration of substance {\it X} in a process gas stream.  If we know that {\it X} happens to generate ions when burned, does this mean we can connect the output of a flame ionization detector (FID) to an indicator and have a guaranteed indication of substance {\it X's} concentration?  Why or why not?

\underbar{file i00927}
%(END_QUESTION)





%(BEGIN_ANSWER)

No, because we don't know what {\it other} substances in the process stream also generate ions when burned, and therefore the measurement we get may not necessarily reflect the concentration of {\it X} in the process.  We can, however, use an FID to measure this substance if it serves as the detector at the end of a {\it chromatograph}, where substance {\it X} is expected to emerge from the column at some definite time.  Here, the non-specific detection capability of the FID is made specific by the selective delay of the chromatograph column.

%(END_ANSWER)





%(BEGIN_NOTES)


%INDEX% Measurement, analytical: applications of different technologies

%(END_NOTES)


