
%(BEGIN_QUESTION)
% Copyright 2014, Tony R. Kuphaldt, released under the Creative Commons Attribution License (v 1.0)
% This means you may do almost anything with this work of mine, so long as you give me proper credit

A theory of medicinal therapy originating in the early 1800's consisted of taking samples of substances known to induce ill effects in patients, and then diluting those substances with a solvent such as alcohol or water to an extreme degree, finally administering the diluted solution to the patient as a cure for an ailment with symptoms similar to the ill effects induced by the pure substance.  Here is a sample of text from a book written by the inventor of this technique:

\vskip 10pt {\narrower \noindent \baselineskip5pt

If two drops of a mixture of equal parts alcohol and the recent juice of any medicinal plant be diluted with 98 drops of alcohol in a vial capable of containing one hundred and thirty drops, and the whole twice shaken together, the medicine becomes exhalted in energy ({\it potenzirt}) to the first development of power, or, as it may be denominated, the first potence.  The process is to be continued through twenty-nine additional vials, each of equal capacity with the first, and each containing ninety-nine drops of spirits of wine; so that every successive vial, after the first, being furnished with one drop from the vial or dilution immediately preceding (which had been just twice shaken), is, in its turn, to be shaken twice, remembering to number the dilution of each vial upon the cork as the operation proceeds.  These manipulations are to be conducted thus through all the vials, from the first up to the thirtieth or decillionth development of power ({\it potenzirte Decillion-Verdunnung, X}) which is the one in most general use.

\par} \vskip 10pt

We will use this dilution recipe to calculate the number of molecules contained in the dilution vials.  To do this, we will assume one drop of pure medicinal plant juice contains 0.0006 moles of the juice's active ingredient, and one drop of pure ethyl alcohol contains the same number of moles (0.0006) of alcohol.  

\vskip 20pt

Calculate the number of moles of active ingredient contained in the first vial (containing a 2:98 ratio of drops of diluted juice to drops of pure solvent):

$n_1$ = \underbar{\hskip 50pt} moles

\vskip 20pt

Calculate the number of moles of active ingredient contained in the second vial (containing a 1/100 dilution of the first vial's solution):

$n_2$ = \underbar{\hskip 50pt} moles

\vskip 20pt

Calculate the number of moles of active ingredient contained in the third vial (containing a 1/100 dilution of the second vial's solution):

$n_3$ = \underbar{\hskip 50pt} moles

\vskip 20pt

Continue this all the way until the thirtieth vial:

$n_{30}$ = \underbar{\hskip 50pt} moles

\vskip 20pt

How many actual {\it molecules} of active ingredient does this molar quantity represent?

$N_{30}$ = \underbar{\hskip 50pt} molecules

\vskip 20pt

\underbar{file i00945}
%(END_QUESTION)





%(BEGIN_ANSWER)

Since the first vial contains {\it two} drops of 50\% plant juice solution, consisting of one drop of pure plant juice with one drop of alcohol, the first vial must contain 0.0006 moles of active ingredient:

\vskip 10pt

$n_1$ = \underbar{0.0006} moles

\vskip 10pt

For each successive vial we take one drop of solution from the previous vial and mix it with 99 drops of pure alcohol, making a $1 \over 100$ dilution ratio.  This means the second vial will have $1 \over 100$ the amount of active ingredient as the first vial:

\vskip 10pt

$n_2$ = \underbar{0.000006} moles

\vskip 10pt

The third vial will contain $1 \over 100$ of the second vial's active ingredient:

\vskip 10pt

$n_3$ = \underbar{0.00000006} moles

\vskip 10pt

In fact, we may generalize the dilution in the following equation:

$$n_m = {0.0006 \over 100^{m-1}}$$

This means the 30th vial will contain the following number of moles of active ingredient:

$$n_m = {0.0006 \over 100^{30-1}} = 6 \times 10^{-62} \hbox{ mol}$$

Recall that a single mole is equal to $6.022 \times 10^{23}$ molecules.  This means a molar quantity of $6 \times 10^{-62}$ moles is equal to $3.61 \times 10^{-38}$ molecules.  Since this number is less than 1, it is purporting to tell us the 30th vial contains just a {\it miniscule fraction of one molecule!}  We know this is impossible, since molecules cannot be divided (at least not without changing their essential properties).  The only way to square this amazingly low molar quantity with the indivisibility of real molecules is to conclude that the 30th vial only has a {\it probability} of containing just a single molecule of the original active ingredient.  To be precise, the 30th vial has but a 1 in $2.77 \times 10^{37}$ chance of containing a single molecule of the original plant juice's active ingredient.

Scientists of this era were aware of the Avogadro number and molecular quantities.  When criticised on this point, the inventor of this therapy responded thusly:

\vskip 10pt {\narrower \noindent \baselineskip5pt

It is of little import whether the attenuation goes so far as to appear almost impossible to ordinary physicians whose minds feed on no other ideas but what are gross and immaterial. 

\par} \vskip 10pt

A footnote to this statement defends extreme dilutions in this way:

\vskip 10pt {\narrower \noindent \baselineskip5pt

Mathematicians will inform them, that in whatever number of parts they may divide a substance, each portion still retains a {\it small share} of the material; that, consequently the most diminutive part that can be conceived never ceases to be {\it something}, and can in no instance be reduced to, nothing.  Physicians may learn from them that there exist immense powers which have no weight, such as light and heat, and which are consequently infinitely lighter than the medicianl contents of the smallest homeopathic doses.  Let them weigh, if they can, the injurious words which excite a bilious fever, or the afflicting news of the death of a son, which terminates the existence of an affectionate mother.  Let them only touch, for a quarter of an hour, a magnet capable of carrying a weight of a hundred pounds, and the pain will soon teach them that even the imponderable bodies can also produce on man the most violent medicinal effects!  Let any of these weak-minded mortals of a delicate constitution but gently apply, during a few minutes, to the pit of the stomach the extremity of the thumb of a vigorous mesmerist who has fixed his intent, and the disagreeable sensations that he experiences will soon make him repent having set limits to the boundless activity of nature.

\par} \vskip 10pt



%(END_ANSWER)





%(BEGIN_NOTES)

This form of treatment was given the name {\it homeopathy} by its inventor, Samuel Hahnemann.  Its purpose was to cure diseases by administering extremely diluted solutions containing substances known to induce ill effects similar to sympyoms of the disease.  So, for example, someone suffering from malaria (with feverish symptoms) might be administered a diluted dose of any substance known to induce fever in a healthy person.  Similarly, someone suffering from poor sleep might be given a diluted dose of caffeine.

Quotes taken from the {\it Organon of Homeopathic Medicine} by Samuel Hahnemann, First American edition (from the British translation of the fourth German edition), published in 1836.

\begin{itemize}
\item{} The paragraph describing the dilution process is paragraph \#270 found on page 200.
\item{} The footnote for paragraph \#272 references the danger of applying multiple homeopathic remedies.
\item{} Paragraph \#280 and its footnote criticizes those who claim extreme dilutions are impossibly weak.
\end{itemize}



%INDEX% Chemistry, stoichiometry: moles

%(END_NOTES)


