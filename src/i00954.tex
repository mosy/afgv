
%(BEGIN_QUESTION)
% Copyright 2015, Tony R. Kuphaldt, released under the Creative Commons Attribution License (v 1.0)
% This means you may do almost anything with this work of mine, so long as you give me proper credit

Decibels are a logarithmic representation of power ratios, with negative decibel values representing power attenuation and positive decibel values representing power gain.  Every addition or subtration of 10 dB exactly represents a 10-fold multiplication or division of power ratio.  A simple approximation that is useful in decibel calculations is that an addition or subtraction of 3 dB approximately represents a 2-fold multiplication or division of power ratio.

\vskip 10pt

Here are some numerical examples based on the equivalence of $\pm$ 10 dB with orders of magnitude:

\begin{itemize}
\item{} Power ratio gain of 10 = {\bf 10 dB}
\vskip 3pt
\item{} Power ratio gain of 100 = {\bf 20 dB}
\vskip 3pt
\item{} Power ratio gain of 1000 = {\bf 30 dB}
\vskip 3pt
\item{} Power ratio gain of 10000 = {\bf 40 dB}
\vskip 3pt
\item{} Power ratio reduction of $1 \over 10$ = {\bf $-$10 dB}
\vskip 3pt
\item{} Power ratio reduction of $1 \over 100$ = {\bf $-$20 dB}
\vskip 3pt
\item{} Power ratio reduction of $1 \over 1000$ = {\bf $-$30 dB}
\vskip 3pt
\item{} Power ratio reduction of $1 \over 10000$ = {\bf $-$40 dB}
\end{itemize}

\vskip 10pt

Here are some numerical examples based on the approximation of $\pm$ 3 dB being equivalent to doubling or halving of power ratio:

\begin{itemize}
\item{} Power ratio gain of 2 $\approx$ {\bf 3 dB}
\vskip 3pt
\item{} Power ratio gain of 4 $\approx$ {\bf 6 dB}
\vskip 3pt
\item{} Power ratio gain of 8 $\approx$ {\bf 9 dB}
\vskip 3pt
\item{} Power ratio gain of 16 $\approx$ {\bf 12 dB}
\vskip 3pt
\item{} Power ratio reduction of $1 \over 2$ $\approx$ {\bf $-$3 dB}
\vskip 3pt
\item{} Power ratio reduction of $1 \over 4$ $\approx$ {\bf $-$6 dB}
\vskip 3pt
\item{} Power ratio reduction of $1 \over 8$ $\approx$ {\bf $-$9 dB}
\vskip 3pt
\item{} Power ratio reduction of $1 \over 16$ $\approx$ {\bf $-$12 dB}
\end{itemize}

\vskip 10pt

Based on these equivalents and approximations, estimate the following decibel values from the given power ratio values:

\begin{itemize}
\item{} Power ratio gain of 20 $\approx$ \underbar{\hskip 50pt}
\vskip 5pt
\item{} Power ratio reduction of $1 \over 40$ $\approx$ \underbar{\hskip 50pt} 
\vskip 5pt
\item{} Power ratio gain of 1600 $\approx$ \underbar{\hskip 50pt}
\vskip 5pt
\item{} Power ratio reduction of $1 \over 800$ $\approx$ \underbar{\hskip 50pt} 
\vskip 5pt
\end{itemize}

\underbar{file i00954}
%(END_QUESTION)





%(BEGIN_ANSWER)

\begin{itemize}
\item{} Power ratio gain of 20 $\approx$ \underbar{\bf +13 dB}
\vskip 5pt
\item{} Power ratio reduction of $1 \over 40$ $\approx$ \underbar{\bf $-$16 dB} 
\vskip 5pt
\item{} Power ratio gain of 1600 $\approx$ \underbar{\bf +32 dB}
\vskip 5pt
\item{} Power ratio reduction of $1 \over 800$ $\approx$ \underbar{\bf $-$29 dB} 
\vskip 5pt
\end{itemize}

%(END_ANSWER)





%(BEGIN_NOTES)


%INDEX% Electronics review: decibel power calculations (mental math)

%(END_NOTES)


