
%(BEGIN_QUESTION)
% Copyright 2015, Tony R. Kuphaldt, released under the Creative Commons Attribution License (v 1.0)
% This means you may do almost anything with this work of mine, so long as you give me proper credit

Suppose a radio transmitter sends 80 watts of power into one end of a twin-lead cable, the other end connected to an antenna.  The cable is 200 feet long, and exhibits a power loss of $-0.03$ dB/foot.  Calculate the power received by the antenna, in both units of watts and units of dBm, {\it without using a calculator}.  Note that an addition or subtraction of 10 dB is exactly equal to a 10-fold multiplication or division ratio (respectively), and that an addition or subtraction of 3 dB is approximately equal to a 2-fold multiplication or division ratio.  

\vskip 10pt

$P_{antenna}$ = \underbar{\hskip 50pt} watts

\vskip 10pt

$P_{antenna}$ = \underbar{\hskip 50pt} dBm

\vskip 20pt \vbox{\hrule \hbox{\strut \vrule{} {\bf Suggestions for Socratic discussion} \vrule} \hrule}

\begin{itemize}
\item{} A technique highly recommended for word-problems is to {\it sketch a picture} of the problem and label elements of that picture with the given information.  Do this, and compare your sketch with those of your classmates.  How, specifically, does this aid your problem-solving?
\end{itemize}

\underbar{file i00956}
%(END_QUESTION)





%(BEGIN_ANSWER)

The 200 foot cable with a loss of $-0.03$ dB per foot will exhibit a total loss of $-6$ dB along its whole length.  We know that every $-3$ dB loss is approximately equal to a halving of power, and so $-6$ dB loss is ${1 \over 2} \times {1 \over 2} = {1 \over 4}$.  Therefore, only one-quarter of the transmitter's power makes it to the far end of the cable where the antenna attaches.  One-quarter of 80 watts is {\bf 20 watts}, which is our first answer.

\vskip 10pt

In order to convert 20 watts into dBm, we must understand what ``dBm'' actually means: the number of dB a certain power value is compared to 1 milliwatt.  In this case, 20 watts is 20,000 times greater than 1 milliwatt, and therefore the dBm value must be the equivalent of 20,000 (in dB).  Since we know every +10 dB is a 10-fold multiplication, and every +3 dB is a 2-fold multiplication, all we need to do is express 20,000 as a product of 10's and 2's which will then tell us the dB value as a sum of 10's and 3's:

$$20000 = 2 \times 10 \times 10 \times 10 \times 10$$

$$20000 = 3 \hbox{ dB} + 10 \hbox { dB} + 10 \hbox { dB} + 10 \hbox{ dB} + 10 \hbox { dB}$$

$$20000 = 43 \hbox{ dB}$$

$$20000 \times 1 \hbox{ mW}= \hbox{\bf 43 dBm}$$

%(END_ANSWER)





%(BEGIN_NOTES)


%INDEX% Electronics review: decibel power calculations (mental math)

%(END_NOTES)


