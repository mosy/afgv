
%(BEGIN_QUESTION)
% Copyright 2012, Tony R. Kuphaldt, released under the Creative Commons Attribution License (v 1.0)
% This means you may do almost anything with this work of mine, so long as you give me proper credit

An automotive performance shop is testing an engine they just built for a race car.  The engine is removed from the car, attached to a dynamometer for measuring its horsepower and other performance parameters.  For temperature measurement, a type K thermocouple has been attached to the metal block of the engine, thermocouple wires run all the way back to the control room where mechanics operate the dynamometer, and and a voltmeter connected to those thermocouple wires inside the control room (which happens to be at a temperature of 58 degrees F).

\vskip 10pt

Before the engine is started up, the voltmeter reads 0.0 millivolts.  The mechanics start up the engine, run it for several seconds, and then shut it down because they realized they forgot to hook up some other sensor for their main test.  In that short run time, the voltmeter's indication increased to 0.089 millivolts.

\vskip 10pt

How hot did the engine get in that short period of time?

\vskip 10pt

Assuming the engine is made of 300 pounds of iron, and filled with 2.5 gallons of water (in the cooling system), how much heat did the engine absorb within its own mass during those few seconds of run time?

\underbar{file i00971}
%(END_QUESTION)





%(BEGIN_ANSWER)

The voltmeter's indication of 0.089 millivolts is the difference between the measurement junction's temperature and the ambient temperature of the control room (58 $^{o}$F).  Since the millivoltage for a type K thermocouple at 58 degrees F is 0.575 mV, and the thermocouple's temperature is 0.089 mV greater than that, the thermocouple must be outputting 0.664 mV, which equates to {\bf 62 degrees F}.

\vskip 10pt

As for the amount of heat required to raise the engine to this new temperature, it is a matter of specific heat calculations: one incorporating the iron engine block's mass and the water's mass.

$$Q = mc \Delta T$$ 
 
$$Q = m_{iron}c_{iron} \Delta T + m_{water}c_{water} \Delta T$$ 

2.5 gallons of water is equivalent to 20.85 pounds of water.  The temperature rise from 58 $^{o}$F to 62 $^{o}$F is 4 $^{o}$F.  Therefore:

$$Q = (300)(0.108)(4)+ (20.85)(1)(4) = {\bf 213.0 \hbox{ BTU}}$$ 


 
%(END_ANSWER)





%(BEGIN_NOTES)


%INDEX% Measurement, temperature: thermocouple millivoltage interpretation
%INDEX% Physics, heat and temperature: calorimetry problem 

%(END_NOTES)


