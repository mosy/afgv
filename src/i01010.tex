
%(BEGIN_QUESTION)
% Copyright 2012, Tony R. Kuphaldt, released under the Creative Commons Attribution License (v 1.0)
% This means you may do almost anything with this work of mine, so long as you give me proper credit

Suppose you need to heat up the 70 gallons of water filling the inside of a hot tub, from ambient temperature (58 $^{o}$F) to 102 $^{o}$F.  Assuming a perfectly insulated hot tub, and ignoring the thermal mass of the hot tub frame itself, how much thermal energy (in units of BTU) will be needed to raise its temperature to the desired operating point?

\underbar{file i01010}
%(END_QUESTION)





%(BEGIN_ANSWER)

This is a problem of specific heat, following this formula:

$$Q = mc \Delta T$$

In order to calculate the amount of heat energy ($Q$) needed for the task, we need to know the mass of the hot tub's water ($m$), the specific heat of water (1 BTU per pound per degree F), and the temperature rise ($\Delta T$ = 102 $^{o}$F $-$ 58$^{o}$F = 44 $^{o}$F).  We were given the volume of water (70 gallons), but not its mass in pounds, so we need to do a units conversion based on water having a density of 62.4 pounds per cubic foot:

$$\left(70 \hbox{ gal} \over 1 \right) \left( 231 \hbox{ in}^3 \over 1 \hbox{ gal} \right) \left(1 \hbox{ ft}^3 \over 1728 \hbox{ in}^3 \right) \left( 62.4 \hbox{ lb} \over \hbox{ft}^3 \right) = 583.9 \hbox{ lb}$$

Now, we are all set to calculate the required heat energy:

$$Q = mc \Delta T$$

$$Q = (583.9 \hbox{ lb}) (1 \hbox{ BTU/lb-}^o\hbox{F}) (102^o\hbox{F} - 58^o\hbox{F})$$

$$Q = 25692.3 \hbox{ BTU}$$

%(END_ANSWER)





%(BEGIN_NOTES)


%INDEX% Physics, heat and temperature: calorimetry problem 

%(END_NOTES)


