
%(BEGIN_QUESTION)
% Copyright 2012, Tony R. Kuphaldt, released under the Creative Commons Attribution License (v 1.0)
% This means you may do almost anything with this work of mine, so long as you give me proper credit

Jane has an ear ache, and decides to apply a hot water bottle to her ear to help loosen the congestion in her ear canals and ease the pain.  Being a pre-med student who recently studied thermodynamics, she estimates a heat input of 19,000 calories necessary to do the job.

\vskip 10pt

Assuming Jane's hot water tap provides water at a temperature of 54 degrees Celsius and that her skin temperature is about 31 degrees Celsius, calculate how much water Jane will need to put into her hot water bottle to deliver the estimated amount of heat to her head as the bottle cools from its initial hot-water temperature down to her skin temperature.  

\vskip 10pt

Also, determine if your calculated value is a {\it high} estimate or a {\it low} estimate, based on simplifications assumed in your calculations.

\underbar{file i01011}
%(END_QUESTION)





%(BEGIN_ANSWER)

What we're trying to solve for here is the water mass necessary to deliver 19000 calories of sensible heat to Jane's ear, given a known drop in temperature.  Clearly, then, we need to apply the specific heat formula, solving for $m$:

$$Q = mc \Delta T$$

$$m = {Q \over c \Delta T}$$

$$m = {19000 \hbox{ cal} \over (1 \hbox{ cal/g-}^o\hbox{C}) (54^o\hbox{C} - 31^o\hbox{C})}$$

$$m = 826.1 \hbox{ grams}$$

Conveniently, 1 liter of water happens to be 1000 grams.  So, what we need is 0.8261 liters of water in Jane's water bottle.  

\vskip 10pt

This will be a {\it low} estimate, as not all of the heat liberated by the cooling water will transfer into Jane's irritated ear canals.  Some heat will transfer into other parts of her head, and a fair amount of heat will be lost to the ambient air.  Thus, Jane will actually need {\it more} than 0.8261 liters of hot water to do the job.

%(END_ANSWER)





%(BEGIN_NOTES)


%INDEX% Physics, heat and temperature: calorimetry problem 

%(END_NOTES)


