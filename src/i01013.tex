
%(BEGIN_QUESTION)
% Copyright 2012, Tony R. Kuphaldt, released under the Creative Commons Attribution License (v 1.0)
% This means you may do almost anything with this work of mine, so long as you give me proper credit

15 grams of iron filings at a temperature of 150 $^{o}$C are sprinkled onto a 25 gram strip of copper metal at room temperature (20 $^{o}$C), and left to equalize in temperature inside of a perfectly insulated chamber.  Calculate the final temperature of the iron/copper mass.

\underbar{file i01013}
%(END_QUESTION)





%(BEGIN_ANSWER)

Sensible heat lost by the iron ($c_{Fe}$ = 0.113 cal/g$\cdot ^{o}$C) as it cools to the final temperature will be equal to the sensible heat gained by the copper ($c_{Cu}$ = 0.093 cal/g$\cdot ^{o}$C) as it warms to the final temperature:

$$Q_{iron-lost} = Q_{copper-gained}$$

$$m_{Fe}c_{Fe} (150 - T) = m_{Cu}c_{Cu} (T - 20)$$

$$(15)(0.113)(150 - T) = (25)(0.093)(T - 20)$$

$$(1.695)(150 - T) = (2.325)(T - 20)$$

$$254.25 - 1.695 T = 2.325 T - 46.5$$

$$300.75 = 4.02 T$$

$$T = {300.75 \over 4.02}$$

$$T = 74.81^o\hbox{C}$$

%(END_ANSWER)





%(BEGIN_NOTES)


%INDEX% Physics, heat and temperature: calorimetry problem 

%(END_NOTES)


