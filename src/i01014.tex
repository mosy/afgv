
%(BEGIN_QUESTION)
% Copyright 2007, Tony R. Kuphaldt, released under the Creative Commons Attribution License (v 1.0)
% This means you may do almost anything with this work of mine, so long as you give me proper credit

Configure a PID simulator for any process control application, and then try tuning this process using both the closed-loop and the open-loop Ziegler-Nichols methods.  Document all the process parameters (e.g. dead time, lag time, reaction rate, ultimate gain, etc.) and then show the P, I, and D tuning parameter values calculated by these techniques.

\vskip 30pt

\noindent
Closed-loop test parameters (I and D actions disabled, P increased until oscillation):

\vskip 10pt

$K_u$ = \underbar{\hskip 50pt} \hskip 50pt $P_u$ = \underbar{\hskip 50pt}

\vskip 10pt

\noindent
Calculated results of Ziegler-Nichols closed-loop (``Ultimate'') tuning method:

\vskip 10pt

P = \underbar{\hskip 50pt} \hskip 50pt I = \underbar{\hskip 50pt} \hskip 50pt D = \underbar{\hskip 50pt}



\vskip 50pt



\noindent
Open-loop test parameters (controller in manual mode):

\vskip 10pt

$L$ = \underbar{\hskip 50pt} \hskip 50pt $R$ = \underbar{\hskip 50pt} \hskip 50pt $\Delta m$ = \underbar{\hskip 50pt} 

\vskip 10pt

\noindent
Calculated results of Ziegler-Nichols open-loop tuning method:

\vskip 10pt

P = \underbar{\hskip 50pt} \hskip 50pt I = \underbar{\hskip 50pt} \hskip 50pt D = \underbar{\hskip 50pt}



\vskip 50pt



\noindent
If you think you can improve on these results by experimenting with P, I, and/or D parameter values, feel free to do so, and then document the settings you found to work best:

\vskip 10pt

P = \underbar{\hskip 50pt} \hskip 50pt I = \underbar{\hskip 50pt} \hskip 50pt D = \underbar{\hskip 50pt}

\vskip 30pt

\vskip 20pt \vbox{\hrule \hbox{\strut \vrule{} {\bf Suggestions for Socratic discussion} \vrule} \hrule}

\begin{itemize}
\item{} A common mistake when students try to apply either Ziegler-Nichols tuning method is ignoring the units of time measurement used by the controller for $\tau_i$ and $\tau_d$.  For example, a student may measure the period of ultimate cycle ($P_u$) in seconds, but then their controller needs to have a $\tau_d$ value in units of {\it minutes}, or worse yet a $\tau_i$ value in units of {\it repeats per minute}.  Explain why the units of measurement matter when you apply either Ziegler-Nichols PID tuning method.
\end{itemize}

\underbar{file i01014}
%(END_QUESTION)





%(BEGIN_ANSWER)

The ``answers'' can only be found by actually tuning a controller!  It is recommended to review the results of your tuning with your instructor to grasp the significance of each process and its PID tuning requirements.

%(END_ANSWER)





%(BEGIN_NOTES)

\vfil \eject

\noindent
{\bf Summary Quiz:}

(This makes an excellent summary quiz, demonstrating the results of three different tuning methods)

%INDEX% Control, PID tuning: computer simulation exercise (three different tuning methods)

%(END_NOTES)


