
%(BEGIN_QUESTION)
% Copyright 2015, Tony R. Kuphaldt, released under the Creative Commons Attribution License (v 1.0)
% This means you may do almost anything with this work of mine, so long as you give me proper credit

Read and outline the ``Typical Calibration Errors'' subsection of the ``Calibration Errors and Testing'' section of the ``Instrument Calibration'' chapter in your {\it Lessons In Industrial Instrumentation} textbook.  Note the page numbers where important illustrations, photographs, equations, tables, and other relevant details are found.  Prepare to thoughtfully discuss with your instructor and classmates the concepts and examples explored in this reading.

\underbar{file i01022}
%(END_QUESTION)





%(BEGIN_ANSWER)

 
%(END_ANSWER)





%(BEGIN_NOTES)

A {\it zero shift} alters the $b$ term of an instrument's $y = mx + b$ function, resulting in the graph being shifted vertically from ideal.

\vskip 10pt

A {\it span shift} alters the $m$ term of an instrument's $y = mx + b$ function, resulting in the graph being at the wrong slope as compared to ideal.

\vskip 10pt

A {\it linearity} error causes the graphed function to not be a straight line.  Some instruments have linearity adjustements, while others do not.

\vskip 10pt

A {\it hysteresis} error causes the instrument to track differently with an increasing signal as opposed to a decreasing signal.  This is almost always caused by friction in some moving part of the instrument.  Instruments containing flexures may exhibit hysteretic behavior if the flexure becomes cracked or bent.

\vskip 10pt

Calibration errors usually come in more than one form at a time (e.g. zero and hysteresis, etc.).  Zero shift almost always accompanies a realistic calibration error, and so a simple one-point calibration check is often used as a quick method for detecting the presence of calibration errors: if the calibration checks out at one point, the zero is most likely good, which means the likelihood of other errors being present is low.  If an error is detected at any one point, recalibration is necessary.

\vskip 10pt

Differential pressure instruments equipped with 3-valve or 5-valve manifolds may be blocked and equalized to perform single-point checks.


%INDEX% Reading assignment: Lessons In Industrial Instrumentation, Instrument Calibration

%(END_NOTES)


