
%(BEGIN_QUESTION)
% Copyright 2015, Tony R. Kuphaldt, released under the Creative Commons Attribution License (v 1.0)
% This means you may do almost anything with this work of mine, so long as you give me proper credit

Read and outline the ```Marking Versus Outlining a Text'' subsection of the ``Active Reading'' section of the ``Problem-Solving and Diagnostic Strategies'' chapter in your {\it Lessons In Industrial Instrumentation} textbook.  Note the page numbers where important illustrations, photographs, equations, tables, and other relevant details are found.  Prepare to thoughtfully discuss with your instructor and classmates the concepts and examples explored in this reading.

\vskip 30pt

In order to ensure all students are familiar with the concept of ``active reading'', you will be required to write an outline of this section in preparation for today's classroom session and have it ready to show your instructor at the beginning of class.  In other words, you must actively read the textbook section on active reading!  {\it Any outline failing to meet the level of detail shown in the textbook (i.e. summary statements on all the major points written in your own words, \underbar{including questions of your own}) will result in a deduction to today's ``preparatory'' quiz score.}

\underbar{file i01024}
%(END_QUESTION)





%(BEGIN_ANSWER)


%(END_ANSWER)





%(BEGIN_NOTES)

Mortimer Adler wrote ``How to Mark a Book'' which gives tips for actively reading a text.  The source text for this exercise is one page from George Burgess and Henry Louis De Chatelier's 1912 text on temperature measurement.

\vskip 10pt

Highlighting and underlining words may be an effective to help memorize facts and figures presented in a text, but it is a poor strategy for understanding.  All you are doing is emphasizing certain words or phrases.  Adler's advice is to have a conversation with the author as you read: to mark points of agreement and disagreement.  This is what is meant by the phrase {\it active reading}.  

\vskip 10pt

In the reader's second approach to the high-temperature text, underlining and highlighting has been replaced by underlines and comments on those underlined passages.  Here, the reader is asking questions and formulating hypotheses about what they read.  Such points may be raised by the reader once he or she arrives in class with the instructor.  Such notes may be taken on a separate piece of paper, or even typed on computer, if desired.

\vskip 10pt

In the reader's third approach to the high-temperature text, a complete outline has been written, expressing the meaning of the text in the reader's own words.  This outline includes sketches as well as sentences.  Such an outline is a great self-check on understanding, since it will become immediately apparent to the reader if and when they have lost focus on the text (since they won't be able to outline anymore).  Outlining like this takes time, but this is appropriate as deep concepts take time to digest.  Outlines are also great study tools, because they document the reader's impressions of the text for future review where the reader can check to see if their understanding of the subject has changed.










\vfil \eject

\noindent
{\bf Prep Quiz:}

Have students submit their written outlines of this reading assignment, showing how they were able to express the major ideas in their own words.



%INDEX% Reading assignment: Lessons In Industrial Instrumentation, marking versus highlighting a text

%(END_NOTES)

