
%(BEGIN_QUESTION)
% Copyright 2012, Tony R. Kuphaldt, released under the Creative Commons Attribution License (v 1.0)
% This means you may do almost anything with this work of mine, so long as you give me proper credit

Suppose you construct a spring catapult to launch a projectile straight up into the air.  Your catapult uses a winch to tense the spring, and a release mechanism to unleash the spring's stored energy to launch the projectile.  While building this catapult, you are interested in predicting the velocity of the projectile as it leaves the catapult, as well as the maximum height the projectile can attain if shot vertically.

\vskip 10pt

First, identify all the relevant variables you would need to determine the values of in order to calculate both projectile velocity and projectile apogee.

\vskip 50pt

Next, write an equation solving for the velocity of the projectile ($v$) as it leaves the catapult, as a function of the spring's initial compression distance ($x$):

\vskip 10pt

$v = f(x) = $

\vskip 50pt

Finally, write an equation solving for the maximum height (apogee) the projectile is able to attain ($h$) as it leaves the catapult, as a function of the spring's initial compression distance ($x$):

\vskip 10pt

$h = f(x) = $

\vfil

\underbar{file i01110}
\eject
%(END_QUESTION)





%(BEGIN_ANSWER)

This is a graded question -- no answers or hints given!
 
%(END_ANSWER)





%(BEGIN_NOTES)

You would need to know the projectile's mass ($m$) as well as the spring rate ($k$) of the catapult spring.

\vskip 10pt

Relevant formulae:

\vskip 10pt

$E_k = {1 \over 2}mv^2$  (kinetic energy of a moving mass)

\vskip 10pt

$E_p = mgh$  (potential energy of a raised mass)

\vskip 10pt

$E_p = {1 \over 2} kx^2$  (Energy stored in a tensed spring)

\vskip 10pt

Assuming all the spring's energy gets transferred to the projectile as kinetic energy upon leaving the catapult, the spring's stored energy must equal the projectile's kinetic energy at exit velocity $v$:

$${1 \over 2} kx^2 = {1 \over 2}mv^2$$

$$kx^2 = mv^2$$

$${kx^2 \over m} = v^2$$

$$v = \sqrt{kx^2 \over m}$$

\vskip 10pt

Assuming all the spring's energy gets transferred to the projectile as potential energy at its apogee, the spring's stored energy must equal the projectile's potential energy at maximum height $h$:

$${1 \over 2} kx^2 = mgh$$

$$h = {kx^2 \over 2mg}$$

%INDEX% Mathematics review: manipulating and combining equations to form a new equation
%INDEX% Physics, energy, work, power: potential energy transformed to kinetic

%(END_NOTES)


