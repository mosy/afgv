
%(BEGIN_QUESTION)
% Copyright 2014, Tony R. Kuphaldt, released under the Creative Commons Attribution License (v 1.0)
% This means you may do almost anything with this work of mine, so long as you give me proper credit

A challenging concept for many students is determining whether the calculus concept of {\it differentiation} or {\it integration} needs to be applied to a particular problem.  What follows here is a list of real-life applications for one or the other of these two calculus operations.  Consider each application, then choose which calculus operation is proper to obtain the desired result, expressing your answer as a mathematical formula using proper calculus notation.  Note that there is not enough information given to actually calculate a numerical answer -- you will only be able to write a formula for each case:

\vskip 20pt

A radiation sensor outputs a signal corresponding to the intensity of nuclear radiation ($I$) measured in {\it millirem per hour}.  You need to be able to calculate the total dosage of radiation ($D$) measured in {\it millirem} received over one work shift (8:00 AM to 5:00 PM).

\vskip 20pt

An open water storage reservoir loses water due to evaporation on hot, dry days.  A level sensor on the reservoir outputs a volume signal ($V$) in units of {\it gallons}.  You need to be able to calculate the loss rate ($L$) in {\it gallons per hour}.

\vskip 20pt

A new business is about to open, and its proprietor wants to know how many advertising fliers to print and distribute.  A local marketing firm provides statistics on the rate of new customer visits per flier ($v$) typically generated given different numbers of fliers ($n$) distributed within a month.  For low numbers of fliers, very few customers will notice.  For large numbers of fliers, the rate tends to go down as you saturate the advertising space.  Therefore, the function of $v$ plotted over $n$ tends to be a bell curve: low at first, then peaking, then dropping off with increasing $n$.  The proprietor wants to know how many fliers to print within the first month in order to have a certain minimum number of new customer visits ($C$).

\vskip 20pt

A piston of area $A$ is moved distance $x$ to compress a gas inside of a cylinder.  When compressed, the gas pressure ($P$) rises.  You need to calculate the amount of work ($W$) done compressing the gas.

\vskip 20pt \vbox{\hrule \hbox{\strut \vrule{} {\bf Suggestions for Socratic discussion} \vrule} \hrule}

\begin{itemize}
\item{} A good problem-solving strategy for quantitative problems such as this one is to first identify what it is you need to solve, then identify all relevant data, identify all units of measurement, identify any general principles or formulae linking the given information to the solution, and finally identify any ``missing pieces'' to a solution.  Demonstrate how to do all these steps on this problem.
\item{} Explain how {\it units of measurement} are especially in determining when to apply differentiation, versus when to apply integration.
\item{} Graphically interpret each application, showing how the given variables could be represented in a graph, and then how the calculus operation could be applied to each graph.
\end{itemize}

\underbar{file i01111}
%(END_QUESTION)





%(BEGIN_ANSWER)

%(END_ANSWER)





%(BEGIN_NOTES)

$$D = \int_{8:00 AM}^{5:00 PM} I \> dt$$

\vskip 20pt

$$L = {dV \over dt}$$

\vskip 20pt

$$C = \int_0^x v \> dn$$

\vskip 20pt

$$W = \int {P A} \> dx$$

\vskip 10pt

This last example may alternatively be written with the piston area ($A$) written outside of the integrand because it is a constant:

$$W = A \int P \> dx$$

\vskip 10pt

The advertising flyer problem is notable in that the variable we would need to solve for is $x$ -- the number of fliers we would need to print in order to achieve a certain value of customer visits ($C$).  Also noteworthy is the fact that this function is best represented as a discrete summation rather than a continuous integral, since fliers only come in whole-number count:

$$C = \sum_{n=0}^x v \> \Delta n$$

%INDEX% Mathematics, calculus: integral versus derivative in real applications

%(END_NOTES)

