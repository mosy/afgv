
%(BEGIN_QUESTION)
% Copyright 2014, Tony R. Kuphaldt, released under the Creative Commons Attribution License (v 1.0)
% This means you may do almost anything with this work of mine, so long as you give me proper credit

In a series circuit, certain general principles may be stated with regard to quantities of voltage, current, resistance, and power.  Complete these sentences, each one describing a fundamental principle of series circuits:

\vskip 10pt

\noindent
``In a series circuit, voltage . . .''

\vskip 10pt

\noindent
``In a series circuit, current . . .''

\vskip 10pt

\noindent
``In a series circuit, resistance . . .''

\vskip 10pt

\noindent
``In a series circuit, power . . .''

\vskip 10pt

For each of these rules, explain {\it why} it is true.

\underbar{file i01140}
%(END_QUESTION)





%(BEGIN_ANSWER)

\noindent
``In a series circuit, voltage {\it drops add to equal the total}.''

\vskip 10pt

This is an expression of Kirchhoff's Voltage Law (KVL), whereby the algebraic sum of all voltages in any loop must be equal to zero.




\vskip 30pt

\noindent
``In a series circuit, current {\it is equal through all components}.''

\vskip 10pt

This is true because a series circuit by definition has only one path for current to travel.  Since charge carriers must move in unison or not at all (a consequence of the Conservation of Charge, whereby electric charges cannot be created or destroyed), the current measured at any one point in a series circuit must be the same as the current measured at any other point in that same circuit, at any given time.



\vskip 30pt

\noindent
``In a series circuit, resistance{\it s add to equal the total}.''

\vskip 10pt

Each resistance in a series circuit acts to oppose electric current.  When resistances are connected in series, their oppositions combine to form a greater total opposition because then same current must travel through every resistance.



\vskip 30pt

\noindent
``In a series circuit, power {\it dissipations add to equal the total}.''

\vskip 10pt

This is an expression of the Conservation of Energy, which states energy cannot be created or destroyed.  Anywhere power is dissipated in any load of a circuit, that power must be accounted for back at the source, no matter how those loads might be connected to each other.

%(END_ANSWER)





%(BEGIN_NOTES)

Rules of series and parallel circuits are very important for students to comprehend.  However, a trend I have noticed in many students is the habit of memorizing rather than understanding these rules.  Students will work hard to memorize the rules without really comprehending {\it why} the rules are true, and therefore often fail to recall or apply the rules properly.

%INDEX% Electronics review: series and parallel circuits

%(END_NOTES)


