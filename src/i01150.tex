
%(BEGIN_QUESTION)
% Copyright 2014, Tony R. Kuphaldt, released under the Creative Commons Attribution License (v 1.0)
% This means you may do almost anything with this work of mine, so long as you give me proper credit

In a parallel circuit, certain general principles may be stated with regard to quantities of voltage, current, resistance, and power.  Complete these sentences, each one describing a fundamental principle of parallel circuits:

\vskip 10pt

\noindent
``In a parallel circuit, voltage . . .''

\vskip 10pt

\noindent
``In a parallel circuit, current . . .''

\vskip 10pt

\noindent
``In a parallel circuit, resistance . . .''

\vskip 10pt

\noindent
``In a parallel circuit, power . . .''

\vskip 10pt

For each of these rules, explain {\it why} it is true.

\underbar{file i01150}
%(END_QUESTION)





%(BEGIN_ANSWER)

\noindent
``In a parallel circuit, voltage {\it is equal across all components}.''

\vskip 10pt

This is true because a parallel circuit by definition is one where the constituent components all share the same two equipotential points.




\vskip 30pt

\noindent
``In a parallel circuit, current{\it s add to equal the total}.''

\vskip 10pt

This is an expression of Kirchhoff's Current Law (KCL), whereby the algebraic sum of all currents entering and exiting a node must be equal to zero.



\vskip 30pt

\noindent
``In a parallel circuit, resistance{\it s diminish to equal the total}.''

\vskip 10pt

Each resistance in a parallel circuit provides another path for electric current.  When resistances are connected in parallel, their combined total paths provide less opposition than any single path because the current is able to split up and proportionately follow these alternative paths.



\vskip 30pt

\noindent
``In a parallel circuit, power {\it dissipations add to equal the total}.''

\vskip 10pt

This is an expression of the Conservation of Energy, which states energy cannot be created or destroyed.  Anywhere power is dissipated in any load of a circuit, that power must be accounted for back at the source, no matter how those loads might be connected to each other.

%(END_ANSWER)





%(BEGIN_NOTES)

Rules of series and parallel circuits are very important for students to comprehend.  However, a trend I have noticed in many students is the habit of memorizing rather than understanding these rules.  Students will work hard to memorize the rules without really comprehending {\it why} the rules are true, and therefore often fail to recall or apply the rules properly.

%INDEX% Electronics review: series and parallel circuits

%(END_NOTES)


