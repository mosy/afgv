
%(BEGIN_QUESTION)
% Copyright 2014, Tony R. Kuphaldt, released under the Creative Commons Attribution License (v 1.0)
% This means you may do almost anything with this work of mine, so long as you give me proper credit

Explain why water (H$_{2}$O) will not ignite, despite the fact that it is comprised of hydrogen which is known to be extremely flammable and oxygen which is known to greatly accelerate combustion.  With these two substances embedded within each water molecule, why isn't water explosive?

\vskip 20pt \vbox{\hrule \hbox{\strut \vrule{} {\bf Suggestions for Socratic discussion} \vrule} \hrule}

\begin{itemize}
\item{} Write a balanced chemical equation showing the electrolysis of water into hydrogen and oxygen gas.  Identify whether the $\Delta H$ value for this equation will be a positive or a negative quantity, and what this quantity means.
\item{} Relate your answer here to the case of carbon dioxide (CO$_{2}$), comprised of carbon (C -- a flammable element) and oxygen (O -- a combustion accelerant).
\end{itemize}

\underbar{file i01199}
%(END_QUESTION)





%(BEGIN_ANSWER)

The answer to this question is rooted in an understanding of what combustion is, and how energy gets released in a chemical reaction such as combustion between hydrogen (H) and oxygen (O).
 
%(END_ANSWER)





%(BEGIN_NOTES)

Combustion -- like any other exothermic chemical process -- involves the formation of bonds between atoms previously separated.  Atoms bound together have a lower energy state than those same atoms separated from each other, and so the process of bond formation releases energy as those atoms transition from a higher-energy state to a lower-energy state.

\vskip 10pt

The reason water doesn't combust is because {\it it has already burned}.  That water molecule formed when two atoms of hydrogen bound to an atom of oxygen and released energy.  There is no energy left to release because those hydrogen and oxygen atoms have already reached their low-energy state in the process of forming water.

\vskip 10pt

It is only when those atoms exist separated from each other that there is the potential for combustion (energy release).

%INDEX% Chemistry, basic: molecular bonds and energy exchange

%(END_NOTES)


