
%(BEGIN_QUESTION)
% Copyright 2012, Tony R. Kuphaldt, released under the Creative Commons Attribution License (v 1.0)
% This means you may do almost anything with this work of mine, so long as you give me proper credit

Everything I know about troubleshooting, I learned from Tyler Durden (with respects to the movie {\it Fight Club}):

\vskip 10pt

\centerline{\bf The Rules of \underbar{Fault Club}}

\begin{itemize}
\item{$(1)$} Don't try to find the fault by looking for it -- perform diagnostic tests instead
\vskip 10pt
\item{$(1)$} {\it Don't try to find the fault by looking for it -- perform diagnostic tests instead!}
\vskip 10pt
\item{$(3)$} The troubleshooting is over when you have correctly identified the nature and location of the fault
\vskip 10pt
\item{$(4)$} It's just you and the fault -- don't ask for help until you have exhausted your resources
\vskip 10pt
\item{$(5)$} Assume one fault at a time in a proven system, unless the data proves otherwise
\vskip 10pt
\item{$(6)$} No new components allowed -- replacing suspected bad components with new is a waste of time and money
\vskip 10pt
\item{$(7)$} We will practice as many times as we have to until you master this
\vskip 10pt
\item{$(8)$} Troubleshooting is not a spectator sport: \underbar{\it you} have to troubleshoot!
\end{itemize}

\vskip 10pt

For each of these rules, explain their rationale.

\underbar{file i01203}
%(END_QUESTION)





%(BEGIN_ANSWER)

{\bf Rules 1 and 2:} A very bad tendency of novice troubleshooters is to waste time visually looking for that which cannot be seen.  Most faults in a system are difficult if not impossible to visually spot.  Instead, the proficient troubleshooter always relies on test data and logical reasoning to identify the fault, all the way to the very end.

\vskip 10pt

{\bf Rule 3:} Novice troubleshooters tend to stop prematurely.  If you cannot pinpoint the exact nature and location of the fault, either you do not know enough about the subsystem you're trying to diagnose, or you need to continue troubleshooting.

\vskip 10pt

{\bf Rule 4:} Novice troubleshooters tend to ask for others' help prematurely.  You are not going to build this vital skill unless and until you are forced to do the thinking on your own!

\vskip 10pt

{\bf Rule 5:} This is nothing more than {\it Occam's Razor} applied to diagnostics.  Single faults are generally more probable than coincidental faults, especially on previously-working systems.

\vskip 10pt

{\bf Rule 6:} Incompetent troubleshooters simply replace all the components they can find until the problem goes away.  This is incredibly wasteful, and worse yet does nothing to develop skill.

\vskip 10pt

{\bf Rule 7:} Practice makes perfect.  Troubleshooting is a difficult skill to master, and it takes lots of time to develop.

\vskip 10pt

{\bf Rule 8:} Never fool yourself by thinking you have learned something complex just because you have watched someone else do it!  Unless and until you have done something yourself, {\it you don't know it.}

\vskip 10pt

 
%(END_ANSWER)





%(BEGIN_NOTES)

\noindent
{\bf Original rules from the movie ``Fight Club''}

\begin{itemize}
\item{$(1)$} You do not talk about Fight Club
\vskip 10pt
\item{$(2)$} {\it You do not talk about Fight Club}
\vskip 10pt
\item{$(3)$} If someone yells ``Stop,'' goes limp, taps out, the fight is over
\vskip 10pt
\item{$(4)$} Only two guys to a fight
\vskip 10pt
\item{$(5)$} One fight at a time
\vskip 10pt
\item{$(6)$} No shirts, no shoes
\vskip 10pt
\item{$(7)$} Fights will go on as long as they have to
\vskip 10pt
\item{$(8)$} If this is your first night at Fight Club, you have to fight!
\end{itemize}


%INDEX% Troubleshooting review: rules of "Fault Club"

%(END_NOTES)


