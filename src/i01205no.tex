
%(BEGIN_QUESTION)
% Copyright 2012, Tony R. Kuphaldt, released under the Creative Commons Attribution License (v 1.0)
% This means you may do almost anything with this work of mine, so long as you give me proper credit

%Suppose an electronic pressure transmitter has an input range of 0 to 400 PSI and an output range of 4 to 20 mA.  When subjected to a series of known pressures to obtain an ``As-Found'' calibration table, it responds as such:

En trykktransmitter har et m{\aa}leomfang p{\aa} 0 til 400 bar og et signalomfang p{\aa} 4-20 mA. En 5-punkts opp/ned sjekk, gir den f{\o}lgende tabell. 

% No blank lines allowed between lines of an \halign structure!
% I use comments (%) instead, so that TeX doesn't choke.

$$\vbox{\offinterlineskip
\halign{\strut
\vrule \quad\hfil # \ \hfil & 
\vrule \quad\hfil # \ \hfil \vrule \cr
\noalign{\hrule}
%
% First row
Applied pressure & Output signal \cr
%
% Another row
(PSI) & (mA) \cr
%
\noalign{\hrule}
%
% Another row
0 & 4.0 \cr
%
\noalign{\hrule}
%
% Another row
100 & 8.0 \cr
%
\noalign{\hrule}
%
% Another row
200 & 12.0 \cr
%
\noalign{\hrule}
%
% Another row
300 & 16.0 \cr
%
\noalign{\hrule}
%
% Another row
400 & 20.0 \cr
%
\noalign{\hrule}
%
% Another row
300 & 16.1 \cr
%
\noalign{\hrule}
%
% Another row
200 & 12.1 \cr
%
\noalign{\hrule}
%
% Another row
100 & 8.1 \cr
%
\noalign{\hrule}
%
% Another row
0 & 4.1 \cr
%
\noalign{\hrule}
} % End of \halign 
}$$ % End of \vbox

\vskip 10pt

%Identify the type of calibration error this transmitter suffers from.
Hvilken type kalibreringsfeil har transmitteren?

\vfil 

\underbar{file i01205no}
\eject
%(END_QUESTION)





%(BEGIN_ANSWER)

This is a graded question -- no answers or hints given!

%(END_ANSWER)





%(BEGIN_NOTES)

Note how the instrument's response is exactly what it should be as it is stimulated from LRV to URV, but exhibits a high error on the way back down to the LRV.  This is characteristic of a {\it hysteresis} error: where the instrument responds differently going up as it does going down.

%INDEX% Calibration: basic terms and graphing
%INDEX% Calibration errors, identifying

%(END_NOTES)


