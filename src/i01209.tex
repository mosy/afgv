
%(BEGIN_QUESTION)
% Copyright 2012, Tony R. Kuphaldt, released under the Creative Commons Attribution License (v 1.0)
% This means you may do almost anything with this work of mine, so long as you give me proper credit

Suppose you are configuring a control system's analog input function block to receive the 4-20 mA signal from a temperature transmitter ranged 150 $^{o}$F to 550 $^{o}$F.  In order to properly ``scale'' this milliamp signal into a measurement value expressed in units of degrees Fahrenheit, the analog input function block requires a linear formula in the form $y = mx + b$ to relate milliamps of signal to degrees Fahrenheit of temperature.

\vskip 10pt

Write such a formula, where $x$ is milliamps (range = 4 to 20) and $y$ is degrees Fahrenheit (range = 150 to 550):

\vskip 10pt

$y =$

\vskip 20pt

Now suppose you are faced with a slightly different challenge, where the $x$ value read by the controller's analog input block comes from a 16-bit analog-to-digital converter with a full-scale range of 0.0 to 21.5 milliamps.  The temperature transmitter is still ranged 150 $^{o}$F to 550 $^{o}$F, 4 to 20 mA.  Write a scaling formula where $x$ is the decimal ``count'' value coming from the ADC and $y$ is in degrees Fahrenheit (range = 150 to 550):

\vskip 10pt

$y =$

\underbar{file i01209}
%(END_QUESTION)





%(BEGIN_ANSWER)

First scenario (mA to deg F):

$$y = {400 \over 16}x + 50 \hskip 50pt y = 25x + 50$$

\vskip 10pt

Second scenario (ADC counts to deg F):

$$y = {400 \over 48770}x + 50 \hskip 50pt = 0.008202x + 50$$

%(END_ANSWER)





%(BEGIN_NOTES)

{\bf This question is intended for exams only and not worksheets!}.

%(END_NOTES)


