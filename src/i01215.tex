
%(BEGIN_QUESTION)
% Copyright 2012, Tony R. Kuphaldt, released under the Creative Commons Attribution License (v 1.0)
% This means you may do almost anything with this work of mine, so long as you give me proper credit

Identify where a HART multivariable transmitter is used in this incinerator process, and how the multiple variables are extracted from it to report on various indicating instruments:

$$\includegraphics[width=15.5cm]{i0004rx01.eps}$$

\underbar{file i01215}
%(END_QUESTION)





%(BEGIN_ANSWER)

Flow transmitter FT-38 is a Coriolis mass flow transmitter, simultaneously measuring mass flow rate, density, and temperature.  Mass flow is the primary variable reported by this transmitter in 4-20 mA form, while the secondary and tertiary variables of density and temperature are extracted by FY-38c (a HART-to-analog converter) and reported to indicating recorders DIR-38 and TIR-38 in 4-20 mA form.

%(END_ANSWER)





%(BEGIN_NOTES)

\vskip 20pt \vbox{\hrule \hbox{\strut \vrule{} {\bf Virtual Troubleshooting} \vrule} \hrule}

This question is a good candidate for a ``Virtual Troubleshooting'' exercise.  Presenting the diagram to students, you first imagine in your own mind a particular fault in the system.  Then, you present one or more symptoms of that fault (something noticeable by an operator or other user of the system).  Students then propose various diagnostic tests to perform on this system to identify the nature and location of the fault, as though they were technicians trying to troubleshoot the problem.  Your job is to tell them what the result(s) would be for each of the proposed diagnostic tests, documenting those results where all the students can see.

During and after the exercise, it is good to ask students follow-up questions such as:

\begin{itemize}
\item{} What does the result of the last diagnostic test tell you about the fault?
\item{} Suppose the results of the last diagnostic test were different.  What then would that result tell you about the fault?
\item{} Is the last diagnostic test the best one we could do?
\item{} What would be the ideal order of tests, to diagnose the problem in as few steps as possible?
\end{itemize}


%INDEX% Fieldbus, HART: multivariable transmitter (realistic P&ID)
%INDEX% Process: incinerator (realistic P&ID shown)

%(END_NOTES)





