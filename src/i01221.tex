
%(BEGIN_QUESTION)
% Copyright 2012, Tony R. Kuphaldt, released under the Creative Commons Attribution License (v 1.0)
% This means you may do almost anything with this work of mine, so long as you give me proper credit

Suppose a FOUNDATION Fieldbus pressure transmitter is connected to the bottom of a liquid storage vessel for the purpose of measuring liquid height based on hydrostatic pressure.  This transmitter happens to be configured with the following Analog Input (AI) block parameters:

% No blank lines allowed between lines of an \halign structure!
% I use comments (%) instead, so that TeX doesn't choke.

$$\vbox{\offinterlineskip
\halign{\strut
\vrule \quad\hfil # \ \hfil & 
\vrule \quad\hfil # \ \hfil \vrule \cr
\noalign{\hrule}
%
% First row
{\tt L\_Type} & Indirect \cr
%
\noalign{\hrule}
%
% Another row
{\tt XD\_Scale} & 38 to 183.488 inches WC \cr
%
\noalign{\hrule}
%
% Another row
{\tt OUT\_Scale} & 0 to 14 feet \cr
%
\noalign{\hrule}
} % End of \halign 
}$$ % End of \vbox

\vskip 10pt

Calculate the height reported by this transmitter when it senses a hydrostatic pressure of 71.8 inches WC.

\vskip 10pt

Calculate the hydrostatic pressure sensed by this transmitter when it reports a liquid height of 9.3 feet.

\vskip 10pt


\underbar{file i01221}
%(END_QUESTION)





%(BEGIN_ANSWER)

Calculate the height reported by this transmitter when it senses a hydrostatic pressure of 71.8 inches WC.  {\bf Height = 3.2525 feet}

\vskip 10pt

Calculate the hydrostatic pressure sensed by this transmitter when it reports a liquid height of 9.3 feet.  {\bf Pressure = 134.65 inches WC}

%(END_ANSWER)





%(BEGIN_NOTES)


%INDEX% Fieldbus, instrument ranging: setting XD_Scale and OUT_Scale parameters for an application

%(END_NOTES)


