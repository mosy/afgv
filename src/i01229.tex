
%(BEGIN_QUESTION)
% Copyright 2016, Tony R. Kuphaldt, released under the Creative Commons Attribution License (v 1.0)
% This means you may do almost anything with this work of mine, so long as you give me proper credit

The maintenance of a working lab facility is extensive, especially for a program such as Instrumentation, where most of the equipment comes in the form of donations which must be pieced together, and where many of the systems are custom-built for the purpose.  Every student bears a responsibility for helping maintain the lab facility, because every student benefits from its provisions.

On the last day of every quarter, time is allocated to the clean-up and re-organization of the lab facility.  This is a full work day, with attendance enforced as per usual.  In order to help students focus on the tasks that need to be done, the following list documents some of the work necessary to make the lab ready for next quarter.  Tasks preceded by a blank line will be assigned to lab teams for completion.

\vskip 10pt

\noindent
{\bf Lab tasks}

\begin{itemize}
\item{} Check to see that all small items bearing BTC inventory tags are painted a bright color to make them easy to spot for each year's inventory check.
\item{} \underbar{\hskip 50pt} Sweep all lab floor areas, recycling or discarding any waste material.
\item{} \underbar{\hskip 50pt} Sweep all storage room floor areas, recycling or discarding any waste material.  Place items found on floor back on shelves where they belong.
\item{} \underbar{\hskip 50pt} Collect all copper tube segments and place them in the copper/brass recycling receptacle.
\item{} \underbar{\hskip 50pt} Collect all aluminum, stainless steel, brass, and copper wire scrap (pieces shorter than 1 foot) in the scrap metal buckets near the north-west exit door.
\item{} \underbar{\hskip 50pt} Haul recyclable metals to a local scrap dealer, and return with cash to buy pizza for today's lunch.
\item{} \underbar{\hskip 50pt} Organize storage bins for danger tags and masking tape.  Collect any unused danger tags from around the lab room and place them in that bin.
\item{} \underbar{\hskip 50pt} Help search for any missing Team Tool Locker items.
\item{} \underbar{\hskip 50pt} Clean all workbench and table surfaces.
\item{} \underbar{\hskip 50pt} Remove items from the compressor room, sweep the floor, and make sure there is no junk being stored there.
\item{} \underbar{\hskip 50pt} Collect lengths of cable longer than 1 foot and place in the storage bins inside the DCS cabinets for future use.
\item{} \underbar{\hskip 50pt} Re-organize wire spool storage area: remove any empty spools from the rack, ensure all boxes and unmounted spools are neatly stacked on the floor.
\item{} \underbar{\hskip 50pt} Collect all plastic tubes and return them to the appropriate storage bin.
\item{} \underbar{\hskip 50pt} Re-organize tube fitting drawers (north-west corner of lab room), ensuring no pipe fittings are mixed in, that all fittings are found in the proper drawers, and that all drawers are properly labeled (these drawers should have sample fittings attached to the fronts).
\item{} \underbar{\hskip 50pt} Re-organize pipe fitting drawers (north-west corner of lab room), ensuring no tube fittings are mixed in.
\item{} \underbar{\hskip 50pt} Re-organize hose fitting drawers (north-west corner of lab room).
\item{} \underbar{\hskip 50pt} Re-organize terminal block and ice-cube relay drawers (north end of lab room).
\item{} \underbar{\hskip 50pt} Drain condensed water out of air compressor tank (in the compressor room).
\item{} \underbar{\hskip 50pt} Return all books and manuals to bookshelves.
\item{} \underbar{\hskip 50pt} Inspect each and every control panel in the lab, removing all wiring except for those which should be permanently installed (120 VAC power, signal cables between junction boxes).  Ensure that each junction box's power cords are securely fastened and grounded.
\item{} \underbar{\hskip 50pt} Inspect each and every signal wiring junction box in the lab, removing all wiring except for those which should be permanently installed (e.g. 120 VAC power, signal cables between junction boxes.).  Ensure that each junction box's power cords are securely fastened and grounded.
\item{} \underbar{\hskip 50pt} Check condition of labels on all junction boxes and control panels, making new labels if the old labels are missing, damaged, or otherwise hard to read.
\item{} \underbar{\hskip 50pt} Check condition of labels on all permanently-installed cables (e.g. between junction boxes), making new labels if the old labels are missing, damaged, or otherwise hard to read.
\item{} \underbar{\hskip 50pt} Check condition of labels on all terminal blocks inside control panels and junction boxes, making new labels if the old labels are missing or otherwise hard to read.
\item{} \underbar{\hskip 50pt} Remove all debris left in control panels and junction boxes throughout the lab room, using a vacuum cleaner if necessary.
\item{} \underbar{\hskip 50pt} Clean up deadweight testers (they tend to leak oil).  {\it Hint: WD-40 works nicely as a solvent to help clean up any leaked oil.}
\item{} \underbar{\hskip 50pt} Maintenance on turbocompressor system: (safety tag-out, check oil level, repair any oil leaks, repair any poor wire connections, clean debris out of control cabinet, re-tighten all power terminal connections).
\item{} \underbar{\hskip 50pt} Return all shared tools (e.g. power drills, saws) to the proper storage locations (hand tools to the tool drawer in the north-east corner of the lab room, and power tools to the tool shelf in the upstairs storage area).
\item{} \underbar{\hskip 50pt} Remove items from all storage cabinets on the north end of the lab room, cleaning all shelves of junk (e.g. pH probes that have been left dry) and returning all items to their proper places.  Install covers on all transmitters missing them, especially on pneumatic transmitters which are vulnerable to damage without their covers attached.
\item{} \underbar{\hskip 50pt} Visually inspect all general-purpose pressure regulators stored in the north storage shelves for missing adjustment bolts, missing tube connectors, damaged port threads, etc.  Make repairs as necessary.
\item{} \underbar{\hskip 50pt} Test all pressure transmitters not labeled ``good'' to see if they are indeed defective.  Repair if possible, salvage parts and discard if not.  {\it Do not discard any instrument with a BTC inventory tag!}
\item{} \underbar{\hskip 50pt} Test all temperature transmitters not labeled ``good'' to see if they are indeed defective.  Repair if possible, salvage parts and discard if not.  {\it Do not discard any instrument with a BTC inventory tag!}
\item{} \underbar{\hskip 50pt} Test all I/P converters not labeled ``good'' to see if they are indeed defective.  Repair if possible, salvage parts and discard if not.  {\it Do not discard any instrument with a BTC inventory tag!}
\item{} \underbar{\hskip 50pt} Test all precision pressure gauges not labeled ``good'' to see if they are indeed defective.  Repair if possible, salvage parts and discard if not.  {\it Do not discard any instrument with a BTC inventory tag!}
\item{} \underbar{\hskip 50pt} Test all precision pressure regulators not labeled ``good'' to see if they are indeed defective.  Repair if possible, salvage parts and discard if not.  {\it Do not discard any instrument with a BTC inventory tag!}
\item{} \underbar{\hskip 50pt} Return all field instruments (e.g. transmitters) and miscellaneous devices (e.g. pressure gauges and regulators) to their proper storage locations.  {\it Note that I/P transducers and valve positioners should remain near their respective control valves rather than be put away in storage!}
\item{} \underbar{\hskip 50pt} Store all 2$\times$2 foot plywood process boards in secure locations, ensuring each one is ready to use next quarter.
\item{} \underbar{\hskip 50pt} Ensure that each and every control valve mounted on the racks in the lab room has an I/P transducer mounted nearby, complete with Swagelok tube connectors in good condition for connecting compressed air supply and signal to the valve.  
\item{} \underbar{\hskip 50pt} Check to make sure that each valve is securely mounted to the rack, and if there is a positioner attached that the feedback arm is properly connected to the valve stem (e.g. no missing tension springs, bent linkages, obvious misalignments).
\item{} \underbar{\hskip 50pt} Remove all items from the flammables cabinet, wipe all shelves of liquid and reside, then re-stock in a neat and safe manner.
\item{} \underbar{\hskip 50pt} Clean all bar-be-que grills of residue left over from lunches and fundraisers.  {\it Note: you may need to take the grill racks and grease drip trays to a car wash station and use the engine degreaser solution to clean them thoroughly enough!}
\item{} \underbar{\hskip 50pt} Re-set all function block parameters in the DCS ``Generic Loops'' to their default settings.  See the documentation on the main BTC\_PPlus workstation for instructions on parameter values.
\item{} \underbar{\hskip 50pt} Check manometers on the calibration bench, ensuring those filled with red fluid are at their fluid levels and that all the others (normally filled with distilled water) are completely drained.
\item{} \underbar{\hskip 50pt} Turn on compressed air to the calibration bench, checking for leaks and ensuring every pressure regulator is functioning as it should.
\item{} \underbar{\hskip 50pt} Clean refrigerators, throwing away any food items remaining within.
\item{} \underbar{\hskip 50pt} Thoroughly clean all food ovens and any other cooking tools.
\item{} \underbar{\hskip 50pt} Return all shelf boards to their appropriate places on the racks.
\item{} \underbar{\hskip 50pt} Clean and re-organize all shelves in classroom DMC 130 storing components for the hands-on mastery assessments.  Throw away any damaged jumper wires, battery clips, etc.  Discard any batteries whose terminal voltages are less than 80\% of their rating (e.g. less than 7.2 volts for a 9-volt battery). 
\item{} \underbar{\hskip 50pt} Shut off power to all control systems except for the DCS.
\item{} \underbar{\hskip 50pt} Store any donated components in the proper locations.
\item{} \underbar{\hskip 50pt} Clean all whiteboards using Windex, so they actually look white again!
\end{itemize}

\filbreak

\begin{itemize}
\item{} {\it Instructors may add items to this list as necessary:}
\vskip 10pt
\item{} \underbar{\hskip 50pt} 
\vskip 10pt
\item{} \underbar{\hskip 50pt} 
\vskip 10pt
\item{} \underbar{\hskip 50pt} 
\vskip 10pt
\item{} \underbar{\hskip 50pt} 
\vskip 10pt
\item{} \underbar{\hskip 50pt} 
\vskip 10pt
\item{} \underbar{\hskip 50pt} 
\vskip 10pt
\item{} \underbar{\hskip 50pt} 
\end{itemize}


\filbreak

\noindent
{\bf Personal tasks}

\begin{itemize}
\item{} Apply ``sick hours'' to missed time this quarter (remember, this is {\it not} automatically done for you!).
\item{} Donate unused ``sick hours'' to classmates in need.
\item{} Take any quizzes missed due to classroom absence this quarter (remember, a quiz not taken will be counted as a failed quiz!).
\end{itemize}

\underbar{file i01229}
%(END_QUESTION)





%(BEGIN_ANSWER)

 
%(END_ANSWER)





%(BEGIN_NOTES)

I recommend pre-assigning at least two tasks to every lab team so that you don't get people hanging around doing nothing while others do all the work.

%INDEX% Lab clean-up and reorganization, work list

%(END_NOTES)


