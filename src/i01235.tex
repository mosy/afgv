
%(BEGIN_QUESTION)
% Copyright 2013, Tony R. Kuphaldt, released under the Creative Commons Attribution License (v 1.0)
% This means you may do almost anything with this work of mine, so long as you give me proper credit

One of the unique features of this program is the inclusion of {\it mastery exams}, where students must answer questions with 100\% accuracy in order to pass.  Conventional ``proportional'' exams allow students to pass if a certain minimum score is achieved.  The problem with this testing strategy is that students may not actually learn {\it all} the concepts they're supposed to, but may still pass the exam if they are strong enough in the other concepts covered in that assessment.  The purpose of mastery exams is to guarantee proficiency in {\it all} critical concepts, by requiring that students achieve 100\% accuracy in order to pass.

\vskip 10pt

Your instructor will hand out copies of the mastery exam for the INST200 Introduction to Instrumentation course, covering several critical concepts of circuit analysis taught in the first year of the Instrumentation program.  Do your best to answer all the questions correctly.  If you get any incorrect on the first attempt, the instructor will mark which {\it sections} (not which {\it questions}) you missed and return it to you for one more attempt.  If it is not passed by the second attempt, it counts as a failed exam and you would ordinarily re-take a different version of the same exam on a different day.

\vskip 10pt

Mastery exams may be re-taken any number of times with no grade penalty.  The purpose is to give students the constructive feedback and practice that they need in order to master all the concepts.  However, the mastery exam must be passed by the specific deadline (usually the date of the next course exam) in order to receive a passing grade for that course.

\vskip 10pt

For the INST230 course, this mastery exam is just for practice.  During the INST200 course (in Fall, Winter, and Spring quarters) this same exam must be passed in its entirety in order to earn a passing grade for the INST200 course, with re-take options scheduled throughout the quarter.  

\vskip 10pt

During the Summer quarter course sequence (INST230, 231, and 232) the concept of relay ladder logic is very important, and so you should pay particular attention to question \#9 on the INST200 mastery exam.  If you have not mastered the concept of how relay circuits work and are analyzed (the same concepts covered in the ELTR140 Digital 1 course during first year), you will have a great deal of difficulty grasping motor controls and PLC programming this quarter.

\vskip 10pt

\underbar{file i01235}
%(END_QUESTION)





%(BEGIN_ANSWER)

 
%(END_ANSWER)





%(BEGIN_NOTES)


%INDEX% Course organization, assessment: mastery (practice)
%INDEX% Mastery exam practice

%(END_NOTES)


