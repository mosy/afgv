
%(BEGIN_QUESTION)
% Copyright 2012, Tony R. Kuphaldt, released under the Creative Commons Attribution License (v 1.0)
% This means you may do almost anything with this work of mine, so long as you give me proper credit

Electrical power distribution ``grids'' were early-adoption platforms for SCADA technology, given the need to communication mission-critical information quickly over long distances.  With modern (digital) protective relay devices, measurement instrumentation, and substation automation controllers in popular use throughout electric power grids, the need to communicate large amounts of system data over reliable channels is more pressing than ever.

\vskip 10pt

Three forms of system data communication are prevalent in the field of electric power distribution, listed in order of their historical introduction (from earliest to latest):

\vskip 10pt

\begin{itemize}
\item{} {\bf Power Line Carrier} (high-frequency AC signals used to encode data along existing power line conductors)
\vskip 10pt
\item{} {\bf Microwave radio} (point-to-point communication between parabolic dish antennas)
\vskip 10pt
\item{} {\bf Fiber optic cables} (pulses of light used to communicate digital 1's and 0's between locations, along glass fiber cables)
\end{itemize}

\vskip 10pt

Identify why these forms of data communication are each well-suited to the task of critical system data communication between substations and other centers of an electric power grid.  One criterion to keep in mind when assessing these modes is resistance to disruption from natural disasters.

\vskip 20pt \vbox{\hrule \hbox{\strut \vrule{} {\bf Suggestions for Socratic discussion} \vrule} \hrule}

\begin{itemize}
\item{} One of these communication modes bears a striking resemblance to the HART standard of instrument data communication.  Identify which one it is, and explain the similarity.
\item{} Identify what a malicious person would have to do to intercept data communicated along these three different modes.  In other words, how {\it secure} is each form of data communication?
\item{} Identify what a malicious person would have to do to interrupt data communicated along these three different modes.  In other words, how {\it robust} is each form of data communication?
\item{} Compare the {\it bandwidth} (i.e. the data transmission rate) of these three communication modes.
\item{} Can you think of any alternate modes of communcation that could be used as alternatives to any of these three, to communicate critical data between power generating stations, substations, and large industrial load centers?
\end{itemize}

\underbar{file i01240}
%(END_QUESTION)





%(BEGIN_ANSWER)

Power line carrier (``PLC'') is the form most closely resembling HART communication, as AC signals are carried simultaneously along the same conductors as the power.  This early technology makes a lot of sense, because the path of communication is the power lines themselves.  If any natural disaster is severe enough to knock power lines down to the ground, then communication for that power route is a moot point.

\vskip 10pt

Microwave radio communication and fiber optic lines both enjoy tremendous bandwidth for signals (much higher than power line carrier!).  Microwave antennas will work well provided their high-gain dish antennas are precisely aimed at each other.  Earthquakes, tornados, and other forces of nature might knock dish antennas out of alignment, but otherwise this is an extremely reliable and secure mode of communication.  Fiber optic lines are typically buried underground, where they are immune to things like tornados, but might be severed in extreme earthquakes.  Like microwave communication, fiber optic is extremely secure as well.

%(END_ANSWER)





%(BEGIN_NOTES)


%INDEX% Electronics review: fiber optics
%INDEX% Electronics review: microwave communications
%INDEX% Electronics review: powerline carrier

%(END_NOTES)

