
%(BEGIN_QUESTION)
% Copyright 2012, Tony R. Kuphaldt, released under the Creative Commons Attribution License (v 1.0)
% This means you may do almost anything with this work of mine, so long as you give me proper credit

Examine the schematic diagrams in Section 9 (pages 49 to 65) of the Pepperl+Fuchs ``Process Automation Engineer's Guide'' on intrinsic safety, and then identify some of the field instrument types (other than 4-20 mA loop-powered transmitters) may be protected by the appropriate type of intrinsic safety barrier.

\vskip 10pt

Examine the minimum ignition curves shown on pages 80 to 84.  Based on the data shown here, how much current is permissible in a resistive circuit for a Group IIC (hydrogen gas) environment when the maximum possible voltage is 28 volts DC?

\vskip 20pt \vbox{\hrule \hbox{\strut \vrule{} {\bf Suggestions for Socratic discussion} \vrule} \hrule}

\begin{itemize}
\item{} Figure 9.18 on page 51 shows the ``most common method'' of connecting a zener diode barrier to a 4-20 mA loop-powered transmitter.  Identify how this circuit differs from the simpler version shown in figure 9.20 on page 52 (the ``most efficient method'' of connecting 2-wire transmitters).
\item{} Figure 9.21 on page 52 shows a method for connecting a zener diode barrier to a 4-20 mA loop-powered transmitter when the I/O module requires a voltage signal input (as opposed to a current signal input).  Identify where the 250 ohm load resistor is located in this circuit, necessary to generate a precision 1-5 VDC signal from the transmitter's 4-20 mA output.
\item{} Figure 9.49 on page 57 shows a zener diode barrier for use with 4-wire RTD sensors.  Explain how this circuit functions, and how that barrier thoroughly protects the RTD from generating a hazardous-energy spark in the hazardous area.
\end{itemize}

\underbar{file i01241}
%(END_QUESTION)




%(BEGIN_ANSWER)


%(END_ANSWER)





%(BEGIN_NOTES)

Devices other than loop-powered transmitters include:

\begin{itemize}
\item{} Switches
\item{} Potentiometers
\item{} Load cells (strain gauge networks)
\item{} Thermocouples
\item{} Vibration sensors
\item{} Speed sensors
\item{} Pulse generators
\item{} I/P converters
\item{} Solenoids
\item{} Lamps
\item{} Audible alerts
\end{itemize}

\vskip 10pt

For a resistive circuit in a hydrogen gas environment, 28 VDC allows up to 175 mA (according to the curve), but the stated maximum permissible working current must be only two-thirds of this value, so the answer is 116.67 mA.

%INDEX% Reading assignment: Pepperl+Fuchs Engineer's Guide (intrinsic safety)

%(END_NOTES)


