
%(BEGIN_QUESTION)
% Copyright 2015, Tony R. Kuphaldt, released under the Creative Commons Attribution License (v 1.0)
% This means you may do almost anything with this work of mine, so long as you give me proper credit

Read and outline the introduction and the ``Potential Transformers'' subsection of the ``Electrical Sensors'' section of the ``Electric Power Measurement and Control'' chapter in your {\it Lessons In Industrial Instrumentation} textbook.  Note the page numbers where important illustrations, photographs, equations, tables, and other relevant details are found.  Prepare to thoughtfully discuss with your instructor and classmates the concepts and examples explored in this reading.

\underbar{file i01243}
%(END_QUESTION)




%(BEGIN_ANSWER)


%(END_ANSWER)





%(BEGIN_NOTES)

A potential (or voltage) transformer steps down and isolates high line voltages to a level that is safe to connect to panel-mounted instruments such as voltmeters.  The fixed ratio of a PT (or VT) provides a proportional representation of either phase or line voltage (depending how the transformer is connected to the high-voltage power conductors).  120 VAC is the standard in industry for representing 100\% normal line voltage.  This standard scaling allows interoperability between different brands and models of PTs and panel-mounted instruments.

\vskip 10pt

A CCVT (capacitively coupled voltage transformer) is used in very high voltage applications to further reduce line voltage down to the 120 VAC nominal signal voltage.  It uses a set of series capacitors as a series voltage divider prior to the step-down transformer.







\vskip 20pt \vbox{\hrule \hbox{\strut \vrule{} {\bf Suggestions for Socratic discussion} \vrule} \hrule}

\begin{itemize}
\item{} What is the purpose of using a PT (or VT) for electrical power instrumentation?
\item{} What is the standard output signal range of a PT?
\item{} Explain why PTs have standardized output signal ranges.
\item{} In what sort of power system application would you expect to see a CCVT used instead of a purely inductive VT?
\end{itemize}


%INDEX% Reading assignment: Lessons In Industrial Instrumentation, potential transformers

%(END_NOTES)


