
%(BEGIN_QUESTION)
% Copyright 2012, Tony R. Kuphaldt, released under the Creative Commons Attribution License (v 1.0)
% This means you may do almost anything with this work of mine, so long as you give me proper credit

Read and outline the ``Current Transformers'' subsection of the ``Electrical Sensors'' section of the ``Electric Power Measurement and Control'' chapter in your {\it Lessons In Industrial Instrumentation} textbook.  Note the page numbers where important illustrations, photographs, equations, tables, and other relevant details are found.  Prepare to thoughtfully discuss with your instructor and classmates the concepts and examples explored in this reading.

\underbar{file i01244}
%(END_QUESTION)




%(BEGIN_ANSWER)


%(END_ANSWER)





%(BEGIN_NOTES)

A current transformer steps down and isolates high line currents to a level that is safe to connect to panel-mounted instruments such as ammeters.  The fixed ratio of a CT provides a proportional representation of line current.  CTs are usually constructed as toroids, with the single pass of the line conductor serving as a single turn of the CT's ``primary winding''.

5 amps is the standard in industry for representing 100\% line current.  This standard scaling allows interoperability between different brands and models of CTs and panel-mounted instruments.  However, it should be noted that some CTs output less current (e.g. 1 amp).

\vskip 10pt

Since a CT acts as a current source when line current is flowing, its secondary winding must never be open-circuited.  Doing so may generate extremely high voltages, because the current step-down ratio of a CT functions as a voltage step-up ratio!  Consequently, the safe way to disconnect a live CT from a measuring instrument is to {\it short-circuit} its secondary output.  Current sources are not harmed by short-circuits like voltage sources are.

\vskip 10pt

Circuit breakers and other high-voltage power devices commonly have CTs installed at the ``bushing'' points of their terminals, for convenient measurement of current in and out of the device.  Some very high-voltage CTs are mounted at the ends of long insulators to separate them from physical ground.







\vskip 20pt \vbox{\hrule \hbox{\strut \vrule{} {\bf Suggestions for Socratic discussion} \vrule} \hrule}

\begin{itemize}
\item{} What is the purpose of using a CT for electrical power instrumentation?
\item{} What is the standard output signal range of a power line CT?
\item{} Note the orientation of the secondary winding's turns for a ``donut'' CT.  Explain why the secondary windings must be wrapped in that orientation, rather than be wrapped parallel to the circumference of the iron core.
\item{} Explain in detail the danger posed by current transformers if their secondary windings are ever open-circuited under power.  
\item{} Explain how to alter the turns ratio of a window-style CT.
\item{} Suppose I passed the power conductor three times through the hole of a CT with a labeled ratio of 400:5.  Calculate the effective turns ratio.
\item{} Suppose I passed the power conductor four times through the hole of a CT with a labeled ratio of 600:5.  Calculate the effective turns ratio.
\item{} Do potential transformers (PTs) exhibit the same type of hazard with open-circuited secondary windings that current transformers (CTs) do?  Why or why not?
\item{} Unused CT secondary windings should be short-circuited.  Explain why short-circuiting the output of an energized CT does not cause any damage to the CT, and in fact is the safest thing you could do with that CT.
\end{itemize}










\vfil \eject

\noindent
{\bf Prep Quiz:}

Why should you never open-circuit the secondary winding of a current transformer?

\begin{itemize}
\item{} You may not be able to re-establish the connection again
\vskip 5pt 
\item{} It can develop dangerously high voltage in the open circuit
\vskip 5pt 
\item{} The transformer's measurement accuracy will suffer a span shift
\vskip 5pt 
\item{} The transformer's measurement accuracy will suffer a zero shift 
\vskip 5pt 
\item{} Doing so may cause the overcurrent protection unit to falsely sense high current
\vskip 5pt 
\item{} It just isn't considered polite
\end{itemize}



%INDEX% Reading assignment: Lessons In Industrial Instrumentation, current transformers

%(END_NOTES)


