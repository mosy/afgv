
%(BEGIN_QUESTION)
% Copyright 2012, Tony R. Kuphaldt, released under the Creative Commons Attribution License (v 1.0)
% This means you may do almost anything with this work of mine, so long as you give me proper credit

Read and outline the ``Circuit Breakers and Disconnects'' section of the ``Electric Power Measurement and Control'' chapter in your {\it Lessons In Industrial Instrumentation} textbook.  Note the page numbers where important illustrations, photographs, equations, tables, and other relevant details are found.  Prepare to thoughtfully discuss with your instructor and classmates the concepts and examples explored in this reading.

\underbar{file i01245}
%(END_QUESTION)




%(BEGIN_ANSWER)


%(END_ANSWER)





%(BEGIN_NOTES)

Circuit breakers are ``final control elements'' for electrical power systems.  Disconnects are knife-style switches intended to safely isolate sections of a power system.  Large circuit breakers usually contain ``trip'' and ``close'' solenoid coils causing them to actuate when those coils are momentarily energized (usually with 125 VDC power).  Status switches provide indication that the circuit breaker has changed state.

\vskip 10pt

Low-voltage (600 volts of less) circuit breakers are self-tripping devices, requiring manual action to re-close.  They are often found installed in ``MCC'' (Motor Control Center) panels where they may be unplugged from the panel for convenient maintenance and replacement.

\vskip 10pt

Medium-voltage (thousands of volts) circuit breakers are operated by ``trip'' and ``close'' coils, and must rely on external devices to tell them to trip open.  MV circuit breakers may be ``racked'' in and out of their panels for testing, maintenance, and safe isolation.  Some MV breakers allow the trip and close circuitry to remain connected while the power conductors have been racked out, allowing the breaker to be cycled without risk of energizing the load conductors.

The arc formed by a circuit breaker's contacts when opening under load must be somehow extinguished to prevent damage to the breaker.  Air blasts, oil baths, and vacuum chambers are different methods of dealing with contact arc.

The energy required to rapidly close and trip a large circuit breaker is typically supplied by tensed springs.  These springs must be ``charged'' by a special motor between operating cycles.  The trip and close events unleash the springs' energy to quickly move the breaker mechanism.  A flag on the circuit breaker panel shows the ``charged'' or ``discharged'' state of the actuating spring.

MV circuit breakers often have ``trip counters'' to track how many times the breaker has been actuated.  Maintenance is based on these trip counter values, much like automobile maintenance is based on mileage accumulated.

\vskip 10pt

High-voltage (46 kV and above) circuit breakers typically use oil immersion or gas quenching to extinguish contact arc.  Contact actuation power may be provided by compressed air (typical for legacy oil breakers) or by springs (typicaly for gas breakers).

The gas used for quenching in gas circuit breakers is sulfur hexafluoride (SF$_{6}$), about 5 times denser than air, and having excellent insulating properties.  It is safe to handle and non-toxic.

HV circuit breakers may be tripped as one unit (three breakers mechanically linked together) or in some cases tripped as three independent mechanisms.  In the case of independent breaker units, the contacts for each phase may be synchronized with current such that they trip when the current is at a minimum (to reduce arcing).










\filbreak

\vskip 20pt \vbox{\hrule \hbox{\strut \vrule{} {\bf Suggestions for Socratic discussion} \vrule} \hrule}

\begin{itemize}
\item{} Identify the {\it disconnects} and the {\it circuit breakers} shown in the book's photographs, describing the function of each.
\item{} If {\it disconnects} and {\it circuit breakers} are the final control elements of the electric power industry, what are the sensors/transmitters, and what are the controllers?
\item{} Explain what would happen if a disconnect switch were used to interrupt load current rather than the circuit breaker.
\item{} Explain how large circuit breakers are remotely controlled.  Specifically, what types of signals are used to command them to open and to close?
\item{} What is an MCC?
\item{} Explain the difference between ``thermal'' versus ``magnetic'' trip breakers, commonly used in low-voltage (600 VAC or less) power systems.
\item{} What does it mean to ``rack out'' a circuit breaker, and why would anyone do such a thing?
\item{} Explain why ``racking out'' a circuit breaker is a necessary step to prepare that breaker for testing.
\item{} What do the terms ``line'' and ``load'' refer to in a three-phase circuit breaker?
\item{} Identify and describe some of the techniques used to extinguish the arc inside of a medium-voltage circuit breaker as its contacts are opened under load.
\item{} What does it mean to ``charge'' the spring of a medium-voltage circuit breaker?
\item{} Identify and describe some of the techniques used to extinguish the arc inside of a high-voltage circuit breaker as its contacts are opened under load.
\item{} What do the colors {\it red} and {\it green} represent for status in the electric power industry?
\item{} Comment on some of the specifications seen on the nameplate of the Areva high-voltage circuit breaker shown in the book.
\item{} Explain what a {\it recloser} is, and what purpose it serves in a power distribution network.
\end{itemize}











\vfil \eject

\noindent
{\bf Prep Quiz:}

What does it mean to {\it rack out} a large circuit breaker?

\begin{itemize}
\item{} To lubricate all the mechanical joints inside
\vskip 5pt 
\item{} To check the time it takes for the contacts to close
\vskip 5pt 
\item{} To test the breaker for proper opening and closing
\vskip 5pt 
\item{} To check the time it takes for the contacts to open
\vskip 5pt 
\item{} To check its contacts for good electrical continuity
\vskip 5pt 
\item{} To unplug it from its socket in a breaker panel
\end{itemize}










\vfil \eject

\noindent
{\bf Prep Quiz:}

Sulfur hexafluoride (SF$_{6}$) gas is used in the electrical power industry for what purpose?

\begin{itemize}
\item{} Serve as a fuel for back-up power generators
\vskip 5pt 
\item{} Extinguish arcing at circuit breaker contacts
\vskip 5pt 
\item{} Dampen shock waves caused by arc blast
\vskip 5pt 
\item{} Suppress electrical fires by displacing air
\vskip 5pt 
\item{} Indicate the status of a line by changing color
\vskip 5pt 
\item{} Deepen your voice to make you talk funny
\end{itemize}


%INDEX% Reading assignment: Lessons In Industrial Instrumentation, circuit breakers and disconnects

%(END_NOTES)


