
%(BEGIN_QUESTION)
% Copyright 2015, Tony R. Kuphaldt, released under the Creative Commons Attribution License (v 1.0)
% This means you may do almost anything with this work of mine, so long as you give me proper credit

Read and outline the ``Instrument Transformer Test Switches'' subsection of the ``Electrical Sensors'' section of the ``Electric Power Measurement and Control'' chapter in your {\it Lessons In Industrial Instrumentation} textbook.  Note the page numbers where important illustrations, photographs, equations, tables, and other relevant details are found.  Prepare to thoughtfully discuss with your instructor and classmates the concepts and examples explored in this reading.

\underbar{file i01246}
%(END_QUESTION)




%(BEGIN_ANSWER)


%(END_ANSWER)





%(BEGIN_NOTES)

Test switches are special knife-style switches used to safely disconnect instruments from PTs and CTs.  PT connections are simply broken open by a pair of test switches.  

\vskip 10pt

CT connections must be handled by special {\it shorting} test switches with a make-before-break contact that shorts out the CT's secondary winding before opening the circuit with the instrument.  The non-shorting pole of a CT test switch pair is usually equipped with a ``test jack'' allowing a special ammeter probe to be inserted.  This non-shorting switch doesn't actually open up when you lift the knife lever, but only opens when an insulating piece is inserted between the test jack pieces.

\vskip 10pt

Special ``shorting'' terminal blocks are often used in CT circuits to both establish a safety ground connection for the CT's secondary circuit, as well as provide a means to short-circuit the transformer output.  The basic construction of a shorting terminal block is that of a metal bar spanning across the top of all terminals, with screw holes through which screws may be threaded to join the metal bar with any terminal underneath.  One terminal (ground) is always screwed to this bar, establishing the metal bar at earth ground potential.  Any other screws inserted into this metal bar force those terminals to also be at ground potential.

When used in conjunction with multi-ratio CTs (current transformers having tapped secondary windings to provide multiple turns ratios), the grounding screw is always located on one of the leads connected to the receiving instrument.








\vskip 20pt \vbox{\hrule \hbox{\strut \vrule{} {\bf Suggestions for Socratic discussion} \vrule} \hrule}

\begin{itemize}
\item{} Point out in the photograph of the CT test switch where the make-before-break action occurs (particularly the cam which touches the leaf-spring as the switch blade first begins to move).
\item{} Explain how the ``test jack'' is supposed to be used on a CT test switch to measure CT secondary current in a live condition.
\item{} Is there any way for the test to go wrong when using the ``test jack'' a CT test switch?  Can you think of a safer way to measure current during a live condition?
\item{} Is it normal for the shorting portion of a CT switch to {\it arc} when it is operated?  Explain why or why not.
\item{} Explain what would happen if a CT were mistakenly wired through a {\it PT test switch}.
\item{} Explain what would happen if a PT were mistakenly wired through a {\it CT test switch}.
\item{} Explain what would happen if a CT were mistakenly wired to the wrong side of a CT test switch.
\item{} Explain what would happen if a PT were mistakenly wired to the wrong side of a CT test switch.
\item{} Examine the pictorial and schematic representations of CT shorting terminal blocks in the textbook, and explain how they relate to one another.
\item{} Explain what would happen if too many bonding screws were left in the shorting bar.
\end{itemize}










\vfil \eject

\noindent
{\bf Prep Quiz:}

Current transformer {\it test switches} are different from regular ``knife'' style electrical switches in what important way?

\begin{itemize}
\item{} They have insulated handles so the user cannot make contact with the metal blades
\vskip 5pt 
\item{} They use a single-pole, double-throw style of contact to minimize arc hazard
\vskip 5pt 
\item{} They always have a place for a padlock so the switch may be locked in the open position
\vskip 5pt 
\item{} They are colored differently from any other type of switch, to avoid confusion
\vskip 5pt 
\item{} They ``make before break,'' ensuring the CT shorts before the rest of the circuit opens
\vskip 5pt 
\item{} They have a built-in diode to allow current measurement without breaking the circuit
\end{itemize}


\vfil \eject

\noindent
{\bf Prep Quiz:}

A current transformer (CT) operates best when:

\begin{itemize}
\item{} A ground loop exists in the secondary circuit
\vskip 5pt 
\item{} The circuit breaker is in the tripped state
\vskip 5pt 
\item{} The secondary winding powers a low resistance
\vskip 5pt 
\item{} The secondary winding is completely ungrounded
\vskip 5pt 
\item{} The secondary winding powers a high resistance 
\vskip 5pt 
\item{} The primary winding is short-circuited
\end{itemize}

%INDEX% Reading assignment: Lessons In Industrial Instrumentation, instrument transformer test switches

%(END_NOTES)


