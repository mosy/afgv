
%(BEGIN_QUESTION)
% Copyright 2015, Tony R. Kuphaldt, released under the Creative Commons Attribution License (v 1.0)
% This means you may do almost anything with this work of mine, so long as you give me proper credit

Read and outline the ``Introduction to Protective Relaying'' section of the ``Electric Power Measurement and Control'' chapter in your {\it Lessons In Industrial Instrumentation} textbook.  Note the page numbers where important illustrations, photographs, equations, tables, and other relevant details are found.  Prepare to thoughtfully discuss with your instructor and classmates the concepts and examples explored in this reading.

\underbar{file i01247}
%(END_QUESTION)




%(BEGIN_ANSWER)


%(END_ANSWER)





%(BEGIN_NOTES)

Large circuit breakers must be commanded to trip by external devices called {\it protective relays}.  ``Time-overcurrent'' relays trip breakers in the event the measured current exceeds a pre-set limit, where the tripping time is based on how far the current exceeds that limit.  ``Reclosing'' relays command a circuit breaker to re-close following an instantaneous overcurrent trip, such as when a tree branch falls across a power line.  The idea here is to re-establish power to see if the fault was transitory.

\vskip 10pt

125 VDC is the standard for power used to trip and close circuit breakers.  A large ``station power'' battery bank maintains this DC power even in the event of a grid power loss.

\vskip 10pt

Legacy protective relays used induction disk elements to sense trip conditions, mechanically actuating a contact to send trip power to a circuit breaker coil.  Time-overcurrent relays are good examples of this.  Such relays could be drawn out of sockets for maintenance and replacement.

Later electronic protective relays used solid-state components to sense fault conditions and determine when to trip the breaker.  Many of these were also ``draw-out'' designs for easy maintenance and replacement.

Modern microprocessor-based protective relays are typically panel-mounted rather than draw-out, because there is less need for regular maintenance and less need to replace them on a regular basis.  These computer-based relays can archive data for later retrieval, and have the ability to communicate digitally with other computers.

\vskip 10pt

Modern digital protective relays still use anachronistic terms which made sense for old electromechanical relays.












\vskip 20pt \vbox{\hrule \hbox{\strut \vrule{} {\bf Suggestions for Socratic discussion} \vrule} \hrule}

\begin{itemize}
\item{} In simple terms, describe the function of a {\it protective relay}.
\item{} What does a {\it reclosing} relay do, and why is this an important function for power distribution systems?
\item{} Explain why {\it reclosing} relays are only used for protecting overhead distribution lines, and never for protecting underground distribution lines.
\item{} Explain why protective relays usually operate off of a 125 volt {\it DC} power source.  Why not use standard 120 VAC power for powering the protective relay circuits?
\item{} Explain what a {\it station battery} is in a substation or other electrical facility.
\item{} Describe what ``pickup'' refers to in a protective relay.
\item{} What does it mean if a protective relay is a ``draw-out'' style?
\item{} Identify some of the benefits offered by digital protective relays over electromechanical protective relays.
\item{} Is the time-delay function of a ``time-overcurrent'' relay a fixed value or is it dependent on the intensity of the current?
\item{} Examine the simple diagram of a protective relay and circuit breaker shown in this section of the textbook, and identify multiple ways in which this circuit could fail (e.g. elements failed shorted, or failed open).  For each of the proposed faults, identify specifically how the protective system would behave.
\end{itemize}









\vfil \eject

\noindent
{\bf Prep Quiz:}

Show the YouTube video of the power line contacting a tree in Bellingham, WA.  Then, ask students to write their explanation of why the arc {\it resumed} after the initial breaker trip.


%INDEX% Reading assignment: Lessons In Industrial Instrumentation, instrument transformer test switches

%(END_NOTES)


