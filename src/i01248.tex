
%(BEGIN_QUESTION)
% Copyright 2015, Tony R. Kuphaldt, released under the Creative Commons Attribution License (v 1.0)
% This means you may do almost anything with this work of mine, so long as you give me proper credit

Read and outline the ``Instantaneous and Time-Overcurrent (50/51) Protection'' section of the ``Electric Power Measurement and Control'' chapter in your {\it Lessons In Industrial Instrumentation} textbook.  Note the page numbers where important illustrations, photographs, equations, tables, and other relevant details are found.  Prepare to thoughtfully discuss with your instructor and classmates the concepts and examples explored in this reading.

\underbar{file i01248}
%(END_QUESTION)




%(BEGIN_ANSWER)


%(END_ANSWER)





%(BEGIN_NOTES)

Excessive current is damaging in a power system because of the heat it generates ($P = I^2 R$).

\vskip 10pt

Instantaneous overcurrent protection (50) is where a protective relay commands a breaker to trip if the measured current exceeds the ``pickup'' value for any length of time.

\vskip 10pt

Time overcurrent protection (51) is where a protective relay commands a breaker to trip if the measured current exceeds the ``pickup'' value for a certain length of time, that time duration inversely proportional to the degree that current exceeds the pickup value.

Electromechanical 51 relays use aluminum disks torqued by the magnetic field generated by a stator coil powered by CT current.  If the CT current exceeds the pickup value, the disk begins to rotate.  When the disk reaches the end of its rotation, a set of contacts close to send ``trip'' power to the circuit breaker.

A seal-in contact is provided in the 51 relay to ensure a ``solid'' trip circuit connection even if the disk's contact closes very lightly.  This helps ensure reliable breaker tripping and reduces arcing at the sensitive disk contact.

\vskip 10pt

Electromechanical 50/51 relays are single-phase, and so three of them are needed to protect against overcurrent in a 3-phase power system.  Each relay has its own CT for sensing line current, and their trip contacts are paralleled so that the circuit breaker will trip if {\it any} of the three relays trips.

\vskip 10pt

Calibration of a 50 overcurrent element is a single-point check: whether or not the trip contacts close when current exceeds the instantaneous pickup value.

Calibration of a 51 overcurrent element consists first of checking the pickup value (the current necessary to begin timing).  Next, the ``time dial'' setting which determines the amount of disk rotation angle needed to close the trip contacts needs to be checked to see that the relay trips in the right amount of time.  The accuracy of relay timing must be checked at multiple points just like process transmitters must have their calibration checked at multiple points.

\vskip 10pt

51 relays exhibit different curves, standardized by name (e.g. inverse, very inverse, extremely inverse).  The purpose for this is to give power system designers the ability to ``coordinate'' when each 51 relay will trip, ensuring the relay closest to the fault will be the one to trip, and that ``upstream'' 51 relays will take a bit longer to trip, thus ensuring power will not be unnecessarily interrupted to loads.












\filbreak

\vskip 20pt \vbox{\hrule \hbox{\strut \vrule{} {\bf Suggestions for Socratic discussion} \vrule} \hrule}

\begin{itemize}
\item{} Explain the meaning of the inverse time-overcurrent trip curve shown in the book.  What does ``pickup'' refer to in this context?  What does ``inverse'' mean with regard to the curve's shape?
\item{} Explain how an electromechanical ``50'' (instantaneous overcurrent) relay functions, based on the diagram shown in the book.
\item{} Explain how an electromechanical ``51'' (time overcurrent) relay functions, based on the diagram shown in the book.
\item{} What does the term ``pickup'' refer to in the context of a ``50'' (instantaneous overcurrent) relay?
\item{} What does the term ``pickup'' refer to in the context of a ``51'' (time overcurrent) relay?
\item{} If you increase the restraining spring force, will this increase or decrease the ``51'' relay's pickup value?
\item{} Explain the purpose of the {\it drag magnet} in an induction disk type of relay.
\item{} How can you make coarse adjustments to the pickup value of a General Electric electromechanical time-overcurrent (51) relay?
\item{} How can you make fine adjustments to the pickup value of a General Electric electromechanical time-overcurrent (51) relay?
\item{} Explain why a ``seal-in'' contact and coil are used in an electromechanical ``51'' (time overcurrent) relay.
\item{} Explain how the ``seal-in'' circuit of an electromechanical ``51'' (time overcurrent) relay becomes unlatched after it latches to send a trip signal to the breaker.
\item{} Explain what the ``time dial'' in an electromechanical 51 relay does, and why one might choose a different time dial setting for a particular application.
\item{} Which is simpler to do: calibrate a 50 relay or a 51 relay?  Explain why.
\item{} Explain what {\it coordination} refers to in the context of overcurrent protection.
\item{} Explain what might happen if overcurrent relays were not properly {\it coordinated} in a power distribution network.
\end{itemize}














\vfil \eject

\noindent
{\bf Prep Quiz:}

Legacy electromechanical time-overcurrent relays use a metal ``induction disk'' to perform the time-delay-trip function.  Identify what happens if a strong current is suddenly sent to such a relay from the connected current transformer, after receiving a very weak current for a long time:

\begin{itemize}
\item{} A microprocessor signals the breaker to trip
\vskip 5pt 
\item{} The induction disk switches direction
\vskip 5pt 
\item{} A fuse inside the relay blows, tripping the breaker
\vskip 5pt 
\item{} The induction disk begins to rotate
\vskip 5pt 
\item{} The induction disk slows down
\vskip 5pt 
\item{} The induction disk stops rotating altogether
\end{itemize}










\vfil \eject

\noindent
{\bf Prep Quiz:}

Define the term ``pickup'' for a legacy electromechanical time-overcurrent relay (ANSI/IEEE code ``51'') using a metal induction disk to perform the time-delay-trip function:

\begin{itemize}
\item{} The number of winding turns inside the operating coil of the relay
\vskip 5pt 
\item{} The amount of current necessary to make the disk begin to rotate
\vskip 5pt 
\item{} The amount of current necessary to instantly trip the circuit breaker
\vskip 5pt 
\item{} The amount of time it takes for the relay to trip when overloaded
\vskip 5pt 
\item{} The number setting on the relay's ``time dial'' adjustment
\vskip 5pt 
\item{} The amount of time it takes for the disk to return to its resting position
\end{itemize}

%INDEX% Reading assignment: Lessons In Industrial Instrumentation, instantaneous and time-overcurrent (50/51) protection

%(END_NOTES)


