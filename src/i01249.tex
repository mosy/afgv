
%(BEGIN_QUESTION)
% Copyright 2015, Tony R. Kuphaldt, released under the Creative Commons Attribution License (v 1.0)
% This means you may do almost anything with this work of mine, so long as you give me proper credit

Read the section entitled ``Distribution Automation System Example'' on pages 16-17 of the article ``Power System Automation'' written by David Dolezilek of Schweitzer Engineering Laboratories, Inc. (SEL document 6091), and answer the following questions:

\vskip 10pt

The diagram shown in Figure 18 is called a {\it single-line electrical diagram}.  Explain why this simplified form of schematic is useful in describing power systems, as opposed to sketching all conductors in the real system.

\vskip 10pt

Identify which of the five switches in Figure 18 are manually-operated, and which are operated automatically.  How is this difference denoted in the single-line diagram?

\vskip 10pt

Where are all the loads in this single-line diagram?  No symbols representing loads are drawn in this diagram, but we do have a verbal hint as to where loads exist in this system!

\vskip 10pt

How do the protective relays in Figure 18 help protect the power system equipment in the event of a fault?

\vskip 10pt

Explain how the system sketched in Figure 19 is better from the perspective of {\it security}.

\underbar{file i01249}
%(END_QUESTION)





%(BEGIN_ANSWER)


%(END_ANSWER)





%(BEGIN_NOTES)

A single-line diagram simplifies power system schematics by showing only one conductor instead of all three.  Power flow tracing is as simple as tracing flow of water through pipes!

\vskip 10pt

In figure 18, switches 1 and 3 are automatically controlled by protective relays, while the other three switches (2, 4, and 5) are manually actuated.  Automatic switches (breakers) are indicated by boxes, while manual disconnects are indicated by SPST switch symbols.

\vskip 10pt

Loads come off of lines 1, 2, 3, and 4, although they are not shown in the single-line diagram.

\vskip 10pt

The protective relays sense overcurrent conditions and act to trip breakers 1 and/or 3 in the event of a fault.

\vskip 10pt

In figure 19, availability of power to customers is improved by making all switches in the system automatic, and connecting all the associated protective relays together so that they may coordinate their actions to automatically isolate one line and restore power as quickly as possible to adjacent lines.

%INDEX% Reading assignment: Power System Automation (Schweitzer Engineering Labs)

%(END_NOTES)


