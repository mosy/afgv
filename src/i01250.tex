
%(BEGIN_QUESTION)
% Copyright 2012, Tony R. Kuphaldt, released under the Creative Commons Attribution License (v 1.0)
% This means you may do almost anything with this work of mine, so long as you give me proper credit

Read and outline the ``ANSI/IEEE Function Number Codes'' section of the ``Electric Power Measurement and Control'' chapter in your {\it Lessons In Industrial Instrumentation} textbook.  Note the page numbers where important illustrations, photographs, equations, tables, and other relevant details are found.  Prepare to thoughtfully discuss with your instructor and classmates the concepts and examples explored in this reading.

\underbar{file i01250}
%(END_QUESTION)




%(BEGIN_ANSWER)

Here is a list of some of the most common ANSI/IEEE function codes:

\begin{itemize}
\item{} {\tt 50} = Instantaneous overcurrent protection
\item{} {\tt 51} = Time overcurrent protection
\item{} {\tt 52} = AC power circuit breaker
\item{} {\tt 67} = Directional overcurrent protection
\item{} {\tt 79} = Automatic reclose protection
\item{} {\tt 86} = Auxiliary/Lockout
\item{} {\tt 81} = Frequency protection
\item{} {\tt 87} = Differential protection
\item{} {\tt 89} = Line power disconnect
\end{itemize}

%(END_ANSWER)





%(BEGIN_NOTES)

ANSI/IEEE-defined number codes refer to specific functions within electrical power systems.  Each number represents a particular {\it function} rather than a particular {\it device}, so that one protective relay might implement multiple ANSI/IEEE functions (e.g. a 50/51 overcurrent relay).

\vskip 10pt

Here is a list of some of the most common ANSI/IEEE function codes, the italic text referring to abstract functions rather than specific pieces of equipment:

\begin{itemize}
\item{} {\tt 50} = {\it Instantaneous overcurrent protection}
\item{} {\tt 51} = {\it Time overcurrent protection}
\item{} {\tt 52} = AC power circuit breaker
\item{} {\tt 67} = {\it Directional overcurrent protection}
\item{} {\tt 79} = {\it Automatic reclose protection}
\item{} {\tt 86} = {\it Auxiliary/Lockout}
\item{} {\tt 81} = {\it Frequency protection}
\item{} {\tt 87} = {\it Differential protection}
\item{} {\tt 89} = Line power disconnect
\end{itemize}

\vskip 10pt

Single-line relay diagrams use solid lines to represent continuous (analog) signal paths and dashed lines to represent discrete (on/off) signal paths or inter-relay communication signal paths.

\vskip 10pt

Trip circuit schematic diagrams show individual wires, unlike single-line diagrams which only show power pathways.  These schematic diagrams are usually shown as pairs: one representing the power conductors and instrument transformer connections to the relay, while the other shows trip circuit conductors and relay contacts connecting to circuit breaker trip/close coils.  A number label such as 51-2 begins with the ANSI code (e.g. 51 = time overcurrent) for the general protective function and ends with a number or lettered abbreviation for the specific relay device (e.g. -2 = relay monitoring phase 2).  Lettered suffixes represent further levels of detail about the specific device (e.g. 51-2/SI = the ``seal-in'' contact for the time-overcurrent relay on phase 2)











\vskip 20pt \vbox{\hrule \hbox{\strut \vrule{} {\bf Suggestions for Socratic discussion} \vrule} \hrule}

\begin{itemize}
\item{} Describe how protective relay symbology in single-line diagrams is similar to ISA instrument symbology in P\&IDs.
\item{} Identify the ANSI/IEEE function code number for the {\it instantaneous overcurrent} function, and give an example of a power system fault that might cause this condition to occur.
\item{} Identify the ANSI/IEEE function code number for the {\it time overcurrent} function, and give an example of a power system fault that might cause this condition to occur.
\item{} Identify the ANSI/IEEE function code number for the {\it undervoltage} function, and give an example of a power system fault that might cause this condition to occur.
\item{} Identify the ANSI/IEEE function code number for the {\it overvoltage} function, and give an example of a power system fault that might cause this condition to occur.
\item{} Identify the ANSI/IEEE function code number for the {\it reclose} function, and give an example of a power system fault that might cause this condition to occur.
\item{} Identify the ANSI/IEEE function code number for the {\it lockout/auxiliary} function, and give an example of a power system fault that might cause this condition to occur.
\item{} Identify the ANSI/IEEE function code number for the {\it differential} function, and give an example of a power system fault that might cause this condition to occur.
\item{} Compare and contrast {\it power} circuit diagrams with {\it trip} circuit diagrams.
\item{} Explain how the seal-in contact in the 51 relay works, and how that latching function is reset once the circuit breaker reaches the open (tripped) position.
\item{} Examine the power/trip circuit diagrams shown in this section of the textbook, and identify specific component symbols within this diagram.  How similar or different are these symbols compared to those used in electrical schematic diagrams or relay ladder-logic diagrams?
\item{} Examine the power/trip circuit diagrams shown in this section of the textbook, and identify ways in which duplication is reduced for easier reading.
\end{itemize}

%INDEX% Reading assignment: Lessons In Industrial Instrumentation, ANSI/IEEE function number codes

%(END_NOTES)


