
%(BEGIN_QUESTION)
% Copyright 2015, Tony R. Kuphaldt, released under the Creative Commons Attribution License (v 1.0)
% This means you may do almost anything with this work of mine, so long as you give me proper credit

Each of these statements is incorrect in some way.  Correct the misconceptions in each:

\vskip 20pt {\narrower \noindent \baselineskip5pt

``Thermal overloads protect against overcurrent conditions in case there is a short-circuit in the power conductors feeding a motor bucket.''

\par} \vskip 100pt


\vskip 20pt {\narrower \noindent \baselineskip5pt

``Thermal overloads protect against motor overheating by sensing the temperature of the motor.  They operate on temperature, rather than on motor current like a circuit breaker.''

\par} \vskip 100pt


\vskip 20pt {\narrower \noindent \baselineskip5pt

``When an overload heater senses an over-loaded condition, it opens up like a fuse to directly interrupt power to the motor.''

\par} \vskip 50pt



\vfil 

\underbar{file i01283}
\eject
%(END_QUESTION)





%(BEGIN_ANSWER)

This is a graded question -- no answers or hints given!
 
%(END_ANSWER)





%(BEGIN_NOTES)

A short-circuit in the lines feeding a motor bucket would generate a fault current, but not through the overload heaters.  Instead, the fault current would stop short of the contactor, existing only line the lines feeding the bucket (L1, L2, and/or L3).  That sort of overcurrent condition would be detected and protected by fuses or a circuit breaker between the contactor and the power source.

\vskip 10pt

Overload heaters do indeed operate on temperature, but this temperature rise is caused by line current through them, just like the motor's temperature rise is a function of its line current.  It is not as though the overloads directly sense motor temperature apart from line current.  There do exist protective relays which use RTD sensors buried in the motor windings to sense motor temperature, but this is far more sophisticated than the overload heaters found in a typical bucket.

\vskip 10pt

Overload heaters are designed to warm up under overload conditions, not burn open like fuses.  This is a very common misconception: that overload heaters are supposed to work like fuses, burning open to interrupt the three-phase power to the motor.

%INDEX% Final Control Elements, motor: overload heater

%(END_NOTES)


