
%(BEGIN_QUESTION)
% Copyright 2013, Tony R. Kuphaldt, released under the Creative Commons Attribution License (v 1.0)
% This means you may do almost anything with this work of mine, so long as you give me proper credit

\noindent
Compare and contrast {\it laminar} versus {\it turbulent} flow regimes with regard to the following criteria:

\vskip 10pt

\begin{itemize}
\item{} Which regime creates the least amount of {\it drag} (frictional energy losses) through a long length of pipe? 
\vskip 10pt
\item{} Which regime is better for mixing fluids together in a piping system?
\vskip 10pt
\item{} Which regime is better for ensuring thorough reaction between chemical reactants in a piping system?
\vskip 10pt
\item{} Which regine is preferable inside of a heat exchanger, to ensure maximum heat transfer?
\end{itemize}

\vskip 30pt

\noindent
Examine the following formulae for calculating Reynolds number:

\vskip 10pt

To calculate Reynolds number given English units (gas flow):

$$\hbox{Re} = {{(6.32) \rho Q} \over {D \mu}}$$

\noindent
Where,

Re = Reynolds number (unitless)

$\rho$ = Mass density of gas, in pounds (mass) per cubic foot (lbm/ft$^{3}$)

$Q$ = Flow rate, standard cubic feet per hour (SCFH)

$D$ = Diameter of pipe, in inches (in)

$\mu$ = Absolute viscosity of fluid, in centipoise (cP)

\vskip 20pt \goodbreak

To calculate Reynolds number given specific gravity instead of density (liquid flow):

$$\hbox{Re} = {{(3160) G_f Q} \over {D \mu}}$$

\noindent
Where,

Re = Reynolds number (unitless)

$G_f$ = Specific gravity of liquid (unitless)

$Q$ = Flow rate, gallons per minute (GPM)

$D$ = Diameter of pipe, in inches (in)

$\mu$ = Absolute viscosity of fluid, in centipoise (cP)

\vskip 20pt

Now, qualitatively identify which direction each variable in the formula must change (e.g. {\it increase} versus {\it decrease}) in order to promote turbulence in a fluid stream, all other factors remaining unchanged.

\vskip 20pt \vbox{\hrule \hbox{\strut \vrule{} {\bf Suggestions for Socratic discussion} \vrule} \hrule}

\begin{itemize}
\item{} For any given fluid {\it velocity}, Reynolds number will decrease if the pipe diameter decreases.  Knowing this, explain why we see $D$ in the denominator of these fractions rather than in the numerator.
\end{itemize}

\underbar{file i01299}
%(END_QUESTION)





%(BEGIN_ANSWER)

\begin{itemize}
\item{} Which regime creates the least amount of {\it drag} (frictional energy losses) through a long length of pipe?  {\it Laminar}
\vskip 5pt
\item{} Which regime is better for mixing fluids together in a piping system?  {\it Turbulent}
\vskip 5pt
\item{} Which regime is better for ensuring thorough reaction between chemical reactants in a piping system?  {\it Turbulent}
\vskip 5pt
\item{} Which regine is preferable inside of a heat exchanger, to ensure maximum heat transfer?  {\it Turbulent}
\end{itemize}

\vskip 10pt

\noindent
All other factors being equal, turbulence will be promoted by the following:

\begin{itemize}
\item{} Increasing flow rate ($Q$)
\vskip 5pt
\item{} Increasing fluid density ($\rho$ or $G_f$)
\vskip 5pt
\item{} Decreasing pipe diameter ($D$)
\vskip 5pt
\item{} Decreasing viscosity ($\mu$)
\end{itemize}



%(END_ANSWER)





%(BEGIN_NOTES)

\noindent
Other Reynolds number formulae:

\vskip 10pt

To calculate Reynolds number given metric units:

$$\hbox{Re} = {{D \overline{V} \rho} \over \mu}$$

\noindent
Where,

Re = Reynolds number (unitless)

$D$ = Diameter of pipe, in meters (m)

$\overline{V}$ = Average velocity of fluid, in meters per second (m/s)

$\rho$ = Mass density of fluid, in kilograms per cubic meter (kg/m$^{3}$)

$\mu$ = Absolute viscosity of fluid, in Pascal-seconds (Pa $\cdot$ s)

\vskip 60pt \goodbreak

To calculate Reynolds number given English units (liquid flow):

$$\hbox{Re} = {{(50.7) \rho Q} \over {D \mu}}$$

\noindent
Where,

Re = Reynolds number (unitless)

$\rho$ = Mass density of liquid, in pounds (mass) per cubic foot (lbm/ft$^{3}$)

$Q$ = Flow rate, gallons per minute (GPM)

$D$ = Diameter of pipe, in inches (in)

$\mu$ = Absolute viscosity of fluid, in centipoise (cP)


%INDEX% Physics, dynamic fluids: laminar versus turbulent flow regimes
%INDEX% Physics, dynamic fluids: Reynolds number

%(END_NOTES)


