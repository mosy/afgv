
%(BEGIN_QUESTION)
% Copyright 2011, Tony R. Kuphaldt, released under the Creative Commons Attribution License (v 1.0)
% This means you may do almost anything with this work of mine, so long as you give me proper credit

Koyo DirectLogic model DL05 programmable logic controllers happen to represent integer values as 16-bit BCD (Binary Coded Decimal) numbers in their memory.  Suppose a DL05 PLC happens to have a BCD number of 5338 stored in memory address {\tt V2001}.  Determine the value interpreted by an HMI panel reading that same memory location in the PLC, supposing the tagname database of the HMI panel has that tag configured for a 16-bit ``unsigned integer'' instead of ``BCD'' as it should be.

\vskip 10pt

HMI panel will read = \underbar{\hskip 50pt}

\underbar{file i01300}
%(END_QUESTION)





%(BEGIN_ANSWER)

5338 in BCD format is 0101001100111000.  This 16-bit string translates to an unsigned integer value of {\bf 21304}.

%(END_ANSWER)





%(BEGIN_NOTES)

{\bf This question is intended for exams only and not worksheets!}.

%(END_NOTES)

