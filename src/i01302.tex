
%(BEGIN_QUESTION)
% Copyright 2012, Tony R. Kuphaldt, released under the Creative Commons Attribution License (v 1.0)
% This means you may do almost anything with this work of mine, so long as you give me proper credit

What is the difference between these two variable expressions?

$$x^2$$

$$x_2$$

\underbar{file i01302}
%(END_QUESTION)





%(BEGIN_ANSWER)

Superscript numbers represent exponents, so that $x^2$ means ``$x$ {\it squared}'', or $x$ multiplied by itself.  Subscript numbers are used to denote separate variables, so that the same alphabetical letter may be used more than once in an equation.  Thus, $x_2$ is a distinct variable, different from $x_0$, $x_1$, or $x_3$.

\vskip 10pt

{\bf Warning:} sometimes subscripts are used to denote specific numerical values of a variable.  For instance, $x_2$ could mean ``the variable $x$ when its value is equal to 2''.  This is almost always the meaning of subscripts when they are 0 ($x_0$ is the variable $x$, set equal to a value of 0).  Confusing?  Yes!

%(END_ANSWER)





%(BEGIN_NOTES)

In the various sciences, certain letters have become standardized to represent specific quantities.  In physics, for instance, {\it mass} is always represented by the variable $m$, {\it velocity} by the variable $v$, etc.  What, then, do we do when we have to represent more than one mass or more than one velocity in a single equation?  Using different letters ($v$ = velocity 1, $x$ = velocity 2, $z$ = velocity 3 . . .) is not practical, since so many of the other letters are already reserved for different quantities in physics. 

A practical solution to this dilemma is to use subscripts to denote different variables of the same quantity {\it type}.  Subscript characters may be numbers, single letters, or even short words: $v_1$, $m_Z$, $T_{original}$.

%INDEX% Mathematics review: basic principles of algebra

%(END_NOTES)


