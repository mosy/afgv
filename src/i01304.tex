
%(BEGIN_QUESTION)
% Copyright 2012, Tony R. Kuphaldt, released under the Creative Commons Attribution License (v 1.0)
% This means you may do almost anything with this work of mine, so long as you give me proper credit

Suppose we begin with this mathematical statement:

$$3 = 3$$

If I were to multiply the right-hand side of the equation by the number ``7'', the quantities on either side of the ``equals'' sign would no longer be equal to each other.  To be proper, I would have to replace the ``equals'' symbol with a ``not equal'' symbol:

$$3 \not = 3 \times 7$$ 

Assuming that we keep ``$3 \times 7$'' on the right-hand side of the statement, what would have to be done to the left-hand side of the statement to turn it into an equality again?

\underbar{file i01304}
%(END_QUESTION)





%(BEGIN_ANSWER)

Without altering the right-hand side of this mathematical expression, the only way to bring both sides into equality again is to multiply the left-hand side of the expression by ``7'' as well:

$$3 \times 7 = 3 \times 7$$ 

%(END_ANSWER)





%(BEGIN_NOTES)

One of the foundational principles of algebra is that any manipulation is allowed in an equation, so long as the same manipulation is applied to both sides.  This question introduces students to this principle in a very gentle way, using numerical values rather than variables.

%INDEX% Mathematics review: basic principles of algebra

%(END_NOTES)


