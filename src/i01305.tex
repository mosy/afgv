
%(BEGIN_QUESTION)
% Copyright 2012, Tony R. Kuphaldt, released under the Creative Commons Attribution License (v 1.0)
% This means you may do almost anything with this work of mine, so long as you give me proper credit

Suppose we begin with this mathematical statement:

$$3 \times 4 = 10 + 2$$

If I were to add the quantity ``1'' to the left-hand side of the equation, the quantities on either side of the ``equals'' sign would no longer be equal to each other.  To be proper, I would have to replace the ``equals'' symbol with a ``not equal'' symbol:

$$(3 \times 4) + 1 \not = 10 + 2$$

What is the simplest and most direct change I can make to the right-hand side of this expression to turn it into an equality again?

\underbar{file i01305}
%(END_QUESTION)





%(BEGIN_ANSWER)

The simplest thing I can do to the right-hand side of the equation to make it equal once again to the left-hand side of the equation is to manipulate it in the same way that I just manipulated the left-hand side (by adding the quantity ``1''):

$$(3 \times 4) + 1 = (10 + 2) + 1$$

%(END_ANSWER)





%(BEGIN_NOTES)

One of the foundational principles of algebra is that any manipulation is allowed in an equation, so long as the same manipulation is applied to both sides.  This question shows how this principle is valid even if the quantities on both sides of the equation are not identical in appearance or form (just so long as they are {\it equal} in value to each other.

%INDEX% Mathematics review: basic principles of algebra

%(END_NOTES)


