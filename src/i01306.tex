
%(BEGIN_QUESTION)
% Copyright 2012, Tony R. Kuphaldt, released under the Creative Commons Attribution License (v 1.0)
% This means you may do almost anything with this work of mine, so long as you give me proper credit

Suppose we were told that this was a mathematically true statement:

$$x = y$$

In other words, variable $x$ represents the exact same numerical value as variable $y$.  Given this assumption, the following mathematical statements must also be true:

$$x + 8 = y + 8$$

$$-x = -y$$

$$9x = 9y$$

$${x \over 25} = {y \over 25}$$

$${1 \over x} = {1 \over y}$$

$$x^2 = y^2$$

$$\sqrt{x} = \sqrt{y}$$

$$e^x = e^y$$

$$\log x = \log y$$

$$x! = y!$$

$${dx \over dt} = {dy \over dt}$$

Explain the general principle at work here.  Why are we able to alter the basic equality of $x = y$ in so many different ways, and yet still have the resulting expressions be equalities? 

\underbar{file i01306}
%(END_QUESTION)





%(BEGIN_ANSWER)

The principle at work is this: you may perform any mathematical operation you wish to an equation, provided you apply the {\it same} operation to both sides of the equation in the exact same way.

%(END_ANSWER)





%(BEGIN_NOTES)

This principle is not only true, but proves to be extremely useful in simplifying and re-arranging algebraic expressions.

%INDEX% Mathematics review: basic principles of algebra

%(END_NOTES)


