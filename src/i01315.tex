
%(BEGIN_QUESTION)
% Copyright 2012, Tony R. Kuphaldt, released under the Creative Commons Attribution License (v 1.0)
% This means you may do almost anything with this work of mine, so long as you give me proper credit

Mechanical, chemical, and civil engineers must often calculate the size of piping necessary to transport fluids.  One equation used to relate the water-carrying capacity of multiple, small pipes to the carrying capacity of one large pipe is as follows:

$$N = \biggl({d_2 \over d_1}\biggr)^{2.5}$$

\noindent
Where,

$N =$ Number of small pipes

$d_1 =$ Diameter of each small pipe

$d_2 =$ Diameter of large pipe

\vskip 10pt

Manipulate this equation as many times as necessary to express it in terms of all its variables.

\underbar{file i01315}
%(END_QUESTION)





%(BEGIN_ANSWER)

$$d_2 = d_1 N^{0.4}$$

$$d_1 = {d_2 \over N^{0.4}}$$

%(END_ANSWER)





%(BEGIN_NOTES)

The most challenging portion of this problem is ``undoing'' the non-integer exponent.

\vskip 10pt

Equation taken from page 3-378 of the \underbar{Standard Handbook of Engineering Calculations}, Editor: Tyler G. Hicks, P.E., ISBN 0-07-028734-1.

%INDEX% Mathematics review: basic principles of algebra
%INDEX% Mathematics review: manipulating literal equations

%(END_NOTES)


