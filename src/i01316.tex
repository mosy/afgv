
%(BEGIN_QUESTION)
% Copyright 2012, Tony R. Kuphaldt, released under the Creative Commons Attribution License (v 1.0)
% This means you may do almost anything with this work of mine, so long as you give me proper credit

The amount of power required to propel a ship is given by this equation:

$$P = {D^{2 \over 3} V^3 \over K}$$

\noindent
Where,

$P =$ Power required to turn propeller(s) (horsepower)

$D =$ Vessel displacement (long tons)

$V =$ Velocity (nautical miles per hour)

$K =$ Admiralty coefficient (approximately 70 for a 30 foot long ship, load waterline)

\vskip 10pt

Manipulate this equation as many times as necessary to express it in terms of all its variables.

\underbar{file i01316}
%(END_QUESTION)





%(BEGIN_ANSWER)

$$D = \biggl({P K \over V^3} \biggr)^{3 \over 2}$$

\vskip 20pt

$$V = \biggl({P K \over D^{2 \over 3}} \biggr)^{1 \over 3}$$

\vskip 20pt

$$K = {D^{2 \over 3} V^3 \over P}$$

%(END_ANSWER)





%(BEGIN_NOTES)

This question provides good practice for students to cancel fractional exponents.

\vskip 10pt

Equation taken from page 9-5 of the \underbar{Standard Handbook of Engineering Calculations}, Editor: Tyler G. Hicks, P.E., ISBN 0-07-028734-1.

%INDEX% Mathematics review: basic principles of algebra
%INDEX% Mathematics review: manipulating literal equations

%(END_NOTES)


