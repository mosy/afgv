
%(BEGIN_QUESTION)
% Copyright 2012, Tony R. Kuphaldt, released under the Creative Commons Attribution License (v 1.0)
% This means you may do almost anything with this work of mine, so long as you give me proper credit

In algebra, any equation may be manipulated in any way desired, so long as the same manipulation is applied to {\it both} sides of the equation equally.  In this example, though, only one term on one side of the equation (${2 \over x}$) is manipulated: we multiply it by the fraction ${3x \over 3x}$.  Is this a ``legal'' thing to do?  Why or why not?

$$y = {2 \over x} + {5 \over 3x^2}$$

$$y = {3x \over 3x}{2 \over x} + {5 \over 3x^2}$$

$$y = {6x \over 3x^2} + {5 \over 3x^2}$$

$$y = {{6x + 5} \over 3x^2}$$

\underbar{file i01317}
%(END_QUESTION)





%(BEGIN_ANSWER)

This type of manipulation is perfectly ``legal'' to do, following the algebraic identity:

$$1a = a$$

%(END_ANSWER)





%(BEGIN_NOTES)

Multiplying any quantity by 1 does not change the value of that quantity, and so performing this manipulation only on one side of an equation does not ``imbalance'' the equation, even if done to just a single term in the equation.

%INDEX% Mathematics review: basic principles of algebra
%INDEX% Mathematics review: manipulating literal equations

%(END_NOTES)


