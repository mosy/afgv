
%(BEGIN_QUESTION)
% Copyright 2012, Tony R. Kuphaldt, released under the Creative Commons Attribution License (v 1.0)
% This means you may do almost anything with this work of mine, so long as you give me proper credit

Equations with identical variables represented on {\it both} sides are often tricky to manipulate.  Take this one, for example:

$${a \over x} = xy + xz$$

Here, the variable $x$ is found on both sides of the equation.  How can we manipulate this equation so as to ``consolidate'' these $x$ variables together so that $x$ is by itself on one side of the equation and everything else is on the other side?

\underbar{file i01319}
%(END_QUESTION)





%(BEGIN_ANSWER)

$$x = \sqrt{{a \over y + z}}$$

%(END_ANSWER)





%(BEGIN_NOTES)

The key here is to use {\it factoring} to consolidate the two $x$ variables on the right-hand side of the equation, then it becomes easy to consolidate the single $x$ variable on the left side with the single $x$ variable on the right.

%INDEX% Mathematics review: basic principles of algebra
%INDEX% Mathematics review: manipulating literal equations

%(END_NOTES)


