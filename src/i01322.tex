
%(BEGIN_QUESTION)
% Copyright 2012, Tony R. Kuphaldt, released under the Creative Commons Attribution License (v 1.0)
% This means you may do almost anything with this work of mine, so long as you give me proper credit

In digital electronic systems based on binary numeration, the number of possible {\it states} representable by the system is given by the following equation:

$$n_s = 2^{n_b}$$

\noindent
Where,

$n_s =$ Number of possible states

$n_b =$ Number of binary ``bits''

\vskip 10pt

How could you manipulate this equation to solve for the number of binary bits necessary to provide a given number of states?

\underbar{file i01322}
%(END_QUESTION)





%(BEGIN_ANSWER)

$$n_b = {\log n_s \over \log 2}$$

%(END_ANSWER)





%(BEGIN_NOTES)

Logarithms provide us a way to easily isolate variable exponents in an equation.

%INDEX% Mathematics review: basic principles of algebra
%INDEX% Mathematics review: manipulating literal equations

%(END_NOTES)


