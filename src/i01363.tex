
%(BEGIN_QUESTION)
% Copyright 2009, Tony R. Kuphaldt, released under the Creative Commons Attribution License (v 1.0)
% This means you may do almost anything with this work of mine, so long as you give me proper credit

Read and outline the ``Force-Balance Pneumatic Positioners'' subsection of the ``Valve Positioners'' section of the ``Control Valves'' chapter in your {\it Lessons In Industrial Instrumentation} textbook.  Note the page numbers where important illustrations, photographs, equations, tables, and other relevant details are found.  Prepare to thoughtfully discuss with your instructor and classmates the concepts and examples explored in this reading.

\underbar{file i01363}
%(END_QUESTION)





%(BEGIN_ANSWER)

 
%(END_ANSWER)





%(BEGIN_NOTES)

Increasing signal adds force to beam; valve moves, stretches spring, adding countering force to beam; when forces balance, equilibrium is achieved.

\vskip 10pt

Rotary example (PMV model 1500) uses a {\it cam} to translate valve's rotary motion into spring stretch into force on beam to balance bellows force.  A pilot valve assembly is used instead of a flapper/nozzle!









\vskip 20pt \vbox{\hrule \hbox{\strut \vrule{} {\bf Suggestions for Socratic discussion} \vrule} \hrule}

\begin{itemize}
\item{} Explain why the force-balance positioner illustrated first in this section of the textbook is actually force-balance and not motion-balance, despite the fact that the valve stem clearly moves.
\item{} Examine the photographs of the PMV model 1500 force-balance positioner and explain how it works.  If you (the instructor) can provide one of these positioners for students to interact with in the classroom, so much the better!
\item{} Examine a random valve positioner provided by the instructor, analyzing the mechanism to see if it is {\it force-balance} or {\it motion-balance}.
\item{} Examine a random valve positioner provided by the instructor, analyzing the mechanism to identify the location and function of the {\it zero} adjustment.
\item{} Examine a random valve positioner provided by the instructor, analyzing the mechanism to identify the location and function of the {\it span} adjustment.
\end{itemize}

%INDEX% Final Control Elements, valve: positioner (Fisher model 3582)
%INDEX% Reading assignment: Lessons In Industrial Instrumentation, valve positioners (force-balance pneumatic)

%(END_NOTES)


