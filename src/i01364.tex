
%(BEGIN_QUESTION)
% Copyright 2009, Tony R. Kuphaldt, released under the Creative Commons Attribution License (v 1.0)
% This means you may do almost anything with this work of mine, so long as you give me proper credit

Read and outline the ``Motion-Balance Pneumatic Positioners'' subsection of the ``Valve Positioners'' section of the ``Control Valves'' chapter in your {\it Lessons In Industrial Instrumentation} textbook.  Note the page numbers where important illustrations, photographs, equations, tables, and other relevant details are found.  Prepare to thoughtfully discuss with your instructor and classmates the concepts and examples explored in this reading.

\underbar{file i01364}
%(END_QUESTION)





%(BEGIN_ANSWER)

Feel free to bring a Fisher model 3582 positioner to class for hands-on learning!
 
%(END_ANSWER)





%(BEGIN_NOTES)

Increasing signal adds motion to the beam; valve moves, making the other end move oppositely; result is that beam becomes angled to reach equilibrium.

\vskip 10pt

The Fisher 3582 positioner example: bellows motion forces D-ring to tilt on vertical axis; valve motion forces D-ring to tile on horizontal axis.  Flapper set at periphery of D-ring moves proportionately to these other motions.

\vskip 10pt

I recommend you have multiple 3582 positioners available for students to ``play'' with.  The mechanism lends itself easily to analysis without air pressure applied, so this can be done in any classroom setting.








\vskip 20pt \vbox{\hrule \hbox{\strut \vrule{} {\bf Suggestions for Socratic discussion} \vrule} \hrule}

\begin{itemize}
\item{} Examine the photographs of the Fisher model 3582 motion-balance positioner and explain how it works.  If you (the instructor) can provide one of these positioners for students to interact with in the classroom, so much the better!
\item{} Examine a random valve positioner provided by the instructor, analyzing the mechanism to see if it is {\it force-balance} or {\it motion-balance}.
\item{} Examine a random valve positioner provided by the instructor, analyzing the mechanism to identify the location and function of the {\it zero} adjustment.
\item{} Examine a random valve positioner provided by the instructor, analyzing the mechanism to identify the location and function of the {\it span} adjustment.
\end{itemize}


%INDEX% Final Control Elements, valve: positioner (Fisher model 3582)
%INDEX% Reading assignment: Lessons In Industrial Instrumentation, valve positioners (motion-balance pneumatic)

%(END_NOTES)


