
%(BEGIN_QUESTION)
% Copyright 2006, Tony R. Kuphaldt, released under the Creative Commons Attribution License (v 1.0)
% This means you may do almost anything with this work of mine, so long as you give me proper credit

Read and outline the ``Electronic Positioners'' subsection of the ``Valve Positioners'' section of the ``Control Valves'' chapter in your {\it Lessons In Industrial Instrumentation} textbook.  Note the page numbers where important illustrations, photographs, equations, tables, and other relevant details are found.  Prepare to thoughtfully discuss with your instructor and classmates the concepts and examples explored in this reading.

\underbar{file i01365}
%(END_QUESTION)





%(BEGIN_ANSWER)

 
%(END_ANSWER)





%(BEGIN_NOTES)

This type of positioner electronically senses valve stem position, then applies air pressure to achieve target position.  Self-diagnostics as an added benefit!

\vskip 10pt

In the event of a position sensor failure, some smart positioners shed to volume-booster mode: simply relaying the approximate amount of pressure needed to position the valve for any given input signal, based on prior pressure/position data.

\vskip 10pt

``Valve signature'' diagnostic graph: a plot of actuator pressure (force) versus stem motion, going up and going down.  Offset between opening and closing graphs represents the amount of friction in the valve mechanism.  The more friction, the more offset the two plots will be from one another.  Lower-left end of plot is the seating profile: a recording of force versus motion as the plug comes into contact with the seat.  Valve signatures may be recorded and archived for later comparison with the valve as it wears.  Electric valve actuators may record the same signatures by graphing motor current versus stem position.


\vskip 10pt

Perhaps the greatest advantage of digital valve positioners is the {\it diagnostic capability} they bring to control valves.  By measuring the amount of pressure needed to move a valve actuator for a given change in input signal, a digital positioner is able to inferentially measure packing friction, plug-to-seat friction, and other parameters indicating the health of the control valve.








\vskip 20pt \vbox{\hrule \hbox{\strut \vrule{} {\bf Suggestions for Socratic discussion} \vrule} \hrule}

\begin{itemize}
\item{} Identify some important capabilities of digital electronic valve positioners that mechanical positioners lack.
\item{} Examine the block diagram of an electronic positioner shown in the textbook, and explain how its two control loops function to maintain a constant valve stem position (given a constant control signal) despite outside influences on the valve stem (e.g. packing friction, forces on the plug from process fluid flow, air supply pressure changes).
\item{} Examine the block diagram of an electronic positioner shown in the textbook, and comment on the ``+'' and ``$-$'' symbols next to each summing ($\Sigma$) function block.  Explain how these symbols are analogous to the noninverting and inverting labels on opamp input terminals.
\item{} Explain what kind(s) of useful data are provided by a ``valve signature'' graph.
\item{} Explain why the ends of the ``valve signature'' graph shown in the book turn sharply vertical.
\item{} Explain how a ``valve signature'' graph may reveal valve seat wear.
\end{itemize}

%INDEX% Final Control Elements, valve: positioner (digital)

%(END_NOTES)


