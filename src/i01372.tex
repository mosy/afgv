
%(BEGIN_QUESTION)
% Copyright 2006, Tony R. Kuphaldt, released under the Creative Commons Attribution License (v 1.0)
% This means you may do almost anything with this work of mine, so long as you give me proper credit

What is the mathematical relationship between valve size (in inches) and maximum flow capacity ($C_v$)?  For example, how does the flow capacity of an 8-inch valve compare with the flow capacity of a 4-inch valve, all other factors being equal?

\underbar{file i01372}
%(END_QUESTION)





%(BEGIN_ANSWER)

$C_v$ is approximately proportional to the {\it square} of the nominal valve size.  The idea here is that the cross-sectional area for flow to go through will increase with the square of the flow path diameter ($A = \pi r^2$).

We see this in the formula for relative flow capacity ($C_d$), which is roughly constant for a given valve type (ball versus globe versus butterfly, etc.):

$$C_d = {C_v \over d^2}$$

Manipulating this to relate diameter ($d$) to flow coefficient ($C_v$):

$$d^2 = {C_v \over C_d}$$

Given a constant $C_d$ value, the flow coefficient ($C_v$) will increase with the {\it square} of the increasing diameter.


%(END_ANSWER)





%(BEGIN_NOTES)

%INDEX% Final Control Elements, valve: sizing

%(END_NOTES)


