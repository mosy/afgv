
%(BEGIN_QUESTION)
% Copyright 2013, Tony R. Kuphaldt, released under the Creative Commons Attribution License (v 1.0)
% This means you may do almost anything with this work of mine, so long as you give me proper credit

One solution to the problem of installed control valve characterization is to eliminate the control valve entirely, using a variable-speed pump to control liquid flow rather than a constant-speed pump and a throttling valve.  The most popular means of controlling the speed of an electric pump is to use a variable-frequency drive (VFD) to adjust the AC frequency of power sent to the driving motor.  VFDs provide the option of analog signal control, with analog input terminals that you may connect to the analog output terminals of a controller, so that the controller can command the pump motor to spin at different speeds just like it could command a throttling valve to go to different stem positions.

\vskip 10pt

Read the appropriate section(s) of the Allen-Bradley ``PowerFlex 4 Adjustable Frequency AC Drive user manual'' (document FRN 5.xx) in order to determine how to connect one of these VFDs to the 4-20 mA analog output of a loop controller.  Sketch a simple diagram showing how to connect a PowerFlex 4 VFD to the output of one of these controller outputs (you choose the controller!):

\vskip 10pt
\filbreak
\hbox{ \vrule
\vbox{ \hrule \vskip 3pt
\hbox{ \hskip 3pt
\vbox{ \hsize=5in \raggedright

\noindent \centerline{\bf Controller options}
\item{} Siemens 352P single-loop
\item{} Siemens 353 four-loop
\item{} Siemens 353R multi-loop with IO-8AO 8-channel analog output module
\vskip 2pt
\item{} Foxboro 762CNA dual-loop 
\item{} Foxboro 716C single-loop temperature
\item{} Foxboro 718TC single-loop temperature
\item{} Moore Industries 535 single-loop
%\item{} Honeywell UDC2000 single-loop
%\item{} Honeywell UDC2500 single-loop
%\item{} Honeywell UDC3200 single-loop
%\item{} Honeywell UDC3500 single-loop
\vskip 2pt
\item{} Emerson ROC800 SCADA/RTU with AO analog output module
\vskip 2pt
\item{} Emerson DeltaV DCS with M-series 2 AO 8-channel 4-20 mA output module
\end{itemize}

} \hskip 3pt}%
\vskip 5pt \hrule}%
\vrule}

\vskip 20pt \vbox{\hrule \hbox{\strut \vrule{} {\bf Suggestions for Socratic discussion} \vrule} \hrule}

\begin{itemize}
\item{} Which parameter(s) in the PowerFlex 4 VFD must be configured to allow 4-20 mA control?  Hint: this is typically referred to as the ``Speed Reference'' parameter for a VFD.
\item{} What other options are available for the ``Speed Reference'' of this VFD?
\end{itemize}

\underbar{file i01384}
%(END_QUESTION)





%(BEGIN_ANSWER)

%(END_ANSWER)





%(BEGIN_NOTES)

The 4-20 mA input on the PowerFlex 4 VFD is found on terminals 15 (+) and 14 ($-$).  Parameter {\tt P038} must be set to option {\tt 3} (``4-20 mA input'').

\vskip 10pt

Page 1-14 shows how a 4-20 mA signal is to be wired to the VFD, and also references the appropriate parameter.  A more comprehensive wiring diagram is shown on page 1-13.  Options for parameter {\tt P038} are found on page 3-11.

%INDEX% Final Control Elements, motor: variable frequency drive

%(END_NOTES)


