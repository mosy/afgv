
%(BEGIN_QUESTION)
% Copyright 2006, Tony R. Kuphaldt, released under the Creative Commons Attribution License (v 1.0)
% This means you may do almost anything with this work of mine, so long as you give me proper credit

Describe how a {\it positioner} may be used to change the inherent characteristic of a control valve, as an alternative to changing out the valve's trim.

\vskip 10pt

Also, explain why using a positioner to do this might not yield the best results.

\vskip 20pt \vbox{\hrule \hbox{\strut \vrule{} {\bf Suggestions for Socratic discussion} \vrule} \hrule}

\begin{itemize}
\item{} Describe a realistic scenario where someone might wish to alter the characteristic of a control valve, and how they would know this is the right thing to do.
\end{itemize}

\underbar{file i01385}
%(END_QUESTION)





%(BEGIN_ANSWER)

Positioners may be intentionally made non-linear, so that the valve stem (or shaft) position is not linearly related to the control signal as it would normally be.  Through the use of specially shaped cams and other mechanical components in the positioner's feedback mechanism, virtually any type of opening characteristic may be obtained from a single type of trim.
 
\vskip 10pt

This does not mean, though, that {\it accurate} equal percentage opening characteristics may be obtained from a valve trim that is inherently quick opening.  Because the positioner would have to move a quick-opening valve mechanism with supreme precision at the bottom end of its range in order to duplicate an equal percentage characteristic (to force an aggressively-opening valve to open gently instead as the control signal increases), this methodology cannot guarantee good accuracy throughout a valve's operating range.  However, it does have the distinct advantage of allowing changes in a valve's opening characteristic without having to disassemble the valve or remove it from the piping!

%(END_ANSWER)





%(BEGIN_NOTES)
 

%INDEX% Final Control Elements, valve: characterization

%(END_NOTES)


