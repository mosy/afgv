
%(BEGIN_QUESTION)
% Copyright 2006, Tony R. Kuphaldt, released under the Creative Commons Attribution License (v 1.0)
% This means you may do almost anything with this work of mine, so long as you give me proper credit

For most electric motors, the amount of {\it torque} output strongly influences the amount of current drawn from the power source.  For some electric motors (most notably, permanent magnet DC motors), current and torque are directly proportional to one another.

Explain what ``torque'' is, why it may be important to measure in a valve actuator mechanism, and how electric motors provide a convenient means for measuring torque.

\underbar{file i01389}
%(END_QUESTION)





%(BEGIN_ANSWER)

{\it Torque} may be simply defined as ``twisting force,'' mathematically defined as the product of force applied to the length of a moment arm.  If you have ever accidently applied enough torque to a hand valve to damage the seat or to jam it shut (so it cannot be opened), you know very well the importance of torque in a valve actuator mechanism.
 
\vskip 10pt

The most mathematically proper definition of torque ($\tau$) is the {\it vector cross-product} of force ($F$) and radius ($r$):

$$\vec \tau = \vec r \times \vec F$$

%(END_ANSWER)





%(BEGIN_NOTES)

It should be noted that while torque is the product of force and distance, it is not the same as {\it work}, which is also a product of force and distance:

$$W = \vec F \cdot \vec d$$

Torque is a cross-product, while work is a dot-product.  Cross-products are always vectors, while dot products are always scalars.  This makes intuitive sense, too: torque always exists along an {\it axis of rotation}, which necessarily has a direction.  Work has no intrinsic direction, but is a simple scalar quantity.

In honor of this distinction, the unit of measurement for torque in the English system is lb-ft, while the unit of measurement for work is the ft-lb.  In the metric system, newton-meters is used for both torque and work, which is unfortunate.

%INDEX% Final Control Elements, valve: electric actuator
%INDEX% Physics, torque: definition

%(END_NOTES)


