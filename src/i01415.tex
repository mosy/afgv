
%(BEGIN_QUESTION)
% Copyright 2006, Tony R. Kuphaldt, released under the Creative Commons Attribution License (v 1.0)
% This means you may do almost anything with this work of mine, so long as you give me proper credit

A gas flow control valve needs to be sized to pass 50,000 SCFH under these full-open conditions:

\medskip 
\item{}Temperature = 280$^{o}$ R
\item{}Upstream pressure = 3000 PSIG
\item{}Downstream pressure = 100 PSIG
\item{}Specific gravity = 1.1
\end{itemize} 
 
Calculate the necessary $C_{v}$ factor for this valve.
 
\underbar{file i01415}
%(END_QUESTION)





%(BEGIN_ANSWER)

If you calculated $C_{v}$ = 0.30248, you made a serious mistake!
 
\vskip 10pt

Hint: this is a ``trick'' question, to determine whether you are really thinking about the dynamics of gas flow, or just plugging numbers into equations.  If you were to actually install a valve with a $C_{v}$ rating of 0.30248 into this process, the resulting flow would be {\it much less} than 50,000 SCFH.

%(END_ANSWER)





%(BEGIN_NOTES)

This is a trick question because the stated conditions ensure ``choked'' flow through the valve, and the standard $C_{v}$ equation for subcritical flow does not apply.  The true $C_{v}$ for this valve must be {\it much} greater than what the ``subcritical'' gas flow equation predicts.

A general rule-of-thumb for predicting choked flow is when the $P_1$ / $P_{vc}$ absolute pressure ratio exceeds 2:1 (Source: {\it Instrument Engineer's Handbook -- Process Control}, third edition, page 605 under ``Choked Flow'' section).  When this occurs, it is almost guaranteed that flow velocity through the valve will reach sonic (critical) speed, thus resulting in much less actual gas flow than what an assumption of subcritical flow would predict.

This rule of thumb is echoed by William J. Palm in his book {\it Control Systems Engineering} on page 85, where he states that air flow through a valve will be subsonic if the ratio of downstream pressure to upstream pressure ($P_2 \over P_1$) is greater than 0.528 (that is to say, the ratio of upstream to downstream pressure $P_1 \over P_2$ is less than 1.894).  This implies that critical flow for a gas will occur somewhere around an absolute pressure ratio of 2:1, upstream to downstream.

\vskip 10pt

In this particular problem, we do not know the pressure at the vena contracta, but we do know the upstream and downstream pressures.  Calculating the ratio $P_1$ / $P_2$, we arrive at a figure of 30.  Since the vena contracta pressure ($P_{vc}$) is bound to be less than the downstream pressure ($P_{2}$), the ratio of $P_1$ / $P_{vc}$ must be greater than $P_1$ / $P_2$, so our actual pressure ratio from inlet to vena contracta must exceed 30:1.  If a ratio of 2:1 virtually guarantees choked flow, then we can be absolutely sure of choked flow in this case!

\vskip 10pt

Now, there are ways to estimate the necessary valve size for an application such as this, but we do not have enough information about the piping or the gas to perform the necessary calculations.  Some of the missing parameters include: $x_T$ (critical pressure drop ratio factor), $Y$ (expansion factor), and $F_K$ (ratio of specific heats factor).

%INDEX% Final Control Elements, valve: sizing

%(END_NOTES)


