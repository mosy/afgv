
%(BEGIN_QUESTION)
% Copyright 2006, Tony R. Kuphaldt, released under the Creative Commons Attribution License (v 1.0)
% This means you may do almost anything with this work of mine, so long as you give me proper credit

If undersizing a control valve leads to the inability to achieve a desired flow rate, why not just oversize all control valves to ensure the desired flow rates may be achieved?  What would be wrong with adopting this philosophy of control valve sizing?

\underbar{file i01422}
%(END_QUESTION)





%(BEGIN_ANSWER)

Oversized control valves, while definitely capable of achieving the desired flow rates, make it impossible to precisely {\it control} flow, especially at low flow rates.

\vskip 10pt

To figure out why, I give you the following ``thought experiment:'' imagine replacing the small hand valve on your kitchen sink faucet with an incredibly large valve, say something the size of a fire hydrant valve.  With this ridiculously oversized valve connected to the same plumbing in your house, what would it be like trying to control the flow of water in your kitchen sink?
 
%(END_ANSWER)





%(BEGIN_NOTES)

%INDEX% Final Control Elements, valve: undersizing

%(END_NOTES)


