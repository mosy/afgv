
%(BEGIN_QUESTION)
% Copyright 2006, Tony R. Kuphaldt, released under the Creative Commons Attribution License (v 1.0)
% This means you may do almost anything with this work of mine, so long as you give me proper credit

Describe what a {\it metering pump} is and what one might be used for.

\underbar{file i01427}
%(END_QUESTION)





%(BEGIN_ANSWER)

%(END_ANSWER)





%(BEGIN_NOTES)

A {\it metering pump} is a low-volume, variable-flow pump usually used as a final control element for adding liquid chemical substances to a process.  The flow output of a metering pump is often pulsed: that is, metering pumps usually deliver liquid in ``spurts'' rather than continuous flows.
 
\vskip 10pt

The pulsating nature of most metering pumps is a function of their design.  Many metering pumps use solenoid-actuated pistons or diaphragms, the actuation strokes timed by an electronic circuit in response to the controlling signal.  For instance, a metering pump may control liquid delivery to a process by varying the number of pump strokes per minute.  Since each pump stroke contains a fixed and known quantity of liquid, strokes per minute equates to a certain volumetric quantity (cc, ml, fl. oz.) per minute.
 
Metering pumps are popularly used for reagent addition in water treatment systems (addition of polymer for solids control, acid/caustic for pH control, bleach for disinfection, corrosion inhibitors, etc.) where small flow quantities are needed and pulsed flow is inconsequential.  In medical applications, metering pumps are used to deliver liquid nutrients and medications to patients through I.V. tubes.

%INDEX% Final Control Elements, pump: metering pump

%(END_NOTES)


