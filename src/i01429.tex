
%(BEGIN_QUESTION)
% Copyright 2006, Tony R. Kuphaldt, released under the Creative Commons Attribution License (v 1.0)
% This means you may do almost anything with this work of mine, so long as you give me proper credit

Roy has the meanest pulling tractor in his county: its engine outputs a maximum torque of 1200 lb-ft, and the total geartrain (transmission combined with rear axle differential gearing) has a 12:1 reduction ratio in the lowest gear.  With 5.5 foot tall tires, how much horizontal pulling force can this tractor (theoretically) exert?

\vskip 10pt

If Roy's tractor drags a weight 300 feet along the ground while pulling at maximum engine torque, how much work was done by the tractor?

\vskip 10pt

Rate the horsepower of Roy's tractor if it took exactly 1 minute to drag that weight 300 feet along the ground.

\vskip 10pt

When Roy goes to the county fair to compete in the tractor-pull contest, he notices that the front end of the tractor tends to raise up off the ground when pulling a heavy load.  Explain to Roy why this happens.

\underbar{file i01429}
%(END_QUESTION)





%(BEGIN_ANSWER)

Maximum pulling force = 5236.36 pounds
 
\vskip 10pt

$$W = F x = (5236.36 \hbox{ lb}) (300 \hbox{ ft}) = 1570909.1 \hbox{ ft-lb}$$

\vskip 10pt

1 horsepower is 550 ft-lb of work done per second.  If Roy's tractor did 1,570,909.1 ft-lb of work in 60 seconds, it is equivalent to 26,181.8 ft-lb/s of power, which is 47.6 horsepower.

\vskip 10pt

As the tractor mechanism exerts torque on the wheels, and the weight of the load opposes the wheels' turning, the tractor experiences this torque about the axis of rotation: the axles.  As the wheels {\it try} to rotate in a forward direction, but are impeded by the resistance of the load, the reaction torque {\it tries to rotate the tractor backward about the same axis}.  This manifests itself in the form of the front tires of the tractor lifting off the ground.

%(END_ANSWER)





%(BEGIN_NOTES)


%INDEX% Machine, gear ratio
%INDEX% Physics, energy, work, power: calculating work and power
%INDEX% Physics, torque: calculation problem

%(END_NOTES)


