
%(BEGIN_QUESTION)
% Copyright 2006, Tony R. Kuphaldt, released under the Creative Commons Attribution License (v 1.0)
% This means you may do almost anything with this work of mine, so long as you give me proper credit

Calculate the mechanical power output by an electric motor (in units of horsepower) as is delivers 1250 lb-ft of torque at 850 RPM.  Then, calculate the line current for this motor if it is a 3-phase unit operating at a line voltage of 480 volts.  Assume 92\% efficiency for the motor.

\vskip 20pt \vbox{\hrule \hbox{\strut \vrule{} {\bf Suggestions for Socratic discussion} \vrule} \hrule}

\begin{itemize}
\item{} How might the results differ if the motor were 100\% efficient instead of 92\% efficient?
\item{} Explain how you may double-check your quantitative answer(s) with a high degree of confidence (i.e. something more rigorous than simply re-working the problem again in the same way).
\item{} Suppose we were to alter this problem to describe a diesel engine turning a three-phase {\it generator} with an efficiency of 92\%, at 1250 lb-ft of torque and a shaft speed of 850 RPM.  At a line voltage of 480 volts, how much line current could we expect the generator to output?  Is the answer the same as in the case of the motor?  Explain why or why not.
\end{itemize}


\underbar{file i01434}
%(END_QUESTION)





%(BEGIN_ANSWER)

$P$ = 202.3 hp

\vskip 10pt

If you calculated 181.5 amps for line current, you're close -- you have assumed 100\% efficiency for the motor!  The actual line current is 197.2 amps if you take the motor's 92\% efficiency into account.
 
\vskip 10pt

Here is a formula you can use to convert torque (lb-ft) and speed (RPM) values into horsepower:

$$P = {S \tau \over 5252.113}$$

I don't expect anyone to memorize a formula like this, but one may derive it from a ``thought experiment.''  It should be intuitively obvious that power ($P$) must be directly proportional to both torque ($\tau$) and speed ($S$), with some constant of proportionality ($k$) included to account for units:

$$P \propto S \tau $$

$$P = k S \tau $$

If we were to imagine a 1-foot radius drum hoisting a 550 pound rate vertically at 1 foot per second as an example of a machine exerting exactly 1 horsepower, we may solve for $\tau$ and $S$, then calculate the necessary constant to make $P$ equal to 1.  The drum's torque would be 550 lb-ft, of course (550 lb of force exerted over a moment arm of 1 foot).  With a circumferential speed of 1 foot per second, it would rotate at $1 \over {2 \pi}$ revolutions per second, or $30 \over \pi$ RPM.  If $\tau$ = 550 and $S$ = $30 \over \pi$ and $P$ = 1 horsepower, then:

$$P = {\pi S \tau \over {30 \times 550}}$$

\vskip 10pt

In answer to the Socratic discussion question, the 92\% efficiency works to diminish output current, rather than increase input current as in the case of the motor.  Thus, the diesel-powered generator will output a line current of 167 amps.

%(END_ANSWER)





%(BEGIN_NOTES)


%INDEX% Electronics review: 3-phase voltage/current/power calculation
%INDEX% Physics, torque: calculation problem

%(END_NOTES)


