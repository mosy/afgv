
%(BEGIN_QUESTION)
% Copyright 2010, Tony R. Kuphaldt, released under the Creative Commons Attribution License (v 1.0)
% This means you may do almost anything with this work of mine, so long as you give me proper credit

Identify the physical phenomenon used in the following analytical instrument types to sense chemical composition.  Note: more than one analyzer type may share the same property (same letter), but there is one {\it best} choice for each analyzer!

\begin{itemize}
\item{} Electrodeless conductivity water analyzer: \underbar{\hskip 50pt}
\item{} pH (potentiometric) water analyzer: \underbar{\hskip 50pt}
\item{} NDIR carbon monoxide analyzer: \underbar{\hskip 50pt}
\item{} NO$_{x}$ gas analyzer: \underbar{\hskip 50pt}
\item{} SO$_{2}$ gas analyzer: \underbar{\hskip 50pt}
\end{itemize}

\vskip 10pt

\begin{itemize}
\item {\bf A}: Absorption of specific light wavelengths by sample
\item {\bf B}: Chemiluminescence
\item {\bf C}: Nernst equation (voltage produced across a membrane)
\item {\bf D}: Optical fluorescence
\item {\bf E}: Emission of specific light wavelengths by sample
\item {\bf F}: Electromagnetic induction
\item {\bf G}: Endothermic chemical reaction with sample
\item {\bf H}: Kelvin (4-wire) resistance measurement 
\item {\bf I}: Paramagnetism
\end{itemize}

\underbar{file i01442}
%(END_QUESTION)





%(BEGIN_ANSWER)

\begin{itemize}
\item{} Electrodeless conductivity water analyzer: \underbar{\bf F}
\item{} pH (potentiometric) water analyzer: \underbar{\bf C}
\item{} NDIR carbon monoxide analyzer: \underbar{\bf A}
\item{} NO$_{x}$ gas analyzer: \underbar{\bf B}
\item{} SO$_{2}$ gas analyzer: \underbar{\bf D}
\end{itemize}

%(END_ANSWER)





%(BEGIN_NOTES)

{\bf This question is intended for exams only and not worksheets!}.

%(END_NOTES)


