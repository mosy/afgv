
%(BEGIN_QUESTION)
% Copyright 2006, Tony R. Kuphaldt, released under the Creative Commons Attribution License (v 1.0)
% This means you may do almost anything with this work of mine, so long as you give me proper credit

Skriv et pseudokode-program for en mikrokontroller som implementerer en enkel av/på-termostat for temperaturregulering. Programmet ditt bør inneholde en variabel kalt {\tt SP} (Setpoint/Skal-verdi) som bestemmer ved hvilken målt temperatur varmeovnen skal slås av. Sørg for at du inkluderer den nødvendige betingelsessetningen for å slå varmeovnen {\it på} igjen!

\underbar{file i01454}
%(END_QUESTION)





%(BEGIN_ANSWER)

Dette er bare ett eksempel, og ikke det eneste korrekte svaret:

\vskip 10pt

\hbox{ \vrule
\vbox{ \hrule \vskip 3pt
\hbox{ \hskip 3pt
\vbox{ \hsize=2.5in \raggedright

\noindent
\underbar{\bf Pseudocode listing}

{\tt LOOP}

\hskip 10pt {\tt IF Temp > SP THEN Heater = Off}

\hskip 10pt {\tt ELSE Heater = On}

{\tt ENDLOOP}
}
\hskip 3pt}%
\vskip 5pt \hrule}%
\vrule}

\vskip 10pt

En mer sofistikert versjon av dette programmet ville inkludere {\it hysterese} (også kalt {\it dødbånd}):

\vskip 10pt

\hbox{ \vrule
\vbox{ \hrule \vskip 3pt
\hbox{ \hskip 3pt
\vbox{ \hsize=2.5in \raggedright

\noindent
\underbar{\bf Pseudocode listing}

{\tt LOOP}

\hskip 10pt {\tt IF Temp > SP\_high THEN Heater = Off}

\hskip 10pt {\tt IF Temp < SP\_low THEN Heater = On}

{\tt ENDLOOP}
}
\hskip 3pt}%
\vskip 5pt \hrule}%
\vrule}

%(END_ANSWER)





%(BEGIN_NOTES)

Be studentene forklare hvorfor et dødbånd (hysterese) ville være en god funksjon å inkludere i en temperaturregulator.

%INDEX% Control, basics: on/off control (implemented in a microcontroller)

%(END_NOTES)
