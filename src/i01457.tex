
%(BEGIN_QUESTION)
% Copyright 2006, Tony R. Kuphaldt, released under the Creative Commons Attribution License (v 1.0)
% This means you may do almost anything with this work of mine, so long as you give me proper credit

Imagine driving an automobile with very sensitive steering: just a few degrees of steering wheel motion at highway speeds is sufficient to quickly change lanes.  Now imagine driving an automobile having significantly less sensitive steering: a whole quarter-turn of the steering wheel is needed to generate the same response as a few degrees of rotation in the first vehicle.

An important process quantity is its {\it gain}.  How would you qualify the two automobile steering systems just described in terms of process gain, from the perspective of lane position as the process variable, steering wheel angle as the manipulated variable, and you (the driver) as the proportional controller?  Which automobile has a high process gain, and which has a low process gain?

\underbar{file i01457}
%(END_QUESTION)





%(BEGIN_ANSWER)

The automobile with ``sensitive'' steering has the greater process gain.  As always, the ``gain'' of a system is a ratio of its output change to its input change ($\Delta \hbox{Out} \over \Delta \hbox{In}$, or $d \hbox{Out} \over d \hbox{In}$), and process gain is no exception.

%(END_ANSWER)





%(BEGIN_NOTES)


%INDEX% Control, proportional: process gain

%(END_NOTES)


