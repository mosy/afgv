
%(BEGIN_QUESTION)
% Copyright 2006, Tony R. Kuphaldt, released under the Creative Commons Attribution License (v 1.0)
% This means you may do almost anything with this work of mine, so long as you give me proper credit

En tekniker er engasjert i å "tune" en prosessregulator (PID) som en del av igangkjøringsprosedyren for et nytt anlegg. Teknikeren finner ut at en regulatorforsterkning på 0,5 fungerer bra for prosessen, og gir en kvart bølges dempning (quarter-wave damping) etter en forstyrrelse.

Senere bestemmer en driftsingeniør seg for å bytte ut reguleringsventilen i denne prosessløyfen med en som har en "trim" som gir dobbelt så stor gjennomstrømning for samme åpningsgrad som den gamle ventilen.

Hva, om noe, må teknikeren gjøre med regulatorens tuning for å opprettholde samme stabilitet i systemet som før?

\vfil

\underbar{file i01460}
\eject
%(END_QUESTION)





%(BEGIN_ANSWER)

Regulatoren må tunes med en ny forsterkning på 0,25 (halvparten av den forrige verdien), for å kompensere for doblingen av ventilforsterkningen.

%(END_ANSWER)





%(BEGIN_NOTES)

En analogi jeg ofte bruker for å forklare denne nødvendigheten av "re-tuning" er en bil som har fått byttet girkasse til en med mye mer aggressivt girforhold. For at bilen fortsatt skal være like smidig å kjøre, må sjåføren tråkke mer forsiktig på gasspedalen (mindre forsterkning i "sjåfør-regulatoren").

En annen analogi er lydanlegget i et auditorium: hvis noen bytter ut effektforsterkeren med en som er mye kraftigere, må personen ved miksebordet skyve "master volume"-spaken ned (mindre forsterkning) for å opprettholde samme lydstyrke som før.

%INDEX% Control, proportional: controller gain

%(END_NOTES)
