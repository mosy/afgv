
%(BEGIN_QUESTION)
% Copyright 2006, Tony R. Kuphaldt, released under the Creative Commons Attribution License (v 1.0)
% This means you may do almost anything with this work of mine, so long as you give me proper credit

Definer følgende begreper med tanke på P-regulatorer:

\begin{itemize}
\item{} Proporsjonalbånd
\item{} Forsterkning (Gain)
\end{itemize}

Hvilket av disse to begrepene forteller oss hvor mye regulatoren reagerer på endringer i avviket? Forklar ditt svar med et eksempel.

\vfil

\underbar{file i01461}
\eject
%(END_QUESTION)





%(BEGIN_ANSWER)

{\it Proporsjonalbånd} er definert som mengden av endring i prosessvariabelen som er nødvendig for å endre regulatorens utgang (pådrag) fra 0\% til 100\%.

{\it Forsterkning} er definert som forholdet mellom endring i utgang og endring i inngang.

$$\mbox{Forsterkning} = \frac{\Delta \mbox{Utgang}}{\Delta \mbox{Inngang}}$$

Begge begrepene kvantifiserer responsiviteten til en regulator.

%(END_ANSWER)





%(BEGIN_NOTES)

Det er viktig at studentene forstår at {\it proporsjonalbånd} og {\it forsterkning} (gain) er to forskjellige måter å uttrykke nøyaktig det samme på: hvor kraftig regulatoren vil reagere på endringer i inngangssignalet (avviket). Sørg for at du diskuterer inversiteten mellom de to: en regulator med høy forsterkning har et smalt proporsjonalbånd, og omvendt.

%INDEX% Control, proportional: proportional band versus gain

%(END_NOTES)
