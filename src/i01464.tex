
%(BEGIN_QUESTION)
% Copyright 2011, Tony R. Kuphaldt, released under the Creative Commons Attribution License (v 1.0)
% This means you may do almost anything with this work of mine, so long as you give me proper credit

Suppose a Limitorque model L120-10 electric valve actuator has a problem: the valve remains in the fully closed position and refuses to open when the local ``open'' switch (pushbutton PB1) is pressed.  Note: you will need to locate the schematic diagram for this MOV in order to diagnose this problem.

Your first diagnostic test is to measure AC voltage between terminals 21 and 20: there your meter shows you 117 volts AC.  After that, you measure 117 volts AC between terminals 1 and 20 with the ``open'' pushbutton PB1 depressed (pushed) by a fellow technician helping you diagnose the problem.

Identify the likelihood of each specified fault for this circuit.  Consider each fault one at a time (i.e. no coincidental faults), determining whether or not each fault could independently account for {\it all} measurements and symptoms in this circuit.

% No blank lines allowed between lines of an \halign structure!
% I use comments (%) instead, so that TeX doesn't choke.

$$\vbox{\offinterlineskip
\halign{\strut
\vrule \quad\hfil # \ \hfil & 
\vrule \quad\hfil # \ \hfil & 
\vrule \quad\hfil # \ \hfil \vrule \cr
\noalign{\hrule}
%
% First row
{\bf Fault} & {\bf Possible} & {\bf Impossible} \cr
%
\noalign{\hrule}
%
% Another row
``Bypass'' switch (\#1) failed open &  &  \cr
%
\noalign{\hrule}
%
% Another row
``Bypass'' switch (\#5) failed open &  &  \cr
%
\noalign{\hrule}
%
% Another row
``Opening'' torque switch (\#18) failed open &  &  \cr
%
\noalign{\hrule}
%
% Another row
``Closing'' torque switch (\#17) failed open &  &  \cr
%
\noalign{\hrule}
%
% Another row
``Open'' limit switch (\#4) failed open &  &  \cr
%
\noalign{\hrule}
%
% Another row
``O'' relay coil failed open &  &  \cr
%
\noalign{\hrule}
%
% Another row
``C'' relay coil failed open &  &  \cr
%
\noalign{\hrule}
%
% Another row
Thermal overload switch(es) tripped &  &  \cr
%
\noalign{\hrule}
%
% Another row
480V fuse(s) blown to input of transformer &  &  \cr
%
\noalign{\hrule}
} % End of \halign 
}$$ % End of \vbox

Finally, identify the {\it next} diagnostic test or measurement you would make on this system.  Explain how the result(s) of this next test or measurement help further identify the location and/or nature of the fault.

\underbar{file i01464}
%(END_QUESTION)





%(BEGIN_ANSWER)

\noindent
{\bf Partial answer:}

% No blank lines allowed between lines of an \halign structure!
% I use comments (%) instead, so that TeX doesn't choke.

$$\vbox{\offinterlineskip
\halign{\strut
\vrule \quad\hfil # \ \hfil & 
\vrule \quad\hfil # \ \hfil & 
\vrule \quad\hfil # \ \hfil \vrule \cr
\noalign{\hrule}
%
% First row
{\bf Fault} & {\bf Possible} & {\bf Impossible} \cr
%
\noalign{\hrule}
%
% Another row
``Bypass'' switch (\#1) failed open &  &  \cr
%
\noalign{\hrule}
%
% Another row
``Bypass'' switch (\#5) failed open &  &  \cr
%
\noalign{\hrule}
%
% Another row
``Opening'' torque switch (\#18) failed open & $\surd$ &  \cr
%
\noalign{\hrule}
%
% Another row
``Closing'' torque switch (\#17) failed open &  &  \cr
%
\noalign{\hrule}
%
% Another row
``Open'' limit switch (\#4) failed open &  &  \cr
%
\noalign{\hrule}
%
% Another row
``O'' relay coil failed open & $\surd$ &  \cr
%
\noalign{\hrule}
%
% Another row
``C'' relay coil failed open &  & $\surd$ \cr
%
\noalign{\hrule}
%
% Another row
Thermal overload switch(es) tripped &  &  \cr
%
\noalign{\hrule}
%
% Another row
480V fuse(s) blown to input of transformer &  & $\surd$ \cr
%
\noalign{\hrule}
} % End of \halign 
}$$ % End of \vbox

%(END_ANSWER)





%(BEGIN_NOTES)

A reading of 120 VAC between terminals 21 and 20 proves that we have 120 VAC control power available.  A reading of 120 VAC between terminals 1 and 20 proves we have power all the way to and through the ``Open'' pushbutton when pressed.  The fault must lie between the blue wire (to the right of the Open pushbutton) and wire P2 at the overload contacts.

% No blank lines allowed between lines of an \halign structure!
% I use comments (%) instead, so that TeX doesn't choke.

$$\vbox{\offinterlineskip
\halign{\strut
\vrule \quad\hfil # \ \hfil & 
\vrule \quad\hfil # \ \hfil & 
\vrule \quad\hfil # \ \hfil \vrule \cr
\noalign{\hrule}
%
% First row
{\bf Fault} & {\bf Possible} & {\bf Impossible} \cr
%
\noalign{\hrule}
%
% Another row
``Bypass'' switch (\#1) failed open &  & $\surd$ \cr
%
\noalign{\hrule}
%
% Another row
``Bypass'' switch (\#5) failed open & $\surd$ &  \cr
%
\noalign{\hrule}
%
% Another row
``Opening'' torque switch (\#18) failed open & $\surd$ &  \cr
%
\noalign{\hrule}
%
% Another row
``Closing'' torque switch (\#17) failed open &  & $\surd$ \cr
%
\noalign{\hrule}
%
% Another row
``Open'' limit switch (\#4) failed open & $\surd$ &  \cr
%
\noalign{\hrule}
%
% Another row
``O'' relay coil failed open & $\surd$ &  \cr
%
\noalign{\hrule}
%
% Another row
``C'' relay coil failed open &  & $\surd$ \cr
%
\noalign{\hrule}
%
% Another row
Thermal overload switch(es) tripped & $\surd$ &  \cr
%
\noalign{\hrule}
%
% Another row
480V fuse(s) blown to input of transformer &  & $\surd$ \cr
%
\noalign{\hrule}
} % End of \halign 
}$$ % End of \vbox

If the ``bypass'' switch (\#5) were failed open, it might prevent the valve from opening if the opening torque switch tripped due to the valve being stuck in the seat.  If the ``opening'' torque switch (\#18) were failed open, the bypass switch (\#5) would allow the motor to move in the ``open'' direction just a bit, but as soon as the bypass switch returned to its normal (open) state the valve would refuse to open any further.

The thermal overload could be tripped if and only if it happened to trip during the last ``close'' cycle, for instance if the ``close'' torque switch failed and allowed the motor to become overloaded.

\vskip 10pt

A good ``next test'' would be to try taking voltage measurements at switch terminals in the ``open'' rung of the circuit with the button pressed as before.  For example, measuring between terminal 18C on the open torque switch and terminal 20 would tell you (by a 0 volt reading) if either switch \#4 was open or both switch \#5 and \#18 were open.







\vskip 20pt \vbox{\hrule \hbox{\strut \vrule{} {\bf Virtual Troubleshooting} \vrule} \hrule}

This question is a good candidate for a ``Virtual Troubleshooting'' exercise.  Presenting the diagram to students, you first imagine in your own mind a particular fault in the system.  Then, you present one or more symptoms of that fault (something noticeable by an operator or other user of the system).  Students then propose various diagnostic tests to perform on this system to identify the nature and location of the fault, as though they were technicians trying to troubleshoot the problem.  Your job is to tell them what the result(s) would be for each of the proposed diagnostic tests, documenting those results where all the students can see.

During and after the exercise, it is good to ask students follow-up questions such as:

\begin{itemize}
\item{} What does the result of the last diagnostic test tell you about the fault?
\item{} Suppose the results of the last diagnostic test were different.  What then would that result tell you about the fault?
\item{} Is the last diagnostic test the best one we could do?
\item{} What would be the ideal order of tests, to diagnose the problem in as few steps as possible?
\end{itemize}














\vfil \eject

\noindent
{\bf Prep Quiz:}

A Limitorque model L120-10 valve actuator uses which technology to move a control valve?

\begin{itemize}
\item{} A pneumatic diaphragm
\vskip 5pt 
\item{} A DC electric motor
\vskip 5pt 
\item{} A hydraulic piston
\vskip 5pt 
\item{} A hydraulic diaphragm
\vskip 5pt 
\item{} An AC electric motor
\vskip 5pt 
\item{} A pneumatic piston
\end{itemize}


%INDEX% Reading assignment: Limitorque L120 electric valve actuator manual
%INDEX% Troubleshooting review: electric circuits

%(END_NOTES)


