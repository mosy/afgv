
%(BEGIN_QUESTION)
% Copyright 2006, Tony R. Kuphaldt, released under the Creative Commons Attribution License (v 1.0)
% This means you may do almost anything with this work of mine, so long as you give me proper credit

{\it Digital} proportional controllers generate their output signal values using a microprocessor to repeatedly evaluate the proportional equation:

$$m = K_p e + b$$
 
What would happen if a {\it negative} value were entered for gain ($K_p$) into the digital controller's program?

\underbar{file i01467}
%(END_QUESTION)





%(BEGIN_ANSWER)

The effect of a negative $K_p$ value in a digital controller's algorithm would be to reverse the control action (from reverse-acting to direct-acting, or from direct-acting to reverse-acting), because a positive error would {\it decrease} the output, and vice-versa.  This is assuming, of course, that the controller is programmed to accept such values.  A wise programmer might make it impossible to enter negative tuning constant values, to avoid confusion from someone accidently entering one and unintentionally reversing the control action.

%(END_ANSWER)





%(BEGIN_NOTES)


%INDEX% Control, proportional: digital electronic controller

%(END_NOTES)


