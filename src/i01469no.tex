
%(BEGIN_QUESTION)
% Copyright 2006, Tony R. Kuphaldt, released under the Creative Commons Attribution License (v 1.0)
% This means you may do almost anything with this work of mine, so long as you give me proper credit

Plott responsen for følgende P-regulator (direktevirkende, gain = 1.5, bias = 0\%), forutsatt en trinnvis endring i prosessvariabelen (PV):

$$\includegraphics{i01469x01.eps}$$

\underbar{file i01469}
%(END_QUESTION)





%(BEGIN_ANSWER)

$$\includegraphics[width=10cm]{i01469x02.eps}$$

%(END_ANSWER)





%(BEGIN_NOTES)

En vanlig feil blant studentene er å glemme bias-verdien. De vil ofte plotte utgangssignalet som starter på 50\% (standard, halv skala) i stedet for den spesifiserte bias-verdien.

De som husker bias-verdien vil kanskje fortsatt gjøre feil med endringsretningen. "Skal den gå opp eller ned?" spør de seg selv. Min metode for å besvare dette spørsmålet er å se for meg at PV øker. Hvis regulatoren er direktevirkende, skal utgangen også øke. Dette betyr at PV og Output beveger seg i samme retning. Derfor, hvis PV går {\it opp}, skal Output gå {\it opp}.

%INDEX% Control, proportional: graphing controller response

%(END_NOTES)
