
%(BEGIN_QUESTION)
% Copyright 2006, Tony R. Kuphaldt, released under the Creative Commons Attribution License (v 1.0)
% This means you may do almost anything with this work of mine, so long as you give me proper credit

Plott responsen for følgende P-regulator (reversvirkende, gain = 4, bias = 40\%), forutsatt en trinnvis endring i prosessvariabelen (PV):

$$\includegraphics{i01484x01.eps}$$

\underbar{file i01484}
%(END_QUESTION)





%(BEGIN_ANSWER)

$$\includegraphics[width=10cm]{i01484x02.eps}$$

%(END_ANSWER)





%(BEGIN_NOTES)

En vanlig feil blant studentene er å glemme bias-verdien. De vil ofte plotte utgangssignalet som starter på 0\% eller 50\% i stedet for 40\%.

De som husker bias-verdien vil kanskje fortsatt gjøre feil med endringsretningen. "Skal den gå opp eller ned?" spør de seg selv. Min metode for å besvare dette spørsmålet er å se for meg at PV øker. Hvis regulatoren er reversvirkende, skal utgangen gå ned. Dette betyr at PV og Output beveger seg i motsatte retninger. Derfor, hvis PV går {\it ned} (som vist i grafen), skal Output gå {\it opp}.

%INDEX% Control, proportional: graphing controller response

%(END_NOTES)
