
%(BEGIN_QUESTION)
% Copyright 2006, Tony R. Kuphaldt, released under the Creative Commons Attribution License (v 1.0)
% This means you may do almost anything with this work of mine, so long as you give me proper credit

Her er et pseudokode-program for en digital prosessregulator:

\vskip 10pt

\hbox{ \vrule
\vbox{ \hrule \vskip 3pt
\hbox{ \hskip 3pt
\vbox{ \hsize=5in \raggedright

\noindent
\underbar{\bf Pseudocode listing}

\vskip 10pt

{\tt Declare Pin0 as an analog input (scale 0 to 5 volts = 0 to 1023)}

{\tt Declare Pin1 as an analog output (scale 0 to 5 volts = 0 to 1023)}

{\tt Declare SP as a variable, initially set to a value of 614}

{\tt Declare GAIN as a variable, initially set to a value of 1.0}

{\tt Declare ERROR as a variable}

{\tt Declare BIAS as a constant = 614}

\vskip 10pt

{\tt LOOP}

\hskip 10pt {\tt SET ERROR = Pin0 - SP}

\hskip 10pt {\tt SET Pin1 = (GAIN * ERROR) + BIAS}

{\tt ENDLOOP}
}
\hskip 3pt}%
\vskip 5pt \hrule}%
\vrule}


\vskip 10pt

Identifiser følgende parametere og funksjoner for dette dataprogrammet:
\begin{itemize}
\item{} Handlingsmodus: {\it direktevirkende} eller {\it reversvirkende}?
\item{} Hvor kommer signalet for prosessvariabelen (PV) fra?
\item{} Hvor går utgangssignalet (output) hen?
\item{} Hva er verdien for proporsjonalbåndet?
\item{} Hva er verdien for skal-verdien (setpoint) i prosent?
\item{} Hva er verdien for bias i prosent?
\item{} Endre dette programmet til å inkludere en PV-alarm, som slår på en LED-alarm hvis PV overstiger en viss verdi, og slår den av igjen når PV faller under en annen verdi.
\end{itemize}


\underbar{file i01486}
%(END_QUESTION)





%(BEGIN_ANSWER)

Regulatorkoden som vist implementerer {\it direkte} virkemåte, siden avviket beregnes som PV $-$ SP.

\vskip 10pt

Følgende tillegg gir denne regulatoren muligheten til å veksle mellom direkte eller revers virkemåte:

$$\includegraphics[width=15.5cm]{i01486x02.eps}$$

\hbox{ \vrule
\vbox{ \hrule \vskip 3pt
\hbox{ \hskip 3pt
\vbox{ \hsize=5in \raggedright

\noindent
\underbar{\bf Pseudocode listing}

\vskip 10pt

{\tt Declare Pin0 as an analog input (scale 0 to 5 volts = 0 to 1023)}

{\tt Declare Pin1 as an analog output (scale 0 to 5 volts = 0 to 1023)}

{\tt Declare Pin7 as a discrete input}

{\tt Declare SP as a variable, initially set to a value of 614}

{\tt Declare GAIN as a variable, initially set to a value of 1.0}

{\tt Declare ERROR as a variable}

{\tt Declare BIAS as a constant = 614}

\vskip 10pt

{\tt LOOP}

\hskip 10pt {\tt IF Pin7 = 0, SET ERROR = Pin0 - SP }

\hskip 10pt {\tt ELSE, SET ERROR = SP - Pin0}

\hskip 10pt {\tt ENDIF}

\vskip 10pt

\hskip 10pt {\tt SET Pin1 = (GAIN * ERROR) + BIAS}

{\tt ENDLOOP}
}
\hskip 3pt}%
\vskip 5pt \hrule}%
\vrule}


\vskip 10pt

Selv om en veldig treg kjøretid for programmet kan være dårlig for kontroll, kan det faktisk tjene en nyttig hensikt i noen prosesser. I prosesser med store dødtider (transportforsinkelser), er en kontrollstrategi kalt {\it sample-and-hold}, som er nettopp hva dette programmet ville være dersom en målrettet og betydelig forsinkelse ble lagt inn i sløyfen.

%(END_ANSWER)





%(BEGIN_NOTES)

\vfil \eject

\noindent
{\bf Summary Quiz:}

Avgjør om dette digitalregulator-programmet er {\it direktevirkende} eller {\it reversvirkende}:

\vskip 10pt

\hbox{ \vrule
\vbox{ \hrule \vskip 3pt
\hbox{ \hskip 3pt
\vbox{ \hsize=5in \raggedright

\noindent
\underbar{\bf Pseudocode listing}

\vskip 10pt

{\tt Declare Pin0 as an analog input (scale 0 to 5 volts = 0 to 1023)} {\it (PV)}

{\tt Declare Pin1 as an analog output (scale 0 to 5 volts = 0 to 1023)}

{\tt Declare SP as a variable, initially set to a value of 512}

{\tt Declare GAIN as a variable, initially set to a value of 1.0}

{\tt Declare ERROR as a variable}

{\tt Declare BIAS as a constant = 512}

\vskip 10pt

{\tt LOOP}

\hskip 10pt {\tt SET Pin1 = (GAIN * (SP - Pin0)) + BIAS}

{\tt ENDLOOP}
}
\hskip 3pt}%
\vskip 5pt \hrule}%
\vrule}

%INDEX% Control, proportional: digital electronic controller

%(END_NOTES)
