
%(BEGIN_QUESTION)
% Copyright 2011, Tony R. Kuphaldt, released under the Creative Commons Attribution License (v 1.0)
% This means you may do almost anything with this work of mine, so long as you give me proper credit

\vbox{\hrule \hbox{\strut \vrule{} {\bf Desktop Process exercise} \vrule} \hrule}

An important feature in any process controller is something called {\it output tracking}.  This feature eases the transition from ``Auto'' mode to ``Manual'' mode.  Using a Desktop Process, demonstrate this feature in action.

\vskip 10pt

Another important feature in any process controller is something called {\it setpoint tracking}.  This feature eases the transition from ``Manual'' mode to ``Auto'' mode.  Using a Desktop Process, demonstrate this feature in action.

\underbar{file i01490}
%(END_QUESTION)





%(BEGIN_ANSWER)

When a controller is in the automatic mode, output tracking means the manual output value follows along (``tracks'') the automatic output value so that when the controller is switched to manual mode, the transition will be bumpless.

\vskip 10pt

When a controller is in the manual mode, setpoint tracking means the setpoint value follows along (``tracks'') the process variable value so that when the controller is switched to automatic mode, the setpoint will begin at the same value as the process variable, and control starts with no error.

\vskip 10pt

In other words, setpoint tracking means the controller assumes the process is where you want it to be at the moment you switch to automatic mode.

%(END_ANSWER)





%(BEGIN_NOTES)


%INDEX% Control: setpoint tracking
%INDEX% Control: output tracking

%(END_NOTES)


