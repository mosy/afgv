
%(BEGIN_QUESTION)
% Copyright 2006, Tony R. Kuphaldt, released under the Creative Commons Attribution License (v 1.0)
% This means you may do almost anything with this work of mine, so long as you give me proper credit

A bored child is traveling in a car with his parents, and decides to pass the time by writing mileage values displayed by the odometer at different times, and then noting those times next to the distances:

% No blank lines allowed between lines of an \halign structure!
% I use comments (%) instead, so that TeX doesn't choke.

$$\vbox{\offinterlineskip
\halign{\strut
\vrule \quad\hfil # \ \hfil & 
\vrule \quad\hfil # \ \hfil \vrule \cr
\noalign{\hrule}
%
% First row
Odometer reading & Time \cr
% 
(miles) & (hour:minute) \cr
%
\noalign{\hrule}
%
% Another row
60,344.1 & 2:14 \cr
%
\noalign{\hrule}
%
% Another row
60,346.3 & 2:17 \cr
%
\noalign{\hrule}
%
% Another row
60,347.1 & 2:18 \cr
%
\noalign{\hrule}
%
% Another row
60,351.7 & 2:25 \cr
%
\noalign{\hrule}
%
% Another row
60,353.9 & 2:27 \cr
%
\noalign{\hrule}
%
% Another row
60,357.4 & 2:30 \cr
%
\noalign{\hrule}
%
% Another row
60,359.5 & 2:35 \cr
%
\noalign{\hrule}
} % End of \halign 
}$$ % End of \vbox

Calculate the average speed of the car between the following times:

\begin{itemize}
\item{} Between 2:17 and 2:18, average speed = \underbar{\hskip 50pt} MPH
\vskip 5pt
\item{} Between 2:18 and 2:25, average speed = \underbar{\hskip 50pt} MPH
\vskip 5pt
\item{} Between 2:25 and 2:27, average speed = \underbar{\hskip 50pt} MPH
\vskip 10pt
\item{} Between 2:17 and 2:27, average speed = \underbar{\hskip 50pt} MPH
\end{itemize}

Then, compare the average speeds taken in the first three intervals with the average speed over the sum of those intervals (2:17 to 2:27).  What does this tell us about the calculation of speed based on distance and time measurements?

\vskip 20pt \vbox{\hrule \hbox{\strut \vrule{} {\bf Suggestions for Socratic discussion} \vrule} \hrule}

\begin{itemize}
\item{} In parts of the country with {\it toll booths}, you can get a speeding ticket automatically issued to you based on the time it took you to drive from one toll station to another.  Explain how this works, and if there is any way to ``beat the system'' (i.e. speed without getting a time-based speeding ticket).
\item{} Identify the arithmetic operations needed to compute rates of change, such as speed.
\item{} This sort of repetitive calculation lends itself well to a programmable calculator, or to a {\it spreadsheet} program running on a personal computer.  If you have some familiarity with spreadsheets, try building one to calculate average speeds given this table of distance values!
\end{itemize}

\underbar{file i01502}
%(END_QUESTION)





%(BEGIN_ANSWER)

\begin{itemize}
\item{} Between 2:17 and 2:18, average speed = \underbar{\bf 48.0} MPH
\vskip 5pt
\item{} Between 2:18 and 2:25, average speed = \underbar{\bf 39.4} MPH
\vskip 5pt
\item{} Between 2:25 and 2:27, average speed = \underbar{\bf 66.0} MPH
\vskip 10pt
\item{} Between 2:17 and 2:27, average speed = \underbar{\bf 45.6} MPH
\end{itemize}

%(END_ANSWER)





%(BEGIN_NOTES)

To calculate each average speed, we must divide the difference in distance ($\Delta x$) by the corresponding difference in time ($\Delta t$):

$$\overline{v} = {\Delta x \over \Delta t}$$

\vskip 10pt

The lesson here is that we get more detailed information about speed if we take distance (and time) measurements {\it more often}.  Taking measurements once in a great while means we only see an {\it average} over time, and we lose the ``fine'' view of the car's speed as we travel on different roads.  The ultimate realization of this is to reduce the time between sampling intervals until we are {\it continuously} measuring differentials in distance ($dx$) and differentials in time ($dt$) to arrive at true speed calculations:

$$v = {dx \over dt}$$

%INDEX% Mathematics, calculus: derivative (calculating velocities from measured distances at specific times)

%(END_NOTES)


