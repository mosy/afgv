
%(BEGIN_QUESTION)
% Copyright 2006, Tony R. Kuphaldt, released under the Creative Commons Attribution License (v 1.0)
% This means you may do almost anything with this work of mine, so long as you give me proper credit

A bored operator is filling a large tank with water from different sources, the flow rates from those sources being variable over time.  He decides to pass the time by writing water volume values displayed by the level indicator (calibrated in gallons) at different times, and then noting those times next to the volumes:

% No blank lines allowed between lines of an \halign structure!
% I use comments (%) instead, so that TeX doesn't choke.

$$\vbox{\offinterlineskip
\halign{\strut
\vrule \quad\hfil # \ \hfil & 
\vrule \quad\hfil # \ \hfil \vrule \cr
\noalign{\hrule}
%
% First row
Level indicator & Time \cr
% 
(gallons) & (hour:minute) \cr
%
\noalign{\hrule}
%
% Another row
120.7 & 9:17 \cr
%
\noalign{\hrule}
%
% Another row
1005.4 & 9:21 \cr
%
\noalign{\hrule}
%
% Another row
1377.8 & 9:23 \cr
%
\noalign{\hrule}
%
% Another row
2050.2 & 9:26 \cr
%
\noalign{\hrule}
%
% Another row
2944.6 & 9:30 \cr
%
\noalign{\hrule}
%
% Another row
4875.1 & 9:40 \cr
%
\noalign{\hrule}
%
% Another row
5101.8 & 9:45 \cr
%
\noalign{\hrule}
} % End of \halign 
}$$ % End of \vbox

Calculate the average flow rate of water into the tank between the following times:

\begin{itemize}
\item{} Between 9:17 and 9:21, average flow = \underbar{\hskip 50pt} GPM
\vskip 5pt
\item{} Between 9:21 and 9:23, average flow = \underbar{\hskip 50pt} GPM
\vskip 5pt
\item{} Between 9:23 and 9:26, average flow = \underbar{\hskip 50pt} GPM
\vskip 10pt
\item{} Between 9:17 and 9:26, average flow = \underbar{\hskip 50pt} GPM
\end{itemize}

Then, compare the average flow rates taken in the first three intervals with the average flow rate over the sum of those intervals (9:17 to 9:26).  What does this tell us about the calculation of water flow based on volume and time measurements?

\vskip 20pt \vbox{\hrule \hbox{\strut \vrule{} {\bf Suggestions for Socratic discussion} \vrule} \hrule}

\begin{itemize}
\item{} Identify the arithmetic operations needed to compute rates of change, such as flow.
\item{} This sort of repetitive calculation lends itself well to a programmable calculator, or to a {\it spreadsheet} program running on a personal computer.  If you have some familiarity with spreadsheets, try building one to calculate average flow rate given this table of volume values!
\end{itemize}

\underbar{file i01503}
%(END_QUESTION)





%(BEGIN_ANSWER)

\begin{itemize}
\item{} Between 9:17 and 9:21, average flow = \underbar{\bf 221.18} GPM
\vskip 5pt
\item{} Between 9:21 and 9:23, average flow = \underbar{\bf 186.2} GPM
\vskip 5pt
\item{} Between 9:23 and 9:26, average flow = \underbar{\bf 224.13} GPM
\vskip 10pt
\item{} Between 9:17 and 9:26, average flow = \underbar{\bf 214.39} GPM
\end{itemize}

%(END_ANSWER)





%(BEGIN_NOTES)

To calculate each average flow rate, we must divide the difference in volume ($\Delta V$) by the corresponding difference in time ($\Delta t$):

$$\overline{Q} = {\Delta V \over \Delta t}$$

\vskip 10pt

The lesson here is that we get more detailed information about flow if we take volume (and time) measurements {\it more often}.  Taking measurements once in a great while means we only see an {\it average} over time, and we lose the ``fine'' view of flow as it peaks and ebbs.  The ultimate realization of this is to reduce the time between sampling intervals until we are {\it continuously} measuring differentials in volume ($dV$) and differentials in time ($dt$) to arrive at true flow calculations:

$$Q = {dV \over dt}$$


%INDEX% Mathematics, calculus: derivative (calculating flow rates from measured volumes at specific times)

%(END_NOTES)


