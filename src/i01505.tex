
%(BEGIN_QUESTION)
% Copyright 2006, Tony R. Kuphaldt, released under the Creative Commons Attribution License (v 1.0)
% This means you may do almost anything with this work of mine, so long as you give me proper credit

In areas of the United States where toll booths are used to monitor passenger vehicle travel, you can get a speeding ticket if your travel time between two toll booths is too short.  Explain how this method of ticketing speeders is valid even if the exact speed of the vehicle at any specific time is unknown between toll booths.  Also, explain how it is possible to exceed the speed limit between toll booths {\it without} getting ticketed based on elapsed time between tolls.

\underbar{file i01505}
%(END_QUESTION)





%(BEGIN_ANSWER)

In order to reach the next toll booth in ``too short'' of time, the driver {\it must} have exceeded the speed limit.  However, it is possible to exceed the speed limit for relatively short periods of time and still not reach the next toll booth ``too soon.''

%(END_ANSWER)





%(BEGIN_NOTES)

The statement that ``all (toll-booth) ticketed persons are speeders'' is true, but the {\it converse} of that statement, ``all speeders will be ticketed'' is not true.  In the same vein, ``all rabbits are mammals'' is a true statement, but the converse ``all mammals are rabbits'' is not.  This is known as the {\it fallacy of illicit conversion} in logic, and it is very commonly made.

Incidentally, this same logical fallacy is made by students when they think that a measurement of zero volts between two points in a circuit means those points must be electrically common.  The statement, ``all electrically common points have zero voltage between them'' is true, but the converse of that statement, ``all points with zero voltage between them are electrically common'' is not true.  This is because electrical commonality is just one way that points in a circuit may become equipotential, not the only way.  Equipotentiality is a condition with multiple, independent causes; just as mammals are a class of creatures with multiple, distinct members.

%INDEX% Mathematics, calculus: derivative (calculating velocities from measured distances at specific times)

%(END_NOTES)


