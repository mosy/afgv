
%(BEGIN_QUESTION)
% Copyright 2006, Tony R. Kuphaldt, released under the Creative Commons Attribution License (v 1.0)
% This means you may do almost anything with this work of mine, so long as you give me proper credit

Estimate the slopes of the tangent lines touching a parabola ($y = x^2$) at $x=3$, $x=4$, and $x=5$ by using secant line approximations:

\vskip 10pt

$$\hbox{If } y = x^2 \hbox{ then,}$$

\vskip 20pt

$${{dy \over dx} \bigg|} _ {x=3} = $$

\vskip 20pt

$${{dy \over dx} \bigg|} _ {x=4} = $$

\vskip 20pt

$${{dy \over dx} \bigg|} _ {x=5} = $$

\vskip 20pt

Do you see a mathematical relationship between the value of $x$ and the slope of the line tangent to that point on the parabola?  Can you express that relationship as a function of $x$?

$${dy \over dx} = \hbox{ ???}$$

\underbar{file i01517}
%(END_QUESTION)





%(BEGIN_ANSWER)

$${{dy \over dx} \bigg|} _ {x=3} = 6$$

\vskip 20pt

$${{dy \over dx} \bigg|} _ {x=4} = 8$$

\vskip 20pt

$${{dy \over dx} \bigg|} _ {x=5} = 10$$

\vskip 20pt

$${dy \over dx} = 2x$$

%(END_ANSWER)





%(BEGIN_NOTES)



%INDEX% Mathematics, calculus: slope of a nonlinear function using secant line approximation

%(END_NOTES)


