
%(BEGIN_QUESTION)
% Copyright 2003, Tony R. Kuphaldt, released under the Creative Commons Attribution License (v 1.0)
% This means you may do almost anything with this work of mine, so long as you give me proper credit

In physics, a very common formula relating position ($x$), velocity ($v$) and constant acceleration ($a$) is as follows:

$$x = x_o + v_ot + {1 \over 2}at^2$$

\noindent
Where,

$x =$ Final position

$x_o =$ Original position

$v_o =$ Original velocity

$a =$ Acceleration

$t =$ Time

\vskip 10pt

Differentiate this equation with respect to time, showing your work at each step.  Also, explain the meaning of the derived equation (what does ${dx \over dt}$ represent in physics?).

\underbar{file i01522}
%(END_QUESTION)





%(BEGIN_ANSWER)

Showing my work, step-by-step:

$$x = x_o + v_ot + {1 \over 2}at^2$$

$${d \over dt}x = {d \over dt}x_o + {d \over dt}v_ot + {d \over dt}{1 \over 2}at^2$$

$${d \over dt}x = 0 + v_o + at$$

$${dx \over dt} = v_o + at$$

\noindent
Where,

${dx \over dt}$ represents velocity.

%(END_ANSWER)





%(BEGIN_NOTES)

The physical variables of position, velocity, and acceleration are excellent examples to use in illustrating both differentiation and integration with respect to time, because everyone has personal experience with them.

%INDEX% Mathematics, calculus: differentiating polynomial functions

%(END_NOTES)


