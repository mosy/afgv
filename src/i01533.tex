
%(BEGIN_QUESTION)
% Copyright 2011, Tony R. Kuphaldt, released under the Creative Commons Attribution License (v 1.0)
% This means you may do almost anything with this work of mine, so long as you give me proper credit

Suppose you are asked to troubleshoot a pH control system in a water treatment facility, where {\it caustic soda} (a white powdery substance with a high pH value) is added to drinking water to raise its pH value.  According to the operator, the quality of control has been steadily growing worse as time wears on.  He first shows you a trend display captured months ago with the pH value holding steady at setpoint, then another trend taken a couple of weeks ago showing some oscillation of the PV around setpoint, then today's trend showing a large oscillation of the PV around setpoint.  Every day he looks at this trend display, he says, the oscillation gets worse.

Examining and comparing the last two trends closely, you notice two things about the more recent oscillation: it is {\it higher amplitude}, and also {\it lower frequency} than the trend captured a couple of weeks ago.  Consulting a more experienced technician about this interesting development, she tells you the pH probe is probably just dirty and needs to be cleaned.

Although this advice seems odd to you, you decide to take it because it certainly cannot hurt to clean the pH probe.  Removing the probe from service, you find it is heavily coated with sludge.  After cleaning it off and recalibrating the pH transmitter, you go check the PV trend and find that indeed this fixed the problem.

\vskip 10pt

Explain {\it why} the other instrument technician's advice was correct.  How, exactly, would a {\it dirty} sensor cause the pH to oscillate?  You can understand how a contaminated sensor might suffer a zero or span error, but neither of those would cause the loop to oscillate!

\vskip 20pt \vbox{\hrule \hbox{\strut \vrule{} {\bf Suggestions for Socratic discussion} \vrule} \hrule}

\begin{itemize}
\item{} Before removing the pH probe from the process and cleaning it, what should you do to ensure the pH control system does not react to your work on the probe?  Remember, this is a {\it running} process you are working on!
\item{} Is there a need to lock-out and/or tag-out any piece of equipment when performing the work described in this scenario?  If so, what?
\item{} For those who have studied pH measurement, how exactly does one re-calibrate a pH transmitter?
\end{itemize}

\underbar{file i01533}
%(END_QUESTION)





%(BEGIN_ANSWER)

A build-up of sludge on a pH probe will tend to add {\it lag time} and possibly even {\it dead time} to its measurement.  Either of these will slow down the loop's natural cycle time (its {\it ultimate period}).

%(END_ANSWER)





%(BEGIN_NOTES)


%INDEX% Basics, control loop troubleshooting: determining effect of specified fault(s)

%(END_NOTES)

