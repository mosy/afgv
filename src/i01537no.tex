
%(BEGIN_QUESTION)
% Copyright 2006, Tony R. Kuphaldt, released under the Creative Commons Attribution License (v 1.0)
% This means you may do almost anything with this work of mine, so long as you give me proper credit

Plott responsen for en derivat-regulator (D-regulator) til følgende inngangssignaler (PV, prosessvariabel), anta en direkte virkemåte:

$$\includegraphics{i01537x01.eps}$$

\underbar{file i01537}
%(END_QUESTION)





%(BEGIN_ANSWER)

$$\includegraphics[width=10cm]{i01537x02.eps}$$

%(END_ANSWER)





%(BEGIN_NOTES)

Et viktig begrep å formidle her er at derivatvirkning (D-del) svarer på {\it endringshastigheten} til avviket (error). Siden skal-verdien (SP) er konstant, er endringshastigheten til PV den samme som endringshastigheten til avviket.

Derfor produserer derivatvirkning et utgangssignal som er proporsjonalt med hvor raskt prosessvariabelen endrer seg.

Merk at jeg ikke har gitt noen verdier for derivattidskonstanten ($\tau_d$) eller forsterkningen ($K_p$). Dette er med vilje: Jeg vil at studentene bare skal kvalitativt bestemme {\it formen} på utgangsresponsen, ikke den spesifikke høyden.

%INDEX% Control, derivative: graphing controller response

%(END_NOTES)
