
%(BEGIN_QUESTION)
% Copyright 2006, Tony R. Kuphaldt, released under the Creative Commons Attribution License (v 1.0)
% This means you may do almost anything with this work of mine, so long as you give me proper credit

Plott responsen for følgende PD-regulator (Proporsjonal + Derivat), anta direkte virkemåte, gain = 2, derivattid = 0,25 minutter, og bias = 30\%:

$$\includegraphics{i01546x01.eps}$$

\underbar{file i01546}
%(END_QUESTION)





%(BEGIN_ANSWER)

$$\includegraphics[width=10cm]{i01546x02.eps}$$

%(END_ANSWER)





%(BEGIN_NOTES)

En vanlig feil blant studentene er å glemme bias-verdien. De vil ofte plotte utgangssignalet som starter på 0\% eller 50\% i stedet for 30\%.

De som husker bias-verdien vil kanskje fortsatt gjøre feil med endringsretningen. "Skal den gå opp eller ned?" spør de seg selv. Min metode for å besvare dette spørsmålet er å se for meg at PV øker. Hvis regulatoren er direktevirkende, skal utgangen også øke. Dette betyr at PV og Output beveger seg i samme retning. Derfor, hvis PV går {\it opp}, skal Output gå {\it opp}.

Det kan være lurt å be studentene tegne P-responsen og D-responsen hver for seg, og deretter summere dem algebraisk for å få den endelige utgangsresponsen.

%INDEX% Control, proportional + derivative: graphing controller response

%(END_NOTES)
