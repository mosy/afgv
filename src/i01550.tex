
%(BEGIN_QUESTION)
% Copyright 2011, Tony R. Kuphaldt, released under the Creative Commons Attribution License (v 1.0)
% This means you may do almost anything with this work of mine, so long as you give me proper credit

Read and outline Case History \#71 (``The Amazing Problem Free Plant'') from Michael Brown's collection of control loop optimization tutorials.  Prepare to thoughtfully discuss with your instructor and classmates the concepts and examples explored in this reading, and answer the following questions:

\begin{itemize}
\item{} At the beginning of this report, Mr. Brown cites an estimate where a plant with 30\% of its automatic controls running inefficiently would incur an {\it opportunity cost} of \$350,000 annually.  Explain what this phrase ``opportunity cost'' means.
\vskip 10pt
\item{} Identify features in the graphs of Figure 1 showing {\it integral} controller action as well as {\it proportional} controller action.  Also, explain how we may determine from this graph the fact that the control valve has problems, and explain how we may determine the direction of controller action (direct or reverse) from the trends.
\vskip 10pt
\item{} Examine the trend shown in Figure 3, and identify the problem revealed in it.  Explain how we can tell this is a control valve ``saturation'' problem and not a transmitter ``saturation'' problem.
\vskip 10pt
\item{} On loop \#7 at this plant -- a level control loop -- Mr. Brown recommended the application of filtering on the PV signal to deal with the large amount of noise.  Why not just reduce the amount of proportional action in the controller and substitute with aggressive integral action instead?
\end{itemize}

\vskip 20pt \vbox{\hrule \hbox{\strut \vrule{} {\bf Suggestions for Socratic discussion} \vrule} \hrule}

\begin{itemize}
\item{} ``Opportunity cost'' is a very useful economic concept, with many practical applications in industrial as well as everyday (personal) life.  Identify some opportunity costs in your own experience.  Hint: what is/are the opportunity cost(s) of going to school full-time?
\item{} The trend shown in Figure 1 reveals quite a bit about the controller's PID tuning.  Examine the PV, SP, and Output trends and then explain what we may discern about the controller's tuning from this.  Is the tuning P, I, or D-dominant?  Can we calculate the controller's gain, reset time, or rate time?
\item{} In Mr. Brown's analysis of loop \#5, he concludes ``Either the valve was passing, or it was stroked incorrectly.''  Explain what this statement means.
\end{itemize}

\underbar{file i01550}
%(END_QUESTION)





%(BEGIN_ANSWER)


%(END_ANSWER)





%(BEGIN_NOTES)

{\it Opportunity cost} is an economic term which means the value of whatever alternative I am forsaking to do what I'm doing now.

\vskip 10pt

Figure 1: proportional action can be seen quite clearly in response to the SP change.  Integral response accounts for the ramping output while PV $>$ SP.  Just before the next SP change, you can see that the PV indeed does come to equal SP.

\vskip 10pt

Figure 3: the PV does not respond when the valve position exceeds about 80\%.  Since the PV is still within measurement range, we know this is likely not a transmitter problem, but more likely a valve saturation problem (or perhaps even a valve positioner problem).

\vskip 10pt

On loop \#7 (level control with noisy PV), aggressive proportional action plus PV filtering was found to be the best solution.  Reducing controller gain and using integral to do most of the work would have allowed the loop to operate with little or no filtering, but it would have resulted in overshooting following SP changes, as is always the case with integral action on integrating loops. 




\vskip 20pt \vbox{\hrule \hbox{\strut \vrule{} {\bf Suggestions for Socratic discussion} \vrule} \hrule}

\begin{itemize}
\item{} Why did Michael Brown suggest limiting the controller's output to 80\% on loop \#6 (Figure 3)?
\item{} Aside from the saturation evident in Figure 3, does the control valve in this loop exhibit any other significant problems?
\end{itemize}

%INDEX% Reading assignment: Michael Brown Case History #71, "The amazing problem free plant"

%(END_NOTES)


