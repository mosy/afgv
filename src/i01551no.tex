
%(BEGIN_QUESTION)
% Copyright 2006, Tony R. Kuphaldt, released under the Creative Commons Attribution License (v 1.0)
% This means you may do almost anything with this work of mine, so long as you give me proper credit

Identifiser hvilken av belgene (bellows) i denne pneumatiske regulatormekanismen som er {\it proporsjonal}-belgen og hvilken som er {\it derivat}-belgen (hastighets-belgen). Det kan være lurt å analysere mekanismens bevegelse for en trinnendring i inngangen (PV) for å bestemme hva hver belg gjør:

$$\includegraphics{i01551x01.eps}$$

Forklar også hvordan du ville justert (tunet) forsterkning og derivattid for denne mekanismen.

\underbar{file i01551}
%(END_QUESTION)





%(BEGIN_ANSWER)

Belgen til venstre er {\it proporsjonal}-belgen, og belgen til høyre er {\it derivat}-belgen.

\vskip 10pt

For å justere forsterkningen (gain), flytt vippeppunktet (fulcrum) mellom de to kraftbjelkene. For å justere derivattiden, endre nåleventilens innstilling.

%(END_ANSWER)





%(BEGIN_NOTES)

Dette er en interessant øvelse i å analysere kraftbalanse-mekanismer. Hvis inngangstrykket plutselig øker ("Step input"), presser inngangsbelgen opp på venstre ende av kraftbjelken. Dette fører til at flapper-ventilen nærmer seg dysen, noe som bygger opp utgangstrykk. Utgangstrykket går umiddelbart til belgen til venstre, som presser ned på kraftbjelken: dette er {\it negativ tilbakekobling}. På samme tid (umiddelbart etter trinnet), har ikke nåleventilen tillatt nok luftstrøm til å fylle belgen til høyre. Dette betyr at belgen til venstre umiddelbart motsetter seg bevegelsen fra inngangsbelgen, og begrenser utgangstrykkets økning (proporsjonal respons).

Men etter hvert som tiden går, fylles belgen til høyre med lufttrykk fra utgangslinjen. Siden denne belgen presser {\it opp} på bjelken, hjelper den inngangsbelgen med å presse flapperen nærmere dysen. Dette fører til at utgangstrykket stiger til en høyere verdi enn det hadde rett etter trinnet. Med andre ord, den forsinkede handlingen fra belgen til høyre "legger til" (øker) forsterkningen til mekanismen over tid. Det motsatte er også sant: forsterkningen er minst til å begynne med. "Mindre forsterkning til å begynne med" betyr samme effekt som å ha "mer forsterkning" i begynnelsen for så å falle tilbake til en lavere forsterkning etter hvert. Dette er definisjonen på derivat-virkning.

Derfor er belgen til venstre proporsjonal-belgen, og belgen til høyre er derivat-belgen.

%INDEX% Control, proportional + derivative: pneumatic force-balance controller

%(END_NOTES)
