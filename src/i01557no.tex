
%(BEGIN_QUESTION)
% Copyright 2006, Tony R. Kuphaldt, released under the Creative Commons Attribution License (v 1.0)
% This means you may do almost anything with this work of mine, so long as you give me proper credit

Skriv et pseudokode-program for en mikrokontroller som beregner den omtrentlige endringshastigheten (derivert) til et analogt inngangssignal ($PV$) i enheter av "volt per minutt".

Anta at programmet skannes (kjøres) 15 ganger per sekund.

\underbar{file i01557}
%(END_QUESTION)





%(BEGIN_ANSWER)

Dette er bare ett eksempel, og ikke det eneste korrekte svaret:

\vskip 10pt

\hbox{ \vrule
\vbox{ \hrule \vskip 3pt
\hbox{ \hskip 3pt
\vbox{ \hsize=4.5in \raggedright

\noindent
\underbar{\bf Pseudocode listing}

{\tt Declare PV as variable}

{\tt Declare Last\_PV as variable}

{\tt Declare Rate as variable}

\vskip 10pt

{\tt LOOP}

\hskip 10pt {\tt SET PV = Analog Input}

\hskip 10pt {\tt SET Rate = (PV - Last\_PV) * 900}

\hskip 10pt {\tt SET Last\_PV = PV}

{\tt ENDLOOP}
}
\hskip 3pt}%
\vskip 5pt \hrule}%
\vrule}

\vskip 10pt

Faktoren "900" kommer fra det faktum at sløyfen kjører med en hastighet på 15 ganger i sekundet (900 ganger i minuttet). Forskjellen mellom nåværende PV og forrige PV vil derfor representere spenningsendringen per $\frac{1}{900}$ del av et minutt. For å skalere dette til "volt per minutt", må vi multiplisere med 900.

%(END_ANSWER)





%(BEGIN_NOTES)

Be studentene forklare hvorfor verdien for {\tt Last\_PV} oppdateres helt på slutten av loopen, og ikke begynnelsen. Hva ville skje hvis vi oppdaterte den ved begynnelsen?

Svaret er ganske pedagogisk: hvis vi satte {\tt Last\_PV = PV} ved starten av loopen, ville de to variablene hatt samme verdi når vi beregnet {\tt Rate}. Dette ville resultere i en {\tt Rate}-verdi på 0 hver gang! For at derivasjonsfunksjonen skal fungere, må differansen tas mellom verdien av PV {\it akkurat nå} og verdien av PV {\it for en tid siden} (i forrige skanning).

%INDEX% Control, derivative: digital algorithm

%(END_NOTES)
