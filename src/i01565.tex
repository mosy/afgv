
%(BEGIN_QUESTION)
% Copyright 2011, Tony R. Kuphaldt, released under the Creative Commons Attribution License (v 1.0)
% This means you may do almost anything with this work of mine, so long as you give me proper credit

Read and outline Case History \#61 (``The Case Of The Mysterious Cycle On A Paper Machine'') from Michael Brown's collection of control loop optimization tutorials.  Prepare to thoughtfully discuss with your instructor and classmates the concepts and examples explored in this reading, and answer the following questions:

\begin{itemize}
\item{} Describe the ``blend chest'' control loops illustrated in this case history, identifying all process variables and final control elements.  Also, explain what a {\it refiner} does to wood pulp.
\vskip 10pt
\item{} One of the process variables common to wood pulp processing is {\it consistency}.  What, exactly, is ``consistency'' of wood pulp, and what is done to the wood pulp in order to influence its consistency?
\vskip 10pt
\item{} In this report, Mr. Brown claims the actual liquid level control in the blend chest was not a critical parameter, and could vary significantly with no ill effect to the process.  Explain why this is useful knowledge when optimizing control loops, especially in this particular case with wood pulp blending.
\vskip 10pt
\item{} Describe the diagnostic test(s) used in an attempt to locate the source of the mysterious (2-minute period) cycle.
\vskip 10pt
\item{} Identify what the source of the 2-minute cycle was determined to be, and why it was so difficult to find.
\vskip 10pt
\item{} In Figure 5 we see an open-loop test of a control valve, with non-linear installed characteristics.  Assuming this valve had a ``linear'' trim from the factory, which trim characteristic might we replace it with, {\it quick-opening} or {\it equal-percentage}?  Explain why, from the graph.
\end{itemize}


\vskip 20pt \vbox{\hrule \hbox{\strut \vrule{} {\bf Suggestions for Socratic discussion} \vrule} \hrule}

\begin{itemize}
\item{} Explain the purpose and operation of the {\it bypass flow} controller shown in the diagram of Figure 2.  What effect does this flow controller have on the level of pulp in the blend chest?
\item{} One of the pulp flows into this blend chest is called {\it broke}, which is pulp made from rejected paper product recovered from later steps in the paper-making process.  The pulp and paper industry is unusually abundant with a range of bizarre terms and labels for parts of the process.  Why do you suppose this particular term is used to refer to re-pulped paper?
\item{} The inconsistent process gains observed in the open-loop test of Figure 5 might very well be caused by the wrong characteristic of valve, but there is another possible cause as well.  Assuming a valve with truly linear (installed) behavior, explain why this same gain problem could have been caused by (wrongly) having square-root extraction programmed into the DP flow transmitter {\it and} the flow controller's input.
\item{} Why would long filter time constants programmed into some of the magnetic flowmeters at this pulp mill pose a problem for good control?
\item{} Explain why there is no refiner on the ``broke'' pulp line, as there are on both of the other pulp lines.  Bear in mind that the purpose of a refiner is to break long wood fibers into shorter wood fibers.
\end{itemize}

\underbar{file i01565}
%(END_QUESTION)





%(BEGIN_ANSWER)


%(END_ANSWER)





%(BEGIN_NOTES)

Blend chest: three slave flow controllers to one master level controller.  A motor-driven machine called a ``refiner'' grinds coarse wood fibers into smaller fibers.  

\vskip 10pt

``Consistency'' is a measure of how much wood fiber is in the pulp.  It is controlled by dilution of the pulp with water.

\vskip 10pt

Level control in the blend chest is actually not critical -- a level controller is only there to prevent chest overflow or running the chest empty.  Because of this, the level controller was intentionally ``detuned'' to keep flows fairly steady, steady flow being an important factor for the operation of the refiners feeding the blend chest.

\vskip 10pt

Slow oscillation of level (even when LIC was in manual) indicated a cycling load somewhere.  Oscillation continued even after three other controllers (consistency, tickler flow, machine chest level) were put into manual mode!

\vskip 10pt

When the cycling was found to continue even with all the controllers placed into manual mode, the machine chest level control valve was discovered to be moving slowly back and forth even with a constant signal sent to it!!!  Another unstable valve positioner was discovered on one of the blend chest flow control valves (open-loop test shown in Figure 5)!

\vskip 10pt

Figure 5: valve with nonlinear characteristic.  Large gains at first, small gains toward 100\% open.  If this valve was linear to begin with, we should replace it with an {\it equal-percentage} trim.

\vskip 10pt

As a side note, some of the magnetic flowmeters had filter times in the order of 1 minute (!!!).








\vfil \eject

\noindent
{\bf Prep Quiz:}

A ``blend chest'' in a paper-making process is where:

\begin{itemize}
\item{} Water is added to wood pulp to dilute and ``thin'' it down
\vskip 5pt 
\item{} Strong caustic is added to wood fibers to ``pre-digest'' them into pulp
\vskip 5pt 
\item{} Wood chips are cooked for hours to fully convert them into pulp
\vskip 5pt 
\item{} Pulp is heated to the proper temperature for paper-making
\vskip 5pt 
\item{} Different types of wood pulp are mixed together in proper proportion
\vskip 5pt 
\item{} Measure the flow rate of wood pulp prior to conversion into paper
\end{itemize}


%INDEX% Reading assignment: Michael Brown Case History #61, "The case of the mysterious cycle on a paper machine"

%(END_NOTES)


