
%(BEGIN_QUESTION)
% Copyright 2010, Tony R. Kuphaldt, released under the Creative Commons Attribution License (v 1.0)
% This means you may do almost anything with this work of mine, so long as you give me proper credit

You are part of a team building a rocket to carry research instruments into the high atmosphere.  One of the variables needed by the on-board flight-control computer is velocity, so it can throttle engine power and achieve maximum fuel efficiency.  The problem is, none of the electronic sensors on board the rocket has the ability to directly measure velocity.  What is available is an {\it altimeter}, which infers the rocket's altitude (its position up from ground in {\it meters}) by measuring ambient air pressure; and also an {\it accelerometer}, which infers acceleration (rate-of-change of velocity in {\it meters per second squared}) by measuring the inertial force exerted by a small mass.

The lack of a ``speedometer'' for the rocket may have been an engineering design oversight, but it is still your responsibility as a development technician to figure out a workable solution to the dilemma.  How do you propose we obtain the electronic velocity measurement the rocket's flight-control computer needs?

\vskip 20pt \vbox{\hrule \hbox{\strut \vrule{} {\bf Suggestions for Socratic discussion} \vrule} \hrule}

\begin{itemize}
\item{} Suppose a ground-based radar is used to continuously track the rocket's altitude.  Explain how data taken from this radar could be used to calculate rocket velocity.
\item{} Explain how {\it units of measurement} are especially in determining when to apply differentiation, versus when to apply integration.
\end{itemize}

\underbar{file i01568}
%(END_QUESTION)





%(BEGIN_ANSWER)

One possible solution is to use an electronic {\it integrator} circuit to derive a velocity measurement from the accelerometer's signal.  However, this is not the only possible solution!  Another solution would be to differentiate the altimeter signal.  Differentiation assumes straight-line, vertical travel for the rocket.

%(END_ANSWER)





%(BEGIN_NOTES)


%INDEX% Mathematics, calculus: derivative vs integral (applied to rocket motion)

%(END_NOTES)


