
%(BEGIN_QUESTION)
% Copyright 2006, Tony R. Kuphaldt, released under the Creative Commons Attribution License (v 1.0)
% This means you may do almost anything with this work of mine, so long as you give me proper credit

A bored child is traveling in a car with his parents, and decides to pass the time by writing speed values displayed by the speedometer at different times, and then noting those times next to the distances:

% No blank lines allowed between lines of an \halign structure!
% I use comments (%) instead, so that TeX doesn't choke.

$$\vbox{\offinterlineskip
\halign{\strut
\vrule \quad\hfil # \ \hfil & 
\vrule \quad\hfil # \ \hfil \vrule \cr
\noalign{\hrule}
%
% First row
Speedometer reading & Time \cr
% 
(miles/hour) & (hour:minute) \cr
%
\noalign{\hrule}
%
% Another row
55 & 2:14 \cr
%
\noalign{\hrule}
%
% Another row
57 & 2:17 \cr
%
\noalign{\hrule}
%
% Another row
60 & 2:18 \cr
%
\noalign{\hrule}
%
% Another row
61 & 2:25 \cr
%
\noalign{\hrule}
%
% Another row
58 & 2:27 \cr
%
\noalign{\hrule}
%
% Another row
55 & 2:30 \cr
%
\noalign{\hrule}
%
% Another row
60 & 2:35 \cr
%
\noalign{\hrule}
} % End of \halign 
}$$ % End of \vbox

Describe how you may calculate the distance traveled by this car between 2:14 and 2:35 based on speed values from this table.

\vskip 20pt \vbox{\hrule \hbox{\strut \vrule{} {\bf Suggestions for Socratic discussion} \vrule} \hrule}

\begin{itemize}
\item{} This sort of repetitive calculation lends itself well to a programmable calculator, or to a {\it spreadsheet} program running on a personal computer.  If you have some familiarity with spreadsheets, try building one to calculate distance traveled given this table of speeds!
\item{} Identify more than one way to calculate distance traveled from the speed and time values given.
\end{itemize}


\underbar{file i01570}
%(END_QUESTION)





%(BEGIN_ANSWER)

I will let you explain the procedure for doing this!  The approximate distance traveled by this car between 2:14 and 2:35 is somewhere between 20.22 miles and 20.65 miles, depending on your procedure.

%(END_ANSWER)





%(BEGIN_NOTES)

The two different answers given reflect two different methods for calculating Riemann sums, the first answer using {\it left endpoints} and the second answer using {\it right endpoints}.

%INDEX% Mathematics, calculus: integral (calculating distances from measured velocities at specific times)

%(END_NOTES)


