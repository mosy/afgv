
%(BEGIN_QUESTION)
% Copyright 2006, Tony R. Kuphaldt, released under the Creative Commons Attribution License (v 1.0)
% This means you may do almost anything with this work of mine, so long as you give me proper credit

Et uungåelig kjennetegn ved P-regulering (kun proporsjonal) er noe som kalles {\it stasjonært avvik} (proportional-only offset). Definer hva "stasjonært avvik" er, og forklar hvorfor en P-regulator ikke klarer å eliminere det.

\vfil

\underbar{file i01584}
\eject
%(END_QUESTION)





%(BEGIN_ANSWER)

"Stasjonært avvik" er forskjellen mellom settpunkt (SP) og prosessvariabel (PV) som oppstår når prosessbelastningen endrer seg fra verdien som bias-innstillingen var basert på. P-regulatorer "trenger" avvik for å generere et annet utgangssignal enn bias-verdien (vanligvis 50\%), og derfor vil de alltid ha et visst avvik (feil) når prosessforholdene krever et pådrag som er forskjellig fra bias-verdien.

%(END_ANSWER)





%(BEGIN_NOTES)

Mange lærebøker forklarer {\it stasjonært avvik} (offset) ved hjelp av eksempler, f.eks. en væskenivå-regulator der en endring i ventilstilling er nødvendig for å opprettholde nivået under en ny strømningsrate, men hvor regulatoren (som kun er proporsjonal) ikke kan flytte ventilen til en ny posisjon uten at nivået endres for å drive den dit. Dette er en god måte å forklare det på, men vær sikker på at studentene også forstår svaret som er gitt her. Rent matematisk genererer en P-regulator utgangssignalet kun basert på to faktorer: feil (avvik) og bias. Hvis utgangssignalet skal være noe annet enn bias-verdien, {\it må} det være en feil tilstede!

%INDEX% Control, proportional: proportional-only offset

%(END_NOTES)
