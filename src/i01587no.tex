
%(BEGIN_QUESTION)
% Copyright 2006, Tony R. Kuphaldt, released under the Creative Commons Attribution License (v 1.0)
% This means you may do almost anything with this work of mine, so long as you give me proper credit

Integralvirkning (også kalt "reset") er en svært nyttig funksjon i prosessregulatorer, ettersom den eliminerer stasjonært avvik (offset). Den har imidlertid en ulempe: det kalles {\it integrasjonsmetning} (integral windup, eller "reset windup").

Forklar hva integrasjonsmetning er, og gi et eksempel på en prosessreguleringstilstand som vil forårsake det.

Forklar også hvordan en menneskelig operatør ville måtte reagere på en "windup"-hendelse. Forestill deg at du er operatør, og en av dine PID-regulatorer har gått i metning (wound up). Hva må du gjøre for å fikse problemet?

\vfil

\underbar{file i01587}
\eject
%(END_QUESTION)





%(BEGIN_ANSWER)

"Integrasjonsmetning" (Windup) er en tilstand der integraldelen i en regulator fortsetter å øke (eller redusere) utgangssignalet i et forgjeves forsøk på å eliminere avviket, selv om utgangen allerede har nådd sin metningsgrense (100\% eller 0\%).

For å komme seg ut av en metningstilstand, må operatøren vanligvis bytte regulatoren til {\it manuell modus} og tvinge utgangen til en verdi som bringer prosessen tilbake til settpunktet.

%(END_ANSWER)





%(BEGIN_NOTES)

Integrasjonsmetning er vanskelig å forstå uten å faktisk ha opplevd det. Hvis du har utstyr for prosess-simulering, eller til og med en enkel op-amp krets som demonstrerer integralvirkning, la studentene eksperimentere med windup/metning, så de kan se det selv. Det kan være en skikkelig øyeåpner!

%INDEX% Control, integral: human operator perspective

%(END_NOTES)
