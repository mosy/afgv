
%(BEGIN_QUESTION)
% Copyright 2006, Tony R. Kuphaldt, released under the Creative Commons Attribution License (v 1.0)
% This means you may do almost anything with this work of mine, so long as you give me proper credit

Integral control action is usually combined with proportional control action in a process controller.  Contrast these two modes of control in the simplest terms possible.

\vskip 20pt \vbox{\hrule \hbox{\strut \vrule{} {\bf Suggestions for Socratic discussion} \vrule} \hrule}

\begin{itemize}
\item{} A good practice when learning new concepts is to imagine yourself having to explain those concepts to an intelligent child, forcing yourself to express the new ideas in the simplest and clearest terms possible.  Explain why this is a good practice to cultivate, especially after you are out of school and in the workplace!
\end{itemize}

\underbar{file i01591}
%(END_QUESTION)





%(BEGIN_ANSWER)

It would be a shame for me to provide a ready-made answer for you!  Do your best to cast these control modes into your own words, and you will understand them deeper than before.

%(END_ANSWER)





%(BEGIN_NOTES)

{\it Proportional control} generates an output signal proportional to the algebraic difference between PV and SP (the {\it error}), while {\it integral control} generates an output signal proportional to the error's accumulation over time.

Proportional control only pays attention to the amount of error between PV and SP.  Integral pays attention to {\it how long} error has existed between PV and SP, as well as the magnitude of the error.

\vskip 10pt

{\bf Proportional control action is where the amount of error tells the output how \underbar{far} to go.}

\vskip 10pt

{\bf Integral control action is where the amount of error tells the output how \underbar{fast} to go.}

\vskip 10pt

{\bf Derivative control action is where \underbar{speed} of the error tells the output how \underbar{far} to go.}

%INDEX% Control, proportional versus integral: comparison between the two control modes

%(END_NOTES)


