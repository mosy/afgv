
%(BEGIN_QUESTION)
% Copyright 2007, Tony R. Kuphaldt, released under the Creative Commons Attribution License (v 1.0)
% This means you may do almost anything with this work of mine, so long as you give me proper credit

Følgende trendlogg viser en PI-regulators respons på en trinnvis endring i avviket (error). Anta en forsterkning på 1 og at regulatoren er reversvirkende (for en økning i PV, vil utgangen synke):

$$\includegraphics[width=15.5cm]{i01595x01.eps}$$

Tegn den responsen du ville forventet å se fra en {\it P-regulator} (kun proporsjonal) og en {\it I-regulator} (kun integral) hver for seg, gitt det samme inngangssignalet. Kombiner disse to responsene grafisk for å bevise at PI-responsen faktisk er summen av P- og I-virkning.

Fra P- og I-grafene dine, estimer integraltidskonstanten ($\tau_i$, i minutter per repetisjon) for denne regulatoren.

\underbar{file i01595}
%(END_QUESTION)





%(BEGIN_ANSWER)

$$\includegraphics[width=15.5cm]{i01595x02.eps}$$

\vskip 10pt

Integraltidskonstanten ($\tau_i$) er ca. 1,6 minutter per repetisjon.

%(END_ANSWER)





%(BEGIN_NOTES)

{\bf Proporsjonalvirkning er hvor \underbar{størrelsen} på avviket forteller utgangen hvor \underbar{langt} den skal gå.}

\vskip 10pt

{\bf Integralvirkning er hvor \underbar{størrelsen} på avviket forteller utgangen hvor \underbar{raskt} den skal gå.}

\vskip 10pt

{\bf Derivatvirkning er hvor \underbar{hastigheten} på avviket forteller utgangen hvor \underbar{langt} den skal gå.}

%INDEX% Control, proportional + integral: graphing controller response

%(END_NOTES)
