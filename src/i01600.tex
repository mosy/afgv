
%(BEGIN_QUESTION)
% Copyright 2010, Tony R. Kuphaldt, released under the Creative Commons Attribution License (v 1.0)
% This means you may do almost anything with this work of mine, so long as you give me proper credit

Research the PID equations available for use in the Rockwell (Allen-Bradley) Logix5000 PLC control system.  You will find the ``General Instructions'' programming reference manual (publication 1756-RM0031-EN-P) to be most useful for this purpose.

\vskip 10pt

How many different PID equations are available to use in the PID instruction?  How do their operations differ from one another?

\vskip 20pt \vbox{\hrule \hbox{\strut \vrule{} {\bf Suggestions for Socratic discussion} \vrule} \hrule}

\begin{itemize}
\item{} Suppose we were using a Logix5000 PLC to control a process with a large first-order lag time, and we needed fast response to setpoint changes.  Would you suggest using the PID equation(s) where derivative is calculated on error, or using the PID equation(s) where derivative is calculated on PV only?
\item{} Examine the structured text routine shown on page 513 of this manual, showing how to set up a 1000 ms timer to execute the PID instruction once every second.  Identify variable assignments, function calls, and conditional statements in this snippet of ST code.  Compare and contrast the structured text example against the ladder-logic programming example of the same algorithm.
\item{} Examine the structured text routine shown on page 515 of this manual, showing how to set up a PID instruction to be executed as soon as the analog input card completes a scan.  Identify variable assignments, function calls, and conditional statements in this snippet of ST code.  Compare and contrast the structured text example against the ladder-logic programming example of the same algorithm.
\item{} Can you spot the typographical error(s) in the equation section of this manual, comparing different equation options for the PID instruction?
\end{itemize}

\underbar{file i01600}
%(END_QUESTION)





%(BEGIN_ANSWER)


%(END_ANSWER)





%(BEGIN_NOTES)

Page 509 lists the different PID equations available, which include ``Dependent Gains'' (Ideal or ISA) and ``Independent Gains'' (parallel).  In each of these two variations, the option exists to calculate derivative action based on the rate-of-change of error (E) or process variable (PV).

%INDEX% Control, PID: different equations used by Rockwell Logix5000 PLCs

%(END_NOTES)


