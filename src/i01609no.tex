
%(BEGIN_QUESTION)
% Copyright 2007, Tony R. Kuphaldt, released under the Creative Commons Attribution License (v 1.0)
% This means you may do almost anything with this work of mine, so long as you give me proper credit

En PI-regulator (P+I) brukes til å kontrollere strømmen av en væske gjennom et rør. Strømningskontrollventilen er imidlertid underdimensjonert, og vil ikke tillate strømmen å oppnå en verdi som er større enn 70\% av måleområdet. Hva vil regulatoren gjøre hvis den får et settpunkt på 80\%? Vær så spesifikk du kan i svaret ditt.

\vskip 10pt

Hva vil skje hvis settpunktet returneres til en oppnåelig verdi som 50\% etter at det har stått på 80\% i lang tid?

\vskip 20pt \vbox{\hrule \hbox{\strut \vrule{} {\bf Forslag til sokratisk diskusjon} \vrule} \hrule}

\begin{itemize}
\item{} Hvordan antar du at et problem som dette manifesterer seg på en prosesstrend (logging)? Finnes det noen avslørende ledetråder på en trend som indikerer at et slikt problem eksisterer?
\end{itemize}

\underbar{file i01609}
%(END_QUESTION)





%(BEGIN_ANSWER)

Siden strømmen ikke kan overstige 70\% på grunn av den underdimensjonerte ventilen, vil regulatoren fortsette å "se" et avvik på 10\%, og regulatorutgangen vil til slutt gå i metning på 100\% (helt åpen ventil). Algoritmens integral-ledd vil fortsette å øke (med mindre det begrenses av en spesiell funksjon i regulatoren designet for å stoppe integrasjonsprosessen hvis og når utgangssignalet når visse grenser), selv om det ikke vil gjøre noen nytte fordi ventilen er mettet helt åpen. Dette fenomenet kalles {\it integrasjonsmetning} (integral windup) eller {\it reset windup}.

Når settpunktet reduseres til 50\%, kan regulatorutgangen forbli mettet på 100\% (ventil helt åpen) i stedet for umiddelbart å avta for å redusere strømmen gjennom røret slik den skal, på grunn av den akkumulerte verdien ("windup") i integral-leddet.

\vskip 10pt

Det bør bemerkes at de fleste moderne (digitale) sløyferegulatorer har grenser for anti-integrasjonsmetning (anti-reset windup) satt som standard til 100\% og 0\%, slik at integral-akkumulatoren ikke vil "vinde" forbi disse grensene. Jeg har imidlertid støtt på regulatorer uten denne funksjonen, hvor regulatorens interne akkumulator er i stand til å gå langt forbi 100\% selv om kontrollventilen selvfølgelig ikke kan åpne mer enn 100\%.

%(END_ANSWER)





%(BEGIN_NOTES)

Andre forhold kan også forårsake metning (windup). En feilet transmitter, en batch-prosess som stenger ned, eller andre diskontinuiteter i prosessen vil føre til at en regulator med integralvirkning "vinder opp" eller "vinder ned" i et forgjeves forsøk på å minimere avviket.

Når settpunktet reduseres til 50\%, kan regulatorutgangen forbli mettet på 100\% (ventil helt åpen) i stedet for umiddelbart å avta for å redusere strømmen gjennom røret slik den skal, på grunn av den akkumulerte verdien ("windup") i integral-leddet. Med andre ord, den akkumulerte verdien av integral-leddet under "windup" da den forgjeves prøvde å øke strømningsraten til 80\%, er fortsatt tilstede når settpunktet synker til 50\%, og den kan være tilstrekkelig til å holde regulatorutgangen på 100\% i lang tid (inntil integral-leddet "vinder ned" igjen langt nok). Dette resulterer selvfølgelig i en forsinket regulatorrespons på det nye settpunktet.

%INDEX% Control, integral: windup

%(END_NOTES)
