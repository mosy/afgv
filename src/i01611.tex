
%(BEGIN_QUESTION)
% Copyright 2007, Tony R. Kuphaldt, released under the Creative Commons Attribution License (v 1.0)
% This means you may do almost anything with this work of mine, so long as you give me proper credit

Some instrument technicians find it helpful to think of the integral tuning constant ($\tau_i$) as the measure of {\it impatience} within a controller.  What is meant by this analogy?  On the same line of thought, is aggressive integral action more desirable for controlling inherently fast-responding processes, or inherently slow-responding processes?

\underbar{file i01611}
%(END_QUESTION)





%(BEGIN_ANSWER)

{\it Impatience} is an apt analogy for integral action, because it relates the controller's output to the amount of {\it time} an error is present between PV and SP.  Correspondingly, aggressive integral action may be appropriate for controlling {\it fast-responding} processes, but not for controlling slow-responding processes.

%(END_ANSWER)





%(BEGIN_NOTES)

This is not to say that integral action cannot be used in slow-responding processes.  To the contrary, judicious application of integral control is the only way to automatically eliminate proportional-only offset.  It's just that slow processes need {\it modest} amounts of integral action, not too much.

%INDEX% Control, integral: analogous to ``impatience''

%(END_NOTES)


