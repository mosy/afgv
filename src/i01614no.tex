
%(BEGIN_QUESTION)
% Copyright 2006, Tony R. Kuphaldt, released under the Creative Commons Attribution License (v 1.0)
% This means you may do almost anything with this work of mine, so long as you give me proper credit

En pneumatisk, fjær-retur aktuator på en reguleringsventil fungerer som en grov proporsjonalregulator. Forklar hvorfor. Hva er prosessvariabelen (PV), settpunktet (SP) og utgangen (Output) for denne "regulatoren"?

\vfil

\underbar{file i01614}
\eject
%(END_QUESTION)





%(BEGIN_ANSWER)

Aktuatoren balanserer kraften fra lufttrykket (PV) mot kraften fra fjæren (SP). Posisjonen til ventilstammen er utgangen (Output). Kraften fra fjæren øker proporsjonalt med kompresjonen (Hookes lov), noe som gir en proporsjonal respons.

%(END_ANSWER)





%(BEGIN_NOTES)

Dette er en konseptuell oppgave som hjelper studentene å se proporsjonal kontroll i en mekanisk sammenheng. Det illustrerer også hvorfor vi ofte trenger en ventilposisjoner: for å overvinne friksjon og gjøre posisjoneringen mer nøyaktig (øke sløyfeforsterkningen lokalt rundt ventilen).

%INDEX% Control, proportional: control valve as crude proportional pressure controller

%(END_NOTES)
