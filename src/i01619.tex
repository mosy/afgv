
%(BEGIN_QUESTION)
% Copyright 2015, Tony R. Kuphaldt, released under the Creative Commons Attribution License (v 1.0)
% This means you may do almost anything with this work of mine, so long as you give me proper credit

Read the ``LabVIEW exercise \#3'' tutorial document in preparation for implementing it in the computer lab, in conjunction with a USB-based data acquisition unit.  Here, you will learn how to configure a digital output function on a DAQ so that your VI has the ability to control a real-world device (in this case, an LED).

\vskip 10pt

Note: the tutorial you need to do this exercise is found on your ``Instrumentation Reference'' (a set of digital document files) your instructor has prepared for you.

\underbar{file i01619}
%(END_QUESTION)





%(BEGIN_ANSWER)


%(END_ANSWER)





%(BEGIN_NOTES)

In this exercise we configure LabVIEW to control the status of an LED connected to one of the discrete output channels of the DAQ, as a high-level indicator.












\vfil \eject

\noindent
{\bf Summary Quiz:}

(Completion of this exercise makes a good summary quiz.)

%INDEX% Reading assignment: LabVIEW exercise #3

%(END_NOTES)

