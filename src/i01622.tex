
%(BEGIN_QUESTION)
% Copyright 2011, Tony R. Kuphaldt, released under the Creative Commons Attribution License (v 1.0)
% This means you may do almost anything with this work of mine, so long as you give me proper credit

A digital temperature transmitter has a calibrated input range of 0 to 200 degrees Celsius, and a 10-bit output (0 to 1023 ``count'' range).  Complete the following table of values for this transmitter, assuming perfect calibration (no error):

% No blank lines allowed between lines of an \halign structure!
% I use comments (%) instead, so that TeX doesn't choke.

$$\vbox{\offinterlineskip
\halign{\strut
\vrule \quad\hfil # \ \hfil & 
\vrule \quad\hfil # \ \hfil & 
\vrule \quad\hfil # \ \hfil & 
\vrule \quad\hfil # \ \hfil \vrule \cr
\noalign{\hrule}
%
% First row
Input temp & Percent of span & Counts & Counts \cr
%
% Another row
($^{o}$C) & (\%) & (decimal) & (hexadecimal) \cr
%
\noalign{\hrule}
%
% Another row
 & 15 &  &  \cr
%
\noalign{\hrule}
%
% Another row
 & 78 &  &  \cr
%
\noalign{\hrule}
} % End of \halign 
}$$ % End of \vbox

\underbar{file i01622}
%(END_QUESTION)





%(BEGIN_ANSWER)

Full credit is given for having either of the alternative answers in each cell of the table (i.e. the student does not have to specify {\it both} count values shown in each cell!):

% No blank lines allowed between lines of an \halign structure!
% I use comments (%) instead, so that TeX doesn't choke.

$$\vbox{\offinterlineskip
\halign{\strut
\vrule \quad\hfil # \ \hfil & 
\vrule \quad\hfil # \ \hfil & 
\vrule \quad\hfil # \ \hfil & 
\vrule \quad\hfil # \ \hfil \vrule \cr
\noalign{\hrule}
%
% First row
Input temp & Percent of span & Counts & Counts \cr
%
% Another row
($^{o}$C) & (\%) & (decimal) & (hexadecimal) \cr
%
\noalign{\hrule}
%
% Another row
30 & 15 & 153 or 154 & 99 or 9A \cr
%
\noalign{\hrule}
%
% Another row
156 & 78 & 797 or 798 & 31D or 31E \cr
%
\noalign{\hrule}
} % End of \halign 
}$$ % End of \vbox

%(END_ANSWER)





%(BEGIN_NOTES)

{\bf This question is intended for exams only and not worksheets!}.

%(END_NOTES)


