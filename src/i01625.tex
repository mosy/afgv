
%(BEGIN_QUESTION)
% Copyright 2007, Tony R. Kuphaldt, released under the Creative Commons Attribution License (v 1.0)
% This means you may do almost anything with this work of mine, so long as you give me proper credit

Suppose you wish to calibrate a current-to-pressure (``I/P'') transducer to an output range of 3 to 15 PSI, with an input range of 4 to 20 mA.  Complete the following calibration table showing the proper test pressures and the ideal input signal levels at those pressures:

% No blank lines allowed between lines of an \halign structure!
% I use comments (%) instead, so that TeX doesn't choke.

$$\vbox{\offinterlineskip
\halign{\strut
\vrule \quad\hfil # \ \hfil & 
\vrule \quad\hfil # \ \hfil & 
\vrule \quad\hfil # \ \hfil \vrule \cr
\noalign{\hrule}
%
% First row
Input signal & Percent of span & Output pressure \cr
%
% Another row
applied (mA) & (\%) & (PSI) \cr
%
\noalign{\hrule}
%
% Another row
 & 35 &  \cr
%
\noalign{\hrule}
%
% Another row
 & 80 &  \cr
%
\noalign{\hrule}
%
% Another row
 & 95 &  \cr
%
\noalign{\hrule}
} % End of \halign 
}$$ % End of \vbox

\underbar{file i01625}
%(END_QUESTION)





%(BEGIN_ANSWER)

% No blank lines allowed between lines of an \halign structure!
% I use comments (%) instead, so that TeX doesn't choke.

$$\vbox{\offinterlineskip
\halign{\strut
\vrule \quad\hfil # \ \hfil & 
\vrule \quad\hfil # \ \hfil & 
\vrule \quad\hfil # \ \hfil \vrule \cr
\noalign{\hrule}
%
% First row
Input signal & Percent of span & Output pressure \cr
%
% Another row
applied (mA) & (\%) & (PSI) \cr
%
\noalign{\hrule}
%
% Another row
{\bf 9.6} & 35 & {\bf 7.2} \cr
%
\noalign{\hrule}
%
% Another row
{\bf 16.8} & 80 & {\bf 12.6} \cr
%
\noalign{\hrule}
%
% Another row
{\bf 19.2} & 95 & {\bf 14.4} \cr
%
\noalign{\hrule}
} % End of \halign 
}$$ % End of \vbox

%(END_ANSWER)





%(BEGIN_NOTES)


%INDEX% Relay, I/P transducer: calibration table

%(END_NOTES)


