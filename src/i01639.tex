
%(BEGIN_QUESTION)
% Copyright 2007, Tony R. Kuphaldt, released under the Creative Commons Attribution License (v 1.0)
% This means you may do almost anything with this work of mine, so long as you give me proper credit

It is sometimes said that in a PID controller, proportional action works on the {\it present}, while integral action works on the {\it past} and derivative action works on the {\it future}.  Explain what this means.

\underbar{file i01639}
%(END_QUESTION)





%(BEGIN_ANSWER)

Proportional action is said to work on the present because its action is instantaneous and does not depend on time.  The value of the proportional term in a PID controller is strictly a function of PV, SP, and gain, without any reference to time.
 
Integral action is said to work on the past, because its action is based on the amount of error (PV $-$ SP) {\it accumulated over time}.  Thus, the value of a PID controller's integral term is a function of past (accumulated) error.

Derivative action is said to work on the future, because its action is based on the rate-of-change over time of the PV, which is a good predictor of overshoot.  This is why derivative action is sometimes called {\it preact}, because it preemptively acts to avoid overshoot of setpoint.  This is analogous to a passenger in a fast-moving automobile, who can ``predict'' that the car's high speed will likely lead to ``overshoot'' of an intersection.

%(END_ANSWER)





%(BEGIN_NOTES)


%INDEX% Control, proportional + integral + derivative: actions contrasted with regard to time

%(END_NOTES)


