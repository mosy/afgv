
%(BEGIN_QUESTION)
% Copyright 2007, Tony R. Kuphaldt, released under the Creative Commons Attribution License (v 1.0)
% This means you may do almost anything with this work of mine, so long as you give me proper credit

En "reversvirkende" regulator er definert som en der utgangen minker når prosessvariabelen (PV) øker.

Hvordan påvirker denne definisjonen fortegnet til P-, I- og D-virkningene i PID-ligningen? Vil de alle jobbe i samme retning (alle negative for revers virkemåte), eller kan de ha forskjellige fortegn?

\underbar{file i01640}
%(END_QUESTION)





%(BEGIN_ANSWER)

I en standard PID-regulator jobber alle tre leddene (P, I og D) sammen i samme retning. Hvis regulatoren er reversvirkende, vil både P-, I- og D-bidragene virke for å redusere utgangen ved en økning i PV.

(Unntaket er visse spesielle implementeringer der man kan velge virkemåte individuelt per ledd, men dette er uvanlig i standard prosesskontroll).

%(END_ANSWER)





%(BEGIN_NOTES)

Det er viktig å forstå at "virkemåte" (Action) er en global innstilling for hele regulatoren, som snur fortegnet på alle leddene samtidig.

%INDEX% Control, proportional + integral + derivative: relative directions of action

%(END_NOTES)
