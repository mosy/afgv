
%(BEGIN_QUESTION)
% Copyright 2010, Tony R. Kuphaldt, released under the Creative Commons Attribution License (v 1.0)
% This means you may do almost anything with this work of mine, so long as you give me proper credit

Read and outline the ``Different PID Equations'' section of the ``Closed-Loop Control'' chapter in your {\it Lessons In Industrial Instrumentation} textbook.  Note the page numbers where important illustrations, photographs, equations, tables, and other relevant details are found.  Prepare to thoughtfully discuss with your instructor and classmates the concepts and examples explored in this reading.

\underbar{file i01642}
%(END_QUESTION)





%(BEGIN_ANSWER)


%(END_ANSWER)





%(BEGIN_NOTES)

In the parallel equation, each tuning constant has independent effect over its action:

$$m = K_p e + {1 \over \tau_i} \int e \> dt + \tau_d {de \over dt} + b \hbox{\hskip 50pt {\bf Parallel PID equation}}$$

\vskip 10pt

In the ideal equation, the gain constant ($K_p$) affects all three actions:

$$m = K_p \left( e + {1 \over \tau_i} \int e \> dt + \tau_d {de \over dt} \right) + b \hbox{\hskip 50pt {\bf Ideal} or {\bf ISA PID equation}}$$

\vskip 10pt

In the series equation, the gain constant ($K_p$) affects all three actions, while the $\tau_i$ and $\tau_d$ constants also affect proportional action:

$$m = K_p \left[ \left({\tau_d \over \tau_i} + 1 \right) e + {1 \over \tau_i} \int e \> dt + \tau_d {de \over dt} \right] + b \hbox{\hskip 25pt {\bf Series} or {\bf Interacting PID equation}}$$

The series equation is an artifact of pneumatic and analog electronic controller design, where it was less expensive to build a controller to implement this equation that it was to implement either of the other two equations.








\vskip 20pt \vbox{\hrule \hbox{\strut \vrule{} {\bf Suggestions for Socratic discussion} \vrule} \hrule}

\begin{itemize}
\item{} Why should we care about there being different versions of the PID equation?  In other words, what practical importance does this hold for you as an instrument technician?
\item{} Explain how to mathematically derive each of the independent action formulae (solving for $\Delta m$, the contribution of each control term to the one output) shown in the textbook.
\item{} Which equation(s) allows us to make an integral-only controller simply by setting $K_p$ and $\tau_d$ both to zero?
\item{} Which equation(s) modify the aggressiveness of all three terms (P, I, and D) when $K_p$ alone is adjusted?
\item{} Why do we even have something as convoluted as the ``series'' or ``interacting'' PID equation at all?
\end{itemize}

%INDEX% Reading assignment: Lessons In Industrial Instrumentation, Closed-Loop Control (different PID equations)

%(END_NOTES)


