
%(BEGIN_QUESTION)
% Copyright 2007, Tony R. Kuphaldt, released under the Creative Commons Attribution License (v 1.0)
% This means you may do almost anything with this work of mine, so long as you give me proper credit

On some PID controllers, an option is given to allow derivative action to act on process variable (PV) changes only, or act on error (PV $-$ SP) changes like integral action always does.  What benefit would it be to have derivative action working on PV changes only and not SP changes?

\underbar{file i01644}
%(END_QUESTION)





%(BEGIN_ANSWER)

When derivative action works on the {\it error} signal, it responds to changes in setpoint (SP) and process variable (PV) equally.  This will result in the controller output saturating (100\% or 0\%) upon step changes in setpoint, which can be a bad thing.  That is why some controllers provide the option of having derivative action work only on PV changes only and not SP changes.

%(END_ANSWER)





%(BEGIN_NOTES)

However, on some processes (notably, those with large first-order time-constants, or {\it lags}), having the output saturate on a setpoint change effectively cancels out the lag and results in much faster response to the new setpoint (up to four times faster, according to some experts!).  So, it should not be assumed that derivative action on the error signal is always undesirable.

%INDEX% Control, derivative: action on PV only

%(END_NOTES)


