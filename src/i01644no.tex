
%(BEGIN_QUESTION)
% Copyright 2007, Tony R. Kuphaldt, released under the Creative Commons Attribution License (v 1.0)
% This means you may do almost anything with this work of mine, so long as you give me proper credit

På mange PID-regulatorer kan man velge om derivatvirkningen skal reagere på endringer i prosessvariabelen (PV) eller på endringer i avviket (Error).

Hva er fordelen med å la derivatvirkningen kun reagere på PV, og ikke på avviket?

\underbar{file i01644}
%(END_QUESTION)





%(BEGIN_ANSWER)

Hvis derivatdelen reagerer på avviket (Error), vil den reagere kraftig på endringer i settpunktet (SP), siden Error = SP - PV. En trinnvis endring i SP representerer en uendelig rask endringshastighet, som vil gi et voldsomt "rykk" (kick) i utgangen.

Ved å la derivatdelen kun se på PV, unngår man dette rykket ved settpunktsendringer, samtidig som man beholder dempningseffekten på prosessforstyrrelser.

%(END_ANSWER)





%(BEGIN_NOTES)

Dette kalles ofte "Derivative on PV" vs "Derivative on Error". I de fleste prosessapplikasjoner foretrekkes "Derivative on PV" for å gi en jevnere respons ved operatørendringer av settpunkt.

%INDEX% Control, derivative: action on PV only

%(END_NOTES)
