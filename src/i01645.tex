
%(BEGIN_QUESTION)
% Copyright 2007, Tony R. Kuphaldt, released under the Creative Commons Attribution License (v 1.0)
% This means you may do almost anything with this work of mine, so long as you give me proper credit

One day on the job you encounter a process under automatic control that seems to be misbehaving.  The process variable (PV) wanders significantly from setpoint (SP) over time, as evidenced by the trend recorder.  Having learned techniques for PID tuning, your first instinct may be to begin adjusting the controller's P, I, and D constants.  However, you know better than to rush into tuning a loop before you understand more about it!

\vskip 10pt

Identify some of the factors {\it other than} a mis-tuned controller that might be responsible for the poor control quality, and also how you might identify them.

\underbar{file i01645}
%(END_QUESTION)





%(BEGIN_ANSWER)

There are many potential sources of control loop instability, including (but not limited to):

\begin{itemize}
\item{}Faulty primary sensing element (PSE), sending erratic or inaccurate signals to the transmitter.
\item{}Faulty transmitter, sending the controller an erratic or inaccurate process variable signal.
\item{}Faulty final control element, not responding properly to commands from the controller.
\item{}Faulty signal path(s), not communicating signals properly to and from the controller.
\item{}Mis-engineered loop, where stable control is essentially {\it impossible} to achieve as built.
\item{}Unusual process load fluctuations, taxing the instrumentation's ability to measure and control.
\end{itemize} 

%(END_ANSWER)





%(BEGIN_NOTES)

Of course, this list is not exhaustive.  It goes to show, however, that the controller should never be assumed to be the source of loop instability simply because it is the most accessible component of the loop!

%INDEX% Control, PID tuning: causes of instability other than controller tuning

%(END_NOTES)


