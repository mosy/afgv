
%(BEGIN_QUESTION)
% Copyright 2007, Tony R. Kuphaldt, released under the Creative Commons Attribution License (v 1.0)
% This means you may do almost anything with this work of mine, so long as you give me proper credit

A very useful technique for testing process control loop response is to subject it to a ``step-change'' in controller output.  In other words, the process is {\it perturbed} (the highly technical term for this is ``bumped'') and the results recorded to learn more about its characteristics.

What practical concerns might surround ``bumping'' a process such as this?  Remember, the process variable (PV) is a real physical measurement such as pressure, level, flow, temperature, pH, or any number of quantities.  What precautions should you take prior to perturbing a process to check its response?

\underbar{file i01652}
%(END_QUESTION)





%(BEGIN_ANSWER)

Some processes may not take well to ``bumps,'' especially large bumps.  Imagine ``bumping'' the coolant flow to a nuclear reactor or the fuel flow to a large steam boiler: the results could be catastrophic!  Not only is it a potential problem to exceed an operating limit (PV too high or too low) in a process, but it may be dangerous to exceed a certain {\it rate of change} over time.  

Short of catastrophe, unacceptable variations in product quality may result from perturbations of the process.  Again, these may be functions of absolute limit (PV too high or too low), and/or rates of change over time.

Remember, the purpose of regulatory control systems is to maintain the PV at or near setpoint.  Any time the control system is disabled and the process purposely ``bumped,'' this purpose is defeated, if only momentarily.  It is essential that operations personnel be consulted prior to manually perturbing a process, so that no safety or operating limit is exceeded in the tuning process.

%(END_ANSWER)





%(BEGIN_NOTES)

A prime example of a process where rate-of-change is critical is temperature control in a high-pressure fluid system such as a catalytic reactor.  Changing the temperature too fast will result in undue stress to welded pipe and vessel joints, and/or refractory linings.

An example of unacceptable variation might be pH control in an alcohol brewing process.  If the pH of the ``mash'' drifts outside of narrow operating limits, the resulting alcohol will be of greatly diminished quality.

%INDEX% Control, PID tuning: manual mode as a diagnostic tool

%(END_NOTES)


