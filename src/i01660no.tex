
%(BEGIN_QUESTION)
% Copyright 2007, Tony R. Kuphaldt, released under the Creative Commons Attribution License (v 1.0)
% This means you may do almost anything with this work of mine, so long as you give me proper credit

Se på denne P\&ID-en:
 
$$\includegraphics[width=15.5cm]{i01660x01.eps}$$

«Matebåndet» (feed conveyor) fører granulært materiale inn i siloen, der nivået av haugen måles av en radar-nivåtransmitter (bølgeleder). «Uttaksbåndet» (demand conveyor) trekker materiale fra bunnen av siloen etter behov for å forsyne en annen prosess.
 
Hva vil skje med nivået inne i siloen over tid hvis et transportbånd flytter mer materiale enn det andre?

\vskip 10pt

Vil du karakterisere denne prosessen som iboende {\it selvregulerende} eller iboende {\it integrerende}?

\underbar{file i01660no}
%(END_QUESTION)





%(BEGIN_ANSWER)

The level inside the hopper will drift either up or down (depending on which conveyor moves more material) at a rate determined by the differential material flow ($Q_{in}$ $-$ $Q_{out}$).  This makes it an {\it integrating} process.

\vskip 10pt

Integrating processes are characterized by the capacity to experience persistent mass and/or energy imbalances, where the out-flow of mass and/or energy does not naturally reach equilibrium the in-flow over time.  Self-regulating processes, by contrast, naturally equalize their mass and energy balances as the process variable changes.

%(END_ANSWER)





%(BEGIN_NOTES)

Integrating processes are characterized by mass and/or energy imbalances, where the out-flow of mass and/or energy does not naturally reach equilibrium the in-flow over time.

%INDEX% Control, process characteristics: self-regulating versus integrating

%(END_NOTES)
