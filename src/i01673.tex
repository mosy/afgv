
%(BEGIN_QUESTION)
% Copyright 2011, Tony R. Kuphaldt, released under the Creative Commons Attribution License (v 1.0)
% This means you may do almost anything with this work of mine, so long as you give me proper credit

Read and outline Case History \#53 (``Another Valve with Weird Characteristics'') from Michael Brown's collection of control loop optimization tutorials.  Prepare to thoughtfully discuss with your instructor and classmates the concepts and examples explored in this reading, and answer the following questions:

\begin{itemize}
\item{} Identify the type of test performed on this valve prior to placing the loop controller in automatic and noting problems.
\vskip 10pt
\item{} Explain what feature(s) of the second valve test revealed the problem.
\vskip 10pt
\item{} Instead of repairing the valve, the refinery personnel decided they would rather fix the problem through PID tuning.  Explain how they were able to do this, even though it was not the ideal solution.
\vskip 10pt
\item{} Note the parameters appearing underneath ``Loop Signature'' in the screen-capture of the Protuner software.  What parameters does Protuner measure about the process, and how might this save a lot of time for the technician or engineer optimizing a control system?
\end{itemize}

\underbar{file i01673}
%(END_QUESTION)





%(BEGIN_ANSWER)


%(END_ANSWER)





%(BEGIN_NOTES)

The first open-loop test of the loop showed what appeared to be nicely linear control valve behavior, with no hysteresis or any other problems (figure 1).  The loop suddenly went unstable in automatic mode (figure 2), and a subsequent open-loop test where the control valve was opened up further than before revealed a nonlinearity (figure 3) at around 68\% output.

\vskip 10pt

The client didn't deem this loop important enough to truly fix, and so they chose to ``de-tune'' the controller so it would be stable in the high-gain region of the control valve.

\vskip 10pt

The final closed-loop tests (figure 5) reveal consistently faster response when the valve is further open, and some instability around full-closed when the valve plug is close to the seat.  This latter problem was deemed okay because the process would never be expected to run in that condition ordinarily.




%INDEX% Reading assignment: Michael Brown Case History #53, "Another valve with weird characteristics"

%(END_NOTES)


