
%(BEGIN_QUESTION)
% Copyright 2011, Tony R. Kuphaldt, released under the Creative Commons Attribution License (v 1.0)
% This means you may do almost anything with this work of mine, so long as you give me proper credit

Read and outline Case History \#59 (``The Regulatory Controls In Most Plants Do Not Work Properly'') from Michael Brown's collection of control loop optimization tutorials.  Prepare to thoughtfully discuss with your instructor and classmates the concepts and examples explored in this reading, and answer the following questions:

\begin{itemize}
\item{} The title of this Case History gives a sweeping indictment of process control quality in industry.  According to Mr. Brown, what percentage of loops does he typically find in poor condition?  Identify the specific industries mentioned in this short collection of (four) case histories.
\vskip 10pt
\item{} How is it possible for industrial plant operators to manage decent production when the underlying control loops tend to behave so poorly?  What, specifically, do some operators do to manage poor loops?
\vskip 10pt
\item{} Describe the problem in Case \#2, and how it was determined from an open-loop test.
\vskip 10pt
\item{} Explain what is so bizarre about the results of the open-loop test in Case \#3, and why there was no possible way to overcome this problem through PID tuning (adjusting the P, I, and D parameters).
\vskip 10pt
\item{} In Case \#4, one of the problems mentioned is the presence of too much ``damping'' in the flowmeter.  Damping often goes by the synonym ``filtering,'' and it can cause a lot of problems in a control loop.  Why was this particular flowmeter configured to have so much damping in it?
\end{itemize}

\vskip 20pt \vbox{\hrule \hbox{\strut \vrule{} {\bf Suggestions for Socratic discussion} \vrule} \hrule}

\begin{itemize}
\item{} For those of you who have studied flowmeter technologies, briefly review the operating principle of a {\it magnetic} flowmeter.  Mr. Brown states the flow range accuracy of a magnetic flowmeter being about 10:1.  How does this amount of turndown compare with other flowmeter types such as orifice plates and Coriolis flowmeters?
\item{} For those of you who have studied flowmeter technologies, explain why a DC-excited magnetic flowmeter might be susceptible to noise caused by solid particles moving through the liquid flowstream. 
\item{} For those of you who have studied transmitters in detail, briefly review the concept of {\it filtering} and identify what useful purpose(s) it may serve in a measurement system.
\end{itemize}

\underbar{file i01674}
%(END_QUESTION)





%(BEGIN_ANSWER)


%(END_ANSWER)





%(BEGIN_NOTES)

According to Michael Brown, the basic loop controls in over 95\% of operating plants are ``completely inefficient in automatic [mode]''.  In this article, he cites examples from a mine and at least one pulp mill.

\vskip 10pt

Operators manage poorly-performing control loops by switching to manual mode for upsets, then returning to automatic only when things are stable again.  Essentially, they act as the real controller!

\vskip 10pt

In case \#2, magnetic flowmeter was grossly over-ranged (normal flow was 70 LPM, span of FT was 6600 LPM!).  Moving the control valve through its entire stroke only resulted in an indicated flow change of about 3\%!!!

\vskip 10pt

In case \#3, pulp consistency showed almost no relationship to valve position!  This indicates an engineering problem, since the final control element seems to have almost no effect at all on the process variable.  There is no amount of PID tuning in the world that can possibly fix a problem like this!

\vskip 10pt

In case \#4, flowmeter was programmed with a lot of damping to try to overcome noise, caused by poor choice of meters for the process (DC magflow with high-solids content pulp).













\filbreak \vskip 20pt \vbox{\hrule \hbox{\strut \vrule{} {\bf Virtual Troubleshooting} \vrule} \hrule}

Briefly introduce the concept of magnetic flow measurement, showing how some magflow meters use DC excitation while others use AC excitation.  Then, have students explain why a DC magflow meter is susceptible to noisy flow measurements due to solids entrained in the flowstream, while AC magflow meters are not.

\vskip 10pt

Answer: {\it solid particles can carry electrostatic charges, which will electrostatically induce noise at the flowtube electrodes as they pass by.  A magnetic flowmeter using DC to excite the field will consequently interpret these electrostatic transients as changes in flow rate, because any fluctuation in electromagnetically induced voltage should logically represent fluctuations in motion (flow rate) given a constant magnetic field.}

{\it If, however, we use AC to excite the field coil of a magnetic flowmeter, however, the expected polarity of the flow signal continually alternates.  An AC magflow meter's electronics are designed to look for the same frequency in the flow signal as the coil's excitation power.  Thus, changes in flow rate should result in changes in an AC voltage signal's magnitude (at the same frequency of excitation).  Any DC transient (caused by charged solids) will merely be a ``blip'' on an AC signal, and therefore filtered out by the magmeter's electronics as not part of that AC signal and therefore will not be interpreted as an actual change in flow.}





%INDEX% Reading assignment: Michael Brown Case History #59, "The regulatory controls in most plants do not work properly"

%(END_NOTES)


