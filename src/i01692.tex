
%(BEGIN_QUESTION)
% Copyright 2007, Tony R. Kuphaldt, released under the Creative Commons Attribution License (v 1.0)
% This means you may do almost anything with this work of mine, so long as you give me proper credit

I et kaskadekoblet reguleringssystem, hvilken sløyfe må \textit{alltid} ha kortere tidskonstant og døtid, den primære eller den sekundære? forklar hvorfor. 

\vskip 20pt \vbox{\hrule \hbox{\strut \vrule{} {\bf Suggestions for Socratic discussion} \vrule} \hrule}

\begin{itemize}
\item{} Explain how the lag and dead times of each loop may be quantitatively measured in a live process.
\end{itemize}

\underbar{file i01692}
%(END_QUESTION)





%(BEGIN_ANSWER)


%(END_ANSWER)





%(BEGIN_NOTES)

The slave (secondary) loop in a cascade control system must always have the shorter dead/lag times, because a cascade control system will not work if a fast loop is ``giving orders'' to a slow loop.  The secondary loop must always respond faster (have shorter dead and lag times) than the primary loop if there is to be any hope of good control.  Otherwise, the primary loop may call for the secondary loop to respond faster than it is capable of responding.

According to F.G. Shinskey and B.G. Lipt\'ak (\underbar{Chapter 1.5: Control -- Cascade Loops}, {\it Instrument Engineer's Handbook, Process Control, Third Edition}, pg. 43), the slave (secondary) process should have a time constant 4 to 10 times faster (shorter) than the master (primary) process, and a period of oscillation 2 to 3 times faster than the master (primary) process.

%INDEX% Control, strategies: cascade

%(END_NOTES)


