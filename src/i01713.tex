
%(BEGIN_QUESTION)
% Copyright 2007, Tony R. Kuphaldt, released under the Creative Commons Attribution License (v 1.0)
% This means you may do almost anything with this work of mine, so long as you give me proper credit

Identify two different methods to help prevent cavitation from happening in a control valve, and explain why each method works.

\vskip 100pt

\underbar{file i01713}
%(END_QUESTION)





%(BEGIN_ANSWER)

I recommend assigning 2.5 points for each correct cavitation-controlling method, and 2.5 points for each correct reason.

\begin{itemize}
\item{} Decrease temperature of fluid (decreases vapor pressure and makes it more difficult to flash in a valve)
\item{} Increase both upstream and downstream pressures on valve (raises $P_{vc}$ and helps avoid dropping below vapor pressure of liquid).
\item{} Install cavitation-control valve trim (reduces pressure in multiple stages instead of in one step, helps raise $P_{vc}$ and avoid flashing).
\item{} Change valve to one with a greater pressure recover factor (larger $F_L$, which is the same as saying less pressure recovery $P_2 - P_{vc}$).  This raises $P_{vc}$ (for any given pressure drop $P_1 - P_2$) and helps avoid flashing.
\end{itemize}

%(END_ANSWER)





%(BEGIN_NOTES)

{\bf This question is intended for exams only and not worksheets!}.

%(END_NOTES)


