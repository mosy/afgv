
%(BEGIN_QUESTION)
% Copyright 2007, Tony R. Kuphaldt, released under the Creative Commons Attribution License (v 1.0)
% This means you may do almost anything with this work of mine, so long as you give me proper credit

Explain what {\it choked flow} refers to for a control valve, and what accounts for ``choking'' in liquid flow control applications and in gas flow control applications.  Also, identify which type of control valve is most susceptible to choked flow: one with a high $F_L$ (pressure recovery factor) or one with a low $F_L$.

\underbar{file i01716}
%(END_QUESTION)





%(BEGIN_ANSWER)

{\it Choked flow} is when the flow rate through a control valve ``saturates'' such that further decreases in downstream pressure do not result in any additional flow:

\begin{itemize}
\item{} Phase change (flashing) in liquid flows
\vskip 5pt
\item{} Sonic velocities in gas flows
\end{itemize}

I will let you figure out how $F_L$ relates to choked flow.

%(END_ANSWER)





%(BEGIN_NOTES)

Although these two different mechanisms of choked flow are quite different from one another, their effects are remarkably similar.  In either case, the valve's flow rate saturates at a certain (maximum) level not predicted by the standard inviscid flow equation:

$$Q = C_v \sqrt{{P_1 - P_2} \over G_f}$$

It should be emphasized that liquid flows may ``choke'' without attaining sonic velocities.  All that is required for liquid choking is for the pressure to fall below the liquid's vapor pressure, thus resulting in boiling.  The two mechanisms of choking (liquid versus gas) are completely different from each other, despite their similar final effects.

%INDEX% Final Control Elements, valve: choked flow
%INDEX% Final Control Elements, valve: pressure recovery factor

%(END_NOTES)


