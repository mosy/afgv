
%(BEGIN_QUESTION)
% Copyright 2011, Tony R. Kuphaldt, released under the Creative Commons Attribution License (v 1.0)
% This means you may do almost anything with this work of mine, so long as you give me proper credit

Read and outline Case History \#87 (``Problems Beyond Belief!'') from Michael Brown's collection of control loop optimization tutorials.  Prepare to thoughtfully discuss with your instructor and classmates the concepts and examples explored in this reading, and answer the following questions:

\begin{itemize}
\item{} In Mr. Brown's experience, how many fast-responding control loops can you reasonably expect to optimize in a working day?  How does this compare with the amount of time you have found yourself spending to characterize (open-loop test) and tune a PID loop?
\vskip 10pt
\item{} Describe what Mr. Brown discovered while trying to optimize the flow-control loop whose trend is shown in Figure 1.  The details here are actually quite funny.
\vskip 10pt
\item{} The next loop described in this case history was another flow-control loop, and it was discovered that the operators had purposely left a bypass valve open around the control valve.  Explain what their purpose was in doing this, and why it compromised the quality of control in this system.  Also explain what sort of problem can be seen in the open-loop trend of Figure 2, and why this loop's control stability was doomed because of it.
\vskip 10pt
\item{} Examine the trend shown in Figure 4.  Even though this is a closed-loop (automatic mode) test, we can still discern what type of problem the control valve has when we look at the PV, SP, and Output plots.  Explain how the shapes of these graphs reveal the poor health of the control valve.
\end{itemize}



\vskip 20pt \vbox{\hrule \hbox{\strut \vrule{} {\bf Suggestions for Socratic discussion} \vrule} \hrule}

\begin{itemize}
\item{} In this case history, Mr. Brown briefly discusses ``advanced control strategies'' and their interface with ``base-layer'' PID controls.  Explain how these advanced control strategies work in a regular control system, especially how they are dependent upon good tuning of the PID control loops.
\item{} The trend shown in Figure 4 also reveals quite a bit about the controller's PID tuning.  Examine the PV, SP, and Output trends and then explain what we may discern about the controller's tuning from this.  Is the tuning P, I, or D-dominant?  Can we calculate the controller's gain, reset time, or rate time?
\end{itemize}

\underbar{file i01717}
%(END_QUESTION)





%(BEGIN_ANSWER)


%(END_ANSWER)





%(BEGIN_NOTES)

In Michael Brown's experience (optimizing loops in over 500 plants), typically only 5\% of loops working well in automatic mode!  His experience with contractors commissioned to install an advanced control system (supervisory setpoint control) is that they often try to optimize the base-layer PID controls but do not allocate sufficient time for the task.

\vskip 10pt

Optimizing 10 to 15 loops a day is a lot if they respond fast (e.g. flow, small-capacity level).  With slow-responding loops such as temperature control, you would be lucky to optimize more than two per day.  A large plant reports optimizing 250 loops in 6 months' time.

\vskip 10pt

Figure 1 loop: system was ``so simple'' that it didn't need a P\&ID diagram!  In reality, a manual bypass valve had been opened wide around the control valve to render the control valve ineffective.  A second control valve had been placed in series with the first, and was part of some other control system!

\vskip 10pt

Figure 2: very sticky valve.  Operators had opened the bypass valve wide to avoid the control problems caused by the sticky control valve.

\vskip 10pt

Figure 4: you can see a long period where the controller's integral action must wind down about 20\% before the valve responds!

Incidentally, Figure 4's trend reveals a controller that is purely integral (I-only), with a reset time of approximately 12 seconds.



\vskip 20pt \vbox{\hrule \hbox{\strut \vrule{} {\bf Suggestions for Socratic discussion} \vrule} \hrule}

\begin{itemize}
\item{} Is the controller in Figure 4's closed-loop trend direct-acting or reverse-acting?
\end{itemize}













\vfil \eject

\noindent
{\bf Prep Quiz:}

In Michael Brown's Case History \#87 (``Problems Beyond Belief!''), he recounts a ``simple'' liquid flow control loop with no supporting P\&ID documentation, which everyone thought was functioning well.  Two major problems were discovered with the field equipment and piping, absolutely preventing any chance of good control.  Identify one of these field problems and then explain why it made good control impossible.


%INDEX% Reading assignment: Michael Brown Case History #87, "Problems beyond belief!"

%(END_NOTES)


