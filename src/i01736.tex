
%(BEGIN_QUESTION)
% Copyright 2007, Tony R. Kuphaldt, released under the Creative Commons Attribution License (v 1.0)
% This means you may do almost anything with this work of mine, so long as you give me proper credit

Modern automobile engines use computer controls to adjust the {\it air/fuel ratio} for an optimum balance between economy, performance, and emissions.  Explain how a typical spark-ignition engine accomplishes this ratio control.  What constitutes the process variable (PV), setpoint, output, ``wild'' flow, and load(s) in such a control system?

\vskip 10pt

Hint: for those who have never worked on a car engine before, the air flow to the engine is controlled by a butterfly valve mechanism directly actuated by the accelerator pedal, while fuel flow to the engine is controlled by precise timing spray valves called {\it injectors} (actuated by the electronic engine control computer).

\vskip 20pt \vbox{\hrule \hbox{\strut \vrule{} {\bf Suggestions for Socratic discussion} \vrule} \hrule}

\begin{itemize}
\item{} For those who have studied engines and/or chemistry, explain why fuel-ratio control is critical in spark-ignition engines.
\item{} Carbureted engines used to be equipped with {\it choke} valves to richen the mixture going into the engine when it was first started, because the normal air/fuel ratio is simply too lean for a cold engine to run on.  Explain why fuel-injected engines don't require choke valves.
\item{} Engines with computer-controlled air:fuel mixture ratios typically employ an {\it oxygen sensor} in the exhaust pipe of the engine to monitor how much unreacted oxygen is left in the exhaust gas stream.  Explain why this is done, and how that sensor's signal may be used to optimize the engine's air:fuel ratio.
\item{} One of the more popular air flowmeters used in engine ratio control systems is the so-called {\it hot-wire anemometer} which places an electrically heated metal wire in the path of the incoming air and senses the temperature of that wire as it is cooled by the air flowing by.  For those who have studied flowmeters, identify the general category of flowmeter that best applies to a hot-wire anemometer, and explain why this is a very good flowmeter choice for this particular application.
\end{itemize}

\underbar{file i01736}
%(END_QUESTION)





%(BEGIN_ANSWER)

\begin{itemize}
\item{} PV = fuel flow through injectors
\item{} ``Wild'' flow = air flow into engine
\item{} SP = multiplied air flow signal
\item{} Output = timing pulses (PWM) to injectors
\item{} Loads = Fuel pump pressure, air density
\end{itemize}

%(END_ANSWER)





%(BEGIN_NOTES)


%INDEX% Control, strategies: ratio
%INDEX% Process: engine air/fuel mixture

%(END_NOTES)


