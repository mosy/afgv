
%(BEGIN_QUESTION)
% Copyright 2007, Tony R. Kuphaldt, released under the Creative Commons Attribution License (v 1.0)
% This means you may do almost anything with this work of mine, so long as you give me proper credit

Identify types of processes that respond well to aggressive {\it proportional} action, and why they do.

\vskip 200pt

Identify types of processes that respond poorly to aggressive {\it proportional} action, and why they do.

\vfil

\underbar{file i01743}
\eject
%(END_QUESTION)





%(BEGIN_ANSWER)

Processes having absolutely no dead time but significant lag time (i.e. pure {\it first-order lag}) processes tend to respond well to aggressive proportional control action.  Also, integrating processes do as well, as strong proportional action in the controller allows the natural integrating action in the process to correct offset.

\vskip 10pt

Processes having significant dead times can oscillate with proportional control action.  Noisy processes don't fare well with aggressive proportional action either because the gain transfers process noise directly to the manipulated variable where it wears out the control valve.

%(END_ANSWER)





%(BEGIN_NOTES)


%INDEX% Control, PID tuning: process responses to proportional action

%(END_NOTES)


