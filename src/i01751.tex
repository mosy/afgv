
%(BEGIN_QUESTION)
% Copyright 2007, Tony R. Kuphaldt, released under the Creative Commons Attribution License (v 1.0)
% This means you may do almost anything with this work of mine, so long as you give me proper credit

In the United States of America, where the Superbowl is a very popular sporting event to watch from home, water and wastewater treatment plant operators have learned to pay attention to the game in order to improve process control.  They know that a significant percentage of the population in any metropolitan area will simultaneously use the bathroom during breaks in the game, particularly the half-time break.  This places unusual demands on both the water supply and the wastewater handling systems at very specific times.

So, when a break begins, these operators at the water and wastewater treatment facilities preemptively turn on spare pumps to handle the impending flow rates, so that the systems do a better job maintaining setpoint.  There is a technical term for this sort of control strategy: {\it feedforward}.  Explain what ``feedforward'' control is, in your own words, and compare it against the more customary ``feedback'' control philosophy.

\vskip 20pt

A practical example of feedforward control on a grand scale is the control of reservoir water level at hydroelectric dams.  The dam must use or ``spill'' excess water when the reservoir is nearing full capacity, in order to avoid over-filling of the reservoir.  A feedforward variable relevant to this control problem is ambient air temperature in the high mountain regions surrounding the reservoir.  Explain how mountain temperature relates to reservoir level, and how such a feedforward control strategy might work in a hydroelectric dam.

\underbar{file i01751}
%(END_QUESTION)





%(BEGIN_ANSWER)


%(END_ANSWER)





%(BEGIN_NOTES)

{\it Feedforward control} works by monitoring all major loads on a process, in order to avoid a major upset in a process.  In other words, while feedback control ``looks back'' to see how it did, feedforward control ``looks forward'' to see what {\it will happen} and takes corrective action before an effect is seen on the process variable.

\vskip 10pt

As mountain temperatures rise, the rate of snow melt increases, which adds more water to the reservoir upstream of a hydroelectric dam.  Therefore, temperature rise is a predictor of imminent reservoir level rise, and could be used as a feedforward variable to preemptively spill water from the reservoir in anticipation of the impending rise.

%INDEX% Control, strategies: feedforward
%INDEX% Process: municipal water flow control

%(END_NOTES)


