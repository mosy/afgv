
%(BEGIN_QUESTION)
% Copyright 2011, Tony R. Kuphaldt, released under the Creative Commons Attribution License (v 1.0)
% This means you may do almost anything with this work of mine, so long as you give me proper credit

\vbox{\hrule \hbox{\strut \vrule{} {\bf Desktop Process exercise} \vrule} \hrule}

Configure your Desktop Process for full proportional-plus-integral-plus-derivative (PID) control.  Experiment with different ``gain,'' ``reset,'' and ``rate'' tuning parameter values until reasonably good control is obtained from the process (i.e. fast response to setpoint changes with minimal ``overshoot,'' good recovery from load changes).  Record these ``optimum'' P, I, and D settings you find for your process, for future reference.

\vskip 10pt

Next, purposely add excessive filtering (``damping'') to the process variable (PV) input of the controller.  If the controller is programmed using function blocks, the filtering parameter will most likely be found in the {\it Analog Input} function block.  The purpose of this exercise is to see how excessive filtering compromises what would otherwise be good control, so be sure to enter a value that is truly too slow for your process.  For motor speed control, a filtering time of 5 seconds should be adequate (equivalent to a filter breakpoint frequency of 0.2 Hz, if your controller accepts frequency rather than time for the filtering parameter).

\vskip 10pt

After entering this filter value into the controller, place it in automatic mode and observe how well (or poorly) it controls the process now.  Pay close attention to the actual process variable value as compared to the process variable display on the controller faceplate.  Does the faceplate PV display overshoot setpoint?  Does the real process variable (as directly observed by you) overshoot as well?  Do they overshoot the same amount?  Do they overshoot at the same time?

Try re-tuning the controller's PID function to achieve better control (i.e. less overshoot) than you have now.  Are you able to achieve better control than the previous (optimum) PID tuning values?  If so, how?  If you succeed in getting the controller's PV display to ``control'' well with different tuning values, does the real process variable (as directly observed by you) control as well as the displayed PV, or worse?

\vskip 10pt

Feel free to set the filtering parameter to different values and re-tuning the controller again to achieve the best control possible.

\vskip 10pt

Discuss with your teammates how you see PV filtering affecting process stability.  This exercise should clearly demonstrate how excessive filtering compromises control quality, but do you think there are beneficial applications for PV filtering in a controller?

\underbar{file i01787}
%(END_QUESTION)





%(BEGIN_ANSWER)


%(END_ANSWER)





%(BEGIN_NOTES)


%INDEX% Desktop Process: effects of excessive input filtering

%(END_NOTES)


