
%(BEGIN_QUESTION)
% Copyright 2011, Tony R. Kuphaldt, released under the Creative Commons Attribution License (v 1.0)
% This means you may do almost anything with this work of mine, so long as you give me proper credit

Read and outline Case History \#51 (``Faulty Positioner Causes Instability'') from Michael Brown's collection of control loop optimization tutorials.  Prepare to thoughtfully discuss with your instructor and classmates the concepts and examples explored in this reading, and answer the following questions:

\begin{itemize}
\item{} Describe the purpose of the control system being optimized by Mr. Brown and his class at this chemical plant.  
\vskip 10pt
\item{} Which loop is the {\it master} and which is the {\it slave}?
\vskip 10pt
\item{} Which loop did they begin to optimize {\it first}, and why do you suppose they began with that one?
\vskip 10pt
\item{} Which test -- closed-loop or open-loop -- provided the most information on the source of the problem?
\vskip 10pt
\item{} The introduction to this Case History states, ``Repeated attempts had been made by the plant over an extended period of time to tune the loop to stop the cycling.''  Explain why you think this was that case.  How come Michael and his class could find and correct the problem when others at the chemical plant could not?  Although the Case History does not tell us, what do you suppose the plant workers tried to do to tame this unruly loop before the optimization class was held?
\end{itemize}

\vskip 20pt \vbox{\hrule \hbox{\strut \vrule{} {\bf Suggestions for Socratic discussion} \vrule} \hrule}

\begin{itemize}
\item{} Suppose during this loop-optimization exercise that no problems were found in any of the field instruments.  Can you think of anything else that might cause the strange ``overshoot'' responses seen in the trend of Figure 3?
\item{} Other than a valve problem, what else may we determine about the fuel pressure process from the tests shown in Figure 3?
\end{itemize}

\underbar{file i01793}
%(END_QUESTION)





%(BEGIN_ANSWER)


%(END_ANSWER)





%(BEGIN_NOTES)

In this process, a cascaded furnace temperature/fuel pressure system experienced cycling (oscillations) on the slave (fuel pressure) control loop as shown in Figure 2.  In this system, temperature is the master PV and fuel pressure is the slave PV.  The temperature controller sends a cascaded setpoint signal to the fuel pressure controller, which in turn controls pressure at the oil nozzles inside the firebox.  The purpose is presumably to compensate for load changes in the fuel delivery system.

\vskip 10pt

The first diagnostic step was to perform open-loop tests on the slave loop to check for problems.  It is noteworthy that they began with the slave loop before the master loop, because the master loop depends on the slave loop being responsive and well-behaved.  Indeed, the control valve was found to be ``overshooting'' its position on step-changes as shown in Figure 3, indicating a field instrument problem in need of correction.

Here, the open-loop test (Figure 3) gave more meaningful diagnostic information than the closed-loop test (Figure 2).  The closed-loop test merely confirmed that oscillations were taking place, while the open-loop test clearly revealed a valve positioner problem.

\vskip 10pt

Most likely, plant personnel tried to fix this problem by tuning and re-tuning the controller(s) in this process.  However, since the root of the problem was a malfunctioning positioner, all such efforts were futile.





\vfil \eject

\noindent
{\bf Summary Quiz:}

In Michael Brown's Case History 51 -- where a problem was identified and corrected in a cascaded temperature-pressure control system -- what did the open-loop test reveal about the faulty system that the first closed-loop test did not?

\begin{itemize}
\item{} That the system was unstable and was not holding closely to setpoint
\vskip 5pt 
\item{} That the control valve had excessive stem friction (hysteresis)
\vskip 5pt 
\item{} That the pressure loop had an unusually long dead time
\vskip 5pt 
\item{} That the pressure transmitter was improperly ranged
\vskip 5pt 
\item{} That the control valve was ``overshooting'' on every step
\vskip 5pt 
\item{} That the temperature sensor was not fully inserted into the thermowell
\end{itemize}

%INDEX% Reading assignment: Michael Brown Case History #51, "Faulty positioner causes instability"

%(END_NOTES)


