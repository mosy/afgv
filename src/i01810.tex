
%(BEGIN_QUESTION)
% Copyright 2007, Tony R. Kuphaldt, released under the Creative Commons Attribution License (v 1.0)
% This means you may do almost anything with this work of mine, so long as you give me proper credit

\vbox{\hrule \hbox{\strut \vrule{} {\bf Desktop Process exercise} \vrule} \hrule}

\noindent
Configure the controller as follows:

\begin{itemize}
\item{} Control action = {\it reverse}
\item{} Gain = 1 (100\% proportional band)
\item{} Reset (Integral) = {\it 0.1 minutes/repeat} = {\it 10 repeats/minute}
\item{} Rate (Derivative) = {\it minimum effect} = {\it 0 minutes} 
\end{itemize}

In this exercise you will watch the action of integral in your controller as it attempts to eliminate a persistent error (difference between PV and SP).  Follow these steps in order:

\begin{itemize}
\item{$1.$} Place controller in manual mode, with the output at 0\%
\vskip 10pt
\item{$2.$} Disconnect electrical power from the motor so it cannot spin 
\vskip 10pt
\item{$3.$} Switch controller to automatic mode
\vskip 10pt
\item{$4.$} Raise setpoint value to 10\%
\vskip 10pt
\item{$5.$} Observe what the controller's output value does in automatic mode
\vskip 10pt
\item{$6.$} Lower setpoint value to 0\%
\vskip 10pt
\item{$7.$} Observe what the controller's output value does in automatic mode
\end{itemize}

Feel free to experiment with different errors (values other than 10\%) and different repeat/minute values to see what effect these parameters have on the controller's integration.

\vskip 20pt \vbox{\hrule \hbox{\strut \vrule{} {\bf Suggestions for Socratic discussion} \vrule} \hrule}

\begin{itemize}
\item{} Identify which units your controller uses to express integral action: minutes per repeat, repeats per minutes, or something else?
\item{} Identify and explain in your own words what factor(s) influence the speed at which the controller output integrates.
\end{itemize}

\underbar{file i01810}
%(END_QUESTION)





%(BEGIN_ANSWER)


%(END_ANSWER)





%(BEGIN_NOTES)

{\bf Lesson:} watching how integral (reset) action behaves when confronted with a persistent error.

%INDEX% Desktop Process: static demonstration of integral action

%(END_NOTES)


