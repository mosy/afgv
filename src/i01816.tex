
%(BEGIN_QUESTION)
% Copyright 2007, Tony R. Kuphaldt, released under the Creative Commons Attribution License (v 1.0)
% This means you may do almost anything with this work of mine, so long as you give me proper credit

Determine a suitable control valve $C_v$ to handle a maximum water flow rate of 750 gallons per minute with an upstream pressure of 125 PSI and a downstream pressure of 110 PSI.  Also, calculate the approximate pipe size (in inches) if the control valve type is a 90$^{o}$ butterfly valve with an offset seat.

\underbar{file i01816}
%(END_QUESTION)





%(BEGIN_ANSWER)

$C_v$ = 193.6

\vskip 10pt

A 2.5 inch valve would almost be large enough.  One size larger (perhaps 3 inches) should be adequate.

%(END_ANSWER)





%(BEGIN_NOTES)

$$Q = C_v \sqrt{\Delta P \over G_f}$$

$$C_d = {C_v \over d^2}$$

\vskip 10pt

Here are some common relative valve capacity factors ($C_d$) for different control valve types, assuming full-area trim, full-open position:
 
\begin{itemize}
\item{} Single-port globe valve, ported plug = 9.5
\item{} Single-port globe valve, contoured plug = 11
\item{} Single-port globe valve, characterized cage = 15
\item{} Double-port globe valve, ported plug = 12.5
\item{} Double-port globe valve, contoured plug = 13
\item{} Rotary ball valve, segmented = 25
\item{} Rotary ball valve, standard port (diameter $\approx$ 0.8$d$) = 30
\item{} Rotary butterfly valve, 60$^{o}$, no offset seat = 17.5
\item{} Rotary butterfly valve, 90$^{o}$, offset seat = 29
\item{} Rotary butterfly valve, 90$^{o}$, no offset seat = 40
\end{itemize}

%INDEX% Final Control Elements, valve: sizing

%(END_NOTES)


