
%(BEGIN_QUESTION)
% Copyright 2011, Tony R. Kuphaldt, released under the Creative Commons Attribution License (v 1.0)
% This means you may do almost anything with this work of mine, so long as you give me proper credit

Read and outline Case History \#70 (``Flow Loops Behaving Badly'') from Michael Brown's collection of control loop optimization tutorials.  Prepare to thoughtfully discuss with your instructor and classmates the concepts and examples explored in this reading, and answer the following questions:

\begin{itemize}
\item{} The first flow-control example shown in this report is graphed in Figure 1.  Mr. Brown tells us how we may conclude this oscillation (``cycling'') is {\it not} the result of an overly-aggressive controller, but rather indicative of a control valve problem.  Explain this in your own words.
\vskip 10pt
\item{} Describe what the open-loop test (Figure 2) reveals about this flow loop's valve condition.  Explain in your own words why an open-loop test of the system is so important to perform when diagnosing a control problem.
\vskip 10pt
\item{} Figure 6 shows a test of a different flow-control system.  Explain how we can tell just by examining the trends that this is an {\it open-loop} test (i.e. controller in manual mode) rather than a {\it closed-loop} test (i.e. controller in automatic mode).
\vskip 10pt
\item{} Toward the end of this case history, Mr. Brown relates the challenges he encountered at a mine.  Describe some of the problems he found there, in your own words.
\vskip 10pt
\item{} An interesting comment Mr. Brown makes toward the end of this report is, ``Luckily the plant uses a lot of variable speed devices for control flows and speeds of feeders.  These loops can be optimized.''  Here, he is referring to the use of variable-speed motors as final control elements (e.g. VFDs powering pumps, etc.) instead of control valves.  Explain why these control loops are so much easier to optimize than those with control valves.  Are there any disadvantages to using VFDs as control elements compared to valves, or do you think VFDs are universally superior?
\end{itemize}

\vskip 20pt \vbox{\hrule \hbox{\strut \vrule{} {\bf Suggestions for Socratic discussion} \vrule} \hrule}

\begin{itemize}
\item{} A theme common to Michael Brown's ``Case History'' reports on control loop optimization is that basic PID controls are often performing very poorly at industrial facilities world-wide.  Explain why you think this is the normal state of affairs.  Why don't the people working at these facilities recognize and correct these problems themselves?
\item{} One of the problem-solving techniques Michael Brown applied to the cycling flow control loop shown in Figure 1 is to compare the cycle's period against what he knew to be the typical period for a flow control loop.  Explain how this relates to the concept of {\it ultimate period} as defined in the Ziegler-Nichols closed-loop tuning technique.
\end{itemize}

\underbar{file i01819}
%(END_QUESTION)





%(BEGIN_ANSWER)


%(END_ANSWER)





%(BEGIN_NOTES)

Triangle output wave and square PV wave is dead giveaway of a valve stiction problem in figure 1 trend (closed-loop).  Open-loop testing revealed a valve that overshoots at each step, and not repeatably either (figure 2).

\vskip 10pt

We can tell figure 6 is open-loop because the PD (output) is stepping in manual increments.  This is a really interesting trend because the oscillations in PV make it almost look like the controller is in automatic mode with too high of a gain!

\vskip 10pt

After 3 days at the new mine, Mr. Brown didn't find a single control loop working properly in automatic!  Level transmitter ``freezes'' its signal on occasion, valve only controls between 30\% and 50\%, etc.  Operator injenuity provides work-arounds to instrument problems.

\vskip 10pt

VFDs make better final control elements because they aren't subject to the same problems as control valves (stiction, weird characteristics, etc.).  However, a variable-speed pump can't provide tight shutoff!


%INDEX% Reading assignment: Michael Brown Case History #70, "Flow loops behaving badly"

%(END_NOTES)


