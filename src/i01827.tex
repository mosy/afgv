
%(BEGIN_QUESTION)
% Copyright 2007, Tony R. Kuphaldt, released under the Creative Commons Attribution License (v 1.0)
% This means you may do almost anything with this work of mine, so long as you give me proper credit

Large combustion systems benefit greatly from {\it oxygen trim control}, keeping the ratio of air to fuel at just the right amount so that there is sufficient oxygen for complete combustion, and little (or no) more.  In such systems, an oxygen analyzer samples flue gas for oxygen content and reports the concentration of exhaust oxygen to the air/fuel ratio control system, which then adjusts (``trims'') the air/fuel ratio accordingly.  By controlling air/fuel ratio as such, several advantages are realized:

\begin{itemize}
\item{} Less heat energy lost out the exhaust (flue gases)
\item{} Reduced NO$_{x}$ emissions
\item{} Fuel conservation
\end{itemize}

Explain {\it why} oxygen trim control, properly implemented, provides these advantages.  Also, identify some hazards if an oxygen trim control system fails in such a way as to provide {\it insufficient} air to a combustion process.

\underbar{file i01827}
%(END_QUESTION)





%(BEGIN_ANSWER)

I will not give away the answer(s) here, but I will propose a ``thought experiment'' to help: imagine a combustion process where the flow rate of air into a burner system grossly exceeded the amount needed to burn the fuel.  Supposing the flame was not blown out by all this excess air, what would all that extra air do flowing through the furnace/boiler/firebox on its way out the exhaust stack?

\vskip 10pt

Insufficient air flow to a burner system is quite dangerous: it may result in an explosion!

%(END_ANSWER)





%(BEGIN_NOTES)


%INDEX% Control, strategies: oxygen trim control

%(END_NOTES)


