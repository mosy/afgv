
%(BEGIN_QUESTION)
% Copyright 2007, Tony R. Kuphaldt, released under the Creative Commons Attribution License (v 1.0)
% This means you may do almost anything with this work of mine, so long as you give me proper credit

One of the byproducts of high-temperature combustion using atmospheric air as the oxidizer is collectively referred to as {\it NO$_{x}$}: oxides of nitrogen.  Nitrogen gas, which comprises the vast majority of the air we breathe, reacts with oxygen at high temperatures to form the following compounds:

\begin{itemize}
\item{} NO
\item{} NO$_{2}$
\item{} N$_{2}$O$_{4}$
\item{} N$_{2}$O$_{5}$
\end{itemize}

These oxides of nitrogen can later form nitric acid in the atmosphere, and are a critical component of {\it smog}.  Thus, reducing NO$_{x}$ emissions in combustion processes is a significant environmental concern.

\vskip 10pt

Oxygen trim control, which minimizes the amount of air drawn in to a combustion process, helps to minimize NO$_{x}$ production: the less air brought into a fire, the less free nitrogen available, and the less oxygen left after combustion to combine with nitrogen to form NO$_{x}$.  However, there are combustion processes where oxygen trim is not practical.  Diesel and gas turbine engines are two such processes, because their air intake must be unrestricted for maximum thermodynamic efficiency (conversion of heat into mechanical energy).

Another method of NO$_{x}$ mitigation is {\it ammonia injection}, sometimes in the form of pure (anhydrous) ammonia gas, other times in the form of {\it urea} (liquid).  In either case, ammonia is sprayed into the exhaust stream of the engine prior to a catalyst, and the resulting chemical reaction reduces NO$_{x}$ molecules to harmless byproducts.  

\vskip 10pt

Determine this chemical reaction between ammonia (NH$_{3}$) and NO$_{x}$, and identify the harmless byproducts produced by it.  Then, determine what sort of control system might control the flow of ammonia into the exhaust of an engine.  In a vehicle engine control system, where a NO$_{x}$ sensor installed in the exhaust pipe might be cost-prohibitive, determine what other variables might predict (feedforward) the amount of ammonia flow necessary to neutralize all the exhaust NO$_{x}$.

\underbar{file i01828}
%(END_QUESTION)





%(BEGIN_ANSWER)

Harmless byproducts: free nitrogen and water vapor.

\vskip 10pt

Follow-up question: write a stoichiometrically balanced chemical reaction for NH$_{3}$ and NO$_{2}$, showing how the only byproducts will be nitrogen (N$_{2}$) and water vapor (H$_{2}$O).

%(END_ANSWER)





%(BEGIN_NOTES)

In the absence of a NO$_{x}$ analyzer, some relevant feedforward variables would be:

\begin{itemize}
\item{} Combustion temperature
\item{} Air flow
\item{} Combustion pressure
\item{} Fuel flow
\item{} Engine load
\end{itemize}

\vskip 10pt

Balanced chemical reaction for NH$_{3}$ and NO$_{2}$:

$$8\hbox{NH}_3 + 6\hbox{NO}_2 \to 7\hbox{N}_2 + 12\hbox{H}_2\hbox{O}$$

%INDEX% Control, strategies: NOx control by ammonia injection
%INDEX% Process: NOx emission mitigation

%(END_NOTES)


