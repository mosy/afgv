
%(BEGIN_QUESTION)
% Copyright 2011, Tony R. Kuphaldt, released under the Creative Commons Attribution License (v 1.0)
% This means you may do almost anything with this work of mine, so long as you give me proper credit

The IEEE 754-1985 standard for representing floating-point numbers uses 32 bits for single-precision numbers.  The first bit is the {\it sign}, the next eight bits are the {\it exponent}, and the last 23 bits are the {\it mantissa}:

% No blank lines allowed between lines of an \halign structure!
% I use comments (%) instead, so that TeX doesn't choke.

$$\vbox{\offinterlineskip
\halign{\strut
\vrule \quad\hfil # \ \hfil & 
\vrule \quad\hfil # \ \hfil & 
\vrule \quad\hfil # \ \hfil \vrule \cr
\noalign{\hrule}
%
% First row
{\bf Sign} & {\bf Exponent} ($E$) & {\bf Mantissa} ($m$) \cr
%
\noalign{\hrule}
%
% Another row
{\tt 0} & {\tt 00000000} & {\tt 00000000000000000000000} \cr
%
\noalign{\hrule}
} % End of \halign 
}$$ % End of \vbox

$$\pm 0.m \times 2^{E-127} \hskip 45pt \hbox{Single-precision, when exponent bits are all zero}$$

$$\pm 1.m \times 2^{E-127} \hskip 30pt \hbox{Single-precision, when exponent bits are not all zero}$$

Based on this standard, determine the values of the following single-precision IEEE 754 floating-point numbers:

\vskip 10pt

% No blank lines allowed between lines of an \halign structure!
% I use comments (%) instead, so that TeX doesn't choke.

$$\vbox{\offinterlineskip
\halign{\strut
\vrule \quad\hfil # \ \hfil & 
\vrule \quad\hfil # \ \hfil & 
\vrule \quad\hfil # \ \hfil \vrule \cr
\noalign{\hrule}
%
% First row
{\bf Sign} & {\bf Exponent} ($E$) & {\bf Mantissa} ($m$) \cr
%
\noalign{\hrule}
%
% Another row
{\tt 1} & {\tt 00101100} & {\tt 00001110000001100000000} \cr
%
\noalign{\hrule}
%
% Another row
{\tt 0} & {\tt 11001100} & {\tt 11010110000000000000000} \cr
%
\noalign{\hrule}
%
% Another row
{\tt 0} & {\tt 00000000} & {\tt 11111100000110001000000} \cr
%
\noalign{\hrule}
} % End of \halign 
}$$ % End of \vbox

Finally, how do you represent the number 1 ($1.0 \times 2^0$) in this floating-point format?

\underbar{file i01851}
%(END_QUESTION)





%(BEGIN_ANSWER)

$-${\tt 1.00001110000001100000000} $\times$ $2^{-83}$

\vskip 10pt

+{\tt 1.11010110000000000000000} $\times$ $2^{77}$

\vskip 10pt

+{\tt 0.11111100000110001000000} $\times$ $2^{-126}$
 
\vskip 10pt

The number 1 is represented as follows (literally, $1.0 \times 2^{127-127}$):

% No blank lines allowed between lines of an \halign structure!
% I use comments (%) instead, so that TeX doesn't choke.

$$\vbox{\offinterlineskip
\halign{\strut
\vrule \quad\hfil # \ \hfil & 
\vrule \quad\hfil # \ \hfil & 
\vrule \quad\hfil # \ \hfil \vrule \cr
\noalign{\hrule}
%
% First row
{\bf Sign} & {\bf Exponent} ($E$) & {\bf Mantissa} ($m$) \cr
%
\noalign{\hrule}
%
% Another row
{\tt 0} & {\tt 01111111} & {\tt 00000000000000000000000} \cr
%
\noalign{\hrule}
} % End of \halign 
}$$ % End of \vbox

%(END_ANSWER)





%(BEGIN_NOTES)


%INDEX%: Digital number format: IEEE floating-point numbers

%(END_NOTES)


