
%(BEGIN_QUESTION)
% Copyright 2007, Tony R. Kuphaldt, released under the Creative Commons Attribution License (v 1.0)
% This means you may do almost anything with this work of mine, so long as you give me proper credit

When attempting to contact potential employers, you will usually encounter automated telephone answering systems where you must leave a message for someone to listen to later.  One problem with voicemail messages is that they tend to sound unprofessional: vocalized pauses (``um''), meandering sentences, and missing information may leave a bad ``first impression'' with the person you are trying to contact.

One helpful way to overcome this is to have a pre-written paragraph ready to read aloud when leaving a voicemail message.  This will help you overcome anxiety when faced with having to leave a voicemail message (``What do I say?''), shorten the message by eliminating wasted time, and ensure no critical information is forgotten.

Write your own introductory paragraph for the purpose of leaving voicemail messages with prospective employers.  Be sure to include the following points:

\begin{itemize}
\item{} Your name, and how (and when!) you may be reached (e.g. phone number)
\item{} The purpose of contacting this company (to explore career options there)
\item{} (Optional) {\it a statement describing your interest in this company}
\item{} Conclude with a repeat of your contact information (e.g. phone number)
\item{} A friendly sign-off (e.g. ``Have a nice day'')
\end{itemize}

\vfil 

\underbar{file i01856}
\eject
%(END_QUESTION)





%(BEGIN_ANSWER)

Here is a sample voicemail message I made in April of 2007 for contacting multiple oil refineries in California (on behalf of students):

\vskip 10pt {\narrower \noindent \baselineskip5pt

Hello.  My name is Tony Kuphaldt, and I'm an instructor of Instrumentation and Control Technology at Bellingham Technical College in Washington state.  My telephone number is 360-752-8477.  

I have a class of 19 students ready to graduate this June with their two-year Associates degree in Instrumentation, and some of them are specifically looking to move to California.  I'm interested in exploring job opportunities with your company, for the benefit of these students.  We have placed graduates in Instrument/Electrical positions at other oil refineries in California, including BP in Carson (near LA), and Tesoro in Martinez (Bay Area).  I believe these students' skill set would be valuable to your operation, and I would like to see if this year's graduates could fulfill some of your technical staffing needs.

Again, my telephone number is 360-752-8477.  Please feel free to contact me about any open positions you may have at the moment, or that you predict you may have later in the year.

I look forward to discussing possibilities with you.  Thank you very much!

\par} \vskip 10pt

%(END_ANSWER)





%(BEGIN_NOTES)



%INDEX% Career, job search: voicemail message draft

%(END_NOTES)


