
%(BEGIN_QUESTION)
% Copyright 2010, Tony R. Kuphaldt, released under the Creative Commons Attribution License (v 1.0)
% This means you may do almost anything with this work of mine, so long as you give me proper credit

Read this fictitious account of an Instrumentation graduate who accepts a position with a company, then decides to leave very soon after:

\vskip 10pt {\narrower \noindent \baselineskip5pt

Gary graduates from the Instrumentation program of a two-year college, and immediately takes a factory maintenance position with Widgets, Inc.  During the interview, Gary is told that this job entails much more than just instrumentation work: Gary will have to perform janitorial duties, light carpentry, and clean windows in addition to performing maintenance on the automation systems at this facility.  Gary has all these skills, and is eager to get started in his new career, so he accepts.

After working a few months at Widgets, Inc., Gary realizes that instrumentation is a very small part of the job.  Most of his time spent working there is occupied with janitorial, carpentry, and window cleaning.  In those months, Gary has only had opportunity to use his instrumentation skills once or twice.  This is not what he expected when he took the job.

While looking through the local newspaper, Gary sees an opening for an instrument technician at a company where he knows the technicians do far more ``traditional'' instrumentation-type work.  He decides to apply for this other job, and eventually gets it.  When he gives two weeks notice to his boss at Widgets, Inc., his boss is quite upset.  Now his boss must put out another posting for a maintenance technician, and incur all the costs of interviewing again.

\par} \vskip 10pt

Explain how Gary could have avoided this unfortunate situation, and also identify some of the far-reaching consequences of his decision to leave Widgets, Inc. after just a few months.  Hint: the general rule of staying with your first employer for one to two years minimum may be helpful to you in brainstorming consequences, and also in determining how this situation could have been avoided.

\vfil

\underbar{file i01857}
\eject
%(END_QUESTION)





%(BEGIN_ANSWER)

This is a graded question -- no answers or hints given!

%(END_ANSWER)





%(BEGIN_NOTES)

Gary should have more carefully investigated the scope of the maintenance job, to the point of clarifying how much of his time (percentage-wise) he would spend dealing with instrumentation as opposed to garbage cans, drywall, and dirty windows.  He should have only accepted the job if he was comfortable with his employer's expectations.

\vskip 10pt

Far-reaching consequences include:

\begin{itemize}
\item{} Unnecessary financial burden on Widgets, Inc.
\item{} Bad publicity for Gary's alma mater
\item{} Bad publicity for Gary (future employers will question the brevity of his employment at Widgets as shown on his r\'esum\'e)
\end{itemize}

\vskip 10pt

A corollary to the rule of staying with your first employer for one to two years (minimum) is this: {\it are you comfortable enough with what you know about this job to commit to staying two years?}  If not, you either don't know enough about the job or are not a good fit for it.

%INDEX% Career, first job: tenure

%(END_NOTES)


