
%(BEGIN_QUESTION)
% Copyright 2007, Tony R. Kuphaldt, released under the Creative Commons Attribution License (v 1.0)
% This means you may do almost anything with this work of mine, so long as you give me proper credit

Under what conditions is it advisable to wear hearing protection?  What are some of the different types of hearing protection available to industrial workers?  Describe what {\it cumulative hearing loss} is.

\underbar{file i01864}
%(END_QUESTION)





%(BEGIN_ANSWER)

I'll let you research the answers to these questions!

%(END_ANSWER)





%(BEGIN_NOTES)

A general rule of thumb is that hearing protection should be worn if the ambient noise level is such that conversation requires a raised voice.  Companies should publish quantitative specifications for hearing protection noise levels.  The BP Cherry Point oil refinery, for instance, requires hearing protection for employees in areas with noise levels exceeding 85 dBA (on page 11 of the 2006 {\it Health, Safety, and Environmental Handbook}).

Hearing protection may take the form of disposable (foam) earplugs, reusable (rubber and silicone) earplugs, and earmuffs.  Some high-tech noise-cancellation headphones are also available using active technology to cancel noise waves before they reach the ears.

Hearing loss is cumulative over time, which is why wearing ear protection regularly in noisy environments is so important.  It is possible, in some cases, for hearing performance to recover if the person starts regularly wearing ear protection.  Generally, however, loss is permanent and cumulative.

%INDEX% Safety, hearing protection

%(END_NOTES)


