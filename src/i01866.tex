
%(BEGIN_QUESTION)
% Copyright 2007, Tony R. Kuphaldt, released under the Creative Commons Attribution License (v 1.0)
% This means you may do almost anything with this work of mine, so long as you give me proper credit

When most people think of electrical safety, the first thing they think of is prevention of {\it electric shock}.  While this is a very important aspect of electrical safety, there is another unique hazard of electricity that is just as important, and responsible for a significant percentage of electrical injuries in industry, as shock.  This hazard comes in two basic forms: {\it arc flash} and {\it arc blast}.

Explain what ``arc flash'' and ``arc blast'' are, what conditions lead to these hazards, and what distinguishes one from the other.  Also, identify the proper personal protective equipment (PPE) to guard against injury or death from either of these hazards.

\underbar{file i01866}
%(END_QUESTION)





%(BEGIN_ANSWER)

I'll let you research the answers to these questions!

%(END_ANSWER)





%(BEGIN_NOTES)

Arc flash and arc blast are hazards associated with electrical power systems, not low-voltage circuits (less than 30 volts).

{\it Arc flash} is the extremely high-temperature discharge produced by an electrical fault in air.  {\it Arc blast} is a high-pressure sound wave caused by a sudden arc fault.  Either are capable of severely burning human flesh.

Both are caused by the generation of an electric arc in air, usually resulting from the opening of a high-voltage, high-current switch contact.  Arc blast usually occurs when an arc stretches far enough to ``connect'' opposite poles in a power system, causing a direct phase-to-phase short in the air.

A common source of each is when an electrician actuates a motor starter (``contactor'') in a shorted motor or locked-rotor condition, especially if he or she has (unwisely and unsafely) removed the arc shields to better watch the contact operation.

\vskip 10pt

Arc flash temperatures may reach upwards of 20,000 degrees Kelvin!!  Note to your students that this temperature is roughly four times that of the Sun's surface!!!  In fact, no materials known are able to withstand such extreme temperatures without {\it vaporizing}.  It takes little imagination to realize what an arc flash can do to human skin.

The PPE recommended for arc flash/blast protection is very similar to the ``bunker gear'' worn by professional firefighters.  

%INDEX% Safety, electrical: arc blast
%INDEX% Safety, electrical: arc flash

%(END_NOTES)


