
%(BEGIN_QUESTION)
% Copyright 2007, Tony R. Kuphaldt, released under the Creative Commons Attribution License (v 1.0)
% This means you may do almost anything with this work of mine, so long as you give me proper credit

{\it Cardio-pulmonary resuscitation}, or {\it CPR}, is a very important procedure to know for certain types of emergencies.  Identify what ``CPR'' is, what it is used for, and what situations require it.

\underbar{file i01870}
%(END_QUESTION)





%(BEGIN_ANSWER)

I'll let you research the answers to this question!

%(END_ANSWER)





%(BEGIN_NOTES)

CPR is a procedure for manually actuating a victim's heart (cardio) and lungs (pulmonary) to ensure an adequate supply of oxygenated blood to their body's cells until professional medical personnel arrive to help the person.  It should be recognized that CPR is a ``maintenance'' procedure only, and generally is insufficient to revive a victim.

CPR is necessary when a person's heart has stopped beating, such as from electric shock or ventricular fibrillation.

\vskip 10pt

In January of 2007 the American Heart Association changed their recommended CPR procedure from 2 breaths and 15 compressions to 2 breaths and {\it 30} chest compressions.  The rationale was to reduce time wasted in the transition from compressions to breaths by reducing the number of transitions.

In March of 2008 the American Heart Association recommended chest compressions {\it only} as a viable alternative to the standard alternating breaths/compressions.  Text from the AHA's report ({\it Hands-Only (Compression-Only) Cardiopulmonary Resuscitation: A Call to Action for Bystander Response to Adults Who Experience Out-of-Hospital Sudden Cardiac Arrest. A Science Advisory for the Public From the American Heart Association Emergency Cardiovascular Care Committee Michael R. Sayre, Robert A. Berg, Diana M. Cave, Richard L. Page, Jerald Potts and Roger D. White Circulation published online Mar 31, 2008; DOI: 10.1161/CIRCULATIONAHA.107.189380}) states the new recommendations:

\vskip 10pt {\narrower \noindent \baselineskip5pt

\noindent
{\bf Recommendations and Call to Action}

\vskip 5pt

All victims of cardiac arrest should receive, at a minimum, high-quality chest compressions (i.e. chest compressions of adequate rate and depth with minimal interruptions). To support that goal and save more lives, the AHA ECC Committee recommends the following. 

\vskip 5pt

When an adult suddenly collapses, trained or untrained bystanders should -- at a minimum -- activate their community emergency medical response system (e.g. call 911) and provide high-quality chest compressions by pushing hard and fast in the center of the chest, minimizing interruptions (Class I).

\begin{itemize}
\item{} If a bystander is not trained in CPR, then the bystander should provide hands-only CPR (Class IIa). The rescuer should continue hands-only CPR until an automated external defibrillator arrives and is ready for use or EMS providers take over care of the victim. 
\vskip 5pt
\item{} If a bystander was previously trained in CPR and is confident in his or her ability to provide rescue breaths with minimal interruptions in chest compressions, then the bystander should provide either conventional CPR using a 30:2 compression-to-ventilation ratio (Class IIa) or hands-only CPR (Class IIa). The rescuer should continue CPR until an automated external defibrillator arrives and is ready for use or EMS providers take over care of the victim. 
\vskip 5pt
\item{} If the bystander was previously trained in CPR but is not confident in his or her ability to provide conventional CPR including high-quality chest compressions (i.e. compressions of adequate rate and depth with minimal interruptions) with rescue breaths, then the bystander should give hands-only CPR (Class IIa). The rescuer should continue hands-only CPR until an automated external defibrillator arrives and is ready for use or EMS providers take over the care of the victim.
\end{itemize}


\par} \vskip 10pt

%INDEX% Safety, first aid: CPR

%(END_NOTES)


