
%(BEGIN_QUESTION)
% Copyright 2010, Tony R. Kuphaldt, released under the Creative Commons Attribution License (v 1.0)
% This means you may do almost anything with this work of mine, so long as you give me proper credit

The most basic type of real-world input to a PLC is a {\it discrete} (on/off) input.  Each discrete input channel on a PLC is associated with a single bit in the PLC's memory.  Use the PLC programming software on your personal computer to ``connect'' to your PLC, then locate the facility within this software that allows you to monitor the status of your PLC's discrete input bits.

\vskip 10pt

Actuate the switches connected to your PLC's discrete input channels while watching the status of the respective bits.  Based on what you see, what does a ``1'' bit status signify, and what does a ``0'' bit status signify?

\vskip 20pt \vbox{\hrule \hbox{\strut \vrule{} {\bf Suggestions for Socratic discussion} \vrule} \hrule}

\begin{itemize}
\item{} How does your PLC address discrete input bits?  In other words, what is the convention it uses to label these bits, and distinguish them from each other?
\item{} How does the programming software for your PLC provide access to discrete input bit status?
\end{itemize}


\vfil 

\noindent
PLC comparison:

\begin{itemize}
\item{} \underbar{Allen-Bradley Logix 5000}: the {\it Controller Tags} folder (typically on the left-hand pane of the programming window set) lists all the tag names defined for the PLC project, allowing you to view the real-time status of them all.  Discrete inputs do not have specific input channel tag names, as tag names are user-defined in the Logix5000 PLC series.
\vskip 5pt
\item{} \underbar{Allen-Bradley PLC-5, SLC 500, and MicroLogix}: the {\it Data Files} listing (typically on the left-hand pane of the programming window set) lists all the data files within that PLC's memory.  Opening a data file window allows you to view the real-time status of these data points.  Discrete inputs are the {\tt I} file addresses (e.g. {\tt I:0/2}, {\tt I:3/5}, etc.).  The letter ``{\tt I}'' represents ``input,'' the first number represents the slot in which the input card is plugged, and the last number represents the bit within that data element (a 16-bit word) corresponding to the input card.
\vskip 5pt
\item{} \underbar{Siemens S7-200}: the {\it Status Chart} window allows the user to custom-configure a table showing the real-time values of multiple addresses within the PLC's memory.  Discrete inputs are the {\tt I} memory addresses (e.g. {\tt I0.1}, {\tt I1.5}, etc.).
\vskip 5pt
\item{} \underbar{Koyo (Automation Direct) DirectLogic and CLICK}: the {\it Data View} window allows the user to custom-configure a table showing the real-time values of multiple addresses within the PLC's memory.  Discrete inputs are the {\tt X} memory addresses (e.g. {\tt X1}, {\tt X2}, etc.).
\end{itemize}

\underbar{file i01876}
\eject
%(END_QUESTION)





%(BEGIN_ANSWER)


%(END_ANSWER)





%(BEGIN_NOTES)

%INDEX% PLC, exploratory question (discrete input status)
%INDEX% PLC, I/O: discrete input status

%(END_NOTES)


