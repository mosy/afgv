
%(BEGIN_QUESTION)
% Copyright 2010, Tony R. Kuphaldt, released under the Creative Commons Attribution License (v 1.0)
% This means you may do almost anything with this work of mine, so long as you give me proper credit

The most basic type of real-world output from a PLC is a {\it discrete} (on/off) output.  Each discrete output channel on a PLC is associated with a single bit in the PLC's memory.  Use the PLC programming software on your personal computer to ``connect'' to your PLC, then locate the facility within this software that allows you to monitor the status of your PLC's discrete output bits.

\vskip 10pt

Use the ``force'' utility in the programming software to force different output bits to a ``1'' status.  Based on what you see, what does a ``1'' bit status signify, and what does a ``0'' bit status signify?

\vskip 10pt

Is there any visual indication that bits have been forced from their normal state(s) in your PLC?  Note that ``forcing'' causes the PLC to output the values you specify, whether or not the programming in the PLC ``wants'' those bits to have those forced values!

\vskip 20pt \vbox{\hrule \hbox{\strut \vrule{} {\bf Suggestions for Socratic discussion} \vrule} \hrule}

\begin{itemize}
\item{} How does your PLC address discrete output bits?  In other words, what is the convention it uses to label these bits, and distinguish them from each other?
\item{} How does the programming software for your PLC provide access to discrete output bit status, and the ability to force them?
\item{} Why would anyone ever wish to force an output bit in a PLC, especially if doing so overrides the logic programmed into the PLC?
\end{itemize}

\vfil 

\noindent
PLC comparison:

\begin{itemize}
\item{} \underbar{Allen-Bradley Logix 5000}: forces may be applied to specific tag names by right-clicking on the tag (in the program listing) and selecting the ``Monitor'' option.  Discrete outputs do not have specific output channel tag names, as tag names are user-defined in the Logix5000 PLC series.
\vskip 5pt
\item{} \underbar{Allen-Bradley PLC-5, SLC 500, and MicroLogix}: the {\it Force Files} listing (typically on the left-hand pane of the programming window set) lists those data files within the PLC's memory liable to forcing by the user.  Opening a force file window allows you to view and set the real-time status of these bits.  Discrete outputs are the {\tt O} file addresses (e.g. {\tt O:0/7}, {\tt O:6/2}, etc.).  The letter ``{\tt O}'' represents ``output,'' the first number represents the slot in which the output card is plugged, and the last number represents the bit within that data element (a 16-bit word) corresponding to the output card.
\vskip 5pt
\item{} \underbar{Siemens S7-200}: the {\it Status Chart} window allows the user to custom-configure a table showing the real-time values of multiple addresses within the PLC's memory, and enabling the user to force the values of those addresses.  Discrete outputs are the {\tt Q} memory addresses (e.g. {\tt Q0.4}, {\tt Q1.2}, etc.).
\vskip 5pt
\item{} \underbar{Koyo (Automation Direct) DirectLogic and CLICK}: the {\it Override View} window allows the user to force variables within the PLC's memory.  Discrete outputs are the {\tt Y} memory addresses (e.g. {\tt Y1}, {\tt Y2}, etc.).
\end{itemize}

\underbar{file i01877}
\eject
%(END_QUESTION)





%(BEGIN_ANSWER)


%(END_ANSWER)





%(BEGIN_NOTES)

%INDEX% PLC, exploratory question (discrete output status)
%INDEX% PLC, I/O: discrete output status
%INDEX% PLC, I/O: forcing bits

%(END_NOTES)


