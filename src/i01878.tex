
%(BEGIN_QUESTION)
% Copyright 2010, Tony R. Kuphaldt, released under the Creative Commons Attribution License (v 1.0)
% This means you may do almost anything with this work of mine, so long as you give me proper credit

All PLCs provide ``special'' locations in memory holding values useful to the programmer, such as status warnings, real-time clock settings, calendar dates, etc.  Use the PLC programming software on your personal computer to ``connect'' to your PLC, then locate the facility within this software that allows you to explore some of these locations in memory.

\vskip 10pt

Identify some of the specific status-related and ``special'' memory locations in your PLC, and comment on those you think might be useful to use in the future.  Note the following memory types you may find associated with these addresses:

\begin{itemize}
\item{} Boolean (discrete) = simply on or off (1 or 0)
\vskip 10pt
\item{} Integer = whole-numbered values
\vskip 10pt
\item{} Floating-point (``real'') = fractional values
\end{itemize}

\vskip 20pt \vbox{\hrule \hbox{\strut \vrule{} {\bf Suggestions for Socratic discussion} \vrule} \hrule}

\begin{itemize}
\item{} Describe some of the ``special'' memory locations you find in your search, and comment on how some of them might be useful.
\item{} One of the useful bits provided by many PLCs is a ``flashing'' bit that simply turns on and off at regular intervals.  How many of these bits can you find in your PLC's memory, and how rapidly does each one oscillate?
\end{itemize}

\vfil 

\noindent
PLC comparison:

\begin{itemize}
\item{} \underbar{Allen-Bradley Logix 5000}: various ``system'' values are accessed via {\tt GSV} (Get System Value) and {\tt SSV} (Save System Value) instructions.
\vskip 5pt
\item{} \underbar{Allen-Bradley PLC-5, SLC 500, and MicroLogix}: the {\it Data Files} listing (typically on the left-hand pane of the programming window set) shows file number 2 as the ``Status'' file, in which you will find various system-related bits and registers.
\vskip 5pt
\item{} \underbar{Siemens S7-200}: the {\it Special Memory} registers contain various system-related bits and registers.  These are the {\tt SM} memory addresses (e.g. {\tt SM0.1}, {\tt SMB8}, {\tt SMW22}, etc.).
\vskip 5pt
\item{} \underbar{Koyo (Automation Direct) DirectLogic and CLICK}: the {\it Special} registers contain various system-related bits and registers.  These are the {\tt SP} memory addresses (e.g. {\tt SP1}, {\tt SP2}, {\tt SP3}, etc.) in the DirectLogic PLC series, and the {\tt SC} and {\tt SD} memory addresses in the CLICK PLC series. 
\end{itemize}

\underbar{file i01878}
\eject
%(END_QUESTION)





%(BEGIN_ANSWER)


%(END_ANSWER)





%(BEGIN_NOTES)

%INDEX% PLC, exploratory question (miscellaneous bits and words)

%(END_NOTES)


