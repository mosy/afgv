
%(BEGIN_QUESTION)
% Copyright 2007, Tony R. Kuphaldt, released under the Creative Commons Attribution License (v 1.0)
% This means you may do almost anything with this work of mine, so long as you give me proper credit

Suppose you are on the job site with an experienced technician, just about to remove a control valve from a piping system that has been taken out of service for you by operations personnel.  You take your wrenches and proceed to loosen the nearest flange bolt, but your work partner stops you.

``Loosen the bolts on the {\it far} side of the pipe first!'' the more experienced technician tells you.  So, you lean over the pipe and work on loosening the bolts on the opposite side of the pipe first.  Why is this a better idea than loosening the closest bolts (the bolts on your side of the pipe) first?  Why should it matter which bolts are loosened, if all the bolts must come out anyway to disconnect the valve from the pipes?

\underbar{file i01881}
%(END_QUESTION)





%(BEGIN_ANSWER)

If there is any residual fluid pressure inside the pipe, loosening the flange bolts on the opposite side of the pipe from where you're standing will ensure that any leak will spray {\it away} from you rather than toward you.

%(END_ANSWER)





%(BEGIN_NOTES)


%INDEX% Mechanics, fluid fittings: pipe flange
%INDEX% Safety, pipe flanges: loosening bolts

%(END_NOTES)


