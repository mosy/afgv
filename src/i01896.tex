
%(BEGIN_QUESTION)
% Copyright 2012, Tony R. Kuphaldt, released under the Creative Commons Attribution License (v 1.0)
% This means you may do almost anything with this work of mine, so long as you give me proper credit

Suppose a control valve with an equal-percentage trim characteristic has a maximum (wide-open) $C_v$ value of 38.5, and a $C_v$ value of 1.9 at 25\% open.  Complete the following table relating stem position values to $C_v$ values for this valve:

% No blank lines allowed between lines of an \halign structure!
% I use comments (%) instead, so that TeX doesn't choke.

$$\vbox{\offinterlineskip
\halign{\strut
\vrule \quad\hfil # \ \hfil & 
\vrule \quad\hfil # \ \hfil \vrule \cr
\noalign{\hrule}
%
% First row
{\bf Stem position} & $C_v$ \cr
%
\noalign{\hrule}
%
% Another row
0\% &  \cr
%
\noalign{\hrule}
%
% Another row
25\% & 1.9 \cr
%
\noalign{\hrule}
%
% Another row
50\% &  \cr
%
\noalign{\hrule}
%
% Another row
60\% &  \cr
%
\noalign{\hrule}
%
% Another row
70\% &  \cr
%
\noalign{\hrule}
%
% Another row
80\% &  \cr
%
\noalign{\hrule}
%
% Another row
90\% &  \cr
%
\noalign{\hrule}
%
% Another row
100\% & 38.5 \cr
%
\noalign{\hrule}
} % End of \halign 
}$$ % End of \vbox

\vskip 20pt \vbox{\hrule \hbox{\strut \vrule{} {\bf Suggestions for Socratic discussion} \vrule} \hrule}

\begin{itemize}
\item{} There is more than one way to algebraically manipulate the equal-percentage $C_v$ formula to solve for the rangeability coefficient $R$.  One method makes use of powers, while another uses logarithms.  Demonstrate both methods of algebraic manipulation.
\item{} Set up a computer spreadsheet to graph the $C_v$ of this control valve for a range of stem positions encompassing the positions specified in the table.
\item{} is your calculated $C_v$ value for a fully closed valve (stem position = 0\%) realistic?  Explain why or why not.
\end{itemize}

\underbar{file i01896}
%(END_QUESTION)





%(BEGIN_ANSWER)

Remember this formula for predicting the behavior of an equal-percentage control valve:

$$C_v = C_{vm} R^{(x - 1)}$$

\noindent
Where,

$C_v$ = Flow coefficient of control valve at stem position $x$

$C_{vm}$ = Flow coefficient of control valve while wide-open ($x$ = 100\%)

$x$ = Stem position, as a per unit value (ranging from 0 to 1) inclusive

$R$ = Rangeability coefficient of equal-percentage trim

\vskip 10pt

\noindent
{\bf Partial answer:}

% No blank lines allowed between lines of an \halign structure!
% I use comments (%) instead, so that TeX doesn't choke.

$$\vbox{\offinterlineskip
\halign{\strut
\vrule \quad\hfil # \ \hfil & 
\vrule \quad\hfil # \ \hfil \vrule \cr
\noalign{\hrule}
%
% First row
{\bf Stem position} & $C_v$ \cr
%
\noalign{\hrule}
%
% Another row
0\% & 0.697 \cr
%
\noalign{\hrule}
%
% Another row
25\% & 1.9 \cr
%
\noalign{\hrule}
%
% Another row
50\% &  \cr
%
\noalign{\hrule}
%
% Another row
60\% &  \cr
%
\noalign{\hrule}
%
% Another row
70\% & 11.6 \cr
%
\noalign{\hrule}
%
% Another row
80\% & 17.3 \cr
%
\noalign{\hrule}
%
% Another row
90\% &  \cr
%
\noalign{\hrule}
%
% Another row
100\% & 38.5 \cr
%
\noalign{\hrule}
} % End of \halign 
}$$ % End of \vbox

%(END_ANSWER)





%(BEGIN_NOTES)

This is an excellent opportunity to model the problem-solving technique of identifying all given values and any relevant formulae to solve a mathematical problem.  Since the question asks the student to solve for $C_v$ values for an {\it equal-percentage} valve, there is really only one formula that applies: $C_v = C_{vm} R^{(x - 1)}$

Examining this formula and the given information, we see we already know the value of $C_{vm}$ (38.5) and the value of $C_v$ at one position $x$ ($C_v$ = 1.9 at $x$ = 0.25).  The only remaining unknown is $R$, which means we can solve for $R$ algebraically by plugging in those known values of $C_v$, $C_{vm}$, and $x$.

\vskip 10pt

Solving for $R$ using logarithms:

$$C_v = C_{vm} R^{(x - 1)}$$

$${C_v \over C_{vm}} = R^{(x - 1)}$$

$$\log \left({C_v \over C_{vm}}\right) = \log \left( R^{(x - 1)} \right)$$

$$\log \left({C_v \over C_{vm}}\right) = (x - 1) \log R $$

$$\log R = {\log \left({C_v \over C_{vm}}\right) \over x - 1}$$

$$10^{\log R} = 10^{{\log \left({C_v \over C_{vm}}\right) \over x - 1}}$$

$$R = 10^{{\log \left({C_v \over C_{vm}}\right) \over x - 1}}$$

$$R = 10^{{\log \left({1.9 \over 38.5}\right) \over 0.25 - 1}} = 55.243$$

\vskip 10pt

Solving for $R$ using powers:

$$C_v = C_{vm} R^{(x - 1)}$$

$${C_v \over C_{vm}} = R^{(x - 1)}$$

$$\left({C_v \over C_{vm}}\right)^{1 \over x - 1} = \left(R^{(x - 1)}\right)^{1 \over x - 1}$$

$$R = \left({C_v \over C_{vm}}\right)^{1 \over x - 1}$$

$$R = \left({1.9 \over 38.5}\right)^{1 \over 0.25 - 1} = 55.243$$

\vskip 10pt

\filbreak

Completing the table:

$$C_v = (38.5) \left(55.243^{(x - 1)}\right)$$

% No blank lines allowed between lines of an \halign structure!
% I use comments (%) instead, so that TeX doesn't choke.

$$\vbox{\offinterlineskip
\halign{\strut
\vrule \quad\hfil # \ \hfil & 
\vrule \quad\hfil # \ \hfil \vrule \cr
\noalign{\hrule}
%
% First row
{\bf Stem position} & $C_v$ \cr
%
\noalign{\hrule}
%
% Another row
0\% & 0.697 \cr
%
\noalign{\hrule}
%
% Another row
25\% & 1.9 \cr
%
\noalign{\hrule}
%
% Another row
50\% & 5.2 \cr
%
\noalign{\hrule}
%
% Another row
60\% & 7.7 \cr
%
\noalign{\hrule}
%
% Another row
70\% & 11.6 \cr
%
\noalign{\hrule}
%
% Another row
80\% & 17.3 \cr
%
\noalign{\hrule}
%
% Another row
90\% & 25.8 \cr
%
\noalign{\hrule}
%
% Another row
100\% & 38.5 \cr
%
\noalign{\hrule}
} % End of \halign 
}$$ % End of \vbox

At a stem position of zero, the formula predicts $C_v = 0.697$.  In real life, the valve would be fully shut and therefore $C_v = 0$.  This is one of the idiosyncrasies of the equal-percentage formula: it fails to accurately predict the true $C_v$ of the control valve (0) at the fully-closed position.


%INDEX% Final Control Elements, valve: characterization

%(END_NOTES)


