
%(BEGIN_QUESTION)
% Copyright 2014, Tony R. Kuphaldt, released under the Creative Commons Attribution License (v 1.0)
% This means you may do almost anything with this work of mine, so long as you give me proper credit

On the first day of this course, we surveyed some of the many industries where measurement and automation are applied, as well as discussed the range and complexity of knowledge and skill areas critical for an instrument technician to master.  You were also left with a question to ponder: 

\vskip 10pt {\narrower \noindent \baselineskip5pt

{\it How is it possible for a program of instruction such as BTC's Instrumentation and Control Technology program to teach students these most difficult-to-learn knowledge and skill areas, and to do so in just two years?} 

\par} \vskip 10pt

After experiencing the structure and expectations of this course, how would you answer this question?

\underbar{file i01929}
%(END_QUESTION)





%(BEGIN_ANSWER)


%(END_ANSWER)





%(BEGIN_NOTES)

In 2009, the {\it Industrial Instrumentation and Control Technology Alliance} (IICTA) conducted a survey of 23 industrial instrumentation experts from across the United States to rank the relative importance of knowledge and skill areas listed on the {\it Texas Skill Standards Board} (TSSB) skill standard for ``Industrial Instrumentation and Controls Technician.''  The following is a list of knowledge/skill areas from this skill standard where ``critically important'' (the absolute highest importance) was the most popular vote of the experts surveyed, along with the percentage of experts voting the knowledge/skill area as ``critial'', and also a qualitative judgment of how difficult it is for someone to acquire the knowledge or skill if new:

% No blank lines allowed between lines of an \halign structure!
% I use comments (%) instead, so that TeX doesn't choke.

$$\vbox{\offinterlineskip
\halign{\strut
\vrule \quad\hfil # \ \hfil & 
\vrule \quad\hfil # \ \hfil & 
\vrule \quad\hfil # \ \hfil \vrule \cr
\noalign{\hrule}
%
% First row
{\bf Knowledge / Skill area} & {\bf \% vote} & {\bf Difficulty}\cr
%
\noalign{\hrule}
%
% Another row
Ability to learn new technology & 65\% & Hard \cr
%
\noalign{\hrule}
%
% Another row
Interpret and use instrument loop diagrams & 65\% & Moderate \cr
%
\noalign{\hrule}
%
% Another row
Configure and calibrate instruments & 65\% & Moderate \cr
%
\noalign{\hrule}
%
% Another row
Knowledge of test equipment & 61\% & Hard \cr
%
\noalign{\hrule}
%
% Another row
Interpret and use process and instrument diagrams & 57\% & Moderate \cr
%
\noalign{\hrule}
%
% Another row
Interpret and use instrument specification sheets & 52\% & Easy \cr
%
\noalign{\hrule}
%
% Another row
Knowledge of basic AC/DC electrical theory & 52\% & Hard \cr
%
\noalign{\hrule}
%
% Another row
Knowledge of basic mathematics & 48\% & Moderate \cr
%
\noalign{\hrule}
%
% Another row
Interpret and use electrical diagrams & 48\% & Moderate \cr
%
\noalign{\hrule}
%
% Another row
Interpret and use motor control logic diagrams & 43\% & Moderate \cr
%
\noalign{\hrule}
%
% Another row
Knowledge of system interactions (e.g. interlocks \& trips) & 43\% & Hard \cr
%
\noalign{\hrule}
%
% Another row
Knowledge of permits and area classifications & 43\% & Easy \cr
%
\noalign{\hrule}
%
% Another row
Understanding consequences of changes & 43\% & Hard \cr
%
\noalign{\hrule}
%
% Another row
Proper use of hand tools & 43\% & Moderate \cr
%
\noalign{\hrule}
%
% Another row
Knowledge of control schemes (e.g. ratio, cascade) & 39\% & Hard \cr
%
\noalign{\hrule}
%
% Another row
Proper tubing and wiring installation & 35\% & Moderate \cr
%
\noalign{\hrule}
%
% Another row
Motor control circuit knowledge & 30\% & Moderate \cr
%
\noalign{\hrule}
%
% Another row
Electrical wiring knowledge & 30\% & Moderate \cr
%
\noalign{\hrule}
} % End of \halign 
}$$ % End of \vbox

On January 24, 2013 the Washington State Workforce Training and Education Coordinating Board presented results of a survey gathering input from over 2800 employers state-wide.  One of the questions on this survey asked employers if they had experienced difficulty with entry-level employees demonstrating the following skills.  A partial listing of results is shown here:

% No blank lines allowed between lines of an \halign structure!
% I use comments (%) instead, so that TeX doesn't choke.

$$\vbox{\offinterlineskip
\halign{\strut
\vrule \quad\hfil # \ \hfil & 
\vrule \quad\hfil # \ \hfil \vrule \cr
\noalign{\hrule}
%
% First row
{\bf Knowledge / Skill area} & {\bf Percentage experiencing difficulty} \cr
%
\noalign{\hrule}
%
% Another row
Solve problems and make decisions & 50\% \cr
%
\noalign{\hrule}
%
% Another row
Take responsibility for learning & 43\% \cr
%
\noalign{\hrule}
%
% Another row
Listen actively & 40\% \cr
%
\noalign{\hrule}
%
% Another row
Observe critically & 38\% \cr
%
\noalign{\hrule}
%
% Another row
Read with understanding & 32\% \cr
%
\noalign{\hrule}
%
% Another row
Use math to solve problems and communicate & 31\% \cr
%
\noalign{\hrule}
} % End of \halign 
}$$ % End of \vbox

\vskip 10pt

\filbreak

In July and August of 2011, the Manufacturing Institute and Deloitte Development LLC worked together to adminster a ``Skills Gap study'' across a range of manufacturing industries in the United States.  Survey results were collected from 1123 respondents, with one of the survey questions asking {\it ``What are the most serious skill deficiencies in your current employees?''}.  The responses to this question are tabulated here:

% No blank lines allowed between lines of an \halign structure!
% I use comments (%) instead, so that TeX doesn't choke.

$$\vbox{\offinterlineskip
\halign{\strut
\vrule \quad\hfil # \ \hfil & 
\vrule \quad\hfil # \ \hfil \vrule \cr
\noalign{\hrule}
%
% First row
{\bf Knowledge / Skill area} & {\bf Percentage experiencing difficulty} \cr
%
\noalign{\hrule}
%
% Another row
Inadequate problem-solving skills & 52\% \cr
%
\noalign{\hrule}
%
% Another row
Lack of basic technical training & 43\% \cr
%
\noalign{\hrule}
%
% Another row
Inadequate ``soft skills'' (attendance, work ethic) & 40\% \cr
%
\noalign{\hrule}
%
% Another row
Inadequate computer skills & 36\% \cr
%
\noalign{\hrule}
%
% Another row
Inadequate math skills & 30\% \cr
%
\noalign{\hrule}
%
% Another row
Inadequate reading/writing/communication skills & 29\% \cr
%
\noalign{\hrule}
} % End of \halign 
}$$ % End of \vbox

\vskip 10pt

In December of 2001, the question ``What qualities should an Instrumentation graduate possess in order to excel in their profession?'' was posed to representatives on the Advisory Committee for BTC's Instrumentation program.  In addition to a firm knowledge of fundamentals (electronics, physics, mathematics, process control), one advisor in particular noted that ``self-direction and the ability to learn on your own'' was even more important.  Why do you suppose self-direction and self-teaching ability would be more highly valued than technical expertise in this career?


\vskip 20pt


\noindent
Here are some strategies applied in the INST courses to achieve these daunting goals:

\begin{itemize}
\item{} Place the student in a role of responsibility whenever possible:
\begin{itemize}

\item{} Responsible for researching answers to questions
\item{} Responsible for demonstrating knowledge and skill every day
\item{} Responsible for meeting deadlines for completion of work
\item{} Responsible for tracking assignments and grades
\end{itemize}
\item{} Demand mastery-level (100\%) competence on all essential objectives
\item{} Require independent and continuous learning on a daily basis
\item{} Emphasize reading as {\it the} single most important foundational skill for learning
\item{} Practice problem-solving every day 
\item{} Assess problem-solving ability in every exam
\end{itemize}

\noindent
{\bf In summary, the way to ensure graduates are self-directed learners with strong problem-solving ability is to require these very traits for successful course completion!}



%INDEX% Course organization, structure: meeting the challenge of the workplace

%(END_NOTES)


