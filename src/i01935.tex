
%(BEGIN_QUESTION)
% Copyright 2010, Tony R. Kuphaldt, released under the Creative Commons Attribution License (v 1.0)
% This means you may do almost anything with this work of mine, so long as you give me proper credit

Identify which of these are true statements:

\begin{itemize}
\item{$1.$} Between two points that are electrically common to each other, there is guaranteed to be zero (or nearly zero) voltage.
\vskip 10pt
\item{$2.$} If zero voltage is measured between two points, those points must be electrically common to each other.
\vskip 10pt
\item{$3.$} Between two points that are not electrically common to each other, there is guaranteed to be voltage.
\vskip 10pt
\item{$4.$} If voltage is measured between two points, those points cannot be electrically common to each other.
\end{itemize}

\underbar{file 01935}
%(END_QUESTION)





%(BEGIN_ANSWER)

Only two out of the four given statements are true:

\begin{itemize}
\item{$1.$} Between two points that are electrically common to each other, there is guaranteed to be zero (or nearly zero) voltage.
\item{$4.$} If voltage is measured between two points, those points cannot be electrically common to each other.
\end{itemize}

The fundamental concept of logic being applied here may be seen by examining the following statements -- representing the same logical pattern of electrical statements given at the beginning of this question:

\begin{itemize}
\item{$1.$} All rabbits are mammals.
\item{$2.$} All mammals are rabbits.
\item{$3.$} All non-rabbits are non-mammals.
\item{$4.$} All non-mammals are non-rabbits.
\end{itemize}

Clearly, only statements 1 and 4 are true.

%(END_ANSWER)





%(BEGIN_NOTES)

What we have here is an exercise in Aristotelian logic.  In either scenario (points in a circuit, or animals), statement 2 is the {\it converse} of statement 1, while statement 3 is the {\it inverse} and statement 4 is the {\it contrapositive}.  Only the contrapositive of a statement is guaranteed to share the same truth value as the original statement.

This is no esoteric exercise.  Rather, it is a hard-learned fact: many students mistakenly think that because there is guaranteed to be no voltage between electrically common points in a circuit, then the absence of voltage between two points must mean those two points are electrically common to each other!  This is not necessarily true, because situations exist where two points may not be electrically common, yet still have no voltage between them.  Electrical commonality is just {\it one way} that two points can have zero voltage between them, not the only way!

The contrapositive of this rule, however, is a valuable troubleshooting tool: if there is substantial voltage measured between two points in a circuit, then we know without a doubt that those two points are {\it not} electrically common to each other!

%INDEX% Electronics review: electrically common points
%INDEX% Troubleshooting review: electric circuits

%(END_NOTES)


