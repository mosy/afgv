
%(BEGIN_QUESTION)
% Copyright 2012, Tony R. Kuphaldt, released under the Creative Commons Attribution License (v 1.0)
% This means you may do almost anything with this work of mine, so long as you give me proper credit

Read Fluke's {\it Transmitter Calibration with the Fluke 750 Series Documenting Process Calibrator} application note (document 3792201B A-EN-N, August 2011) and answer the following questions:

\vskip 10pt

Identify the four different instrument calibration examples in this application note.  Are any of these similar to an instrument calibration you have done?

\vskip 10pt

Explain the advantage of using the ``Auto Test'' feature of the Fluke DPC to perform an instrument calibration, compared to performing a manual calibration test.

\vskip 10pt

Explain why the fourth calibration example in this application note cannot be done using the ``Auto Test'' capability of the Fluke DPC.  

\vskip 10pt

When manually providing the input values for the instrument under test as is the case in the last calibration example, is it necessary for you to exactly settle at each test point?  Explain why or why not.

\vskip 20pt \vbox{\hrule \hbox{\strut \vrule{} {\bf Suggestions for Socratic discussion} \vrule} \hrule}

\begin{itemize}
\item{} One of the Auto Test features not mentioned in this application note is the ability to perform an ``Up/Down'' test.  Explain why this feature might be useful for certain calibration procedures, specifically identifying the sort of calibration error it would be intended to detect.
\item{} Explain what would have to be different about the Fluke 750 series DPC in order for it to perform all the calibration tests described in automatic mode.  In other words, devise a solution to the ``manual-only'' test option given in the fourth calibration example.
\end{itemize}

\underbar{file i01940}
%(END_QUESTION)





%(BEGIN_ANSWER)


%(END_ANSWER)





%(BEGIN_NOTES)

The four different instrument calibration examples are: {\it thermocouple transmitter, RTD transmitter, I/P converter, and pressure transmitter (or P/I converter)}.

\vskip 10pt

The ``Auto Test'' feature greatly speeds up calibrations by automatically stepping through a series of input stimulus levels and automatically recording the corresponding output signal levels (and error in percent of span).  If any error during the test exceeds the pre-set tolerance (the default is $\pm$ 0.25\% for the Fluke 750 series), the error value is shown in inverse coloring (i.e. white numbers on a black background).

\vskip 10pt

The fourth calibration shown (a pressure transmitter) cannot be done using ``Auto Test'' because the Fluke DPC has no means to automatically generate a series of air pressures to stimulate the transmitter.  Instead, the technician must manually generate these pressures using a hand pump.

\vskip 10pt

When performing a ``manual test'' you need only settle the instrument's input stimulus at some ``reasonably close'' value to the desired test point (step \#9 on page 7 explains this in detail).  The DPC is smart enough to calculate what the transmitter's output signal {\it should be} at your reasonably close pressure, so that your pressure sourcing deviation (from ideal) does not get counted as transmitter error.







%INDEX% Calibration: Documenting Process Calibrator (Fluke 744/754)

%(END_NOTES)


