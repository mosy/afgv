
%(BEGIN_QUESTION)
% Copyright 2012, Tony R. Kuphaldt, released under the Creative Commons Attribution License (v 1.0)
% This means you may do almost anything with this work of mine, so long as you give me proper credit

Read Fluke's {\it Calibrating Pressure Switches with a DPC} application note (document 2069058B A-EN-N, July 2011) and answer the following questions:

\vskip 10pt

Define ``deadband'' as used in this document with reference to a pressure switch, and explain why this is an important parameter for a process switch. 

\vskip 10pt

Explain why it is important to tell the DPC whether the setpoint type is ``low'' or ``high''.

\vskip 10pt

Explain why a pressure switch calibration check cannot be done using the ``Auto Test'' capability of the Fluke DPC, but rather must be done using the ``Manual Test'' feature.  


\vskip 20pt \vbox{\hrule \hbox{\strut \vrule{} {\bf Suggestions for Socratic discussion} \vrule} \hrule}

\begin{itemize}
\item{} Explain why this document advises you to repeatedly cycle the test pressure past the set and reset switch states, instead of traversing those values just once.
\item{} At the conclusion of the described test, the Fluke DPC displays both {\it setpoint error} and {\it deadband error} figures.  Explain the meaning of each.
\item{} Explain what would have to be different about the Fluke 750 series DPC in order for it to perform all the calibration tests described in automatic mode.  In other words, devise a solution to the ``manual-only'' test option given in the fourth calibration example.
\end{itemize}

\underbar{file i01941}
%(END_QUESTION)





%(BEGIN_ANSWER)


%(END_ANSWER)





%(BEGIN_NOTES)

``Deadband'' for a pressure switch is defined as the measured difference in applied pressure when the switch changes from set to reset states (page 1).

\vskip 10pt

It is important to specify the setpoint type as either ``high'' or ``low'' because this tells the Fluke DPC whether to check the switch's action on an increasing pressure or on a decreasing pressure, respectively (page 2).

\vskip 10pt

A pressure switch calibration cannot be done using ``auto test'' because the Fluke DPC has no means to automatically generate a series of air pressures to stimulate the transmitter.  Instead, the technician must manually generate these pressures using a hand pump.










\vskip 20pt \vbox{\hrule \hbox{\strut \vrule{} {\bf Suggestions for Socratic discussion} \vrule} \hrule}

\begin{itemize}
\item{} Suppose a high pressure switch (PSH) with a desired trip value of 35 PSI is tested using a Fluke 754 calibrator, with the calibrator set for ``high'' setpoint type and ``open'' set state.  Is this switch {\it normally open} or {\it normally closed}?
\vskip 5pt
\item{} Suppose we wish to test a low pressure switch (PSL) with a desired trip value of 24 PSI and NC contacts using a Fluke model 754 calibrator.  Determine the setpoint type (high or low) and the set state (open or short) for this test.
\vskip 5pt
\item{} Suppose we wish to test a low pressure switch (PSL) with a desired trip value of 15 PSI and NO contacts using a Fluke model 754 calibrator.  Determine the setpoint type (high or low) and the set state (open or short) for this test.
\vskip 5pt
\item{} Suppose we wish to test a high pressure switch (PSH) with a desired trip value of 100 PSI and NC contacts using a Fluke model 754 calibrator.  Determine the setpoint type (high or low) and the set state (open or short) for this test.
\vskip 5pt
\item{} Suppose we wish to test a high pressure switch (PSH) with a desired trip value of 250 PSI and NO contacts using a Fluke model 754 calibrator.  Determine the setpoint type (high or low) and the set state (open or short) for this test.
\end{itemize}










\vfil \eject

\noindent
{\bf Prep Quiz:}

The definition of ``deadband'' with reference to a pressure switch is:

\begin{itemize}
\item{} The measured difference between a switch's set pressure versus reset pressure
\vskip 5pt 
\item{} A specific type of electrical contact designed to minimze sparking
\vskip 5pt 
\item{} The maximum amount of pressure the switch can withstand without failing (dying)
\vskip 5pt 
\item{} An accessory for a pressure switch to absorb mechnical shock and vibration
\vskip 5pt 
\item{} A spring-steel band the technician tightens or loosens to adjust the setpoint
\vskip 5pt 
\item{} The ``dead'' weight of the pressure switch without any wires or tubes connected
\end{itemize}



%INDEX% Calibration: Documenting Process Calibrator (Fluke 744/754)

%(END_NOTES)


