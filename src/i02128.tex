
%(BEGIN_QUESTION)
% Copyright 2012, Tony R. Kuphaldt, released under the Creative Commons Attribution License (v 1.0)
% This means you may do almost anything with this work of mine, so long as you give me proper credit

Decoding a Manchester-encoded waveform is challenging to many students, because it's not clear at first how to differentiate ``real'' signal transitions (i.e. those pulse edges representing actual binary data bits) from transitions that are merely ``reversals'' (i.e. those pulse edges that are simply set-up for the next ``real'' data pulse).  Here, I will show you a practical problem-solving method to gain a deeper understanding.

We will apply the problem-solving technique of {\it working backwards} to understand the concept better.  If the part we're struggling with is how to convert a waveform into a series of bits, then we'll turn the problem backwards by starting with a known series of bits and working to convert that series of bits into a waveform:

\vskip 10pt

Here we see a series of bits aligned with a set of grey lines which we know will be pulse edges:

$$\includegraphics[width=15.5cm]{i02128x01.eps}$$

Begin by tracing the rising- and falling-edge pulses for each bit, following the standard of a rising edge representing a ``1'' bit and a falling edge representing a ``0'' bit.  Feel free to draw small arrows distinguishing the rising versus falling transitions.  Then, figure out how to connect these rising and falling edges together to form an actual pulse waveform.

It should quickly become apparent to you where ``reversals'' are necessary in order to ``set up'' properly for the next data pulse.

\vskip 10pt

After you have done this, cover up the ``1'' and ``0'' bits so you can only see the waveform you've skeched.  Erase any arrow-heads you might have sketched, so there is nothing visible to you except a clean pulse waveform.  Now, explain to yourself how you would interpret this waveform to know which pulse edges represented real data as opposed to reversals.


\underbar{file i02128}
%(END_QUESTION)





%(BEGIN_ANSWER)

The ``clean'' waveform, with no bit markings:

$$\includegraphics[width=15.5cm]{i02128x02.eps}$$

Hint: {\it all} the pulse edges representing ``real'' data bits are evenly spaced!  {\it All} ``reversal'' pulse edges are out-of-step with the regular pattern of data bit pulse edges.

%(END_ANSWER)





%(BEGIN_NOTES)


%INDEX% Electronics review: Manchester encoding

%(END_NOTES)


