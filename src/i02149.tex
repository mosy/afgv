
%(BEGIN_QUESTION)
% Copyright 2014, Tony R. Kuphaldt, released under the Creative Commons Attribution License (v 1.0)
% This means you may do almost anything with this work of mine, so long as you give me proper credit

Suppose an archer's longbow has a draw of 0.8 meters and a holding force of 340 newtons.  Assuming a perfectly linear force-draw function, calculate the maximum speed of a 17 gram arrow shot by this bow.

\vskip 20pt \vbox{\hrule \hbox{\strut \vrule{} {\bf Suggestions for Socratic discussion} \vrule} \hrule}

\begin{itemize}
\item{} A good problem-solving strategy for quantitative problems such as this one is to first identify what it is you need to solve, then identify all relevant data, identify all units of measurement, identify any general principles or formulae linking the given information to the solution, and finally identify any ``missing pieces'' to a solution.  Demonstrate how to do all these steps on this problem.
\end{itemize}

\underbar{file i02149}
%(END_QUESTION)





%(BEGIN_ANSWER)

Drawing this bow stores potential energy in it.  Releasing the bowstring transfers that potential energy to the arrow, where it becomes kinetic energy.  Thus, this is really a work/energy problem.  If we calculate the amount of work done drawing this bow, we will know how much kinetic energy the arrow possesses when it is shot from the bow, and from that energy value we may calculate the arrow's velocity.

\vskip 10pt

First, calculating the work done drawing the bow:

$$W = \int_{0 \hbox{ m}}^{0.8 \hbox{ m}} F \> dx$$

Since we know this longbow has a perfectly linear force-draw function, we know its graphical plot will look like a perfect right-triangle, with force starting at 0 newtons at 0 draw, linearly climbing to 340 newtons at a draw of 0.8 meters.  Thus, the work done (i.e. the area under this force-draw triangle) will be one half times the peak height (340 newtons) times the peak draw (0.8 meters):

$$W = \left({1 \over 2}\right) (340 \hbox{ N}) (0.8 \hbox{ m}) = 136 \hbox{ Nm}$$

\vskip 10pt

Now that we know the fully-drawn bow stores 136 newton-meters of potential energy at full draw, we may set this quantity equal to the kinetic energy of the flying arrow and solve for velocity:

$$W = E_k = 136 \hbox{ Nm}$$

$$E_k = {1 \over 2} m v^2$$

$$2 E_k = m v^2$$

$${2 E_k \over m} =v^2$$

$$v = \sqrt{ 2 E_k \over m}$$

$$v = \sqrt{ (2) (136 \hbox{ Nm}) \over 0.017 \hbox{ kg}}$$

$$v = 126 \hbox{ m/s}$$

This calculated velocity of 126 meters per second translates to approximately 411 feet per second.  It is the speed of the arrow as it leaves the bow, assuming no energy lost in friction (i.e. 100\% of the bow's potential energy gets transferred to the arrow).

%(END_ANSWER)





%(BEGIN_NOTES)


%INDEX% Mathematics, calculus: integral (work)

%(END_NOTES)


