
%(BEGIN_QUESTION)
% Copyright 2007, Tony R. Kuphaldt, released under the Creative Commons Attribution License (v 1.0)
% This means you may do almost anything with this work of mine, so long as you give me proper credit

The most important assignment for this course is to conduct an informational interview over the telephone of a company {\it not} in your local area.  Please use a land-line telephone for this (not a mobile phone), as the poor voice quality of most cell phone connections is viewed as unprofessional.

People appropriate to interview when you are completely new to the company include Human Resources personnel, instrument technicians, and instrument shop supervisors (in that order of preference).  Your first point of contact with a new company should be the Human Resources department, not the instrument shop!

Informational interviews of a local company are permitted only if the interview is performed in person (not over the telephone) and a business card is attached to this question upon submission by Day 5.  If this is your plan, please be sure to dress and behave appropriately for the interview: treat it as if you are being interviewed for a job!  Bear in mind that this informational interview may very well open a door for your employment at this company, so treat it with the sense of importance it deserves!

\vskip 10pt

Document several of the questions you asked during your informational interview, and the responses you received to each one:

\vskip 10pt

\begin{itemize}
\item{} 
\vskip 60pt
\item{} 
\vskip 60pt
\item{} 
\vskip 60pt
\item{} 
\vskip 60pt
\item{} 
\vskip 60pt
\item{} 
\end{itemize}

\vfil

\underbar{file i02160}
\eject
%(END_QUESTION)





%(BEGIN_ANSWER)

There is no answer for me to give on this sort of question!

%(END_ANSWER)





%(BEGIN_NOTES)

Sources to use when searching for prospective employers:

\begin{itemize}
\item{} Washington Manufacturer's Register (BTC library has two printed copies)
\item{} Thomas Register ({\tt http://www.thomasnet.com})
\item{} Visit ``Chamber of Commerce'' (or website for Chamber) for any large city, then search employers.
\item{} ProQuest online database -- ABI/INFORM Trade and Industry link
\item{} LexisNexis -- Business link -- Company Profiles link
\item{} Bellingham Public Library -- indices and database link -- business/company resource center link
\end{itemize}

\vskip 10pt

Industry sectors to search:

\begin{itemize}
\item{} Oil and Gas
\item{} Pharmaceutical / Biopharmaceutical
\item{} Aerospace
\item{} Power generation and utilities
\item{} Pulp and paper
\item{} Food manufacturing (large-scale)
\item{} Air separation
\item{} Mining
\item{} Building automation
\item{} Contract calibration services (e.g. ESC Automation)
\item{} City Public Works
\item{} Public Utility Departments (PUDs)
\item{} Instrumentation contractors
\item{} Contract engineering firms (VECO, Parsons, Bechtel, Halliburton)
\item{} Instrument manufacturers (Emerson/Rosemount/Fisher, Foxboro, Honeywell, etc.)
\item{} Manufacturing of all kinds
\end{itemize}

%INDEX% Career, job search: informational interview

%(END_NOTES)


