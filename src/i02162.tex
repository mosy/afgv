
%(BEGIN_QUESTION)
% Copyright 2011, Tony R. Kuphaldt, released under the Creative Commons Attribution License (v 1.0)
% This means you may do almost anything with this work of mine, so long as you give me proper credit

Read the section entitled ``Input Polarity and Range'' (pages 2-2 through 2-5) of the National Instruments ``DAQ E Series User Manual'' (document 370503K-01) and answer the following questions:

\vskip 10pt

Based on the resolution (``precision'') you see listed in Table 2-1 for the NI 6020E data acquisition module, how many bits does its ADC use?

\vskip 10pt

Based on the resolution (``precision'') you see listed in Table 2-1 for the NI 6052E data acquisition module, how many bits does its ADC use?

\vskip 10pt

Describe the difference between {\it unipolar} and {\it bipolar} analog ranges. 

\vskip 10pt

Page 2-21 contains a table (2-6) of diagrams showing how to connect various analog signal sources to a DAQ.  Explain why the one connection scheme (middle row, right column) is not recommended.

\vskip 20pt \vbox{\hrule \hbox{\strut \vrule{} {\bf Suggestions for Socratic discussion} \vrule} \hrule}

\begin{itemize}
\item{} Why do you suppose the E-series DAQ modules use a PGIA (Programmable Gain Instrumentation Amplifier) as a front-end to the ADC (Analog-to-Digital Converter)?  Why not use a fixed-gain instrumentation amplifier instead?
\item{} Note the two different ground symbols used in the ``Not Recommended'' diagram in table 2-6.  What specifically are the authors trying to convey to you by using two different ground symbols?
\item{} Explain the necessity of $R_{ext}$ in some of the diagrams shown in table 2-6.
\item{} Identify the purpose for each function in the block diagram of figure 2-1.
\end{itemize}

\underbar{file i02162}
%(END_QUESTION)





%(BEGIN_ANSWER)

The NI 6020E DAQ uses 12-bit (4096 count) conversion.  The NI 6052E uses 16-bit (65,536 count) conversion.

%(END_ANSWER)





%(BEGIN_NOTES)

Table 2-1 on page 2-3:
\item{} NI 6020E DAQ -- (10 V / 2.44 mV = 4098 counts = 12 bits)
\item{} NI 6052E DAQ -- (10 V / 153 $\mu$V = 65359 counts = 16 bits)
\end{itemize}

\vskip 10pt

A ``unipolar'' range only accepts positive voltage signals.  A ``bipolar'' range accepts both positive and negative polarities.

\vskip 10pt

Ground loops are to be avoided at all cost, which is the problem experienced connecting a ground-referenced signal source to a single-ended (ground-referenced) DAQ.  The two different ground symbols here reinforce the idea that although both points are ``ground,'' there might be some potential difference existing in the earth between those two points.

%INDEX% Reading assignment: National Instruments E-Series DAQ user manual

%(END_NOTES)


