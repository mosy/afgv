
%(BEGIN_QUESTION)
% Copyright 2003, Tony R. Kuphaldt, released under the Creative Commons Attribution License (v 1.0)
% This means you may do almost anything with this work of mine, so long as you give me proper credit

A numeration system often used as a ``shorthand'' way of writing large binary numbers is the {\it octal}, or base-eight, system.  Based on what you know of place-weighted numeration systems, describe how many valid ciphers exist in the octal system, and the respective ``weights'' of each place in an octal number.

\vskip 10pt

Also, perform the following conversions:

\begin{itemize}
\item{} $35_{8}$ into decimal:
\vskip 5pt
\item{} $16_{10}$ into octal:
\vskip 5pt
\item{} $110010_{2}$ into octal:
\vskip 5pt
\item{} $51_{8}$ into binary:
\end{itemize}

\vskip 20pt \vbox{\hrule \hbox{\strut \vrule{} {\bf Suggestions for Socratic discussion} \vrule} \hrule}

\begin{itemize}
\item{} If binary is the ``natural language'' of digital electronic circuits, why do we even bother with other numeration systems such as hex and octal?
\item{} Why is octal considered a ``shorthand'' notation for binary numbers?
\end{itemize}

\underbar{file i02166}
%(END_QUESTION)





%(BEGIN_ANSWER)

There are only eight valid ciphers in the octal system (0, 1, 2, 3, 4, 5, 6, and 7), with each successive place carrying eight times the ``weight'' of the place before it.

\begin{itemize}
\item{} $35_{8}$ into decimal: $29_{10}$
\vskip 5pt
\item{} $16_{10}$ into octal: $20_{8}$
\vskip 5pt
\item{} $110010_{2}$ into octal: $62_{8}$
\vskip 5pt
\item{} $51_{8}$ into binary: $101001_2$
\end{itemize}

%(END_ANSWER)





%(BEGIN_NOTES)

There are many references from which students may learn to perform these conversions.  You assistance should be minimal, as these procedures are simple to comprehend and easy to find.

\vfil \eject

\noindent
{\bf Prep Quiz:}

Convert the decimal number {\tt 183} into hexadecimal:

\begin{itemize}
\item{} F3
\vskip 5pt 
\item{} 267
\vskip 5pt 
\item{} B7
\vskip 5pt 
\item{} 1B
\vskip 5pt 
\item{} A3
\vskip 5pt 
\item{} 387
\end{itemize}


%INDEX% Electronics review: conversions to and from octal
%INDEX% Electronics review: numeration system conversions

%(END_NOTES)


