
%(BEGIN_QUESTION)
% Copyright 2003, Tony R. Kuphaldt, released under the Creative Commons Attribution License (v 1.0)
% This means you may do almost anything with this work of mine, so long as you give me proper credit

A numeration system often used as a ``shorthand'' way of writing large binary numbers is the {\it hexadecimal}, or base-sixteen, system.

Based on what you know of place-weighted numeration systems, describe how many valid ciphers exist in the hexadecimal system, and the respective ``weights'' of each place in a hexadecimal number.

Also, perform the following conversions:

\begin{itemize}
\item{} $35_{16}$ into decimal:
\vskip 5pt
\item{} $34_{10}$ into hexadecimal:
\vskip 5pt
\item{} $11100010_{2}$ into hexadecimal:
\vskip 5pt
\item{} $93_{16}$ into binary:
\end{itemize}

\underbar{file i02167}
%(END_QUESTION)





%(BEGIN_ANSWER)

There are sixteen valid ciphers in the hexadecimal system (0, 1, 2, 3, 4, 5, 6, 7, 8, 9, A, B, C, D, E, and F), with each successive place carrying sixteen times the ``weight'' of the place before it.

\begin{itemize}
\item{} $35_{16}$ into decimal: $53_{10}$
\vskip 5pt
\item{} $34_{10}$ into hexadecimal: $22_{16}$
\vskip 5pt
\item{} $11100010_{2}$ into hexadecimal: E2$_{16}$
\vskip 5pt
\item{} $93_{16}$ into binary: $10010011_2$
\end{itemize}

\vskip 10pt

Follow-up question: why is hexadecimal considered a ``shorthand'' notation for binary numbers?

%(END_ANSWER)





%(BEGIN_NOTES)

There are many references from which students may learn to perform these conversions.  You assistance should be minimal, as these procedures are simple to comprehend and easy to find.


%INDEX% Electronics review: conversions to and from hexadecimal
%INDEX% Electronics review: numeration system conversions

%(END_NOTES)


