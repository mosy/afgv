
%(BEGIN_QUESTION)
% Copyright 2003, Tony R. Kuphaldt, released under the Creative Commons Attribution License (v 1.0)
% This means you may do almost anything with this work of mine, so long as you give me proper credit

In digital computer systems, binary numbers are often represented by a fixed number of bits, such as 8, or 16, or 32.  Such bit groupings are often given special names, because they are so common in digital systems:

\begin{itemize}
\item{} byte
\item{} nybble
\item{} word
\end{itemize}

How many binary bits is represented by each of the above terms?

\vskip 10pt

And, for those looking for more challenge, try defining these terms:

\begin{itemize}
\item{} nickle
\item{} deckle
\item{} chawmp
\item{} playte
\item{} dynner
\end{itemize}

\underbar{file i02171}
%(END_QUESTION)





%(BEGIN_ANSWER)

\begin{itemize}
\item{} byte = 8 bits
\item{} nybble = 4 bits
\item{} word = {\it depends on the system}
\end{itemize}

The term ``word,'' is often used to represent 16 bits, but it really depends on the particular system being spoken of.  A binary ``word'' is more accurately defined as the default width of a binary bit grouping in a digital system.

\vskip 10pt

Follow-up question: what binary grouping corresponds to a single hexadecimal character?

%(END_ANSWER)





%(BEGIN_NOTES)

Definitions taken from {\it The New Hacker's Dictionary}, available on Internet terminals everywhere.

%INDEX% Electronics review: bit, defined
%INDEX% Electronics review: byte, defined
%INDEX% Electronics review: nybble, defined
%INDEX% Electronics review: word, defined

%(END_NOTES)


