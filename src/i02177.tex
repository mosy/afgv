
%(BEGIN_QUESTION)
% Copyright 2007, Tony R. Kuphaldt, released under the Creative Commons Attribution License (v 1.0)
% This means you may do almost anything with this work of mine, so long as you give me proper credit

Encode the following text message in ASCII format (expressing your codes in hexadecimal characters):

\vskip 10pt

\centerline{\tt Zero \& Span} 

\vskip 50pt

Then, encode the exact same message with {\it odd parity}.  Since the parity bit will be an eighth bit in the message (ASCII codes only requiring 7 bits each), include the parity bit as the MSB when you express each code in hexedecimal form.

\vskip 50pt

\underbar{file i02177}
%(END_QUESTION)





%(BEGIN_ANSWER)

\noindent
ASCII codes (no parity):

\vskip 10pt

5A 65 72 6F 20 26 20 53 70 61 6E

\vskip 30pt

\noindent
ASCII codes (with odd parity as the MSB):

\vskip 10pt

DA E5 F2 EF 20 26 20 D3 70 61 6E

%(END_ANSWER)





%(BEGIN_NOTES)


%INDEX% Electronics review: encoding ASCII (with parity)

%(END_NOTES)


