
%(BEGIN_QUESTION)
% Copyright 2011, Tony R. Kuphaldt, released under the Creative Commons Attribution License (v 1.0)
% This means you may do almost anything with this work of mine, so long as you give me proper credit

Read and outline the ``Antennas'' section of the ``AC Electricity'' chapter in your {\it Lessons In Industrial Instrumentation} textbook.  Note the page numbers where important illustrations, photographs, equations, tables, and other relevant details are found.  Prepare to thoughtfully discuss with your instructor and classmates the concepts and examples explored in this reading.

\underbar{file i02183}
%(END_QUESTION)





%(BEGIN_ANSWER)


%(END_ANSWER)





%(BEGIN_NOTES)

Tank (LC) circuits resonate.  So do unterminated transmission lines, but with multiple resonance frequencies (not just one): a fundamental frequency plus harmonics (integer multiples of the fundamental).  Tubes used in musical instruments do the same thing (in audio tones).

\vskip 10pt

A splayed-wire end to a cable radiates its resonating energy away in the form of electromagnetic waves because neither its electric nor its magnetic fields are contained as is the case in a typical tank circuit with discrete $L$ and $C$ components.  Electromagnetic waves were predicted by Maxwell in 1873, and later demonstrated by Hertz in 1887 (sadly, after Maxwell's death).

\vskip 10pt

Symmetrical wire antennas fundamentally resonate at half the wavelength of the signal ($\lambda = {v \over f}$), and at any harmonic of that fundamental frequency.  Quarter-wave antennas fundamentally resonate at one-quarter the signal wavelength.

\vskip 10pt

Transmitting and receiving antenna conductors should be parallel to each other for maximum coupling.  Whip antennas are omnidirectional; Yagi antennas are directional.  Dish antennas are highly directional.









\vskip 20pt \vbox{\hrule \hbox{\strut \vrule{} {\bf Suggestions for Socratic discussion} \vrule} \hrule}

\begin{itemize}
\item{} Identify how to alter the resonant frequency of a tank circuit.
\item{} Identify how to alter the resonant frequency of a transmission line.
\item{} Identify how to alter the resonant frequency of an antenna.
\item{} Explain why a transmission line has multiple frequencies of resonance, while a tank circuit only has one resonant frequency.
\item{} Examine the schematic diagram for Hertz's experimental apparatus, and explain how it works.
\item{} Examine how you could perform a modern version of Hertz's experiment using modern lab equipment.
\item{} Contrast the advice to keep conductors {\it perpendicular} to each other when minimum ``coupling'' is desired between them (e.g. wiring inside of an industrial control panel), versus the advice to keep transmitting and receiving antenna conductors {\it parallel} to each other.  Identify the fundamental principles at work in both these applications.
\item{} Identify applications for a {\it directional} antenna.
\end{itemize}

%INDEX% Reading assignment: Lessons In Industrial Instrumentation, AC Electricity (antennas -- intro)

%(END_NOTES)


