
%(BEGIN_QUESTION)
% Copyright 2007, Tony R. Kuphaldt, released under the Creative Commons Attribution License (v 1.0)
% This means you may do almost anything with this work of mine, so long as you give me proper credit

In data communications terminology, a binary ``1'' state is called a {\it mark} while a binary ``0'' state is called a {\it space}.  Research the minimum allowable voltage levels constituting a ``mark'' and a ``space'' in each of the following serial communication standards, and then explain why the standards differ so much in this respect:

% No blank lines allowed between lines of an \halign structure!
% I use comments (%) instead, so that TeX doesn't choke.

$$\vbox{\offinterlineskip
\halign{\strut
\vrule \quad\hfil # \ \hfil & 
\vrule \quad\hfil # \ \hfil & 
\vrule \quad\hfil # \ \hfil \vrule \cr
\noalign{\hrule}
%
% First row
Standard & Mark (1) voltage transmitted & Space (0) voltage transmitted \cr
%
\noalign{\hrule}
%
% Another row
EIA/TIA-232 &  &  \cr
%
\noalign{\hrule}
%
% Another row
EIA/TIA-422 &  &  \cr
%
\noalign{\hrule}
%
% Another row
EIA/TIA-485 &  &  \cr
%
\noalign{\hrule}
} % End of \halign 
}$$ % End of \vbox


% No blank lines allowed between lines of an \halign structure!
% I use comments (%) instead, so that TeX doesn't choke.

$$\vbox{\offinterlineskip
\halign{\strut
\vrule \quad\hfil # \ \hfil & 
\vrule \quad\hfil # \ \hfil & 
\vrule \quad\hfil # \ \hfil \vrule \cr
\noalign{\hrule}
%
% First row
Standard & Mark (1) voltage received & Space (0) voltage received \cr
%
\noalign{\hrule}
%
% Another row
EIA/TIA-232 &  &  \cr
%
\noalign{\hrule}
%
% Another row
EIA/TIA-422 &  &  \cr
%
\noalign{\hrule}
%
% Another row
EIA/TIA-485 &  &  \cr
%
\noalign{\hrule}
} % End of \halign 
}$$ % End of \vbox


\vskip 10pt

\noindent
Note that in digital communications, the following terms are synonymous:

\vskip 10pt

0 = Space = ``on''

\vskip 10pt

1 = Mark = ``off'' = ``idle'' (line state in a condition of no communication)

\vskip 10pt

Explain how {\it noise margin} relates to the minimum voltage signal levels specified in these standards.  Which of these serial communication standards enjoys the greatest immunity to electrical noise, based on the figures you research?



\vskip 20pt \vbox{\hrule \hbox{\strut \vrule{} {\bf Suggestions for Socratic discussion} \vrule} \hrule}

\begin{itemize}
\item{} Identify the difference between an EIA/TIA-422 device and an EIA/TIA-485 device.
\item{} Explain how the terms ``mark'' and ``space'' relate historically to {\it Morse Code}, an early serial communication standard.
\item{} Some cheap USB-to-serial converters you can buy for your personal computer do not fully comply with the EIA/TIA-232 standard, in that they only output +5 V and 0 V, not +5 V and $-5$ V.  Explain why this might be a problem.
\end{itemize}


\underbar{file i02193}
%(END_QUESTION)





%(BEGIN_ANSWER)

% No blank lines allowed between lines of an \halign structure!
% I use comments (%) instead, so that TeX doesn't choke.

$$\vbox{\offinterlineskip
\halign{\strut
\vrule \quad\hfil # \ \hfil & 
\vrule \quad\hfil # \ \hfil & 
\vrule \quad\hfil # \ \hfil \vrule \cr
\noalign{\hrule}
%
% First row
Standard & Mark (1) voltage transmitted & Space (0) voltage transmitted \cr
%
\noalign{\hrule}
%
% Another row
EIA/TIA-232 & $-5$ V & +5 V \cr
%
\noalign{\hrule}
%
% Another row
EIA/TIA-422 & $-2$ V & +2 V \cr
%
\noalign{\hrule}
%
% Another row
EIA/TIA-485 & $-1.5$ V & +1.5 V \cr
%
\noalign{\hrule}
} % End of \halign 
}$$ % End of \vbox


% No blank lines allowed between lines of an \halign structure!
% I use comments (%) instead, so that TeX doesn't choke.

$$\vbox{\offinterlineskip
\halign{\strut
\vrule \quad\hfil # \ \hfil & 
\vrule \quad\hfil # \ \hfil & 
\vrule \quad\hfil # \ \hfil \vrule \cr
\noalign{\hrule}
%
% First row
Standard & Mark (1) voltage received & Space (0) voltage received \cr
%
\noalign{\hrule}
%
% Another row
EIA/TIA-232 & $-3$ V & +3 V \cr
%
\noalign{\hrule}
%
% Another row
EIA/TIA-422 & $-200$ mV & +200 mV \cr
%
\noalign{\hrule}
%
% Another row
EIA/TIA-485 & $-200$ mV & +200 mV \cr
%
\noalign{\hrule}
} % End of \halign 
}$$ % End of \vbox

Note that the latter two communications standards, being differential rather than single-ended (ground-referenced), can operate with much less signal amplitude due to the relative immunity to external noise.

%(END_ANSWER)





%(BEGIN_NOTES)

With EIA/TIA-232, a negative voltage with respect to ground defines the ``1'' or {\it mark} state (oddly enough, this is the state the line is in when it is idle, with no communication.  It is also called the ``off'' state).  Conversely, a positive voltage with respect to ground defines the ``0'' or {\it space} state.

\vskip 10pt

With both EIA/TIA-422 and EIA/TIA-485, a negative $V_{AB}$ (terminal A negative and terminal B positive) defines the ``1'' or {\it mark} state.  This is also the default ``idle'' state when the line is not in communication, and it is also called the ``off'' state here.  Conversely, a positive $V_{AB}$ (terminal A positive and terminal B negative) defines the ``0'' or {\it space} state.  

Terminal A is sometimes labeled ($-$) and terminal B labeled (+) in respect of the default (idle) state, also called ``1'' or mark.

\vskip 10pt

The only difference in signal voltages between RS-422 and RS-485 is the minimum transmit levels.  RS-485 calls for $\pm$ 1.5 volts, while RS-422 calls for $\pm$ 2 volts.

\vskip 10pt

EIA/TIA-422 clearly has the greatest noise margin of the differential signaling schemes (2 volts $-$ 0.2 volts = {\bf 1.8 volts}).  EIA/TIA-485 comes next (1.5 volts $-$ 0.2 volts = {\bf 1.3 volts}).  EIA/TIA-232 actually has the greatest noise margin of them all (5 volts $-$ 3 volts = {\bf 2 volts}), but since this is a single-ended network rather than differential, its real-world noise immunity is less than either EIA/TIA-422 or EIA/TIA-485.

\vskip 10pt

With some of the early Morse sounder machines, a paper tape recorded the on/off states of the telegraph line, with a pencil making marks on the tape in accordance with the signals.  The blank areas of tape between ``marks,'' of course, were called ``spaces.''

%INDEX% Networking, signal voltages: EIA/TIA-232
%INDEX% Networking, signal voltages: EIA/TIA-422
%INDEX% Networking, signal voltages: EIA/TIA-485

%(END_NOTES)


