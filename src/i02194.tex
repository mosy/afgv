
%(BEGIN_QUESTION)
% Copyright 2007, Tony R. Kuphaldt, released under the Creative Commons Attribution License (v 1.0)
% This means you may do almost anything with this work of mine, so long as you give me proper credit

When configuring two EIA/TIA-232 devices for communication, their parameters must be identically set.  These parameters include the following:

\begin{itemize}
\item{} Stop bits (1, 1.5, or 2)
\item{} Data bits (5, 6, 7, or 8)
\item{} Parity (even, odd, or none)
\item{} Baud rate
\end{itemize}

Explain what each of these parameters mean, and why they must be set identically between the two serial devices.

\underbar{file i02194}
%(END_QUESTION)





%(BEGIN_ANSWER)

Asynchronous data communication is based on the concept of multiple ``packets'' of data, each one announced by a ``start'' bit, concluded with one or more ``stop'' bits, with a set number of data bits in between.  This is necessary for asynchronous communication because one cannot asynchronously send an arbitrarily long data packet without eventually encountering synchronization problems.  Packets must be limited to a set length and devices re-synchronized each time to prepare for transfer of the next packet, given inevitable differences in clock frequencies between transmitter and receiver.

%(END_ANSWER)





%(BEGIN_NOTES)


%INDEX% Networking, serial data: bit rate (or baud rate)
%INDEX% Networking, serial data: start bit
%INDEX% Networking, serial data: stop bit

%(END_NOTES)


