
%(BEGIN_QUESTION)
% Copyright 2010, Tony R. Kuphaldt, released under the Creative Commons Attribution License (v 1.0)
% This means you may do almost anything with this work of mine, so long as you give me proper credit

When multiple digital devices ``talk'' together on a common network, they must have some way of arbitrating who gets to talk, and in what order.  Otherwise there will be data loss as multiple devices inevitably attempt to transmit at the same time.

A variety of protocols exist to handle this problem.  A few are listed here:

\begin{itemize}
\item{} Master/slave
\item{} Token passing
\item{} TDMA ({\it Time Division Multiple Access})
\item{} CSMA/CD ({\it Carrier Sense Multiple Access / Collision Detect})
\item{} CSMA/BA ({\it Carrier Sense Multiple Access / Bitwise Arbitration})
\item{} CSMA/CA ({\it Carrier Sense Multiple Access / Collision Avoidance})
\end{itemize}

Explain how each of these protocols works, and identify which one is used in the {\it Ethernet} communication standard (IEEE 802.3).

\vskip 20pt \vbox{\hrule \hbox{\strut \vrule{} {\bf Suggestions for Socratic discussion} \vrule} \hrule}

\begin{itemize}
\item{} {\it Wireless} communications such as WLAN (IEEE 802.11) absolutely {\it cannot} use CSMA/CD protocol.  Explain why this is.
\item{} Explain what ``jabbering'' is, in your own words, and what effect it will have on networks using each of these five different protocols.
\item{} Identify ways to identify a ``jabbering'' problem occurring in a network.
\end{itemize}

\underbar{file i02200}
%(END_QUESTION)





%(BEGIN_ANSWER)

I'll let you research the answers here!

%(END_ANSWER)





%(BEGIN_NOTES)

\begin{itemize}
\item{} Master/slave: {\it Only select devices (``masters'') get to initiate communication.  All others (``slaves'') are merely allowed to respond to queries from a master.} 
\item{} Token passing {\it Each device gets a time to initiate communication, temporary mastership being designated by a packet of data called a ``token'' that gets passed from one device to the next.}
\item{} TDMA {\it Each device has a pre-assigned slot in a repeating schedule in which to speak.}
\item{} CSMA/CD {\it Any device may initiate communication at any time, leading to inevitable ``collisions.''  Collisions are detected, and then the colliding devices each try speaking again after a random ``back-off'' time.}
\item{} CSMA/BA  {\it Any device may initiate communication at any time, leading to inevitable ``collisions.'' Collisions are detected, and then the colliding devices speak again after a time delay set by a unique, pre-programmed ``back-off'' time.} 
\item{} CSMA/CA  {\it Devices each have a unique, pre-programmed wait time which must elapse after detecting a silent network before they may transmit.  Since each device's wait time is unique, no two devices will ever collide because one will always begin to transmit before the other.}
\end{itemize}

%INDEX% Networking, protocols: master-slave
%INDEX% Networking, protocols: token passing
%INDEX% Networking, protocols: CSMA/CD
%INDEX% Networking, protocols: CSMA/CA
%INDEX% Networking, protocols: CSMA/BA

%(END_NOTES)


