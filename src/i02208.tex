
%(BEGIN_QUESTION)
% Copyright 2007, Tony R. Kuphaldt, released under the Creative Commons Attribution License (v 1.0)
% This means you may do almost anything with this work of mine, so long as you give me proper credit

Twisted-pair Ethernet devices share a similar problem with EIA/TIA-232, 422, and 485 devices in that the {\it transmit} (TD) terminals on one device need to connect to the {\it receive} (RD) terminals on another.  If a DTE device is connected to a DCE device, the cable wiring will be ``straight'' (pin 1 on one end connects to pin 1 on the other end, pin 2 with pin 2, pin 3 with pin 3, etc.).  If, however, you wish to do something such as connect two computers together without a hub in between, a ``straight'' cable will not work, because the transmit pins on one computer will be connected through to the transmit pins on the other computer.

\vskip 10pt

The fix for this problem is a special cable called a {\it crossover} cable.  In the context of EIA/TIA-232, this is called a {\it null modem cable}, and it involves the same principle.  Describe what that principle is.

\underbar{file i02208}
%(END_QUESTION)





%(BEGIN_ANSWER)

The cable is built so that the transmit and receive lines ``cross over'' to the opposite terminals between one plug and the other.

%(END_ANSWER)





%(BEGIN_NOTES)

Note: crossover cables are sometimes necessary to link two hubs together.  Some hubs are able to automatically switch TD and RD pins, though, and some other hubs have a switch that may be flipped to reverse the TD/RD pins to allow the use of any cable type for linking hubs.

%INDEX% Networking, Ethernet: crossover cable

%(END_NOTES)


