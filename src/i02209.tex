
%(BEGIN_QUESTION)
% Copyright 2007, Tony R. Kuphaldt, released under the Creative Commons Attribution License (v 1.0)
% This means you may do almost anything with this work of mine, so long as you give me proper credit

{\it Ethernet} actually encompasses several similar network standards, varying by cable type and bit rate (speed).  The IEEE has designated special identifier names to denote each variety of Ethernet media.  Identify which type of Ethernet each of these names refers to:

\begin{itemize}
\item{} 10BASE2:
\vskip 5pt
\item{} 10BASE5:
\vskip 5pt
\item{} 10BASE-T:
\vskip 5pt
\item{} 10BASE-F:
\vskip 5pt
\item{} 100BASE-TX:
\vskip 5pt
\item{} 100BASE-FX:
\vskip 5pt
\item{} 1000BASE-T:
\vskip 5pt
\item{} 1000BASE-SX:
\vskip 5pt
\item{} 1000BASE-LX:
\end{itemize}

\vskip 20pt \vbox{\hrule \hbox{\strut \vrule{} {\bf Suggestions for Socratic discussion} \vrule} \hrule}

\begin{itemize}
\item{} Ethernet networks are considered ``baseband'' rather than ``broadband.''  Explain the distinction between these two terms.
\end{itemize}

\underbar{file i02209}
%(END_QUESTION)





%(BEGIN_ANSWER)

\begin{itemize}
\item{} 10BASE2: 10 Mbps, thin coaxial cable, 185 meters length maximum
\vskip 5pt
\item{} 10BASE5: 10 Mbps, thick coaxial cable, 500 meters length maximum
\vskip 5pt
\item{} 10BASE-T: 10 Mbps, two twisted wire pairs, Category 3 cable or better
\vskip 5pt
\item{} 10BASE-F: 10 Mbps, two optical fibers (includes 10BASE-FB, 10BASE-FP, and 10BASE-FL)
\vskip 5pt
\item{} 100BASE-TX: 100 Mbps, two twisted wire pairs, Category 5 cable or better
\vskip 5pt
\item{} 100BASE-FX: 100 Mbps, two optical fibers, multimode
\vskip 5pt
\item{} 1000BASE-T: 1 Gbps, four twisted wire pairs, Category 5 cable or better
\vskip 5pt
\item{} 1000BASE-SX: 1 Gbps, two optical fibers, short-wavelength
\vskip 5pt
\item{} 1000BASE-LX: 1 Gbps, two optical fibers, long-wavelength
\end{itemize}

%(END_ANSWER)





%(BEGIN_NOTES)


%INDEX% Networking, Ethernet: different versions of

%(END_NOTES)


