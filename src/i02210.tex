
%(BEGIN_QUESTION)
% Copyright 2007, Tony R. Kuphaldt, released under the Creative Commons Attribution License (v 1.0)
% This means you may do almost anything with this work of mine, so long as you give me proper credit

Open up a {\it terminal emulator} program on a computer.  This is a program that allows you to directly send ASCII characters out to a serial port on the computer, and also read ASCII characters coming in from another computer.  

If using Microsoft Windows (pre-Vista), try the stock {\tt hyperterminal} program in the Accessories folder.  For Windows Vista and 7, a version of {\tt hyperterminal} (from Hilgraeve) may be installed as an add-on.  If using Linux, {\tt minicom} is a good choice.  Another good one, available for a wide variety of operating systems, is {\tt PuTTY}.  Whatever your choice, the software needs to be installed on your computer prior to the class session.

Examine the communication port settings (data bits, stop bits, baud rate, etc.) and see what the allowable ranges are for each:

\begin{itemize}
\item{} Baud rate:
\vskip 5pt
\item{} Stop bits:
\vskip 5pt
\item{} Data bits:
\vskip 5pt
\item{} Parity options:
\end{itemize}

\underbar{file i02210}
%(END_QUESTION)





%(BEGIN_ANSWER)

No answers provided here.  You'll have to try it yourself on a real computer!

%(END_ANSWER)





%(BEGIN_NOTES)

{\tt Hyperterminal} also works with TCP/IP connections, using Winsock!  One note for successful use, though: one Hyperterminal session must be set up for ``Wait for Call'' while the other terminal places the call.  Otherwise, they won't connect.



\vfil \eject

\noindent
{\bf Prep Quiz:}

(A good prep quiz is to check and see that students have a terminal emulator program installed on their personal computers)

%INDEX% Networking, practical exercise: serial terminal emulator configuration

%(END_NOTES)


