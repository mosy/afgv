
%(BEGIN_QUESTION)
% Copyright 2007, Tony R. Kuphaldt, released under the Creative Commons Attribution License (v 1.0)
% This means you may do almost anything with this work of mine, so long as you give me proper credit

Add the necessary parity bits to each of these one-byte (eight-bit) data words.  The data is shown in hexadecimal format, and your answers should be in hexadecimal format as well, making the parity bit a ninth (most-significant) bit for each word:

\begin{itemize}
\item{} {\tt A7} \hskip 30pt Even parity = \underbar{\hskip 50pt} \hskip 30pt Odd parity = \underbar{\hskip 50pt}
\vskip 5pt
\item{} {\tt 58} \hskip 30pt Even parity = \underbar{\hskip 50pt} \hskip 30pt Odd parity = \underbar{\hskip 50pt}
\vskip 5pt
\item{} {\tt 20} \hskip 30pt Even parity = \underbar{\hskip 50pt} \hskip 30pt Odd parity = \underbar{\hskip 50pt}
\vskip 5pt
\item{} {\tt CA} \hskip 30pt Even parity = \underbar{\hskip 50pt} \hskip 30pt Odd parity = \underbar{\hskip 50pt}
\end{itemize}

Be sure to show your work in converting these hex numbers into binary so as to count the bits!

\vfil

\underbar{file i02220}
\eject
%(END_QUESTION)





%(BEGIN_ANSWER)

This is a graded question -- no answers or hints given!

%(END_ANSWER)





%(BEGIN_NOTES)

The general principle of parity is that the parity bit will be made a ``1'' or a ``0'' by the transmitting device as necessary to bring the total number of ``1'' bits in the data frame to either an even or an odd count as specified in the device configuration.  This parity bit must be appended to the original data word, which in this case converts every eight-bit word into a nine-bit word in preparation for transmission over a network.

For example, the data word ``A7'' is written in binary as 10100111, and happens to have an odd number of ``1'' bits.  If the transmitting device is configured for ``even'' parity, the parity bit will become a ``1'' in order to bring the total number of ``1'' bits to an even number.  The answer then becomes 110100111 which is written as 1A7 in hexadecimal.  If the transmitting device is configured for ``odd'' parity, the transmitting device will set the parity bit to ``0'' in order to keep the total number of ``1'' bits at an odd value.  This would make the answer 0101001110 which is written as 0A7 in hexadecimal.

\begin{itemize}
\item{} {\tt A7} \hskip 30pt Even parity = \underbar{\tt 1A7} \hskip 30pt Odd parity = \underbar{\tt 0A7}
\vskip 5pt
\item{} {\tt 58} \hskip 30pt Even parity = \underbar{\tt 158} \hskip 30pt Odd parity = \underbar{\tt 058}
\vskip 5pt
\item{} {\tt 20} \hskip 30pt Even parity = \underbar{\tt 120} \hskip 30pt Odd parity = \underbar{\tt 020}
\vskip 5pt
\item{} {\tt CA} \hskip 30pt Even parity = \underbar{\tt 0CA} \hskip 30pt Odd parity = \underbar{\tt 1CA}
\end{itemize}

%INDEX% Electronics review: calculating parity bits

%(END_NOTES)


