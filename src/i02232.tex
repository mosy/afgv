
%(BEGIN_QUESTION)
% Copyright 2007, Tony R. Kuphaldt, released under the Creative Commons Attribution License (v 1.0)
% This means you may do almost anything with this work of mine, so long as you give me proper credit

When I first heard of the ``Internet,'' I made the mistake of thinking it was a special cable stretching across vast portions of the world, dedicated to transporting web-page digital data.  To my surprise, the essential thing that makes up the Internet is not a physical object at all, but rather a {\it network protocol} specifying how digital data may be transparently communicated across all manner of digital networks (dedicated cables, satellite links, fiber optics, radio, etc.).  This protocol permits the exchange of data across an ad-hoc collection of networks between points far and wide.  Without a platform- and network-independent protocol, Internet really would have to be a dedicated cable or radio link stretching across the United States in order for people across the country to digitally communicate.

This protocol comes in two parts: {\it TCP} and {\it IP}.  Sometimes it is referred to as a single standard: {\it TCP/IP}.  Explain what ``TCP'' and ``IP'' represent, and how these network protocols are independent of specific details such as cable type, data rate, ``mark'' and ``space'' voltage levels, and other parameters associated with digital network hardware.

\vskip 20pt \vbox{\hrule \hbox{\strut \vrule{} {\bf Suggestions for Socratic discussion} \vrule} \hrule}

\begin{itemize}
\item{} An alternative to TCP is UDP, often used in industrial Ethernet networks.  Explain why UDP is more popular within industry, and how it differs from TCP.
\item{} SCADA systems used for the monitoring and control of such things as pipelines and electric power transmission networks typically rely on their own dedicated communication channels rather than the ``internet'' to communicate digital data over long distances.  Explain why.
\item{} Suppose an instrument technician got bored and decided to build a SCADA system for her home.  Using a PLC to acquire data from sensors installed throughout the house and also to control lights and valves, the technician is able to monitor her home from a ``smart'' phone with internet access.  Identify some of the network standards that might be employed in this system to transfer data between the PLC and her phone.
\end{itemize}

\underbar{file i02232}
%(END_QUESTION)





%(BEGIN_ANSWER)

{\it IP} stands for {\it Internet Protocol}, and it specifies how long blocks of data may be divided into smaller chunks called {\it packets}, how those packets may be re-assembled at the receiving end, and also how devices connected to a large network may be addressed both individually and by group.

\vskip 10pt

{\it TCP} stands for {\it Transmission Control Protocol}, and it specifies (among other things) how to ensure  integrity of communication in a network where data has been broken down into individual packets.  In essence, TCP guarantees deliver of all data packets even when network connections are less than perfectly reliable, by acknowledging correct receipt of each packet and requesting re-transmission in the event of corrupted or lost packets.

%(END_ANSWER)





%(BEGIN_NOTES)


%INDEX% Networking, protocol: TCP/IP

%(END_NOTES)


