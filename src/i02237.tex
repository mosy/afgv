
%(BEGIN_QUESTION)
% Copyright 2007, Tony R. Kuphaldt, released under the Creative Commons Attribution License (v 1.0)
% This means you may do almost anything with this work of mine, so long as you give me proper credit

One day a new instrument technician goes to connect a PLC (programmable logic controller) to an operator display panel and notices the communication ports on both the PLC and on the display panel are labeled {\it Modbus}, which the technician figures is some sort of networking standard.  Upon inspection of the screw terminals the technician also notices the wiring is similar to RS-485 (EIA/TIA-485), with TD(+), TD($-$), RD(+), and RD($-$) terminals.  Later on, when reading the user manuals for both devices, the technician notices the ports described as being ``RS-485'' as well as being ``Modbus.''

Asking a more experienced technician about this, the answer is that the ports are {\it both} EIA/TIA-485 and Modbus.  These two network standards are not exclusive, but complementary.

\vskip 10pt

Elsewhere in the world, a new computer network technician is going to download some software from a website, and notices it is possible to use either {\it HTTP} (Hyper-Text Transfer Protocol) or {\it FTP} (File Transfer Protocol) to do the job.  Looking behind the computer, the technician notices a regular Ethernet cable (twisted-pair) plugged into the network port.  Later, the technician asks someone more experienced, ``Which network standard am I using when I download files, HTTP, FTP, or Ethernet?''  The answer is similar to that given to the instrument technician: HTTP and FTP are alternatives to each other, both being complementary to Ethernet as parts of a complete protocol ``pathway'' from file to user.  It is never a question of HTTP {\it or} Ethernet, just as it is never a question of Modbus {\it or} RS-485.

\vskip 10pt

How would you explain either situation, using the OSI seven-layer model as a guide?

\vskip 20pt \vbox{\hrule \hbox{\strut \vrule{} {\bf Suggestions for Socratic discussion} \vrule} \hrule}

\begin{itemize}
\item{} A commonly-heard criticism of the OSI model is that ``no communications standard fully agrees with it,'' i.e. no single standard has specifications in all seven layers of the model.  Explain why this is not really a problem at all, and how it represents a fundamental misunderstanding of the OSI model.
\item{} {\it ASCII} is a layer-6 standard for encoding alphanumeric text in digital form.  Give an example where this standard works in conjunction with lower-level standards to communicate text data between two computers.
\item{} {\it S-HTTP} is a layer-7 standard used for encrypting and decrypting digital data over networks.  This is what you are using when you access a web page beginning with {\tt https://}.  Give an example where this standard works in conjunction with ASCII and other lower-level standards to securely communicate a credit card number between two computers during an online purchase transaction.
\end{itemize}

\underbar{file i02237}
%(END_QUESTION)





%(BEGIN_ANSWER)

The different networking standards listed describe different layers of the OSI model.  EIA/TIA-485 describes layer 1; Ethernet describes layers 1 and 2; Modbus, HTTP, and FTP all describe layer 7. 

\vskip 10pt

The OSI model is fundamentally a {\it framework} for defining the potential purposes of a network standard.  No existing standard occupies all seven layers because no existing standard is absolutely comprehensive from physical layer all the way up to application layer, exhibiting all possible functions and features in between.  Such a standard would be completely stand-alone.  Most networks are a conglomeration of different standards, working together to achieve a practical end.

%(END_ANSWER)





%(BEGIN_NOTES)


%INDEX% Networking, OSI model: necessity of describing different layers

%(END_NOTES)


