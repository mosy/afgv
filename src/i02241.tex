
%(BEGIN_QUESTION)
% Copyright 2007, Tony R. Kuphaldt, released under the Creative Commons Attribution License (v 1.0)
% This means you may do almost anything with this work of mine, so long as you give me proper credit

The fourth version of the {\it IP} (Internet Protocol) standard, known as IPv4, specifies an address that is 32 bits wide.  The address is usually expressed in the form of four ``octets'' translated into decimal form and separated by periods.  Here is an example of an IPv4 address:

\vskip 10pt

\centerline{\tt 196.252.70.183}

\vskip 10pt

What are the largest and the smallest IPv4 addresses possible in this format?  How many total unique addresses does this work out to be?

\vskip 10pt

The next version of IP is version 6 (version 5 was experimental).  IPv6 uses a 128-bit wide address.  How many ``octets'' does it take to express an IPv6 address?  How many unique addresses can be represented in the IPv6 field?

\vskip 20pt \vbox{\hrule \hbox{\strut \vrule{} {\bf Suggestions for Socratic discussion} \vrule} \hrule}

\begin{itemize}
\item{} Are there any special IP addresses reserved for specific purposes?
\item{} How do the address spaces of IPv4 and IPv6 compare to that of Ethernet MAC address space?
\item{} Why are IP addresses required in addition to Ethernet MAC addresses in a network where most devices are Ethernet-based?
\item{} Explain what {\tt 192.168.25.7/24} means as an IP address.
\item{} Explain what {\tt 192.168.25.7/8} means as an IP address.
\end{itemize}

\underbar{file i02241}
%(END_QUESTION)





%(BEGIN_ANSWER)

\noindent
{\bf Partial answer:}

\vskip 10pt

Largest IPv4 address: {\tt 255.255.255.255}

\vskip 10pt

Smallest IPv4 address: {\tt 0.0.0.0}

\vskip 10pt

IPv4 gives almost 4.3 {\it billion} unique addresses, but believe it or not we have already run out of IPv4 addresses (i.e. there are more IP-capable devices in existence than there are IPv4 addresses)!

%(END_ANSWER)





%(BEGIN_NOTES)

Smallest IPv4 address = 0.0.0.0  ;  largest IPv4 address = 255.255.255.255

IPv4 uses a 32-bit address field.  Total number of IPv4 addresses = $2^{32}$ = 4,294,967,296

\vskip 10pt

IPv6 uses a 128 bit address field.  Total number of IPv6 addresses = $2^{128}$ = $3.4028 \times 10^{38}$

%INDEX% Networking, IP: IPv4
%INDEX% Networking, IP: IPv6

%(END_NOTES)


