
%(BEGIN_QUESTION)
% Copyright 2007, Tony R. Kuphaldt, released under the Creative Commons Attribution License (v 1.0)
% This means you may do almost anything with this work of mine, so long as you give me proper credit

An improvement over direct-contact limit switches for many applications is the {\it inductive proximity switch}.  This type of switch actuates simply when an object gets {\it near} it -- no direct physical contact necessary!  Explain how these devices work, and what kinds of material they are able to detect.

Inductive proximity switches are powered devices by necessity.  They usually require a DC voltage for power, and their output is usually {\it not} a dry switch contact.  Instead, it is usually a transistor, with the output signal being standard TTL logic (0 to 5 volts).  Inductive proximity switches are often manufactured as three-wire devices:

$$\includegraphics[width=15.5cm]{i02243x01.eps}$$

\vskip 30pt

Show how you would connect the limit switch in the above illustration so that it makes the LED turn {\it on} when actuated, assuming the switch's internal transistor is configured to {\it source} current through the output lead.

\vskip 20pt \vbox{\hrule \hbox{\strut \vrule{} {\bf Suggestions for Socratic discussion} \vrule} \hrule}

\begin{itemize}
\item{} Identify an object a capacitive proximity switch would be able to detect.
\item{} Identify an object an ultrasonic proximity switch would be able to detect.
\item{} Identify an object an inductive proximity switch would {\it not} be able to detect.
\item{} Identify an object an optical proximity switch would {\it not} be able to detect.
\end{itemize}

\underbar{file i02243}
%(END_QUESTION)





%(BEGIN_ANSWER)

$$\includegraphics[width=15.5cm]{i02243x02.eps}$$

%(END_ANSWER)





%(BEGIN_NOTES)


%INDEX% Switch, proximity: inductive

%(END_NOTES)


