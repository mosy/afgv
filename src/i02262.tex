
%(BEGIN_QUESTION)
% Copyright 2010, Tony R. Kuphaldt, released under the Creative Commons Attribution License (v 1.0)
% This means you may do almost anything with this work of mine, so long as you give me proper credit

\noindent
{\bf Programming Challenge and Comparison -- Combination lock}

\vskip 10pt

Suppose we wish to have a PLC act as a combination lock, with four electrical toggle switches acting as input devices, such that the user must actuate those four switches in a specific sequence in order for a lamp to turn on.  For example, consider the following switch sequence:

\begin{itemize}
\item{} Turn all four switches off (re-sets lock to starting condition)
\item{} Flip switch \#3 on
\item{} Flip switch \#2 on
\item{} Flip switch \#4 on
\end{itemize}

In the end, toggle switches 2, 3, and 4 are all in the ``on'' position, but those switches must be turned on in the sequence shown in order for the lamp to energize.  In other words, it isn't enough simply to have those three switches turned on in any arbitrary order.

Write a PLC program performing this function, and demonstrate its operation using switches connected to its inputs to simulate the discrete inputs in a real application.  

\vskip 10pt

When your program is complete and tested, capture a screen-shot of it as it appears on your computer, and prepare to present your program solution to the class in a review session for everyone to see and critique.  The purpose of this review session is to see multiple solutions to one problem, explore different programming techniques, and gain experience interpreting PLC programs others have written.  When presenting your program, prepare to discuss the following points:

\begin{itemize}
\item{} Identify the ``tag names'' or ``nicknames'' used within your program to label I/O and other bits in memory
\item{} Follow the sequence of operation in your program, simulating the system in action
\item{} Identify any special or otherwise non-standard instructions used in your program, and explain why you decided to take that approach
\item{} Show the comments placed in your program, to help explain how and why it works
\item{} How you designed the program (i.e. what steps you took to go from a concept to a working program)
\end{itemize}

\vskip 20pt \vbox{\hrule \hbox{\strut \vrule{} {\bf Suggestions for Socratic discussion} \vrule} \hrule}

\begin{itemize}
\item{} How might it be possible to add one more feature to this program: an {\it alarm} that sounds if a person enters an incorrect sequence?
\end{itemize}


\underbar{file i02262}
%(END_QUESTION)





%(BEGIN_ANSWER)


%(END_ANSWER)





%(BEGIN_NOTES)

A sophisitcated and flexible solution to this programming challenge (when using an Allen-Bradley PLC) is to utilize a sequencer compare (SQC) instruction to compare switch states against a pre-programmed set of combination codes stored in {\tt B3} memory.  This way, the lock's combination may be altered without changing any of the logic in the program.

\vfil \eject

\noindent
{\bf Summary Quiz:}

(The recommended summary quiz is to have \underbar{each student} demonstrate their PLCs running this particular program)

%INDEX% PLC, programming challenge: combination lock

%(END_NOTES)


