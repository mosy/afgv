
%(BEGIN_QUESTION)
% Copyright 2010, Tony R. Kuphaldt, released under the Creative Commons Attribution License (v 1.0)
% This means you may do almost anything with this work of mine, so long as you give me proper credit

\noindent
{\bf Programming Challenge -- Reaction time measurement} 

\vskip 10pt

Program your PLC to measure a person's reaction time in flipping a switch.  The PLC should energize a light (or simply one of the discrete output indicating LEDs) telling the user when to flip an input switch, and then the PLC will measure how long it takes for the person to react to the light and flip the switch.

\vskip 20pt \vbox{\hrule \hbox{\strut \vrule{} {\bf Suggestions for Socratic discussion} \vrule} \hrule}

\begin{itemize}
\item{} How can you program the PLC to turn on the signaling light in a way that the person being tested cannot anticipate it?
\item{} How must you configure the reaction time timer to count in units appropriate for this very quick time delay?
\item{} What type of timer instruction is best suited for the reaction time timer, a {\it retentive} or a {\it non-retentive} timer?
\end{itemize}

\vfil 

\underbar{file i02266}
\eject
%(END_QUESTION)





%(BEGIN_ANSWER)


%(END_ANSWER)





%(BEGIN_NOTES)

I strongly recommend students save all their PLC programs for future reference, commenting them liberally and saving them with special filenames for easy searching at a later date!

\vskip 10pt

I also recommend presenting these programs as problems for students to work on in class for a short time period, then soliciting screenshot submissions from students (on flash drive, email, or some other electronic file transfer method) when that short time is up.  The purpose of this is to get students involved in PLC programming, and also to have them see other students' solutions to the same problem.  These screenshots may be emailed back to students at the conclusion of the day so they have other students' efforts to reference for further study.

%INDEX% PLC, programming challenge: reaction time measurement

%(END_NOTES)


