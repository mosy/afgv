
%(BEGIN_QUESTION)
% Copyright 2011, Tony R. Kuphaldt, released under the Creative Commons Attribution License (v 1.0)
% This means you may do almost anything with this work of mine, so long as you give me proper credit

%A {\it Programmable Logic Controller} or {\it PLC} is an industrial control computer designed to input and output many types of signals.  To handle different signal types (on/off, analog, digital networking), large-scale PLCs use different ``cards'' that plug into a common frame to provide I/O capacity to the processor:

En PLS er en styringsenhet som skal behandler mange typer in- og utgangssignaler. For {\aa} f{\aa} dette til er det vanlig {\aa} bruke ulike l{\o}sninger for tilkobling av ekstrakort. 

$$\epsfbox{i02271x01.eps}$$

%Read selected portions of the Allen-Bradley PLC ``1756 ControlLogix I/O Modules'' publication (document 1756-TD002A-EN-E, May 2009), and answer the following questions:

Les i instruskjonsmanualen for V200-18-E2B, og svar p{\aa} f{\o}lgende sp{\o}rsm{\aa}l. 

\vskip 10pt

%Locate the sample 4-20 mA device wiring diagrams for the 1756-IF6CIS ``sourcing current loop analog input module'', identifying the different types of 4-20 mA field devices supported.
Finn ut hvordan h.h.v 0-10 V og 4-20mA transmittere tilkobles kortet. 

\vskip 10pt

Finn ut hvilke inngangsomr{\aa}der som kan brukes av kortet, og hvor mange steg de deles opp i. 

\vskip 10pt

%Calculate the number of counts per milliamp of signal with this analog input card, and also the resolution (mA per count).
Regn ut antal steg per milliamper av signal p{\aa} dette kortet. Finn ogs{\aa} oppl{\o}sningen i mA pr. steg

\vskip 10pt

%Calculate the ``User counts'' value for a 8.51 mA signal value input to this analog card.
Regn ut antal steg for et signal p{\aa} 8.51 mA tilf{\o}rt inngangen. 


\vskip 10pt

%Calculate the mA current signal value at a ``User counts'' value of +4592.

Regn ut hvor stort inngangsignal i mA n{\aa}r antal steg p{\aa} inngangen er +754

\underbar{file i02271}
%(END_QUESTION)




%(BEGIN_ANSWER)

%Pages 28 and 29: this card may connect to loop-powered (2-wire) 4-20 mA devices as well as self-powered (4-wire) devices.  Three connection terminals are provided per channel: a positive voltage terminal (VOUT), an input terminal (IN), and a ground (RTN).

\vskip 10pt

%3106.8 counts per mA (equivalent to 0.3 microamps per count).

%(END_ANSWER)





%(BEGIN_NOTES)

%Pages 28 and 29: this card may connect to loop-powered (2-wire) 4-20 mA devices as well as self-powered (4-wire) devices.  Three connection terminals are provided per channel: a positive voltage terminal (VOUT), an input terminal (IN), and a ground (RTN).

\vskip 10pt

%Page 29: 0 to 21.09376 mA ; -32768 to +32767 counts.  This is clearly a 16-bit ADC expressing its count value as a signed integer (2's complement notation).

\vskip 10pt

${65535 \over 21.09376}= 3106.84$ counts per mA (equivalent to 0.3 microamps per count).

\vskip 10pt

8.51 mA = ${8.51 \over 21.09376}$ = 40.343\% of 65535 count span = 26439 counts above bottom = -6329 counts (-6328.766312 calculated)

\vskip 10pt

4592 counts = ${4592 - (-32768) \over 65535}$ = 57.0077\% of 21.09376 mA span = 12.02506864 mA

%INDEX% PLC, I/O: analog resolution and scaling
%INDEX% Reading assignment: Allen-Bradley 1756 ControlLogix I/O Modules publication

%(END_NOTES)

