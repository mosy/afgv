
%(BEGIN_QUESTION)
% Copyright 2007, Tony R. Kuphaldt, released under the Creative Commons Attribution License (v 1.0)
% This means you may do almost anything with this work of mine, so long as you give me proper credit

When fabricating control panels and custom instrument enclosures, it is often necessary to cut a hole in the side of the metal enclosure for such things as switches, connectors, and conduit to penetrate the enclosure.  While there is certainly more than one way to cut a neat hole in an electrical enclosure, one method is generally superior to all others: using a tool called a {\it knockout punch}.

Research knockout punches, and describe how they function.  Hint: the Greenlee company makes a line of knockout punches called the {\it Slug Buster}.  You might want to try researching their website for detailed information.

\underbar{file i02278}
%(END_QUESTION)





%(BEGIN_ANSWER)

I'll let you do the research here!

%(END_ANSWER)





%(BEGIN_NOTES)

All knockout punches first require that a hole has been drilled through the enclosure, to allow the bolt to go through and join both halves of the punch.  Usually, this drill hole is quite a bit smaller than the final hole made by the punch, and may be drilled with relative ease.  For larger punches having larger bolts, and requiring larger starting holes, you may need to drill a small hole, use a small knockout punch to expand the hole, then upgrade to the size of punch you need for the final hole.

%INDEX% Good practices, wiring: knockout punches for making holes in cabinets

%(END_NOTES)


