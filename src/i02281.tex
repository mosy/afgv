
%(BEGIN_QUESTION)
% Copyright 2010, Tony R. Kuphaldt, released under the Creative Commons Attribution License (v 1.0)
% This means you may do almost anything with this work of mine, so long as you give me proper credit

Suppose you must run two signal cables from field-mounted instruments to a central room where the control system is located.  Two different electrical conduits stretch from the field location to the control system room: one with 480 VAC power wiring in it, and another with low-level control signal wiring in it.  The two cables you must run through these conduits are as follows:

\begin{itemize}
\item{} One twisted-pair cable carrying a 4-20 mA analog DC signal
\vskip 5pt
\item{} One twisted-pair cable carrying a Modbus digital signal (RS-485 physical layer)
\end{itemize}

At first, you plan to run both these cables through the signal wire conduit.  However, you soon discover this signal conduit only has room to accommodate one cable but not both.  The power wire conduit, however, has plenty of available room.

\vskip 10pt

Which cable would you run through which conduit, and why?

\vskip 20pt \vbox{\hrule \hbox{\strut \vrule{} {\bf Suggestions for Socratic discussion} \vrule} \hrule}

\begin{itemize}
\item{} Explain why it is best to run all {\it signal} cables in conduit completely separate from {\it power} cables.
\end{itemize}

\underbar{file i02281}
%(END_QUESTION)





%(BEGIN_ANSWER)


%(END_ANSWER)





%(BEGIN_NOTES)

Run the 4-20 mA analog cable through the signal conduit, and the Modbus digital cable through the power conduit, because digital signals can tolerate more noise corruption than analog signals.

%INDEX% Good practices, wiring: separation of power and signal wires

%(END_NOTES)


