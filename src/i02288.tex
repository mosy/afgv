
%(BEGIN_QUESTION)
% Copyright 2007, Tony R. Kuphaldt, released under the Creative Commons Attribution License (v 1.0)
% This means you may do almost anything with this work of mine, so long as you give me proper credit

Explain why the following wiring practices are standard:

\vskip 10pt

{\bf Separating power wiring from signal wiring in an enclosure:}

\vskip 200pt

{\bf Using wire duct and/or wire ``looms'' to organize wires in electrical enclosures:}

\vskip 200pt

\vfil

\underbar{file i02288}
\eject
%(END_QUESTION)





%(BEGIN_ANSWER)

This is a graded question -- no answers or hints given!

%(END_ANSWER)





%(BEGIN_NOTES)

General principles associated with wire separation include minimization of capacitance between wires ($C = {\epsilon A \over d}$), as well as minimization of mutual inductance ($M$) between wires.

\vskip 10pt

The use of wire duct is legitimized by the need to keep wiring orderly, in order to help facilitate troubleshooting and to reduce the possibility of wires becoming tangled and caught by objects.

%INDEX% Good practices, wiring: separation of power and signal wires
%INDEX% Good practices, wiring: use of wire duct and wire looms

%(END_NOTES)


