
%(BEGIN_QUESTION)
% Copyright 2010, Tony R. Kuphaldt, released under the Creative Commons Attribution License (v 1.0)
% This means you may do almost anything with this work of mine, so long as you give me proper credit

Suppose two computer devices are networked together using RS-232 serial data ports.  The cable connecting the devices' two 9-pin ports together has three wires: {\it transmit}, {\it receive}, and {\it ground}.  The problem is, the devices refuse to ``talk'' with one another over this network: no data sent by the first device is received by the second, or vice-versa.

\vskip 10pt

This is a brand-new system, and the technician who built it has just finished checking the continuity of the data cable's three conductors (from one end to the other) using an ohmmeter.  Furthermore, the technician verified that the proper communication software has been installed and is running in both computer devices.  What else other than the cable might be set up incorrectly to prevent these two devices from ``talking'' via RS-232?  Identify {\it five} different potential problems that could prevent communication between these two devices.

\vskip 20pt

\begin{itemize}
\item{$(1)$}
\vskip 70pt 
\item{$(2)$} 
\vskip 70pt 
\item{$(3)$} 
\vskip 70pt 
\item{$(4)$} 
\vskip 70pt 
\item{$(5)$} 
\end{itemize}

\vfil

\underbar{file i02290}
\eject
%(END_QUESTION)





%(BEGIN_ANSWER)

This is a graded question -- no answers or hints given!

%(END_ANSWER)





%(BEGIN_NOTES)

\begin{itemize}
\item{} Cable not connected as a ``null modem'' (transmit connecting to receive and vice-versa)
\item{} Bit rate (baud rate) not equal on both devices
\item{} Number of stop bits not equal on both devices
\item{} Parity not set equal on both devices
\item{} Cable too long, and/or bit rate too high (corrupting data)
\item{} Devices not using software handshaking (Xon/Xoff), but expecting hardware handshaking instead
\item{} Ground loop between two devices
\end{itemize}

Lack of terminating resistors is a common suggestion made by students, but RS-232 networks are not commonly known to employ resistors.  The ratio of minimum received voltage to minimum transmitted voltage in the RS-232 standard doesn't leave much room for the loading imposed by termination resistors, unlike RS-422 and RS-485 which can function all the way down to $\pm$200 mV received signal.

%INDEX% Networking, serial: EIA/TIA-232 (formerly RS-232)

%(END_NOTES)


