
%(BEGIN_QUESTION)
% Copyright 2014, Tony R. Kuphaldt, released under the Creative Commons Attribution License (v 1.0)
% This means you may do almost anything with this work of mine, so long as you give me proper credit

A Siemens S7-200 PLC uses a ``resistive transducer'' analog input module to convert a sensor's variable resistance into an integer ``count'' value spanning this range:

% No blank lines allowed between lines of an \halign structure!
% I use comments (%) instead, so that TeX doesn't choke.

$$\vbox{\offinterlineskip
\halign{\strut
\vrule \quad\hfil # \ \hfil & 
\vrule \quad\hfil # \ \hfil \vrule \cr
\noalign{\hrule}
%
% First row
{\bf Input resistance} & {\bf Count value} \cr
%
\noalign{\hrule}
%
% Another row
0 $\Omega$ & 0 counts \cr
%
\noalign{\hrule}
%
% Another row
355.54 $\Omega$ & 32767 counts \cr
%
\noalign{\hrule}
} % End of \halign 
}$$ % End of \vbox

Given this measurement range, calculate the following:

\vskip 30pt

\noindent
Count value when $R_{sensor}$ is 125 $\Omega$ = \underbar{\hskip 50pt} counts

\vskip 30pt

\noindent
$R_{sensor}$ when count value is 19837 = \underbar{\hskip 50pt} $\Omega$

\vskip 20pt

\underbar{file i02307}
%(END_QUESTION)





%(BEGIN_ANSWER)

Deduct 2 points for any ``count'' answers that are not whole numbers.

\vskip 10pt

\noindent
Count value when $R_{sensor}$ is 125 $\Omega$ = \underbar{\bf 11520 or 11521} counts

\vskip 10pt

\noindent
$R_{sensor}$ when count value is 19837 = \underbar{\bf 215.24} $\Omega$

%(END_ANSWER)





%(BEGIN_NOTES)

{\bf This question is intended for exams only and not worksheets!}.

%(END_NOTES)

