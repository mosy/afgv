
%(BEGIN_QUESTION)
% Copyright 2013, Tony R. Kuphaldt, released under the Creative Commons Attribution License (v 1.0)
% This means you may do almost anything with this work of mine, so long as you give me proper credit

Følgende P-regulering kildekode (pseudokode) inneholder logikk for å begrense utgangssignalet mellom 0\% og 100\%:

\vskip 10pt

\hbox{ \vrule
\vbox{ \hrule \vskip 3pt
\hbox{ \hskip 3pt
\vbox{ \hsize=5in \raggedright

\noindent
\underbar{\bf Pseudocode listing}

{\tt Set ERROR = SP - PV}

{\tt Set OUTPUT = (GAIN * ERROR) + BIAS}

{\tt IF OUTPUT > 100\% THEN OUTPUT = 100\%}

{\tt IF OUTPUT < 0\% THEN OUTPUT = 0\%}

}
\hskip 3pt}%
\vskip 5pt \hrule}%
\vrule}
\vskip 10pt

Forklar hvorfor det er nødvendig å inkludere lignende grense-logikk i den integrerende (I) delen av algoritmen for en PID-regulator, og forklar hva konsekvensene er for regulatorens drift hvis denne logikken utelates.

\underbar{file i02325}
%(END_QUESTION)





%(BEGIN_ANSWER)

Hvis integraldelen av regulator-algoritmen ikke er begrenset, vil verdien fortsette å integrere oppover (eller nedover) over metningsgrensen på 100\% (eller under 0\%) hvis den utsettes for et vedvarende avvik over lengre tid. Dette kalles {\it integrasjonsmetning} (integral windup), og resulterer i en treg respons fra regulatoren når avviket endrer fortegn.

%(END_ANSWER)





%(BEGIN_NOTES)

Be studentene beskrive en situasjon der integrasjonsmetning kan oppstå.

%INDEX% Control, integral: windup
%INDEX% Control, digital algorithm: saturation limits

%(END_NOTES)
