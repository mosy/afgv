
%(BEGIN_QUESTION)
% Copyright 2010, Tony R. Kuphaldt, released under the Creative Commons Attribution License (v 1.0)
% This means you may do almost anything with this work of mine, so long as you give me proper credit

\noindent
{\bf Programming Challenge -- Blinking alarm light} 

\vskip 10pt

Suppose we wish to create a high-pressure alarm system for operators to alert them when the pressure inside a process vessel exceeds certain pressure limits.  Two normally-open pressure switches connect to this vessel, and to separate inputs on the PLC.  The first is a PSH (pressure switch high) set for a trip pressure of 150 PSI, and the second is a PSHH (pressure switch high-high) set for a trip pressure of 220 PSI.  If the first pressure switch (150 PSI) activates, we want to make a warning light blink on and off.  If the second pressure switch activates (220 PSI), {\it we want that same warning light to remain on steady}.

\vskip 10pt

Write a PLC program to perform this function, and demonstrate its operation using switches connected to its inputs to simulate the discrete inputs in a real application.


\vskip 20pt \vbox{\hrule \hbox{\strut \vrule{} {\bf Suggestions for Socratic discussion} \vrule} \hrule}

\begin{itemize}
\item{} How do you make the alarm light blink on and off if all the PLC input points are steady (non-pulsing)?
\item{} Can you think of any alternatives to having the same light energize differently to represent two different process alarm conditions?
\end{itemize}


\vfil 

\underbar{file i02341}
\eject
%(END_QUESTION)





%(BEGIN_ANSWER)


%(END_ANSWER)





%(BEGIN_NOTES)

I strongly recommend students save all their PLC programs for future reference, commenting them liberally and saving them with special filenames for easy searching at a later date!

\vskip 10pt

I also recommend presenting these programs as problems for students to work on in class for a short time period, then soliciting screenshot submissions from students (on flash drive, email, or some other electronic file transfer method) when that short time is up.  The purpose of this is to get students involved in PLC programming, and also to have them see other students' solutions to the same problem.  These screenshots may be emailed back to students at the conclusion of the day so they have other students' efforts to reference for further study.


%INDEX% PLC, programming challenge: high pressure alarm (blinking/steady)

%(END_NOTES)




