
%(BEGIN_QUESTION)
% Copyright 2011, Tony R. Kuphaldt, released under the Creative Commons Attribution License (v 1.0)
% This means you may do almost anything with this work of mine, so long as you give me proper credit

In 2010, the first computer virus designed to infect an industrial control system was detected, with the capability of infecting and compromising Siemens S7-315-2 PLCs.  This virus became to be known as {\it Stuxnet}.  The majority of Stuxnet infections were encountered in Iran, leading many to conclude it was a targeted attack on Iran's nuclear fuel enrichment efforts.

\vskip 10pt

Search the internet to find the video of Ralph Langner's talk at the 2011 TED conference in Long Beach, California, where he summarizes the apparent design and his team's investigation of Stuxnet ({\tt http://www.ted.com} is probably your best source!).  Based on Ralph's presentation, how were they able to identify the intended target of the Stuxnet virus, and what exactly was the virus trying to accomplish in the Siemens S7 control system?

Feel free to research other references on this virus, identifying how it exploited vulnerabilities in Microsoft Windows operating systems.  What lessons may we take away from this event as instrument technicians, charged with the responsibility of ensuring integrity of PLC-controlled processes?

\underbar{file i02345}
%(END_QUESTION)





%(BEGIN_ANSWER)

Stuxnet is a fascinating subject to research, with many technical references to be found on the Internet.  Be warned: you may find yourself occupied for hours on end finding and reading various documents on Stuxnet, partly because of the technical complexity of the subject and partly because it's so fascinating to consider the ramifications of a {\it PLC virus}.  Good sources to research include these:

\vskip 10pt

\noindent
Falliere, Nicolas; Murchu, Liam O; Chien, Eric; ``W32.Stuxnet dossier'', Version 1.4, Symantec Security Response, February, 2011.

\vskip 10pt

\noindent
Byres, Eric; Ginter, Andrew; Langill, Joel; ``How Stuxnet Spreads -- A Study of Infection Paths in Best Practice Systems'', Version 1.0, Tofino Security, Abterra Technologies, ScadaHacker.com, February 22, 2011.

\vskip 10pt

\noindent
Matrosov, Aleksandr; Rodionov, Eugene; Harley, David; Malcho, Juraj; ``Stuxnet Under the Microscope'', Version 1.31, ESET, 2011.

\vskip 10pt

Another excellent video to watch is Symantec's demonstration of a Sutxnet infection, found on YouTube under the title ``Stuxnet: How it Infects PLCs''.

Another YouTube video you may wish to view is a talk given by Bruce Dang of Microsoft.  Dang's lively (and laugh-out-loud profane) description of how his team at Microsoft cracked the ``dropper'' portion is worth watching, although much more technical than Langner's.  Dang's presentation is filled with code-specific references and hacker slang (e.g. he says ``vuln'' when he means ``vulnerability'') which may make it difficult for non-programmers to understand.

%(END_ANSWER)





%(BEGIN_NOTES)


%INDEX% Research assignment: Stuxnet PLC virus

%(END_NOTES)

