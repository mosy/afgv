
%(BEGIN_QUESTION)
% Copyright 2014, Tony R. Kuphaldt, released under the Creative Commons Attribution License (v 1.0)
% This means you may do almost anything with this work of mine, so long as you give me proper credit

An Allen-Bradley MicroLogix PLC uses an analog input module to convert a DC current signal into an integer ``count'' value spanning this range:

% No blank lines allowed between lines of an \halign structure!
% I use comments (%) instead, so that TeX doesn't choke.

$$\vbox{\offinterlineskip
\halign{\strut
\vrule \quad\hfil # \ \hfil & 
\vrule \quad\hfil # \ \hfil \vrule \cr
\noalign{\hrule}
%
% First row
{\bf Input current} & {\bf Count value} \cr
%
\noalign{\hrule}
%
% Another row
0 mA & 0 counts \cr
%
\noalign{\hrule}
%
% Another row
21.0 mA & 32767 counts \cr
%
\noalign{\hrule}
} % End of \halign 
}$$ % End of \vbox

Given this measurement range, calculate the following:

\vskip 30pt

\noindent
Count value when $I_{input}$ is 5.9 mA = \underbar{\hskip 50pt} counts

\vskip 30pt

\noindent
$I_{input}$ when count value is 28417 = \underbar{\hskip 50pt} mA

\vskip 20pt

\underbar{file i02348}
%(END_QUESTION)





%(BEGIN_ANSWER)

Deduct 2 points for any ``count'' answers that are not whole numbers.

\vskip 10pt

\noindent
Count value when $I_{input}$ is 5.9 mA = \underbar{\bf 9205 or 9206} counts

\vskip 10pt

\noindent
$I_{input}$ when count value is 28417 = \underbar{\bf 18.21} mA

%(END_ANSWER)





%(BEGIN_NOTES)

{\bf This question is intended for exams only and not worksheets!}.

%(END_NOTES)

