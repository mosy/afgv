
%(BEGIN_QUESTION)
% Copyright 2010, Tony R. Kuphaldt, released under the Creative Commons Attribution License (v 1.0)
% This means you may do almost anything with this work of mine, so long as you give me proper credit

\noindent
{\bf Programming Challenge -- HMI control of sprinkler valves} 

\vskip 10pt

Suppose an instrument technician wishes to have a PLC-controlled sprinkler system in his yard, with an HMI panel inside his house with ``pushbutton'' graphics on the screen where he may conveniently activate sprinkler water solenoid valves.  To begin this project, the technician connects two solenoid valves to two discrete outputs on his PLC: one valve opens up to send water to his fruit tree sprinkler nozzles while the other valve opens up to fire a jet of water at the nearby fire hydrant where his neighbor's dog likes to mark his territory.

\vskip 10pt

Create a simple HMI project with two ``pushbutton'' icons on the screen.  The first icon will directly activate the PLC output bit for the fruit tree sprinklers, and it needs to have a {\it toggle} action: pressing this icon once turns the bit on, and pressing it a second time turns it off.  The second icon will directly activate the PLC output bit for the anti-dog water cannon, and it needs to have a {\it momentary} action: pressing this icon activates the water jet, and releasing it stops the water jet.

The PLC itself should have no instructions programmed in it (except perhaps for an {\tt END} rung to avoid a processor error).

\vskip 20pt \vbox{\hrule \hbox{\strut \vrule{} {\bf Suggestions for Socratic discussion} \vrule} \hrule}

\begin{itemize}
\item{} What types of HMI data tags (boolean, integer, floating-point, ASCII, etc.) should be used for both these ``pushbutton'' objects?
\item{} How do you specify the action (toggle, momentary, etc.) of the ``pushbutton'' icons on the HMI screen?
\item{} What steps must you take to create appropriate tag names in the HMI for the PLC's data points?
\item{} Explain why the PLC should have no program in it, or conversely, what bad things could happen if a program existed in the PLC with coils addressed to the same output bits the HMI was attempting to write to.
\end{itemize}

\vfil 

\underbar{file i02350}
\eject
%(END_QUESTION)





%(BEGIN_ANSWER)


%(END_ANSWER)





%(BEGIN_NOTES)

I strongly recommend students save all their PLC programs for future reference, commenting them liberally and saving them with special filenames for easy searching at a later date!

\vskip 10pt

I also recommend presenting these programs as problems for students to work on in class for a short time period, then soliciting screenshot submissions from students (on flash drive, email, or some other electronic file transfer method) when that short time is up.  The purpose of this is to get students involved in PLC programming, and also to have them see other students' solutions to the same problem.  These screenshots may be emailed back to students at the conclusion of the day so they have other students' efforts to reference for further study.



%INDEX% PLC, programming challenge: HMI control of sprinkler valves

%(END_NOTES)


