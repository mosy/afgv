
%(BEGIN_QUESTION)
% Copyright 2010, Tony R. Kuphaldt, released under the Creative Commons Attribution License (v 1.0)
% This means you may do almost anything with this work of mine, so long as you give me proper credit

The {\it Hall} process for converting alumina (Al$_{2}$O$_{3}$) into metallic aluminum (Al$_{2}$) involves electrolysis with carbon electrodes.  A powerful electric current forces carbon to join with the oxygen atoms in the alumina, forming carbon dioxide (CO$_{2}$) gas.  The chemical equation describing this reaction is shown here:

$$\hbox{Al}_2\hbox{O}_3 + \hbox{C}_2 \to \hbox{Al}_2 + \hbox{CO}_2$$

Balance this equation to that all molecules appear in their proper proportions.  Also, determine how many pounds of carbon electrode must be consumed in this process to yield three tons (6000 pounds) of metallic aluminum, assuming perfect conversion efficiency.

\underbar{file i02353}
%(END_QUESTION)





%(BEGIN_ANSWER)

$$4\hbox{Al}_2\hbox{O}_3 + 3\hbox{C}_2 \to 4\hbox{Al}_2 + 6\hbox{CO}_2$$

%(END_ANSWER)





%(BEGIN_NOTES)

The molar ratio between aluminum and carbon in this equation is 4:3 (an atomic ratio of 8:6).  Aluminum has an atomic mass of 27, while carbon is 12.  Thus, the mass ratio for this equation is 216:72, or 3:1.  

If we desire to produce 6000 pounds of aluminum, we will need to consume 2000 pounds (one-third the mass) of carbon.  In other words, we will need 1 ton of carbon to produce 3 tons of aluminum.

%INDEX% Chemistry, stoichiometry: balancing a chemical equation
%INDEX% Process: aluminum electrolysis (Hall process)

%(END_NOTES)


