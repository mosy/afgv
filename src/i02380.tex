
%(BEGIN_QUESTION)
% Copyright 2010, Tony R. Kuphaldt, released under the Creative Commons Attribution License (v 1.0)
% This means you may do almost anything with this work of mine, so long as you give me proper credit

Suppose a PLC program contains a counter instruction that counts in unsigned 16-bit integer format.  An HMI connected to this PLC, however, is configured to read and interpret this counter's accumulated value as a {\it BCD} number instead of unsigned binary.

\vskip 10pt

If the PLC's counter has an accumulated value of 38199 (decimal), determine how the HMI will incorrectly interpret and display it.

\vfil
\underbar{file i02380}
\eject
%(END_QUESTION)





%(BEGIN_ANSWER)

This is a graded question -- no answers or hints given!

%(END_ANSWER)





%(BEGIN_NOTES)

A good problem-solving strategy to apply here is writing out all the binary bits comprising the unsigned 16-bit value of 38199, and then seeing how those bits would be interpreted as BCD.  In an unsigned binary number, each successive place-weight is twice the value of the one before it (e.g. 1's place, 2's place, 4's place, 8's place, 16's place, etc.).  In a BCD number, every four bits are translated as a single decimal digit much like hexadecimal translates every four bit group into one hex character.

\vskip 10pt

38199 as a 16-bit unsigned integer: {\tt 1001010100110111}

That same number interpreted as BCD = {\tt 1001 0101 0011 0111} = 9537

\vskip 10pt

Therefore, the HMI will display the number as 9537.

%INDEX% Digital number format: binary versus BCD integer representation

%(END_NOTES)


