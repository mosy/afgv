
%(BEGIN_QUESTION)
% Copyright 2009, Tony R. Kuphaldt, released under the Creative Commons Attribution License (v 1.0)
% This means you may do almost anything with this work of mine, so long as you give me proper credit

Identify a few of the ``function codes'' specified within the {\it Modbus} standard used to read and write data between industrial devices, and provide the appropriate Modbus address ranges for each of the function codes you list.

\vfil
\underbar{file i02384}
\eject
%(END_QUESTION)





%(BEGIN_ANSWER)

This is a graded question -- no answers or hints given!

%(END_ANSWER)





%(BEGIN_NOTES)

The only way to answer this question is to locate a reference that gives you the definitions of all the various Modbus function codes and address ranges.

% No blank lines allowed between lines of an \halign structure!
% I use comments (%) instead, so that TeX doesn't choke.

$$\vbox{\offinterlineskip
\halign{\strut
\vrule \quad\hfil # \ \hfil & 
\vrule \quad\hfil # \ \hfil \vrule \cr
\noalign{\hrule}
%
% First row
Modbus code & Function \cr
(decimal) &  \cr
%
\noalign{\hrule}
%
% Another row
01 & Read one or more PLC output ``coils'' (1 bit each) \cr
%
\noalign{\hrule}
%
% Another row
02 & Read one or more PLC input ``contacts'' (1 bit each) \cr
%
\noalign{\hrule}
%
% Another row
03 & Read one or more PLC ``holding'' registers (16 bits each) \cr
%
\noalign{\hrule}
%
% Another row
04 & Read one or more PLC analog input registers (16 bits each) \cr
%
\noalign{\hrule}
%
% Another row
05 & Write (force) a single PLC output ``coil'' (1 bit) \cr
%
\noalign{\hrule}
%
% Another row
06 & Write (preset) a single PLC ``holding'' register (16 bits) \cr
%
\noalign{\hrule}
%
% Another row
15 & Write (force) multiple PLC output ``coils'' (1 bit each) \cr
%
\noalign{\hrule}
%
% Another row
16 & Write (preset) multiple PLC ``holding'' registers (16 bits each) \cr
%
\noalign{\hrule}
} % End of \halign 
}$$ % End of \vbox

\vskip 10pt

Modbus ``984'' addressing defines sets of fixed numerical ranges where various types of data may be found in a PLC or other control device.  The absolute address ranges (according to the Modbus 984 scheme) are shown in this table: 

% No blank lines allowed between lines of an \halign structure!
% I use comments (%) instead, so that TeX doesn't choke.

$$\vbox{\offinterlineskip
\halign{\strut
\vrule \quad\hfil # \ \hfil & 
\vrule \quad\hfil # \ \hfil \vrule \cr
\noalign{\hrule}
%
% First row
Address range (decimal) & Purpose \cr
%
\noalign{\hrule}
%
% Another row
00001 to 09999 & discrete outputs (``coils'') \cr
%
\noalign{\hrule}
%
% Another row
10001 to 19999 & discrete inputs (``contacts'') \cr
%
\noalign{\hrule}
%
% Another row
30001 to 39999 & analog input registers \cr
%
\noalign{\hrule}
%
% Another row
40001 to 49999 & ``holding'' registers \cr
%
\noalign{\hrule}
} % End of \halign 
}$$ % End of \vbox


%INDEX% PLC, ladder logic programming: Modbus

%(END_NOTES)


