
%(BEGIN_QUESTION)
% Copyright 2010, Tony R. Kuphaldt, released under the Creative Commons Attribution License (v 1.0)
% This means you may do almost anything with this work of mine, so long as you give me proper credit

\noindent
{\bf Programming Challenge -- Four-function calculator} 

\vskip 10pt

Write a PLC program and corresponding HMI project to make a simple four-function (add, subtract, multiply, and divide) calculator taking two integer values input by the user and displaying the sum, difference, product, and quotient of those two input values on the HMI screen.

\vskip 20pt \vbox{\hrule \hbox{\strut \vrule{} {\bf Suggestions for Socratic discussion} \vrule} \hrule}

\begin{itemize}
\item{} Determine how you could safely ``experiment'' with your PLC's math instructions to determine how it handles conditions such as ``divide-by-zero''.
\end{itemize}



\vfil 

\noindent
PLC comparison:

\begin{itemize}
\item{} \underbar{Allen-Bradley Logix 5000}: {\tt CMP}, {\tt ADD}, {\tt SUB}, {\tt MUL}, and {\tt DIV} instructions
\vskip 5pt
\item{} \underbar{Allen-Bradley SLC 500}: {\tt ADD}, {\tt SUB}, {\tt MUL}, and {\tt DIV} instructions
\vskip 5pt
\item{} \underbar{Siemens S7-200}: {\tt ADD\_I}, {\tt SUB\_I}, {\tt MUL\_I}, and {\tt DIV\_I} instructions
\vskip 5pt
\item{} \underbar{Koyo (Automation Direct) DirectLogic}: {\tt ADD}, {\tt SUB}, {\tt MUL}, and {\tt DIV} instructions
\end{itemize}

\underbar{file i02385}
\eject
%(END_QUESTION)





%(BEGIN_ANSWER)


%(END_ANSWER)





%(BEGIN_NOTES)

I strongly recommend students save all their PLC programs for future reference, commenting them liberally and saving them with special filenames for easy searching at a later date!

\vskip 10pt

I also recommend presenting these programs as problems for students to work on in class for a short time period, then soliciting screenshot submissions from students (on flash drive, email, or some other electronic file transfer method) when that short time is up.  The purpose of this is to get students involved in PLC programming, and also to have them see other students' solutions to the same problem.  These screenshots may be emailed back to students at the conclusion of the day so they have other students' efforts to reference for further study.

\vfil \eject

\noindent
{\bf Summary Quiz:}

(The recommended summary quiz is to have students demonstrate their PLCs running this particular program)

%INDEX% PLC, programming challenge: fillage/ullage calculator

%(END_NOTES)


