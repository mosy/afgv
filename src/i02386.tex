
%(BEGIN_QUESTION)
% Copyright 2010, Tony R. Kuphaldt, released under the Creative Commons Attribution License (v 1.0)
% This means you may do almost anything with this work of mine, so long as you give me proper credit

\noindent
{\bf Programming Challenge -- Reversing motor restart delay} 

\vskip 10pt

Suppose a large machine is turned by an electric motor with reversing contactors.  Due to the high inertia of the machine's moving parts, the motor requires time to coast to a stop before reversing direction, or else the motor will become overloaded in the effort to switch direction while overcoming inertia spinning the wrong way.  

After some experimentation with the machine, it is determined that 20 seconds provides an adequate delay following a ``Stop'' command prior to starting the motor in the opposite direction.  To provide this delay automatically, you are asked to program a PLC inhibiting a re-start in the opposite direction for 20 seconds following a ``stop'' command.  The ``Forward'' and ``Reverse'' commands are to come from an HMI panel (not hard-wired pushbutton switches), and the HMI panel should also indicate ``inhibit'' status to the operator so he or she knows they must wait before pushing a different directional button.

\vskip 10pt

Write a PLC program to provide this forward/reverse/restart lockout functionality, and an HMI screen with the appropriate ``pushbutton'' icons and ``inhibit'' display.

\vskip 20pt \vbox{\hrule \hbox{\strut \vrule{} {\bf Suggestions for Socratic discussion} \vrule} \hrule}

\begin{itemize}
\item{} What type of timer instruction is best suited for this application, an {\it on-delay} or an {\it off-delay} timer?
\end{itemize}

\vfil 

\underbar{file i02386}
\eject
%(END_QUESTION)





%(BEGIN_ANSWER)


%(END_ANSWER)





%(BEGIN_NOTES)

I strongly recommend students save all their PLC programs for future reference, commenting them liberally and saving them with special filenames for easy searching at a later date!

\vskip 10pt

I also recommend presenting these programs as problems for students to work on in class for a short time period, then soliciting screenshot submissions from students (on flash drive, email, or some other electronic file transfer method) when that short time is up.  The purpose of this is to get students involved in PLC programming, and also to have them see other students' solutions to the same problem.  These screenshots may be emailed back to students at the conclusion of the day so they have other students' efforts to reference for further study.

%INDEX% PLC, programming challenge: reversing motor restart delay (with HMI)

%(END_NOTES)


