
%(BEGIN_QUESTION)
% Copyright 2010, Tony R. Kuphaldt, released under the Creative Commons Attribution License (v 1.0)
% This means you may do almost anything with this work of mine, so long as you give me proper credit

\noindent
{\bf Programming Challenge -- HMI-driven up/down setpoint control} 

\vskip 10pt

In an industrial process controlled by a PLC, the operators desire to have pushbutton control over the setpoint of a control loop.  In other words, they want one pushbutton to increment the setpoint value of the loop whenever it is pushed, and another pushbutton to decrement the setpoint value of the loop whenever pushed.  Furthermore, they want both these ``pushbuttons'' to be on an HMI screen rather than be real hard-wired pushbutton switches.

\vskip 10pt

Write a PLC program to provide this up/down control over an integer value, and an HMI screen with the appropriate ``pushbutton'' icons and numerical display for the setpoint.

\vskip 20pt \vbox{\hrule \hbox{\strut \vrule{} {\bf Suggestions for Socratic discussion} \vrule} \hrule}

\begin{itemize}
\item{} What type of counter instruction is best suited for this application?
\item{} Would there be an easy way to build a high limit into this system, so the operators could not increment the setpoint value any greater than a pre-set limit value?
\end{itemize}

\vfil 

\underbar{file i02389}
\eject
%(END_QUESTION)





%(BEGIN_ANSWER)


%(END_ANSWER)





%(BEGIN_NOTES)

I strongly recommend students save all their PLC programs for future reference, commenting them liberally and saving them with special filenames for easy searching at a later date!

\vskip 10pt

I also recommend presenting these programs as problems for students to work on in class for a short time period, then soliciting screenshot submissions from students (on flash drive, email, or some other electronic file transfer method) when that short time is up.  The purpose of this is to get students involved in PLC programming, and also to have them see other students' solutions to the same problem.  These screenshots may be emailed back to students at the conclusion of the day so they have other students' efforts to reference for further study.

%INDEX% PLC, programming challenge: HMI-driven up/down setpoint control

%(END_NOTES)


