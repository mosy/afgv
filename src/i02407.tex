
%(BEGIN_QUESTION)
% Copyright 2012, Tony R. Kuphaldt, released under the Creative Commons Attribution License (v 1.0)
% This means you may do almost anything with this work of mine, so long as you give me proper credit

Read selected portions of the Fluke ``Series 80 Series V Multimeters users manual'' (May 2004 revision 2, 11/08) and answer the following questions:

\vskip 10pt

Pages 32-33 describe the {\it MIN/MAX} measuring mode.  Explain what this does, in your own words, identifying a practical application for using this mode.

\vskip 10pt

Suppose you trying to measure a 60 Hz AC voltage signal that you suspected was being corrupted by high-frequency noise voltage.  Explain how the Low Pass Filter mode available in the model 87 multimeter (described on page 15) could be useful to you in this situation.

\vskip 10pt

Page 31 describes the {\it HiRes} measuring mode available on the model 87.  Explain what this does, in your own words, identifying a practical application for using this mode.

\vskip 10pt

Page 34 describes the {\it Relative} measuring mode available on the model 87.  Explain what this does, in your own words, identifying a practical application for using this mode.

\vskip 20pt \vbox{\hrule \hbox{\strut \vrule{} {\bf Suggestions for Socratic discussion} \vrule} \hrule}

\begin{itemize}
\item{} Suppose you attempted to measure the voltage of an AC+DC ``mixed'' signal (i.e. a signal with an AC component as well as a DC bias, like what you might encounter at the base terminal of the transistor in a class-A audio amplifier).  It would be good to know how well your mulitmeter discriminates between AC and DC in both measurement modes.  Devise a test by which you could determine how well your multimeter is able to selectively measure just the AC or just the DC portion of a ``mixed'' signal voltage.
\item{} Is the MIN/MAX mode of your multimeter fast enough to capture the peak voltage value of a 60 Hz sine-wave?  Demonstrate this if you can.
\item{} Demonstrate some of the other special features of your multimeter, describing practical applications of these features.
\item{} For those who have studied variable frequency motor drives (VFDs), explain why the Low Pass Filter mode is useful for taking AC voltage and current measurements on VFD power conductors.
\end{itemize}

\underbar{file i02407}
%(END_QUESTION)





%(BEGIN_ANSWER)


%(END_ANSWER)





%(BEGIN_NOTES)

MIN/MAX mode records the lowest and highest measurements, and allows you to view them later.  Good for tracking intermittent problems.  Time duration required for MIN/MAX capture is 100 ms (0.1 seconds), but only 250 microseconds for ``peak'' mode (model 87 only).

\vskip 10pt

Low-pass filter mode engages a 1 kHz lowpass filter on the input, blocking frequencies higher than that from being measured and interpreted by the meter in any AC mode.  If you suspected your signal was being corrupted by high-frequency noise, you could engage the filter mode and see if it makes any difference in your reading!

\vskip 10pt

HiRes mode switches from 3-1/2 digit to 4-1/2 digit resolution for greater precision of measurement.

\vskip 10pt

Relative mode acts like the ``tare'' button on a weigh scale: it ``zeroes'' the meter at any reference level you set it to.  You can reference your meter to some measured value by pressing the REL $\Delta$ button, then have the meter calculate how much more or less the present signal is compared to the referenced signal value.

%INDEX% Electronics review: DMM (test equipment) -- using high-resolution mode
%INDEX% Reading assignment: Fluke 80 series DMM manual

%(END_NOTES)


