
%(BEGIN_QUESTION)
% Copyright 2010, Tony R. Kuphaldt, released under the Creative Commons Attribution License (v 1.0)
% This means you may do almost anything with this work of mine, so long as you give me proper credit

The National Electrical Code, or {\it NEC} -- otherwise known as {\it NFPA 70} -- is a standard published by the National Fire Protection Agency.  Among (many!) other things, it specifies the following maximum ``fill'' percentages for electrical conduit, given the number of conductors contained within:

\vskip 10pt

1 conductor: 53\%

\vskip 10pt

2 conductors: 31\%

\vskip 10pt

3 or more conductor: 40\%

\vskip 10pt

Explain the rationale behind these figures.  Why be worried about how ``full'' a conduit is at all?  Additionally, why should the most conservative rating be for a condition where there are only {\it two} conductors in a conduit?

\underbar{file i02413}
%(END_QUESTION)





%(BEGIN_ANSWER)

Here is a hint:

$$\includegraphics[width=15.5cm]{i02413x01.eps}$$

%(END_ANSWER)





%(BEGIN_NOTES)

Maximum fill capacity for conduit is primarily based on stress during wire-pulling.  The relatively low percentage limit for two conductors should be obvious if one examines a pictorial representation of what a ``full'' conduit would look like holding 1, 2, and more than 2 conductors (shown in the answer).

As you can see, a single round conductor fills a round conduit most efficiently, while two round conductors provide an inefficient fill.  Greater numbers of smaller round conductors approaches the fill-efficiency of a single conductor, but never quite manages to equal it.

%INDEX% Good practices, wiring: conduit fill

%(END_NOTES)


