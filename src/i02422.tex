
%(BEGIN_QUESTION)
% Copyright 2010, Tony R. Kuphaldt, released under the Creative Commons Attribution License (v 1.0)
% This means you may do almost anything with this work of mine, so long as you give me proper credit

Suppose a single-phase AC load draws a current of 16.5 amps at 237 volts (RMS).  If the measured power factor of this load is 0.85, calculate the {\it true power} ($P$) dissipated by the load as well as its {\it apparent power} ($S$).  Be sure to include the proper unit of measurement (e.g. VA, VAR, or W) with each answer!

\vskip 10pt

$P$ = 

\vskip 40pt

$S$ = 

\vfil 

\underbar{file i02422}
\eject
%(END_QUESTION)





%(BEGIN_ANSWER)

This is a graded question -- no answers or hints given!

%(END_ANSWER)





%(BEGIN_NOTES)

Apparent power ($S$) is the simplest to calculate, being voltage times current (power factor ignored).  Any time you multiply these two quantities in a single-phase AC circuit you will end up with a value that is ``apparently'' the circuit's power, although true power in an AC circuit is a bit more complex than this:

\vskip 10pt

$S$ = (16.5)(237) = 3910.5 VA = 3.911 kVA

\vskip 30pt

True power ($P$), however, is that fraction of apparent power doing real work.  True power in a single-phase AC circuit is always equal to the apparent power multiplied by the power factor of that circuit:

\vskip 10pt

$P$ = (16.5)(237)(0.85) = 3323.9 kW = 3.324 kW

\vskip 10pt

To further contrast apparent versus true power, consider the case of a purely resistive AC circuit: here apparent power is the exact same value as true power, because every bit of the current does useful work in the form of heat dissipation at the resistor (i.e. a power factor of 1).  In a purely reactive circuit (e.g. a pure inductor or a pure capacitor), {\it none} of the current does useful work, but instead it's all apparent power and zero true power (i.e. a power factor of 0).

%INDEX% Electronics review: power factor

%(END_NOTES)

