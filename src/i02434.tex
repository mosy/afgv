
%(BEGIN_QUESTION)
% Copyright 2010, Tony R. Kuphaldt, released under the Creative Commons Attribution License (v 1.0)
% This means you may do almost anything with this work of mine, so long as you give me proper credit

Download a free FOUNDATION Fieldbus segment design tool from a Fieldbus instrument manufacturer's website or the Fieldbus Foundation website ({\tt http://www.fieldbus.org}), and use this software to lay out a simple H1 segment complete with power supply, enough field instruments to make one working control loop, and coupling devices to connect the instruments together.  I highly recommend {\it DesignMATE} software, which is a free download.

\vskip 10pt

When you have the segment layout complete, use the ``Print Screen'' function of your personal computer to capture the screen image and save it as a graphic file.  The instructor may collect some of these screenshots from you to display for the entire class.

\vskip 20pt \vbox{\hrule \hbox{\strut \vrule{} {\bf Suggestions for Socratic discussion} \vrule} \hrule}

\begin{itemize}
\item{} Describe some of the practical benefits that may come from using segment design software to plan a Fieldbus H1 segment.
\item{} Would FOUNDATION Fieldbus segment design software work to plan a Profibus PA network?  Why or why not?
\item{} Intentionally introduce problems into your H1 segment to see how the design tool flags those errors.  Examples include:
\itemitem{} Excessive cable length
\itemitem{} Too many terminating resistors
\itemitem{} Too few terminating resistors
\itemitem{} Wrong cable type for the application
\end{itemize}

\underbar{file i02434}
%(END_QUESTION)





%(BEGIN_ANSWER)


%(END_ANSWER)





%(BEGIN_NOTES)

\vfil \eject

\noindent
{\bf Summary Quiz:}

(An alternative to a summary quiz is to have students show the H1 segment layouts they've developed using this software tool)


%INDEX% Fieldbus, FOUNDATION (H1): segment design tool software

%(END_NOTES)


