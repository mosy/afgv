
%(BEGIN_QUESTION)
% Copyright 2014, Tony R. Kuphaldt, released under the Creative Commons Attribution License (v 1.0)
% This means you may do almost anything with this work of mine, so long as you give me proper credit

The different {\it Virtual Communication Relationships} or {\it VCRs} in FOUNDATION Fieldbus H1 networks can be hard to understand without illustrative examples.  In order to make better sense of how the {\it Publisher/Subscriber}, {\it Client/Server}, and {\it Source/Sink} communication models work, we will match these VCR types to the following classroom interactions between an instructor and students.  In each of these cases, identify who plays the role of the LAS, and which VCR is being implemented:

\vskip 10pt

\noindent
{\bf Scenario \#1}

The instructor calls upon students one at a time to voluntarily share any progress they have made on their lab projects.  Some students give a report, while others pass.  When each student is finished with their turn, the instructor calls upon the next student.  There is no student-to-student dialogue.


\vskip 10pt

\noindent
{\bf Scenario \#2}

The instructor is playing the role of a moderator in a debate between two students in a classroom.  Neither student is allowed to speak unless directed to by the instructor, and each one has a limited time to speak once they begin.  The two debaters are listening to each other's arguments, responding to the points raised, confirming correct comprehension of what they heard, and asking questions of each other.




\vskip 10pt

\noindent
{\bf Scenario \#3}

The instructor is managing a classroom exercise where groups of students are studying probability and random processes.  Five different students have been asked to each roll a pair of dice ten times and tally the total count.  The instructor gives the five students enough time to roll their dice and tally the results, then calls upon each one of these students (one at a time) to report their tally.  A sixth student in the classroom listens to each of the five students' tallies and then calculates the average value.  After waiting another short interval, the instructor then calls upon the sixth student to report the calculated average.


\vskip 20pt \vbox{\hrule \hbox{\strut \vrule{} {\bf Suggestions for Socratic discussion} \vrule} \hrule}

\begin{itemize}
\item{} In the Publisher/Subscriber example, who is publishing data and who is subscribing to it?  Are there any people in the room who overhear published data but are not subscribed to that data?
\item{} In the Client/Server example, who plays the role of the Client and who plays the role of the Server?
\item{} Extend these analogies to include a failure of the LAS, with a backup LAS taking over.
\end{itemize}

\underbar{file i02437}
%(END_QUESTION)





%(BEGIN_ANSWER)


%(END_ANSWER)





%(BEGIN_NOTES)

In each scenario the instructor plays the role of the LAS.

\begin{itemize}
\item{} {\bf Scenario \#1:} Source/Sink {\it (Queued User-triggered Unidirectional)}
\item{} {\bf Scenario \#2:} Client/Server {\it (Queued User-triggered Bidirectional)}
\item{} {\bf Scenario \#3:} Publisher/Subscriber {\it (Buffered Network-scheduled Unidirectional)}
\end{itemize}

Note how Scenario \#2 (Client/Server) allows students (FF devices) to exchange information back and forth with each other, and is the only truly bidirectional (duplex) VCR described.  In the other two scenarios the data flows from students (FF devices) to the classroom (network), but there is no reciprocation from other students (FF devices). 

Also note the limited scope of the instructor's ``token'' (permission to speak) in each case: in Scenario \#3 permission is granted by the instructor to report specific data (tallies from the dice-rolling experiment); in the other scenarios permission is granted to speak for a limited time but for a range of purposes (Scenario \#1 is voluntary, Scenario \#2 allows for student-to-student dialogue).

%INDEX% Fieldbus, FOUNDATION (H1): cyclic vs. acyclic communication
%INDEX% Fieldbus, FOUNDATION (H1): scheduled vs. unscheduled communication

%(END_NOTES)


