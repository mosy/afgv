
%(BEGIN_QUESTION)
% Copyright 2007, Tony R. Kuphaldt, released under the Creative Commons Attribution License (v 1.0)
% This means you may do almost anything with this work of mine, so long as you give me proper credit

In FOUNDATION Fieldbus H1, there are three distinct classes of bus communication (Virtual Communication Relationships, or {\it VCR}s):

\begin{itemize}
\item{} Publisher / Subscriber
\item{} Client / Server
\item{} Source / Sink (also called Report Distribution)
\end{itemize}

Explain how each of these works, and identify the types of data exchanged by each in a Fieldbus network.

\underbar{file i02439}
%(END_QUESTION)





%(BEGIN_ANSWER)

\noindent
{\bf Partial answer:}

\underbar{Publisher / Subscriber}: scheduled communication whereby the Link Active Scheduler (LAS) calls upon a specific device on the network to transmit specific data for time-critical control purposes.  The message calling for this data is called a Compel Data (CD) request.  Multiple devices on the network ``subscribing'' to this published data receive it simultaneously.

\vskip 10pt

\underbar{Client / Server}: unscheduled communication between individual devices enabled whenever the LAS grants a Pass Token (PT) to a device.  Each device maintains a queue (list) of data requests issued by other devices (clients), and responds to them in order.  By responding to client requests, the device acts as a server.  Likewise, each device can use this time to act as a client, posting their own requests to other devices, which will act as servers when they receive the token from the LAS.

\vskip 10pt

\underbar{Source / Sink}: unscheduled communication from one device to a ``group address'' representing many devices, enabled whenever the LAS grants a Pass Token (PT) to a device. 

%(END_ANSWER)





%(BEGIN_NOTES)

\underbar{Publisher / Subscriber}: scheduled communication whereby the Link Active Scheduler (LAS) calls upon a specific device on the network to transmit specific data for time-critical control purposes.  The message calling for this data is called a Compel Data (CD) request.  Multiple devices on the network ``subscribing'' to this published data receive it simultaneously.  {\it Used to communicate process-related parameters between function blocks, for control purposes.  The publisher/subscriber VCR is highly deterministic.}

\vskip 10pt

\underbar{Client / Server}: unscheduled communication between individual devices enabled whenever the LAS grants a Pass Token (PT) to a device.  Each device maintains a queue (list) of data requests issued by other devices (clients), and responds to them in order.  By responding to client requests, the device acts as a server.  Likewise, each device can use this time to act as a client, posting their own requests to other devices, which will act as servers when they receive the token from the LAS.  {\it Used to communicate maintenance and device configuration data, operator setpoint changes, diagnostics, adjusting alarm and PID tuning values, etc.  Client/server communications are checked for data corruption, to ensure reliable data flow.}

\vskip 10pt

\underbar{Source / Sink}: unscheduled communication from one device to a ``group address'' representing many devices, enabled whenever the LAS grants a Pass Token (PT) to a device.  {\it Used to communicate trends and alarms (alarm acknowledgments are handled through client/server VCR, however).  Source/sink communications are not checked for data corruption, so it is technically possible for data to go missing here!} 

\vskip 10pt

One of the better resources I've found for information on this is the Fieldbus Foundation's {\it FOUNDATION Specification System Architecture} report, especially page 7 (Section 2.1.1.6).

%INDEX% Fieldbus, FOUNDATION (H1): Virtual Communication Relationships (VCRs)

%(END_NOTES)


