
%(BEGIN_QUESTION)
% Copyright 2011, Tony R. Kuphaldt, released under the Creative Commons Attribution License (v 1.0)
% This means you may do almost anything with this work of mine, so long as you give me proper credit

Suppose you have recently installed a pressure transmitter ranged from 40 to 240 PSI, complete with a field-mounted analog loop indicator registering 4 to 20 milliamps.  The installation is brand-new, and you have not yet received the custom scale for the analog indicator showing 40 to 240 PSI.  Instead, the indicator's face simply reads 4 to 20 milliamps.

\vskip 10pt

For the time being, the operators need a way to translate the ``milliamp'' number value read on the indicator into a ``PSI'' number value they can relate to the process.  Write simple instructions for calculating PSI from any milliamp value they happen to read off this pressure indicator's face.

\vfil

\underbar{file i02452}
\eject
%(END_QUESTION)





%(BEGIN_ANSWER)

This is a graded question -- no answers or hints given!

%(END_ANSWER)





%(BEGIN_NOTES)

The solution to this problem is nothing more than the writing of a $y = mx + b$ equation in a form understandable by operators.  Placing 40 to 240 PSI on the vertical ($y$) axis of a graph and 4 to 20 mA on the horizontal ($x$) axis of a graph allows us to visually set up the problem in order to solve for $m$ and $b$:

$$m = {\hbox{Rise} \over \hbox{Run}} = {240 - 40 \over 20 - 4} = 12.5$$

Now that we know the value of $m$, we may substitute known $x$ and $y$ values for any given point on the function and solve for $b$.  In this example, I will choose 4 milliamps for $x$ and 40 PSI for $y$:

$$40 = (4)(12.5) + b$$

$$40 = 50 + b$$

$$b = -10$$

Thus, the $y = mx + b$ formula is $y = 12.5x - 10$.  Expressed in simpler terms:
$$\hbox{PSI} = (\hbox{milliamps}) \times 12.5 - 10$$

{\it ``Take the milliamp number and multiply by 12.5, then subtract 10 to get PSI.''}

\vskip 10pt

A common mistake made by students is assuming that the value of $b$ will always be the lower-range value of the output (in this case, 40).  However, this is only true when the range of $x$ has an LRV of zero, such as the case in which $x$ is scaled in percent (i.e. 0\% to 100\%).

An even more fundamental error, however, is in students not double-checking their work.  Whenever you develop a formula for doing some calculation -- especially if your intent is to give that formula to someone else for them to use and trust -- you should always test it by inputting some values for which you already know the correct answers.  In this particular example, you should input 4 for $x$ and verify that you get an answer of 40 for $y$; also when $x$ is 20 you should get an answer of 240 for $y$.  Failure to run such a simple and basic test of your work is failure to regard your own potential for error.  {\it Always check your work!}

%INDEX% Basics, transmitter: input and output ranges

%(END_NOTES)


