
%(BEGIN_QUESTION)
% Copyright 2010, Tony R. Kuphaldt, released under the Creative Commons Attribution License (v 1.0)
% This means you may do almost anything with this work of mine, so long as you give me proper credit

Match the best data type to each of the following industrial measurement applications.  Note that more than one of the answers may be identical (i.e. same data type for multiple applications), and that it is possible one or more data types listed will not be used:

\begin{itemize}
\item{} (A) Discrete
\item{} (B) Unsigned integer
\item{} (C) Signed integer
\item{} (D) Floating-point
\item{} (E) ASCII
\end{itemize}

\begin{itemize}
\item{} Counting soup cans passing by on a unidirectional conveyor at a cannery: \underbar{\hskip 50pt}
\vskip 10pt
\item{} Measuring the temperature of a furnace, in degrees Fahrenheit: \underbar{\hskip 50pt}
\vskip 10pt
\item{} Displaying the status of an across-the-line electric motor starter: \underbar{\hskip 50pt}
\vskip 10pt
\item{} Showing the difference between number of units assembled and a daily production goal: \underbar{\hskip 50pt}
\vskip 10pt
\item{} Timing the total run-time of an electric motor, in whole seconds: \underbar{\hskip 50pt}
\end{itemize}

\underbar{file i02454}
%(END_QUESTION)





%(BEGIN_ANSWER)

\begin{itemize}
\item{} Counting soup cans passing by on a unidirectional conveyor at a cannery: \underbar{\bf B}
\vskip 10pt
\item{} Measuring the temperature of a furnace, in degrees Fahrenheit: \underbar{\bf D}
\vskip 10pt
\item{} Displaying the status of an across-the-line electric motor starter: \underbar{\bf A}
\vskip 10pt
\item{} Showing the difference between number of units assembled and a daily production goal: \underbar{\bf C}
\vskip 10pt
\item{} Timing the total run-time of an electric motor, in whole seconds: \underbar{\bf B}
\end{itemize}


%(END_ANSWER)





%(BEGIN_NOTES)

{\bf This question is intended for exams only and not worksheets!}.

%(END_NOTES)

