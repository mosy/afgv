
%(BEGIN_QUESTION)
% Copyright 2010, Tony R. Kuphaldt, released under the Creative Commons Attribution License (v 1.0)
% This means you may do almost anything with this work of mine, so long as you give me proper credit

Suppose an electrician decides to check a newly installed FOUNDATION Fieldbus H1 segment trunk cable for short-circuit faults by clipping the test leads of an insulation tester (often referred to by the brand name ``Megger'') between the (+) and ($-$) terminals of the segment cable accessible in a junction box, then using the insulation tester to measure resistance.  The electrician knows to shut off the DC power to the network first so that the DC voltage does not interfere with his resistance test.

\vskip 10pt

Explain why the electrician's method of cable testing is very risky, especially for any Fieldbus devices connected to the segment.

\vfil

\underbar{file i02457}
\eject
%(END_QUESTION)





%(BEGIN_ANSWER)

This is a graded question -- no answers or hints given!

%(END_ANSWER)





%(BEGIN_NOTES)

A ``megger'' outputs hundreds of volts when measuring resistance.  This high voltage will surely damage any Fieldbus devices connected to the segment, as well as possibly damaging terminating resistors, network coupling devices (equipped with active features such as short-circuit protection), and/or power supply isolators.

Such high-voltage resistance testing is useful for detecting otherwise hard-to-find faults in cables, motor windings, and other applications where conductors lie in close proximity to each other.  However, if one does not first disconnect all voltage-sensitive electronic devices from the conductors to be tested, the result may be a lot of damaged equipment!

%INDEX% Fieldbus, FOUNDATION (H1): segment troubleshooting

%(END_NOTES)


