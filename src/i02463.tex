
%(BEGIN_QUESTION)
% Copyright 2007, Tony R. Kuphaldt, released under the Creative Commons Attribution License (v 1.0)
% This means you may do almost anything with this work of mine, so long as you give me proper credit

\noindent
Calculate energy and/or power for the following scenarios:

\vskip 10pt {\narrower \noindent \baselineskip5pt

\noindent
One gallon of gasoline stores approximately 121 megajoules (MJ) of energy in chemical form.  If this gallon of gasoline is burned completely and steadily over a period of 2 hours, how much power is output by the fire?

\par} \vskip 10pt



\vskip 10pt {\narrower \noindent \baselineskip5pt

\noindent
A battery charger connected to a secondary-cell battery applies a constant charging voltage of 14.3 volts at a constant charging current of 6.8 amps for 2.5 hours.  How much power does this represent (in watts), and how much energy has been delivered to the battery by the charger (in Joules)?

\par} \vskip 10pt


\vskip 10pt {\narrower \noindent \baselineskip5pt

\noindent
One metric ton of bituminous/anthracite coal contains approximately 28 gigajoules of energy in chemical form.  If a coal-burning power plant operating at 35\% conversion efficiency (chemical to electrical energy) is to output 500 MW of power, what rate must the coal be burned in units of metric tons per minute?

\par} \vskip 10pt



\underbar{file i02463}
%(END_QUESTION)





%(BEGIN_ANSWER)

One gallon of gasoline burned over a period of 2 hours = 16.8 kW

\vskip 10pt

The battery charger outputs a total of 875.16 kJ over the 2.5 hour time period, at a rate of 97.24 watts.

\vskip 10pt

The power plant must burn coal at a rate of 3.061 metric tons per minute.

\vskip 10pt

Data for the heat value of various fuels taken from the following web document:

{\tt http://bioenergy.ornl.gov/papers/misc/energy\_conv.html}

%(END_ANSWER)





%(BEGIN_NOTES)


%INDEX% Physics, energy, work, power: calculating energy and power

%(END_NOTES)


