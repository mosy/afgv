
%(BEGIN_QUESTION)
% Copyright 2007, Tony R. Kuphaldt, released under the Creative Commons Attribution License (v 1.0)
% This means you may do almost anything with this work of mine, so long as you give me proper credit

\vbox{\hrule \hbox{\strut \vrule{} $\int f(x) \> dx$ \hskip 5pt {\sl Calculus alert!} \vrule} \hrule}

Capacitors store energy in the form of an electric field.  We may calculate the energy stored in a capacitance by integrating the product of capacitor voltage and capacitor current ($P = IV$) over time, since we know that power is the rate at which work ($W$) is done, and the amount of work done to a capacitor taking it from zero voltage to some non-zero amount of voltage constitutes energy stored ($U$):

$$P = {dW \over dt}$$

$$dW = P \> dt$$

$$U = W = \int P \> dt$$

Find a way to substitute capacitance ($C$) and voltage ($V$) into the integrand so you may integrate to find an equation describing the amount of energy stored in a capacitor for any given capacitance and voltage values.

\underbar{file i02464}
%(END_QUESTION)





%(BEGIN_ANSWER)

$$U = {1 \over 2}CV^2$$

%(END_ANSWER)





%(BEGIN_NOTES)

The integration required to obtain the answer is commonly found in calculus-based physics textbooks, and is an easy (power rule) integration:

$$U = \int P \> dt$$

$$U = \int VI \> dt$$

$$U = \int VC{dv \over dt} \> dt$$

$$U = C \int V {dv \over dt} \> dt$$

$$U = C \int V \> dv$$

$$U = {1 \over 2}C V^2$$

%INDEX% Electronics review: energy stored in a capacitor

%(END_NOTES)


