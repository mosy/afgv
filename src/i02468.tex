
%(BEGIN_QUESTION)
% Copyright 2011, Tony R. Kuphaldt, released under the Creative Commons Attribution License (v 1.0)
% This means you may do almost anything with this work of mine, so long as you give me proper credit

Read and outline the ``Limit Controls'' subsection of the ``Limit, Selector, and Override Controls'' section of the ``Basic Process Control Strategies'' chapter in your {\it Lessons In Industrial Instrumentation} textbook.  Note the page numbers where important illustrations, photographs, equations, tables, and other relevant details are found.  Prepare to thoughtfully discuss with your instructor and classmates the concepts and examples explored in this reading.

\underbar{file i02468}
%(END_QUESTION)





%(BEGIN_ANSWER)


%(END_ANSWER)





%(BEGIN_NOTES)

High- and low-limit function blocks may be used to limit the value of cascaded setpoints in a cascade control system.  Note that a low-select block may perform the same function as a high-limit, and that a high-select block may perform the same function as a low-limit.

\vskip 10pt

Provision should be made for halting integral wind-up in a controller whose output is overridden by a limiting or selecting function block.  This is one of the functions of the ``BKCAL'' signal in a FOUNDATION Fieldbus function block program: to let upstream function blocks ``know'' whether or not their outputs are still actively controlling anything downstream.






\vskip 20pt \vbox{\hrule \hbox{\strut \vrule{} {\bf Suggestions for Socratic discussion} \vrule} \hrule}

\begin{itemize}
\item{} Explain the purpose of the high-limit function in the cascaded furnace temperature control system.  Can you think of a solution to this problem that does not require modifying the control strategy from simple cascade?
\item{} Explain how the cascade furnace temperature control system uses a low-select function block to provide a high-limit function.
\item{} Explain what ``integral wind-up'' is, and why it can be a problem.
\item{} Explain how the two different solutions for ``integral wind-up'' in the cascaded furnace temperature control scheme work.
\item{} Explain how the {\tt BKCAL} signal provided in FOUNDATION Fieldbus systems helps to avoid ``integral wind-up'' in the master controller of a cascade control scheme.
\item{} Based on what you know of {\tt BKCAL} signal in FOUNDATION Fieldbus function blocks, explain how a PID function block might behave if its {\tt BKCAL} signal were left disconnected.
\item{} Describe different ways of mitigating ``integral wind-up'' when a limit function overrides the output of a PID controller.
\end{itemize}



%INDEX% Reading assignment: Lessons In Industrial Instrumentation, basic control strategies (limit controls)

%(END_NOTES)


