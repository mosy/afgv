
%(BEGIN_QUESTION)
% Copyright 2007, Tony R. Kuphaldt, released under the Creative Commons Attribution License (v 1.0)
% This means you may do almost anything with this work of mine, so long as you give me proper credit

Many consumer-grade electronic devices such as mobile telephones are not allowed inside explosion-hazardous (classified) industrial areas.  Explain in detail why this is.  What danger, specifically, would such a device pose?

\vskip 10pt

When an electronics manufacturer designs a device such as a mobile computer or telephone specifically for use in classified areas, how do you suppose they reduce the hazard posed by the device?  Specifically, {\it what} do they alter about its design to make it safe to use in a classified area?

\underbar{file i02475}
%(END_QUESTION)





%(BEGIN_ANSWER)

Obviously, some consumer-grade electronic devices are capable of producing sparks which could ignite a hazardous atmosphere.  What I'm looking for is an analysis (or an educated guess) of what specific things {\it inside} a typical consumer device such as a mobile phone, portable computer, or camera might cause a spark!

%(END_ANSWER)





%(BEGIN_NOTES)

Ideas include:

\begin{itemize}
\item{} Switch contacts opening/closing
\item{} Battery connectors coming loose
\item{} Flash bulbs (for photos) creating sparks
\end{itemize}

If your students own industrial-rated digital multimeters, have them read the ratings for their meters to see if there is mention of hazardous atmosphere use.

%INDEX% Safety, explosion: forbidding consumer-grade electronic devices in classified areas

%(END_NOTES)


