
%(BEGIN_QUESTION)
% Copyright 2007, Tony R. Kuphaldt, released under the Creative Commons Attribution License (v 1.0)
% This means you may do almost anything with this work of mine, so long as you give me proper credit

Some intrinsic safety barriers provide {\it galvanic isolation} in addition to voltage and current limiting.  Explain what ``isolation'' means in this context, and why it can be a useful addition to an instrument circuit.

\vskip 20pt \vbox{\hrule \hbox{\strut \vrule{} {\bf Suggestions for Socratic discussion} \vrule} \hrule}

\begin{itemize}
\item{} Explain how you could test a circuit for galvanic isolation.
\end{itemize}

\underbar{file i02476}
%(END_QUESTION)





%(BEGIN_ANSWER)

Intrinsically safe circuit grounding must be limited to a single point.  Otherwise, a ground loop would form which would have no current limiting.  Isolated barriers prevent the formation of ground loops.  I'll let you explain how they work!

%(END_ANSWER)





%(BEGIN_NOTES)

Galvanic isolation simply means there is no path for DC current from input to output of the barrier device.  This is accomplished either through transformer or optical isolation, allowing information (data) to pass from input to output but nothing else.  Isolation barriers are obviously active (not passive) devices, requiring much in the way of internal circuitry to drive the isolating components (e.g. transformer or optical transmit/receive pair).

Students should beware of simplified diagrams showing DC instrument signals going into and out of isolating transformers in these ``active'' barriers.  Obviously, DC signals cannot pass through passive transformers!  Modulating circuitry impresses the information conveyed by the DC signal onto an AC carrier signal which is then coupled electromagnetically through the transformer.  Demodulation circuitry on the other side strips away the carrier signal, leaving the original DC signal to go on to further portions of the loop.

This modulating circuitry is obviously active in nature (rather than passive), meaning that it requires an external source of power to function.  The power supply for an isolating barrier then becomes a potential point of failure for the system (no power, no amplification or modulation, therefore no instrument signal!) and must be considered from a standpoint of reliability.

%INDEX% Safety, intrinsic: isolating barriers

%(END_NOTES)


