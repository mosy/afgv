
%(BEGIN_QUESTION)
% Copyright 2011, Tony R. Kuphaldt, released under the Creative Commons Attribution License (v 1.0)
% This means you may do almost anything with this work of mine, so long as you give me proper credit

While working in an instrument/electrical shop at a chemical processing plant, a fellow technician approaches you and asks a question about intrinsic safety:

\vskip 10pt {\narrower \noindent \baselineskip5pt

``We need to install a 10 horsepower electric motor in the propane storage area, where it will be in a Class 1 Div 1 location.  The motor is not explosion-proof, so can we make it intrinsically safe by routing the power through an intrinsically safe barrier?''

\par} \vskip 10pt

How would you answer this technician?  Be sure to explain your reasoning.

\vskip 20pt

Next, you encounter another technician who tells you that a FOUNDATION Fieldbus H1 segment with 12 devices connected to it (each device drawing 10 milliamps of current) cannot be intrinsically safe, because Article 504 of the National Electrical Code (NFPA 70) specifies a ``simple apparatus'' as allowing no more than 100 milliamps of current.  How would you respond to this technician?  Be sure to explain your reasoning.

\vskip 20pt \vbox{\hrule \hbox{\strut \vrule{} {\bf Suggestions for Socratic discussion} \vrule} \hrule}

\begin{itemize}
\item{} Explain how the following proposal illustrates the absurdity of the first technician's idea: ``Let's just put a really {\it big} intrinsic safety barrier on the power lines feeding the entire plant, and that way {\it everything} inside the plant will be intrinsically safe!''
\end{itemize}

\underbar{file i02479}
%(END_QUESTION)





%(BEGIN_ANSWER)

Here's a hint: both technicians are wrong.

%(END_ANSWER)





%(BEGIN_NOTES)

The first technician's question reveals a fundamental misconception about intrinsic safety.  The point of intrinsic safety, and intrinsic safety barrier circuits by inclusion, is that electrical power to the field device is limited to a very low level that cannot possibly ignite a flammable atmosphere.  A 10 horsepower motor will need more than 7460 watts of power to run, which clearly makes intrinsic safety an impossibility.

Incidentally, I was once told by a fellow technician while working at an oil refinery that such a thing (powering an AC motor through an intrinsic safety barrier) was possible!  He did not ask me for my opinion; he actually {\it argued} that this would work, as if an intrinsic safety barrier is some sort of magic device that prevents explosions regardless of the application.  What I should have said in response is, ``why not just put a really big intrinsic safety barrier on the power lines coming in to the refinery?  That way {\it everything} will be protected and we can never have an explosion from an electrical fault!'' and let cognitive dissonance do the rest.

\vskip 10pt

The second technician is confusing a ``simple apparatus'' with a powered intrinsically safe circuit.  A ``simple apparatus'' is a device {\it generating} its own power, unlike a Fieldbus H1 segment which receives power from a remotely-located power supply.  The 100 mA limit in Article 504 applies only to these self-powered devices.  Current limits for other intrinsically safe circuits vary depending on the environment (Group) of the explosive hazard, and typically exceed 100 mA.

%INDEX% Safety, intrinsic: limited to low-power instrumentation

%(END_NOTES)


