
%(BEGIN_QUESTION)
% Copyright 2007, Tony R. Kuphaldt, released under the Creative Commons Attribution License (v 1.0)
% This means you may do almost anything with this work of mine, so long as you give me proper credit

In May 2007, a major instrument manufacturer placed an advertisement in an industry periodical touting the reliability of a new wireless instrument product line.  According to the add, the data reliability for this wireless (radio) instrumentation product was ``greater than 99\%''.  Bear in mind that {\it reliability} is nothing more than a probability value that a device or a system will faithfully perform its design function.  A system that is 99\% reliable is one that properly performs its design function 99 times out of 100.

\vskip 10pt

Sounds pretty good, doesn't it?  It might, until you put this probability into perspective.  Calculate what a 99.5\% reliability rate would mean in the context of:

\begin{itemize}
\item{} Reliability of electric power delivered to your home (how much outage in one year's time).
\item{} How often the brakes would fail in your automobile (assuming 10 instances of using the brakes per short trip, and the number of short trips you take in your car during one year).
\item{} The number of skipped heart beats in an average day.
\end{itemize}

\underbar{file i02480}
%(END_QUESTION)





%(BEGIN_ANSWER)

\noindent
{\bf Partial answer:}

\vskip 10pt

At a reliability rate of 99.5\%, your home would be out of power almost 2 whole days per year (43.8 hours per year), {\it every year}.

%(END_ANSWER)





%(BEGIN_NOTES)

Reliability of power to home.  365 days in a year.  99.5\% of 365 is 363.175 days.  The complement of this is 1.825 days, or 43.8 hours without power during a year.

\vskip 10pt

The brake failure rate is a real eye-opener.  Assuming each student uses the brakes 10 times during each trip to school, and 10 times each trip back (20 braking events per school day per vehicle), and an average of 50 days per quarter (150 days for a three-quarter school year), the number of brake failures is rather staggering.  At 20 braking events per school day per vehicle and 150 school days in a year, I calculate each student suffering 15 brake failures per school year, and this is just during the trips to and from school, not anywhere else!

\vskip 10pt

Assuming an average of 70 beats per minute, the number of heartbeats in a full 24 hour day is 100,800.  99.5\% of this is 100,296 successful beats.  The complement is 504 beats, which would mean 504 incomplete or skipped heartbeats per day!

%INDEX% Safety, system reliability: percentage calculation

%(END_NOTES)


