
%(BEGIN_QUESTION)
% Copyright 2011, Tony R. Kuphaldt, released under the Creative Commons Attribution License (v 1.0)
% This means you may do almost anything with this work of mine, so long as you give me proper credit

Read and outline the introduction to the ``Limit, Selector, and Override Controls'' section of the ``Basic Process Control Strategies'' chapter in your {\it Lessons In Industrial Instrumentation} textbook.  Note the page numbers where important illustrations, photographs, equations, tables, and other relevant details are found.  Prepare to thoughtfully discuss with your instructor and classmates the concepts and examples explored in this reading.

\underbar{file i02482}
%(END_QUESTION)





%(BEGIN_ANSWER)


%(END_ANSWER)





%(BEGIN_NOTES)

Function blocks capable of selecting between different signal values based on magnitude.  Greater-than symbol means ``high select'' while less-than symbol means ``low select.''  Placement of input lines has nothing whatsoever to do with which signal gets selected -- only the signal values matter.  High-limit (greater-than with a vertical line) prevents a signal from exceeding a pre-set value.  Low-limit (less-than with a vertical line) prevents a signal from going below a pre-set value.  Rate limit functions (symbols look like high-limit and low-limit turned sideways) prevent a signal from exceeding a pre-set rate-of-change over time.





\vskip 20pt \vbox{\hrule \hbox{\strut \vrule{} {\bf Suggestions for Socratic discussion} \vrule} \hrule}

\begin{itemize}
\item{} Describe the difference between a {\it low-select} function and a {\it low-limit} function.
\item{} Describe the difference between a {\it high-select} function and a {\it high-limit} function.
\item{} Demonstrate how a {\it low-select} function could be used as a {\it high-limit}.
\item{} Demonstrate how a {\it high-select} function could be used as a {\it low-limit}.
\item{} Sketch a diagram showing a low-select function with input signals of 41\% and 64\%, and determine its output signal value.
\item{} Sketch a diagram showing a low-select function with input signals of 18\% and 74\%, and determine its output signal value.
\item{} Sketch a diagram showing a high-select function with input signals of 22\% and 35\%, and determine its output signal value.
\item{} Sketch a diagram showing a high-select function with input signals of 74\% and 92\%, and determine its output signal value.
\end{itemize}

%INDEX% Reading assignment: Lessons In Industrial Instrumentation, basic control strategies (limit, selector, and override controls introduction)

%(END_NOTES)


