
%(BEGIN_QUESTION)
% Copyright 2007, Tony R. Kuphaldt, released under the Creative Commons Attribution License (v 1.0)
% This means you may do almost anything with this work of mine, so long as you give me proper credit

When I was a young child, pondering the concepts of right and wrong, I noticed a friend of mine based his own ethical decisions on a simple formula: his decision to do something punishable by adults was inversely proportional to both the risk of being caught and the severity of the punishment.  Thus, if the risk of getting caught was small, and/or the punishment was not severe, he would do whatever he wanted.

\vskip 10pt

Despite the ethical shortcomings of my young friend's approach, there is a certain practicality to it.  Risk analysis for industrial process systems follows a similar formula.  Two vital factors to consider when assessing the risk posed by an industrial process is:

\begin{itemize}
\item{$1.$} The probability of the dangerous event occurring
\item{$2.$} The severity of the danger
\end{itemize}

The following tables quantify probability and severity on a simple number scale:

% No blank lines allowed between lines of an \halign structure!
% I use comments (%) instead, so that TeX doesn't choke.

$$\vbox{\offinterlineskip
\halign{\strut
\vrule \quad\hfil # \ \hfil & 
\vrule \quad\hfil # \ \hfil & 
\vrule \quad\hfil # \ \hfil \vrule \cr
\noalign{\hrule}
%
% First row
Risk level & Description & Frequency \cr
%
\noalign{\hrule}
%
% Another row
1 & Improbable & Once per 10,000 yrs \cr
%
\noalign{\hrule}
%
% Another row
2 & Remote & Once per 1,000 yrs \cr
%
\noalign{\hrule}
%
% Another row
3 & Occasional & Once per 100 yrs \cr
%
\noalign{\hrule}
%
% Another row
4 & Probable & Once per 10 yrs \cr
%
\noalign{\hrule}
%
% Another row
5 & Frequent & Once per yr \cr
%
\noalign{\hrule}
} % End of \halign 
}$$ % End of \vbox



% No blank lines allowed between lines of an \halign structure!
% I use comments (%) instead, so that TeX doesn't choke.

$$\vbox{\offinterlineskip
\halign{\strut
\vrule \quad\hfil # \ \hfil & 
\vrule \quad\hfil # \ \hfil & 
\vrule \quad\hfil # \ \hfil \vrule \cr
\noalign{\hrule}
%
% First row
Risk level & Description & Consequences \cr
%
\noalign{\hrule}
%
% Another row
1 & Negligible & No injuries \cr
%
\noalign{\hrule}
%
% Another row
2 & Minor & Medical treatment \cr
%
\noalign{\hrule}
%
% Another row
3 & Serious & Lost-time accident \cr
%
\noalign{\hrule}
%
% Another row
4 & Severe & Death \cr
%
\noalign{\hrule}
%
% Another row
5 & Catastrophic & Multiple deaths \cr
%
\noalign{\hrule}
} % End of \halign 
}$$ % End of \vbox

Determine the most logical way to combine the two ratings to form a single numerical ``risk factor'' describing an industrial event.

\underbar{file i02483}
%(END_QUESTION)





%(BEGIN_ANSWER)

Multiplication makes the most sense, resulting in a number ranging between 1 and 25.  Some might think addition is the proper way to mathematically combine the two risk figures, but multiplication makes more sense especially when you note the exponential scale of frequency.  Now, addition {\it would} be most appropriate if the result was taken to mean a power of ten!

\vskip 10pt

The Emerson (Fisher/Rosemount) corporation gives the following suggestions for relating SIL level requirements with risk product:

\begin{itemize}
\item{} SIL 1 = risk product between 1 and 6
\item{} SIL 2 = risk product between 6 and 15
\item{} SIL 3 = risk product between 15 or greater
\end{itemize}

%(END_ANSWER)





%(BEGIN_NOTES)


%INDEX% Safety, analysis: risk level frequency
%INDEX% Safety, analysis: risk level severity

%(END_NOTES)


