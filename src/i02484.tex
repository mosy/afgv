
%(BEGIN_QUESTION)
% Copyright 2011, Tony R. Kuphaldt, released under the Creative Commons Attribution License (v 1.0)
% This means you may do almost anything with this work of mine, so long as you give me proper credit

Read the report ``Improving Thermocouple Service Life in Slagging Gasifiers'' written by James Bennett of the US Department of Energy (DOE) Albany Research Center (document DOE/ARC-2005-043), and answer the following questions:

\vskip 10pt

What exactly does a {\it gasifier} do?  What ``feedstocks'' does it input, and what products does it output?  How do these facts make gasification a technology of interest for future energy production?

\vskip 10pt

Identify the typical pressures and temperatures encountered in the interior of a slagging gasifier.  Express these pressure and temperature ranges in {\it bar} and $^{o}$F, respectively.

\vskip 10pt

What type of thermocouple has traditionally been applied to this service?  Explain why this choice is more appropriate than some other thermocouple types.

\vskip 10pt

How long have thermocouples typically lasted in this type of severe service, and how much does each one cost to replace?

\vskip 10pt

One of the alternative temperature-measurement technologies cited in this report is the use of sapphire process probes with fiber-optic cables to conduct light to a remote location where it may be sensed.  Explain how this novel temperature-measurement technology works, based on the description alone (and/or your own independent research on the topic).

\vskip 10pt

An interesting style of thermocouple experimented with for gasifier applications is the so-called {\it open thermocouple}.  Explain how this sensing device is constructed, and how it differs from a regular thermocouple.

\vskip 10pt

Identify some of the specific modes of thermocouple failure (i.e. how exactly is it that they are failing?).

\vskip 10pt

Explain how ``bridging'' inside a gasifier may lead to incorrect thermocouple EMFs (millivolt signals).

\vskip 20pt \vbox{\hrule \hbox{\strut \vrule{} {\bf Suggestions for Socratic discussion} \vrule} \hrule}

\begin{itemize}
\item{} Examine the graph of thermocouple readings over a 3-hour test period as shown in figure 10.  Which slag thermocouple performed better as the sample warmed up?  Which slag thermocouple performed better after the sample had reached 880 $^{o}$C?
\item{} Explain how ``syngas analysis'' may be alternatively used to infer temperatures inside the gasifier, and why this technique is not suitable for all phases of gasifier operation.
\item{} Explain chemical interactions between thermocouple wires and compounds inside the gasifier can lead to measurement errors, even before complete failure of the thermocouple.
\item{} Explain what it means to {\it sequester} carbon dioxide gas from a process, and why this is important from the perspective of environmental impact.
\item{} Under what operating conditions may a gasifier be {\it carbon-neutral} (i.e. no net release of carbon dioxide gas to the atmosphere)?  
\item{} Under what operating conditions may a gasifier be {\it carbon-negative} (i.e. it removes more carbon dioxide gas from the atmosphere than it releases)?  
\end{itemize}

\underbar{file i02484}
%(END_QUESTION)





%(BEGIN_ANSWER)


%(END_ANSWER)





%(BEGIN_NOTES)

Gasifiers use high temperatures to break down carbon-containing fuel feedstocks (biomass, coal, woodpulp liquor) into heat and useful chemicals such as hydrogen, methane, and carbon monoxide gas.  Since all gas streams are closed (not vented to atmosphere in a stack), it is relatively easy to separate, capture, and sequester the CO$_{2}$ gas to make it carbon-neutral (feed = coal) or even carbon-negative (feed = biomass or woodpulp liquor).

\vskip 10pt

Typical pressure is up to 1000 PSI (68.95 bar), with temperatures ranging from 1250 to 1550 degrees Celsius (2282 to 2822 degrees Fahrenheit).  The metal reactor vessel is lined with refractory material (typically compounds with a high concentration of chromium oxides) to withstand the high internal temperatures of gasification.

\vskip 10pt

{\it Type S} (platinum/rhodium) thermocouples have traditionally been the thermocouple of choice for slagging gasifiers, due to their extremely high temperature range and relative resistance to corrosion (page 2).  The thermocouples are encased in a castable refractory thermowell (``thermocouple protection tube'') for protection against the slag inside the reactor.

\vskip 10pt

Typical thermocouple life has been only 40 to 90 days (first paragraph of report).  The cost of each is cited in this report as \$2500 (page 3).  On page 3 we are told that improvements to controls and sensor measurements on the order of just 1 percent would potentially save \$409 million annually in the fuel costs for a slagging gasifier, and up to 5000 megawatts more power generated industry-wide(!).

\vskip 10pt

Sapphire crystals are briefly mentioned in this report in conjunction with fiber optics, suggesting an optical method of temperature measurement.  For more detail, research outside of this report is necessary.

According to the DOE paper ``Single-Crystal Sapphire Optical Fiber Sensor Instrumentation for Coal Gasifiers'', sapphire crystals may function as temperature-sensing elements by producing dual reflections of an incident light wave (one reflection off the front face of a sapphire crystal, another reflection off the back face).  The interference patterns created by the two reflected light waves may be then used to measure the {\it optical thickness} (``OT'') of the sapphire crystal, which in turn is a function of crystal temperature.  This is known as the {\it Fabry-Perot interferometry method}.  As the sapphire crystal heats, both its thickness and its index of refraction change with temperature, causing a definite and repeatable phase shift in the light waves measurable by an external instrument coupled to the sapphire crystal via fiber optic cabling.  These sapphire temperature sensors have exhibited very long service lives compared to thermocouples, with one example lasting 7 months in a slagging gasifier (according to this Sapphire Sensor report).

\vskip 10pt

Open thermocouples have no welded junction at their tip, but rather just two open-ended wires contacting the process medium.  The process medium itself joins the wires and comprises the measurement junction.

\vskip 10pt

Thermocouples may fail due to chemical attack from the slag and from gases such as H$_{2}$S and SO$_{x}$, from shear forces as the wall components of the gasifier (refractory, steel) shift position, and/or from fabrication defects.  Figure 5 on page 5 lists several mechanisms of failure.  Chemical corrosion is thought to be the dominant failure mode at present (page 7).  Chemical amalgamation between iron and the platinum/rhodium of a thermocouple may alter the characteristics of the thermocouple, causing erroneous readings.

\vskip 10pt

``Bridging'' is when foreign material such as slag or carbon creates a parallel resistance across the thermocouple, effectively loading the voltage source and decreasing its output signal.  Several forms of bridging are shown in table 2 on page 8.





\vfil \eject

\noindent
{\bf Prep Quiz:}

Identify the purpose of a {\it gasifier} as described in the DOE report.

\begin{itemize}
\item{} Boils water into a gas phase so it can power a steam turbine
\vskip 5pt
\item{} Captures CO$_{2}$ gas from the atmosphere for sequestration
\vskip 5pt 
\item{} Converts coal into valuable fuels and chemical feedstocks
\vskip 5pt 
\item{} Chemically converts natural gas into value-added liquid fuels
\vskip 5pt 
\item{} Senses temperature using sapphire crystals and fibers
\vskip 5pt 
\item{} Stabilizes the voltage of a thermocouple for better accuracy
\end{itemize}




%INDEX% Process: biomass gasification (slagging gasifier)
%INDEX% Process: coal gasification (slagging gasifier)
%INDEX% Reading assignment: DOE research report on thermocouple failures in slagging gasifier systems

%(END_NOTES)


