
%(BEGIN_QUESTION)
% Copyright 2010, Tony R. Kuphaldt, released under the Creative Commons Attribution License (v 1.0)
% This means you may do almost anything with this work of mine, so long as you give me proper credit

\noindent
{\bf Programming Challenge and Comparison -- Motor control with HMI and pushbutton} 

\vskip 10pt

Suppose we need to have a PLC provide start/stop control for an electric motor, from two different locations.  Near the motor we have a pair of momentary-contact pushbutton switches: one to start the motor and another to stop it.  However, at the control room we have an HMI panel where the operators would like to have an additional set of momentary start and stop ``pushbuttons'' so they may start and stop the motor from that location as well.

\vskip 10pt

Write a PLC program integrating these two sets of start/stop controls together, so that the motor may be controlled from either location.  Also provide a ``run'' indicator on the HMI panel so operators there know when the motor is actually running.  Demonstrate the program's operation using switches connected to its inputs to simulate the discrete inputs in a real application.  

\vskip 10pt

When your program is complete and tested, capture a screen-shot of it as it appears on your computer, and prepare to present your program solution to the class in a review session for everyone to see and critique.  The purpose of this review session is to see multiple solutions to one problem, explore different programming techniques, and gain experience interpreting PLC programs others have written.  When presenting your program (either individually or as a team), prepare to discuss the following points:

\begin{itemize}
\item{} Identify the ``tag names'' or ``nicknames'' used within your program to label I/O and other bits in memory
\item{} Follow the sequence of operation in your program, simulating the system in action
\item{} Identify any special or otherwise non-standard instructions used in your program, and explain why you decided to take that approach
\item{} Show the comments placed in your program, to help explain how and why it works
\item{} How you designed the program (i.e. what steps you took to go from a concept to a working program)
\end{itemize}


\vskip 20pt \vbox{\hrule \hbox{\strut \vrule{} {\bf Suggestions for Socratic discussion} \vrule} \hrule}

\begin{itemize}
\item{} Why would it be a bad idea for the HMI to write to the same PLC addresses used for discrete inputs (e.g. {\tt X1}, {\tt I:0/0}, {\tt I0.0}, etc.) in an attempt to have dual control over the motor?  What bits in the PLC memory should the HMI write to in order to avoid this problem?
\item{} How may we ensure consistent action at the motor if conflicting commands are given from the pushbuttons and HMI (e.g. ``Start'' pushbutton pressed while ``Stop'' HMI control activated)?
\end{itemize}


\underbar{file i02485}
%(END_QUESTION)





%(BEGIN_ANSWER)


%(END_ANSWER)





%(BEGIN_NOTES)

\vfil \eject

\noindent
{\bf Summary Quiz:}

(The recommended summary quiz is to have \underbar{each student} demonstrate their PLCs running this particular program)

%INDEX% PLC, programming challenge: motor start/stop control with dual HMI/pushbutton controls 

%(END_NOTES)


