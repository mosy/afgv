
%(BEGIN_QUESTION)
% Copyright 2011, Tony R. Kuphaldt, released under the Creative Commons Attribution License (v 1.0)
% This means you may do almost anything with this work of mine, so long as you give me proper credit

\noindent
{\bf Programming Challenge and Comparison -- Positive displacement flowmeter rate} 

\vskip 10pt

A common design of flowmeter for residential water flow measurement is the {\it positive displacement} design, where the movement of water volume through the meter causes a mechanism to rotate, passing a known and fixed quantity of water volume through the meter for each revolution.  The rotation of the flowmeter mechanism may be electrically transmitted by a magnetic reed switch actuated by a magnet on the flowmeter mechanism's rotating shaft.  Actuating (closed and opened) one cycle per revolution, the reed switch produces a pulse signal representing a known and fixed measurement of water volume per switch ``pulse.''

\vskip 10pt

Write a PLC program continuously calculating the flow rate of water through such a meter, given a meter factor of 1 gallon per switch pulse.  The calculated flowrate needs to be displayed on an HMI, units of ``GPM'' (gallons per minute). 

\vskip 10pt

When your program is complete and tested, capture a screen-shot of it as it appears on your computer, and prepare to present your program solution to the class in a review session for everyone to see and critique.  The purpose of this review session is to see multiple solutions to one problem, explore different programming techniques, and gain experience interpreting PLC programs others have written.  When presenting your program (either individually or as a team), prepare to discuss the following points:

\begin{itemize}
\item{} Identify the ``tag names'' or ``nicknames'' used within your program to label I/O and other bits in memory
\item{} Follow the sequence of operation in your program, simulating the system in action
\item{} Identify any special or otherwise non-standard instructions used in your program, and explain why you decided to take that approach
\item{} Show the comments placed in your program, to help explain how and why it works
\item{} How you designed the program (i.e. what steps you took to go from a concept to a working program)
\end{itemize}

\vfil 

\vskip 20pt \vbox{\hrule \hbox{\strut \vrule{} {\bf Suggestions for Socratic discussion} \vrule} \hrule}

\begin{itemize}
\item{} How can you write your program to update the flow calculation more often than once per minute?
\item{} One way to calculate flow is to count the number of gallons passed in one minute of time.  Is there a way to calculate inversely: determining flow rate by measuring the amount of time elapsed between pulses?
\end{itemize}

\underbar{file i02486}
\eject
%(END_QUESTION)





%(BEGIN_ANSWER)


%(END_ANSWER)





%(BEGIN_NOTES)

\vfil \eject

\noindent
{\bf Summary Quiz:}

(The recommended summary quiz is to have \underbar{each student} demonstrate their PLCs running this particular program)

%INDEX% PLC, programming challenge: positive displacement flowmeter rate 

%(END_NOTES)


