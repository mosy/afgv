
%(BEGIN_QUESTION)
% Copyright 2011, Tony R. Kuphaldt, released under the Creative Commons Attribution License (v 1.0)
% This means you may do almost anything with this work of mine, so long as you give me proper credit

Match the best data type to each of the following industrial measurement applications.  Note that more than one of the answers may be identical (i.e. same data type for multiple applications), and that it is possible one or more data types listed will not be used:

\begin{itemize}
\item{} (A) Signed integer
\item{} (B) Floating-point
\item{} (C) ASCII
\item{} (D) Unsigned integer
\item{} (E) Discrete
\end{itemize}

\begin{itemize}
\item{} A text message to be sent when a certain condition is detected by the PLC: \underbar{\hskip 50pt}
\vskip 10pt
\item{} Measuring the time elapsed between motor start events, in whole seconds: \underbar{\hskip 50pt}
\vskip 10pt
\item{} Displaying the status of a pressure switch: \underbar{\hskip 50pt}
\vskip 10pt
\item{} Measuring the rotational speed of an electric motor, in RPM: \underbar{\hskip 50pt}
\vskip 10pt
\item{} Showing the difference between ideal weight and actual weight of a produced brick, in whole ounces: \underbar{\hskip 50pt}
\end{itemize}

\underbar{file i02454}
%(END_QUESTION)





%(BEGIN_ANSWER)

\begin{itemize}
\item{} A text message to be sent when a certain condition is detected by the PLC: \underbar{\bf C}
\vskip 10pt
\item{} Measuring the time elapsed between motor start events, in whole seconds: \underbar{\bf D}
\vskip 10pt
\item{} Displaying the status of a pressure switch: \underbar{\bf E}
\vskip 10pt
\item{} Measuring the rotational speed of an electric motor, in RPM: \underbar{\bf A}, \underbar{\bf B}, or \underbar{D}
\vskip 10pt
\item{} Showing the difference between ideal weight and actual weight of a produced brick, in whole ounces: \underbar{\bf A}
\end{itemize}


%(END_ANSWER)





%(BEGIN_NOTES)

{\bf This question is intended for exams only and not worksheets!}.

%(END_NOTES)

