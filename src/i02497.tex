
%(BEGIN_QUESTION)
% Copyright 2011, Tony R. Kuphaldt, released under the Creative Commons Attribution License (v 1.0)
% This means you may do almost anything with this work of mine, so long as you give me proper credit

Perform an experiment where you roll a set of dice to simulate a test of component reliability.  Each die represents a single component being tested, with a random chance of failure following each roll.  This experiment works better the more dice you have to roll (10 dice is a good minimum).  

The purpose of this experiment is to contrast the scenarios of no component replacement versus immediate repair of any failed components, and also to gain a more intuitive understanding of MTBF (or MTTF).  Count any die landing on ``1'' as a failed component, and landing on any other number as a surviving component.

\vskip 10pt

First, perform successive rolls, removing any failed components after each roll so that the group of failed components grows over time while the group of surviving components (the only dice being rolled the next throw) dwindles over time.  Record the sizes of the ``failed'' and ``surviving'' groups after each roll.  For example:

% No blank lines allowed between lines of an \halign structure!
% I use comments (%) instead, so that TeX doesn't choke.

$$\vbox{\offinterlineskip
\halign{\strut
\vrule \quad\hfil # \ \hfil & 
\vrule \quad\hfil # \ \hfil & 
\vrule \quad\hfil # \ \hfil \vrule \cr
\noalign{\hrule}
%
% First row
{\bf Roll} & {\bf Total number failed} & {\bf Number surviving} \cr
%
\noalign{\hrule}
%
% Another row
0 & 0 & 15 \cr
%
\noalign{\hrule}
%
% Another row
1 & 2 & 13 \cr
%
\noalign{\hrule}
%
% Another row
2 & 3 & 12 \cr
%
\noalign{\hrule}
%
% Another row
3 & 5 & 10 \cr
%
\noalign{\hrule}
} % End of \halign 
}$$ % End of \vbox

Plot the results on a graph, noting the number of rolls required until 63.2\% of the original components had failed.

\vskip 10pt

Next, perform successive rolls, ``repairing'' any failed components after each roll so that each roll begins with the same number of total components.  Record an accumulating total of ``failures'' after each roll.  For example:

% No blank lines allowed between lines of an \halign structure!
% I use comments (%) instead, so that TeX doesn't choke.

$$\vbox{\offinterlineskip
\halign{\strut
\vrule \quad\hfil # \ \hfil & 
\vrule \quad\hfil # \ \hfil & 
\vrule \quad\hfil # \ \hfil \vrule \cr
\noalign{\hrule}
%
% First row
{\bf Roll} & {\bf Total number failed} & {\bf Number surviving} \cr
%
\noalign{\hrule}
%
% Another row
0 & 0 & 15 \cr
%
\noalign{\hrule}
%
% Another row
1 & 2 & 15 \cr
%
\noalign{\hrule}
%
% Another row
2 & 4 & 15 \cr
%
\noalign{\hrule}
%
% Another row
3 & 7 & 15 \cr
%
\noalign{\hrule}
} % End of \halign 
}$$ % End of \vbox

Plot the results on a graph, noting the number of rolls required until the accumulated number of failed components equals the total number of components maintained.

\underbar{file i02497}
%(END_QUESTION)





%(BEGIN_ANSWER)


%(END_ANSWER)





%(BEGIN_NOTES)


%INDEX% Mathematics, probability: flipping coins and rolling dice

%(END_NOTES)


