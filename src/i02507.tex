
%(BEGIN_QUESTION)
% Copyright 2007, Tony R. Kuphaldt, released under the Creative Commons Attribution License (v 1.0)
% This means you may do almost anything with this work of mine, so long as you give me proper credit

A common misconception is the assumed equivalence between the MTBF and the {\it expected life} of a component or device.  Just because a device has an MTBF of 4500 hours does not mean it is guaranteed to function for 4500 hours.

The equation shown here relates the reliability ($R$) of a device to its continuous operating time and MTBF.  Like all probabilities, $R$ is a unitless ratio between 0 and 1, inclusive:

$$R = e^{-{t \over m}}$$

\noindent
Where,

$R$ = Reliability of the device at time $t$

$t$ = Continuous operating time of the device

$m$ = Mean Time Between Failures (MTBF)

\vskip 10pt

Suppose you are considering the reliability of a control valve positioner with a calculated MTBF of 100,000 hours.  Use the above formula to predict the following:

\begin{itemize}
\item{} The positioner's reliability after 3 years of continuous operation
\vskip 5pt 
\item{} The age at which the positioner's reliability decreases to 0.4
\vskip 5pt 
\item{} The age at which the positioner's reliability is 1 (guaranteed success)
\vskip 5pt 
\item{} The age at which the positioner's reliability is 0 (guaranteed failure)
\end{itemize}

\vskip 20pt \vbox{\hrule \hbox{\strut \vrule{} {\bf Suggestions for Socratic discussion} \vrule} \hrule}

\begin{itemize}
\item{} If MTBF is not the same as ``expected life,'' then what value does it have for anyone?
\end{itemize}

\underbar{file i02507}
%(END_QUESTION)





%(BEGIN_ANSWER)

\noindent
{\bf Partial answer:}

\begin{itemize}
\item{} The positioner's reliability after 3 years of continuous operation = {\bf 0.7689}
\vskip 5pt 
\item{} The age at which the positioner's reliability decreases to 0.4 
\vskip 5pt 
\item{} The age at which the positioner's reliability is 1 (guaranteed success)
\vskip 5pt 
\item{} The age at which the positioner's reliability is 0 (guaranteed failure) = $\infty$
\end{itemize}

%(END_ANSWER)





%(BEGIN_NOTES)

\begin{itemize}
\item{} The positioner's reliability after 3 years of continuous operation = {\bf 0.7689}
\vskip 5pt 
\item{} The age at which the positioner's reliability decreases to 0.4 = {\bf 91,629 hours}, or about {\bf 10.46 years} 
\vskip 5pt 
\item{} The age at which the positioner's reliability is 1 (guaranteed success) = {\bf 0 hours} (brand-new)
\vskip 5pt 
\item{} The age at which the positioner's reliability is 0 (guaranteed failure) = $\infty$
\end{itemize}

\vskip 10pt

You can see from the equation that the MTBF is a sort of ``time constant'' for reliability rather than being a typical (or even maximum!) expected age.

%INDEX% Safety, system reliability: expected life
%INDEX% Safety, system reliability: Mean Time Between Failures (MTBF) or Mean Time To Failure (MTTF)

%(END_NOTES)


