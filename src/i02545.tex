
%(BEGIN_QUESTION)
% Copyright 2012, Tony R. Kuphaldt, released under the Creative Commons Attribution License (v 1.0)
% This means you may do almost anything with this work of mine, so long as you give me proper credit

Look in the manufacturer's documentation for information on the analog input channel(s) for your PLC.  You will need to identify this important information before connecting any signal source to your PLC's analog input!

\vskip 10pt

\begin{itemize}
\item{} What is the absolute maximum voltage that your PLC's analog input can withstand?
\vskip 5pt
\item{} Determine how you may wire a potentiometer to the analog input of your PLC in order to test that input.
\vskip 5pt
\item{} Design a circuit with a potentiometer, where the variable voltage is limited to a value that {\it cannot} exceed the PLC's analog input rating no matter what position the potentiometer is set to.  {\it Hint: include some fixed-value resistors in the potentiometer circuit!}
\vskip 5pt
\item{} Identify the ``full count'' number value produced by the PLC's analog-to-digital converter (ADC) when it is receiving a maximum-scale voltage signal.
\vskip 5pt
\item{} Identify which register in the PLC's memory this ``count'' value is written to, and what format it is in (e.g. signed integer, floating-point, etc.).
\end{itemize}

\vskip 10pt

If your PLC does not have any analog input channels, you will need to partner with a classmate whose PLC does have analog inputs.

\underbar{file i02545}
%(END_QUESTION)





%(BEGIN_ANSWER)


%(END_ANSWER)





%(BEGIN_NOTES)


%INDEX% PLC, I/O: analog I/O device wiring

%(END_NOTES)


