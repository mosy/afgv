
%(BEGIN_QUESTION)
% Copyright 2012, Tony R. Kuphaldt, released under the Creative Commons Attribution License (v 1.0)
% This means you may do almost anything with this work of mine, so long as you give me proper credit

\noindent
%{\bf Programming Challenge -- analog input scaling using math instructions} 
Programeringsoppgave

\vskip 10pt

Potmeteret p{\aa} simulatorstasjonen g{\aa}r simulerer en m{\aa}lt temperatur fra 0-70�C, skaler inngangsignalet slik at det g{\aa}r fra 0-700. (Utfordring: Finner du en mulighet for {\aa} vise det som 0.0-70.0)

%When any PLC receives an analog voltage or current signal from some device (such as a sensor or a potentiometer), the number value generated by the PLC's analog-to-digital converter (ADC) will be proportional to that signal, but not identical to it.  For example, a PLC receiving a 5.00 volt DC signal may yield an ADC ``count'' value of 32,767 (equivalent to the binary number {\tt 0111111111111111}).  Usually the DC signal represents some physical measurement, such as machine position, temperature, pressure, speed, etc.  Our desire is to have the PLC ``scale'' the ADC count value into a number range representing that real-world quantity, which means we must program the PLC to mathematically manipulate the ADC's count value into a number range more meaningful to us.

\vskip 10pt

%A common method for doing this is to program the PLC to evaluate a linear equation of the form $y = mx + b$, where $x$ is the raw ADC count value and $y$ is the scaled quantity the analog signal represents.  For instance, if our PLC receives an analog voltage signal ranging 0 to 5.00 volts, converting that voltage into a count value ranging 0 to 32,767, that voltage in turn representing the temperature detected by a sensor over an equivalent temperature range of 30 degrees to 100 degrees Fahrenheit, we may sketch a linear graph of the count-to-temperature relationship and develop a linear equation expressing it:


%Program your PLC to convert the full range of the analog input count value into a 30 to 100 degree Fahrenheit scaled temperature value, using only simple math instructions (e.g. multiplication, division, addition, subtraction).  Note that you must derive your own custom $y = mx + b$ equation for your PLC, because the ADC count value range will likely not be 0 to 32767 as it was in this example.

\underbar{file i02546no}
%(END_QUESTION)





%(BEGIN_ANSWER)


%(END_ANSWER)





%(BEGIN_NOTES)


%INDEX% PLC, I/O: analog resolution and scaling

%(END_NOTES)


