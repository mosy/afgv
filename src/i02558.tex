
%(BEGIN_QUESTION)
% Copyright 2015, Tony R. Kuphaldt, released under the Creative Commons Attribution License (v 1.0)
% This means you may do almost anything with this work of mine, so long as you give me proper credit

Read and outline the ``Determining The Design Purpose Of Override Controls'' subsection of the ``Techniques For Analyzing Control Strategies'' section of the ``Basic Process Control Strategies'' chapter in your {\it Lessons In Industrial Instrumentation} textbook.  Note the page numbers where important illustrations, photographs, equations, tables, and other relevant details are found.  Prepare to thoughtfully discuss with your instructor and classmates the concepts and examples explored in this reading.

\underbar{file i02558}
%(END_QUESTION)





%(BEGIN_ANSWER)


%(END_ANSWER)





%(BEGIN_NOTES)

The following procedure was illustrated for analyzing any override control system:

\begin{itemize}
\item{$(1)$} Identify the actions of each controller (i.e. direct or reverse) by considering each one at a time with the assumption that it is being selected to manipulate the FCE. 
\item{$(2)$} Identify the high/low selection action of each selector function.
\item{$(3)$} Based on the high/low selection action and the direct/reverse controller action, determine what PV condition (high or low) causes that controller to be selected over another.  {\it This is the condition the controller exists to regulate!}
\end{itemize}








\vskip 20pt \vbox{\hrule \hbox{\strut \vrule{} {\bf Suggestions for Socratic discussion} \vrule} \hrule}

\begin{itemize}
\item{} 
\item{} 
\item{} 
\item{} 
\item{} 
\end{itemize}


%INDEX% Reading assignment: Lessons In Industrial Instrumentation, basic control strategies (override controls)

%(END_NOTES)


