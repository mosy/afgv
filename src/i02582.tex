
%(BEGIN_QUESTION)
% Copyright 2012, Tony R. Kuphaldt, released under the Creative Commons Attribution License (v 1.0)
% This means you may do almost anything with this work of mine, so long as you give me proper credit

A sample of ``table'' salt (sodium chloride, NaCl) has a mass of 10 kg.  How many {\it moles} of salt is this equal to?

\underbar{file i02582}
%(END_QUESTION)





%(BEGIN_ANSWER)

``Table'' salt is sodium chloride (NaCl), with 1 atom of sodium bound to 1 atom of chlorine.  Together, the number of atomic mass units (amu) for each sodium chloride molecule is the sum of the individual atoms' atomic masses:

\vskip 10pt

\medskip 
\item{} {\it Each molecule of NaCl contains:}
\item{} 1 atom of Na at 22.99 amu each
\item{} 1 atom of Cl at 35.45 amu each
\end{itemize} 

[(1 atom)(22.99 amu/atom) + (1 atom)(35.45 amu/atom)] = 58.44 g per mole of NaCl

\vskip 10pt

Since we now know the number of grams per mole for NaCl, we may calculate the number of moles needed to make 10 kg (10000 g) of salt:

\vskip 10pt

(10000 g)(1 mol / 58.44 g) = 171.1 mol

\vskip 10pt

Therefore, 10 kg of ``table'' salt is equal to 171.1 moles.


%(END_ANSWER)





%(BEGIN_NOTES)


%INDEX% Chemistry, stoichiometry: moles

%(END_NOTES)


