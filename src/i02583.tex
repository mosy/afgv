
%(BEGIN_QUESTION)
% Copyright 2012, Tony R. Kuphaldt, released under the Creative Commons Attribution License (v 1.0)
% This means you may do almost anything with this work of mine, so long as you give me proper credit

Dilute chemical concentrations are often measured in the unit of {\it parts per million}, abbreviated {\it ppm}.  For extremely dilute solutions, the unit {\it parts per billion} (ppb) is used.  These are nothing more than ratios, much like {\it percentage}.  In fact, the unit of ``percent'' may be thought of as nothing more than ``parts per hundred'' although it is never conventionally expressed as such.

\vskip 10pt

In light of this definition for {\it ppm}, calculate the following volumetric and mass quantities:

\begin{itemize}
\item{} The quantity of hydrazine vapor in 34000 cubic feet of air, where the volumetric concentration of hydrazine is 2.3 ppm
\vskip 10pt
\item{} The quantity of H$_{2}$S gas in a room measuring 25 feet by 8 feet by 31 feet, where the volumetric concentration of hydrogen sulfide gas is 93 ppm
\vskip 10pt
\item{} The quantity of sulfuric acid in 50 kg of water, where the mass concentration of acid is 247 ppm 
\end{itemize}

\underbar{file i02583}
%(END_QUESTION)





%(BEGIN_ANSWER)

\begin{itemize}
\item{} The quantity of 2.3 ppm (by volume) hydrazine vapor in 34000 cubic feet of air = {\bf 0.0782 cubic feet of pure hydrazine vapor}
\vskip 10pt
\item{} The quantity of 93 ppm (by volume) H$_{2}$S gas in a room measuring 25 feet by 8 feet by 31 feet = {\bf 0.5766 cubic feet of pure H$_{2}$S gas}
\vskip 10pt
\item{} The quantity of 247 ppm (by mass) sulfuric acid in 50 kg of water = {\bf 12.35 grams of pure sulfuric acid}
\end{itemize}

%(END_ANSWER)





%(BEGIN_NOTES)


%INDEX% Chemistry, safety: parts per billion concentration
%INDEX% Chemistry, safety: parts per million concentration

%(END_NOTES)


