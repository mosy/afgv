
%(BEGIN_QUESTION)
% Copyright 2012, Tony R. Kuphaldt, released under the Creative Commons Attribution License (v 1.0)
% This means you may do almost anything with this work of mine, so long as you give me proper credit

If an object is lifted up against the force of Earth's gravity, it gains {\it potential energy} (stored energy): the potential to do useful work if released to fall back down to the ground.  The exact amount of this potential energy may be calculated using the following formula:

$$E_p = m g h$$

\noindent
Where,

$E_p =$ Potential energy (Joules)

$m =$ Mass (kilograms)

$g =$ Acceleration of gravity (9.81 meters per second)

$h =$ Height lifted above the ground (meters)

\vskip 20pt

When that mass is released to free-fall back to ground level, its potential energy becomes converted into {\it kinetic energy} (energy in motion).  The relationship between the object's velocity and its kinetic energy may be calculated using the following formula:

$$E_k = {1 \over 2} m v^2$$

\noindent
Where,

$E_k =$ Kinetic energy (Joules)

$m =$ Mass (kilograms)

$v =$ Velocity (meters per second)

\vskip 20pt

Knowing that the amount of potential energy ($E_p$) an object possesses at its peak height ($h$) will be the same amount as its kinetic energy ($E_k$) in the last moment before it hits the ground, combine these two formulae to arrive at one formula predicting the maximum velocity ($v$) given the object's initial height ($h$).

\vskip 10pt

\underbar{file i02591}
%(END_QUESTION)





%(BEGIN_ANSWER)

If 100\% of the potential energy at the peak height gets converted into kinetic energy just before contact with the ground, we may set $E_p = E_K$:

$$E_p = m g h = E_k = {1 \over 2} m v^2$$

$$m g h = {1 \over 2} m v^2$$

$$g h = {1 \over 2} v^2$$

$$v^2 = 2 g h$$

$$\sqrt{v^2} = \sqrt{2 g h}$$

$$v = \sqrt{2 g h}$$

The paradoxical conclusion we reach from this combination and manipulation of formulae is that the amount of mass ($m$) doesn't matter: any object dropped from a certain height will hit the ground with the same velocity.  Of course, this assumes 100\% conversion of energy from potential to kinetic with no losses, which is never completely practical, and explains why lighter objects do in fact fall slower than heavy objects in air.  However, in a vacuum (no air resistance), the fall velocities are precisely equal!

%(END_ANSWER)





%(BEGIN_NOTES)


%INDEX% Mathematics review: basic principles of algebra
%INDEX% Mathematics review: manipulating literal equations

%(END_NOTES)


