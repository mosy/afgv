
%(BEGIN_QUESTION)
% Copyright 2012, Tony R. Kuphaldt, released under the Creative Commons Attribution License (v 1.0)
% This means you may do almost anything with this work of mine, so long as you give me proper credit

Suppose a crane picks up a shipping container weighing 32000 pounds, lifting it 17 feet above the ground, moving it horizontally 260 feet to a warehouse, and then setting it down into a pit 5 feet below ground level.  Calculate the total (net) amount of work done by the crane on the shipping container from its starting point (on the ground) to its destination (in the pit).

\underbar{file i02592}
%(END_QUESTION)





%(BEGIN_ANSWER)

The {\it total} amount of work done on the shipping container may be calculated by multiplying its weight (32000 pounds) by the total height change from beginning to end.  Assuming it started at ground level, and ended up 5 feet below ground level (i.e. the displacement is {\it down} when the crane's force on the container is {\it up}), the work done on the container is:

$$W = F x \cos \theta$$

$$W = (32000 \hbox{ lb}) (5 \hbox{ ft}) \cos 180^o$$

$$W = -160000 \hbox{ ft-lb of work done on the container}$$

Technically, the crane didn't do any work on the shipping container, but rather {\it the shipping container did work on the crane!}

\vskip 10pt

Note how we are completely ignoring the crane's horizontal motion, because this displacement vector is at right angles (90$^{o}$) to the container's weight vector as well as the crane's upward force vector, and therefore does not count toward or against work done on the container.

%(END_ANSWER)





%(BEGIN_NOTES)


%INDEX% Physics, energy, work, power

%(END_NOTES)


