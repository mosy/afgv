
%(BEGIN_QUESTION)
% Copyright 2012, Tony R. Kuphaldt, released under the Creative Commons Attribution License (v 1.0)
% This means you may do almost anything with this work of mine, so long as you give me proper credit

A laborer working on the top of a building uses a manual hoist to lift 20 gallons of water from ground level.  The height of this lift is 31 feet:

$$\includegraphics[width=15.5cm]{i02610x01.eps}$$

The rope is counterweighted with a mass equal to that of the bucket (empty), so that the bucket's weight does not have to be lifted, only the water inside the bucket.  Assuming a vertical lift distance of 31 feet, and ignoring the weight of the rope itself, how much work does the laborer do in lifting the water up?   Please express your answer in both English and metric units of work.

\underbar{file i02610}
%(END_QUESTION)





%(BEGIN_ANSWER)

To calculate work in this system, we need to know both the displacement (31 feet), and the force exerted in the direction of the displacement.  We were not given this force, but we do know it is 20 gallons of water being lifted, and we know the density of water to be 62.4 pounds per cubic foot.  So, 20 gallons of water weighs:

$$\left( 20 \hbox{ gal} \over 1 \right) \left(231 \hbox{ in}^3 \over 1 \hbox{ gal} \right) \left(1 \hbox{ ft}^3 \over 1728 \hbox{ in}^3 \right) \left(62.4 \hbox{ lb} \over \hbox{ ft}^3 \right) = 166.83 \hbox{ lb of water in 20 gallons}$$

Now, knowing both the force (166.83 lb) and the displacement (31 ft), we may calculate the work done:

$$W = F x \cos \theta$$

$$W = (166.83 \hbox{ lb})(31 \hbox{ ft})(\cos 0^o) = 5171.83 \hbox{ ft-lb of work}$$

Converting foot-pounds into joules:

$$\left(5171.83 \hbox{ ft-lb} \over 1 \right) \left(1055.06 \hbox{ J} \over 778.169 \hbox{ ft-lb} \right) = 7012.09 \hbox{ J of work}$$

%(END_ANSWER)





%(BEGIN_NOTES)


%INDEX% Physics, energy, work, power

%(END_NOTES)


