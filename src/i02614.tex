
%(BEGIN_QUESTION)
% Copyright 2012, Tony R. Kuphaldt, released under the Creative Commons Attribution License (v 1.0)
% This means you may do almost anything with this work of mine, so long as you give me proper credit

Two people are working to move a dead car, one pushing and the other pulling.  The person pushing exerts 150 pounds of force, while the one pulling exerts 170 pounds of force.  What is the total (resultant) force from the efforts of these two people?

$$\includegraphics[width=15.5cm]{i02614x01.eps}$$

Assuming they are able to drag the car a total distance of 45 feet before collapsing in exhaustion, calculate the total work done by these two people.

\underbar{file i02614}
%(END_QUESTION)





%(BEGIN_ANSWER)

When multiple forces act in the same direction, the resultant force will be the simple sum of the individual forces.  In this case, 150 pounds plus 170 pounds is equal to 320 pounds.

\vskip 10pt

Work is calculated by taking this total (net) force of 320 pounds and multiplying by the displacement of 45 feet, since both the force and the displacement vectors are pointed in the same direction (i.e. $\theta = 0^o$):

$$W = F x \cos \theta$$

$$W = (320 \hbox{ lb}) (45 \hbox{ ft}) \cos 0^o$$

$$W = 14400 \hbox{ ft-lb of work}$$

%(END_ANSWER)





%(BEGIN_NOTES)


%INDEX% Physics, energy, work, power

%(END_NOTES)


