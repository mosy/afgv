
%(BEGIN_QUESTION)
% Copyright 2012, Tony R. Kuphaldt, released under the Creative Commons Attribution License (v 1.0)
% This means you may do almost anything with this work of mine, so long as you give me proper credit

If 40 pounds of books are lifted from floor level to a bookshelf 5 feet above, then later those same books are taken off the shelf and returned to floor level, what is the total amount of work done by the person moving the books?

\underbar{file i02620}
%(END_QUESTION)





%(BEGIN_ANSWER)

Contrary to intuition, {\it no work has been done}.  Lifting the 40 pounds of books 5 feet up constitutes 200 ft-lb of work done on the books (i.e. potential energy invested in the books), but returning those books back to floor level constitutes 200 ft-lb of energy {\it released} (negative work done).  Thus, in physics terms, there was no net work performed.

\vskip 10pt

The person tasked with this pointless exercise, however, may beg to differ.

\vskip 10pt

This same principle of storing and releasing energy is employed in electric vehicles to recover braking energy.  Instead of converting electrical energy into mechanical potential and vice-versa as happens with the elevator, electric vehicles convert electrical energy into kinetic form (vehicle motion) and vice-versa.  Thus, accelerating an electric car from a full stop to some speed and then regeneratively decelerating it back to full stop is another example of zero (net) work.  This energy-recovering capability is what makes electric vehicles so attractive for stop-and-go travel.

%(END_ANSWER)





%(BEGIN_NOTES)


%INDEX% Physics, energy, work, power

%(END_NOTES)


