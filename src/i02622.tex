
%(BEGIN_QUESTION)
% Copyright 2012, Tony R. Kuphaldt, released under the Creative Commons Attribution License (v 1.0)
% This means you may do almost anything with this work of mine, so long as you give me proper credit

Suppose a crate full of bowling balls weighing a total of 5000 newtons (i.e. the crate has a total {\it mass} of 509.68 kilograms) is lifted 20 meters vertically into the air and then dropped.  How fast will its velocity be just before hitting the ground?

\vskip 20pt \vbox{\hrule \hbox{\strut \vrule{} {\bf Suggestions for Socratic discussion} \vrule} \hrule}

\begin{itemize}
\item{} What happens to all the potential energy that was ``invested'' in this crate after it impacts the ground?
\end{itemize}

\underbar{file i02622}
%(END_QUESTION)





%(BEGIN_ANSWER)

The amount of work required to lift this crate of bowling balls 20 meters into the air is 100,000 newton-meters, or 100,000 joules.  This is also the amount of potential energy the crate has at its apogee.

\vskip 10pt

When dropped, the crate's potential energy converts into kinetic energy, making the crate fall at a faster and faster velocity as its height approaches ground level (0).  If we neglect the effects of air friction, we may say that the potential energy of the crate at its peak height will be precisely equal to its kinetic energy at the moment it contacts the ground:

$$E_p \hbox{ (max.)} = E_k \hbox{ (max)}$$

$$m g h = {1 \over 2} m v^2$$

Note how the term of mass ($m$) cancels out of both sides of the equation, letting us know that the mass of the crate will be irrelevant to its free-fall velocity:

$$g h = {1 \over 2} v^2$$

Solving for $v$ given its initial height of 20 meters, and Earth's acceleration of gravity being 9.81 meters per second squared:

$$2 g h = v^2$$

$$v = \sqrt{2 g h}$$

$$v = \sqrt{(2) (9.81 \hbox{ m/s}^2) (20 \hbox{ m})} = 19.81 \hbox{ m/s}$$

Bowling balls everywhere!!!

%(END_ANSWER)





%(BEGIN_NOTES)


%INDEX% Physics, energy, work, power

%(END_NOTES)


