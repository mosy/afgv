
%(BEGIN_QUESTION)
% Copyright 2012, Tony R. Kuphaldt, released under the Creative Commons Attribution License (v 1.0)
% This means you may do almost anything with this work of mine, so long as you give me proper credit

Suppose two vehicles are traveling down a highway: a pickup truck and a sports car.  The truck weighs twice as much as the sports car, but is traveling only half as fast.

\vskip 10pt

All other factors being equal, which vehicle can stop in the shortest distance?

\underbar{file i02624}
%(END_QUESTION)





%(BEGIN_ANSWER)

The truck can stop in a shorter distance than the sports car (about {\it half} the stopping distance).  The reason for this is due to how much kinetic energy each vehicle possesses as it travels down the highway.  Mass has a linear effect on kinetic energy, but velocity has a square effect on kinetic energy.  Thus, while doubling a vehicle's mass will double its $E_k$ at any given velocity, doubling a vehicle's velocity will {\it quadruple} its $E_k$ for any given mass.  

\vskip 10pt

A helpful thought experiment to try here is to imagine a vehicle traveling at the slow speed of the pickup truck, and with the light mass of the sports car, and then use this vehicle as the ``standard'' of comparison.  The truck will have twice the kinetic energy of this third vehicle.  The sports car will have four times as much kinetic energy as this third vehicle.  Therefore, the sports car will have twice the kinetic energy of the truck.

%(END_ANSWER)





%(BEGIN_NOTES)


%INDEX% Physics, energy, work, power

%(END_NOTES)


