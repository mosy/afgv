
%(BEGIN_QUESTION)
% Copyright 2012, Tony R. Kuphaldt, released under the Creative Commons Attribution License (v 1.0)
% This means you may do almost anything with this work of mine, so long as you give me proper credit

A stupid person fires a gun straight up into the air.  The bullet leaves the gun's muzzle with a velocity of 1100 feet per second.  The bullet has a mass of 150 grains.  Ignoring the effects of air friction (I know, aerodynamic friction is no minor effect at supersonic velocities, but just work with me here . . .) how long will it take for the bullet to return to ground level?

\underbar{file i02655}
%(END_QUESTION)





%(BEGIN_ANSWER)

When the bullet is fired upward at ground level, its kinetic energy is at a maximum but its potential energy is near zero.  In rising against gravity, that kinetic energy will be translated into potential energy until the bullet's apogee, where there will be zero kinetic energy and maximum potential energy.  As it falls, the reverse will happen: potential energy will translate into kinetic energy again, resulting in the bullet traveling at 1100 feet per second when it returns to ground level.

\vskip 10pt

The time for the bullet to rise to its apogee will be equal to the time required for it to fall back to ground level from that maximum altitude.  So, if we were able to calculate that rise time (or that fall time), all we would have to do is multiply by two to obtain the total time spent in the air.

\vskip 10pt

The attraction of Earth's gravity results in a downward acceleration of 32.2 feet per second squared.  This means that any object dropped from altitude will accelerate at this rate, increasing in velocity by 32.2 ft/s every second.  It also means that any object rising by its own inertia (no propulsive force upward) will {\it slow down} by 32.2 ft/s each and every second until its vertical velocity is zero and it begins to fall back down.  This rate of acceleration holds true for {\it any} object, aerodynamic resistance notwithstanding.

\vskip 10pt

For this rate of acceleration to increase a falling object's velocity from 0 ft/s to 1100 ft/s will require this much time:

$$v = a t$$

$$t = {v \over a}$$

$$t = {1100 \hbox{ ft/s} \over 32.2 \hbox{ ft/s}^2} = 34.16 \hbox{ seconds}$$

This is the time required to either accelerate the bullet from zero ft/s (at apogee) to 1100 ft/s downward, or to {\it de}celerate it from 1100 ft/s upward to 0 ft/s (at apogee).  Doubling this figure gives us 68.32 seconds total time.  In other words, the stupid person has a little over a minute to duck and cover before the bullet finds its way back to {\it terra firma}.

%(END_ANSWER)





%(BEGIN_NOTES)


%INDEX% Physics, energy, work, power

%(END_NOTES)


