
%(BEGIN_QUESTION)
% Copyright 2015, Tony R. Kuphaldt, released under the Creative Commons Attribution License (v 1.0)
% This means you may do almost anything with this work of mine, so long as you give me proper credit

A technician needs to write a PLC program to control a water pump driven by an electric motor.  This water pump will be manually started and stopped by pushbutton switches, and shut down automatically by any one of several ``permissive'' switches.  The operating statuses of these switches are listed here:

\begin{itemize}
\item{} {\bf Start pushbutton} (normally-open): open when unpressed, closed when pressed
\item{} {\bf Stop pushbutton} (normally-closed): closed when unpressed, open when pressed
\item{} {\bf Low water level} (normally-closed): closed when level is low, open when level is adequate
\item{} {\bf Low oil pressure} (normally-open): open when pressure is low, closed when pressure is adequate
\item{} {\bf High vibration} (normally-closed): closed when still, open when vibrating
\item{} {\bf Water leak detector} (normally-open): open when dry, closed when wet (leak detected)
\end{itemize}

The technician's first attempt is shown here, but it contains a serious error.  Identify and correct this error:

$$\includegraphics[width=15.5cm]{i02657x01.eps}$$

\underbar{file i02657}
%(END_QUESTION)





%(BEGIN_ANSWER)

The {\tt Oil\_press\_low} contact instruction should be drawn as normally-open rather than normally-closed as shown in the technician's first draft of the PLC program.  The program is designed to shut down the motor if ever the {\tt Shutdown} bit goes to a 0 state.  This means the motor will shut down if any of the permissive contact instructions become uncolored (i.e. fails to ``conduct'' virtual power).  We have been told that the real-world oil pressure switch is NO, which means its contact opens when oil pressure becomes too low.  This means a low oil pressure condition causes that bit to be 0, which necessitates an NO contact instruction so that it will un-color under that condition.

%(END_ANSWER)





%(BEGIN_NOTES)


%INDEX% PLC, diagnosing programming error
%INDEX% PLC, relating I/O status to virtual elements (troubleshooting)
%INDEX% PLC, troubleshooting: motor start/stop control circuit

%(END_NOTES)


