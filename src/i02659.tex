
%(BEGIN_QUESTION)
% Copyright 2015, Tony R. Kuphaldt, released under the Creative Commons Attribution License (v 1.0)
% This means you may do almost anything with this work of mine, so long as you give me proper credit

The {\tt grades\_template} spreadsheet provided for you on the Y: network drive allows you to calculate your grade for any course (by entering exam scores, attendance data, etc.) as well as project to the future for courses you have not yet taken.  Download the spreadsheet file (if you have not done so yet) and enter all the data you can for grade calculation at this point in the quarter.

\vskip 20pt

Also, locate the pages in your course worksheet entitled ``Sequence of Second-Year Instrumentation Courses'' to identify which courses you will need to register for next quarter.




\vskip 20pt \vbox{\hrule \hbox{\strut \vrule{} {\bf Suggestions for Socratic discussion} \vrule} \hrule}

\begin{itemize}
\item{} If you do not yet have enough data to calculate a final grade for a course (using the spreadsheet), experiment with plugging scores into the spreadsheet to obtain the grade you would like to earn.  How might this be a useful strategy for you in the future?
\item{} Why do you suppose this spreadsheet is provided to you, rather than the instructor simply posting your grades or notifying you of your progress in the program courses?
\item{} Identify any courses that are {\it elective} rather than required for your 2-year AAS degree.
\end{itemize}

\underbar{file i02659}
%(END_QUESTION)





%(BEGIN_ANSWER)

You may locate the {\tt grades\_template} on the {\tt Y:} network drive at BTC, provided you log in to the computer system using your individual student ID and password (not a generic login such as ``btc'').  It is also available for download at the {\it Socratic Instrumentation} website.

%(END_ANSWER)





%(BEGIN_NOTES)


%INDEX% Course organization, grading: spreadsheet-based grade calculator

%(END_NOTES)


