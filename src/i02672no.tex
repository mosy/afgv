
%(BEGIN_QUESTION)
% Copyright 2008, Tony R. Kuphaldt, released under the Creative Commons Attribution License (v 1.0)
% This means you may do almost anything with this work of mine, so long as you give me proper credit

%Sketch a circuit whereby two loop-powered pressure transmitters send signals to two isolated channels on a datalogger, a device that records electronic signals over long periods of time.  Include one DC power supply in your circuit, along with any other components necessary to make the two 4-20 mA loops work:

Tegn en krets der to trykktransmittere sender signal til to forskjellige kanaler p{\aa} en datalogger, en komponent som tar opp electroniske signaler over tid. Ta med str{\o}mforsyning og eller andre komponenter som trengs for {\aa} f{\aa} kretsen til {\aa} fungere. 


$$\epsfbox{i02672x01.eps}$$

\vfil 

\underbar{file i02672}
\eject
%(END_QUESTION)





%(BEGIN_ANSWER)

This is a graded question -- no answers or hints given!

%(END_ANSWER)





%(BEGIN_NOTES)

A good problem-solving technique to apply here is sketching the directions of current through each device (based on its voltage polarity marks and whether it is a {\it load} or a {\it source}).  Then, we will know how the wires must connect in order to keep all the arrows facing the same direction in series loops:

\vskip 10pt

This is just one possible solution:

$$\epsfbox{i02672x02.eps}$$

150 $\Omega$ is the {\it maximum} resistance value practical in this circuit, due to the upper limit of 3 volts on the datalogger.  This, of course, makes the calibrated range of measurement 0.6 volts to 3.0 volts between the datalogger terminals.

%INDEX% Pictorial circuit review (4-20 mA loop)

%(END_NOTES)


