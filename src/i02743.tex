
%(BEGIN_QUESTION)
% Copyright 2015, Tony R. Kuphaldt, released under the Creative Commons Attribution License (v 1.0)
% This means you may do almost anything with this work of mine, so long as you give me proper credit

Read selected portions of the ``SEL Fiber-Optic Products and Applications'' data sheet (February 2015) and answer the following questions:

\vskip 10pt

Schweitzer Engineering Laboratories (SEL) manufactures electronic equipment designed to monitor and protect high-voltage electrical power grid components, including specialized devices called {\it protective relays} designed to monitor voltage and current conditions in power grids and take automatic protective action to ``clear'' faults before they cause irreparable damage to the system and/or threaten human safety.  Why do you suppose a company like SEL would offer fiber-optic components such as serial data converters and fiber optic cables?  Of what use would fiber optic devices have in a high-voltage substation, generating station, or similar facility?

\vskip 10pt

Page 1 of this document highlights features of the SEL-2800 series of fiber optic data {\it transceivers} manufactured by Schweitzer Engineering Laboratories.  Based on the information found here, describe where you might find such transceivers used.  Identify where you could use these in the school's lab, at least to demonstrate the concept.

\vskip 10pt

Read the section of this document entitled ``Reflection Issues When Running Short-Distance, Single-Mode Fiber'' and describe what the potential problems are and how they are mitigated.

\vskip 10pt

Find the photographs of ``splice bushings'' shown in this document and explain their use.




\vskip 20pt \vbox{\hrule \hbox{\strut \vrule{} {\bf Suggestions for Socratic discussion} \vrule} \hrule}

\begin{itemize}
\item{} Table 1 (found on page 2 of this document) provides ``link budget'' figures for each model of fiber optic transceiver.  Explain what a ``link budget'' value represents, and why it might be important.
\item{} Table 3 (found on page 5 of this document) provides typical attenuation values for different types of optical fiber.  Identify which optical fiber type exhibits the least amount of loss, and explain why.  Also, explain why single-mode fiber shows ``N/A'' ({\it not applicable}) for attenuation at a light wavelength of 850 nm.
\end{itemize}

\underbar{file i02743}
%(END_QUESTION)





%(BEGIN_ANSWER)



%(END_ANSWER)





%(BEGIN_NOTES)

Optical fibers provide complete immunity to electrical interference as well as differences in ground potential, which are valid concerns in high-power electrical substation environments.

\vskip 10pt

The SEL-2800 series transceivers convert electrical serial data (RS-232 or RS-485) into optical equivalent pulses, allowing serial communications to occur between electronic devices via optical media rather than electrical.

\vskip 10pt

The section of this document entitled ``Reflection Issues When Running Short-Distance, Single-Mode Fiber'' discusses what may happen in short-length single-mode fiber networks, where reflections between connectors in a cable may cause ``echo'' signals to reach the receiver after the incident light pulses reach it, thus appearing as false (extra) data bits.  To combat this, SEL sells an {\it attenuator kit} to install in each optical fiber path, introducing $-13$ db of intentional power loss so as to make the reflected ``echo'' signals so weak they cannot be falsely interpreted as valid data.  So long as the valid data magnitude exceeds the $-50$ dB receiver sensitivity threshold, it will be received without error.

\vskip 10pt

The splice bushings (``butt connectors'') are shown on page 11, and are fairly self-explanatory.


%INDEX% Reading assignment: SEL fiber-optic products and applications

%(END_NOTES)


