
%(BEGIN_QUESTION)
% Copyright 2007, Tony R. Kuphaldt, released under the Creative Commons Attribution License (v 1.0)
% This means you may do almost anything with this work of mine, so long as you give me proper credit

The Boolean representation of this truth table is:

% No blank lines allowed between lines of an \halign structure!
% I use comments (%) instead, so that TeX doesn't choke.

$$\vbox{\offinterlineskip
\halign{\strut
\vrule \quad\hfil # \ \hfil & 
\vrule \quad\hfil # \ \hfil & 
\vrule \quad\hfil # \ \hfil & 
\vrule \quad\hfil # \ \hfil \vrule \cr
\noalign{\hrule}
%
% First row
A & B & C & Output \cr
%
\noalign{\hrule}
%
% Another row
0 & 0 & 0 & 0 \cr
%
\noalign{\hrule}
%
% Another row
0 & 0 & 1 & 0 \cr
%
\noalign{\hrule}
%
% Another row
0 & 1 & 0 & 1 \cr
%
\noalign{\hrule}
%
% Another row
0 & 1 & 1 & 0 \cr
%
\noalign{\hrule}
%
% Another row
1 & 0 & 0 & 0 \cr
%
\noalign{\hrule}
%
% Another row
1 & 0 & 1 & 0 \cr
%
\noalign{\hrule}
%
% Another row
1 & 1 & 0 & 1 \cr
%
\noalign{\hrule}
%
% Another row
1 & 1 & 1 & 0 \cr
%
\noalign{\hrule}
} % End of \halign 
}$$ % End of \vbox

\begin{itemize}
\item{(A)} $B \overline{C}$
\vskip 5pt 
\item{(B)} $A \overline{C}$
\vskip 5pt 
\item{(C)} $B$
\vskip 5pt 
\item{(D)} $\overline{A} \> \overline{B} \> C$
\vskip 5pt 
\item{(E)} $AB + \overline{C}$
\end{itemize}

\underbar{file i02770}
%(END_QUESTION)





%(BEGIN_ANSWER)

{\bf (A)} $B \overline{C}$
 
%(END_ANSWER)





%(BEGIN_NOTES)


%INDEX% Certification exam: Digital circuit theory 

%(END_NOTES)


