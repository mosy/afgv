
%(BEGIN_QUESTION)
% Copyright 2012, Tony R. Kuphaldt, released under the Creative Commons Attribution License (v 1.0)
% This means you may do almost anything with this work of mine, so long as you give me proper credit

Suppose a steam turbine engine in a power plant receives steam from the boiler at a temperature of 583 $^{o}$F and a pressure of 88 PSIG, and discharges steam at a temperature of 91 $^{o}$F and a pressure of $-14$ PSIG.  The mass flow rate of this steam is 2300 pounds per minute. 

\vskip 10pt

Assuming a turbine efficiency of 80\%, calculate the mechanical power output by this steam turbine, in the unit of Watts.


\underbar{file i02855}
%(END_QUESTION)





%(BEGIN_ANSWER)

We may calculate the heat rate input to this turbine by subtracting enthalpy values ($h_{in} - h_{out}$) and multiplying by the mass flow rate of the steam:

\vskip 10pt

Input enthalpy (from steam table; 583 $^{o}$F and 88 PSIG) = 1314.4 BTU/lb

\vskip 10pt

Output enthalpy (from steam table; 91 $^{o}$F and $-14$ PSIG) = 1099.6 BTU/lb

\vskip 10pt

Difference in enthalpy values from inlet to discharge of turbine = 1314.4 BTU/lb $-$ 1099.6 BTU/lb = 214.8 BTU/lb

\vskip 10pt

Heat rate = (214.8 BTU/lb)(2300 lb/min) = 494,040 BTU/min = 29,642,400 BTU/h
 
\vskip 10pt

We may convert this heat rate (power) into watts by using the conversion equivalence of 745.7 watts and 2544.43 BTU/h:

\vskip 10pt

(29642400 BTU/h) (745.7 W / 2544.43 BTU/h) = 8,687,343.6 W = 8.69 MW

\vskip 10pt

Since the turbine is 80\% efficient, only 80\% of this heat rate gets converted into mechanical shaft power.  Therefore,

\vskip 10pt

(8.69 MW)(0.80) = {\bf 6.95 MW} shaft power


%(END_ANSWER)





%(BEGIN_NOTES)


%INDEX% Physics, heat and temperature: steam table

%(END_NOTES)


