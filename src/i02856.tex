
%(BEGIN_QUESTION)
% Copyright 2015, Tony R. Kuphaldt, released under the Creative Commons Attribution License (v 1.0)
% This means you may do almost anything with this work of mine, so long as you give me proper credit

Read and outline the ``Introduction to Power System Automation'' section of the ``Electric Power Measurement and Control'' chapter in your {\it Lessons In Industrial Instrumentation} textbook.  Note the page numbers where important illustrations, photographs, equations, tables, and other relevant details are found.  Prepare to thoughtfully discuss with your instructor and classmates the concepts and examples explored in this reading.

\underbar{file i02856}
%(END_QUESTION)




%(BEGIN_ANSWER)


%(END_ANSWER)





%(BEGIN_NOTES)







\vskip 20pt \vbox{\hrule \hbox{\strut \vrule{} {\bf Suggestions for Socratic discussion} \vrule} \hrule}

\begin{itemize}
\item{} Examine the large single-line diagram shown in this section and identify how power gets from one particular generating station to one particular load.
\item{} Examine the large single-line diagram shown in this section and identify how power may be re-routed from sources to loads if a particular breaker is tripped.
\item{} Examine the large single-line diagram shown in this section and identify how power may be isolated from a section of the power grid in the event of a fault (e.g. a transmission line that has fallen to the ground).
\item{} Examine the large single-line diagram shown in this section and compare the topologies of any two substations, contrasting their designs (e.g. number of circuit breakers, ways to alternatively route power to any particular load).   
\item{} What is the purpose of using a PT (or VT) for electrical power instrumentation?
\item{} What is the standard output signal range of a PT?
\item{} What is the purpose of using a CT for electrical power instrumentation?
\item{} What is the standard output signal range of a CT?
\item{} Describe what a {\it protective relay} does in an electric power system.
\item{} Differentiate between a {\it disconnect} and a {\it circuit breaker} in an electric power system.
\item{} Identify and describe some of the different methods used to extinguish arcing inside circuit breakers.
\item{} Examine the final schematic diagram in this section of the textbook and describe a way you could force the circuit breaker to trip (open) using a jumper wire judiciously placed in the control circuitry.
\end{itemize}


%INDEX% Reading assignment: Lessons In Industrial Instrumentation, introduction to power system automation

%(END_NOTES)


