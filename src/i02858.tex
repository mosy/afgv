
%(BEGIN_QUESTION)
% Copyright 2015, Tony R. Kuphaldt, released under the Creative Commons Attribution License (v 1.0)
% This means you may do almost anything with this work of mine, so long as you give me proper credit

Read and outline the ``Introduction to Power System Automation'' section of the ``Electric Power Measurement and Control'' chapter in your {\it Lessons In Industrial Instrumentation} textbook.  Prepare to thoughtfully discuss with your instructor and classmates the concepts and examples explored in this reading.

\vskip 10pt

Your instructor will ask you to show your outline of this reading assignment: a summary written \underbar{in your own words} of what you learned from the text.  The instructor will query you on any sections of your outline that appear weak or missing.

\vskip 10pt

After assessing your outline, the instructor may ask questions similar or identical to the ``Suggestions for Socratic discussion'' listed below in order to spur an engaging discussion on this topic.  The purpose of this Socratic dialogue is to challenge your reasoning on this subject and to thereby foster your analytical abilities.

\vskip 20pt \vbox{\hrule \hbox{\strut \vrule{} {\bf Suggestions for Socratic discussion} \vrule} \hrule}

\begin{itemize}
\item{} Arbitrarily choose one of the loads represented in the large single-line diagram shown near the beginning of this textbook section, and identify the path(s) by which electrical power flows to that load from one or more sources (generators).
\item{} Arbitrarily choose a single power line or component represented in the large single-line diagram shown near the beginning of this textbook section, and identify which circuit breaker(s) would have to be opened (``tripped'') in order to cut the flow of electrical power to that line or component.
\item{} What is the purpose of having both {\it disconnects} and {\it circuit breakers} in an electrical power system?  Isn't one form of circuit interruption sufficient?
\item{} Explain the meaning of the phrase ``silent sentinel'' with regard to protective relays in an electrical power system.
\item{} List all the number codes referenced in this section of the textbook describing different protective relay functions.
\item{} Identify any concepts in today's reading that seem similar (or identical) to concepts you have learned previously about industrial instrumentation.
\item{} {\it Kirchhoff's Current Law} is a fundamental principle of electric circuits, and it finds application in an 87 (differential current) protective relay.  Explain this basic principle, and how we might exploit it to detect abnormal conditions in an electric power system component.
\item{} Formulate your own question based on the reading, as though you were an instructor querying students on their understanding of the text.
\end{itemize}

\underbar{file i02858}
%(END_QUESTION)





%(BEGIN_ANSWER)

No answers given here!  A recommendation for writing your outline, though, is to write approximately one sentence of your own thoughts per paragraph of text read.  It's okay to have questions and uncertainties about the reading, too -- {\it bring those to class ready to discuss as well!}  Feel free to write a digital version of your outline, copying and pasting images from the electronic text files into your own document, if you prefer to write in electronic format rather than by hand.

\vskip 10pt

One of the purposes of an ``inverted'' classroom structure where students encounter new topics on their own through reading is to develop technical reading skills in addition to learning about the topic at hand.  If you struggle with the subject matter or just with technical reading in general, that's okay.  {\it This is a skill you will gain by doing!}  Do your best, come to class fully prepared to ask questions and explain what does make sense to you, and you will be well on your way to becoming an autonomous technical learner.
 
%(END_ANSWER)





%(BEGIN_NOTES)

Automation is used in electric power systems to monitor and control the flow of electricity.  Like industrial automation, there are three basic system components controlling the process: (1) sensors to detect the process variable(s), (2) controllers to decide what to do, and (3) final control devices to alter process conditions.  Instrument transformers serve as sensors, stepping high voltage and current levels down to manageable quantities.  Protective relays make control decisions based on sensed voltage and current levels.  Circuit breakers redirect power flow and isolate sections of the power system.

\vskip 10pt

Single-line diagrams are analogous to PFDs for industrial systems, showing the routing of electricity through pathways from sources to loads.  Individual conductors are not shown in single-line diagrams.  Generators and loads appear as circles, circuit breakers as squares, disconnects as knife-switch symbols, and transformers as regular schematic transformer symbols (a series of half-circles).  {\it Current transformers} (CTs) step high line current values proportionately down to levels easily measured by instruments, while {\it Potential transformers} (PTs) step high line voltage values proportionately down for the same reason.  The standard output of a PT is 0 to 120 volts AC.  The standard output of a CT is 0 to 5 amps.  Various turns ratios are available to properly scale voltage and current for any given application.

Single-line diagrams showing protective relay and measurement functions also represent those as circles, containing numerical codes representing each protective function.  Solid lines represent power and analog signal paths, while dashed lines represent discrete (on/off) control signals.  Substation control panels are often built as representations of single-line diagrams, with plastic traces illustrating lines between components.

\vskip 10pt

{\it Supervisory Control And Data Acquisition} (SCADA) systems interpret instrument transformer and breaker status signals, and present that data to human operators on graphical displays.  These SCADA systems use different technologies (microwave, fiber optics) to communicate power system data over long distances.

\vskip 10pt

{\it Protective relays} sense abnormal conditions and command circuit breakers to trip (open) in order to protect the system against damage.  ANSI-standardized number codes represent different protective functions including:

\begin{itemize}
\item{} 21 (Distance), sensing impedance of a power line and load
\item{} 87 (Differential current), sensing mismatch of current in and out of a device
\item{} 63 (Pressure), sensing excessive pressure inside of a device
\item{} 50 (Instantaneous overcurrent), sensing very high overload currents and
\item{} 51 (Time overcurrent), sensing lesser overload currents over longer time periods
\item{} 67 (Directional overcurrent), discriminating between direction of overcurrent
\end{itemize}

Modern microprocessor-based protective relays are much more accurate and discriminating than legacy electromechanical relays.  They also record fault events and may be configured via electronic files.

\vskip 10pt

{\it Disconnects} are open-air knife switches used to manually isolate sections of a power network.  {\it Circuit breakers} are enclosed switches designed to interrupt high levels of current by extinguishing the arc in various ways (oil immersion, air blast, SF$_{6}$ gas, or vacuum).  Circuit breaker mechanisms are operated by spring or compressed air action, triggered by solenoid coils for remote operation by human operators or protective relays.

\vskip 10pt

A typical protective relay system uses instrument transformers to sense line conditions, a relay to decide when to trip the breaker, and a DC trip control circuit whereby the relay closes a contact to energize a {\it trip coil} inside the circuit breaker.









\vfil \eject

\noindent
{\bf Prep Quiz:}

Describe the purpose of a {\it protective relay} in your own words.









\vfil \eject

\noindent
{\bf Prep Quiz:}

Describe the purpose of a {\it SCADA} system in an electric power system.









\vfil \eject

\noindent
{\bf Prep Quiz:}

Explain why large power circuit breakers utilize mechanisms operated by either springs or compressed air.




%INDEX% Electric power systems: overview of automation and basic devices

%(END_NOTES)


