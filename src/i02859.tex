
%(BEGIN_QUESTION)
% Copyright 2015, Tony R. Kuphaldt, released under the Creative Commons Attribution License (v 1.0)
% This means you may do almost anything with this work of mine, so long as you give me proper credit

Read and outline the first four pages of Chapter 1 (``What Are Protective Relays'') in the book {\it Protective Relays} by Victor H. Todd.  Note that this book is available through Google Books as well as the Socratic Instrumentation web page as a downloadable PDF file free of copyright due to its age (copyright 1922).  Prepare to thoughtfully discuss with your instructor and classmates the concepts and examples explored in this reading.

\vskip 10pt

Your instructor will ask you to show your outline of this reading assignment: a summary written \underbar{in your own words} of what you learned from the text.  The instructor will query you on any sections of your outline that appear weak or missing.

\vskip 10pt

After assessing your outline, the instructor may ask questions similar or identical to the ``Suggestions for Socratic discussion'' listed below in order to spur an engaging discussion on this topic.  The purpose of this Socratic dialogue is to challenge your reasoning on this subject and to thereby foster your analytical abilities.

\vskip 20pt \vbox{\hrule \hbox{\strut \vrule{} {\bf Suggestions for Socratic discussion} \vrule} \hrule}

\begin{itemize}
\item{} Describe in your own words the evolution of electric power system protection, from the days of no protection to the ``state of the art'' circa 1922 shown in this book.  What, exactly, are these power systems being protected against, and how are protective relays superior in their functionality?
\item{} Two important concepts are used in the electric power industry to express the reliability of the system: {\it dependability} and {\it security}.  Dependability is the likelihood that a protective device or system will shut off power as designed in the event of an fault.  Security is the likelihood that a protective device will allow power to remain on when there is no fault (i.e. the likelihood it will not needlessly trip).  Apply these terms in describing the evolution of power system protection as described by this reading.
\item{} Formulate your own question based on the reading, as though you were an instructor querying students on their understanding of the text.
\end{itemize}

\underbar{file i02859}
%(END_QUESTION)





%(BEGIN_ANSWER)

No answers given here!  A recommendation for writing your outline, though, is to write approximately one sentence of your own thoughts per paragraph of text read.  It's okay to have questions and uncertainties about the reading, too -- {\it bring those to class ready to discuss as well!}  Feel free to write a digital version of your outline, copying and pasting images from the electronic text files into your own document, if you prefer to write in electronic format rather than by hand.

\vskip 10pt

One of the purposes of an ``inverted'' classroom structure where students encounter new topics on their own through reading is to develop technical reading skills in addition to learning about the topic at hand.  If you struggle with the subject matter or just with technical reading in general, that's okay.  {\it This is a skill you will gain by doing!}  Do your best, come to class fully prepared to ask questions and explain what does make sense to you, and you will be well on your way to becoming an autonomous technical learner.
 
%(END_ANSWER)





%(BEGIN_NOTES)

Early electrical power systems had no overcurrent protection because reliability wasn't paramount and often they were incapable of self-harm through overload.  {\it Fuses} were the first overcurrent protection devices, designed to be intentional ``weak spots'' that would burn out first.  Fuses still used even today (1922) due to their extreme reliability, often in conjunction with other protective devices.

\vskip 10pt

Fuses can be expensive and time-consuming to replace.  Lack of proper fuse inventory invites improper replacement, causing other problems.  {\it Circuit breakers} solve both problems, but do not solve another problem of fuses: loss of service due to transient faults because this overcurrent protection is not ``smart'' enough to discriminate between different degrees of fault. 

\vskip 10pt

{\it Protective relays} sense power line conditions and command a circuit breaker to trip only if absolutely needed.  The advantage of protective relaying over self-tripping circuit breakers is greater discrimination of faults, leading to less unnecessary interruption of power to customers (in one case, a 25-fold decrease in interruptions!).  Relays are built to detect the following abnormal conditions (in addition to others):

\begin{itemize}
\item{} Excessive or insufficient current
\item{} Excessive or insufficient voltage
\item{} Excessive or insufficient power
\item{} Excessive or insufficient frequency
\item{} Excessive or insufficient temperature
\item{} Reverse current or power
\end{itemize}

\vskip 10pt

Practically all protective relays are {\it electromagnetic} in principle: using the magnetic field created by an electric current to (1) move a coil against a magnet, (2) attract an armature, or (3) rotate an induction disk.









\vfil \eject

\noindent
{\bf Prep Quiz:}

Describe the purpose of a {\it protective relay} in your own words.

%(END_NOTES)


