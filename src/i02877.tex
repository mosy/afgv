
%(BEGIN_QUESTION)
% Copyright 2015, Tony R. Kuphaldt, released under the Creative Commons Attribution License (v 1.0)
% This means you may do almost anything with this work of mine, so long as you give me proper credit

The following equation predicts the tripping time ($t$) for a time-overcurrent relay given the time-dial setting ($T$) and overcurrent value in multiples of pick-up current ($M$):

%$$t = T \left(0.18 + {5.95 \over {M^2 - 1}} \right) \hskip 30pt \hbox{Inverse curve}$$

$$t = T \left(0.0963 + {3.88 \over {M^2 - 1}} \right) \hskip 30pt \hbox{Very inverse curve}$$

%$$t = T \left(0.0352 + {5.67 \over {M^2 - 1}} \right) \hskip 30pt \hbox{Extremely inverse curve}$$

\noindent
Where,

$t$ = Trip time (seconds)

$T$ = Time Dial setting

$M$ = Multiples of pickup current

\vskip 20pt

\noindent
First, determine the proper pickup current setting for this time-overcurrent relay if a line current value of 1130 amps is considered full-load (i.e. 100\%) and the CT ratio is 1200:5.

\vskip 10pt

$I_{pickup}$ = \underbar{\hskip 50pt}

\vskip 20pt

\noindent
Next, calculate the length of time necessary for this relay to trip given a constant line current of 3000 amps and a time-dial setting of 8:

\vskip 20pt

$t$ = \underbar{\hskip 50pt} seconds

\vskip 10pt

\underbar{file i02877}
%(END_QUESTION)





%(BEGIN_ANSWER)

$I_{pickup}$ = \underbar{\bf 4.708} amps

\vskip 10pt

$t$ = \underbar{\bf 5.902} seconds

%(END_ANSWER)





%(BEGIN_NOTES)

{\bf This question is intended for exams only and not worksheets!}.

%(END_NOTES)


