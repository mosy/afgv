
%(BEGIN_QUESTION)
% Copyright 2015, Tony R. Kuphaldt, released under the Creative Commons Attribution License (v 1.0)
% This means you may do almost anything with this work of mine, so long as you give me proper credit

Electromechanical time-overcurrent (51) relays are equipped with a {\it seal-in} circuit.  Explain what this portion of the relay does, and why it is necessary.

\underbar{file i02879}
%(END_QUESTION)





%(BEGIN_ANSWER)

The moving contact on an induction-disc 51 relay operates with very small amounts of force, which makes firm electrical contact a practical impossibility.  The ``seal-in'' function of a 51 relay is a parallel contact actuated by trip circuit current which closes around the disc's moving contact once the relay reaches its trip state, in order to provide a reliable pathway for trip current to flow to the breaker's trip coil.

%(END_ANSWER)





%(BEGIN_NOTES)

{\bf This question is intended for exams only and not worksheets!}.

%(END_NOTES)


