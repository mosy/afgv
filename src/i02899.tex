
%(BEGIN_QUESTION)
% Copyright 2011, Tony R. Kuphaldt, released under the Creative Commons Attribution License (v 1.0)
% This means you may do almost anything with this work of mine, so long as you give me proper credit

A company named GlassPoint designed and built a rather ingenious solar-powered steam generating system for enhanced oilfield recovery cost-competitive to natural gas in regions with plentiful direct sunshine.  Their website in 2011 claimed these performance figures:

\begin{itemize}
\item{} Steam pressure = 2500 PSIG
\item{} Steam temperature = 750 degrees Fahrenheit
\item{} Steam volume = 90 barrels of boiler feedwater consumption per day, per acre of collection area
\end{itemize}

Based on these performance metrics, answer the following questions:

\vskip 10pt

\noindent
Is this saturated steam or superheated steam?  How can you tell?

\vskip 10pt

\noindent
Calculate the water flow rate in units of gallons per minute (per acre of collection).  Hint: one ``barrel'' is equal to 42 gallons.  Assume the unit operates for 12 hours each day, not 24 hours since the sun does not shine at night.

\vskip 10pt

\noindent
Calculate the heat output rate in units of BTU per minute per acre of collection, assuming the boiler feedwater starts at an ambient temperature of 70 degrees Fahrenheit. 

\vskip 10pt

\noindent
Calculate the heat output rate in units of horsepower per acre of collection. 

\vskip 10pt

Hint: the {\it Fisher Control Valve Handbook} has a set of steam tables in the Appendix section covering the pressures and temperatures experienced in the GlassPoint process.

\vskip 10pt

\underbar{file i02899}
%(END_QUESTION)





%(BEGIN_ANSWER)

\noindent
{\bf Partial answer:}

\vskip 10pt

This is definitely superheated steam, because its temperature is well above the boiling point for water at 2500 PSIG (approximately 670 $^{o}$F, as my steam table shows water at 2526 PSIA to boil at 670 $^{o}$F).

\vskip 10pt

Your calculated value for water flow rate should be 5.25 gallons per minute, or 43.79 pounds per minute.

\vskip 10pt

Heat output = approximately 1249.9 horsepower per acre of collection area.

%(END_ANSWER)





%(BEGIN_NOTES)

Even my old steam table (dated from 1920) shows saturated steam at 706.1 $^{o}$F (cooler than the GlassPoint steam) at a saturated pressure of 3200 PSIA (greater than the GlassPoint steam), which tells us the GlassPoint steam must be superheated.

\vskip 10pt

We may calculate heat output per acre of collection by converting the flow rate of feedwater per acre (90 barrels per day) into BTU per minute.  90 barrels per 12-hour day is 315 GPH, which is 5.25 GPM.  At 62.4 lbs per cubic foot, this equates to a mass flow rate of 43.79 pounds per minute of water being boiled into steam:

$$\left( {5.25 \hbox{ gal} \over \hbox{min}} \right) \left( {231 \hbox{ in}^3 \over 1 \hbox{ gal}} \right) \left( {1 \hbox{ ft}^3 \over 1728 \hbox{ in}^3} \right) \left( {62.4 \hbox{ lb} \over 1 \hbox{ ft}^3} \right) = 43.79 \hbox{ lb/min}$$

Interpolating from the Fisher Control Valve Handbook's superheated steam table, steam at 2500 PSIG and 750 degrees F has an enthalpy of 1248.3 BTU per pound.  Subtracting the enthalpy of 70 $^{o}$F water, this gives us a heat of 1210.3 BTU per pound of boiler feedwater turned into steam.  Given a mass flow rate of 43.79 pounds per minute, our heat output from one acre of GlassPoint collection is 53,003.1 BTU per minute (with no rounding).

Converting 53003.1 BTU/minute into horsepower:

$$\left( {53003.1 \hbox{ BTU/min} \over 1} \right) \left( {1 \hbox{ HP} \over 2544.43 \hbox{ BTU/hr}} \right) \left( {60 \hbox{ min} \over 1 \hbox{ hr}} \right) = 1249.9 \hbox{ HP}$$

Thus, one acre of GlassPoint solar collection is equivalent to {\bf 1249.9 horsepower}.

\vskip 20pt

\noindent
{\bf Interpolation work:}

\vskip 10pt

Fisher Control Valve Handbook shows the following enthalpies (table values in {\bf bold}, interpolated values in {\it italics}):

% No blank lines allowed between lines of an \halign structure!
% I use comments (%) instead, so that TeX doesn't choke.

$$\vbox{\offinterlineskip
\halign{\strut
\vrule \quad\hfil # \ \hfil & 
\vrule \quad\hfil # \ \hfil & 
\vrule \quad\hfil # \ \hfil & 
\vrule \quad\hfil # \ \hfil \vrule \cr
\noalign{\hrule}
%
% First row
Pressure & Temperature & {\it Temperature} & Temperature \cr
%
\noalign{\hrule}
%
% Another row
 & {\bf 740 $^{o}$F} &  {\it 750 $^{o}$F} & {\bf 760 $^{o}$F} \cr
%
\noalign{\hrule}
%
% Another row
{\bf 2500 PSIA} & 1237.6 BTU/lb & {\it 1249.7 BTU/lb} & 1261.8 BTU/lb \cr
%
\noalign{\hrule}
%
% Another row
{\bf 2600 PSIA} & 1227.3 BTU/lb & {\it 1240.1 BTU/lb} & 1252.9 BTU/lb \cr
%
\noalign{\hrule}
} % End of \halign 
}$$ % End of \vbox

Now, all we need to do is take the interpolated enthalpies at 750 $^{o}$F and interpolate {\it them} to a pressure that is 2514.7 PSIA.  Mathematically, this is nothing more than a scaling problem:

$${{2514.7 - 2500} \over {2600 - 2500}} = {14.7 \over 100} = 0.147$$

$$0.147 (1240.1 - 1249.7) = -1.4112$$

$$-1.4112 + 1249.7 = 1248.3 \hbox{ BTU/lb}$$






\vfil \eject

\noindent
{\bf Summary Quiz:}

Referencing a steam table, determine the rate of heat released by steam flowing through a heat exchanger at a flow rate of 10 pounds per hour, assuming a entry temperature of 400 $^{o}$F and pressure of 230 PSIG, exiting as hot water (condensate) at 140 $^{o}$F and atmospheric pressure.

\begin{itemize}
\item{} 11690 BTU/hr
\vskip 5pt 
\item{} 10930 BTU/hr
\vskip 5pt 
\item{} 12010 BTU/hr
\vskip 5pt 
\item{} 10610 BTU/hr
\vskip 5pt 
\item{} 13020 BTU/hr
\end{itemize}

%INDEX% Physics, heat and temperature: steam table
%INDEX% Process: solar steam generation (GlassPoint)

%(END_NOTES)


