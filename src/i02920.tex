
%(BEGIN_QUESTION)
% Copyright 2010, Tony R. Kuphaldt, released under the Creative Commons Attribution License (v 1.0)
% This means you may do almost anything with this work of mine, so long as you give me proper credit

Read and outline the ``Introduction to Pseudocode'' subsection of the ``Digital PID Algorithms'' section of the ``Closed-Loop Control'' chapter in your {\it Lessons In Industrial Instrumentation} textbook.  Note the page numbers where important illustrations, photographs, equations, tables, and other relevant details are found.  Prepare to thoughtfully discuss with your instructor and classmates the concepts and examples explored in this reading.

\vskip 10pt

Feel free to skip past the portions of this subsection discussing {\it branching} and {\it functions}.

\underbar{file i02920}
%(END_QUESTION)





%(BEGIN_ANSWER)

 
%(END_ANSWER)





%(BEGIN_NOTES)

Text-based computer languages consist of instructions given to the computer, followed in order from top to bottom like reading a text.  ``Pseudocode'' is an imaginary language, but similar enough to real computer languages to translate well to them.  ``LOOP'' and ``ENDLOOP'' markers delineate a section of code that is repeatedly re-executed.  Double-slash marks (//) represent the beginning of comment statements, aimed at explaining the program to a human reader, not instructions to be followed by the computer as it executes the program.

\vskip 10pt

Variables for mathematical operations may be declared in a program, allocating space in the computer's memory to hold numerical values of different types (boolean, integer, floating-point, etc.).  ``SET'' statements assign numerical values to variables.  The ``equals'' symbol (=) represents the assigning of the value, not a true expression of equality.  Thus, SET $x = 2-x$ is an instruction for the computer to calculate the quantity $2-x$ and then assign that quantity as the new value of $x$.  In the context of an ``IF'' statement, an equals symbol becomes a test for equality rather than an assignment of new value.  Some formal computer languages make this distinction more explicit, using a symbol such as (==) as an equality test (with = meaning assignment).  Structured text programming for PLCs uses a single equals sign (=) as an equality test while using a colon and an equals sign (:=) to represent assignment.

\vskip 10pt

Branching refers to a program's ability to change the flow of execution based on conditions: to alter which instructions get executed in what order.  Encapsulating a section of code into its own ``function'' or ``subroutine'' which may then be conditionally {\it called} by the main program is one way to do this.  In this version of pseudocode, BEGIN and END markers declare the boundaries of a subroutine or function.  Numerical values may be {\it passed} to functions and then manipulated under different variable names within that function.  The numerical result of a function is then {\it returned} to the main program routine.

Functions may even call themselves, a process known as {\it recursion}.





\vskip 20pt \vbox{\hrule \hbox{\strut \vrule{} {\bf Suggestions for Socratic discussion} \vrule} \hrule}

\begin{itemize}
\item{} Explain what each part of the example pseudocode programs in the textbook do, starting with the most basic ``Hello World'' program.  Be sure to differentiate between instructions, arguments, comments, loops, etc.
\item{} Explain what a ``loop'' is in a text-based computer program.
\item{} Give examples of where you might use the ``='' operator to {\it assign} numerical values, versus using it to test for equality.
\item{} Refer to the pseudocode {\it differentiator} function in the ``Numerical Differentiation'' section of the ``Calculus'' chapter in {\it Lessons In Industrial Instrumentation}, explaining how that program functions.  Feel free to refer to the graph of pipeline pressure, and relate the code to the analysis of pipeline pressure over time.
\item{} Refer to the pseudocode {\it integrator} function in the ``Numerical Integration'' section of the ``Calculus'' chapter in {\it Lessons In Industrial Instrumentation}, explaining how that program functions.  Feel free to refer to a graph of flow rate, and relate the code to the analysis of flow rate over time.  
\item{} Write a simple pseudocode program that generates one of the following numerical sequences:
\begin{itemize}

\item{} 1, 2, 4, 8, 16, 32 . . .
\item{} 1, 3, 9, 27, 81, 243 . . .
\item{} Edit the simple pseudocode program to stop executing when a certain value is reached.
\end{itemize}
\end{itemize}




%INDEX% Reading assignment: Lessons In Industrial Instrumentation, Closed-Loop Control -- introduction to pseudocode)

%(END_NOTES)


