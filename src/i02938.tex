
%(BEGIN_QUESTION)
% Copyright 2007, Tony R. Kuphaldt, released under the Creative Commons Attribution License (v 1.0)
% This means you may do almost anything with this work of mine, so long as you give me proper credit

Complete the following table of equivalent pressures:

% No blank lines allowed between lines of an \halign structure!
% I use comments (%) instead, so that TeX doesn't choke.

$$\vbox{\offinterlineskip
\halign{\strut
\vrule \quad\hfil # \ \hfil & 
\vrule \quad\hfil # \ \hfil & 
\vrule \quad\hfil # \ \hfil & 
\vrule \quad\hfil # \ \hfil \vrule \cr
\noalign{\hrule}
%
% First row
\hskip 20pt PSIG \hskip 20pt & \hskip 20pt PSIA \hskip 20pt & \hskip 20pt inches Hg (G) \hskip 20pt & \hskip 20pt inches W.C. (G) \hskip 20pt \cr
%
\noalign{\hrule}
%
% Another row
18 &  &  &  \cr
%
\noalign{\hrule}
%
% Another row
  & 400 &  &  \cr
%
\noalign{\hrule}
%
% Another row
  &  & 33 &  \cr
%
\noalign{\hrule}
%
% Another row
  &  &  & 60 \cr
%
\noalign{\hrule}
%
% Another row
  &  & 452 &  \cr
%
\noalign{\hrule}
%
% Another row
  &  &  & 12 \cr
%
\noalign{\hrule}
%
% Another row
  & 1 &  &  \cr
%
\noalign{\hrule}
%
% Another row
-5  &  &  &  \cr
%
\noalign{\hrule}
} % End of \halign 
}$$ % End of \vbox

\vskip 10pt

There is a technique for converting between different units of measurement called ``unity fractions'' which is imperative for students of Instrumentation to master.  For more information on the ``unity fraction'' method of unit conversion, refer to the ``Unity Fractions" subsection of the ``Unit Conversions and Physical Constants'' section of the ``Physics'' chapter in your {\it Lessons In Industrial Instrumentation} textbook.

\underbar{file i02938}
%(END_QUESTION)





%(BEGIN_ANSWER)

$$\vbox{\offinterlineskip
\halign{\strut
\vrule \quad\hfil # \ \hfil & 
\vrule \quad\hfil # \ \hfil & 
\vrule \quad\hfil # \ \hfil & 
\vrule \quad\hfil # \ \hfil \vrule \cr
\noalign{\hrule}
%
% First row
\hskip 20pt PSIG \hskip 20pt & \hskip 20pt PSIA \hskip 20pt & \hskip 20pt inches Hg (G) \hskip 20pt & \hskip 20pt inches W.C. (G) \hskip 20pt \cr
%
\noalign{\hrule}
%
% Another row
18 & 32.7 & 36.65 & 498.25 \cr
%
\noalign{\hrule}
%
% Another row
385.3 & 400 & 784.5 & 10665 \cr
%
\noalign{\hrule}
%
% Another row
16.21 & 30.91 & 33 & 448.6 \cr
%
\noalign{\hrule}
%
% Another row
2.168 & 16.87 & 4.413 & 60 \cr
%
\noalign{\hrule}
%
% Another row
222.0 & 236.7 & 452 & 6145.1 \cr
%
\noalign{\hrule}
%
% Another row
0.4335 & 15.13 & 0.8826 & 12 \cr
%
\noalign{\hrule}
%
% Another row
-13.7 & 1 & -27.89 & -379.2 \cr
%
\noalign{\hrule}
%
% Another row
-5 & 9.7 & -10.18 & -138.4 \cr
%
\noalign{\hrule}
} % End of \halign 
}$$ % End of \vbox

%(END_ANSWER)





%(BEGIN_NOTES)

\vfil \eject

\noindent
{\bf Summary Quiz:}

Convert a pressure of 35.2 PSIG into inches of water column (gauge):

\begin{itemize}
\item{} 62.88 "W.C
\vskip 5pt 
\item{} 71.67 "W.C.
\vskip 5pt 
\item{} 1.272 "W.C.
\vskip 5pt 
\item{} 242.7 "W.C.
\vskip 5pt 
\item{} 974.3 "W.C.
\vskip 5pt 
\item{} 7.52 "W.C.
\end{itemize}


%INDEX% Physics, units and conversions: pressure

%(END_NOTES)


