
%(BEGIN_QUESTION)
% Copyright 2007, Tony R. Kuphaldt, released under the Creative Commons Attribution License (v 1.0)
% This means you may do almost anything with this work of mine, so long as you give me proper credit

Complete the following table of equivalent pressures:

% No blank lines allowed between lines of an \halign structure!
% I use comments (%) instead, so that TeX doesn't choke.

$$\vbox{\offinterlineskip
\halign{\strut
\vrule \quad\hfil # \ \hfil & 
\vrule \quad\hfil # \ \hfil & 
\vrule \quad\hfil # \ \hfil & 
\vrule \quad\hfil # \ \hfil \vrule \cr
\noalign{\hrule}
%
% First row
\hskip 20pt Atm \hskip 20pt & \hskip 20pt PSIG \hskip 20pt & \hskip 20pt inches W.C. (G) \hskip 20pt & \hskip 20pt PSIA \hskip 20pt \cr
%
\noalign{\hrule}
%
% Another row
3.5 &  &  &  \cr
%
\noalign{\hrule}
%
% Another row
  & 81 &  &  \cr
%
\noalign{\hrule}
%
% Another row
  &  & 8834 &  \cr
%
\noalign{\hrule}
%
% Another row
  &  &  & 0 \cr
%
\noalign{\hrule}
%
% Another row
  &  & 7.12 &  \cr
%
\noalign{\hrule}
%
% Another row
  &  &  & 368 \cr
%
\noalign{\hrule}
%
% Another row
  & 2 &  &  \cr
%
\noalign{\hrule}
%
% Another row
100 &  &  &  \cr
%
\noalign{\hrule}
} % End of \halign 
}$$ % End of \vbox

\vskip 10pt

There is a technique for converting between different units of measurement called ``unity fractions'' which is imperative for students of Instrumentation to master.  For more information on the ``unity fraction'' method of unit conversion, refer to the ``Unity Fractions" subsection of the ``Unit Conversions and Physical Constants'' section of the ``Physics'' chapter in your {\it Lessons In Industrial Instrumentation} textbook.

\underbar{file i02939}
%(END_QUESTION)





%(BEGIN_ANSWER)

$$\vbox{\offinterlineskip
\halign{\strut
\vrule \quad\hfil # \ \hfil & 
\vrule \quad\hfil # \ \hfil & 
\vrule \quad\hfil # \ \hfil & 
\vrule \quad\hfil # \ \hfil \vrule \cr
\noalign{\hrule}
%
% First row
\hskip 20pt Atm \hskip 20pt & \hskip 20pt PSIG \hskip 20pt & \hskip 20pt inches W.C. (G) \hskip 20pt & \hskip 20pt PSIA \hskip 20pt \cr
%
\noalign{\hrule}
%
% Another row
3.5 & 36.75 & 1017.3 & 51.45 \cr
%
\noalign{\hrule}
%
% Another row
6.51 & 81 & 2242 & 95.7 \cr
%
\noalign{\hrule}
%
% Another row
22.71 & 319.1 & 8834 & 333.8 \cr
%
\noalign{\hrule}
%
% Another row
0 & -14.7 & -406.9 & 0 \cr
%
\noalign{\hrule}
%
% Another row
1.017 & 0.2572 & 7.12 & 14.96 \cr
%
\noalign{\hrule}
%
% Another row
25.03 & 353.3 & 9779.6 & 368 \cr
%
\noalign{\hrule}
%
% Another row
1.136 & 2 & 55.36 & 16.7 \cr
%
\noalign{\hrule}
%
% Another row
100 & 1455.3 & 40284 & 1470 \cr
%
\noalign{\hrule}
} % End of \halign 
}$$ % End of \vbox

%(END_ANSWER)





%(BEGIN_NOTES)

%INDEX% Physics, units and conversions: pressure

%(END_NOTES)


