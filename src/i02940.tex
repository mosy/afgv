
%(BEGIN_QUESTION)
% Copyright 2007, Tony R. Kuphaldt, released under the Creative Commons Attribution License (v 1.0)
% This means you may do almost anything with this work of mine, so long as you give me proper credit

Explain what is wrong with this attempt to convert a gauge pressure of 65 PSI into units of atmospheres (atm):

$$\left({{65 \hbox{ PSI}} \over {1}}\right) \left({{1 \hbox{ atm}} \over {14.7 \hbox{ PSI}}}\right) = 4.422 \hbox{ atm}$$

\vskip 20pt \vbox{\hrule \hbox{\strut \vrule{} {\bf Suggestions for Socratic discussion} \vrule} \hrule}

\begin{itemize}
\item{} The mistake made here is common for new students to make as they learn to do pressure unit conversions.  Identify a ``sure-fire'' way to identify and avoid this mistake.
\end{itemize}

\underbar{file i02940}
%(END_QUESTION)





%(BEGIN_ANSWER)

The given pressure is in units of PSI {\it gauge} (PSIG), while the final unit (atmospheres) is an {\it absolute} pressure unit.  In order for this conversion to be correct, there must somewhere be an offset (addition) in the calculation to account for the 14.7 PSI shift between gauge pressure and absolute pressure.

\vskip 10pt

Here is the proper conversion technique:

$$\hbox{\bf Step 1: }65 \hbox{ PSIG} + 14.7 \hbox{ PSI} = 79.7 \hbox{ PSIA}$$

$$\hbox{\bf Step 2: } \left({{79.7 \hbox{ PSIA}} \over {1}}\right) \left({{1 \hbox{ atm}} \over {14.7 \hbox{ PSIA}}}\right) = 5.422 \hbox{ atm}$$

%(END_ANSWER)





%(BEGIN_NOTES)


%INDEX% Physics, units and conversions: pressure

%(END_NOTES)


