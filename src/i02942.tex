
%(BEGIN_QUESTION)
% Copyright 2007, Tony R. Kuphaldt, released under the Creative Commons Attribution License (v 1.0)
% This means you may do almost anything with this work of mine, so long as you give me proper credit

2.036 inches of mercury ("Hg) is an equivalent pressure to 27.68 inches of water ("W.C. or "H$_{2}$O).  This fact allows us to create a ``unity fraction'' from these two quantities for use in converting pressure units from inches mercury to inches water or vice-versa.  Two examples are shown here:

$$\left({{310 \hbox{ "Hg}} \over {1}}\right) \left({{27.68 \hbox{ "W.C.}} \over {2.036 \hbox{ "Hg}}}\right) = 4215 \hbox{ "W.C.}$$

$$\left({{45 \hbox{ "W.C.}} \over {1}}\right) \left({2.036 \hbox{ "Hg}} \over {{27.68 \hbox{ "W.C.}}}\right) = 3.31 \hbox{ "Hg}$$

But what if we are performing a unit conversion where the initial pressure is given in inches of mercury or inches of water {\it absolute}?  Can we properly make a unity fraction with the quantities 2.036 "HgA and 27.68 "W.C.A as in the following examples?  

$$\left({{310 \hbox{ "HgA}} \over {1}}\right) \left({{27.68 \hbox{ "W.C.A}} \over {2.036 \hbox{ "HgA}}}\right) = 4215 \hbox{ "W.C.A}$$

$$\left({{45 \hbox{ "W.C.A}} \over {1}}\right) \left({2.036 \hbox{ "HgA}} \over {{27.68 \hbox{ "W.C.A}}}\right) = 3.31 \hbox{ "HgA}$$

Explain why or why not.

\underbar{file i02942}
%(END_QUESTION)





%(BEGIN_ANSWER)

This is perfectly legitimate, because in either case all the pressure units involved in each conversion are of the same type: either all gauge or all absolute.  Where we encounter difficulties is if we try to mix different units in the same ``unity fraction'' conversion that do not share a common zero point.  

A classic example of this mistake is trying to do a temperature conversion from degrees F to degrees C using unity fractions (e.g. 100$^{o}$ C = 212$^{o}$ F):

$$\left({{60^o \hbox{ F}} \over {1}}\right)  \left({{100^o \hbox{ C}} \over {212^o \hbox{ F}}}\right) \neq 28.3^o \hbox{ C}$$

This cannot work because the technique of unity fractions is based on proportion, and there is no simple proportional relationship between degrees F and degrees C; rather, there is an {\it offset} of 32 degrees between the two temperature scales.  The only way to properly manage this offset in the calculation is to include an appropriate addition or subtraction (as needed).

However, if there is no offset between the units involved in a conversion problem, there is no need to add or subtract anything, and we may perform the entire conversion using nothing but multiplication and division (unity fractions).  Such is the case if we convert pressure units that are all gauge, or if we convert pressure units that are all absolute.

\vskip 10pt

To summarize, it is perfectly acceptable to construct a unity fraction of $27.68 \hbox{ "W.C.} \over 2.036 \hbox{ "Hg}$ because 0 "W.C. is the same as 0 "Hg (i.e. they share the same zero point; there is no offset between units "W.C. and "Hg).  Likewise, it is perfectly acceptable to construct a unity fraction of $27.68 \hbox{ "W.C.A} \over 2.036 \hbox{ "HgA}$ because 0 "W.C.A is the same as 0 "HgA (i.e. they share the same zero point; there is no offset between units "W.C.A and "HgA).

%(END_ANSWER)





%(BEGIN_NOTES)

%INDEX% Physics, units and conversions: pressure

%(END_NOTES)


