
%(BEGIN_QUESTION)
% Copyright 2007, Tony R. Kuphaldt, released under the Creative Commons Attribution License (v 1.0)
% This means you may do almost anything with this work of mine, so long as you give me proper credit

Elevation ($z$) and pressure ($P$) readings are taken at two different points in a piping system carrying liquid benzene ($\gamma$ = 56.1 lb/ft$^{3}$):

\vskip 10pt

$z_1$ = 50 inches \hskip 50pt $z_2$ = 34 inches

\vskip 10pt

$P_1$ = 70 PSI \hskip 59pt $P_2$ = 69 PSI

\vskip 10pt

Calculate the fluid velocity at point 2 ($v_2$) if the velocity at point 1 is known to be equal to 5 feet per second ($v_1$ = 5 ft/s).

\vskip 10pt

\noindent
{\bf Bernoulli's equation:}

$$z_1 \rho g + {v_1^2 \rho \over 2} + P_1 = z_2 \rho g + {v_2^2 \rho \over 2} + P_2$$

\underbar{file i02988}
%(END_QUESTION)





%(BEGIN_ANSWER)

$v_2$ = 16.57 ft/s

\vskip 10pt

Note that the two pressures are given in units of PSI (not pounds per square {\it foot}), and that the two heights are given in inches instead of feet.  Also, $\rho_{benzene}$ = 1.753 slugs/ft$^{3}$.

%(END_ANSWER)





%(BEGIN_NOTES)


%INDEX% Physics, dynamic fluids: Bernoulli's equation

%(END_NOTES)


