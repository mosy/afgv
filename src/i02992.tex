
%(BEGIN_QUESTION)
% Copyright 2007, Tony R. Kuphaldt, released under the Creative Commons Attribution License (v 1.0)
% This means you may do almost anything with this work of mine, so long as you give me proper credit

An absolute pressure gauge is connected to a hollow metal sphere containing a gas:

$$\includegraphics[width=15.5cm]{i02992x01.eps}$$

According to the Ideal Gas Law, the relationship between the gauge's pressure indication and the sphere's temperature is as follows:

$$PV = nRT$$

Unfortunately, though, we do not happen to know the volume of the sphere ($V$) or the number of moles of gas contained within ($n$).  At best, all we can do is express the relationship between $P$ and $T$ as a proportionality, or as an equality with a {\it constant of proportionality} ($k$) accounting for all the unknown variables and unit conversions:

$$P \propto T \hbox{\hskip 50pt} P = kT$$

Calculate the value of this constant ($k$) if you happen to know that the pressure gauge registers 1.5 bar (absolute) at a temperature of 280 K.  Then, predict the temperature when the pressure gauge reads 1.96 bar (absolute).

\underbar{file i02992}
%(END_QUESTION)





%(BEGIN_ANSWER)

$k$ = 0.00536

\vskip 10pt

$T$ = 365.9 K at $P$ = 1.96 bar

%(END_ANSWER)





%(BEGIN_NOTES)


%INDEX% Mathematics, proportionalities converted into equalities

%(END_NOTES)


