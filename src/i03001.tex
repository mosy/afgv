
%(BEGIN_QUESTION)
% Copyright 2015, Tony R. Kuphaldt, released under the Creative Commons Attribution License (v 1.0)
% This means you may do almost anything with this work of mine, so long as you give me proper credit

When an internal combustion engine starts up cold, {\it water} may be seen coming out of the exhaust pipe, at least until the exhaust pipes become hot enough to vaporize the water so you can't see it anymore.  Explain why water is a byproduct of any hydrocarbon fuel (e.g. gasoline or diesel) combustion, based on principles of chemistry.

\vskip 10pt

When coal is partially combusted (i.e. burned in a low-oxygen environment), one of the byproduct gases is the flammable and toxic gas {\it carbon monoxide} (CO).  This is also one of the major products of {\it biomass gasification}, where carbon-containing biomass fuels are heated in low-oxygen environments.  Explain why an engine running on carbon monoxide gas as a fuel will {\it not} produce water as a byproduct of combustion.

\vskip 20pt \vbox{\hrule \hbox{\strut \vrule{} {\bf Suggestions for Socratic discussion} \vrule} \hrule}

\begin{itemize}
\item{} Is it possible to eliminate water from the exhaust gases of a car's engine operating on gasoline?
\item{} Identify how we may identify fuels producing water vapor when burned, strictly by examining their chemical formulae.
\item{} Identify how we may identify fuels which inherently produce no water vapor when burned, strictly by examining their chemical formulae.
\end{itemize}

\underbar{file i03001}
%(END_QUESTION)





%(BEGIN_ANSWER)

Hint: gasoline fuel molecules contain lots of {\it hydrogen} atoms!

%(END_ANSWER)





%(BEGIN_NOTES)

Any hydrogenous (hydrogen-containing) fuel will produce water as a reaction product when burned with oxygen.  The only fuels that don't produce water are those containing no hydrogen, such as carbon monoxide (CO).

\vskip 10pt

Carbon monoxide gas (CO) combusts with oxygen to produce carbon dioxide (CO$_{2}$) as the only product of combustion.  It is indeed possible to run an internal combustion engine on CO gas, as it has been done by home experimenters in the past.  This very thing was done during World War II in Germany, when fuel rations prevented German citizens from burning gasoline in their cars.  Many built ``wood gasifiers'' to generate carbon monoxide gas from incomplete combustion of wood.

Mother Earth News magazine once had an article describing how to build a wood gasifier to produce CO gas from incomplete wood combustion, which could then be piped to the engine of a pickup truck.  The gasifier assembly fit in the bed of the truck.  Incidentally, the truck engine was a 454 cubic inch ``big-block'' Chevrolet V-8 with a raised compression ratio (10:1 or 11:1 if memory serves correctly).  CO gas has a very high octane number, much like methane gas, and so the engine's compression ratio needed to be very high in order to achieve good efficiency.

FEMA (Federal Emergency Management Agency) has also published a document showing how to construct a wood gasifier from simple components for use in petroleum shortage emergencies.

%INDEX% Chemistry: products of combustion
%INDEX% Process: biomass gasification

%(END_NOTES)


