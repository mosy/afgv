
%(BEGIN_QUESTION)
% Copyright 2007, Tony R. Kuphaldt, released under the Creative Commons Attribution License (v 1.0)
% This means you may do almost anything with this work of mine, so long as you give me proper credit

An inventor claims to have built an automobile that runs on water.  Opening the hood of this car, he points to a device he calls an ``electrolyzer,'' which uses electricity from the car's battery to split ordinary water into hydrogen and oxygen gas.  These gases are then fed into the car's internal combustion engine as fuel and oxidizer, respectively.  The car's engine turns a generator which replenishes the battery powering the electrolyzer.  His claim is that the only consumable is the water itself, and points to the fact that the electrolyzer must be periodically refilled with water after driving the car.

\vskip 10pt

Explain how we can tell this is a fraudulent claim, based on what you know of chemistry and physics.

\underbar{file i03002}
%(END_QUESTION)





%(BEGIN_ANSWER)

The inventor's real claim is not that he made a car run on water, but that he found a way to (magically) amplify the energy contained in the battery to not only fully replenish the battery's reserves but also to propel the car.  This is really a claim of perpetual motion: getting more energy out of a system than what is put in.

Some will argue we cannot dismiss the inventor's claim simply because it contradicts well-established laws of physics.  Science does not yield absolute knowledge, and so there must always be some room for open-mindedness, right?  The problem with this argument is that it ignores scale.  The Law of Energy Conservation is {\it so well established}, on {\it so many levels}, and in {\it so many different fields} of research, that it stands as one of the most well-substantiated principles ever discovered in science.  On the other hand, we all know full well the temptation for people to lie and deceive (even themselves) when there is a vested interest at hand (e.g. fame, fortune, and/or strong emotional satisfaction).  The odds are literally millions to one that this claim is false, and that the inventor is either knowingly perpetrating a fraud or terribly self-deceived.  The plea for open-mindedness comes at the expense of ignoring overwhelming odds.  True, there is always a {\it chance} that someone will discover an exception to a physical Law, but we need to recognize just how slim that chance is when the Law in question is the Conservation of Energy (or Conservation of Mass, for that matter).  And, if the claim is indeed true, there is a prize waiting in Stockholm for anyone able to overthrow one of the best-established laws in all of science.

\vskip 10pt

The argument that oil companies actively suppress this technology is laughable on several fronts:

\begin{itemize}
\item{} Oil companies are not just {\it oil} companies.  If they were that small-minded, they would have gone out of business long ago.  Oil companies are in fact {\it energy} companies.  This is why many of them support solar energy divisions and other renewable research efforts.  They can see the inevitability of oil shortage as well as anyone else, and they have no desire to be caught off guard when it happens.
\vskip 5pt
\item{} If the inventor wished, he could easily build his own stationary power plant using this technology and power his home for (virtually) free.  These power plants could then be marketed to anyone and everyone with little capital investment, forming a brand-new market that would not compete with existing oil companies' markets.
\vskip 5pt
\item{} The military (of any country) would be {\it extremely} interested in finding ways to power their trucks, tanks, airplanes, ships, and submarines with nothing but water.  The tactical advantage of water-fueled machines would be so great that no attempted interference by private industry would stop a major military power (such as the United States of America, China, or Russia) from exploiting it to the fullest.  Those who may claim that the military is controlled by oil interests and therefore forbidden to pursue alternatives conveniently ignore the face that the major world militaries are the biggest users of {\it nuclear power}, having abandoned oil as the fuel of choice for aircraft carriers and submarines long ago for precisely the same reasons they would be willing to embrace ``water power'' if it existed.
\end{itemize}

%(END_ANSWER)





%(BEGIN_NOTES)


%INDEX% Chemistry, basic: molecular bonds and energy exchange

%(END_NOTES)


