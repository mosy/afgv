
%(BEGIN_QUESTION)
% Copyright 2007, Tony R. Kuphaldt, released under the Creative Commons Attribution License (v 1.0)
% This means you may do almost anything with this work of mine, so long as you give me proper credit

When an electric current is passed through a tube containing pure hydrogen gas, a characteristic red glow is emitted.  If the tube is emptied and then filled with a different gaseous element (such as neon), a different color will be emitted, also characteristic to that element.

The fact that each element emits a color of light characteristic to that element tells us something about the electrons in the atoms of those elements.  What does it indicate?  

\vskip 10pt

Hint: this has something to do with electrons inhabiting discrete ``shells'' around an atom's nucleus, in contrast to the oversimplified image of electrons whirling around nuclei like tiny satellites.

\underbar{file i03003}
%(END_QUESTION)





%(BEGIN_ANSWER)

The color of light emitted by an ``excited'' atom is due to energy emitted by its electrons after having absorbed energy from an external source (in this case, the electric current through the tube).  It is known that free electrons (unbound to an atom) are able to absorb and release energy over a wide range of levels (corresponding to a wide range of optical colors).  If electrons really were free-ranging ``satellites'' whirling around atomic nuclei, then they ought to be able to release a wide range of light colors just as they do when free.  However, the electrons bound to an atom's nucleus do not do so.  Rather, they emit light of very specific frequencies (colors) in the same way a tuning fork emits the same characteristic sound pitch, no matter how hard or how soft it has been struck.

This phenomenon clearly shows that electrons fall into discrete ``states'' when part of an atom.  It is this discrete, or {\it quantum} nature of atomic electrons that gives rise to predictable chemical behavior (bonds between atoms formed by electron interaction), and the periodicity of Mendeleyev's table.

%(END_ANSWER)





%(BEGIN_NOTES)

%INDEX% Physics, atomic: electron shells 
%INDEX% Chemistry, basic principles: electron shells 

%(END_NOTES)


