
%(BEGIN_QUESTION)
% Copyright 2007, Tony R. Kuphaldt, released under the Creative Commons Attribution License (v 1.0)
% This means you may do almost anything with this work of mine, so long as you give me proper credit

If a 2 kilogram sample of pure water is mixed with 1 gram of pure H$_{2}$SO$_{4}$, what will the resulting sulfuric acid's {\it molality} be?  Note: I am asking you to calculate molality ($m$), not molarity ($M$), of the acid solution!

\underbar{file i03005}
%(END_QUESTION)





%(BEGIN_ANSWER)

{\it Molality} is defined as the number of moles of solute per kilograms of solvent (not of the total solution!).  We were given the solvent mass in kilograms (2 kg) already.  All we need to know is the molar quantity of the pure acid and we can solve for molality.

First, let's tally the number of grams per mole for H$_{2}$SO$_{4}$, based on the atomic mass units (amu) for the constituent elements:

\vskip 10pt

\noindent
For H$_{2}$SO$_{4}$, each molecule contains:

\begin{itemize}
\item{} 2 atoms of H at 1.01 amu each
\item{} 1 atom of S at 32.06 amu each
\item{} 4 atoms of O at 16 amu each
\end{itemize}

This gives a total of 98.08 grams per mole of pure H$_{2}$SO$_{4}$.

\vskip 10pt

This figure may be used as a unity fraction to convert moles into grams, or grams into moles.  For our application, we need to convert the given mass of 1 gram into moles:

\vskip 10pt

(1 g)(1 mol / 98.08 g) = 0.0102 mol

\vskip 10pt

Taking this quantity in moles and dividing by the mass of solvent (2 kg) gives us the molality of the acid solution:

\vskip 10pt

0.0102 mol / 2 kg = 0.0051 $m$

%(END_ANSWER)





%(BEGIN_NOTES)


%INDEX% Chemistry: molality

%(END_NOTES)


