
%(BEGIN_QUESTION)
% Copyright 2007, Tony R. Kuphaldt, released under the Creative Commons Attribution License (v 1.0)
% This means you may do almost anything with this work of mine, so long as you give me proper credit

Calculate the pH of the following aqueous solutions and the voltage (ideally) generated by a pH electrode pair, given the hydrogen ion molarity of each solution.  Assume a solution temperature of 25$^{o}$ C:

% No blank lines allowed between lines of an \halign structure!
% I use comments (%) instead, so that TeX doesn't choke.

$$\vbox{\offinterlineskip
\halign{\strut
\vrule \quad\hfil # \ \hfil & 
\vrule \quad\hfil # \ \hfil & 
\vrule \quad\hfil # \ \hfil \vrule \cr
\noalign{\hrule}
%
% First row
[H$^{+}$] & pH & $V_{probe}$ \cr
%
\noalign{\hrule}
%
% Another row
0.001995 $M$ &  & \cr
%
\noalign{\hrule}
%
% Another row
6.309 $\times$ 10$^{-7}$ $M$ &  & \cr
%
\noalign{\hrule}
%
% Another row
7.943 $\times$ 10$^{-13}$ $M$ &  & \cr
%
\noalign{\hrule}
%
% Another row
3.881 $\times$ 10$^{-5}$ $M$ &  & \cr
%
\noalign{\hrule}
%
% Another row
1.452 $\times$ 10$^{-11}$ $M$ &  & \cr
%
\noalign{\hrule}
} % End of \halign 
}$$ % End of \vbox

\vskip 20pt \vbox{\hrule \hbox{\strut \vrule{} {\bf Suggestions for Socratic discussion} \vrule} \hrule}

\begin{itemize}
\item{} How does a pH instrument tell the difference between a pH value above 7 versus one below 7?
\item{} Demonstrate how to {\it estimate} numerical answers for this problem without using a calculator.
\end{itemize}

\underbar{file i03006}
%(END_QUESTION)





%(BEGIN_ANSWER)

% No blank lines allowed between lines of an \halign structure!
% I use comments (%) instead, so that TeX doesn't choke.

$$\vbox{\offinterlineskip
\halign{\strut
\vrule \quad\hfil # \ \hfil & 
\vrule \quad\hfil # \ \hfil & 
\vrule \quad\hfil # \ \hfil \vrule \cr
\noalign{\hrule}
%
% First row
[H$^{+}$] & pH & $V_{probe}$ \cr
%
\noalign{\hrule}
%
% Another row
0.001995 $M$ & 2.7 pH & 254.4 mV \cr
%
\noalign{\hrule}
%
% Another row
6.309 $\times$ 10$^{-7}$ $M$ & 6.2 pH & 47.32 mV \cr
%
\noalign{\hrule}
%
% Another row
7.943 $\times$ 10$^{-13}$ $M$ & 12.1 pH & $-301.7$ mV \cr
%
\noalign{\hrule}
%
% Another row
3.881 $\times$ 10$^{-5}$ $M$ & 4.41 pH & 153.2 mV \cr
%
\noalign{\hrule}
%
% Another row
1.452 $\times$ 10$^{-11}$ $M$ & 10.84 pH & $-227.1$ mV \cr
%
\noalign{\hrule}
} % End of \halign 
}$$ % End of \vbox

%(END_ANSWER)





%(BEGIN_NOTES)

\noindent
{\bf Mathematical definition of pH}:

$$\hbox{pH} = - \log [\hbox{H}^{+}]$$

\vskip 10pt

\noindent
{\bf Nernst equation}:

$$V = {{R T} \over {nF}} \ln \left({C_1 \over C_2}\right)$$

\noindent
Where,

$V$ = Voltage produced across membrane due to ion exchange, in volts (V)

$R$ = Universal gas constant (8.315 J/mol$\cdot$K)

$T$ = Absolute temperature, in Kelvin (K)

$n$ = Number of electrons transferred per ion exchanged (unitless)

$F$ = Faraday constant, in coulombs per mole (96,485 C/mol e$^{-}$)

$C_1$ = Concentration of ion in measured solution, in moles per liter of solution ($M$)

$C_2$ = Concentration of ion in reference solution (on other side of membrane), in moles per liter of solution ($M$)

\vskip 20pt

\noindent
For this particular problem,

\vskip 10pt

$C_1$ = [H$^{+}$] (Hydrogen ion molarity for the solution)

$C_1$ = 1.00 $\times$ 10$^{-7}$ (Hydrogen ion molarity for the KCl buffer solution within the pH probe)


%INDEX% Chemistry, pH: molarity calculation

%(END_NOTES)


