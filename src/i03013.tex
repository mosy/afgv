
%(BEGIN_QUESTION)
% Copyright 2014, Tony R. Kuphaldt, released under the Creative Commons Attribution License (v 1.0)
% This means you may do almost anything with this work of mine, so long as you give me proper credit

Read selected portions of the NFPA 70E document ``Standard for Electrical Safety in the Workplace'' and answer the following questions:

\vskip 10pt

Annex K (``General Categories of Electrical Hazards'') is a short appendix section of this document, giving some background information on {\it electric shock}, {\it arc flash}, and {\it arc blast}.  Read the definitions given for these hazards, and then define them using your own words.

\vskip 10pt

Article 100 (``Definitions'') of this document defines some key terms used throughout the standard.  Read the definitions given for the following terms and then define them using your own words:

\begin{itemize}
\item{} Arc flash hazard analysis
\item{} Arc flash suit
\item{} Incident energy
\item{} Ground fault
\item{} Qualified person
\end{itemize}

\underbar{file i03013}
%(END_QUESTION)





%(BEGIN_ANSWER)

 
%(END_ANSWER)





%(BEGIN_NOTES)

\begin{itemize}
\item{} {\bf Electrical shock} is a condition where electricity of sufficient magnitude passes through the body of a living organism to cause ill effect.  {\bf Electrocution} is death by electric shock, with an incident rate of about 1000 annually.  More than half of all electrocutions occur on systems less than 600 volts.
\item{} {\bf Arc flash} is the high temperature created by an electric arc in open air, capable of causing severe burns to the body.  Over 2000 people annually hurt due to arc flash.  Can be lethal at distances of 10 feet.
\item{} {\bf Arc blast} is the percussive effect of rapid expansion of air and metal in the arc path (e.g. copper conductors), capable of causing shrapnel and other violent effects.  Speeds of up to 700 miles per hour have been estimated for shrapnel jettisoned by arc blast!
\end{itemize}

\vskip 10pt

\begin{itemize}
\item{} {\bf Arc flash hazard analysis} -- a technical investigation of arc flash energy levels for the purpose of establishing boundaries and PPE requirements.
\item{} {\bf Arc flash suit} -- a complete garment designed to shield the wearer from thermal energy resulting from an arc flash incident.
\item{} {\bf Incident energy} -- the amount of energy delivered to a surface from an arc event, typically measured in calories per square centimeter or joules per square centimeter.
\item{} {\bf Ground fault} -- an unintentional connection of an ungrounded conductor to earth ground.
\item{} {\bf Qualified person} -- one skilled and knowledgeable in the hazards, able to independently recognize and avoid those hazards.
\end{itemize}








\vskip 20pt \vbox{\hrule \hbox{\strut \vrule{} {\bf Suggestions for Socratic discussion} \vrule} \hrule}

\begin{itemize}
\item{} Compare the relative hazards of electric shock versus arc flash.  {\it The majority of hospital admissions from electrical injuries are from arc flash, not electric shock!}
\item{} Explain why the expansion rate of copper (67,000$\times$ from solid to vapor) is relevant to certain electrical hazards.
\item{} Is the incident energy of an arc flash event affected by the distance from the fault, or is it strictly a function of arc intensity?
\item{} Are all arc flash suits created equal, or are there substantial differences between them?  How could you (as a wearer of such a suit) tell?
\item{} At what point do you think a graduate of the Instrumentation program might be regarded as a ``qualified person''?
\end{itemize}

%INDEX% Safety, electrical: arc flash
%INDEX% Safety, electrical: shock
%INDEX% Safety, electrical: NFPA 70E Standard for Electrical Safety in the Workplace

%(END_NOTES)


