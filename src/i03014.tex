
%(BEGIN_QUESTION)
% Copyright 2014, Tony R. Kuphaldt, released under the Creative Commons Attribution License (v 1.0)
% This means you may do almost anything with this work of mine, so long as you give me proper credit

Read selected portions of the NFPA 70E document ``Standard for Electrical Safety in the Workplace'' and answer the following questions:

\vskip 10pt

Annex C (``Limits of Approach'') and Article 100 (``Definitions'') both define some key terms used to describe how close one can be to an electrical hazard.  Read the definitions given for the following terms and then define them using your own words:

\begin{itemize}
\item{} Arc flash boundary
\item{} Limited approach boundary 
\item{} Restricted approach boundary 
\item{} Prohibited approach boundary
\end{itemize}

\vskip 10pt

Article 130 provides tables specifying limited approach boundary, restricted approach boundary, and prohibited boundary distances for a variety of circuit voltages and types.  Identify the boundary distances for each of these circuits, assuming all conductors within are non-moving:

\begin{itemize}
\item{} 480 VAC motor control circuit (``bucket'') 
\item{} 110 VAC control power wiring for a PLC
\item{} 125 VDC uninterruptible ``station power'' for a substation facility
\item{} 4160 VAC motor control circuit
\end{itemize}

\vskip 20pt \vbox{\hrule \hbox{\strut \vrule{} {\bf Suggestions for Socratic discussion} \vrule} \hrule}

\begin{itemize}
\item{} One of the informational notes in the 2012 edition of NFPA 70E, Article 130.4(B), reads {\it ``In certain circumstances the arc flash boundary might be a greater distance from the energized electrical conductors or circuit parts than the limited approach boundary.  The shock protection boundaries and the arc flash boundary are independent of each other.''}  Explain why this is, linking your answer to fundamental principles of electric circuits.
\end{itemize}

\underbar{file i03014}
%(END_QUESTION)





%(BEGIN_ANSWER)

 
%(END_ANSWER)





%(BEGIN_NOTES)

\begin{itemize}
\item{} {\bf Arc flash boundary} -- the distance at which a person could suffer a second-degree burn from arc flash (typically defined as an incident energy level of 1.2 calories per square centimeter or 5 joules per square centimeter.  No one is to cross this boundary without appropriate arc-rated PPE.
\item{} {\bf Limited approach boundary} -- the distance at which an exposed conductor presents an electrical shock hazard.  One must be ``qualified'' in order to cross this boundary and enter {\it limited space}, or be continuously escorted by a qualified person.
\item{} {\bf Restricted approach boundary} -- the distance at which a worker is at elevated risk of electrical shock from electrical arc-over and accompanying inadvertent movement.  Unqualified persons should never cross this boundary.
\item{} {\bf Prohibited approach boundary} -- the distance considered tantamount to making direct contact with the energized conductor.
\end{itemize}

\vskip 10pt

\filbreak

\begin{itemize}
\item{} {\bf 480 VAC motor control circuit (``bucket'')}:
\itemitem{} {\it Limited approach distance} = 1 meter (3' 6")
\itemitem{} {\it Restricted approach distance} = 0.3 meter (1' 0")
\itemitem{} {\it Prohibited approach distance} = 25 mm (1")
\vskip 2pt
\item{} {\bf 110 VAC control power wiring for a PLC}:
\itemitem{} {\it Limited approach distance} = 1 meter (3' 6")
\itemitem{} {\it Restricted approach distance} = ``avoid contact''
\itemitem{} {\it Prohibited approach distance} = ``avoid contact'' 
\vskip 2pt
\item{} {\bf 125 VDC uninterruptible ``station power'' for a substation facility}:
\itemitem{} {\it Limited approach distance} = 1 meter (3' 6")
\itemitem{} {\it Restricted approach distance} = ``avoid contact''
\itemitem{} {\it Prohibited approach distance} = ``avoid contact'' 
\vskip 2pt
\item{} {\bf 4160 VAC motor control circuit}: 
\itemitem{} {\it Limited approach distance} = 1.5 meters (5' 0")
\itemitem{} {\it Restricted approach distance} = 0.7 meter (2' 2")
\itemitem{} {\it Prohibited approach distance} = 0.2 meter (0' 7")
\end{itemize}

\vskip 10pt

The arc flash boundary for any circuit is a function of its maximum {\it power-delivering} capability to an open-air arc.  The boundary dimensions given in the tables are based on avoidance of electrical shock, and are a simpler function of voltage alone (it's assumed there is more than enough current capacity in any power system to deliver a harmful shock given the requisite voltage, and so current does not factor into the NFPA 70E's shock hazard analysis).  A low-voltage, high-current power system is a good example where the limited, restricted, and prohibited approach boundaries may be far smaller than the arc flash boundary.













\vskip 20pt \vbox{\hrule \hbox{\strut \vrule{} {\bf Suggestions for Socratic discussion} \vrule} \hrule}

\begin{itemize}
\item{} Which boundaries refer to electric shock hazard and which refer to arc flash hazard?
\item{} Note that ``Arc Flash Boundary'' is not listed under the definitions in Article 100 of NFPA 70E.  Instead, it is listed under ``Boundary, Arc Flash''.  What does this suggest about reading strategies for technical literature?
\item{} What special legal responsibilities do you think might be associated with being deemed a ``qualified person'' on an industrial job site?
\item{} {\tt C.1.2.2} states that only qualified people are to cross the limited approach boundary and enter limited space, while {\tt C.1.1} states unqualified people may do so when continuously escorted by a qualified person.  Is this a contradiction?  Why or why not?
\end{itemize}

%INDEX% Safety, electrical: arc flash
%INDEX% Safety, electrical: shock
%INDEX% Safety, electrical: NFPA 70E Standard for Electrical Safety in the Workplace

%(END_NOTES)


