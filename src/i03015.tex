
%(BEGIN_QUESTION)
% Copyright 2014, Tony R. Kuphaldt, released under the Creative Commons Attribution License (v 1.0)
% This means you may do almost anything with this work of mine, so long as you give me proper credit

Read selected portions of the NFPA 70E document ``Standard for Electrical Safety in the Workplace'' and answer the following questions:

\vskip 10pt

Article 120 of this document (``Establishing an Electrically Safe Work Condition'') outlines the requirements for two different types of lockout/tagout procedures: a {\it simple} and a {\it complex} procedure.  Annex G provides a sample procedure for an industrial workplace meeting the requirements of the ``simple'' procedure type.  Read these sections and then write your own outline of a simple lockout/tagout procedure.

\vskip 10pt

What criteria make a lockout/tagout procedure ``complex'' rather than ``simple''?

\vskip 20pt \vbox{\hrule \hbox{\strut \vrule{} {\bf Suggestions for Socratic discussion} \vrule} \hrule}

\begin{itemize}
\item{} As with NFPA 70 (the {\it National Electrical Code}), NFPA 70E does not intrinsically possess the force of law.  That is to say, NFPA has no legal power of its own to declare or to enforce professional standards.  However, the Authority Having Jurisdiction (AHJ) in the area where the work is being done has the ability to adopt any version of the NFPA's standards as enforceable law.  Generally, local government agencies specify which version of the NFPA document(s) they adopt as law, and enjoy a certain freedom of interpretation of those standards.  Why is this important for you as a technical worker to know?
\item{} An important safety policy at many industrial facilities is something called {\it stop-work authority}, which means any employee has the right to stop work they question as unsafe.  Describe a scenario involving electrical power where one might invoke stop-work authority.
\end{itemize}

\underbar{file i03015}
%(END_QUESTION)





%(BEGIN_ANSWER)

 
%(END_ANSWER)





%(BEGIN_NOTES)

A ``simple'' lockout/tagout procedure according to {\tt 120.2(D)(1)} is the sole responsibility of the lone (qualified) worker.  No written plans are necessary in such cases.

\vskip 10pt

\noindent
Based on Annex G:

\begin{itemize}
\item{} A qualified person disconnects all electrical sources and discharges stored energy.
\item{} Lock out all disconnects and tag out.  If tagging only, one additional safety measure must be employed (e.g. opening another circuit element).
\item{} Test the disconnect to see that it cannot be closed.
\item{} Inspect voltmeter and verify proper operation of voltmeter on a known source.
\item{} Test for absence of voltage in the locked-out circuit.
\item{} Once again verify proper operation of voltmeter on a known source. 
\item{} If required, connect grounding conductors to ensure no hazardous potential may arise.
\end{itemize}

\vskip 10pt

A ``complex'' lockout/tagout procedure involves a multiplicity of hazards, including multiple sources, multiple personnel working on the system, specific sequences of safe operation, and/or long durations of work.  A group lockout device must be used when there are multiple people in harm's way ({\tt 120.2(D)(2)}). 









\vskip 20pt \vbox{\hrule \hbox{\strut \vrule{} {\bf Suggestions for Socratic discussion} \vrule} \hrule}

\begin{itemize}
\item{} Washington State interprets any lockout procedure involving more than one worker to be a ``complex'' procedure and therefore subject to a written plan.  Do you see this as a conservative (i.e. cautious) interpretation of NFPA 70E or does it fit well within the plain reading of the standard?  {\it Note: {\tt 120.2.D.2} states ``A complex lockout/tagout plan shall be permitted . . .'' rather than saying it ``shall be required''.}
\end{itemize}

%INDEX% Safety, electrical: lock-out / tag-out
%INDEX% Safety, electrical: NFPA 70E Standard for Electrical Safety in the Workplace

%(END_NOTES)


