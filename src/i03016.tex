
%(BEGIN_QUESTION)
% Copyright 2015, Tony R. Kuphaldt, released under the Creative Commons Attribution License (v 1.0)
% This means you may do almost anything with this work of mine, so long as you give me proper credit

Read selected portions of the NFPA 70E document ``Standard for Electrical Safety in the Workplace'' and answer the following questions:

\vskip 10pt

Article 130 specifies Hazard/Risk categories as well as rubber glove and insulated tool requirements for work done on different classifications of electrical circuits.  Identify these ratings for a 480 volt motor control system, where a technician wishes to remove the front panel and take some voltage measurements on the live system.  Then, identify what the particular ``Hazard/Risk'' number means in terms of Personal Protective Equipment (PPE).

\vskip 30pt

For those looking for a challenge, perform an arc flash boundary analysis for an open 208 VAC motor starter circuit, such as the type we build in the lab during this course.  Assume the following:

\begin{itemize}
\item{} 3 kVA transformer bank feeding the motor starter
\item{} Transformers have 5\% impedance rating
\item{} Fault-clearing time is $1 \over 2$ cycle at 60 Hz with fuses on the 208 VAC lines
\end{itemize}

\underbar{file i03016}
%(END_QUESTION)





%(BEGIN_ANSWER)


The bolted-fault MVA rating of the 3 kVA transformer bank feeding a 208 volt motor starter will be its base MVA rating (0.003 MVA) divided by its impedance (5\%, or 0.05 per-unit):

$$\hbox{MVA}_{bf} = {\hbox{ MVA}_{base} \over Z}$$

$$\hbox{MVA}_{bf} = {0.003 \over 0.05} = 0.060$$

Arc flash boundary distance may be calculated one of two different ways:

$$D_c = \sqrt{2.65 \times \hbox{MVA}_{bf} \times t}$$
 
$$D_c = \sqrt{53 \times 1.25 \times \hbox{MVA}_{base} \times t}$$

The time to clear a fault was specified as half of one cycle, which is 8.333 milliseconds at 60 Hz.  Calculating both ways:
 
$$D_c = \sqrt{2.65 \times 0.060 \times 0.008333} = 0.036 \hbox{ ft} = 0.44 \hbox{ in}$$
 
$$D_c = \sqrt{53 \times 1.25 \times 0.003 \times 0.008333} = 0.041 \hbox{ ft} = 0.49 \hbox{ in}$$

%(END_ANSWER)





%(BEGIN_NOTES)

Taking live voltage measurements on a 480 VAC system with fault current capacity 25 kA and 2-cycle fault clearing yields an arc flash boundary of 30 inches.  This carries a Hazard/Risk rating of ``2'' and requires both the use of rubber insulating gloves and insulated hand tools for protection against electric shock.

\vskip 10pt

A Hazard/Risk Category (HRC) value of 2 represents the PPE necessary to protect against arc flash, in this case necessitating arc-rated clothing with a minimum arc rating of 8 calories per square centimeter covering all skin (including a balaclava and face shield), hard hat, safety glasses, earplugs, heavy-duty leather gloves, and leather work shoes.  Table {\tt 130.7(C)(16)} is where this information is found.










\vskip 20pt \vbox{\hrule \hbox{\strut \vrule{} {\bf Suggestions for Socratic discussion} \vrule} \hrule}

\begin{itemize}
\item{} Do the Hazard/Risk categories shown in Article 130 refer to electric shock hazard, arc flash hazard, or both?  {\it Note: informational note \#2 in {\tt 130.7(C)(16)} gives a direct answer to this question.}
\item{} Do the Hazard/Risk categories shown in Article 130 address arc blast hazard?
\end{itemize}

%INDEX% Safety, electrical: lock-out / tag-out
%INDEX% Safety, electrical: NFPA 70E Standard for Electrical Safety in the Workplace

%(END_NOTES)


