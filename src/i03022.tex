
%(BEGIN_QUESTION)
% Copyright 2015, Tony R. Kuphaldt, released under the Creative Commons Attribution License (v 1.0)
% This means you may do almost anything with this work of mine, so long as you give me proper credit

Read and outline the ``Electrical Power Grids'' section of the ``Electric Power Measurement and Control'' chapter in your {\it Lessons In Industrial Instrumentation} textbook.  Note the page numbers where important illustrations, photographs, equations, tables, and other relevant details are found.  Prepare to thoughtfully discuss with your instructor and classmates the concepts and examples explored in this reading.

\vskip 10pt

Alternatively, you may read the ``Introduction to Power System Automation'' section found in the same chapter of the {\it Lessons In Industrial Instrumentation} textbook.  This section gives a more comprehensive overview of electrical power grids and associated instrumentation, as well as the {\it single-line diagrams} used to document power grids.

\underbar{file i03022}
%(END_QUESTION)




%(BEGIN_ANSWER)


%(END_ANSWER)





%(BEGIN_NOTES)

A power ``grid'' is an interconnecting network of components and conductors between electrical power sources and electrical loads.  A typical grid contains the following things:

\begin{itemize}
\item{} Generating stations: {\it convert various forms of energy into electricity, run through step-up transformers to increase voltage and decrease current}
\item{} Transmission lines: {\it convey electrical power at high voltage from the generating stations to locations far away}
\item{} Substations: {\it connect transmission lines together from various generating stations and step voltage down for distribution to loads}
\item{} Distribution lines: {\it convey electrical power at lower voltage (and higher current) to points of use}
\item{} Customer transformers: {\it step voltage down (and current up) for end-use at each customer site}
\end{itemize}

Alternating current (AC) is what made long-distance power grids practical, because transformers (which only work with AC) can easily step voltage up and current down to permit the use of economical transmission and distribution lines.  DC could not be stepped up or down so easily, which meant that early DC grids operated at one voltage level with heavy distribution conductors and significant voltage losses from generating stations to loads.


\filbreak

\vskip 20pt \vbox{\hrule \hbox{\strut \vrule{} {\bf Suggestions for Socratic discussion} \vrule} \hrule}

\begin{itemize}
\item{} Define the following terms:
\itemitem{} Transmission line
\itemitem{} Distribution line
\itemitem{} Substation
\item{} Explain why AC was chosen over DC for the long-distance transmission electrical power.
\end{itemize}









%INDEX% Reading assignment: Lessons In Industrial Instrumentation, electrical power grids

%(END_NOTES)


