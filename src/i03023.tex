
%(BEGIN_QUESTION)
% Copyright 2014, Tony R. Kuphaldt, released under the Creative Commons Attribution License (v 1.0)
% This means you may do almost anything with this work of mine, so long as you give me proper credit

Read and outline the ``Single-line Electrical Diagrams'' section of the ``Electric Power Measurement and Control'' chapter in your {\it Lessons In Industrial Instrumentation} textbook.  Note the page numbers where important illustrations, photographs, equations, tables, and other relevant details are found.  Prepare to thoughtfully discuss with your instructor and classmates the concepts and examples explored in this reading.

\underbar{file i03023}
%(END_QUESTION)




%(BEGIN_ANSWER)


%(END_ANSWER)





%(BEGIN_NOTES)

Power grid conductors are often represented in {\it single line diagram} form for simplicity: showing each set of three-phase conductors as a single line on the diagram.  This way, needless complexity is eliminated for those tasked with quickly interpreting a power system diagram to determine the route power takes from source(s) to load(s).

\vskip 10pt

When shown on computer-based SCADA system displays, the square symbols used to represent circuit breakers are often colored to reveal their status: green represents a circuit breaker that is off (tripped) while red represents a circuit breaker that is on (closed).



\filbreak

\vskip 20pt \vbox{\hrule \hbox{\strut \vrule{} {\bf Suggestions for Socratic discussion} \vrule} \hrule}

\begin{itemize}
\item{} Identify some of the details condensed in the transition from schematic diagrams to single-line diagrams.
\item{} Examine the large single-line diagram shown in this section and identify how power gets from one particular generating station to one particular load.
\item{} Examine the large single-line diagram shown in this section and identify how power may be re-routed from sources to loads if a particular breaker is tripped.
\item{} Examine the large single-line diagram shown in this section and identify how power may be isolated from a section of the power grid in the event of a fault (e.g. a transmission line that has fallen to the ground).
\item{} Examine the large single-line diagram shown in this section and compare the topologies of any two substations, contrasting their designs (e.g. number of circuit breakers, ways to alternatively route power to any particular load).   
\end{itemize}











%INDEX% Reading assignment: Lessons In Industrial Instrumentation, single-line electrical diagrams

%(END_NOTES)


