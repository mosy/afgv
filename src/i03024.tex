
%(BEGIN_QUESTION)
% Copyright 2015, Tony R. Kuphaldt, released under the Creative Commons Attribution License (v 1.0)
% This means you may do almost anything with this work of mine, so long as you give me proper credit

Read selected portions of ``Informative Annex D -- Incident Energy and Arc Flash Boundary Calculation Methods'' in the NFPA 70E document ``Standard for Electrical Safety in the Workplace'' and answer the following questions:

\vskip 10pt

Section D.2 mentions the amount of {\it clearing time} for different overcurrent protective devices.  List those device types and comment on their relative operating times.  What kind of electrical fault do you think is being referred to in this Annex, {\it opens} or {\it shorts}?

\vskip 10pt

The {\it impedance} of a power transformer or power system is mentioned throughout this section of the NFPA document as a limiting factor in fault current and arc flash energy.  Identify what causes impedance in a power system, and whether a high percentage or a low percentage is more limiting to fault current.

\vskip 10pt

Table D.7.7 lists different types of circuit breakers and their ``Trip Unit Types''.  Identify some of these different trip unit types and how they work.  Which of these trip unit types provides the smallest (i.e. safest) arc flash boundary?  Explain why this is so.

\vskip 20pt \vbox{\hrule \hbox{\strut \vrule{} {\bf Suggestions for Socratic discussion} \vrule} \hrule}

\begin{itemize}
\item{} Explain exactly why clearing time is a relevant parameter for calculating arc flash energy and arc flash boundary size.
\item{} An important safety policy at many industrial facilities is something called {\it stop-work authority}, which means any employee has the right to stop work they question as unsafe.  Describe a scenario involving either arc flash or arc blast potential where one might invoke stop-work authority.
\item{} When a substation operator closes circuit breakers to energize a bus, they typically begin with the source having the greatest impedance.  Explain why this is a wise choice.
\end{itemize}

\underbar{file i03024}
%(END_QUESTION)





%(BEGIN_ANSWER)

 
%(END_ANSWER)





%(BEGIN_NOTES)

{\bf Fuses} are mentioned as clearing a fault in approximately $1 \over 4$ cycle (0.004 seconds).  A medium-voltage {\bf circuit breaker} is mentioned as having a clearing time of approximately 6 cycles (0.1 second), itemized as 2 cycles of breaker operating time, 1.74 cycles of relay operating time, and 2 cycles of margin.  Circuit breakers having a ``time delay function'' may take longer to trip.

\vskip 10pt

Power system impedance may come from a number of different sources, including generator winding impedance, power line resistance and inductance, transformer leakage inductance, transformer winding resistance, line reactor impedance, and circuit breaker contact resistance to name a few.  Judging by the formulae in Annex D, a larger impedance ($Z$) has a more limiting effect on short-circuit fault current ($I_{sc}$).

\vskip 10pt

Trip unit types listed in this table include:

\begin{itemize}
\item{} {\bf TM}: Thermal-magnetic
\item{} {\bf M}: Magnetic (instantaneous)
\item{} {\bf E}: Electronic (may have different time characteristics used either separately or in combination)
\begin{itemize}

\item{} {\bf S} = Short time 
\item{} {\bf I} = Instantaneous
\end{itemize}
\end{itemize}

The smallest arc flash boundary is exhibited in systems with circuit breakers having magnetic trip units, which operate instantaneously.  The largest arc flash boundary is exhibited with circuit breakers lacking an ``instantaneous'' trip characteristic.

%INDEX% Electric power systems: protective relays (instantaneous overcurrent)
%INDEX% Electric power systems: protective relays (time-overcurrent)
%INDEX% Safety, electrical: NFPA 70E Standard for Electrical Safety in the Workplace

%(END_NOTES)


