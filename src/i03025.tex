
%(BEGIN_QUESTION)
% Copyright 2015, Tony R. Kuphaldt, released under the Creative Commons Attribution License (v 1.0)
% This means you may do almost anything with this work of mine, so long as you give me proper credit

Read and outline the ``Transformer Polarity'' subsection of the ``Electrical Sensors'' section of the ``Electric Power Measurement and Control'' chapter in your {\it Lessons In Industrial Instrumentation} textbook.  Note the page numbers where important illustrations, photographs, equations, tables, and other relevant details are found.  Prepare to thoughtfully discuss with your instructor and classmates the concepts and examples explored in this reading.

\underbar{file i03025}
%(END_QUESTION)




%(BEGIN_ANSWER)


%(END_ANSWER)





%(BEGIN_NOTES)

The phase relationships between windings on a transformer are designated by the use of ``polarity'' markings, often in the form of dots, squares, ``X'' marks, or common numbering (e.g. H1 versus X1).  The notion is that the instantaneous voltage polarity at any point in time will be common between similarly-marked terminals.  For example, at a point in time where the ``dot'' terminal of the transformer's primary winding is positive with respect to the ``non-dot'' terminal of the same winding, the ``dot'' terminal of that transformer's secondary winding will also be positive with respect to the ``non-dot'' secondary terminal.

\vskip 10pt

Polarity marks on transformer windings always refer to {\it voltage}, not {\it current}.  Since the primary winding of a transformer acts as a load (drawing power from an external power source) while the secondary winding of a transformer acts as a source (supplying power to an external load), current will always be flowing in opposite directions with respect to the polarity dot (e.g. current in at the primary dot means current out at the secondary dot).











\vskip 20pt \vbox{\hrule \hbox{\strut \vrule{} {\bf Suggestions for Socratic discussion} \vrule} \hrule}

\begin{itemize}
\item{} Explain in your own words what the dots are supposed to mean on transformer windings.
\item{} Identify alternative symbols to dots for transformer polarity markings.
\item{} Sketch a transformer circuit that introduces a 180$^{o}$ phase shift from input to output.
\item{} If we are using a CT to drive an ammeter, does polarity matter?  Explain why or why not.
\item{} Note how the bushing CTs shown in the power transformer diagram have multiple ``taps'' on their secondary windings.  What do you suppose those are for?
\end{itemize}


%INDEX% Reading assignment: Lessons In Industrial Instrumentation, transformer polarity

%(END_NOTES)


