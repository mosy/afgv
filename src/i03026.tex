
%(BEGIN_QUESTION)
% Copyright 2014, Tony R. Kuphaldt, released under the Creative Commons Attribution License (v 1.0)
% This means you may do almost anything with this work of mine, so long as you give me proper credit

Read and outline the ``Instrument Transformer Safety'' subsection of the ``Electrical Sensors'' section of the ``Electric Power Measurement and Control'' chapter in your {\it Lessons In Industrial Instrumentation} textbook.  Note the page numbers where important illustrations, photographs, equations, tables, and other relevant details are found.  Prepare to thoughtfully discuss with your instructor and classmates the concepts and examples explored in this reading.

\underbar{file i03026}
%(END_QUESTION)




%(BEGIN_ANSWER)


%(END_ANSWER)





%(BEGIN_NOTES)

Potential transformers (PTs) act as voltage sources driving high-impedance loads (voltmeters).  As such, PTs are safe to open-circuit.  Current transformers (CTs) act as current sources driving low-impedance loads (ammeters).  As such, CTs are safe to short-circuit.

\vskip 10pt

In fact, a CT {\it must} be short-circuited if disconnected from its load, in order to avoid developing an extremely high voltage at its secondary terminals!

\vskip 10pt

The secondary winding of any instrument transformer (PT or CT) must be solidly grounded in order to avoid elevated potentials from developing on the instrument side of the circuit.  The ideal location for the ground connection is the first point of use (where the transformer's secondary leads terminate at a test switch or terminal block).



\vskip 20pt \vbox{\hrule \hbox{\strut \vrule{} {\bf Suggestions for Socratic discussion} \vrule} \hrule}

\begin{itemize}
\item{} Why bother grounding the secondary winding of a PT or a CT, if the stepped-down quantity is relatively small (e.g. 120 volts or less ; 5 amps or less)?
\item{} Suppose you were using a standard digital multimeter to measure CT secondary current, and the meter's fuse suddenly opened?
\item{} Explain what a {\it ground loop} is, and why they should be avoided in instrument transformer wiring.
\item{} Explain how you could use simple test equipment (i.e. nothing more complicated than a multimeter) to check for the presence of a ground loop in an instrument signal cable.
\end{itemize}

%INDEX% Reading assignment: Lessons In Industrial Instrumentation, instrument transformer safety

%(END_NOTES)


