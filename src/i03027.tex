
%(BEGIN_QUESTION)
% Copyright 2015, Tony R. Kuphaldt, released under the Creative Commons Attribution License (v 1.0)
% This means you may do almost anything with this work of mine, so long as you give me proper credit

Read and outline the ``Circles, Sine Waves, and Cosine Waves'' subsection of the ``Phasors'' section of the ``AC Electricity'' chapter in your {\it Lessons In Industrial Instrumentation} textbook.  Note the page numbers where important illustrations, photographs, equations, tables, and other relevant details are found.  Prepare to thoughtfully discuss with your instructor and classmates the concepts and examples explored in this reading.

\underbar{file i03027}
%(END_QUESTION)




%(BEGIN_ANSWER)


%(END_ANSWER)





%(BEGIN_NOTES)

A sine wave is the vertical projection of a rotating vector (where horizontal pointing right is defined as an angle of zero).  A cosine wave is the horizontal projection of a rotating vector.  Euler related the phasor's angle to those vertical and horizontal projections through the use of imaginary numbers ($i = j = \sqrt{-1}$) in what is called Euler's Relation:

$$e^{j \theta} = \cos \theta + j \sin \theta$$

\noindent
Where,

$e$ = Euler's number (approximately equal to 2.718281828)

$\theta$ = Angle of vector, in radians

$\cos \theta$ = Horizontal projection of a unit vector (along a real number line) at angle $\theta$

$j$ = Imaginary ``operator'' equal to $\sqrt{-1}$, represented by $i$ or $j$

$j \sin \theta$ = Vertical projection of a unit vector (along an imaginary number line) at angle $\theta$

\vskip 10pt

For vectors having a length of $A$, the coefficient $A$ defines the length of the vector while $e^{j \theta}$ defines the direction it points:

$$Ae^{j \theta} = A \cos \theta + j A \sin \theta$$

\vskip 10pt

In a two-pole generator, the one stator winding voltage is represented by $A \cos \theta$ while $j A \sin \theta$ represents the voltage of an imaginary stator winding placed 90$^{o}$ out of step with the real stator winding.

\vskip 10pt

In electrical studies these special vectors are called {\it phasors}.  Expressing a phasor as a magnitude (length) and angle is known as {\it polar notation} while expressing a phasor as its horizontal and vertical projections is known as {\it rectangular notation}.

We may express phasors as time-based functions by replacing $\theta$ with $\omega t$, where $\omega$ is the angular velocity and $t$ is time.  The function may be visualized as a corkscrew trajectory where time ($t$) is the centerline axis, the horizontal axis is real ($A \cos \omega t$) and the vertical axis is imaginary ($j A \sin \omega t$):

$$Ae^{j \omega t} = A \cos \omega t + j A \sin \omega t$$




\vskip 20pt \vbox{\hrule \hbox{\strut \vrule{} {\bf Suggestions for Socratic discussion} \vrule} \hrule}

\begin{itemize}
\item{} Demonstrate how to plot a sine wave using the technique explained in the opening pages of this section.
\item{} Demonstrate how to plot a cosine wave using the technique explained in the opening pages of this section.
\item{} Explain how the expression $Ae^{j \omega t} = A \cos \omega t + j A \sin \omega t$ may be derived from the simpler expression $e^{j \theta} = \cos \theta + j \sin \theta$.  What does $A$ represent, exactly?  What does $\omega$ represent, exactly?  What does $t$ represent, exactly?
\item{} A favorite expression of math teachers is $e^{i \pi} = -1$.  Show how this is true using Euler's Relation, and sketch this on a phasor diagram.
\item{} Determine the value of $\theta$ that will yield a phasor pointing straight up, and prove this using Euler's Relation.
\item{} Determine the value of $\theta$ that will yield a phasor pointing straight down, and prove this using Euler's Relation.
\item{} View the textbook's flip-book animation of a phasor graphed over time, and explain how both a circular shape (complex domain) and a sinusoidal shape (time domain) emerge from this three-dimensional trace.
\item{} The dominant power system frequency in the United States is 60 Hz.  Convert 60 Hz into units of {\it radians per second}, which is the standard unit of measurement for angular velocity ($\omega$).
\item{} Explain why it is impossible in most cases to determine the direction a phasor is pointing by examining only the value $A \cos \theta$ or only $j A \sin \theta$.  In other words, explain why {\it both} the real and imaginary quantities are necessary to completely define a phasor's angle.
\end{itemize}

%INDEX% Reading assignment: Lessons In Industrial Instrumentation, circles, sine waves, and cosine waves

%(END_NOTES)


