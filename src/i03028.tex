
%(BEGIN_QUESTION)
% Copyright 2015, Tony R. Kuphaldt, released under the Creative Commons Attribution License (v 1.0)
% This means you may do almost anything with this work of mine, so long as you give me proper credit

Read and outline the ``Phasor Expressions of Phase Shifts'' subsection of the ``Phasors'' section of the ``AC Electricity'' chapter in your {\it Lessons In Industrial Instrumentation} textbook.  Note the page numbers where important illustrations, photographs, equations, tables, and other relevant details are found.  Prepare to thoughtfully discuss with your instructor and classmates the concepts and examples explored in this reading.

\underbar{file i03028}
%(END_QUESTION)




%(BEGIN_ANSWER)


%(END_ANSWER)





%(BEGIN_NOTES)

Euler's Relation states an eqivalence between an imaginary exponential function and trigonometric functions:

$$Ae^{j \theta} = A \cos \theta + A j \sin \theta$$

\noindent
Where,

$A$ = Peak amplitude of waveform

$e$ = Euler's number (approximately equal to 2.718281828)

$\theta$ = Angle of vector, in radians

$A \cos \theta$ = Horizontal projection of a unit vector (along a real number line) at angle $\theta$

$j$ = Imaginary ``operator'' equal to $\sqrt{-1}$, represented by $i$ or $j$

$A j \sin \theta$ = Vertical projection of a unit vector (along an imaginary number line) at angle $\theta$

\vskip 10pt

Two sinusoids at the same frequency are phase-shifted from each other, it is equivalent to two rotating phasors with an angular displacement between each other.  As phasors are assumed to rotate counter-clockwise, the phasor more CCW than the other is the one that ``leads''.  Differences (subtraction) between two phasors is the geometric distance between those phasors' tips.  Phasors may be relocated to different places on the phasor diagram so long as their angles are preserved.

When expressing AC voltages in phasor form, it is very important to specify which lead of the test instrument (oscilloscope, phasometer) is which point in the circuit.  $V_{AB}$

\vskip 10pt

In reality, phasors are in constant motion, rotating at the same speed as the AC frequency.  As such, their absolute angles are constantly changing ($\theta = \omega t$).  Expressing phasors with fixed angles only makes sense when we are comparing two phasors of the same frequency with a fixed phase shift between them.  That is to say, fixed phase angles only make sense when the angle represents a {\it comparison of phase} between two sinusoids.

\vskip 10pt

A contrived instrument useful to explain and explore phasors is the so-called {\it phasometer}, which is just a synchronous AC motor with a pointer attached to its shaft, calibrated so that the pointer shows 0$^{o}$ when its red lead is positive and its black lead is negative.  If the phasometer is constructed such that the pointer always rotates counter-clockwise, the pointer's position at any point in time will match the angle of the phasor at that same point in time -- it is a mechanical analogue of an electrical phasor.  Spinning at 3600 RPM (60 Hz), the phasometer's needle will be nothing but a blur.  If viewed under a {\it strobe light} synchronized to pulse once every cycle, however, the phasometer needle will appear to be ``frozen'' at one position representing that waveform's angular position at the time of the strobe's pulse.



\vskip 20pt \vbox{\hrule \hbox{\strut \vrule{} {\bf Suggestions for Socratic discussion} \vrule} \hrule}

\begin{itemize}
\item{} Suppose two sinusoidal waveforms are out of phase with each other.  Which one is {\it leading} -- the one shifted to the left or the one shifted to the right -- on the oscilloscope display?  Explain why the waves should be interpreted as such.
\item{} Suppose two sinusoidal waveforms are out of phase with each other.  Which one is {\it lagging} -- the one shifted to the left or the one shifted to the right -- on the oscilloscope display?  Explain why the waves should be interpreted as such.
\item{} Suppose two sinusoidal waveforms are out of phase with each other.  Which one is {\it leading} -- the one whose phasor is shifted clockwise or the one whose phasor is shifted counter-clockwise -- on a phasor diagram?  Explain why the phasors should be interpreted as such.
\item{} Suppose two sinusoidal waveforms are out of phase with each other.  Which one is {\it lagging} -- the one whose phasor is shifted clockwise or the one whose phasor is shifted counter-clockwise -- on a phasor diagram?  Explain why the phasors should be interpreted as such.
\item{} In the textbook example where two 5-volt AC voltage sources are compared with a voltmeter, does it matter to the meter's displayed voltage which lead of the voltmeter connects to which source?  Why or why not?
\item{} In the textbook example where two 5-volt AC voltage sources are compared with a voltmeter, does it matter to the measured voltage phasor's direction which lead of the voltmeter connects to which source?  Why or why not?
\item{} Explain how the {\it phasometer} introduced in the textbook is supposed to function, and how this instrument helps make sense of phasors.  In particular, what purpose does the strobe light serve?
\item{} Examining the two textbook illustrations showing three phasometers registering the phase angles of three different voltages, why do all the phasometer angles appear to change when the strobe light is connected to a different source?  What does this tell us about phasor measurements in general?
\item{} Explain what {\it syncrophasors} are and how they relate to phasor measurements of voltage and current in power systems.
\item{} Describe how {\it syncrophasors} could be used to detect ``islanding'' conditions in a power grid.
\item{} Describe how phasors in an AC power system would appear from the perspective of a {\it syncrophasor} measurement system, if the line frequency drifted away from 60 Hz.
\end{itemize}

%INDEX% Reading assignment: Lessons In Industrial Instrumentation, phasor expressions of phase shifts

%(END_NOTES)


