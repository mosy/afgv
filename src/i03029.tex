
%(BEGIN_QUESTION)
% Copyright 2015, Tony R. Kuphaldt, released under the Creative Commons Attribution License (v 1.0)
% This means you may do almost anything with this work of mine, so long as you give me proper credit

Read and outline the ``Phasors and Circuit Measurements'' subsection of the ``Phasors'' section of the ``AC Electricity'' chapter in your {\it Lessons In Industrial Instrumentation} textbook.  Note the page numbers where important illustrations, photographs, equations, tables, and other relevant details are found.  Prepare to thoughtfully discuss with your instructor and classmates the concepts and examples explored in this reading.

\underbar{file i03029}
%(END_QUESTION)




%(BEGIN_ANSWER)


%(END_ANSWER)





%(BEGIN_NOTES)

A phasor with a fixed angle expresses the {\it relative} phase shift between two different phasor quantities.  In reality, all phasors are constantly rotating over time.  The imaginary instrument invented to show phasor quantities (the ``phasometer'') is only able to show a constant angle when illuminated by a strobe light connected so as to flash once per cycle.

\vskip 10pt

Phasor angles are to AC quantities as mathematical signs are to DC quantities.  As such, they depend as much on how the measuring instrument connects to the circuit as they do to the inherent electrical quantities.

\vskip 10pt

When analyzing AC circuits, we may still draw arrows showing direction of current and draw voltage polarity symbols (+, $-$) showing orientation of voltages.  However, these symbols must be understood to represent the instantaneous orientation of those quantities {\it at their zero-degree points}.  In other words, those markings define current and direction at the point in time when a phasometer would point to 0$^{o}$ with the strobe light referenced to that point or component.

\vskip 10pt

It is perfectly permissible to have current arrows and voltage polarities pointed in what appears to be conflicting directions, so long as the phase angles of those quantities do not mathematically conflict.  For example, a Wye-connected three-phase load may be sketched with each phase current pointed toward the center node of the Wye so long as it is understood those currents have phase angles of 0$^{0}$, 120$^{0}$, and 240$^{o}$ respectively, and as such their phasor sum will be zero at that node in accordance with Kirchhoff's Current Law.  At any given instant in time the real directions of those currents never conflict: current entering that node must be balanced by current exiting that same node, according to KVL.





\vskip 20pt \vbox{\hrule \hbox{\strut \vrule{} {\bf Suggestions for Socratic discussion} \vrule} \hrule}

\begin{itemize}
\item{} If phasors in AC electrical systems are actually moving all the time (counter-clockwise), how can we rationally speak of them as having fixed angles?  Isn't this a contradiction?
\item{} How is it correct to label the currents in a Wye-connected three-phase load as all pointed toward the center node of the Wye?  Does this not violate Kirchhoff's Current Law (KCL) which tells us for every amp headed into a node there must be an amp of current {\it exiting} it too?
\end{itemize}










\vfil \eject

\noindent
{\bf Prep Quiz:}

What does it mean to say that a phasor has an angle of $180^o$?

\begin{itemize}
\item{} Its signal has a constant (DC) negative polarity at all points in time
\vskip 5pt 
\item{} Its waveform's sign is the same as some reference waveform at any moment in time
\vskip 5pt 
\item{} Its waveform's sign is opposite some reference waveform at any moment in time
\vskip 5pt 
\item{} Its signal has a constant (DC) positive polarity at all points in time
\vskip 5pt 
\item{} It is a little more than 30 degrees away from reaching the boiling point of water
\vskip 5pt 
\item{} It can do cool skateboard tricks
\end{itemize}










\vfil \eject

\noindent
{\bf Prep Quiz:}

What does it mean when we label components within an AC circuit with + and $-$ voltage polarity marks and current arrows as though it were a DC circuit?

\begin{itemize}
\item{} These symbols define the polarities and directions at each phasor's 0$^{o}$ point
\vskip 5pt 
\item{} That is how we absolutely must connect meter leads when measuring voltage and current
\vskip 5pt 
\item{} Those are simultaneous polarities and directions (i.e. at the same point in time)
\vskip 5pt 
\item{} The circuit is really DC and not AC at all like we first thought
\vskip 5pt 
\item{} The + and $-$ marks represent positive and negative phase angles 
\vskip 5pt 
\item{} We are really nostalgic about working with simple DC circuits
\end{itemize}


%INDEX% Reading assignment: Lessons In Industrial Instrumentation, phasors and circuit measurements

%(END_NOTES)


