
%(BEGIN_QUESTION)
% Copyright 2014, Tony R. Kuphaldt, released under the Creative Commons Attribution License (v 1.0)
% This means you may do almost anything with this work of mine, so long as you give me proper credit

Read and outline the ``Interconnected Generators'' section of the ``Electric Power Measurement and Control'' chapter in your {\it Lessons In Industrial Instrumentation} textbook.  Note the page numbers where important illustrations, photographs, equations, tables, and other relevant details are found.  Prepare to thoughtfully discuss with your instructor and classmates the concepts and examples explored in this reading.

\underbar{file i03030}
%(END_QUESTION)




%(BEGIN_ANSWER)


%(END_ANSWER)





%(BEGIN_NOTES)



\filbreak

\vskip 20pt \vbox{\hrule \hbox{\strut \vrule{} {\bf Suggestions for Socratic discussion} \vrule} \hrule}

\begin{itemize}
\item{} Describe the conditions which must be met before we parallel two operating AC generators.
\item{} Describe how a synchronization lamp will behave when one AC generator is running at full speed and the other generator is stopped.
\item{} Describe how a synchronization lamp will behave when both AC generators are running at the same speed and are in-phase with each other but their voltages are unequal.
\item{} Describe how a synchronization lamp will behave when both AC generators are running at the same speed and are 180$^{o}$ out of phase with each other.
\item{} Describe how a synchronization lamp will behave when both AC generators are running at slightly different speeds but outputting exactly the same amount of voltage.
\item{} Describe how a synchronization lamp will behave when both AC generators are running at slightly different speeds and are outputting different amounts of voltage.
\end{itemize}

%INDEX% Reading assignment: Lessons In Industrial Instrumentation, interconnected generators

%(END_NOTES)


