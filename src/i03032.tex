
%(BEGIN_QUESTION)
% Copyright 2015, Tony R. Kuphaldt, released under the Creative Commons Attribution License (v 1.0)
% This means you may do almost anything with this work of mine, so long as you give me proper credit

Read and outline the ``Auxiliary and Lockout (86) Relays'' section of the ``Electric Power Measurement and Control'' chapter in your {\it Lessons In Industrial Instrumentation} textbook.  Note the page numbers where important illustrations, photographs, equations, tables, and other relevant details are found.  Prepare to thoughtfully discuss with your instructor and classmates the concepts and examples explored in this reading.

\underbar{file i03032}
%(END_QUESTION)




%(BEGIN_ANSWER)


%(END_ANSWER)





%(BEGIN_NOTES)

An 86 (lockout, or auxiliary) relay is one connected between a protective relay and one or more circuit breakers, serving as an intermediary element in the protection system.  Lockout relays mechanically ``latch'' when tripped and therefore require human intervention to re-set.

The ``normal'' or resting state of an 86 relay is when it is in the Reset (operating) position.  The Tripped position is considered the ``actuated'' state.  This is important to bear in mind when interpreting switch symbols for an 86 relay.  86 relay trip coils typically have an auxiliary 86 contact (normally-closed) connected in series to break the trip circuit once the 86 relay has made it to the tripped position.  This auxiliary contact is necessary to dis-engage the seal-in coils of any protective relays driving the 86 relay's trip coil, and is analogous to the 52a contact in a circuit breaker's trip coil circuit.

\vskip 10pt

A lockout relay may be tripped by a single protective function, or by multiple functions.  The multiple contacts provided by an 86 relay allow that one relay to trip more than one circuit breaker and/or signal more than one monitoring system.  The latching nature of the 86 relay encourages a human operator to survey which protective function triggered the lockout, by forcing the operator to walk over to the 86 reset handle and turn it.







\filbreak

\vskip 20pt \vbox{\hrule \hbox{\strut \vrule{} {\bf Suggestions for Socratic discussion} \vrule} \hrule}

\begin{itemize}
\item{} How may we visually tell when an 86 relay needs to be manually reset?
\item{} Refer to the schematic diagram of the generator protection system shown in the LIII textbook and explain the purpose of the 86 relay in that system.
\item{} Refer to the schematic diagram of the generator protection system shown in the LIII textbook and explain what would happen if the second NO 86 contact (connected to 52-N/TC) failed open in this system.
\item{} Locate a trip circuit diagram in one of the SEL protective relay manuals (e.g. the SEL-387A instruction manual, figure 2.8) showing an 86 relay and explain its purpose there.
\end{itemize}

%INDEX% Reading assignment: Lessons In Industrial Instrumentation, differential (87) protection

%(END_NOTES)


