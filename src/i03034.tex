
%(BEGIN_QUESTION)
% Copyright 2014, Tony R. Kuphaldt, released under the Creative Commons Attribution License (v 1.0)
% This means you may do almost anything with this work of mine, so long as you give me proper credit

Examine the single-line electrical diagram example contained in ``Informative Annex D -- Incident Energy and Arc Flash Boundary Calculation Methods'' of the NFPA 70E document ``Standard for Electrical Safety in the Workplace'' and answer the following questions:

\vskip 10pt

Identify how each of the following devices are represented in this diagram:

\begin{itemize}
\item{} Transformers
\item{} Disconnects
\item{} Circuit breakers
\item{} Fuses
\end{itemize}

\vskip 10pt

Several ``tie breakers'' are shown in this diagram, connecting bus segments together to form a larger bus.  Identify where these tie breakers are in the diagram, and explain their purpose in the power distribution system.

\vskip 10pt

Each transformer shown in this diagram bears a percentage rating.  What, exactly, does this percentage rating refer to?

\vskip 10pt

Suppose the 5 MVA transformer feeding bus 2B suffers a catastrophic internal short-circuit.  Identify the points at which power will be automatically disconnected from this failed transformer, either by fuses or by circuit breakers tripped by protective relays.  Is it possible to restore power to bus 2B while this transformer is still out of service?  If so, explain how this could be accomplished.

\vskip 10pt

Suppose the 2.5 MVA transformer feeding bus 7A suffers a catastrophic internal short-circuit.  Identify the points at which power will be automatically disconnected from this failed transformer, either by fuses or by circuit breakers tripped by protective relays.  Is it possible to restore power to bus 7A while this transformer is still out of service?  If so, explain how this could be accomplished.

\vskip 20pt \vbox{\hrule \hbox{\strut \vrule{} {\bf Suggestions for Socratic discussion} \vrule} \hrule}

\begin{itemize}
\item{} Should tie breakers and tie switches typically be left in their open or closed states?  Explain your reasoning.
\item{} Calculate the maximum bolted-fault current for a three-phase transformer rated at 480 volt output and 1.5 MVA, with 6\% impedance.
\item{} Note the tie breaker arrangement between busses 1A and 1B: there is a direct tie breaker in parallel with an inductor having its own pair of breakers.  When initially connecting these two busses together, the protocol is to close the direct tie breaker {\it last}, relying on the other two breakers and the associated inductor to make the first connection between the busses.  Explain why this is so.
\end{itemize}

\underbar{file i03034}
%(END_QUESTION)





%(BEGIN_ANSWER)

 
%(END_ANSWER)





%(BEGIN_NOTES)

\begin{itemize}
\item{} Transformers: {\it parallel zig-zag or hooped lines}
\item{} Disconnects: {\it knife-switch symbols}
\item{} Circuit breakers: {\it square boxes}
\item{} Fuses: {\it rectangles with a dark band at each end}
\end{itemize}

\vskip 10pt

``Tie breakers'' connect the A and B segments of each bus together.  They allow two busses to be powered by a single transformer.

\vskip 10pt

Each percentage value is an expression of transformer {\it impedance}, in terms of rated load.  For example, if a transformer has a full-load rating of 5 MVA (5 million volt-amps) and an impedance of 5.5\%, it means its own internal impedance is only 5.5\% that of the load's impedance drawing full power.  Conversely, it means if that transformer's secondary is short-circuited by a ``bolted'' fault (essentially 0 ohms for the short-circuit fault itself), the fault current will be $1 \over 5.5\%$ of the full-load current (i.e. the bolted-fault current will be 18.18 times the normal full-load current).

\vskip 10pt

If the transformer feeding bus 2B fails, it should be disconnected from its 13.8 kV source by the next breaker upstream (which also feeds the transformer powering bus 3B).  If the protection scheme is differential current (ANSI/IEEE type 87), then the secondary circuit breaker will be tripped by the protective relay as well.  Power may be restored to bus 2B by closing the tie breaker between busses 2A and 2B.

\vskip 10pt

If the transformer feeding bus 7A fails, it should be disconnected from its 13.8 kV source by the fuse on bus 1A/1.  If the tie switch between bus 7A and bus 7B happens to be open at the time of the transformer fault, this is all that will happen.  However, if the tie switch was closed during the fault, it is likely that the fuse on the secondary side of the {\it other} transformer (feeding bus 7B) will blow instead because it not only carries all the fault current but also any load current on busses 7A and 7B.  

Usually tie switches and breakers are left in their open states, and so the former scenario is more likely.  If this is the case, power may be restored to bus 7A by opening the disconnect switch on that transformer's secondary winding, shutting down all loads on bus 7A, closing the tie switch, and re-starting loads on bus 7A.  This sequence of operations is necessary in order to avoid closing or opening the disconnect switches while they carry substantial current.

\vskip 10pt

A three-phase transformer rated at 480 volt output and 1.5 MVA, with 6\% impedance, will have a bolted-fault current capacity of 30,070 amps.









\vfil \eject

\noindent
{\bf Summary Quiz:}

Suppose a paricular single-phase transformer steps 2.4 kV down to 240 volts, with a power rating of 450 kVA.  If this transformer's impedance is rated at 7.3\%, calculate the amount of current it can deliver to a bolted fault.

\begin{itemize}
\item{} 1875 amps
\vskip 5pt 
\item{} 2568 amps
\vskip 5pt 
\item{} 14829 amps
\vskip 5pt 
\item{} 187.5 amps
\vskip 5pt 
\item{} 25685 amps
\vskip 5pt 
\item{} 137 amps
\end{itemize}


%INDEX% Safety, electrical: NFPA 70E Standard for Electrical Safety in the Workplace

%(END_NOTES)


