
%(BEGIN_QUESTION)
% Copyright 2014, Tony R. Kuphaldt, released under the Creative Commons Attribution License (v 1.0)
% This means you may do almost anything with this work of mine, so long as you give me proper credit

Read selected portions of the sales brochure for the Siemens model 3AP1/2 live-tank high-voltage circuit breakers and answer the following questions:

\vskip 10pt

Explain what the phrase ``live tank'' means with regard to a circuit breaker, referencing illustrations or photographs in the document to aid in your explanation.

\vskip 10pt

Identify at least two different styles of high-voltage circuit breakers described in this document, showing where power goes in and out of the breaker units.

\vskip 10pt

Explain how the spring mechanism works to store mechanical energy to trip and close the circuit breaker contacts.  Note that the diagram shows different colored arrows to designate the ``opening'' and ``closing'' directions.

\vskip 10pt

Compare the following current ratings of a 550 kV model 3AP2 circuit breaker, and explain why each one is different:

\begin{itemize}
\item{} Rated normal current = \underbar{\hskip 50pt}
\vskip 5pt
\item{} Rated peak withstand current = \underbar{\hskip 50pt}
\vskip 5pt
\item{} Rated short-circuit breaking current = \underbar{\hskip 50pt}
\vskip 5pt
\item{} Rated short-circuit making current = \underbar{\hskip 50pt}
\end{itemize}

\vskip 10pt

\underbar{file i03038}
%(END_QUESTION)





%(BEGIN_ANSWER)

 
%(END_ANSWER)





%(BEGIN_NOTES)

A ``live tank'' circuit breaker has its current-interrupting contacts located in a ``tank'' assembly that is elevated in both height and electrical potential.  This is plain to see with the Siemens 3AP2 model circuit breakers, where the contacts are located in a horizontal tank atop a vertical insulating pole.

\vskip 10pt

The model 3AP2 unit looks like a letter ``T'' with the high-voltage contacts in the elevated horizontal section.  The 3AP1 unit consists of vertical assemblies, with the high-voltage contacts in the upper half of each insulated assembly.  Both models are single-pole in design, three being required to make a complete three-phase circuit breaker assembly.

\vskip 10pt

Two springs are in this mechanism: one to close the breaker and another to trip it.  The closing spring is charged by an electric charging motor, and is larger in diameter than the opening spring.  The opening spring is charged by the decompression of the closing spring, during the closing cycle.  

\vskip 10pt

These current ratings are given in the table on the last page of the document.

\begin{itemize}
\item{} Rated normal current = \underbar{\bf 5 kA}
\vskip 5pt
\item{} Rated peak withstand current = \underbar{\bf 170 kA}
\vskip 5pt
\item{} Rated short-circuit breaking current = \underbar{\bf 63 kA}
\vskip 5pt
\item{} Rated short-circuit making current = \underbar{\bf 170 kA}
\end{itemize}

The normal current is the continuous rating with contacts closed (obviously).  The peak withstand rating is with contacts closed as well, which is more than the normal rating because its time duration is limited.  The ``making'' and ``breaking'' current ratings deal with contacts in motion: ``making'' = ``closing'' and ``breaking'' = ``tripping'' (opening).  Arcs tend to be greater in intensity for opening contacts than for closing contacts, since an opening contact encourages arc formation in its progressively increasing arc gap distance.  Closing contacts, by contrast, decrease in gap distance once the arc forms.  The longer an arc is, the more resistance it has, thus the more voltage it will drop, thus the greater power it will dissipate for any given amount of arc current.

%INDEX% Reading assignment: Siemens model 3AP1/2 live-tank high-voltage circuit breaker brochure

%(END_NOTES)


