
%(BEGIN_QUESTION)
% Copyright 2015, Tony R. Kuphaldt, released under the Creative Commons Attribution License (v 1.0)
% This means you may do almost anything with this work of mine, so long as you give me proper credit

Read and outline the ``Phasor Arithmetic'' subsection of the ``Phasors'' section of the ``AC Electricity'' chapter in your {\it Lessons In Industrial Instrumentation} textbook.  Note the page numbers where important illustrations, photographs, equations, tables, and other relevant details are found.  Prepare to thoughtfully discuss with your instructor and classmates the concepts and examples explored in this reading.

\underbar{file i03043}
%(END_QUESTION)




%(BEGIN_ANSWER)


%(END_ANSWER)





%(BEGIN_NOTES)

Mathematics teachers love to demonstrate $e^{i \pi} = -1$ because it links several important constants in math.  This expression, however, is just a special case of Euler's Relation $e^{i \theta} = \cos \theta + i \sin \theta$ where the angle $\theta$ is set to a value of $\pi$ radians or 180$^{o}$.  It describes a unit phasor pointed to the {\it left} on the complex plane.

\vskip 10pt

Setting the angle $\theta$ to different values merely points the phasor in different directions, thus:

$$e^{i0} = \hbox{ Unit phasor pointed right}$$

$$e^{i \pi / 2} = \hbox{ Unit phasor pointed up}$$

$$e^{i \pi} = \hbox{ Unit phasor pointed left}$$

$$e^{i 3 \pi / 2} = \hbox{ Unit phasor pointed down}$$

Complex numbers used to represent voltage, current, and impedance in AC circuits make AC circuit analysis equivalent to DC circuit analysis: all the same rules and laws apply (except with regard to calculating power).  Therefore, an important tool to have for AC circuit analysis is the ability to add, subtract, multiply, and divide phasors, so that we may apply Ohm's and Kirchhoff's Laws:

$$Ae^{jM} + Be^{jN} = (A \cos M + B \cos N) + j (A \sin M + B \sin N)$$

$$Ae^{jM} - Be^{jN} = (A \cos M - B \cos N) + j (A \sin M - B \sin N)$$

$$Ae^{jM} \times Be^{jN} = ABe^{j(M + N)}$$

$$Ae^{jM} \div Be^{jN} = {A \over B} \left[ e^{j(M - N)} \right]$$





\vskip 20pt \vbox{\hrule \hbox{\strut \vrule{} {\bf Suggestions for Socratic discussion} \vrule} \hrule}

\begin{itemize}
\item{} Why do we care about phasor arithmetic?  What is the practical application of it in our field?
\end{itemize}

%INDEX% Reading assignment: Lessons In Industrial Instrumentation, phasor arithmetic

%(END_NOTES)


