
%(BEGIN_QUESTION)
% Copyright 2007, Tony R. Kuphaldt, released under the Creative Commons Attribution License (v 1.0)
% This means you may do almost anything with this work of mine, so long as you give me proper credit

The Nernst equation yields a ``slope'' figure of 59.17 millivolts per pH unit for a perfect pH electrode probe pair.  Explain where this numerical value comes from.

\underbar{file i03071}
%(END_QUESTION)





%(BEGIN_ANSWER)

From the common logarithm version of the Nernst equation:

$$V = {{2.303 R T} \over {nF}} \log \left({C_1 \over C_2}\right)$$

\noindent
Where,

$V$ = Voltage produced across membrane due to ion exchange, in volts (V)

$R$ = Universal gas constant (8.315 J/mol$\cdot$K)

$T$ = Absolute temperature, in Kelvin (K)

$n$ = Number of electrons transferred per ion exchanged (1 for H$^{+}$ ions)

$F$ = Faraday constant, in coulombs per mole (96,485 C/mol e$^{-}$)

$C_1$ = Concentration of ion in measured solution, in moles per liter of solution ($M$)

$C_2$ = Concentration of ion in reference solution (on other side of membrane), in moles per liter of solution ($M$)

\vskip 10pt

$${{2.303 R T} \over {nF}} = 59.17 \hbox{ mV}$$

\vskip 10pt

Since the common logarithm of hydrogen ion concentration is the very definition of pH, the logarithm term of the Nernst equation may be re-written as the difference between the solution pH and the measurement probe's internal buffer pH (7.0):

$$V = {{2.303 R T} \over {nF}} \> (7 - \hbox{pH})$$

%(END_ANSWER)





%(BEGIN_NOTES)


%INDEX% Chemistry, electro-: Nernst equation
%INDEX% Measurement, analytical: pH

%(END_NOTES)


