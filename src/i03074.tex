
%(BEGIN_QUESTION)
% Copyright 2007, Tony R. Kuphaldt, released under the Creative Commons Attribution License (v 1.0)
% This means you may do almost anything with this work of mine, so long as you give me proper credit

A conductivity probe is used to measure the conductivity of a water sample.  Supposing the drive (excitation) current to the probe is 150 $\mu$A, the sensed (output) voltage is 1.64 volts, and the cell constant ($\theta$) is 1.2, calculate the specific conductivity ($k$) of the water sample.

\vskip 10pt

Relevant formulae:

$$G = k{A \over d} \hbox{\hskip 100pt} \theta = {d \over A} \hbox{\hskip 100pt} k = G \theta$$

\noindent
Where,

$G$ = Conductance, in Siemens (S)

$k$ = Specific conductivity of liquid, in Siemens per centimeter (S/cm)

$A$ = Electrode area (each), in square centimeters (cm$^{2}$)

$d$ = Electrode separation distance, in centimeters (cm)

$\theta$ = cell constant

\vskip 10pt

\underbar{file i03074}
%(END_QUESTION)





%(BEGIN_ANSWER)

$k$ = 110 $\mu$S/cm

%(END_ANSWER)





%(BEGIN_NOTES)

%INDEX% Measurement, analytical: conductivity

%(END_NOTES)


