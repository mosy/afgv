
%(BEGIN_QUESTION)
% Copyright 2007, Tony R. Kuphaldt, released under the Creative Commons Attribution License (v 1.0)
% This means you may do almost anything with this work of mine, so long as you give me proper credit

A toroidal conductivity sensor may suffer from calibration error if located too close to the wall of a metal pipe.  Explain why.

\underbar{file i03075}
%(END_QUESTION)





%(BEGIN_ANSWER)

The pipe provides a higher-conductance current path than the liquid, making it seem (from the probe's perspective) as though the liquid is more conductive than it really is.

In a sense, a metal pipe wall ``shorts out'' part of the normal current pathway through the liquid.  Electric current must still travel through the liquid as it goes through the center of the toroids, but the outer (return) path of the electric current which normally flows through liquid as well may be shunted through a metal pipe wall.

\vskip 10pt

Solutions to the problem include use of non-metal pipes, plastic-lined metal pipes, and/or oversizing the pipe so that the probe is far from the walls.

%(END_ANSWER)





%(BEGIN_NOTES)


%INDEX% Measurement, analytical: conductivity

%(END_NOTES)


