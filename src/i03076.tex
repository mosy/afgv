
%(BEGIN_QUESTION)
% Copyright 2007, Tony R. Kuphaldt, released under the Creative Commons Attribution License (v 1.0)
% This means you may do almost anything with this work of mine, so long as you give me proper credit

Is it physically possible for a solution to have a pH of less than zero or greater than 14?

\underbar{file i03076}
%(END_QUESTION)





%(BEGIN_ANSWER)

Yes.  Concentrated acids may have {\it negative} pH values, while concentrated bases can exceed 14 pH.  In such extreme cases, where standard pH probe life is extremely short, a toroidal conductivity probe may be the best way to infer acid or base concentration.

\vskip 10pt

Toroidal conductivity probes are encased in non-reactive plastic, and so can withstand very harsh pH conditions.  Given that additional concentrations of either acid or caustic both increase solution conductivity to the point of ``swamping'' other sources of conductivity (such as mineral or salt concentration), extreme levels of pH may be reliably inferred from conductivity measurements.

Conductivity measurement is definitely {\it not} a reliable method for measuring pH in weak (or even most strong) acid or base solutions, because the effect of pH changes on conductivity is not strong enough in these cases to overshadow the effect of other conductivity factors (such as salt concentration).

%(END_ANSWER)





%(BEGIN_NOTES)


%INDEX% Measurement, analytical: pH

%(END_NOTES)


