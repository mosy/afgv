
%(BEGIN_QUESTION)
% Copyright 2007, Tony R. Kuphaldt, released under the Creative Commons Attribution License (v 1.0)
% This means you may do almost anything with this work of mine, so long as you give me proper credit

Each element on the periodic table has its own preferred state of ionization in liquid solutions.  Hydrogen, Sodium, and Potassium atoms, for instance, ``want'' to ionize with a single positive charge (H$^{+}$, Na$^{+}$, and K$^{+}$, respectively).  Magnesium, Calcium, and Barium atoms ``want'' to ionize with a double positive charge (Mg$^{2+}$, Ca$^{2+}$, and Ba$^{2+}$, respectively).  Oxygen atoms ``prefer'' to ionize with a double negative charge (O$^{2-}$).  The ``halogen'' elements (Fluorine, Chlorine, Bromine, Iodine, etc.) tend to ionize with a single negative charge (F$^{-}$, Cl$^{-}$, Br$^{-}$, I$^{-}$, respectively).  Nitrogen atoms readily take on a triple negative charge (N$^{3-}$).  

\vskip 10pt

Just as the individual elements have their own ionization preferences, certain molecular combinations of elements also have preferred ionization states.

Shown here is a list of some common polyatomic ions:

\begin{itemize}
\item{} Ammonium = NH$_{4}^{+}$
\vskip 5pt
\item{} Hydroxide = OH$^{-}$
\vskip 5pt
\item{} Hydronium = H$_{3}$O$^{+}$
\vskip 5pt
\item{} Cyanide = CN$^{-}$
\vskip 5pt
\item{} Nitrite = NO$_{2}^{-}$
\vskip 5pt
\item{} Nitrate = NO$_{3}^{-}$
\vskip 5pt
\item{} Phosphate = PO$_{4}^{3-}$
\vskip 5pt
\item{} Hydrogen phosphate = HPO$_{4}^{2-}$
\vskip 5pt
\item{} Sulphite = SO$_{3}^{2-}$
\vskip 5pt
\item{} Sulphate = SO$_{4}^{2-}$
\vskip 5pt
\item{} Hydrogen sulphide = HS$^{-}$
\vskip 5pt
\item{} Hydrogen sulphite = HSO$_{3}^{-}$
\vskip 5pt
\item{} Hydrogen sulphate = HSO$_{4}^{-}$
\vskip 5pt
\item{} Carbonate = CO$_{3}^{2-}$
\vskip 5pt
\item{} Hypochlorite = ClO$^{-}$
\vskip 5pt
\item{} Chlorate = ClO$_{3}^{-}$
\vskip 5pt
\item{} Borate = BO$_{3}^{3-}$
\end{itemize}

Identify which of these ions are {\it cations} and which are {\it anions}, and then explain the meanings of the subscript and superscript numbers.

\vskip 20pt \vbox{\hrule \hbox{\strut \vrule{} {\bf Suggestions for Socratic discussion} \vrule} \hrule}

\begin{itemize}
\item{} Note that the hydrogenated versions of sulphur ions have one less negative charge (one more positive charge) than the unhydrogenated versions.  Explain this pattern!
\item{} Compare all the ions whose names end in ``-ide'' versus ``-ite'' versus ``-ate''.  Do you see any pattern to the naming of ions?
\end{itemize}


\underbar{file i03078}
%(END_QUESTION)





%(BEGIN_ANSWER)


%(END_ANSWER)





%(BEGIN_NOTES)

An ``anion'' is a negatively charged ion, which is attracted to the {\it anode} of a DC cell.  A ``cation'' is a positively charged ion, which is attracted to the {\it cathode} of a DC cell.

\vskip 10pt

Ammonium and hydronium are the only cations shown in the list, being the only ions with net positive charges.  All the other ions in the list, all having net negative charges of varying degree, are classified as anions.

\vskip 10pt

Subscripts show how many of a particular atom there are in the polyatomic ion.  Superscripts show how many electron's worth of charge is held by the ion.  Hydrogen sulphate, for instance, has 4 atoms of oxygen and a single negative charge (HSO$_{4}^{-}$).  Ammonium has 4 atoms of hydrogen and a single positive charge (NH$_{4}^{+}$).  Carbonate has three atoms of oxygen and a double negative charge (CO$_{3}^{2-}$).

\vskip 10pt

Hydrogen, which likes to exist as a single-positive charge ion (H$^{+}$), lends a single positive charge to the sulphide/sulphite/sulphate ion, raising its electrical charge (toward positive) by one electron's worth.

%INDEX% Chemistry, ions: list of polyatomic

%(END_NOTES)


