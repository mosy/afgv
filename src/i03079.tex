
%(BEGIN_QUESTION)
% Copyright 2007, Tony R. Kuphaldt, released under the Creative Commons Attribution License (v 1.0)
% This means you may do almost anything with this work of mine, so long as you give me proper credit

What is pH {\it neutralization?}  In what types of processes is this procedure relevant to, and why is it important (from an environmental perspective)?

\vskip 10pt

Is there any chemical byproduct of the pH neutralization process?

\underbar{file i03079}
%(END_QUESTION)





%(BEGIN_ANSWER)

pH neutralization is the minimization of either excessive acidity or alkalinity of aqueous liquids, usually prior to discharge into a natural body of water.  In plain language, this means controlling the pH of process water to a range usually between 6 and 9.  A pH value equal to exactly 7 is considered ``neutral.''

Wastewater treatment processes almost always involve some form of pH neutralization.  It is important for the pH of treated water to be within a range close to that of natural bodies of water, so as not to disturb the habitat of flora and fauna there.

%(END_ANSWER)





%(BEGIN_NOTES)


%INDEX% Measurement, analytical: pH

%(END_NOTES)


