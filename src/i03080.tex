
%(BEGIN_QUESTION)
% Copyright 2011, Tony R. Kuphaldt, released under the Creative Commons Attribution License (v 1.0)
% This means you may do almost anything with this work of mine, so long as you give me proper credit

Read and outline Case History \#90 (``Tuning Is Also Important'') from Michael Brown's collection of control loop optimization tutorials.  Prepare to thoughtfully discuss with your instructor and classmates the concepts and examples explored in this reading, and answer the following questions:

\begin{itemize}
\item{} Figure 1 shows the response of a gas flow control loop (supplying gas to a chemical reactor).  Based on your observation of this trend, is the controller in automatic or manual mode?  In other words, is this a {\it closed-loop} test or an {\it open-loop} test, and how can you tell for certain?
\vskip 10pt
\item{} After properly tuning this gas flow loop, how much quicker did it respond to setpoint changes than it did before with the old PID tuning?  Calculate a ratio, if possible, from the data shown in the trends (Figures 1 versus 2).
\vskip 10pt
\item{} The day after Mr. Brown optimized this reactor flow loop, another section of this South African chemical plant issued a complaint.  Describe what their complaint was, and why it was caused by the work done on this flow control loop.  Also describe the ``compromise'' solution implemented after this complaint.
\vskip 10pt
\item{} Explain why Mr. Brown began optimizing the flow controller before optimizing the temperature controller on this distillation tower reflux control system.
\end{itemize}

\vskip 20pt \vbox{\hrule \hbox{\strut \vrule{} {\bf Suggestions for Socratic discussion} \vrule} \hrule}

\begin{itemize}
\item{} Although Mr. Brown does not provide the ``before'' or ``after'' P-I-D tuning parameter values for the gas pressure control system shown in figures 1 and 2, try to discern what sort of tuning was there before (e.g. ``predominantly proportional; predominantly integral; how much gain?''), and what sort of tuning was used to make it respond so much faster.
\item{} Mr. Brown states that ``quarter-wave damping'' is often taught as the optimum response for a well-tuned PID control loop.  Where do you think this popular standard for control quality came from?
\end{itemize}

\underbar{file i03080}
%(END_QUESTION)





%(BEGIN_ANSWER)


%(END_ANSWER)





%(BEGIN_NOTES)

Michael Brown claims that approximately 85\% of industrial control loops operate inefficiently in automatic.  He has many skeptics, but is usually able to prove his point once he gets inside the plant and starts working.

\vskip 10pt

Figure 1 shows an automatic (closed-loop) test, because the output responded to the SP change (albeit slightly), and also because the output responds in kind to the PV curve.

\vskip 10pt

The time ratio from old tuning to new tuning was 1 hour to 20 seconds = {\it 180:1 ratio of time savings!}

\vskip 10pt

After optimizing the gas pressure control loop, the upstream plant supplying the gas complained because the faster valve motions were causing loads in their operation.  The compromise solution was to de-tune the gas pressure controller for slower response (about half as fast as before).

\vskip 10pt

The ``reflux'' flow on a distillation tower is a cascaded (slave) loop to control the temperature at the top of the column.  They optimized this flow loop before the temperature loop because the slave must be optimized first: the master's operation depends on the slave doing its job fast and well.


%INDEX% Reading assignment: Michael Brown Case History #90, "Tuning is also important"

%(END_NOTES)


