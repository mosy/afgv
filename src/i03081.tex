
%(BEGIN_QUESTION)
% Copyright 2007, Tony R. Kuphaldt, released under the Creative Commons Attribution License (v 1.0)
% This means you may do almost anything with this work of mine, so long as you give me proper credit

The common-logarithm form of the Nernst equation is very insightful for pH applications, because it lends itself to significant simplification:

$$V = {{2.303 R T} \over {nF}} \log \left({C_1 \over C_2}\right)$$

$$\hbox{. . . after simplification . . .}$$

$$V = {{2.303 R T} \over {nF}} \> (7 - \hbox{pH})$$

\vskip 10pt

Demonstrate mathematically how it is possible to manipulate the first Nernst equation into the form of the second.

\vfil 

\underbar{file i03081}
\eject
%(END_QUESTION)





%(BEGIN_ANSWER)

This is a graded question -- no answers or hints given!

%(END_ANSWER)





%(BEGIN_NOTES)

$$V = {{2.303 R T} \over {nF}} \log \left({C_1 \over C_2}\right)$$

$$V = {{2.303 R T} \over {nF}} \log C_1 - \log C_2$$

$C_1$ and $C_2$ refer to the hydrogen ion activities outside and inside of the glass bulb, respectively.  Therefore:

$$V = {{2.303 R T} \over {nF}} ( -\hbox{pH} - (-7))$$

$$V = {{2.303 R T} \over {nF}} (7 -\hbox{pH})$$


%INDEX% Chemistry, electro-: Nernst equation
%INDEX% Measurement, analytical: pH

%(END_NOTES)


