
%(BEGIN_QUESTION)
% Copyright 2015, Tony R. Kuphaldt, released under the Creative Commons Attribution License (v 1.0)
% This means you may do almost anything with this work of mine, so long as you give me proper credit

Suppose an 800:5 current transformer with an internal winding resistance of 0.25 ohms is connected to the input terminals of a protective relay presenting a purely resistive burden of 0.7 ohms.  The wire used to connect the CT to the relay is 12 gauge, 1500 feet total circuit length (i.e. 750 feet cable length for a 2-conductor cable).  

\vskip 10pt

From this information, determine both the voltage dropped across the relay's terminals at a fault current value of 4.5 kA (power line current) and the amount of voltage the CT secondary must produce internally to overcome both the total circuit burden (i.e. relay plus wiring plus the CT's own winding resistance) during this same fault condition.

\vskip 10pt

$V_{relay}$ = \underbar{\hskip 50pt} volts

\vskip 10pt

$V_{CT-winding}$ = \underbar{\hskip 50pt} volts

\vskip 10pt

\underbar{file i03088}
%(END_QUESTION)





%(BEGIN_ANSWER)

If you are having trouble knowing where to start, {\it begin by sketching a schematic diagram showing all components and how they connect to each other.}  Represent each resistance as its own resistor symbol, and annotate the diagram with all given information.  Treat the CT as a {\it current source} driving power to the protective relay.
 
%(END_ANSWER)





%(BEGIN_NOTES)

$$I_{CT} = \left({4500 \hbox{ A} \over 1} \right) \left({5 \over 800} \right) = 28.125 \hbox{ A}$$

$$R_{wire} = (1500 \hbox{ ft}) \left( {e^{(12)(0.232) - 2.32} \> \Omega} \over 1000 \hbox{ ft} \right) = 2.3856 \> \Omega$$

$$V_{relay} = I_{CT} R_{relay} = (28.125 \hbox{ A})(0.7 \> \Omega) = 19.6875 \hbox{ V}$$

$$V_{CT-winding} = I_{CT} R_{total} = (28.125 \hbox{ A})(0.7 \> \Omega + 2.3856 \> \Omega) = 86.7835 \hbox{ V}$$

%INDEX% Electronics review: current transformer (CT)

%(END_NOTES)


