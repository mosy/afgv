
%(BEGIN_QUESTION)
% Copyright 2014, Tony R. Kuphaldt, released under the Creative Commons Attribution License (v 1.0)
% This means you may do almost anything with this work of mine, so long as you give me proper credit

Read selected portions of the ``Lessons Learned From Commissioning Protective Relay Systems'' whitepaper (written by Karl Zimmerman and David Costello of Schweitzer Engineering Labs) and answer the following questions:

\vskip 10pt

In Application Example ``A'' (pp. 7-8) we are shown a case where a pair of line differential relays are protecting a short transmission line.  Figure 10 shows a single-line diagram of the system and Figure 11 shows a ``live'' phasor diagram of currents measured by these two relays.  Describe what the problem was in this case, and explain how the phasor diagram reveals this to be so.  It may be helpful to sketch a ``corrected'' phasor diagram showing what the phasors ought to have looked like during initial testing.

\vskip 10pt

In Application Example ``H'' (pp. 14-15) a case is explored involving a differential current relay installed on a step-down transformer.  Figure 34 on page 15 shows the current phasor diagram for transformer primary and secondary sides.  Identify the problem by careful examination of this phasor diagram.  Also, identify how we can tell which winding (W1 or W2) of this transformer is the high-voltage side based on the phasor diagram alone.

\vskip 10pt

Appendix C beginning on page 22 describes a test called {\it primary current injection}.  Describe what would be necessary to perform this kind of commissioning test, and identify the different elements of the protective relay system verified by it.

\vskip 20pt \vbox{\hrule \hbox{\strut \vrule{} {\bf Suggestions for Socratic discussion} \vrule} \hrule}

\begin{itemize}
\item{} Application Example ``C'' on page 9 cites the use of {\it synchrophasors} to commission instrument transformers.  Explain what ``synchrophasors'' are and how they were used in this case to verify proper VT function.
\end{itemize}

\underbar{file i03089}
%(END_QUESTION)





%(BEGIN_ANSWER)

 
%(END_ANSWER)





%(BEGIN_NOTES)

\noindent
{\bf Application Example ``A''}

The current phasors should be 180 degrees apart (facing opposite directions) when comparing local and remote relay currents (IAL, IBL, and ICL currents at the local relay versus IAX, IBX, and ICX currents at the remote relay).  Instead they are only 120 degrees apart, the result of incorrect wiring.

\vskip 10pt

\noindent
{\bf Application Example ``H''}

The phasor diagram shows a standard ABC phase rotation for winding 2 (IAW2, IBW2, and ICW2 phasors), but a CBA phase rotation for winding 1 (IAW1, IBW1, and ICW1 phasors).  We can tell that winding 2 is the higher-voltage side because its current phasors are shorter in length (i.e. less current means that winding must have more voltage). 

\vskip 10pt

\noindent
{\bf Appendix C}

Primary current injection works by passing high current from some temporary power source through the actual power system components and conductors, and measuring the results back at the protective relay.  This is a very thorough test because all elements of the protective relay system from instrument transformers to the relay itself come into play.



%INDEX% Reading assignment: SEL whitepaper on protective relay system commissioning

%(END_NOTES)


