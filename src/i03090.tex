
%(BEGIN_QUESTION)
% Copyright 2014, Tony R. Kuphaldt, released under the Creative Commons Attribution License (v 1.0)
% This means you may do almost anything with this work of mine, so long as you give me proper credit

A set of current transformers located in the bushings of a substation circuit breaker are C200 class with 800:5 ratios and winding resistance values of 0.15 ohms each.  Each CT secondary connects to a digital protective relay having an input resistance of 6 milliohms.  The wire used for this CT secondary circuit is 12 AWG and the distance between the circuit breaker and the relay is 150 feet.

\vskip 10pt

An electrical engineer performs a ``system study'' analysis and determines the maximum fault current for this power circuit to be 11,000 amps.  Given these parameters, will the CTs be able to deliver accurate representations of fault current to the protective relay?  Consider only AC current (not DC transients) in your analysis of this system, for simplicity.

\vskip 20pt \vbox{\hrule \hbox{\strut \vrule{} {\bf Suggestions for Socratic discussion} \vrule} \hrule}

\begin{itemize}
\item{} A useful problem-solving technique is to sketch a simple diagram of the system you are asked to analyze.  This is useful even when you already have some graphical representation of the problem given to you, as a simple sketch often reduces the complexity of the problem so that you can solve it more easily.  Draw your own sketch showing how the given information in this problem inter-relates, and use this sketch to explain your solution.
\item{} Quantitative problems requiring many calculations to arrive at a result are good candidates for solution using a computer spreadsheet program such as Microsoft Excel.  Try creating a spreadsheet to solve this problem, and you will have a tool useful for solving other problems like it!
\item{} Suppose the $X \over R$ ratio of this power system were significant enough that certain faults could produce DC transients worthy of consideration for CT performance.  Calculate the maximum $X \over R$ ratio we could tolerate with this CT circuit as specified.
\item{} Suppose the wire resistance in this circuit happened to be excessive for the fault conditions and CT capability specified.  What could we alter in this system to improve matters?
\end{itemize}

\underbar{file i03090}
%(END_QUESTION)





%(BEGIN_ANSWER)

We should begin by analyzing what each CT will be able to do under ideal circumstances.  A ``C200'' class CT is supposed to be able to generate 200 volts at its terminals while experiencing 20$\times$ its rated amount of current (i.e. 100 amps secondary current).  In order to generate 200 volts at its terminals while overcoming the internal voltage drop of its own winding resistance (0.15 ohms carrying 100 amps), the CT's winding must internally generate 215 volts (200 volts + [100 amps $\times$ 0.15 ohms]).  This represents the maximum amount of AC voltage the CT is capable of internally producing at full magnetic flux.

The absence of DC transients in this system means we don't have to worry about de-rating the CT for the sake of DC bias magnetization.  In other words, we may assume the full 215 volts of generating capacity will be available for this CT to drive current through its own internal resistance, the wire resistance, and the relay's burden.

\vskip 10pt

Now that we know what the CT is ideally capable of, we may apply these CT values (215 volts $V_{max}$, 0.15 ohms $R_{winding}$) to the scenario at hand and determine if the CT will be able to deliver its rated performance under fault conditions.

\vskip 10pt

A system fault current of 11000 amps will be stepped down by the CT's 800:5 ratio to become 68.75 amps of current in the secondary circuit.  Thus, the CT must be able to produce enough voltage internally in its secondary winding to push 68.75 amps of current through its own winding resistance, the wire resistance, and the relay's burden resistance.  The next piece of information we need to do this analysis is the resistance of the wire.

150 feet of distance between the circuit breaker and the protective relay means we have 300 feet of wire connecting each CT to the relay.  With a wire size of 12 AWG, its resistance per 1000 feet of length is 1.59 ohms.  300 feet of this wire will therefore give us 0.477 ohms of wire resistance in the loop.

Summing up all these resistances and using Ohm's Law to predict the necessary voltage generated by the CT:

$$V = IR$$

$$V = I (R_{CT\_winding} + R_{wire} + R_{relay})$$

$$V = (68.75 \hbox{ A})(0.15 \> \Omega + 0.477 \> \Omega + 0.006 \> \Omega)$$

$$V = (68.75 \hbox{ A})(0.633 \> \Omega)$$

$$V = 43.53 \hbox{ V}$$

This necessary voltage of 43.53 volts is substantially less than the CT's maximum of 215 volts (internal, at the secondary winding), and so this CT circuit should perform quite well under the stated conditions.
 
%(END_ANSWER)





%(BEGIN_NOTES)

In answer to the Socratic Question of $X \over R$ ratio for the power system, we know that CT de-rating follows the factor $1 + {X \over R}$.  Since we know the CT is capable of internally generating up to 215 volts without saturation, and we know the total circuit driving voltage requirement is only 43.53 volts (4.939 times less), we may predict a maximum $X \over R$ ratio for the power system of 3.939.

%INDEX% Electronics review: current transformer (CT)
%INDEX% Protective relay: CT circuit wire resistance

%(END_NOTES)


