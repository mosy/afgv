
%(BEGIN_QUESTION)
% Copyright 2014, Tony R. Kuphaldt, released under the Creative Commons Attribution License (v 1.0)
% This means you may do almost anything with this work of mine, so long as you give me proper credit

Suppose you need to lock out electrical power to a 30 HP three-phase electric motor in order to disconnect its power wires and prepare it for replacement.  This motor receives its power from a ``bucket'' (a starter circuit) located in a nearby motor control center (MCC).  The motor bucket is equipped with a disconnect switch which is lockable.

List the essential steps you must follow to lock out this motor in accordance with NFPA 70E guidelines for a ``simple'' lockout/tagout procedure:

\vskip 100pt

\underbar{file i03111}
%(END_QUESTION)





%(BEGIN_ANSWER)

Answers \underbar{must} include these 5 elements.  Additional steps may be included before, after, or in-between these elements, but the order of these elements must not be compromised:

\begin{itemize}
\item{} Throw the disconnect switch into the open position
\item{} Apply the lock and tag
\item{} Verify proper voltmeter operation against a known source
\item{} Test for absence of voltage using the meter
\item{} Verify proper voltmeter operation against a known source (again)
\end{itemize}
 
%(END_ANSWER)





%(BEGIN_NOTES)

{\bf This question is intended for exams only and not worksheets!}.

%(END_NOTES)


