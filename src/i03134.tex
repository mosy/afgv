
%(BEGIN_QUESTION)
% Copyright 2008, Tony R. Kuphaldt, released under the Creative Commons Attribution License (v 1.0)
% This means you may do almost anything with this work of mine, so long as you give me proper credit

Mechanical, chemical, and civil engineers must often calculate the size of piping necessary to transport fluids.  One equation used to relate the water-carrying capacity of multiple, small pipes to the carrying capacity of one large pipe is as follows:

$$N = \biggl({d_2 \over d_1}\biggr)^{2.5}$$

\noindent
Where,

$N =$ Number of small pipes

$d_1 =$ Diameter of each small pipe

$d_2 =$ Diameter of large pipe

\vskip 30pt

Manipulate this equation to solve for $d_1$, and again to solve for $d_2$.  Be sure to show all your work!

\vskip 50pt

$d_1 = $

\vskip 100pt

$d_2 = $


\vfil 

\underbar{file i03134}
\eject
%(END_QUESTION)





%(BEGIN_ANSWER)

This is a graded question -- no answers or hints given!

%(END_ANSWER)





%(BEGIN_NOTES)

$$d_1 = {d_2 \over N^{0.4}}$$

$$d_2 = d_1 N^{0.4}$$

\vskip 10pt

Taken from page 3-378 of the \underbar{Standard Handbook of Engineering Calculations}, Editor: Tyler G. Hicks, P.E., ISBN 0-07-028734-1.

%INDEX% Mathematics review: manipulating literal equations

%(END_NOTES)


