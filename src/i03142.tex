
%(BEGIN_QUESTION)
% Copyright 2011, Tony R. Kuphaldt, released under the Creative Commons Attribution License (v 1.0)
% This means you may do almost anything with this work of mine, so long as you give me proper credit

Suppose a farmer working hard in his field on a hot summer day loses 2 pounds of water due to sweat (evaporative body cooling).  Use a {\it steam table} to calculate the heat lost due to the phase change from liquid (sweat droplets) to vapor (water vapor into the air) on his skin at a temperature of 90 degrees Fahrenheit.  Express your answer in the unit of {\it dietary Calories} (the type of ``Calorie'' used to rate food energy).

\vfil 

\underbar{file i03142}
\eject
%(END_QUESTION)





%(BEGIN_ANSWER)

This is a graded question -- no answers or hints given!

%(END_ANSWER)





%(BEGIN_NOTES)

A steam table gives the latent heat of vaporization value of 1041.2 BTU/lb at a temperature of 90 degrees F.  2 pounds of water evaporated thus yields a heat loss of:

\vskip 10pt

$Q = mL = (2)(1041.2) =$ 2082.4 BTU

\vskip 10pt

This converts to 524,756 calories, or 524.8 dietary calories (kilocalories). 

\vskip 20pt

It is noteworthy how the latent heat of evaporation for water at 90 degrees Fahrenheit is {\it not} 970.3 BTU/lb as it is at the boiling point of water.  If $L$ were stable over wide temperature ranges, it would render references such as steam tables superfluous.  As it is, the varying values of $L$ (and also of specific heat $c$) complicate calculations of heat when using formulae such as $Q = mc \Delta T$ and $Q = mL$.  This is why steam tables are so tremendously useful: the values contained therein are empirically (i.e. experimentally) based, and therefore account for all the nonlinearities and changing values of $L$ and $c$.


%INDEX% Physics, heat and temperature: latent heat calculation
%INDEX% Physics, heat and temperature: steam table

%(END_NOTES)


