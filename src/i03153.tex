
%(BEGIN_QUESTION)
% Copyright 2010, Tony R. Kuphaldt, released under the Creative Commons Attribution License (v 1.0)
% This means you may do almost anything with this work of mine, so long as you give me proper credit

Suppose a voltmeter registers 0 volts between test points {\bf C} and {\bf B} in this series-parallel circuit:

$$\includegraphics[width=15.5cm]{i03153x01.eps}$$

Identify the likelihood of each specified fault for this circuit.  Consider each fault one at a time (i.e. no coincidental faults), determining whether or not each fault could independently account for {\it all} measurements and symptoms in this circuit.

% No blank lines allowed between lines of an \halign structure!
% I use comments (%) instead, so that TeX doesn't choke.

$$\vbox{\offinterlineskip
\halign{\strut
\vrule \quad\hfil # \ \hfil & 
\vrule \quad\hfil # \ \hfil & 
\vrule \quad\hfil # \ \hfil \vrule \cr
\noalign{\hrule}
%
% First row
{\bf Fault} & {\bf Possible} & {\bf Impossible} \cr
%
\noalign{\hrule}
%
% Another row
$R_1$ failed open &  &  \cr
%
\noalign{\hrule}
%
% Another row
$R_2$ failed open &  &  \cr
%
\noalign{\hrule}
%
% Another row
$R_3$ failed open &  &  \cr
%
\noalign{\hrule}
%
% Another row
$R_1$ failed shorted &  &  \cr
%
\noalign{\hrule}
%
% Another row
$R_2$ failed shorted &  &  \cr
%
\noalign{\hrule}
%
% Another row
$R_3$ failed shorted &  &  \cr
%
\noalign{\hrule}
%
% Another row
Voltage source dead &  &  \cr
%
\noalign{\hrule}
} % End of \halign 
}$$ % End of \vbox

Finally, identify the {\it next} diagnostic test or measurement you would make on this system.  Explain how the result(s) of this next test or measurement help further identify the location and/or nature of the fault.

\vfil 

\underbar{file i03153}
\eject
%(END_QUESTION)





%(BEGIN_ANSWER)

This is a graded question -- no answers or hints given!

%(END_ANSWER)





%(BEGIN_NOTES)

If two test points which {\it should} have voltage dropped between them don't actually have voltage dropped between them, we're either dealing with a short-circuit between those two points, or an open fault separating one of those two points from the power supply, or a dead power supply.

% No blank lines allowed between lines of an \halign structure!
% I use comments (%) instead, so that TeX doesn't choke.

$$\vbox{\offinterlineskip
\halign{\strut
\vrule \quad\hfil # \ \hfil & 
\vrule \quad\hfil # \ \hfil & 
\vrule \quad\hfil # \ \hfil \vrule \cr
\noalign{\hrule}
%
% First row
{\bf Fault} & {\bf Possible} & {\bf Impossible} \cr
%
\noalign{\hrule}
%
% Another row
$R_1$ failed open &  & $\surd$ \cr
%
\noalign{\hrule}
%
% Another row
$R_2$ failed open &  & $\surd$ \cr
%
\noalign{\hrule}
%
% Another row
$R_3$ failed open &  & $\surd$ \cr
%
\noalign{\hrule}
%
% Another row
$R_1$ failed shorted &  & $\surd$ \cr
%
\noalign{\hrule}
%
% Another row
$R_2$ failed shorted & $\surd$ &  \cr
%
\noalign{\hrule}
%
% Another row
$R_3$ failed shorted &  & $\surd$ \cr
%
\noalign{\hrule}
%
% Another row
Voltage source dead & $\surd$ &  \cr
%
\noalign{\hrule}
} % End of \halign 
}$$ % End of \vbox

A good ``next test'' would be to measure current through R2.  If there is current, we know R2 is shorted and the source is good.  If there is no current, we know we have a dead source.

%INDEX% Troubleshooting review: electric circuits

%(END_NOTES)


