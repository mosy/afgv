
%(BEGIN_QUESTION)
% Copyright 2008, Tony R. Kuphaldt, released under the Creative Commons Attribution License (v 1.0)
% This means you may do almost anything with this work of mine, so long as you give me proper credit

The following motor control ``bucket'' has a problem.  When the ``Start'' button is pressed, the motor refuses to start, although an audible ``clunk'' may be heard from the contactor when the ``Start'' switch is pressed, and again when the ``Stop'' switch is pressed:

$$\includegraphics[width=15.5cm]{i03201x01.eps}$$

Using your digital voltmeter, you measure 480 volts AC between TP2 and TP3 with the ``Start'' switch pressed.  From this information, identify two possible faults (either one of which could account for the problem and all measured values in this circuit), and also identify two circuit elements that could not possibly be to blame (i.e. two things that you know {\it must} be functioning properly, no matter what else may be faulted).  The circuit elements you identify as either possibly faulted or properly functioning can be wires, traces, and connections as well as components.  Be as specific as you can in your answers, identifying both the circuit element and the type of fault.

\medskip
\goodbreak
\item{} Circuit elements that are possibly faulted
\item{1.}
\item{2.} 
\end{itemize}

\medskip
\goodbreak
\item{} Circuit elements that must be functioning properly
\item{1.} 
\item{2.} 
\end{itemize}

\vfil 

\underbar{file i03201}
\eject
%(END_QUESTION)





%(BEGIN_ANSWER)

This is a graded question -- no answers or hints given!

%(END_ANSWER)





%(BEGIN_NOTES)

Note: the following answers are not exhaustive.  There may be more circuit elements possibly at fault and more circuit elements known to be functioning properly!

\begin{itemize}
\item{} Circuit elements that are possibly faulted
\item{1.} Motor winding(s) failed open
\item{2.} One or more power conductors to motor failed open
\item{3.} One or more overload heater elements failed open
\item{4.} Power contact (one of the three M1 contacts to the motor) failed open
\end{itemize}

\begin{itemize}
\item{} Circuit elements that must be functioning properly
\item{1.} 3-phase power source good
\item{2.} Control power transformer (480/120 volt unit supplying L1 and L2)
\item{3.} Both fuses
\item{4.} ``Start'' switch contacts
\end{itemize}

\vskip 20pt \vbox{\hrule \hbox{\strut \vrule{} {\bf Virtual Troubleshooting} \vrule} \hrule}

This question is a good candidate for a ``Virtual Troubleshooting'' exercise.  Presenting the diagram to students, you first imagine in your own mind a particular fault in the system.  Then, you present one or more symptoms of that fault (something noticeable by an operator or other user of the system).  Students then propose various diagnostic tests to perform on this system to identify the nature and location of the fault, as though they were technicians trying to troubleshoot the problem.  Your job is to tell them what the result(s) would be for each of the proposed diagnostic tests, documenting those results where all the students can see.

During and after the exercise, it is good to ask students follow-up questions such as:

\begin{itemize}
\item{} What does the result of the last diagnostic test tell you about the fault?
\item{} Suppose the results of the last diagnostic test were different.  What then would that result tell you about the fault?
\item{} Is the last diagnostic test the best one we could do?
\item{} What would be the ideal order of tests, to diagnose the problem in as few steps as possible?
\end{itemize}

%INDEX% Troubleshooting review: electric circuits

%(END_NOTES)


