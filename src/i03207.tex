
%(BEGIN_QUESTION)
% Copyright 2008, Tony R. Kuphaldt, released under the Creative Commons Attribution License (v 1.0)
% This means you may do almost anything with this work of mine, so long as you give me proper credit

Small relays often come packaged in clear, rectangular, plastic cases.  These so-called ``ice cube'' relays have either eight or eleven pins protruding from the bottom, allowing them to be plugged into a special socket for connection with wires in a circuit.  Note the labels near terminals on the relay socket, showing the locations of the coil terminals and contact terminals:

$$\includegraphics[width=15.5cm]{i03207x01.eps}$$

Draw the necessary connecting wires between terminals in this circuit, so that actuating the normally-open pushbutton switch sends power from the battery to the coil to energize the relay:

\vskip 20pt

$$\includegraphics[width=15.5cm]{i03207x02.eps}$$

\vfil 

\underbar{file i03207}
\eject
%(END_QUESTION)





%(BEGIN_ANSWER)

This is a graded question -- no answers or hints given!
 
%(END_ANSWER)





%(BEGIN_NOTES)

This is by no means the only solution, but it works:

$$\includegraphics[width=15.5cm]{i03207x03.eps}$$

``Ice cube'' style relays are very common in industry, and it is important that students understand how to interpret the pin diagrams on the cases in order to use them in new circuits and to troubleshoot relay circuits that are already built.

\vskip 10pt

An important problem-solving skill to apply here is to sketch the intended circuit in an orderly {\it schematic diagram} before attempting to sketch wires in the {\it pictorial} diagram.  Schematic diagrams give you the freedom to position components where you want them to be, allowing you to think more clearly about the circuit's function and pay less attention to layout.  Your conceptual understanding of the circuit must be firm or else your pictorial wire-sketching efforts will surely be error-prone.  Here is an example schematic:

$$\includegraphics[width=15.5cm]{i03207x04.eps}$$

%INDEX% Pictorial circuit review (relay circuit)

%(END_NOTES)


