
%(BEGIN_QUESTION)
% Copyright 2008, Tony R. Kuphaldt, released under the Creative Commons Attribution License (v 1.0)
% This means you may do almost anything with this work of mine, so long as you give me proper credit

Small relays often come packaged in clear, rectangular, plastic cases.  These so-called ``ice cube'' relays have either eight or eleven pins protruding from the bottom, allowing them to be plugged into a special socket for connection with wires in a circuit.  Note the labels near terminals on the relay socket, showing the locations of the coil terminals and contact terminals:

$$\includegraphics[width=15.5cm]{i03211x01.eps}$$

Draw the necessary connecting wires between terminals in this circuit, so that actuating the normally-open pushbutton switch sends power from the battery to the coil to energize the relay, with one of the relay's normally-open contacts turning the lamp on.  The pushbutton switch should not carry any lamp current, just enough current to energize the relay coil:

\vskip 20pt

$$\includegraphics[width=15.5cm]{i03211x02.eps}$$

\vfil 

\underbar{file i03211}
\eject
%(END_QUESTION)





%(BEGIN_ANSWER)

This is a graded question -- no answers or hints given!
 
%(END_ANSWER)





%(BEGIN_NOTES)

This is by no means the only solution, but it works:

$$\includegraphics[width=15.5cm]{i03211x03.eps}$$

``Ice cube'' style relays are very common in industry, and it is important that students understand how to interpret the pin diagrams on the cases in order to use them in new circuits and to troubleshoot relay circuits that are already built.

%INDEX% Pictorial circuit review (relay circuit)

%(END_NOTES)


