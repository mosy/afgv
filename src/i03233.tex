
%(BEGIN_QUESTION)
% Copyright 2010, Tony R. Kuphaldt, released under the Creative Commons Attribution License (v 1.0)
% This means you may do almost anything with this work of mine, so long as you give me proper credit

Note the oscilloscope measurement of AC voltage between test points {\bf D} and {\bf F} in this series-parallel circuit:

$$\includegraphics[width=15.5cm]{i03233x01.eps}$$

Identify the likelihood of each specified fault for this circuit.  Consider each fault one at a time (i.e. no coincidental faults), determining whether or not each fault could independently account for {\it all} measurements and symptoms in this circuit.

% No blank lines allowed between lines of an \halign structure!
% I use comments (%) instead, so that TeX doesn't choke.

$$\vbox{\offinterlineskip
\halign{\strut
\vrule \quad\hfil # \ \hfil & 
\vrule \quad\hfil # \ \hfil & 
\vrule \quad\hfil # \ \hfil \vrule \cr
\noalign{\hrule}
%
% First row
{\bf Fault} & {\bf Possible} & {\bf Impossible} \cr
%
\noalign{\hrule}
%
% Another row
$R_1$ failed open &  &  \cr
%
\noalign{\hrule}
%
% Another row
$R_2$ failed open &  &  \cr
%
\noalign{\hrule}
%
% Another row
$R_3$ failed open &  &  \cr
%
\noalign{\hrule}
%
% Another row
$R_1$ failed shorted &  &  \cr
%
\noalign{\hrule}
%
% Another row
$R_2$ failed shorted &  &  \cr
%
\noalign{\hrule}
%
% Another row
$R_3$ failed shorted &  &  \cr
%
\noalign{\hrule}
%
% Another row
Voltage source output too high &  &  \cr
%
\noalign{\hrule}
} % End of \halign 
}$$ % End of \vbox

Finally, identify the {\it next} diagnostic test or measurement you would make on this system.  Explain how the result(s) of this next test or measurement help further identify the location and/or nature of the fault.

\vfil 

\underbar{file i03233}
\eject
%(END_QUESTION)





%(BEGIN_ANSWER)

This is a graded question -- no answers or hints given!
 
%(END_ANSWER)





%(BEGIN_NOTES)

% No blank lines allowed between lines of an \halign structure!
% I use comments (%) instead, so that TeX doesn't choke.

$$\vbox{\offinterlineskip
\halign{\strut
\vrule \quad\hfil # \ \hfil & 
\vrule \quad\hfil # \ \hfil & 
\vrule \quad\hfil # \ \hfil \vrule \cr
\noalign{\hrule}
%
% First row
{\bf Fault} & {\bf Possible} & {\bf Impossible} \cr
%
\noalign{\hrule}
%
% Another row
$R_1$ failed open &  & $\surd$ \cr
%
\noalign{\hrule}
%
% Another row
$R_2$ failed open &  & $\surd$ \cr
%
\noalign{\hrule}
%
% Another row
$R_3$ failed open &  & $\surd$ \cr
%
\noalign{\hrule}
%
% Another row
$R_1$ failed shorted &  & $\surd$ \cr
%
\noalign{\hrule}
%
% Another row
$R_2$ failed shorted & $\surd$ &  \cr
%
\noalign{\hrule}
%
% Another row
$R_3$ failed shorted &  & $\surd$ \cr
%
\noalign{\hrule}
%
% Another row
Voltage source output too high & $\surd$ &  \cr
%
\noalign{\hrule}
} % End of \halign 
}$$ % End of \vbox

A good ``next test'' to do here is check source voltage with the oscilloscope to make sure it is outputting only 12 volts.  If so, the problem is with $R_2$.  A test that would confirm a too-high source voltage (rather than a shorted $R_2$) would be to measure voltage across $R_2$: if that voltage is zero, then $R_2$ is shorted; if that voltage is much greater than it should be, an excess power supply voltage is the problem.

%INDEX% Electronics review: oscilloscope usage
%INDEX% Troubleshooting review: electric circuits

%(END_NOTES)


