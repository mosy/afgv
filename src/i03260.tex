
%(BEGIN_QUESTION)
% Copyright 2011, Tony R. Kuphaldt, released under the Creative Commons Attribution License (v 1.0)
% This means you may do almost anything with this work of mine, so long as you give me proper credit

Read and outline Case History \#79 (``Are Smart Positioners The Answer To All Valve Problems?'') from Michael Brown's collection of control loop optimization tutorials.  Prepare to thoughtfully discuss with your instructor and classmates the concepts and examples explored in this reading, and answer the following questions:

\begin{itemize}
\item{} Explain how even a smart positioner can exhibit poor behavior if the valve actuator is under-sized.
\vskip 10pt
\item{} Examine Figure 1 showing an open-loop test of a control valve equipped with a smart positioner.  What {\it should} the trend look like if the positioner were properly doing its job?
\vskip 10pt
\item{} Describe some of the awful valve behavior evident in the trend of Figure 3, another open-loop test of a control valve equipped with a smart positioner.
\end{itemize}

\vskip 20pt \vbox{\hrule \hbox{\strut \vrule{} {\bf Suggestions for Socratic discussion} \vrule} \hrule}

\begin{itemize}
\item{} Should the pitiful performance of these control valves be taken as an indictment of smart valve positioners?  Why or why not?
\item{} Suppose you were the instrument technician tasked with {\it repairing} one of the bad control valves identified by a loop optimization team.  What kind of diagnostic test would you consider running, using the smart positioner's diagnostic capabilities?
\item{} How do you suppose the Fisher DVC6000 series of control valve positioners perform in light of Michael Brown's concerns?
\end{itemize}

\underbar{file i03260}
%(END_QUESTION)





%(BEGIN_ANSWER)


%(END_ANSWER)





%(BEGIN_NOTES)

Some ``smart'' positioners don't use P+I control (internally), but are P-only.  This means some smart positioners will inevitably exhibit P-only offset and therefore fail to perfectly achieve the commanded position.

\vskip 10pt

Even the ``smartest'' valve positioner cannot do its job if the valve actuator is under-powered and therefore lacks the ``muscle'' to move the control valve properly.

\vskip 10pt

If positioner in Figure 1 were properly functioning, the flow would quickly step up and down with the step-changes in PD.  In other words, the flow signal should be nearly a copy of the Output trend.

\vskip 10pt

Figure 3 shows a ``smart'' positioner with ``mental defects'': valve is completely non-repeatable.










%INDEX% Reading assignment: Michael Brown Case History #79, "Are smart positioners the answer to all valve problems?"

%(END_NOTES)


