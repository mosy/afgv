
%(BEGIN_QUESTION)
% Copyright 2011, Tony R. Kuphaldt, released under the Creative Commons Attribution License (v 1.0)
% This means you may do almost anything with this work of mine, so long as you give me proper credit

Read selected sections of the US Chemical Safety and Hazard Investigation Board's report (2008-3-I-FL) of the 2008 chemical reactor explosion at T2 Laboratories in Jacksonville, Florida, and answer the following questions.

\vskip 10pt

Explain in your own words what the CSB's ``only credible cause'' was for the chemical reactor vessel to violently explode.

\vskip 10pt

Page 20 shows a simple diagram of the reactor vessel and associated apparatus.  Identify in this diagram the single over-pressure protection device, and explain how it is supposed to function in the event of an emergency.

\vskip 10pt

Page 25 describes some chemical tests the CSB performed in the wake of this accident, and one of their recommendations is that the over-pressure relief limit should have been set much lower than it was.  Explain the rationale for this recommendation.

\vskip 10pt

On page 24, it says {\it ``Interviews with employees indicated that T2 ran cooling system components to failure . . .''}.  Explain what this phrase means, in your own words, and how this policy adversely affects the probability of failure on demand (PFD) for any safety-related control system.

\vskip 10pt

Page 24 lists several potential single-point failures in the reactor's cooling system.  Brainstorm specific design changes which could have been made to this process in order to mitigate each one of these potential failures.

\vskip 10pt

Pages 29 and 30 describe a rather sad state of affairs in American chemical engineering curricula.  Describe what the Mary Kay O'Connor Process Safety Center discovered in their 2006 survey of 180 chemical engineering departments at universities around the United States.

\vskip 20pt \vbox{\hrule \hbox{\strut \vrule{} {\bf Suggestions for Socratic discussion} \vrule} \hrule}

\begin{itemize}
\item{} When combining the probability value of a single-point failure (such as any listed on page 24) with other PFD values for other components in a system to arrive at a total system PFD, do we use the AND function or the OR function?  Explain your answer.
\end{itemize}

\underbar{file i03290}
%(END_QUESTION)





%(BEGIN_ANSWER)


%(END_ANSWER)





%(BEGIN_NOTES)

The CSB's ``only credible cause'' for the accident was a failure in the cooling system (page 24).  The deceased operator is claimed to have reported a ``cooling problem'' prior to the accident (pages 9 and 24).

\vskip 10pt

The only overpressure protection device for the reactor was a rupture disk installed in the 4-inch line.  A pressure control valve venting hydrogen gas through a 1 inch line is the only other pathway for pressurized fluids to escape this reactor vessel.

\vskip 10pt

Two exothermic reactions were possible with the chemistry of this process.  The first at 350 degrees F was expected, and relatively mild in terms of its heat output.  The second kicked in at 390 degrees F and was much more energetic.  Based on the energy of the second exothermic reaction, the CSB determined that no overpressure protection device set to lift at 400 PSI or above could have prevented an overpressure event -- only a device lifting at a substantially lower pressure would have snubbed the pressure rise in time.  A rupture disk pressure of 75 PSIG (compared to the disk's burst rating of 400 PSIG, as specified on page 19) would have probably prevented the accident.

\vskip 10pt

``Running a system to failure'' means that things are run until they break, rather than to preventively perform maintenance.  This virtually guaranteed an accident given the fact that a failure in this reactor's cooling system could not be mitigated by the inadequate overpressure protection.

\vskip 10pt

Ways to mitigate single-point cooling system failures: 

\begin{itemize}
\item{} Install redundant supply valves
\item{} Have an emergency reservoir of cooling water on site
\item{} Redundant temperature sensors for reactor
\item{} Use fail-open cooling water valves to default to full cooling in the event of air failure
\item{} Have better overpressure protection!
\item{} Perform routine system testing to identify and correct problems before they become major
\end{itemize}

\vskip 10pt

According to the results of the Mary Kay survey (180 colleges and universities surveyed), only 11\% of them required process safety coursework as part of the engineering degree!  An additional 13\% offered safety coursework as elective credit.  The inclusion of reactivity hazard analysis was found to be ``limited at best'' in American chemical engineering curricula.  Neither the AIChE nor ABET required process safety or reactive hazard content in Chemical Engineering curricula!


\vfil \eject

\noindent
{\bf Prep Quiz:}

What was the contributing cause of the explosion at the T2 Laboratories processing facility?

\begin{itemize}
\item{} The rupture disk pipe size was too large, causing gas to vent too rapidly
\vskip 5pt 
\item{} Water accidently entered the reactor, reacting violently with the sodium metal
\vskip 5pt 
\item{} The reactor vessel was not rated for the normal operating pressure of the process
\vskip 5pt 
\item{} Inadequate cooling on the reactor vessel, and inadequate pressure relief capacity
\vskip 5pt 
\item{} An electrical sensor on the reactor caused a spark, igniting a vapor cloud
\vskip 5pt 
\item{} The DCS output card failed, shutting off the coolant valve prematurely
\end{itemize}



%INDEX% Reading assignment: USCSB report on the 2008 reactor explosion in Jacksonville, Florida

%(END_NOTES)


