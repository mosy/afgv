
%(BEGIN_QUESTION)
% Copyright 2008, Tony R. Kuphaldt, released under the Creative Commons Attribution License (v 1.0)
% This means you may do almost anything with this work of mine, so long as you give me proper credit

The process trend shown below reveals a controller's response to the process variable signal and the setpoint.  Based on what you see in this trend, determine whether the controller is direct or reverse acting, and also whether it implements a P-only, I-only, P+I, I+D, or P+D control algorithm.

$$\includegraphics[width=15.5cm]{i03307x01.eps}$$

\vskip 20pt \vbox{\hrule \hbox{\strut \vrule{} {\bf Suggestions for Socratic discussion} \vrule} \hrule}

\begin{itemize}
\item{} A useful problem-solving technique to apply here is working the problem {\it backwards}: ask youself what the output trend would look like for each action (P, I, D) and then see what the given output trend most resembles.
\item{} Re-draw the output trend if this controller implemented a full PID algorithm.
\end{itemize}

\underbar{file i03307}
%(END_QUESTION)





%(BEGIN_ANSWER)

This is a {\it direct-acting}, {\it proportional + derivative} controller.

%(END_ANSWER)





%(BEGIN_NOTES)

\vskip 20pt \vbox{\hrule \hbox{\strut \vrule{} {\bf Suggestions for Socratic discussion} \vrule} \hrule}

\begin{itemize}
\item{} Estimate the proportional band of this controller.  (PB = 200\% -- gain of 0.5)
\item{} Estimate the derivative time constant of this controller, assuming each horizontal division is one minute in length.  ($\tau_d \approx$ 1 minute assuming parallel PID equation, 2 minutes assuming Ideal equation)
\end{itemize}

%INDEX% Control: determining P, I, D from graph of controller response

%(END_NOTES)


