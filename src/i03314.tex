
%(BEGIN_QUESTION)
% Copyright 2015, Tony R. Kuphaldt, released under the Creative Commons Attribution License (v 1.0)
% This means you may do almost anything with this work of mine, so long as you give me proper credit

A current transformer with a C200 class rating and a 400:5 ratio is used to measure current through a power line where the maximum symmetrical fault current is 4500 amps and the relay burden is 1.0 ohm (resistive).  The CT's secondary winding resistance is 0.25 ohms.  Assume no DC transients in this power system and 500 feet of distance between the CT and the relay (1000 feet total wire length).

\vskip 10pt

Calculate the maximum amount of resistance for the wire connecting the CT to the relay not exceeding the CT's rated capability, and then choose the {\it smallest} copper wire gauge size satisfying this criterion:

\vskip 10pt

$R_{wire}$ (maximum) = \underbar{\hskip 50pt} $\Omega$

\vskip 10pt
\begin{itemize}
\item{} 6 AWG
\item{} 8 AWG
\item{} 10 AWG
\item{} 12 AWG
\item{} 14 AWG
\end{itemize}

\underbar{file i03313}
%(END_QUESTION)





%(BEGIN_ANSWER)

$R_{wire}$ (maximum) = \underbar{\bf 2.750} $\Omega$

\vskip 10pt

{\bf 14 AWG} is the minimum wire size necessary in this application.  However, 12 AWG is also accepted as a correct answer because some sources cite 12 AWG as the recommended minimum wire gauge for {\it any} CT application.

%(END_ANSWER)





%(BEGIN_NOTES)

{\bf This question is intended for exams only and not worksheets!}.

%(END_NOTES)


