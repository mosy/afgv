
%(BEGIN_QUESTION)
% Copyright 2014, Tony R. Kuphaldt, released under the Creative Commons Attribution License (v 1.0)
% This means you may do almost anything with this work of mine, so long as you give me proper credit

The following equation predicts the tripping time ($t$) for a time-overcurrent relay given the time-dial setting ($T$) and overcurrent value in multiples of pick-up current ($M$):

$$t = T \left(0.18 + {5.95 \over {M^2 - 1}} \right) \hskip 30pt \hbox{Inverse curve}$$

%$$t = T \left(0.0963 + {3.88 \over {M^2 - 1}} \right) \hskip 30pt \hbox{Very inverse curve}$$

%$$t = T \left(0.0352 + {5.67 \over {M^2 - 1}} \right) \hskip 30pt \hbox{Extremely inverse curve}$$

\noindent
Where,

$t$ = Trip time (seconds)

$T$ = Time Dial setting

$M$ = Multiples of pickup current

\vskip 20pt

\noindent
First, algebraically manipulate this equation to solve for $M$:

\vskip 10pt

$M = $

\vskip 20pt

\noindent
Next, calculate the amount of constant line current causing such a relay to trip in 2.5 seconds given a 800:5 CT ratio, a pick-up current setting of 4.5 amps, and a time-dial setting of 10:

\vskip 20pt

$I_{line}$ = \underbar{\hskip 50pt} amps

\vskip 10pt

\underbar{file i03322}
%(END_QUESTION)





%(BEGIN_ANSWER)

$$M = \sqrt{{5.95 \over {{t \over T} - 0.18}} + 1}$$

The overcurrent is 9.27 times pickup, which equates to 41.73 amps CT secondary current.  With an 800:5 ratio, this equates to a sustained overcurrent of {\bf 6.677 kA}.

%(END_ANSWER)





%(BEGIN_NOTES)

{\bf This question is intended for exams only and not worksheets!}.

%(END_NOTES)


