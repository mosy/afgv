
%(BEGIN_QUESTION)
% Copyright 2014, Tony R. Kuphaldt, released under the Creative Commons Attribution License (v 1.0)
% This means you may do almost anything with this work of mine, so long as you give me proper credit

Suppose two three-phase AC generators are ready to be synchronized and connected to the same bus.  One of them is outputting 485 volts (line), while the other is outputting 478 volts (line), both at the exact same frequency (60 Hz).  However, at the present moment they are 18$^{o}$ out of phase with each other.

\vskip 10pt

A set of synchronization lamps connected across the open circuit breaker contacts indicates the two generators' readiness to synchronize.  Calculate the voltage across each of the lamps in this out-of-phase condition, as well as the frequency of that lamp voltage.

\underbar{file i03328}
%(END_QUESTION)





%(BEGIN_ANSWER)

You can solve this by sketching two phasor diagrams (one for each generator), with an 18$^{o}$ shift between the two sets of phasors, and calculate the distance between any one pair of corresponding phasor tips.  With line voltages of 485 volts and 478 volts, the phase voltages will be $1 \over \sqrt{3}$ of those values, or 280 volts and 276 volts, respectively:

$$\includegraphics[width=15.5cm]{i03328x01.eps}$$

$$V_{lamp} = (280 \hbox{ V} \angle 0^o) - (276 \hbox{ V} \angle -18^o)$$

$$V_{lamp} = 87.1 \hbox{ V} \angle 78.4^o$$

Since lamps care not for phase angle, the voltage across each lamp will be 87.1 volts.  The frequency of this voltage across the lamp will be the same as the generators: 60 Hz.

%(END_ANSWER)





%(BEGIN_NOTES)


%INDEX% Electronics review, phasor expressions of circuit quantities

%(END_NOTES)


