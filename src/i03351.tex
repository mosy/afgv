
%(BEGIN_QUESTION)
% Copyright 2013, Tony R. Kuphaldt, released under the Creative Commons Attribution License (v 1.0)
% This means you may do almost anything with this work of mine, so long as you give me proper credit

Would it be appropriate to use a chemiluminescence analyzer to measure the concentration of oxygen gas in air?  Explain why or why not.

\underbar{file i03351}
%(END_QUESTION)





%(BEGIN_ANSWER)

No, because oxygen molecules do not chemiluminesce.  One should recall how chemiluminescense detectors work by reacting the molecule of interest with ozone gas.  Since ozone is simply O$_{3}$, it will in no way react with O$_{2}$ to produce light.

\vskip 10pt

I recommend granting half-credit for the yes/no answer, and half-credit for the explanation of why or why not.

\vskip 10pt

However . . . a really clever answer would be to supply the chemiluminescence detector with a steady stream of NO gas, then let the {\it sample} get ionized to produce ozone.  This way, the intensity of the chemiluminescence would be a function of oxygen concentration in the sample!  If anyone thinks of this solution, grant them +2 points extra credit!!

%(END_ANSWER)





%(BEGIN_NOTES)

{\bf This question is intended for exams only and not worksheets!}.

%(END_NOTES)


