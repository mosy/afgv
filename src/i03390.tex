
%(BEGIN_QUESTION)
% Copyright 2012, Tony R. Kuphaldt, released under the Creative Commons Attribution License (v 1.0)
% This means you may do almost anything with this work of mine, so long as you give me proper credit

Determine the proper mode to place a loop calibrator in ({\it measure}, {\it source}, or {\it simulate}) to monitor the signal sent to a control valve by a loop controller in a working process, and also calculate the amount of current needed to position the control valve at 81\% open, assuming a 4-20 mA range and a control valve that is air-to-close (with a regular direct-acting I/P transducer supplying air to the valve).

\vskip 10pt

Proper loop calibrator mode = \underbar{\hskip 50pt}

\vskip 10pt

Current = \underbar{\hskip 50pt} mA

\underbar{file i03390}
%(END_QUESTION)





%(BEGIN_ANSWER)

Proper loop calibrator mode = \underbar{\bf Measure}

Current = \underbar{\bf 7.04} mA

%(END_ANSWER)





%(BEGIN_NOTES)

{\bf This question is intended for exams only and not worksheets!}.

%(END_NOTES)


