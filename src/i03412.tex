
%(BEGIN_QUESTION)
% Copyright 2013, Tony R. Kuphaldt, released under the Creative Commons Attribution License (v 1.0)
% This means you may do almost anything with this work of mine, so long as you give me proper credit

A digital pH transmitter has a calibrated input range of 4 to 12 pH, and a 12-bit output (0 to 4095 ``count'' range).  Complete the following table of values for this transmitter, assuming perfect calibration (no error):

% No blank lines allowed between lines of an \halign structure!
% I use comments (%) instead, so that TeX doesn't choke.

$$\vbox{\offinterlineskip
\halign{\strut
\vrule \quad\hfil # \ \hfil & 
\vrule \quad\hfil # \ \hfil & 
\vrule \quad\hfil # \ \hfil & 
\vrule \quad\hfil # \ \hfil \vrule \cr
\noalign{\hrule}
%
% First row
Input pH & Percent of span & Counts & Counts \cr
%
% Another row
 & (\%) & (decimal) & (hexadecimal) \cr
%
\noalign{\hrule}
%
% Another row
 & 22 &  &  \cr
%
\noalign{\hrule}
%
% Another row
 & 73 &  &  \cr
%
\noalign{\hrule}
} % End of \halign 
}$$ % End of \vbox

\underbar{file i03412}
%(END_QUESTION)





%(BEGIN_ANSWER)

Full credit is given for having either of the alternative answers in each cell of the table (i.e. the student does not have to specify {\it both} count values shown in each cell!):

% No blank lines allowed between lines of an \halign structure!
% I use comments (%) instead, so that TeX doesn't choke.

$$\vbox{\offinterlineskip
\halign{\strut
\vrule \quad\hfil # \ \hfil & 
\vrule \quad\hfil # \ \hfil & 
\vrule \quad\hfil # \ \hfil & 
\vrule \quad\hfil # \ \hfil \vrule \cr
\noalign{\hrule}
%
% First row
Input pH & Percent of span & Counts & Counts \cr
%
% Another row
 & (\%) & (decimal) & (hexadecimal) \cr
%
\noalign{\hrule}
%
% Another row
5.76 & 22 & 900 or 901 & 384 or 385 \cr
%
\noalign{\hrule}
%
% Another row
9.84 & 73 & 2989 or 2990 & BAD or BAE \cr
%
\noalign{\hrule}
} % End of \halign 
}$$ % End of \vbox

%(END_ANSWER)





%(BEGIN_NOTES)

{\bf This question is intended for exams only and not worksheets!}.

%(END_NOTES)


