
%(BEGIN_QUESTION)
% Copyright 2010, Tony R. Kuphaldt, released under the Creative Commons Attribution License (v 1.0)
% This means you may do almost anything with this work of mine, so long as you give me proper credit

This is an electronic thermal conductivity sensor, used to detect different chemical components exiting the column of a chromatograph.  Both the measurement and reference filaments are heated by the electric current going through them, and a varying voltage will develop across each one as each is cooled differently by the gases:

$$\includegraphics[width=15.5cm]{i03416x01.eps}$$

Suppose a technician is setting up this detector for the very first time, and notes the voltmeter registers an abnormally large signal (in the polarity shown) even with the same test gas passed through each cell.

Identify the likelihood of each specified fault for this circuit.  Consider each fault one at a time (i.e. no coincidental faults), determining whether or not each fault could independently account for {\it all} measurements and symptoms in this circuit.

% No blank lines allowed between lines of an \halign structure!
% I use comments (%) instead, so that TeX doesn't choke.

$$\vbox{\offinterlineskip
\halign{\strut
\vrule \quad\hfil # \ \hfil & 
\vrule \quad\hfil # \ \hfil & 
\vrule \quad\hfil # \ \hfil \vrule \cr
\noalign{\hrule}
%
% First row
{\bf Fault} & {\bf Possible} & {\bf Impossible} \cr
%
\noalign{\hrule}
%
% Another row
$R_1$ resistance set too low &  &  \cr
%
\noalign{\hrule}
%
% Another row
$R_1$ resistance set too high &  &  \cr
%
\noalign{\hrule}
%
% Another row
$R_2$ resistance set too low &  &  \cr
%
\noalign{\hrule}
%
% Another row
$R_2$ resistance set too high &  &  \cr
%
\noalign{\hrule}
%
% Another row
$R_3$ resistance set too low &  &  \cr
%
\noalign{\hrule}
%
% Another row
$R_3$ resistance set too high &  &  \cr
%
\noalign{\hrule}
%
% Another row
Reference filament burned open &  &  \cr
%
\noalign{\hrule}
%
% Another row
Measurement filament burned open &  &  \cr
%
\noalign{\hrule}
%
% Another row
Voltmeter failed open &  &  \cr
%
\noalign{\hrule}
%
% Another row
Voltmeter failed shorted &  &  \cr
%
\noalign{\hrule}
} % End of \halign 
}$$ % End of \vbox


\underbar{file i03416}
%(END_QUESTION)





%(BEGIN_ANSWER)

% No blank lines allowed between lines of an \halign structure!
% I use comments (%) instead, so that TeX doesn't choke.

$$\vbox{\offinterlineskip
\halign{\strut
\vrule \quad\hfil # \ \hfil & 
\vrule \quad\hfil # \ \hfil & 
\vrule \quad\hfil # \ \hfil \vrule \cr
\noalign{\hrule}
%
% First row
{\bf Fault} & {\bf Possible} & {\bf Impossible} \cr
%
\noalign{\hrule}
%
% Another row
$R_1$ resistance set too low &  & $\surd$ \cr
%
\noalign{\hrule}
%
% Another row
$R_1$ resistance set too high &  & $\surd$ \cr
%
\noalign{\hrule}
%
% Another row
$R_2$ resistance set too low &  & $\surd$ \cr
%
\noalign{\hrule}
%
% Another row
$R_2$ resistance set too high & $\surd$ &  \cr
%
\noalign{\hrule}
%
% Another row
$R_3$ resistance set too low & $\surd$ &  \cr
%
\noalign{\hrule}
%
% Another row
$R_3$ resistance set too high &  & $\surd$ \cr
%
\noalign{\hrule}
%
% Another row
Reference filament burned open &  & $\surd$ \cr
%
\noalign{\hrule}
%
% Another row
Measurement filament burned open & $\surd$ &  \cr
%
\noalign{\hrule}
%
% Another row
Voltmeter failed open &  & $\surd$ \cr
%
\noalign{\hrule}
%
% Another row
Voltmeter failed shorted &  & $\surd$ \cr
%
\noalign{\hrule}
} % End of \halign 
}$$ % End of \vbox

%(END_ANSWER)





%(BEGIN_NOTES)

{\bf This question is intended for exams only and not worksheets!}.

%INDEX% Measurement, analytical: chromatography
%INDEX% Troubleshooting review: electric circuits

%(END_NOTES)


