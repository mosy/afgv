
%(BEGIN_QUESTION)
% Copyright 2011, Tony R. Kuphaldt, released under the Creative Commons Attribution License (v 1.0)
% This means you may do almost anything with this work of mine, so long as you give me proper credit

An instrumentation intern is asked to fabricate a {\it heat sink} for a power transistor in an industrial circuit.  The existing circuit keeps burning up the transistor, which had no heat sink attached at all in the original design.  The intern asks her fellow technicians for advice on how to fabricate one out of metal, the intern never having done this before.  

Technician ``A'' says the heat sink should have as large a surface area as practical, suggesting the intern cut one from the sheet aluminum of a soda pop can.  Technician ``B'' disagrees, saying the heat sink needs to be as massive as possible, suggesting the intern cut one from a short piece of 1/4 inch round metal bar.

\vskip 10pt

Recognizing that these two technicians are presenting contradicting answers, the intern must choose between the two heat sink forms: {\it thin and wide}, or {\it thick and heavy}.  Based on what you know about thermodynamics, which technician is giving the better advice to the intern?  Be sure to explain your answer!

\underbar{file i03437}
%(END_QUESTION)





%(BEGIN_ANSWER)

Technician ``A'' has the better solution, because with both radiative and convective heat transfer the goal is to optimize surface area.  Large mass will only slow down the temperature rise of the sink, not result in a lower final temperature!

\vskip 10pt

5 points for correct answer, 5 points for good explanation.

%(END_ANSWER)





%(BEGIN_NOTES)

{\bf This question is intended for exams only and not worksheets!}.

%(END_NOTES)

