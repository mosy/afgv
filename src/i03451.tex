
%(BEGIN_QUESTION)
% Copyright 2011, Tony R. Kuphaldt, released under the Creative Commons Attribution License (v 1.0)
% This means you may do almost anything with this work of mine, so long as you give me proper credit

A process vessel uses a displacer-type level transmitter to measure the amount of liquid toluene contained at the bottom, over a range of 0 to 20 inches.  The cylindrical displacer is 20 inches long and 2.5 inches in diameter.  Some years later, this same vessel is converted to hold acetone instead of toluene, but no one bothers to re-range the level transmitter for the new liquid.

Assuming a calibration based on the full length of the displacer (20 inches) for the wrong liquid (toluene), calculate the amount of acetone the transmitter ``thinks'' is in the vessel when the actual acetone level is 11.1 inches (verified by a sightglass).

\vskip 10pt

Level (perceived) = \underbar{\hskip 50pt} inches

\vskip 10pt

Also, determine the proper LRV and URV buoyant forces for dry-calibrating the transmitter, if it is to accurately measure the liquid height of acetone over the 20 inch range.  Express your answers in units of pounds:

\vskip 10pt

LRV = \underbar{\hskip 50pt} pounds \hskip 30pt URV = \underbar{\hskip 50pt} pounds

\underbar{file i03451}
%(END_QUESTION)





%(BEGIN_ANSWER)

Level (perceived) = \underbar{\bf 10.14} inches

\vskip 10pt

LRV = \underbar{\bf 0} pounds \hskip 30pt URV = \underbar{\bf 2.81} pounds

\vskip 10pt

4 points for perceived level calculation, 3 points for each range value.

%(END_ANSWER)





%(BEGIN_NOTES)

{\bf This question is intended for exams only and not worksheets!}.

%(END_NOTES)


