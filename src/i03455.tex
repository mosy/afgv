
%(BEGIN_QUESTION)
% Copyright 2011, Tony R. Kuphaldt, released under the Creative Commons Attribution License (v 1.0)
% This means you may do almost anything with this work of mine, so long as you give me proper credit

A salesperson for a capacitive level transmitter manufacturer is trying to tell your supervisor that these instruments are inherently immune to changes in process liquid density, unlike some traditional technologies for measuring liquid level.  The application being considered is one with a single nonconductive liquid beneath a vapor, in a vessel operating at atmospheric pressure.  You know this claim to be basically true, but you also know there is something more to this picture that the salesperson is not telling your boss.

\vskip 10pt

Explain how a capacitive level transmitter {\it can} be misled by changes in process operating conditions.  Be as specific as you can in your answer:

\vskip 50pt

\underbar{file i03455}
%(END_QUESTION)





%(BEGIN_ANSWER)

Nonconductive capacitive sensors use the process liquid as the dielectric, and as such rely on that liquid having a known and stable permittivity value for accurate indication.  Anything causing the process liquid's permittivity to change will cause inaccurate level measurement.  Thus, while the claim that changes in liquid density do not affect capacitive level transmitters is essentially true, it does not mean capacitive level transmitters are immune to any and all changes in process liquid characteristics.

%(END_ANSWER)





%(BEGIN_NOTES)

{\bf This question is intended for exams only and not worksheets!}.

%(END_NOTES)

