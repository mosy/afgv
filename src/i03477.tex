
%(BEGIN_QUESTION)
% Copyright 2010, Tony R. Kuphaldt, released under the Creative Commons Attribution License (v 1.0)
% This means you may do almost anything with this work of mine, so long as you give me proper credit

Suppose you are going to install a Rosemount model 1151 ``Alphaline'' (analog) differential pressure transmitter in a process, calibrated to a range of 0 to 100 inches W.C.  The transmitter's model number shows the following specifications:

\begin{itemize}
\item{} Model = {\it 1151DP}
\item{} Pressure range code = {\it 4}
\item{} Output code = {\it E}
\item{} Material code = {\it 22}
\end{itemize}

Answer the following questions regarding this transmitter as it applies to the application you intend to install it in:

\vskip 20pt

\item{} Does this transmitter have sufficient turndown (``rangeability'') for the application?  Show your calculation to prove whether it does or not.

\vskip 50pt

\item{} Calculate the expected accuracy for this transmitter once installed, expressed in $\pm$ inches of water column.

\vskip 50pt

\item{} Calculate the six-month calibration stability of this transmitter after installation, expressed in $\pm$ inches of water column.

\vskip 50pt

\item{} Calculate the amount of total measurement error this transmitter may exhibit given an ambient temperature shift of 65 degrees Fahrenheit, expressed in $\pm$ inches of water column.



\vfil 

\underbar{file i03477}
\eject
%(END_QUESTION)





%(BEGIN_ANSWER)

This is a graded question -- no answers or hints given!

%(END_ANSWER)





%(BEGIN_NOTES)

{\it Note: all performance data taken from pages 6-6 through 6-10 of the Rosemount manual (document 00809-0100-4360, Revision AA, copyright 1997).}

\vskip 10pt

The transmitter's {\bf 6:1} turndown ratio is more than adequate for the application.  Range code 4 gives an upper range limit (URL) of {\bf 150 "W.C.}, which when coupled with a turndown of 6:1 yields a minimum span of {\bf 25 "W.C.}, four times less than our intended span of 100 "W.C.

\vskip 10pt

Expected accuracy is $\pm$ 0.2\% of calibrated span, so the accuracy will be {\bf $\pm$ 0.2 "W.C.}

\vskip 10pt

Six-month stability is $\pm$ 0.2\% of URL, which is 150 "W.C. for this range code.  Thus, the six-month stability will be {\bf $\pm$ 0.3 "W.C.}

\vskip 10pt

Total error is $\pm$ (0.5\% of URL plus 0.5\% of calibrated span) per 100 degree Fahrenheit shift.  Thus, the total error for a 65 degree shift will be {\bf $\pm$ 0.8125 "W.C.}  It should be noted that the effect of temperature changes on pressure measurement (shown on page 6-7) is not the same as the effect of temperature on the LCD display's reading (shown on page 4-7).  This is a good lesson on the importance of {\it identifying context} when reading a technical document.  Just because you did a word-search and happened to find something that talks about temperature effect doesn't mean it's the same temperature effect you're interested in!  In this case, you had to read the previous page (4-6) to realize that the 0.01\% of range per degree C specified on page 4-7 isn't talking about pressure sensing at all, but rather the LCD unit option.

%INDEX% Calibration, stability: applied to pressure transmitter application
%INDEX% Calibration, turndown: applied to pressure transmitter application

%(END_NOTES)


