
%(BEGIN_QUESTION)
% Copyright 2012, Tony R. Kuphaldt, released under the Creative Commons Attribution License (v 1.0)
% This means you may do almost anything with this work of mine, so long as you give me proper credit

You may find the course structure and format of the INST courses to be quite different from what you have experienced elsewhere in your education.  For each of the following examples, discuss and explain the rationale.  What do you think is the greater purpose for each of these course attributes and policies?

\vskip 10pt

\begin{itemize}
\item{} Homework consists of studying new subjects prior to arriving to class for the theory sessions.  Students' primary source of new information is in the form of written materials: textbooks, reports, and manufacturer's literature.  Daily quizzes at the start of each class session hold students accountable for this preparatory learning.  {\it Why study new subjects outside of class, instead of doing normal homework that reviews subjects previously covered in class?  Why the strong emphasis on reading as a mode of learning?}
\vskip 20pt
\item{} Classroom sessions are not lecture-oriented.  Rather, classroom sessions place students in an active role discussing, questioning, and investigating what they're learned from their independent studies.  Learning new facts (knowledge) and how to interpret them (comprehension) is the students' responsibility, and it happens before class rather than during class.  Class time is devoted to higher-level thinking (application, analysis, synthesis, and evaluation).  {\it What's wrong with lecture, especially when the overwhelming majority of classes in the world are taught this way?}
\vskip 20pt
\item{} Students are expected to track their own academic progress using a computer spreadsheet to calculate their own course grades as they progress through each school quarter.  {\it Why not simply present the grades to students?}
\vskip 20pt
\item{} Students must explicitly apply ``sick hours'' to their absences (this is not automatically done by the instructor!), and seek donations from classmates if they exceed their allotment for a quarter.  {\it Why not simply allow a fixed number of permitted absence for each student, or let the instructor judge the merits of each student's absence on a case-by-case basis?}
\vskip 20pt
\item{} Mastery exams, where students must answer all questions with 100\% accuracy.  {\it What's wrong with regular exams, where a certain minimum percentage of correct answers is all that's necessary to pass?}
\vskip 20pt
\item{} Students may submit optional, ungraded assignments called ``feedback questions'' to the instructor at the end of most course sections in order to check their preparedness for the higher-level thinking challenges of the upcoming exam.  {\it Why in the world would anyone do work that doesn't contribute to their grade?}
\vskip 20pt
\item{} Troubleshooting exercises in lab and diagnostic questions in homework, where students must demonstrate sound reasoning in addition to properly identifying the problem(s).  {\it Isn't it enough that the student simply finds the fault?}
\vskip 20pt
\item{} Extra credit is offered for students wishing to improve their grades, but this extra credit is always in the form of practical and realistic work relevant to the specific course in which the extra credit is desired.  {\it Why doesn't unrelated work count?}
\end{itemize}

\underbar{file i03484}
%(END_QUESTION)





%(BEGIN_ANSWER)

The general philosophy of education in these courses may be summed up in a proverb: 

\vskip 10pt

{\bf ``Give a man a fish and you feed him for a day.  Give a man a fishing pole and you feed him for life.''}

\vskip 10pt

Instrumentation is a highly complex, fast-changing career field.  You will not survive, much less thrive, in this field if all you can ever learn is what someone directly teaches you.  In order to stay up-to-date with new technology, figure out solutions to novel problems, and adapt to a changing profession, you absolutely {\it must} possess independent learning ability.  You must be able to ``fish'' for new knowledge and understanding on your own.  These courses are designed to foster this higher-level skill.

%(END_ANSWER)





%(BEGIN_NOTES)


%INDEX% Course organization, philosophy: independent learning and self-responsibility

%(END_NOTES)

