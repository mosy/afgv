%(BEGIN_QUESTION)
% Copyright 2011, Tony R. Kuphaldt, released under the Creative Commons Attribution License (v 1.0)
% This means you may do almost anything with this work of mine, so long as you give me proper credit

Dette P\&ID viser en avgassforbrenningsovn (pipe) som brukes til å brenne giftige gasser på en sikker måte. Den høye temperaturen i gassflammen reduserer de giftige forbindelsene til relativt ufarlig vanndamp, karbondioksid, og oksider av svovel og nitrogen:

$$\includegraphics[width=15.5cm]{i0004rx01.eps}$$

Identifiser hvor måle-/styringssystemet utfører {\it integrasjon} for å ``totalisere'' (summere) en mengde over tid. Hvis mulig, identifiser realistiske måleenheter for de(n) integrerte verdien(e).

\vskip 10pt


\underbar{file i03507}
%(END_QUESTION)





%(BEGIN_ANSWER)

FIQ-38 er en totalteller for massestrøm, som integrerer signalet for {\it massestrømningsrate} fra mengdemåleren FT-38 for å komme frem til en total masse brenngass som er brent over tid. Realistiske måleenheter ville være kilogram eller pund (masse).

\vskip 10pt

For å gjøre det samme for utslipp fra pipen, måtte vi måle strømningsraten av eksos gjennom forbrenningsovnen, deretter multiplisere det strømningssignalet med hvert forurensende stoffs målte konsentrasjon (for å komme frem til strømningsraten for hvert forurensende stoff), og så integrere hver av disse gassstrømmene over tid for å beregne volumet av forurensning.

%(END_ANSWER)





%(BEGIN_NOTES)

\vskip 20pt \vbox{\hrule \hbox{\strut \vrule{} {\bf Virtuell feilsøking} \vrule} \hrule}

Dette spørsmålet er en god kandidat for en øvelse i ``Virtuell feilsøking''. Når du presenterer diagrammet for studentene, ser du først for deg en bestemt feil i systemet. Deretter presenterer du ett eller flere symptomer på den feilen (noe som kan merkes av en operatør eller annen bruker av systemet). Studentene foreslår så ulike diagnostiske tester som kan utføres på systemet for å identifisere feilens art og plassering, som om de var teknikere som prøvde å feilsøke problemet. Din jobb er å fortelle dem hva resultatet/resultatene ville bli for hver av de foreslåtte testene, og dokumentere disse resultatene slik at alle studentene kan se dem.

Under og etter øvelsen er det lurt å stille studentene oppfølgingsspørsmål som:

\begin{itemize}
\item{} Hva forteller resultatet av den siste diagnostiske testen deg om feilen?
\item{} Anta at resultatene av den siste testen var annerledes. Hva ville det resultatet da fortalt deg om feilen?
\item{} Er den siste diagnostiske testen den beste vi kunne gjort?
\item{} Hva ville være den ideelle rekkefølgen av tester for å diagnostisere problemet i så få trinn som mulig?
\end{itemize}

%INDEX% Basics, control loop troubleshooting (realistic P&ID shown)
%INDEX% Control, strategies: cascade (realistic P&ID shown)
%INDEX% Fieldbus, HART: multivariable instrument (realistic P&ID shown)
%INDEX% Process: incinerator (realistic P&ID shown)

%(END_NOTES)
