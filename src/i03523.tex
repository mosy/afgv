
%(BEGIN_QUESTION)
% Copyright 2008, Tony R. Kuphaldt, released under the Creative Commons Attribution License (v 1.0)
% This means you may do almost anything with this work of mine, so long as you give me proper credit

A student attempts to manipulate the equation $x = {{a-b} \over {b-c}}$ to solve for $b$, but makes a mistake somewhere along the way:

\vskip 10pt

$$x = {{a-b} \over {b-c}}$$

\vskip 10pt

$$x(b - c) = a - b$$

\vskip 10pt

$$bx - cx = a - b$$

\vskip 10pt

$$bx - b = a - cx$$

\vskip 10pt

$$b(x - 1) = a - cx$$

\vskip 10pt

$$b = {{a - cx} \over {x - 1}}$$

\vskip 10pt

Identify which step the student made their mistake, then write the correct solution for $b$.

\vfil 

\underbar{file i03523}
\eject
%(END_QUESTION)





%(BEGIN_ANSWER)

This is a graded question -- no answers or hints given!

%(END_ANSWER)





%(BEGIN_NOTES)


Instead of writing $bx - b = a - cx$, the student should have written $bx + b = a + cx$.  The correct answer is:

$$b = {{a + cx} \over {x + 1}}$$

%INDEX% Mathematics review: detecting and correcting errors in literal equations

%(END_NOTES)


