
%(BEGIN_QUESTION)
% Copyright 2013, Tony R. Kuphaldt, released under the Creative Commons Attribution License (v 1.0)
% This means you may do almost anything with this work of mine, so long as you give me proper credit

Suppose an apprentice instrument technician goes up to a working Siemens model 353 loop controller and loads FCO101 into its configuration.  Suddenly, the controller stops working properly!  Explain {\it in detail} what may have changed within the controller with the loading of FCO101 to make it inoperational.  You will need to give multiple examples of controller parameters possibly altered by loading FCO101, based on the exercise we did in class using the Siemens 353 controllers.

\vskip 100pt

\underbar{file i03534}
%(END_QUESTION)





%(BEGIN_ANSWER)

5 points for each parameter possibly altered by loading FCO101.  A list of possible parameters include (but are not limited to):

\begin{itemize}
\item{} Input range scaling (FCO101 sets range to 0-100\%)
\item{} Input and output channels
\item{} Engineering units for PV/SP (FCO101 is ``PRCT'')
\item{} Loop tag name (FCO101 is ``Loop01'')
\item{} Alarm values and functionality
\item{} PID tuning (FCO101 sets gain to 1, reset and rate to minimums)
\item{} Controller action (FCO101 sets to reverse)
\item{} Controller program (FCO101 is a single-loop controller with analog I/O)
\item{} Enabling of any special features (e.g. Auto-tune, quick-set)
\end{itemize}

%(END_ANSWER)





%(BEGIN_NOTES)

{\bf This question is intended for exams only and not worksheets!}

%(END_NOTES)


