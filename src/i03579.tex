
%(BEGIN_QUESTION)
% Copyright 2011, Tony R. Kuphaldt, released under the Creative Commons Attribution License (v 1.0)
% This means you may do almost anything with this work of mine, so long as you give me proper credit

A common mistake among novice PLC/HMI programmers when configuring separate ``Start'' and ``Stop'' pushbutton objects in an HMI (for a PLC-controlled motor control system also containing hard-wired Start and Stop pushbuttons) is to configure the HMI pushbutton objects for {\it toggle} operation rather than {\it momentary} operation.   Explain why it is a mistake to have a ``toggling'' Start pushbutton object as well as a ``toggling'' Stop pushbutton object in an HMI display for a motor control.

\underbar{file i03579}
%(END_QUESTION)





%(BEGIN_ANSWER)

If the HMI pushbutton objects toggle, it will require {\bf two} pushes of each ``button'' to return the system to its normal latched state rather than just one.

%(END_ANSWER)





%(BEGIN_NOTES)

{\bf This question is intended for exams only and not worksheets!}.

%(END_NOTES)

