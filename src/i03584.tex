
%(BEGIN_QUESTION)
% Copyright 2011, Tony R. Kuphaldt, released under the Creative Commons Attribution License (v 1.0)
% This means you may do almost anything with this work of mine, so long as you give me proper credit

A common mistake among novice PLC/HMI programmers is to configure an HMI ``pushbutton'' object to write to one of the input register bits in a PLC's memory (e.g. write to bit {\tt X1} in a Koyo, write to bit {\tt I:0/0} in an Allen-Bradley, write to bit {\tt I0.0} in a Siemens).  Explain why this is a mistake.

\underbar{file i03584}
%(END_QUESTION)





%(BEGIN_ANSWER)

When the HMI attempts to write to an input register bit inside the PLC, that bit will become over-written by information from the real, hard-wired inputs at the beginning of each scan cycle.  The HMI's write operation(s) will therefore be in vain.

%(END_ANSWER)





%(BEGIN_NOTES)

{\bf This question is intended for exams only and not worksheets!}.

%(END_NOTES)

