
%(BEGIN_QUESTION)
% Copyright 2011, Tony R. Kuphaldt, released under the Creative Commons Attribution License (v 1.0)
% This means you may do almost anything with this work of mine, so long as you give me proper credit

Read and outline Loop Problem Signatures \#3 (``Process Dynamics -- Process Gain'') from Michael Brown's collection of control loop optimization tutorials.  Prepare to thoughtfully discuss with your instructor and classmates the concepts and examples explored in this reading, and answer the following questions:

\begin{itemize}
\item{} According to Mr. Brown, why do simple algorithmic tuning procedures such as those outlined by Ziegler and Nichols often fail to produce good results?
\vskip 10pt
\item{} In this tutorial, Mr. Brown states, ``[a] disadvantage of oversized valves is that the process gain actually amplifies the valve imperfections, and results in a degradation of control performance.''  Explain how an oversized control valve ``amplifies'' its mechanical imperfections.
\vskip 10pt
\item{} Examine the trends shown in Figure 2 (taken on a flow control loop at a South African oil refinery).  How is it evident that the control valve is oversized from this trend display?
\vskip 10pt
\item{} How did the controller on that flow loop have to be tuned in order to achieve good control with the hugely oversized valve?
\vskip 10pt
\item{} How does Mr. Brown define ``process gain'' for an integrating process that does not self-regulate following a step-change in valve position?
\end{itemize}

\vskip 20pt \vbox{\hrule \hbox{\strut \vrule{} {\bf Suggestions for Socratic discussion} \vrule} \hrule}

\begin{itemize}
\item{} Michael Brown uses the abbreviation ``PD'' to show the output of loop controllers in many of the trend graphs found within his articles.  Identify what ``PD'' means.
\item{} In this article, Michael Brown makes a mathematical error converting proportional band into gain.  Identify and correct this error!
\item{} Michael Brown's definition of process gain for an integrating process looks a lot like Ziegler and Nichols' {\it reaction rate}.  Identify and explain the similarities between the two.
\end{itemize}

\underbar{file i03593}
%(END_QUESTION)





%(BEGIN_ANSWER)


%(END_ANSWER)





%(BEGIN_NOTES)

Loop optimization begins with a thorough understanding of the process's dynamics: how it behaves to changes.  Quantitative techniques such as Ziegler-Nichols fail to account for some important process dynamics, which is one reason they often fail to give good results.

\vskip 10pt

Process gain amplifies valve imperfections!  If a valve is sticky, that stickiness will result in more error with a high process gain than with a low process gain.  Process gain should lie between 0.5 and 2.0.

\vskip 10pt

Figure 2 shows a hugely oversized control valve: just a few percent of valve stem motion resulted in large PV changes!  Good control was achievable only by turning the controller gain all the way down to 0.0125 (PB = 8000\%).  Mr. Brown made a mistake converting 8000\% into a gain: he said the gain was 0.000125.  Fortunately, the amount of hysteresis in this control valve was small enough to permit good control with tiny stem motions.

\vskip 10pt

Mr. Brown defines process gain for integrating processes as a rate of change (seconds$^{-1}$).  This is basically the same concept as the time constant of an integrator: the amount of time required for the integrator to repeat the input magnitude.





%INDEX% Reading assignment: Michael Brown Loop Problem Signature #3, "Process dynamics -- process gain"

%(END_NOTES)


