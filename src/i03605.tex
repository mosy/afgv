
%(BEGIN_QUESTION)
% Copyright 2011, Tony R. Kuphaldt, released under the Creative Commons Attribution License (v 1.0)
% This means you may do almost anything with this work of mine, so long as you give me proper credit

Read selected portions of the Siemens ``SIMATIC S7-200 Programmable Controller System Manual'' (document A5E00307987-04, August 2008) and answer the following questions:

\vskip 10pt

Locate the section discussing the PLC's {\it scan cycle} and describe the sequence of operations conducted by the PLC on an ongoing basis.

\vskip 10pt

Locate the section discussing the PLC's memory types (``Permanent Memory'' versus ``Retentive Data Memory'' and such), and describe the functions of each. 

\vskip 30pt

A very important aspect to learn about any PLC is how to specify various locations within its memory.  Each manufacturer and model of PLC has its own way of ``addressing'' memory locations, analogous to the different ways each postal system within each country of the world specifies its mailing addresses.  Locate the section of the manual discussing addressing conventions (``Accessing the Data of the S7-200''), and then answer these questions:

\vskip 10pt

Identify the proper address notation for a particular bit in the Siemens PLC's memory: bit number {\it 4} of byte {\it 1} within the {\it process-image input register}.

\vskip 10pt

Identify the proper address notation for a particular bit in the Siemens PLC's memory: bit number {\it 2} of byte {\it 0} within the {\it process-image output register}.

\vskip 10pt

Identify the proper address notation for a ``double word'' of data in the Siemens PLC's memory beginning at byte {\it 105} within the {\it variable memory area}.  How many bits are contained in a double word?

\underbar{file i03605}
%(END_QUESTION)





%(BEGIN_ANSWER)

Input register, byte 1, bit 4: {\tt I1.4}

\vskip 10pt

Output register, byte 0, bit 2: {\tt Q0.2}

\vskip 10pt

Variable memory double word, starting at byte 105: {\tt VD105} (a double-word consisting of 4 bytes, or 32 bits)

%(END_ANSWER)





%(BEGIN_NOTES)

The scan cycle is discussed on page 24 (execution of the control program may be halted by placing the PLC into ``Stop'' mode):

\begin{itemize}
\item{} Read discrete inputs (analog inputs will be read on scan cycle if filtering enabled, otherwise will be read immediately within program execution)
\item{} Execute control program
\item{} Communication processing
\item{} Self-diagnostic checks
\item{} Write discrete outputs (analog outputs written immediately within program execution)
\end{itemize}

A generic scan is shown in graphical form on page 24.  A much more detailed diagram of a PLC's scan is shown on page 26.  Detailed explanations of these steps are found on pages 25 through 27.

\vskip 10pt

Retentive Data Memory used to retain program parameters through a power cycle.  Parameters such as Timer and Counter values may be made retentive.  Permanent Memory is also retained through a power cycle, and stores such data as the user's program, system values, etc.  Interestingly, a memory cartridge may be installed in the S7-200 PLC to store such things as documentation files (e.g. Adobe .pdf documents!).

\vskip 10pt

The section titled ``Accessing the Data of the S7-200'' begins on page 27 and ends on page 36.  Memory areas include:

\begin{itemize}
\item{} {\tt I} = input register
\item{} {\tt Q} = output register
\item{} {\tt V} = variable memory
\item{} {\tt M} = bit memory
\item{} {\tt T} = timer memory
\item{} {\tt C} = counter memory
\item{} {\tt HC} = high-speed counter memory
\item{} {\tt AC} = accumulator memory
\item{} {\tt SM} = special memory
\item{} {\tt L} = local memory (similar to {\tt V} memory, but with local rather than global scope)
\item{} {\tt AI} = analog input register
\item{} {\tt AQ} = analog output register
\item{} {\tt S} = sequence control memory
\end{itemize}

\vskip 10pt

A special pair of registers inside the S7-200 PLC (SMB31 and SMW32) can instruct the PLC to save one value stored in V memory to Permanent memory (the same place where the program is stored).

\vskip 10pt

Input register, byte 1, bit 4: {\tt I1.4}

\vskip 10pt

Output register, byte 0, bit 2: {\tt Q0.2}

\vskip 10pt

Variable memory double word, starting at byte 105: {\tt VD105} (a double-word consisting of 4 bytes, or 32 bits)








\vskip 20pt \vbox{\hrule \hbox{\strut \vrule{} {\bf Suggestions for Socratic discussion} \vrule} \hrule}

\begin{itemize}
\item{} Page 26 of this manual shows a timing diagram for a typical scan cycle.  Examine this diagram and identify some of the events taking place in this particular scan.  
\item{} Explain what an {\it interrupt} event is according to the timing diagram shown on page 26.
\item{} How many bits comprise a {\it byte} in the S7-200 PLC's memory?
\item{} How many bits comprise a {\it word} in the S7-200 PLC's memory?
\item{} How many bits comprise a {\it double word} in the S7-200 PLC's memory?
\item{} Identify the proper address notation for a ``word'' of data in the Siemens PLC's memory beginning at byte {\it 81} within the {\it variable memory area}.  {\tt VW81}
\item{} Identify the proper address notation for a particular bit in the Siemens PLC's memory: bit number {\it 5} of byte {\it 17} within the {\it variable memory area}.  {\tt V17.5}
\item{} Identify the proper address notation for a particular bit in the Siemens PLC's memory: the output bit of timer instruction number {\it 2}.  {\tt T2}
\item{} Identify the proper address notation for a ``word'' of data in the Siemens PLC's memory: the current value of timer instruction number {\it 44}.  {\tt T44}
\item{} Identify the proper address notation for a particular bit in the Siemens PLC's memory: the output bit of counter instruction number {\it 8}.  {\tt C8}
\item{} Identify the proper address notation for a ``word'' of data in the Siemens PLC's memory: the current value of counter instruction number {\it 12}.  {\tt C12}
\item{} Identify the proper address notation for a ``word'' of data in the Siemens PLC's memory beginning at byte {\it 2} within the {\it special memory area}.  {\tt SMB2}
\item{} Identify the proper address notation for a particular bit in the Siemens PLC's memory: bit number {\it 6} of byte {\it 65} within the {\it special memory area}.  {\tt SM65.6}
\end{itemize}

%INDEX% Reading assignment: Siemens S7-200 system manual (scan cycle, addressing I/O)

%(END_NOTES)


