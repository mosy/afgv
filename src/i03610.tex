
%(BEGIN_QUESTION)
% Copyright 2014, Tony R. Kuphaldt, released under the Creative Commons Attribution License (v 1.0)
% This means you may do almost anything with this work of mine, so long as you give me proper credit

Read and outline the ``Phase Diagrams and Critical Points'' subsection of the ``Elementary Thermodynamics'' section of the ``Physics'' chapter in your {\it Lessons In Industrial Instrumentation} textbook.  Note the page numbers where important illustrations, photographs, equations, tables, and other relevant details are found.  Prepare to thoughtfully discuss with your instructor and classmates the concepts and examples explored in this reading.

\underbar{file i03610}
%(END_QUESTION)





%(BEGIN_ANSWER)


%(END_ANSWER)





%(BEGIN_NOTES)

{\it Phase diagrams} relate pressure, temperature, and substance phase in one graph.  Each line (or curve) on a phase diagram represents a boundary between two different phases of that substance.  Every point falling on a line represents a coordinate temperature and pressure for that substance's phase change.  The point of intersection is called the {\it triple point} of the substance, where solid, liquid, and vapor phases can coexist.

\vskip 10pt

The {\it critical temperature} of a substance is that temperature above which the substance can only exist as a gas (vapor).  No amount of pressure can liquify or solidify a substance above its critical temperature.  Oxygen is a good example of critical temperature, as it cannot remain in liquid or solid form at ambient temperature.

\vskip 10pt

The {\it critical pressure} of a substance is that pressure below which the substance cannot exist as a liquid in stable form.  No temperature change can liquify a solid or gas below its critical pressure.  The critical pressure is the same as the pressure at the substance's triple point.  Carbon dioxide is a good example of critical pressure, as it cannot remain in liquid form at atmospheric pressure, no matter what its temperature.

\vskip 10pt

The slope of the solid/liquid line in a phase diagram reveals how its density changes with temperature: a right-leaning slope represents a substance which expands as it melts and contracts when it freezes.  A left-leaning slope (e.g. water) represents a substance which contracts as it melts and expands when it freezes.






\vskip 20pt \vbox{\hrule \hbox{\strut \vrule{} {\bf Suggestions for Socratic discussion} \vrule} \hrule}

\begin{itemize}
\item{} Consult a phase diagram and explain how to interpret the lines and curves shown.
\item{} Consult a phase diagram for water and identify the point representing the state of water at room temperature and atmospheric pressure.
\item{} Consult a phase diagram for water and identify the point representing the state of water in a boiling tea kettle.
\item{} Consult a phase diagram for water and identify the point representing the state of water in a paper cup set in the middle of a roaring campfire.
\item{} Consult a phase diagram for water and identify the point representing the state of water in a pressure cooker operating at 15 PSIG.
\item{} Consult a phase diagram for water and identify the point representing the state of ice cubes in a domestic freezer.
\item{} Consult a phase diagram for water and identify the point representing a {\it supercritical} boiler at a modern power plant.
\item{} Explain this sentence from the textbook: ``[T]he whole concept of a singular boiling {\it point} for water becomes quaint in the light of a phase diagram''.
\item{} Explain what the ``triple point'' of a substance is.
\item{} Explain what the ``critical temperature'' of a substance is.
\item{} Explain what the ``critical pressure'' of a substance is.
\item{} Consult a phase diagram for water and sketch two sets of temperature changes on that diagram, then ask which temperature change represents the greatest change in enthalpy.
\end{itemize}


%INDEX% Reading assignment: Lessons In Industrial Instrumentation, Elementary Thermodynamics (phase diagrams and critical points)

%(END_NOTES)


