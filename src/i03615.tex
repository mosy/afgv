
%(BEGIN_QUESTION)
% Copyright 2010, Tony R. Kuphaldt, released under the Creative Commons Attribution License (v 1.0)
% This means you may do almost anything with this work of mine, so long as you give me proper credit

Examine the bulletin for the Fisher ``Wizard'' model 4196 pneumatic temperature indicating controller (Bulletin 34.6:4196, February 1997), especially the diagrams on page 5 showing the proportional-only control option, and answer the following questions about this controller:

\vskip 50pt

\begin{itemize}
\item{} Is this controller's mechanism {\it force-balance} or is it {\it motion-balance}?
\vskip 70pt
\item{} How is ``bumpless'' transfer between automatic and manual modes achieved?
\vskip 70pt
\item{} The model 4196S is called a ``Differential Gap'' controller.  Explain what this means, and how the model 4196S controller's behavior differs from that of a normal proportional-only 4196 controller.
\end{itemize}

\vfil

\underbar{file i03615}
\eject
%(END_QUESTION)





%(BEGIN_ANSWER)

This is a graded question -- no answers or hints given!

%(END_ANSWER)





%(BEGIN_NOTES)

\begin{itemize}
\item{} Is this controller's mechanism {\it force-balance} or is it {\it motion-balance}?  {\bf This is a motion-balance mechanism, reminiscent of the Fisher 3582 valve positioner.  The illustration at the top of page 5 shows a ``D-ring'' mechanism where error (PV versus SP) is compared by motion to the output (proportional + reset bellows).}
\vskip 10pt
\item{} How is ``bumpless'' transfer between automatic and manual modes achieved?  {\bf By a ball-in-tube balance indicator between the automatic and manual signal pressure tubes, indicating a ``null'' condition when it is safe to make the mode switch.  This is described on page 8.}
\vskip 10pt
\item{} The model 4196S is called a ``Differential Gap'' controller.  Explain what this means, and how the model 4196S controller's behavior differs from that of a normal proportional-only 4196 controller.  {\bf ``Differential Gap'' is just a synonym for ``on-off'' or ``bang-bang'' control action.  In this model, the pneumatic mechanism uses positive feedback to saturate the output signal at either full supply or zero pressure, as described on page 7.}
\end{itemize}

%INDEX% Control, proportional: pneumatic motion-balance controller

%(END_NOTES)


