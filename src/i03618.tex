
%(BEGIN_QUESTION)
% Copyright 2008, Tony R. Kuphaldt, released under the Creative Commons Attribution License (v 1.0)
% This means you may do almost anything with this work of mine, so long as you give me proper credit

The resolution of an analog-to-digital or digital-to-analog converter circuit is a function of how many bits of binary data it inputs or outputs.  For example, an 8-bit converter resolves the signal into 256 discrete states (counts).  The formula for determining the number of counts ($M$) from the number of binary bits ($n$) is as follows:

$$M = 2^n$$

In order to calculate the resultion of a converter circuit in units of {\it volts} (i.e. how many volts of analog signal each digital ``count'' value represents), we divide the analog span by the number of digital counts less one:

$$\hbox{Resolution} = {\hbox{Span} \over {\hbox{Counts} - 1}}$$

\vskip 10pt

Based on this knowledge, write a single mathematical formula solving for the resolution of an analog/digital converter circuit given the analog span and number of bits.  Use only the following variables in your formula:

\vskip 10pt

$V_S$ = Analog span (volts)

$V_R$ = Resolution (volts)

$n$ = Number of digital bits

\vskip 100pt

Next, manipulate this formula to solve for the number of bits needed given a specified resolution and span.

\vfil

\underbar{file i03618}
\eject
%(END_QUESTION)





%(BEGIN_ANSWER)

This is a graded question -- no answers or hints given!
 
%(END_ANSWER)





%(BEGIN_NOTES)

$$V_R = {V_S \over 2^n - 1}$$

\vskip 10pt

Manipulating the equation to solve for $n$:

$$2^n - 1 = {V_S \over V_R}$$

$$2^n = {V_S \over V_R} + 1$$

In order to ``un-do'' this formula to solve for $n$, we must apply a logarithm function:

$$\log 2^n = \log {V_S \over V_R} + 1$$

$$n \log 2 = \log {V_S \over V_R} + 1$$

$$n = {{\log ({V_S \over V_R} + 1)} \over {\log 2}}$$

%INDEX% Mathematics review: powers and logarithms

%(END_NOTES)


