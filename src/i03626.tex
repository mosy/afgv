
%(BEGIN_QUESTION)
% Copyright 2015, Tony R. Kuphaldt, released under the Creative Commons Attribution License (v 1.0)
% This means you may do almost anything with this work of mine, so long as you give me proper credit

Suppose an instrument salesperson comes to your shop and tells you his company's radar level transmitter product is superior to all hydrostatic and displacer level transmitters because those instruments' accuracy depends on a fixed process liquid density, whereas radar transmitters do not.  Thus, he tells you, his radar transmitters will give accurate level measurements even when process pressures and temperatures change.

What do you think of this claim?  Is the salesperson's claim true, or not?  Explain.

\vskip 10pt

If a magnetostrictive level transmitter salesperson made a comparable claim -- that the accuracy of a magnetostrictive instrument would not be affected by changes in process liquid density -- would you believe it?

\underbar{file i03626}
%(END_QUESTION)





%(BEGIN_ANSWER)

{\it Caveat emptor!} (Latin for ``buyer beware'')

\vskip 10pt

Radar level transmitters {\it are} definitely affected by certain properties of the gas or vapor above the liquid surface.  I'll let you identify what those properties are!

%(END_ANSWER)





%(BEGIN_NOTES)

Each of the salespersons' claims is true, but only for some conditions.  It is true that radar instruments do not infer liquid level from hydrostatic pressure or buoyant force, and as such are not strictly dependent on liquid density for accurate measurement.  However, the one variable radar instruments do depend on is {\it permittivity} of the substance through which the radar pulse must travel, and this permittivity will be affected by changes in density of the substance.

The dielectric permittivity of the gas or vapor through which the radar pulse travels through {\it will} change with changes in pressure and with changes in temperature, because both pressure and temperature changes will cause gases and vapors to change density.  Basically, it means packing fewer or more gas molecules in a given volume, which affects the bulk permittivity of the gas.  This phenomenon is commonly referred to as the {\it gas phase effect}.  The following formula predicts the change in gas permittivity resulting from changes in gas pressure and temperature:

$$\epsilon_r = 1 + (\epsilon_{ref} - 1) {P T_{ref} \over P_{ref} T}$$

\noindent
Where,

$\epsilon_r$ = Relative permittivity of a gas at a given pressure ($P$) and temperature ($T$)

$\epsilon_{ref}$ = Relative permittivity of the same gas at standard pressure ($P_{ref}$) and temperature ($T_{ref}$)

$P$ = Absolute pressure of gas (bars)

$P_{ref}$ = Absolute pressure of gas under standard conditions ($\approx$ 1 bar)

$T$ = Absolute temperature of gas (Kelvin)

$T_{ref}$ = Absolute temperature of gas under standard conditions ($\approx$ 273 K)

\vskip 10pt

Temperature changes also affect the permittivity of liquids, which will affect the accuracy of interface level measurement (where the radar signal must travel twice through the upper liquid layer).  For example, the dielectric permittivity of water near room temperature is about 80, whereas the permittivity of water at its boiling point (100$^{o}$ C) is only about 48.

\vskip 20pt

Magnetostrictive level instruments are far less affected by changes in liquid density, since the calibration depends solely on the speed of sound through the metal waveguide rod.  However, the float riding along the waveguide must still be buoyant, and buoyancy does depend on liquid density!  In all honesty, this is less of a concern because the float should still float so long as its average density is less than the liquid it is sensing.  In other words, the liquid density would have to experience a {\it huge} density change before a magnetostrictive sensor would fail to register the proper level.

%INDEX% Measurement, level: radar

%(END_NOTES)


