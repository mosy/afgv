
%(BEGIN_QUESTION)
% Copyright 2008, Tony R. Kuphaldt, released under the Creative Commons Attribution License (v 1.0)
% This means you may do almost anything with this work of mine, so long as you give me proper credit

Build your own thermocouple by taking a piece of thermocouple cable (type J or K works well) and twisting together the wires at one end to form a measurement junction.  Clip the test leads of a sensitive milli-voltmeter to the other wire ends (the reference junction).  Heat the measurement junction using body heat or an open flame (e.g. butane lighter) and then use a thermocouple reference table to infer the temperature of the measurement junction.

\vskip 10pt

A helpful feature of high-end digital multimeters is {\it high-resolution mode}, where the meter displays an extra digit on the screen at the expense of slower update times.  This extra resolution will help you better read the small millivoltage signals output by your thermocouple.  

Another helpful feature of most DMMs is {\it min/max mode}.  Try engaging this mode on your multimeter and see how it might be helpful to you reading minimum and maximum millivoltage values, as well as potentially a troubleshooting tool!

\vskip 20pt \vbox{\hrule \hbox{\strut \vrule{} {\bf Suggestions for Socratic discussion} \vrule} \hrule}

\begin{itemize}
\item{} Why is it important to know the ambient (room) temperature when calculating the temperature of the heated measurement junction?
\item{} Based on what you know of {\it measurement} versus {\it reference} junctions in thermocouple circuits, explain the millivoltage measurement you obtain with your meter when the thermocouple is at ambient (room) temperature.
\item{} What would happen if you touched the thermocouple's measurement junction to an ice cube instead of a flame?
\item{} What would happen if you warmed up the reference junction instead of the measurement junction?
\item{} Research what a {\it thermopile} is, and try building one with thermocouple wire.  Demonstrate with a heat source how a thermopile's operation differs from that of a regular thermcouple's.
\end{itemize}

\underbar{file i03629}
%(END_QUESTION)





%(BEGIN_ANSWER)


%(END_ANSWER)





%(BEGIN_NOTES)

%INDEX% Electronics review: DMM (test equipment) -- using high-resolution mode
%INDEX% Electronics review: DMM (test equipment) -- using min/max capture mode
%INDEX% Measurement, temperature: thermocouple

%(END_NOTES)


