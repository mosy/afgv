
%(BEGIN_QUESTION)
% Copyright 2011, Tony R. Kuphaldt, released under the Creative Commons Attribution License (v 1.0)
% This means you may do almost anything with this work of mine, so long as you give me proper credit

En veldig vanlig kontrollstrategi for store dampkjeler er såkalt {\it tre-element} (three-element) regulering av vannstanden i dampsylinderen (trommelen). Følgende illustrasjon viser systemet:

$$\includegraphics[width=15.5cm]{i03632x01.eps}$$

Forklar formålet med denne komplekse kontrollstrategien. Hvorfor er det ikke godt nok å bare la en enkel regulator måle vannstanden i trommelen (LT) og justere kontrollventilen for fødevannet (feedwater) deretter? Hvilket problem spesielt er denne strategien designet for å motvirke?

\vfil 

\underbar{file i03632}
\eject
%(END_QUESTION)





%(BEGIN_ANSWER)

Tre-element kjelestyring er designet for å overvinne problemet med "svelle og krympe" (shrink and swell) i damptrommelen, så vel som å motvirke svingninger i fødevannstrykket. Ved plutselig økning i damputtaket vil trykket i trommelen falle, noe som fører til at vannbobler (dampbobler under vannoverflaten) utvider seg og får vannstanden til å stige midlertidig ("svelle") selv om kjelen faktisk mister total vannmasse. En enkel nivåregulator ville reagert med å stenge ventilen for å senke nivået, noe som er feil reaksjon ettersom vi faktisk trenger mer vann.

Tre-element strategien bruker massebalanse (Dampstrøm vs. Vannstrøm) som primær styring for vannventilen, med nivået i trommelen som en korreksjon ("master") for å holde nivået over tid. Dermed, hvis dampstrømmen øker, vil regulatoren umiddelbart øke vannstrømmen tilsvarende, uavhengig av hva nivået gjør midlertidig.

%(END_ANSWER)





%(BEGIN_NOTES)

Dette er klassisk kaskaderegulering med foroverkobling (feedforward). Nivåregulatoren (LIC) er master, og strømningsregulatoren (FIC) er slave. Dampstrømmen virker som en foroverkobling til strømningsregulatoren.

%INDEX% Control, strategies: three-element boiler level control
%INDEX% Process: boiler water level control

%(END_NOTES)
