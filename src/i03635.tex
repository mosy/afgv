
%(BEGIN_QUESTION)
% Copyright 2008, Tony R. Kuphaldt, released under the Creative Commons Attribution License (v 1.0)
% This means you may do almost anything with this work of mine, so long as you give me proper credit

Green plants convert carbon dioxide (CO$_{2}$) and water (H$_{2}$O) into glucose sugar (C$_{6}$H$_{12}$O$_{6}$) and oxygen gas (O$_{2}$) using sunlight as the energy input, which is why this reaction is called {\it photo}synthesis.  This reaction is catalyzed, with the catalyst being a green-colored substance called {\it chlorophyll}.  Write a balanced equation showing all reactants and all reaction products in the proper proportions.

\vskip 10pt

Also, describe the role photosynthesis plays in the sustenance of biological life on earth, especially the life of mammals such as ourselves who use glucose as a fuel to sustain our own bodies' vital functions.  Identify where chemical bonds are formed in the food chain, and where chemical bonds are broken in the food chain, and how this relates to energy.

\vskip 20pt \vbox{\hrule \hbox{\strut \vrule{} {\bf Suggestions for Socratic discussion} \vrule} \hrule}

\begin{itemize}
\item{} Explain how to check your work to make sure the final equation is properly balanced.
\item{} Is the photosynthesis reaction {\it exothermic} or {\it endothermic}?
\item{} Should the photosynthesis reaction have a {\it positive} or {\it negative} $\Delta H$ value written next to it?
\item{} What happens to the chlorophyll after the reaction completes?  Does it need to be replenished?
\item{} Explain why plants will not continue to grow if starved of sunlight.
\item{} Explain why some plants actually will grow (at least for a time) if cut off at ground level where there is no longer any exposed surface area to collect sunlight.
\item{} Explain why photosynthesis does not occur in the absence of chlorophyll despite the presence of all necessary reactants plus sunshine.
\item{} Would it be possible to grow plants under artificial light (from special electric lamps), then harvest those plants and burn them as fuel to power a generator to keep the grow-lights lit?  Explain why or why not.
\end{itemize}

\underbar{file i03635}
%(END_QUESTION)





%(BEGIN_ANSWER)

$$\hbox{6H}_2\hbox{O} + \hbox{6CO}_2 \to \hbox{C}_6\hbox{H}_{12}\hbox{O}_6 + \hbox{6O}_2$$

\vskip 10pt

The energy required to sustain most life on Earth comes from the sun.  Sunlight powers the photosynthesis reaction, storing energy in the form of glucose, which animals consume in their food.  When digested, the glucose becomes available to the animal body as fuel, and is ``burned'' by combination with oxygen.  This slow ``combustion'' of glucose provides animals' bodies with power to operate, and it also produces CO$_{2}$ which the plants then re-process (with sunlight as the energy source) into more glucose.

%(END_ANSWER)





%(BEGIN_NOTES)

Balancing this reaction using simultaneous linear equations:

% No blank lines allowed between lines of an \halign structure!
% I use comments (%) instead, so Tex doesn't choke.

$$\vbox{\offinterlineskip
\halign{\strut
\vrule \quad\hfil # \ \hfil & 
\vrule \quad\hfil # \ \hfil & 
\vrule \quad\hfil # \ \hfil & 
\vrule \quad\hfil # \ \hfil & 
\vrule \quad\hfil # \ \hfil \vrule \cr
\noalign{\hrule}
%
% First row
1 & x & = & $y$ & $z$ \cr
%
\noalign{\hrule}
%
% Another row
H$_{2}$O & CO$_{2}$ & $\to$ & C$_{6}$H$_{12}$O$_{6}$ & O$_{2}$ \cr
%
\noalign{\hrule}
} % End of \halign 
}$$ % End of \vbox

% No blank lines allowed between lines of an \halign structure!
% I use comments (%) instead, so Tex doesn't choke.

$$\vbox{\offinterlineskip
\halign{\strut
\vrule \quad\hfil # \ \hfil & 
\vrule \quad\hfil # \ \hfil \vrule \cr
\noalign{\hrule}
%
% First row
{\bf Element} & {\bf Balance equation} \cr
%
\noalign{\hrule}
%
% Another row
Hydrogen & $2 + 0x = 12y + 0z$ \cr
%
\noalign{\hrule}
%
% Another row
Oxygen & $1 + 2x = 6y + 2z$ \cr
%
\noalign{\hrule}
%
% Another row
Carbon & $0 + 1x = 6y + 0z$ \cr
%
\noalign{\hrule}
} % End of \halign 
}$$ % End of \vbox

From the hydrogen balance equation ($2 + 0x = 12y + 0z$) we find that $y = {1 \over 6}$.  Plugging this result into the carbon balance equation ($0 + 1x = 6({1 \over 6}) + 0z$) we find that $x = 1$.  Plugging both of these results into the oxygen balance equation ($1 + 2(1) = 6({1 \over 6}) + 2z$) we find that $z = 1$.  In the interest of having whole-number molecular multipliers we will express the solutions as $x = 6$ and $y = 1$ and $z = 6$, with the original constant multiplier of water molecules bumped up from 1 to 6 as well:

$$\hbox{6H}_2\hbox{O} + \hbox{6CO}_2 \to \hbox{C}_6\hbox{H}_{12}\hbox{O}_6 + \hbox{6O}_2$$

\vskip 10pt

The energy required to sustain most life on Earth comes from the sun.  Sunlight powers the photosynthesis reaction, storing energy in the form of glucose, which animals consume in their food.  When digested, the glucose becomes available to the animal body as fuel, and is ``burned'' by combination with oxygen.  This slow ``combustion'' of glucose provides animals' bodies with power to operate, and it also produces CO$_{2}$ which the plants then re-process (with sunlight as the energy source) into more glucose.

Although energy is released in the formation of sugar and in the formation of diatomic oxygen molecules, far more energy input is required to break the strong bonds of carbon dioxide and of water.  Both water and carbon dioxide are very stable molecules (i.e. very strong bonds between their constituent atoms), as evidenced by the relative un-reactivity of both compounds.  In fact, the stability of both water and carbon dioxide molecular bonds is part of the reason they are both excellent fire-extinguishing agents: they do not react or separate even in the presence of significant heat, but rather act as oxygen displacers and coolants to snuff out fires.

%INDEX% Chemistry, catalyst
%INDEX% Chemistry, stoichiometry: balancing a chemical equation
%INDEX% Process: photosynthesis in plants

%(END_NOTES)


