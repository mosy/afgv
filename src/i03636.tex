
%(BEGIN_QUESTION)
% Copyright 2011, Tony R. Kuphaldt, released under the Creative Commons Attribution License (v 1.0)
% This means you may do almost anything with this work of mine, so long as you give me proper credit

The formula for calculating {\it path loss} in a radio system (the power lost due to spreading of RF energy from the transmitting antenna) looks like this:

$$L_p = -20 \log \left({4 \pi D} \over \lambda \right)$$

\noindent
Where,

$L_p$ = Loss of power (decibels)

$D$ = Distance between transmitting and receiving antennas

$\lambda$ = Wavelength of radio wave

\vskip 10pt

Why does this formula have a multiplier of {\it 20} instead of the more customary 10 we are used to seeing in decibel formulae?

\vfil 

\underbar{file i03636}
\eject
%(END_QUESTION)





%(BEGIN_ANSWER)

This is a graded question -- no answers or hints given!

%(END_ANSWER)





%(BEGIN_NOTES)

Path loss follows the {\it inverse square law}, which means power loss due to RF energy spreading will be proportional to the square of distance.  The basis for the inverse square law is that the energy spreads out proportional to the "surface area" of an imaginary sphere with the source as its center and the sphere's radius representing the distance between source and receiver, and that the surface area of a sphere is a function of the {\it square} of that sphere's radius:

$$L_p \hbox{ (as a fraction)} = \left({4 \pi D} \over \lambda \right)^2$$

$$L_p \hbox{ (as a decibel value)} = -10 \log \left({4 \pi D} \over \lambda \right)^2$$

One rule of logarithms is that we may translate an exponent into a multiplier, following this general rule:

$$\log x^a = a \log x$$

So, the power of ``2'' may be moved to become a multiplying constant in front of the logarithm function:

$$L_p = (2)(-10) \log \left({4 \pi D} \over \lambda \right)$$

$$L_p = -20 \log \left({4 \pi D} \over \lambda \right)$$

%(END_NOTES)


