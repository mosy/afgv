
%(BEGIN_QUESTION)
% Copyright 2008, Tony R. Kuphaldt, released under the Creative Commons Attribution License (v 1.0)
% This means you may do almost anything with this work of mine, so long as you give me proper credit

If a hydrocarbon fuel is completely mixed with just the right amount of air and thoroughly burned, the only products of combustion will be carbon dioxide (CO$_{2}$) and water vapor (H$_{2}$O).  If the mixing ratio is not stoichiometrically perfect, however, one or the other of the unburned reactants (either fuel or oxygen) will remain after combustion and be detectable in the exhaust gas stream.

Suppose a furnace operates with pure pentane (C$_{5}$H$_{12}$) as the fuel, and the balance of pentane to oxygen in the mix is as follows:

$$\hbox{C}_5\hbox{H}_{12} + \hbox{11O}_2 \to \hbox{water vapor} + \hbox{carbon dioxide} + \hbox{heat} + \hbox{unburned reactant}$$

First, determine whether or not this mix is too rich (excessive fuel) or too lean (excessive oxygen).  Then, determine what the output signal of an oxygen transmitter located in the exhaust pipe of this furnace will do ({\it increase}, {\it decrease}, or {\it remain the same}) if the mix changes to this:

$$\hbox{C}_5\hbox{H}_{12} + \hbox{9O}_2 \to \hbox{water vapor} + \hbox{carbon dioxide} + \hbox{heat} + \hbox{unburned reactant}$$

Assume the oxygen transmitter is {\it direct-acting} (i.e. outputs a greater milliamp signal with greater oxygen concentration).

\vskip 20pt \vbox{\hrule \hbox{\strut \vrule{} {\bf Suggestions for Socratic discussion} \vrule} \hrule}

\begin{itemize}
\item{} A common product of high-temperature combustion is NO$_{x}$ emissions (NO, NO$_{2}$, etc.).  Explain where the nitrogen comes from to form NO$_{x}$ compounds, since pentane fuel contains no nitrogen.
\item{} Will NO$_{x}$ emissions (NO, NO$_{2}$, etc.) in this combustion process increase, decrease, or remain at the same concentration level with the change in air/fuel ratio?  Explain why.
\item{} A common method for mitigating NO$_{x}$ emissions (NO, NO$_{2}$, etc.) is to react the exhaust gases with ammonia (NH$_{3}$).  Identify the harmless byproducts of a complete NO$_{x}$-ammonia reaction.
\end{itemize}

\underbar{file i03638}
%(END_QUESTION)





%(BEGIN_ANSWER)

This mixture is {\bf too lean}.  The ideal (stoichiometric) pentane-to-oxygen mix is:

$$\hbox{C}_5\hbox{H}_{12} + \hbox{8O}_2 \to \hbox{6H}_2\hbox{O} + \hbox{5CO}_2$$

If the oxygen content is reduced, the mixture will be closer to ideal, but still too lean.  Thus, the oxygen transmitter signal will {\bf decrease} (become closer to 4 mA).
 
%(END_ANSWER)





%(BEGIN_NOTES)

Balancing this reaction using simultaneous linear equations:

% No blank lines allowed between lines of an \halign structure!
% I use comments (%) instead, so Tex doesn't choke.

$$\vbox{\offinterlineskip
\halign{\strut
\vrule \quad\hfil # \ \hfil & 
\vrule \quad\hfil # \ \hfil & 
\vrule \quad\hfil # \ \hfil & 
\vrule \quad\hfil # \ \hfil & 
\vrule \quad\hfil # \ \hfil \vrule \cr
\noalign{\hrule}
%
% First row
1 & x & = & $y$ & $z$ \cr
%
\noalign{\hrule}
%
% Another row
C$_{5}$H$_{12}$ & O$_{2}$ & $\to$ & H$_{2}$O & CO$_{2}$ \cr
%
\noalign{\hrule}
} % End of \halign 
}$$ % End of \vbox

% No blank lines allowed between lines of an \halign structure!
% I use comments (%) instead, so Tex doesn't choke.

$$\vbox{\offinterlineskip
\halign{\strut
\vrule \quad\hfil # \ \hfil & 
\vrule \quad\hfil # \ \hfil \vrule \cr
\noalign{\hrule}
%
% First row
{\bf Element} & {\bf Balance equation} \cr
%
\noalign{\hrule}
%
% Another row
Hydrogen & $12 + 0x = 2y + 0z$ \cr
%
\noalign{\hrule}
%
% Another row
Oxygen & $0 + 2x = y + 2z$ \cr
%
\noalign{\hrule}
%
% Another row
Carbon & $5 + 0x = 0y + 1z$ \cr
%
\noalign{\hrule}
} % End of \halign 
}$$ % End of \vbox

From the hydrogen balance equation ($12 + 0x = 2y + 0z$) we find that $y = 6$.  From the carbon balance equation ($5 + 0x = 0y + 1z$) we find that $z = 5$.  Plugging these results into the oxygen balance equation ($0 + 2x = (6) + (2)(5)$) we find that $x = 8$:

$$\hbox{C}_5\hbox{H}_{12} + \hbox{8O}_2 \to \hbox{6H}_2\hbox{O} + \hbox{5CO}_2$$

%INDEX% Chemistry, stoichiometry: balancing a chemical equation

%(END_NOTES)


