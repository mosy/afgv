
%(BEGIN_QUESTION)
% Copyright 2008, Tony R. Kuphaldt, released under the Creative Commons Attribution License (v 1.0)
% This means you may do almost anything with this work of mine, so long as you give me proper credit

Two technicians are arguing about the suitability of a chromatograph in a process application.  Technician ``A'' says a chromatograph should be able to measure the parts-per-million concentrations of different compounds present in a sample stream, but technician ``B'' claims chromatographs are only useful for identifying the types of compounds present and not their quantities.

Identify which technician is correct, and explain why by referencing how a chromatograph functions.  Bonus for identifying which technician graduated from Perry Technical Institute and which graduated from Bellingham Technical College.

\vskip 50pt

\underbar{file i03641}
%(END_QUESTION)





%(BEGIN_ANSWER)

Chromatographs are most definitely able to both qualify and quantify species in a sample, so long as those species are sufficiently separated by the column.  When species exit a chromatograph column, the detector outputs a signal proportional to the {\it quantity} (mass flow rate) of that species exiting, which when integrated with respect to time yields a total quantity value of that species in the sample.

\vskip 10pt

I recommend granting half-credit for the A/B answer, and half-credit for the explanation of why.

%(END_ANSWER)





%(BEGIN_NOTES)

{\bf This question is intended for exams only and not worksheets!}.

%(END_NOTES)


