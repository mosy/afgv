
%(BEGIN_QUESTION)
% Copyright 2010, Tony R. Kuphaldt, released under the Creative Commons Attribution License (v 1.0)
% This means you may do almost anything with this work of mine, so long as you give me proper credit

Identify the integer data types your PLC uses for the ``accumulator'' or ``current'' value registers in its {\it counter} instructions and {\it timer} instructions.  Note that different data types may be used for each of these instruction types!  Are the registers 16 bit or 32 bit?  Are they signed or unsigned?  How can you tell?

\vskip 20pt \vbox{\hrule \hbox{\strut \vrule{} {\bf Suggestions for Socratic discussion} \vrule} \hrule}

\begin{itemize}
\item{} Why does the data type matter to us?
\item{} What is the largest number value reachable with each of the identified data types?
\item{} Suppose you needed the PLC to be able to count (or to time) to a value greater than that possible by the basic counter or timer instruction on its own.  For example, your counter uses a 16-bit signed integer register, and you need to be able to count up to +1 million.  How could you do this given the limitations of the signed 16-bit counter registers?
\end{itemize}

\underbar{file i03677}
%(END_QUESTION)





%(BEGIN_ANSWER)

 
%(END_ANSWER)





%(BEGIN_NOTES)

Allen-Bradley MicroLogix 1000 and 1100 PLCs: counter preset and accumulator registers are signed 16-bit integers, having a number value range of $-32768$ to +32767.  Timer preset and accumulator registers are also signed 16-bit numbers, but since timers never count down, only the positive half of the range is used (e.g. 0 to +32767).

\vskip 10pt

Koyo CLICK PLCs use 16 bit signed integers for timer instructions, and 32 bit signed integers (``double'' integers) for counter instructions.

%\vskip 10pt

%Koyo DirectLogic PLCs use 16-bit BCD integers for ??? instructions, and ??? for ??? instructions.

%INDEX% PLC, exploratory question (HMI programming)

%(END_NOTES)


