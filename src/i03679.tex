
%(BEGIN_QUESTION)
% Copyright 2010, Tony R. Kuphaldt, released under the Creative Commons Attribution License (v 1.0)
% This means you may do almost anything with this work of mine, so long as you give me proper credit

A salesperson for a radar level transmitter manufacturer is trying to tell your supervisor that guided-wave radar transmitters are inherently immune to changes in process liquid density, unlike some traditional technologies for measuring liquid level.  The application being considered is one with a single liquid beneath a vapor, in a high-pressure process vessel.  You know this claim to be basically true, but you also know there is something more to this picture that the salesperson is not telling your boss.

\vskip 10pt

Explain how a guided-wave level transmitter {\it can} be misled by changes in process operating conditions.  Be as specific as you can in your answer:

\vskip 50pt

\underbar{file i03679}
%(END_QUESTION)





%(BEGIN_ANSWER)

Radar gauge accuracy depends on knowing the speed of light through the vapor space above the liquid.  This speed of light varies with vapor permittivity, which also happens to vary with the {\it density} of the vapor.  Thus, a GWR transmitter may be misled into thinking the liquid level is changing when it fact it is only the process pressure (or temperature) changing!

Guided-wave radar gauges may also be ``fooled'' by accumulated solids on the waveguide, and sometimes by foam or emulsions.

%(END_ANSWER)





%(BEGIN_NOTES)

{\bf This question is intended for exams only and not worksheets!}.

%(END_NOTES)

