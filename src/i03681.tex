
%(BEGIN_QUESTION)
% Copyright 2010, Tony R. Kuphaldt, released under the Creative Commons Attribution License (v 1.0)
% This means you may do almost anything with this work of mine, so long as you give me proper credit

An industrial fuel depot uses a DP transmitter to hydrostatically measure the amount of liquid kerosene stored in a tank, over a range of 0 to 15 feet.  Suppose the decision is made to use this same storage tank for holding methanol instead of kerosene, but no one bothers to re-range the level transmitter for the new liquid.

Assuming no elevation or suppression in the DP transmitter's installation and a calibration based on the wrong liquid (kerosene), calculate the amount of methanol the transmitter ``thinks'' is in the tank when the actual methanol level is 12.4 feet (verified with a hand tape).

\vskip 10pt

Level (perceived) = \underbar{\hskip 50pt} feet

\vskip 10pt

Also, determine the proper LRV and URV points for the transmitter, if it is to accurately measure the liquid height of methanol over the original height range.  Express your answers in units of PSI:

\vskip 10pt

LRV = \underbar{\hskip 50pt} PSI \hskip 30pt URV = \underbar{\hskip 50pt} PSI

\underbar{file i03681}
%(END_QUESTION)





%(BEGIN_ANSWER)

Level (perceived) = \underbar{\bf 12.23} feet

\vskip 10pt

LRV = \underbar{\bf 0} PSI \hskip 30pt URV = \underbar{\bf 5.26} PSI

\vskip 10pt

4 points for perceived level calculation, 3 points for each range value.

%(END_ANSWER)





%(BEGIN_NOTES)

{\bf This question is intended for exams only and not worksheets!}.

%(END_NOTES)


