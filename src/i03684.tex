
%(BEGIN_QUESTION)
% Copyright 2010, Tony R. Kuphaldt, released under the Creative Commons Attribution License (v 1.0)
% This means you may do almost anything with this work of mine, so long as you give me proper credit

\noindent
{\bf Programming Challenge -- Parking garage counter} 

\vskip 10pt

Suppose we wish to count the number of cars inside a parking garage at any given time, by incrementing a counter each time a car enters the garage through the entry lane, and decrementing the same counter each time a car leaves the garage through the exit lane.  One discrete input of the PLC will connect to a switch detecting the passing of each car through the garage entry, and another discrete input of the PLC will connect to a switch detecting cars passing out the garage exit.  The PLC must be equipped with a way to for the garage attendant to manually reset the counter to zero.

Write a PLC program to perform this function, and demonstrate its operation using switches connected to its inputs to simulate the discrete inputs in a real application.

\vskip 20pt \vbox{\hrule \hbox{\strut \vrule{} {\bf Suggestions for Socratic discussion} \vrule} \hrule}

\begin{itemize}
\item{} What type of switches would you recommend to detect cars driving into the parking garage?
\item{} How are you able to view the counter instruction's current count value as the program runs?
\item{} Is there any way to ``fool'' this system so that it does not hold an accurate count of cars inside the garage?
\end{itemize}


\vfil 

\noindent
PLC comparison:

\begin{itemize}
\item{} \underbar{Allen-Bradley Logix 5000}: {\tt CTUD} count-up/down instruction
\vskip 5pt
\item{} \underbar{Allen-Bradley SLC 500}: {\tt CTU} and {\tt CTD} instructions.
\vskip 5pt
\item{} \underbar{Siemens S7-200}: {\tt CTUD} count-up/down instruction
\vskip 5pt
\item{} \underbar{Koyo (Automation Direct) DirectLogic}: {\tt UDC} counter instruction
\end{itemize}

\underbar{file i03684}
\eject
%(END_QUESTION)





%(BEGIN_ANSWER)


%(END_ANSWER)





%(BEGIN_NOTES)

I strongly recommend students save all their PLC programs for future reference, commenting them liberally and saving them with special filenames for easy searching at a later date!

\vskip 10pt

I also recommend presenting these programs as problems for students to work on in class for a short time period, then soliciting screenshot submissions from students (on flash drive, email, or some other electronic file transfer method) when that short time is up.  The purpose of this is to get students involved in PLC programming, and also to have them see other students' solutions to the same problem.  These screenshots may be emailed back to students at the conclusion of the day so they have other students' efforts to reference for further study.


%INDEX% PLC, programming challenge: parking garage counter application (up/down)

%(END_NOTES)


