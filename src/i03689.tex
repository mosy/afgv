
%(BEGIN_QUESTION)
% Copyright 2010, Tony R. Kuphaldt, released under the Creative Commons Attribution License (v 1.0)
% This means you may do almost anything with this work of mine, so long as you give me proper credit

Identify the physical property of a liquid used in the following level-measuring instrument types to distinguish that liquid from the vapor above it.  Note: more than one level instrument may share the same property (same letter)!

\begin{itemize}
\item{} Displacer: \underbar{\hskip 50pt}
\item{} Guided-wave radar: \underbar{\hskip 50pt}
\item{} Magnetostriction: \underbar{\hskip 50pt}
\item{} Nuclear: \underbar{\hskip 50pt}
\item{} Capacitive: \underbar{\hskip 50pt}
\end{itemize}

\vskip 10pt

\begin{itemize}
\item {\bf A}: Electrical conductivity
\item {\bf B}: Specific gravity
\item {\bf C}: Speed of sound
\item {\bf D}: Viscosity
\item {\bf E}: Temperature
\item {\bf F}: Color
\item {\bf G}: Dielectric constant (permittivity)
\item {\bf H}: Corrosive potential
\item {\bf I}: Ability to attenuate or reflect radiation
\end{itemize}

\underbar{file i03689}
%(END_QUESTION)





%(BEGIN_ANSWER)

\begin{itemize}
\item{} Displacer: \underbar{\bf B}
\item{} Guided-wave radar: \underbar{\bf G}
\item{} Magnetostriction: \underbar{\bf B}
\item{} Nuclear: \underbar{\bf I} (or possibly {\bf B})
\item{} Capacitive: \underbar{{\bf A} or {\bf G}} (depending on type of instrument)
\end{itemize}

%(END_ANSWER)





%(BEGIN_NOTES)

{\bf This question is intended for exams only and not worksheets!}.

%(END_NOTES)


