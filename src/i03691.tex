
%(BEGIN_QUESTION)
% Copyright 2010, Tony R. Kuphaldt, released under the Creative Commons Attribution License (v 1.0)
% This means you may do almost anything with this work of mine, so long as you give me proper credit

\noindent
{\bf Programming Challenge -- Hour/Minute/Second timer} 

\vskip 10pt

Many PLCs provide a range of {\it special contacts} to the programmer.  Among these ``special contacts'' is typically one that cycles on and off at a rate of once per second, like a 1 Hz clock pulse.  

Research the special contact for this function in your PLC, then write a PLC program for an Hour/Minute/Second timer using three counter instructions: one to count seconds (0 to 59), one to count minutes (0 to 59), and one to count hours.

\vskip 20pt \vbox{\hrule \hbox{\strut \vrule{} {\bf Suggestions for Socratic discussion} \vrule} \hrule}

\begin{itemize}
\item{} What is the address of the special contact in your PLC for the 1 Hz clock pulse?
\item{} How do you make three counters count in the correct sequence, so that one represents seconds, the next minutes, and the next hours?
\end{itemize}


\vfil 

\noindent
PLC comparison:

\begin{itemize}
\item{} \underbar{Allen-Bradley SLC 500}: status bit {\tt S:4/0} is a free-running clock pulse with a period of 20 milliseconds, which may be used to clock a counter instruction up to 50 to make a 1-second pulse (because 50 times 20 ms = 1000 ms = 1 second).
\vskip 5pt
\item{} \underbar{Siemens S7-200}: Special Memory bit {\tt SM0.5} is a free-running clock pulse with a period of 1 second.
\vskip 5pt
\item{} \underbar{Koyo (Automation Direct) DirectLogic}: Special relay {\tt SP4} is a free-running clock pulse with a period of 1 second. 
\end{itemize}

\underbar{file i03691}
\eject
%(END_QUESTION)





%(BEGIN_ANSWER)


%(END_ANSWER)





%(BEGIN_NOTES)

I strongly recommend students save all their PLC programs for future reference, commenting them liberally and saving them with special filenames for easy searching at a later date!

\vskip 10pt

I also recommend presenting these programs as problems for students to work on in class for a short time period, then soliciting screenshot submissions from students (on flash drive, email, or some other electronic file transfer method) when that short time is up.  The purpose of this is to get students involved in PLC programming, and also to have them see other students' solutions to the same problem.  These screenshots may be emailed back to students at the conclusion of the day so they have other students' efforts to reference for further study.



%INDEX% PLC, programming challenge: hour/minute/second timer (using counters and a 1-second pulse bit)

%(END_NOTES)


