
%(BEGIN_QUESTION)
% Copyright 2015, Tony R. Kuphaldt, released under the Creative Commons Attribution License (v 1.0)
% This means you may do almost anything with this work of mine, so long as you give me proper credit

\noindent
{\bf Demonstration Program -- timer instructions} 

\vskip 10pt

An important technique for learning any programming language -- Ladder Diagram PLC programming included -- is to write simple ``demonstration'' programs showcasing and explaining how particular instructions and programming constructs are supposed to work.  Since you have access to your own personal PLC, you can explore the elements of your PLC's programming language like a scientist would explore new specimens: subject them to tests and record how they respond.  This is how you will be able to teach yourself new models of PLC when you are working in your career, when you won't have textbooks to follow or training to show you exactly what to do.

\vskip 10pt

Write such a ``demonstration'' program for your PLC's {\it timer} instructions, where discrete inputs on your PLC control discrete outputs on your PLC.  An acceptable demonstration program must meet these three criteria:

\begin{itemize}
\item{} {\bf Simple} -- nothing ``extra'' included in the program to detract from the fundamental behavior of the instruction(s) being explored
\vskip 5pt
\item{} {\bf Complete} -- nothing missing from the program relevant to the fundamental behavior of the instruction(s) being explored.  {\it For a timer demonstration program, this includes on-delay timers, off-delay timers, and retentive timers.}
\vskip 5pt
\item{} {\bf Clearly documented} -- every rung clearly commented in your own words, every variable named
\end{itemize}

{\it Your instructor will challenge you to use this demonstration program to illustrate what you have learned about PLC counter instructions.}



\vskip 20pt \vbox{\hrule \hbox{\strut \vrule{} {\bf Suggested questions your demonstration program should answer:} \vrule} \hrule}

\begin{itemize}
\item{} What are the different timer instruction types offered on your PLC?  What does each one of them do?
\item{} Where in the PLC's memory is each timer storing its data?
\item{} Where in the PLC programming editor can you view the ``live'' status of a timer instruction?
\item{} What are the ``timebase'' options for timers in your PLC?  How fast or slow can they time?
\item{} How long of a time period can a timer time?  What is the maximum ``count'' value for a timer?
\item{} When a timer instruction reaches its preset (setpoint) value, does it keep timing or does it stop at that value?
\item{} How does the operation of an off-delay timer differ from that of an on-delay timer?
\item{} How does the operation of a retentive on-delay timer differ from that of a non-retentive on-delay timer?
\item{} How does each type of timer get reset?
\item{} When a timer instruction is reset, does its current value go to zero or one?
\item{} How does each type of timer get to control something, like an output bit?
\item{} Is it possible to ``preload'' a timer instruction so that it doesn't have to begin at the starting value when the PLC program runs anew?
\item{} What happens to the timer's current value when it reaches its maximum value?  Does the timer instruction stop timing, or does it do something else?
\end{itemize}

\underbar{file i03692}
%(END_QUESTION)





%(BEGIN_ANSWER)

There are no answers provided here!  For help, consult the ``instruction set'' reference manual for your PLC, which will describe in detail how each type of instruction is supposed to function in your PLC.

%(END_ANSWER)





%(BEGIN_NOTES)

\noindent
{\bf Summary Quiz:}

The recommended summary quiz is to have \underbar{each student} demonstrate their PLCs running this demonstration program.  Demonstration programs should not be accepted unless and until they meet all three criteria listed in the question: they need to be {\it simple} (no extraneous features), {\it complete} (demonstrating {\it all} the necessary instructions), and {\it thoroughly documented} in the student's own words (every rung commented intelligently, every variable named).

\vskip 10pt

Some students may opt to write separate demonstration programs for each type of timer instruction supported by their PLC.  This is a perfectly acceptable alternative to writing a single demonstration program showcasing all timer instruction types.

\vskip 10pt

When checking students' demonstation programs, have them run the programs with their laptop PCs in online mode so they can view status highlighting and numerical values live, and {\it ask them to provide a running commentary of what they see the instructions doing.  If their commentary is lacking, challenge them to explore instruction behaviors that they have not yet done.}  Refer to the ``suggested questions'' for specific ideas on instruction behavior that may be explored in a demonstration program.

\vskip 10pt

This is by far the most important PLC program students will write today.  If there are other programs assigned in this day's plan, {\it prioritize this one}.  This point bears mentioning because a common error students make is to disregard the importance of a well-written demo program, instead tending to treat it as ``busywork'' and move on to ``more important'' assignments.  As the instructor your task is to explain why demonstration programs are important and to ensure students take them seriously:

\begin{itemize}
\item{} Demonstration programs are a powerful tool for self-instruction later when graduates will be challenged to learn some new PLC system.
\vskip 5pt
\item{} Demonstration programs may be re-run at a later date, which makes them ideal tools for review near the conclusion of the course.
\vskip 5pt
\item{} Demonstration programs require students to scientifically observe the PLC's programmed behavior, learning by {\it experiment} rather than learning by {\it direct instruction}.
\vskip 5pt
\item{} Well-written comments demand critical reflection on the part of the student, and also serve as practice for technical expression.
\end{itemize}









\vfil \eject

\noindent
{\bf Sample demonstration program for one type of timer instructions:} (for an Allen-Bradley MicroLogix PLC, TON instruction only)

$$\epsfxsize=6in \includegraphics[width=15.5cm]{i03692x01.eps}$$

%INDEX% PLC, demonstration program: timer instructions

%(END_NOTES)


