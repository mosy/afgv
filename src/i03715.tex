
%(BEGIN_QUESTION)
% Copyright 2010, Tony R. Kuphaldt, released under the Creative Commons Attribution License (v 1.0)
% This means you may do almost anything with this work of mine, so long as you give me proper credit

\noindent
{\bf Programming Challenge -- Engine auto-start sequence}

\vskip 10pt

Suppose we wish to have a PLC start up an engine automatically on demand.  We need the PLC to follow this sequence in starting the engine:

% No blank lines allowed between lines of an \halign structure!
% I use comments (%) instead, so that TeX doesn't choke.

$$\vbox{\offinterlineskip
\halign{\strut
\vrule \quad\hfil # \ \hfil & 
\vrule \quad\hfil # \ \hfil & 
\vrule \quad\hfil # \ \hfil & 
\vrule \quad\hfil # \ \hfil & 
\vrule \quad\hfil # \ \hfil \vrule \cr
\noalign{\hrule}
%
% First row
Step \# & Throttle (idle/run) & Choke & Ignition & Starter \cr
%
\noalign{\hrule}
%
% Another row
1 & 0 & 0 & 0 & 0 \cr
%
\noalign{\hrule}
%
% Another row
2 & 0 & 1 & 0 & 0 \cr
%
\noalign{\hrule}
%
% Another row
3 & 0 & 1 & 1 & 1 \cr
%
\noalign{\hrule}
%
% Another row
4 & 1 & 0 & 1 & 0 \cr
%
\noalign{\hrule}
} % End of \halign 
}$$ % End of \vbox

The program needs to have two discrete inputs and four discrete outputs:

\begin{itemize}
\item{} Input\_start: Start-up command signal (0 = shut down ; 1 = begin start-up sequence)
\item{} Input\_run\_detector: Running sensor (0 = not firing ; 1 = engine running)
\vskip 5pt
\item{} Output\_throttle: (0 = idle position; 1 = run position)
\item{} Output\_choke: (0 = off (run position) ; 1 = choked position)
\item{} Output\_ignition: (0 = off ; 1 = on)
\item{} Output\_starter: (0 = off ; 1 = cranking)
\end{itemize}

Steps 1 through 3 should happen according to a timed schedule, but the transition from step 3 (cranking the engine) to step 4 (engine running) should occur {\it only} if Input\_run\_detector shows the engine has fired.  The sequence should immediately revert to step 1 if the ``Input\_start'' command signal ever turns off.


\vskip 20pt \vbox{\hrule \hbox{\strut \vrule{} {\bf Suggestions for Socratic discussion} \vrule} \hrule}

\begin{itemize}
\item{} How will your sequencer ``know'' when to advance from one step to the next, especially given the change of criteria from steps 1 through 3, to step 4?
\end{itemize}


\vfil

\noindent
PLC comparison:

\begin{itemize}
\item{} \underbar{Allen-Bradley Logix 5000}: relevant ladder-logic commands include {\tt SQI}, {\tt SQO}, and {\tt SQL}.
\vskip 5pt
\item{} \underbar{Allen-Bradley SLC 500}: relevant ladder-logic commands include {\tt SQI}, {\tt SQO}, {\tt SQC}, and {\tt SQL}. 
\vskip 5pt
\item{} \underbar{Siemens S7-200}: relevant ladder-logic commands include {\tt SCR}, {\tt SCRE}, and {\tt SCRT}.
\vskip 5pt
\item{} \underbar{Koyo (Automation Direct) DirectLogic}: relevant ladder-logic commands include {\tt DRUM} and {\tt EDRUM}.
\end{itemize}

\underbar{file i03715}
\eject
%(END_QUESTION)





%(BEGIN_ANSWER)


%(END_ANSWER)





%(BEGIN_NOTES)


%INDEX% PLC, programming challenge: Engine auto-start sequence

%(END_NOTES)


