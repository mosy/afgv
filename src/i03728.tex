
%(BEGIN_QUESTION)
% Copyright 2010, Tony R. Kuphaldt, released under the Creative Commons Attribution License (v 1.0)
% This means you may do almost anything with this work of mine, so long as you give me proper credit

Read and outline the ``Analyzer Sample Systems'' section of the ``Continuous Analytical Measurement'' chapter in your {\it Lessons In Industrial Instrumentation} textbook.  Note the page numbers where important illustrations, photographs, equations, tables, and other relevant details are found.  Prepare to thoughtfully discuss with your instructor and classmates the concepts and examples explored in this reading.

\underbar{file i03728}
%(END_QUESTION)




%(BEGIN_ANSWER)


%(END_ANSWER)





%(BEGIN_NOTES)

Analytical instruments installed within the process line or vessel are called {\it in situ}, which is Latin for ``in the place.''  In many applications, however, this is impractical and the sample must be transported away from the process for measurement.  A sample system is the network of tubes, filters, pumps, and other equipment needed to conduct and condition that sample before it enters into an analyzer.

\vskip 10pt

Sample systems represent an additional point of failure in the instrument loop, and are often to blame for analyzer troubles if not properly designed and maintained.

\vskip 10pt

Some sample systems incorporate {\it automatic calibration} ability with span gas cylinders, solenoid valves, and a PLC to routinely send these calibration gases to the analyzer(s) to check their ``drift'' at regular intervals.  Such systems save a lot of maintenance labor.








\vskip 20pt \vbox{\hrule \hbox{\strut \vrule{} {\bf Suggestions for Socratic discussion} \vrule} \hrule}

\begin{itemize}
\item{} {\bf In what ways may an analyzer sample system fail in such a way as to cause one or more analyzers to report false measurements?}
\item{} Note some of the practical features found in the illustrated CEMS sample system.
\item{} Explain what could happen if the chiller in this sample system were to fail.
\item{} Explain what could happen if the heat tracing for the sample line in this sample system were to fail.
\item{} Explain what could happen if the air dryer in this sample system were to fail.
\item{} Explain what could happen if the backpressure regulator in this sample system were to fail wide-open.
\item{} Explain the purpose for having the ``calibration'' line leading up to the sample port.
\item{} Explain how {\it automated calibration} works and why it is important.
\end{itemize}








\vfil \eject

\noindent
{\bf Prep Quiz:}

Describe at least one of the ways in which an analyzer sample system {\it conditions} a gas sample to get it ready for one or more analyzer instruments to measure.









\vfil \eject

\noindent
{\bf Prep Quiz:}

Explain how it is possible for gas analyzers to automatically {\it self-calibrate}.  Specifically, what components exist in the system to enable this capability?


%INDEX% Reading assignment: Lessons In Industrial Instrumentation, Analytical (sample systems)

%(END_NOTES)


