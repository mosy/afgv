
%(BEGIN_QUESTION)
% Copyright 2010, Tony R. Kuphaldt, released under the Creative Commons Attribution License (v 1.0)
% This means you may do almost anything with this work of mine, so long as you give me proper credit

Suppose you wish to simulate a type S thermocouple at a temperature of 2670 degrees Fahrenheit for a temperature transmitter with reference junction compensation enabled.  Knowing the ambient temperature at the transmitter (using a portable thermometer) is 73 degrees Fahrenheit, how many millivolts must you send to the transmitter's input terminals?

\vskip 10pt

Suppose you wish to simulate a type T thermocouple at a temperature of 155 degrees Celsius for a temperature transmitter with reference junction compensation enabled.  Knowing the ambient temperature at the transmitter (using a portable thermometer) is 20 degrees Celsius, how many millivolts must you send to the transmitter's input terminals?

\underbar{file i03733}
%(END_QUESTION)





%(BEGIN_ANSWER)

Type S thermocouple: $V$ = 15.037 mV

\vskip 10pt

Type T thermocouple: $V$ = 6.166 mV

%(END_ANSWER)





%(BEGIN_NOTES)


%INDEX% Measurement, temperature: thermocouple

%(END_NOTES)

