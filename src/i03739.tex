
%(BEGIN_QUESTION)
% Copyright 2009, Tony R. Kuphaldt, released under the Creative Commons Attribution License (v 1.0)
% This means you may do almost anything with this work of mine, so long as you give me proper credit

Configure a personal computer to act as an ASCII data terminal, receiving ASCII-coded data through a serial port and displaying the alphanumeric characters on the screen.  Several ``terminal emulator'' programs exist for this purpose, perhaps the most common Microsoft Windows-based variant being {\it Hyperterminal} (this used to be one of the ``stock'' accessory programs included with Windows installations).  Another popular terminal emulator program is {\it Kermit}.  Locate a terminal emulator program, install it on your computer, and create a simple serial-port cable which you may connect between your computer and your PLC for data communications (using the {\it transmit}, {\it receive}, and {\it ground} pins only).

You may wish to test the terminal software and cable by connecting two personal computers together and exchanging messages back and forth using the same software.




\vskip 20pt \vbox{\hrule \hbox{\strut \vrule{} {\bf Suggestions for Socratic discussion} \vrule} \hrule}

\begin{itemize}
\item{} Identify how communication parameters for your terminal emulator program are set (including bit rate, number of data bits, number of stop bits, parity, etc.)
\item{} Which pins on the 9-pin or 25-pin serial connector will you use for transmit, receive, and ground?
\item{} Why is it important to have the computer's {\it transmit} pin connected to the PLC's {\it receive} pin, and visa versa?
\end{itemize}



\vfil 

\underbar{file i03739}
\eject
%(END_QUESTION)





%(BEGIN_ANSWER)


%(END_ANSWER)





%(BEGIN_NOTES)


%INDEX% PLC, exploratory question (data communications)

%(END_NOTES)


