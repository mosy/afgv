
%(BEGIN_QUESTION)
% Copyright 2009, Tony R. Kuphaldt, released under the Creative Commons Attribution License (v 1.0)
% This means you may do almost anything with this work of mine, so long as you give me proper credit

Write a relay ladder-logic (RLL) program in your PLC that {\it repeatedly} sends a short ASCII-encoded message to a data terminal (e.g. a personal computer running a terminal emulator program) at 10-second intervals as long as a discrete input channel is energized. 




\vskip 20pt \vbox{\hrule \hbox{\strut \vrule{} {\bf Suggestions for Socratic discussion} \vrule} \hrule}

\begin{itemize}
\item{} What is the easiest way to send messages on repeating intervals?  If using a timer function, how do you get the timing sequence to repeat itself?
\end{itemize}



\vfil 

\noindent
PLC comparison:

\begin{itemize}
\item{} \underbar{Allen-Bradley Logix 5000}: the ``ASCII Write'' instructions {\tt AWT} and {\tt AWA} may be used to do this.  The ``ASCII Write Append'' instruction ({\tt AWA}) is convenient to use because it may be programmed to automatically insert linefeed and carriage-return commands at the end of a message string.
\vskip 5pt
\item{} \underbar{Allen-Bradley SLC 500}: the ``ASCII Write'' instructions {\tt AWT} and {\tt AWA} may be used to do this.  The ``ASCII Write Append'' instruction ({\tt AWA}) is convenient to use because it may be programmed to automatically insert linefeed and carriage-return commands at the end of a message string.
\vskip 5pt
\item{} \underbar{Siemens S7-200}: the ``Transmit'' instruction ({\tt XMT}) is useful for this task when used in Freeport mode.
\vskip 5pt
\item{} \underbar{Koyo (Automation Direct) DirectLogic}: the ``Print Message'' instruction ({\tt PRINT}) is useful for this task.
\end{itemize}

\underbar{file i03742}
\eject
%(END_QUESTION)





%(BEGIN_ANSWER)


%(END_ANSWER)





%(BEGIN_NOTES)


%INDEX% PLC, exploratory question (data communications)

%(END_NOTES)


