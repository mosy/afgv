
%(BEGIN_QUESTION)
% Copyright 2009, Tony R. Kuphaldt, released under the Creative Commons Attribution License (v 1.0)
% This means you may do almost anything with this work of mine, so long as you give me proper credit

Write a relay ladder-logic (RLL) program in your PLC that increments a counter whenever a discrete input channel is energized, sending an ASCII message to a data terminal showing the counter's current value appended to the end of a text string, like this:

\vskip 10pt

{\tt The counter value is: }$X$

\vskip 10pt

{\it (Where $X$ is the current value of the counter)}


\vskip 20pt \vbox{\hrule \hbox{\strut \vrule{} {\bf Suggestions for Socratic discussion} \vrule} \hrule}

\begin{itemize}
\item{} How do you distinguish the counter's integer count value from the text string when sending the entire ASCII message to the terminal?
\end{itemize}



\vfil 

\noindent
PLC comparison:

\begin{itemize}
\item{} \underbar{Allen-Bradley Logix 5000}: the ``ASCII Write'' instructions {\tt AWT} and {\tt AWA} may be used to do this.  The ``ASCII Write Append'' instruction ({\tt AWA}) is convenient to use because it may be programmed to automatically insert linefeed and carriage-return commands at the end of a message string.
\vskip 5pt
\item{} \underbar{Allen-Bradley SLC 500}: the ``ASCII Write'' instructions {\tt AWT} and {\tt AWA} may be used to do this.  The ``ASCII Write Append'' instruction ({\tt AWA}) is convenient to use because it may be programmed to automatically insert linefeed and carriage-return commands at the end of a message string.
\vskip 5pt
\item{} \underbar{Siemens S7-200}: the ``Transmit'' instruction ({\tt XMT}) is useful for this task when used in Freeport mode.
\vskip 5pt
\item{} \underbar{Koyo (Automation Direct) DirectLogic}: the ``Print Message'' instruction ({\tt PRINT}) is useful for this task.
\end{itemize}

\underbar{file i03745}
\eject
%(END_QUESTION)





%(BEGIN_ANSWER)


%(END_ANSWER)





%(BEGIN_NOTES)


%INDEX% PLC, exploratory question (data communications)

%(END_NOTES)


