
%(BEGIN_QUESTION)
% Copyright 2013, Tony R. Kuphaldt, released under the Creative Commons Attribution License (v 1.0)
% This means you may do almost anything with this work of mine, so long as you give me proper credit

Your instructor will choose one 4-20 mA field instrument and one control system from the lists shown below, for which you must sketch an accurate circuit diagram showing how the two instruments would connect to each other.  If this interconnection between controller and field instrument requires additional electrical components to function (e.g. DC or AC power source, precision 250 $\Omega$ resistor, diode, relay, etc.), those must be incorporated into your diagram as well.  Instruction manuals for all instrument listed are available on the electronic Instrumentation Reference for your convenience.  When your sketch is complete, you must show the relevant manual pages to your instructor for verification of correct connections.

This exercise tests your ability to locate appropriate information in technical manuals and sketch a correct 4-20 mA loop circuit for a given pair of instruments.  The electronic Instrumentation Reference will be available to you in order to answer this question.

\vskip 10pt

Since all 4-20 mA ``loops'' are basically series DC circuits, it is highly recommended that you approach their design the same as for any other DC circuit: carefully identify all {\it sources} and {\it loads} in the circuit, trace directions of all currents, and mark the polarities of all voltages.  Most of the mistakes made in this type of circuit design challenge may be remedied by careful consideration of these specific circuit-analysis details.


%%%%%%%%%%%%%%%%%%%%%%%%%%%%%%%
\vskip 10pt
\filbreak
\hbox{ \vrule
\vbox{ \hrule \vskip 3pt
\hbox{ \hskip 3pt
\vbox{ \hsize=5in \raggedright

\noindent \centerline{\bf 4-20 mA transmitter options}
\begin{itemize}
\item{} Pressure
	\begin{itemize}

	\item{} Yokogawa DPharp EJX110A or EJX910 
	\item{} Honeywell ST3000
	\end{itemize}
\item{} Level

\item{} Temperature
	\begin{itemize}

	\item{} Foxboro RTT15 or RTT30
	\item{} Moore Industries SPT with sourcing (4-wire) 4-20 mA output
	\item{} Moore Industries SPT with sinking (2-wire) 4-20 mA output
	\item{} Moore Industries TRX or TDY
	%\item{} Honeywell STT173
	\end{itemize}
\item{} Flow

	\vskip 2pt
	\item{} Analytical
	\begin{itemize}

	\item{} Daniel 700 gas chromatograph (4 analog output channels)
	\item{} Foxboro 876PH (pH/ORP/ISE)
	\end{itemize}
\end{itemize}

} \hskip 3pt}%
\vskip 5pt \hrule}%
\vrule}
%%%%%%%%%%%%%%%%%%%%%%%%%%%%%%%




%%%%%%%%%%%%%%%%%%%%%%%%%%%%%%%
\vskip 10pt
\filbreak
\hbox{ \vrule
\vbox{ \hrule \vskip 3pt
\hbox{ \hskip 3pt
\vbox{ \hsize=5in \raggedright

\noindent \centerline{\bf Controller options}
\begin{itemize}
\item{} Monolithic
	\begin{itemize}

	\item{} Siemens 353
	\item{} Foxboro 716C
	\item{} Foxboro 718TC
	\item{} Foxboro 762CNA 
	\item{} Moore Industries 535
	\item{} Honeywell UDC2300
	\item{} Honeywell UDC3500 
\end{itemize}
\item{} Modular -- {\it you choose the appropriate I/O module}
	\begin{itemize}

	\item{} Emerson ROC800 SCADA/RTU 
	\end{itemize}
\item{} Distributed Control System (DCS) -- {\it you choose the appropriate I/O module}
	\begin{itemize}

	\item{} Emerson DeltaV with S-series I/O 
	\item{} Honeywell Experion with 2MLF series I/O 
		\end{itemize}
\item{} Programmable Logic Controller (PLC) -- {\it you choose the appropriate I/O module}
	\begin{itemize}

	\item{} Rockwell ControlLogix (catalog number 1756)
	\item{} Rockwell CompactLogix (catalog number 1769)
	\end{itemize}
\end{itemize}

} \hskip 3pt}%
\vskip 5pt \hrule}%
\vrule}
%%%%%%%%%%%%%%%%%%%%%%%%%%%%%%%




%%%%%%%%%%%%%%%%%%%%%%%%%%%%%%%
\vskip 10pt
\filbreak
\hbox{ \vrule
\vbox{ \hrule \vskip 3pt
\hbox{ \hskip 3pt
\vbox{ \hsize=5in \raggedright

\noindent \centerline{\bf 4-20 mA Final Control Element options}
\begin{itemize}
\item{} Pneumatic control valve positioners
	\begin{itemize}

	\item{} Fisher DVC6000 positioner 
	\end{itemize}
\item{} Electrically actuated valves (MOV)
	\begin{itemize}

	\item{} Rotork AQ with Folomatic controller 
	\end{itemize}
\item{} AC motor drives (VFD)
	\begin{itemize}

	\item{} Automation Direct GS1 
	\end{itemize}
\end{itemize}

} \hskip 3pt}%
\vskip 5pt \hrule}%
\vrule}
%%%%%%%%%%%%%%%%%%%%%%%%%%%%%%%


\vfil

Study reference: the ``Analog Electronic Instrumentation'' chapter of {\it Lessons In Industrial Instrumentation}, particularly the section on HART.

\vskip 10pt

Note: a very effective problem-solving strategy for determining how to connect different components together to create a working 4-20 mA current loop is to first identify whether each component acts as a {\it source} or a {\it load} in that loop circuit.  Then, label voltage polarities (+ , $-$) and directions of current accordingly.  Knowing which way current must flow through each component and which polarity each voltage must have is key to ensuring the inter-component connections are correct.

\underbar{file i03773}
%(END_QUESTION)





%(BEGIN_ANSWER)


%(END_ANSWER)





%(BEGIN_NOTES)


%INDEX% Mastery exam performance exercise (circuit), HART transmitter ranging

%(END_NOTES)


