
%(BEGIN_QUESTION)
% Copyright 2011, Tony R. Kuphaldt, released under the Creative Commons Attribution License (v 1.0)
% This means you may do almost anything with this work of mine, so long as you give me proper credit

Is it possible for a radio antenna to have a {\it negative} dBi value?  Why or why not?

\vskip 10pt

Is it possible for a radio antenna to have a {\it negative} dBd value?  Why or why not?

\vskip 10pt

\underbar{file i03840}
%(END_QUESTION)





%(BEGIN_ANSWER)

In order for any antenna to have a negative gain, it must be {\it more} omnidirectional (less directional) than the reference antenna it's being compared to.  An isotropic antenna is perfectly omnidirectional, and so no antenna could ever have a negative dBi value (i.e. be less directional than an isotropic antenna).  However, since an isotropic antenna is less directional than a dipole, an isotropic antenna (if it existed) would have a negative {\it dBd} value.

\vskip 10pt

Of course, poor performance may be realized by other means: damage, mis-operation (e.g. operating at the wrong frequency), etc.

%(END_ANSWER)





%(BEGIN_NOTES)


%INDEX% Electronics review: antennas

%(END_NOTES)

