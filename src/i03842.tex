
%(BEGIN_QUESTION)
% Copyright 2010, Tony R. Kuphaldt, released under the Creative Commons Attribution License (v 1.0)
% This means you may do almost anything with this work of mine, so long as you give me proper credit

Suppose a fellow instrument technician suspects the AC line power in a certain area of the facility is ``corrupted'' by excessive harmonics generated by the many variable-frequency motor drives (VFDs) in that area.  Some sensitive electronic equipment in that area has been doing strange things, and he thinks harmonics may be to blame.

Of course, you could contract an outside expert to perform a {\it power quality audit} within the facility where you work, but your supervisor is too stubborn (and cheap) to hire anyone to figure this out when he has a shop full of bright, eager instrument technicians to do it instead.

\vskip 10pt

Your task is to figure out a way to measure harmonic ``distortion'' of the AC power using relatively simple test equipment already existing in the instrument shop.  In your case (as students), limit yourself to test equipment you might find in a first-year electronics course lab.

\vskip 10pt

After devising your own solution, a grizzled old electrician comes up to you saying ``You don't need any fancy equipment for this job!  Just measure the line voltage using a true-RMS voltmeter and an old-fashioned analog voltmeter, then compare the readings!''  At first you are skeptical, judging by the scorch marks on this electrician's screwdriver (always a bad sign).  But then you begin to wonder if there might be merit to this suggestion.  Explain how simultaneous measurement of line voltage using two different styles of AC voltmeter would tell you {\it anything} about harmonics.

\underbar{file i03842}
%(END_QUESTION)





%(BEGIN_ANSWER)

Analog AC voltmeters are average-responding devices, their calibrations skewed to show voltage in volts AC RMS.  However, this calibration skew is based on the assumption of a pure sine-wave signal.  Thus, an analog AC voltmeter measuring line voltage alongside a modern true-RMS AC voltmeter should agree to within the combined calibration tolerances of the meters.  However, if the line voltage is significantly distorted from pure sinusoidal, the two voltmeters' readings will no longer agree!  The greater the disparity, the worse the distortion!

%(END_ANSWER)





%(BEGIN_NOTES)


%INDEX% Electronics review: detecting harmonic distortion on AC line power

%(END_NOTES)

