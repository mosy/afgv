
%(BEGIN_QUESTION)
% Copyright 2010, Tony R. Kuphaldt, released under the Creative Commons Attribution License (v 1.0)
% This means you may do almost anything with this work of mine, so long as you give me proper credit

Suppose a large grinding machine used in a production machine shop is powered by an induction motor, which in turn receives its electrical power from a VFD.  The time for this machine to coast to a stop after running at full speed is quite long, owing to the mass of the spinning griding wheel.  This ``coast'' time has a negative effect on production, because the operators must wait until the wheel finally stops before they can take the freshly-ground parts off the machine and replace them with new parts to be ground.

Your supervisor would like to shorten this ``stopping'' time by using the {\it dynamic braking} feature of the VFD, which up to this point in time had never been configured for use.  Explain where the stored (kinetic) energy of the spinning grinding wheel goes when the VFD dynamically brakes it to a quick stop.

\vskip 20pt \vbox{\hrule \hbox{\strut \vrule{} {\bf Suggestions for Socratic discussion} \vrule} \hrule}

\begin{itemize}
\item{} What are some alternative braking techniques to dynamic braking?  In each of these techniques, where does the grinding wheel's kinetic energy go during the braking process?
\end{itemize}

\underbar{file i03843}
%(END_QUESTION)





%(BEGIN_ANSWER)

The braking energy here will be dissipated in a braking resistor connected to the VFD.

%(END_ANSWER)





%(BEGIN_NOTES)


%INDEX% Final Control Elements, motor: variable frequency drive (dynamic braking)

%(END_NOTES)

