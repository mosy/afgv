
%(BEGIN_QUESTION)
% Copyright 2010, Tony R. Kuphaldt, released under the Creative Commons Attribution License (v 1.0)
% This means you may do almost anything with this work of mine, so long as you give me proper credit

If you read through the ``Student Performance Objectives'' portion of the syllabus, you will note how many of the assessments within these courses are classified as {\it mastery}.  Based on the descriptions of ``mastery'' objectives you read here, explain what this term means.  How is a ``mastery'' objective different from a regular assignment or test where you receive a point score or letter grade?  Why do you think there are so many mastery-based objectives in a program such as Instrumentation?

\vskip 10pt

The penalty specified in the syllabi for letter-graded courses is to cap the grade at 70\% (a C-) if any mastery objective is not completed by the end of the last day of a course.  If any mastery objective is not complete by the end of the following school day, you will fail the course.  Why do you think these penalties exist for mastery objectives?  Is this comparable to the penalties for failing to complete an assignment by the deadline date at a job?

\underbar{file i03855}
%(END_QUESTION)





%(BEGIN_ANSWER)

``Mastery'' objectives are pass/fail, with opportunities given to re-try until successful.  Every exam in the second year of Instrumentation, for example, contains a ``mastery'' portion, which must be completed 100\% correctly to pass.  You will be given a maximum of two attempts to pass each mastery exam issued to you (with no clues given as to which question or questions were incorrectly answered in the first attempt).  If you do not pass the mastery exam within these two attempts, you may re-take a different version of the same exam on another day, with another two attempts allowed for that version of the exam as well.

There is no limit to the number of mastery exam versions you may sit for (with two attempts allowed on each).  You {\it must} eventually pass the exam, however, to pass a course.  Failing to pass the mastery exam before the next (regular) scheduled exam date will result in failure of that course.

Similarly, there is no limit to the number of times you may attempt any other mastery objective, other than the deadline specified in the course schedule.

%(END_ANSWER)





%(BEGIN_NOTES)


%INDEX% Course organization, assessment: mastery versus proportional assessments

%(END_NOTES)


