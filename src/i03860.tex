
%(BEGIN_QUESTION)
% Copyright 2010, Tony R. Kuphaldt, released under the Creative Commons Attribution License (v 1.0)
% This means you may do almost anything with this work of mine, so long as you give me proper credit

Use a clamp-on ammeter to measure line current to the VFD, and/or from the VFD (to the motor).  Compare the measurement given by the ammeter when it is clamped around a single conductor, versus when it is clamped around multiple conductors.  Does it register differently?  Why or why not?

\vskip 10pt

Another interesting experiment is to loop a single current-carrying conductor around the ammeter's jaws so that it passes through the center of the clamp more than once.  What effect does this arrangement have on the ammeter's reading, and why?

\vskip 20pt \vbox{\hrule \hbox{\strut \vrule{} {\bf Suggestions for Socratic discussion} \vrule} \hrule}

\begin{itemize}
\item{} Identify a practical application for passing all power conductors feeding a VFD through the center of a single current transformer (CT).  What, exactly, would that CT's output signal represent?
\end{itemize}

\underbar{file i03860}
%(END_QUESTION)





%(BEGIN_ANSWER)


%(END_ANSWER)





%(BEGIN_NOTES)

When a clamp-on ammeter is placed around all conductors leading to a load, it should measure zero, because the algebraic sum of all currents to and from a load should be zero.

\vskip 10pt

Interestingly, this application is coined a {\it zero sequence} CT, in honor of ``zero sequence'' equivalent currents that arise from certain imbalanced three-phase conditions when analyzed using the mathematical technique of {\it symmetrical components}.  ``Zero sequence'' currents are currents found in all three lines of a three-phase power system that are precisely in-phase with each other rather than being 120$^{o}$ out of phase with each other as is customary.

%INDEX% Electronics review: clamp-on ammeter (test equipment)

%(END_NOTES)

