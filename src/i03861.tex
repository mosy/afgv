
%(BEGIN_QUESTION)
% Copyright 2010, Tony R. Kuphaldt, released under the Creative Commons Attribution License (v 1.0)
% This means you may do almost anything with this work of mine, so long as you give me proper credit

Read and discuss the bullet-point suggestions given in ``Question 0'' of this worksheet on how to maximize your reading effectiveness.  Then, apply these tips to an actual document: pages 81 through 89 of the {\it Report of the President's Commission on The Accident at Three Mile Island}, where the prologue to the ``Account of the Accident'' chapter explains the basic workings of a nuclear power plant.

\vskip 10pt

After taking about half an hour in class to actively read these nine pages -- either individually or in groups -- discuss what you were able to learn about nuclear power plant operation from the text, and also how active reading helps you maximize the learning experience.

\underbar{file i03861}
%(END_QUESTION)





%(BEGIN_ANSWER)

An anecdote to relate regarding active reading on challenging subjects is when I had to study policy statements at BTC in preparation for an accreditation audit.  The texts were long, boring, and I had little interest in their particulars.  I found myself nodding off as I tried to read the policy statements, and unable to explain the meaning of what I had just read.  Finally, I forced myself to outline each section of these policy papers in my own words, paragraph by paragraph, until I could articulate their meaning.  To be sure, this technique took longer than simply reading the text, but it was {\it far} more effective than plain reading (even with underlining and highlighting!).

I've successfully applied similar strategies studying labor contracts for my work with the union at BTC.  Several times I've been called upon to research policies in other college contracts, and I have done so (again) by summarizing their statements in my own words to ensure I am comprehending them as I read.

%(END_ANSWER)





%(BEGIN_NOTES)

Here are my own active-reading notes from this exercise, typed over a period of 17 minutes actively reading the text:

\vskip 10pt

Nuclear reactors produce heat to turn water into steam.  This heat comes from fission -- the splitting of uranium nuclei by free neutrons in a chain reaction.  Impacting neutrons cause uranium nuclei to split, releasing heat and more neutrons to split other uranium nuclei.

Uranium fuel encased in ``Zircalloy'' tubes (fuel rods) to allow heat to transfer to water while being transparent to neutrons.  Reactor 2 contains over 36,000 fuel rods, with 69 control rods and 52 instrumentation rods.  

Control rods absorb neutrons, damping or quenching chain-reaction.  Inserting rods stops reaction, withdrawing rods speeds it up.  Power output of reactor controlled by number and depth of control rod insertion.  Magnetic release of control rods (gravity-drop) is called a ``scram'' to shut down the reactor immediately.

Reactors cannot explode like bombs, but they have potential to release dangerous radioactive materials.  Three barriers to release: fuel rods, reactor vessel and piping, and finally the concrete containment building.

Reactor water pressurized (2155 PSI) to prevent boiling.  Pressure generated by a unit called the ``pressurizer'' nearly half-full of water and half-full of steam.  If boiling occurs in the reactor, water level may drop and expose fuel rods.  This may lead to overheating, and this to hydrogen production and potential release of radioactive materials into the cooling water.

``Primary'' water coolant loop transfers heat from the reactor core to the steam generator (heat exchanger), where ``secondary'' water boils and becomes steam to turn the steam turbine to make electricity.  Four 9000 HP pumps circulate the primary coolant.  Steam exiting turbine gets re-condensed into water in another heat exchanger (the condenser) by another water loop sent to the large cooling towers.  Only the primary water is radioactive during normal operation.  All equipment circulating this primary water is housed inside the containment building.  The secondary loop equipment is outside the containment.

In a Loss-Of-Cooling-Accident (LOCA), an emergency coolant system exists to maintain reactor core water coverage.  Operators did not keep the High Pressure Injection (HPI) emergency coolant pumps running as they should have!

%INDEX% Reading assignment: worksheet information (question 0)

%(END_NOTES)


