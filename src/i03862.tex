
%(BEGIN_QUESTION)
% Copyright 2014, Tony R. Kuphaldt, released under the Creative Commons Attribution License (v 1.0)
% This means you may do almost anything with this work of mine, so long as you give me proper credit

Read and outline the introduction to the ``Introduction to Industrial Instrumentation'' chapter in your {\it Lessons In Industrial Instrumentation} textbook.  Note the page numbers where important illustrations, photographs, equations, tables, and other relevant details are found.  Prepare to thoughtfully discuss with your instructor and classmates the concepts and examples explored in this reading, including the following definitions:

\vskip 10pt

\begin{itemize}
\item{} {\bf Process}
\vskip 10pt
\item{} {\bf Process Variable (PV)}
\vskip 10pt
\item{} {\bf Setpoint (SP)}
\vskip 10pt
\item{} {\bf Primary Sensing Element (PSE)}
\vskip 10pt
\item{} {\bf Transducer}
\vskip 10pt
\item{} {\bf Lower Range Value (LRV)}
\vskip 10pt
\item{} {\bf Upper Range Value (URV)}
\vskip 10pt
\item{} {\bf Zero}
\vskip 10pt
\item{} {\bf Span}
\vskip 10pt
\item{} {\bf Controller}
\vskip 10pt
\item{} {\bf Final Control Element (FCE)}
\vskip 10pt
\item{} {\bf Manipulated Variable (MV)} or {\bf Output}
\vskip 10pt
\medskip

\vskip 20pt \vbox{\hrule \hbox{\strut \vrule{} {\bf Suggestions for Socratic discussion} \vrule} \hrule}

\begin{itemize}
\item{} As a student in an ``inverted'' classroom, your role as a learner is substantially different from that of a student in a lecture-based classroom.  Rather than receive information from the instructor via lecture, you are tasked with gathering this information on your own outside of class.  What, then, will you do during class time with the instructor?  If there is no lecture, how is class time spent and for what purpose?
\item{} What should you do if you arrive to class having not understood parts of what you studied in preparation?
\item{} If you are new to an inverted classroom format, describe how this shift will affect your approach to learning.
\end{itemize}

\underbar{file i03862}
%(END_QUESTION)





%(BEGIN_ANSWER)


%(END_ANSWER)





%(BEGIN_NOTES)

Instrumentation is the science of measurement and control.

\vskip 10pt

Measuring device $\to$ Controller $\to$ Final control element $\to$ Process $\to$ (loop)

\vskip 10pt

\noindent
Home heating/cooling system:
\item{} Thermostat acts as measuring device and control device
\item{} Heater/AC acts as final control element
\item{} ``Setpoint'' = desired temperature
\end{itemize}

\begin{itemize}
\item{} Process = what it is we're trying to measure and/or control
\item{} PV = specific variable being measured/controlled
\item{} SP = desired value of PV
\item{} PSE = sensor doing the measurement
\item{} Transducer = converts between standard signals, or processes that signal
\item{} LRV = 0\% point of a transmitter's calibrated range
\item{} URV = 100\% point of a transmitter's calibrated range
\item{} Zero = LRV
\item{} Span = URV - LRV
\item{} Controller = the decision-making part of a control system
\item{} FCE = ``muscle'' of the control system
\item{} MV = controller output signal
\end{itemize}










\vskip 20pt \vbox{\hrule \hbox{\strut \vrule{} {\bf Suggestions for Socratic discussion} \vrule} \hrule}

\begin{itemize}
\item{} Feedback control systems are often referred to as {\it control loops}.  Explain why the word ``loop'' is appropriate to the description of a control system.
\item{} If a pressure transmitter has a calibrated range of 100 to 1000 PSI, what is its LRV and URV?  What is its zero and span?
\item{} If a pressure transmitter has a calibrated range of $-10$ to +10 PSI, what is its LRV and URV?  What is its zero and span?
\item{} If a pressure transmitter has a calibrated range of $-25$ kPa to $-70$ kPa, what is its LRV and URV?  What is its zero and span?
\item{} If a temperature transmitter has a calibrated range of 500 to 1200 degrees C, what is its LRV and URV?  What is its zero and span?
\item{} If a temperature transmitter has a calibrated range of $-250$ to +300 degrees F, what is its LRV and URV?  What is its zero and span?
\item{} If a flow transmitter has a calibrated range of 0 to 40 liters per minute, what is its LRV and URV?  What is its zero and span?
\item{} If a level transmitter has a calibrated range of 10 to 50 inches, what is its LRV and URV?  What is its zero and span?
\item{} If a pH transmitter has a calibrated range of 4 to 12 pH, what is its LRV and URV?  What is its zero and span?
\item{} If an oxygen transmitter has a calibrated range of 0 to 90\%, what is its LRV and URV?  What is its zero and span?
\end{itemize}





%INDEX% Reading assignment: Lessons In Industrial Instrumentation, Introduction to Industrial Instrumentation

%(END_NOTES)


