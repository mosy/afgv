
%(BEGIN_QUESTION)
% Copyright 2015, Tony R. Kuphaldt, released under the Creative Commons Attribution License (v 1.0)
% This means you may do almost anything with this work of mine, so long as you give me proper credit

Read and outline the ``Example: Boiler Water Level Control System'' section of the ``Introduction to Industrial Instrumentation'' chapter in your {\it Lessons In Industrial Instrumentation} textbook.  Note the page numbers where important illustrations, photographs, equations, tables, and other relevant details are found.  Prepare to thoughtfully discuss with your instructor and classmates the concepts and examples explored in this reading.

\vskip 20pt \vbox{\hrule \hbox{\strut \vrule{} {\bf Active reading tip} \vrule} \hrule}

In order to read a text {\it actively}, your mind needs to be fully attentive to the words on the page.  This is why you are asked to {\it write an outline} of the text you've been assigned to read.  This means much more than just highlighting and underlining words, but actually expressing what you have learned {\it in your own words}.  Your instructor will check your outline for this level of engagement when you come to the ``inverted'' class session to present what you have learned.

If you discover a section of the text that you just can't seem to summarize in your own words, it is an indication to you that you're not comprehending that section of text.  This is one of the benefits of writing an outline: it serves as a self-check for understanding, whereas highlighting and underlining does not.

\vskip 10pt

\underbar{file i03863}
%(END_QUESTION)





%(BEGIN_ANSWER)


%(END_ANSWER)





%(BEGIN_NOTES)

Control system needed to maintain a steady level of water in the upper drum of a steam boiler.

\vskip 10pt

LT senses drum water level, reports as a pneumatic (air pressure) signal.  LIC receives LT's signal, compares against SP, then sends MV signal to valve.  LV moved by controller's air pressure signal, returned by spring action, throttles make-up water flow into drum.

\vskip 10pt

A controller in automatic mode moves the valve to whatever position is needed to keep PV = SP.  This means the valve will be moved by the controller in response not only to SP changes but also steam demand changes.  In other words, the valve's position is not just a function of the SP value -- valve position is also a function of ``how hard'' the control system is trying in order to maintain PV = SP.

\vskip 10pt

A controller placed in manual mode moves the valve to whatever position deemed by human operator.  It will still display the PV, but a controller in manual mode takes no action to regulate the PV.  In other words, its output signal will be unresponsive to changes in PV or SP while in manual mode.  Manual mode is a useful diagnostic tool, for de-coupling the PV from the Output, allowing you to diagnose (for instance) the cause behind an erratic PV signal.  If placing the controller in manual mode stabilizes the PV, then the controller is at fault; if placing the controller in manual mode results in an even wilder PV, then a process load is at fault.

\vskip 10pt

This particular control system uses compressed air as the signaling medium.  3 to 15 PSI represents 0\% to 100\% of signal range:

\begin{itemize}
\item{} 3 PSI = 0\%
\item{} 9 PSI = 50\%
\item{} 15 PSI = 100\%
\item{} 0\% and 100\% of measurement range need not be absolute limits of process!
\end{itemize}

\vskip 10pt

Even though the PV and MV signals in a control loop typically use the same range (e.g. 3-15 PSI or 4-20 mA), we have no reason to expect that PV = MV except by chance.  PV and MV represent two entirely different variables in the process, despite using the same type and range of representative signal.





\vskip 20pt \vbox{\hrule \hbox{\strut \vrule{} {\bf Suggestions for Socratic discussion} \vrule} \hrule}

\begin{itemize}
\item{} {\bf This is a good opportuity to emphasize active reading strategies as you check students' comprehension of today's homework, because it will set the pace for your students' homework completion from here on out.  I strongly recommend challenging students to apply the ``Active Reading Tips'' given in this and other questions in today's assignment, making this the primary focus and the instrumentation concepts the secondary focus.}
\item{} Explain how manual mode differs from automatic mode in a controller, using a practical example to illustrate.
\item{} Explain why a process transmitter might not be ranged to indicate the full physical limits of the process variable (e.g. a steam drum level transmitter outputting 0\% at 40\% full and outputting 100\% at 60\% full).
\item{} How will the steam drum control loop respond to an increase in steam demand with the controller in automatic mode?
\item{} How will the steam drum control loop respond to an increase in steam demand with the controller in manual mode?
\item{} What would happen in this process if the LT failed with a low signal, with the controller in automatic mode?
\item{} What would happen in this process if the LT failed with a high signal, with the controller in automatic mode?
\item{} What would happen in this process if the LT failed with a low signal, with the controller in manual mode?
\item{} What would happen in this process if the LT failed with a high signal, with the controller in manual mode?
\item{} What would happen in this process if the air tube connecting the LIC to the valve sprung a leak, with the controller in automatic mode?
\item{} What would happen in this process if the air tube connecting the LIC to the valve sprung a leak, with the controller in manual mode?
\item{} Why might we calibrate the range of a transmitter to be something other than the extreme limits of the process?  For example, in this steam drum level control system, the LT might be calibrated so that it registered 0\% level when the drum was actually 30\% full, and register 100\% level when the drum was actually 70\% full.
\end{itemize}









\vfil \eject

\noindent
{\bf Prep Quiz:}

\vskip 10pt

\noindent
\vbox{\hrule \hbox{\strut \vrule{} {Part A -- multiple-choice} \vrule} \hrule}
If the person responsible for operating a steam boiler in a power plant places the steam drum water level controller into ``manual'' mode, it means:

\begin{itemize}
\item{} Any alarm indications or warning buzzers will be ``mute'' and unable to alert anyone
\vskip 5pt
\item{} The controller's setpoint value will now be fixed at a constant value of 50\% 
\vskip 5pt
\item{} Additional automatic safety measures will now be in effect to guard against overflowing
\vskip 5pt
\item{} That feedwater control valve will strictly do only what the human operator tells it to
\vskip 5pt
\item{} The controller will now be hyper-responsive to any changes in steam drum water level
\vskip 5pt
\item{} The controller's action will be switched from direct to reverse, or vice-versa
\end{itemize}

\vskip 20pt

\noindent
\vbox{\hrule \hbox{\strut \vrule{} {Part B -- written response} \vrule} \hrule}
Explain how you can identify the specific topics covered on any upcoming ``mastery'' exam.

\vskip 20pt

\noindent
\vbox{\hrule \hbox{\strut \vrule{} {Part C -- written response} \vrule} \hrule}
Explain in your own words how the content of a ``proportional'' exam differs from that of a ``mastery'' exam.  In other words, what kinds of challenge(s) will you find on a proportional exam that you will not find on a mastery exam?

\vskip 20pt

{\it Note: your explanations need to be \underbar{complete} and \underbar{clearly written}.  Expressing your ideas clearly and completely is every bit as important as having those ideas correct in your own mind!}















\vfil \eject

\noindent
{\bf Prep Quiz:}

\vskip 10pt

\noindent
\vbox{\hrule \hbox{\strut \vrule{} {Part A -- multiple-choice} \vrule} \hrule}
A common misconception among students first learning about instrumentation and control systems is that the signal output by a controller must have the same value at all times as the signal input to the controller from the transmitter.  The reason this is {\it not} true is because:

\begin{itemize}
\item{} As the transmitter signal increases, the controller's output signal must decrease
\vskip 5pt
\item{} These two signals represent entirely different aspects of the process being controlled
\vskip 5pt
\item{} These two signals are always of a different type (e.g. 3-15 PSI versus 4-20 mA)
\vskip 5pt
\item{} In manual mode the output of a controller is fixed by the human operator
\vskip 5pt
\item{} In automatic mode the output of a controller always follows the PV signal
\vskip 5pt
\item{} Tony says so
\end{itemize}

\vskip 20pt

\noindent
\vbox{\hrule \hbox{\strut \vrule{} {Part B -- written response} \vrule} \hrule}
Explain how you can identify the specific topics covered on any upcoming ``mastery'' exam.

\vskip 20pt

\noindent
\vbox{\hrule \hbox{\strut \vrule{} {Part C -- written response} \vrule} \hrule}
Explain in your own words how the content of a ``proportional'' exam differs from that of a ``mastery'' exam.  In other words, what kinds of challenge(s) will you find on a proportional exam that you will not find on a mastery exam?

\vskip 20pt

{\it Note: your explanations need to be \underbar{complete} and \underbar{clearly written}.  Expressing your ideas clearly and completely is every bit as important as having those ideas correct in your own mind!}



%INDEX% Reading assignment: Lessons In Industrial Instrumentation, Introduction to Industrial Instrumentation

%(END_NOTES)


