
%(BEGIN_QUESTION)
% Copyright 2009, Tony R. Kuphaldt, released under the Creative Commons Attribution License (v 1.0)
% This means you may do almost anything with this work of mine, so long as you give me proper credit

Read and outline the ``Example: Wastewater Disinfection'' section of the ``Introduction to Industrial Instrumentation'' chapter in your {\it Lessons In Industrial Instrumentation} textbook.  Note the page numbers where important illustrations, photographs, equations, tables, and other relevant details are found.  Prepare to thoughtfully discuss with your instructor and classmates the concepts and examples explored in this reading.

\vskip 20pt \vbox{\hrule \hbox{\strut \vrule{} {\bf Active reading tip} \vrule} \hrule}

One of the distinctive differences between {\it technical} reading and the reading of other document types is the degree to which the reader needs to jump back and forth between the words of the text and the illustrations.  Identify portions of this reading assignment where it would be wise to stop reading the words and switch your attention to one or more illustrations, in order to put context to those words.

\vskip 10pt

\underbar{file i03864}
%(END_QUESTION)





%(BEGIN_ANSWER)


%(END_ANSWER)





%(BEGIN_NOTES)

Mixing chlorine gas with wastewater to kill off harmful bacteria before the water enters the environment.  It is important not to inject too much chlorine into the water, or inject too little.

\vskip 10pt

AT = ``Analytical Transmitter'' measures chlorine concentration.  Dashed line means electrical signal.  All instruments in a loop are labeled with a tag, the first letter of which defines the variable being controlled.  Therefore, in this chlorine loop, the transmitter is ``AT'' and the controller is ``AIC''.

\vskip 10pt

This system uses analog electronic signaling instead of pneumatic (air pressure) signals.  The signal range here is 4 to 20 mA DC:

\begin{itemize}
\item{} 4 mA = 0\%
\item{} 12 mA = 50\%
\item{} 20 mA = 100\%
\end{itemize}

Dashed lines in the diagram symbolize an electronic signal path, just as solid lines with double-slash marks represented pneumatic signal paths in the last (boiler) system.

\vskip 10pt

Even though both the PV and MV in this process are represented by 4-20 mA analog signals, we should not expect PV = MV except by chance.

\vskip 10pt

AIC is controller deciding how much more chlorine to add.  Valve is motor-actuated, and may be made reverse-acting (i.e. 4 mA = wide open and 20 mA = shut) if desired.

\vskip 10pt

Motor-actuated control valve has positioner inside making shaft position match signal value.








\vskip 20pt \vbox{\hrule \hbox{\strut \vrule{} {\bf Suggestions for Socratic discussion} \vrule} \hrule}

\begin{itemize}
\item{} {\bf This is a good opportuity to emphasize active reading strategies as you check students' comprehension of today's homework, because it will set the pace for your students' homework completion from here on out.  I strongly recommend challenging students to apply the ``Active Reading Tips'' given in this and other questions in today's assignment, making this the primary focus and the instrumentation concepts the secondary focus.}
\item{} What would happen in this process if the AT failed with a low signal, with the controller in automatic mode?
\item{} What would happen in this process if the AT failed with a high signal, with the controller in automatic mode?
\item{} What would happen in this process if the AT failed with a low signal, with the controller in manual mode?
\item{} What would happen in this process if the AT failed with a high signal, with the controller in manual mode?
\item{} What would happen in this process if the cable connecting the AIC to the valve failed open, with the controller in automatic mode?
\item{} What would happen in this process if the cable connecting the AIC to the valve failed open, with the controller in manual mode?
\item{} What would happen in this process if the chlorine gas supply ran out, with the controller in automatic mode?
\item{} What would happen in this process if the chlorine gas supply ran out, with the controller in manual mode?
\item{} What would happen in this process if the influent flow rate increased, with the controller in automatic mode?
\item{} What would happen in this process if the influent flow rate increased, with the controller in manual mode?
\end{itemize}


%INDEX% Reading assignment: Lessons In Industrial Instrumentation, Introduction to Industrial Instrumentation

%(END_NOTES)


