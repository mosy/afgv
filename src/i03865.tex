
%(BEGIN_QUESTION)
% Copyright 2009, Tony R. Kuphaldt, released under the Creative Commons Attribution License (v 1.0)
% This means you may do almost anything with this work of mine, so long as you give me proper credit

Read and outline the ``Example: Chemical Reactor Temperature Control'' section of the ``Introduction to Industrial Instrumentation'' chapter in your {\it Lessons In Industrial Instrumentation} textbook.  Note the page numbers where important illustrations, photographs, equations, tables, and other relevant details are found.  Prepare to thoughtfully discuss with your instructor and classmates the concepts and examples explored in this reading.

\vskip 20pt \vbox{\hrule \hbox{\strut \vrule{} {\bf Active reading tip} \vrule} \hrule}

Well-written technical texts always model problem-solving strategies for the reader.  In this particular section of text, a problem-solving technique called a {\it thought experiment} was applied to a particular point.  Explain what a ``thought experiment'' is, and how this technique was applied in the problem at hand in the text.  Furthermore, identify ways you might be able to apply ``thought experiments'' of your own when reading technical texts in the future.

\vskip 10pt

\underbar{file i03865}
%(END_QUESTION)





%(BEGIN_ANSWER)


%(END_ANSWER)





%(BEGIN_NOTES)

Heat added to reactor by introducing steam to a ``jacket'' surrounding the vessel. 

\vskip 10pt

TT is digital (fieldbus) instead of analog (4-20 mA).  Temperature represented by electrical pulses conveying binary number data.  Fieldbus transmitters can also report other data, like self-diagnostics.

\vskip 10pt

TIC output goes to TY, converting 4-20 mA into 3-15 PSI (transducer).  TV is a pneumatically actuated valve, air-to-open (ATO).

\vskip 10pt

The controller in this process must be configured for {\it reverse action}, so that when temperature rises above setpoint, the controller will output a decreasing signal to the control valve which will cause that valve to close more and reduce steam to the reactor.  A direct-acting controller in the steam-heating system would move the valve the wrong way (further open) if the temperature became too hot.  However, direct action is perfect for the glycol-chilled beer fermenting system shown later.  Controller action is a user-configurable parameter.

This section of the book introduces the concept of a {\it thought experiment} to the reader: imagining the response of a control system to a change in its process variable.  By considering which direction the automatic controller's output signal must vary to stabilize the process as a result of the PV's change, we may determine the necessary direction of control action.

\vskip 10pt

PT is a digital wireless (radio) unit.  Like fieldbus, only no copper wires!  At this writing (2011), wireless not recommended for mission-critical control applications, but mostly monitoring.  Battery power and potential blockage of radio signal pathways makes wireless a less reliable technology than wired instrumentation.








\vskip 20pt \vbox{\hrule \hbox{\strut \vrule{} {\bf Suggestions for Socratic discussion} \vrule} \hrule}

\begin{itemize}
\item{} {\bf This is a good opportuity to emphasize active reading strategies as you check students' comprehension of today's homework, because it will set the pace for your students' homework completion from here on out.  I strongly recommend challenging students to apply the ``Active Reading Tips'' given in this and other questions in today's assignment, making this the primary focus and the instrumentation concepts the secondary focus.}
\item{} A powerful problem-solving technique is performing a {\it thought experiment} where you mentally simulate the response of a system to some imagined set of conditions.  Explain how such a ``thought experiment'' was used in the text to explore the operation of the chemical reactor temperature control system.
\item{} What dictates the correct action (either direct or reverse) for a loop controller?  How do we determine the correct action for any given process?  What might happen if a controller is configured for the wrong direction of action?
\item{} What would happen in this process if the TT failed with a low signal, with the controller in automatic mode?
\item{} What would happen in this process if the TT failed with a high signal, with the controller in automatic mode?
\item{} What would happen in this process if the TT failed with a low signal, with the controller in manual mode?
\item{} What would happen in this process if the TT failed with a high signal, with the controller in manual mode?
\item{} What would happen in this process if the air tube connecting the TY to the valve sprung a leak, with the controller in automatic mode?
\item{} What would happen in this process if the air tube connecting the TY to the valve sprung a leak, with the controller in manual mode?
\item{} What would happen in this process if the boiler supplying steam shut down, with the controller in automatic mode?
\item{} What would happen in this process if the boiler supplying steam shut down, with the controller in manual mode?
\item{} What would happen in this process if the PT failed with a low signal, with the controller in automatic mode?
\item{} What would happen in this process if the PT failed with a high signal, with the controller in automatic mode?
\item{} What dictates the proper direction of action (e.g. {\it direct} or {\it reverse}) for a loop controller?
\end{itemize}


%INDEX% Reading assignment: Lessons In Industrial Instrumentation, Introduction to Industrial Instrumentation

%(END_NOTES)


