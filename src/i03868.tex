
%(BEGIN_QUESTION)
% Copyright 2009, Tony R. Kuphaldt, released under the Creative Commons Attribution License (v 1.0)
% This means you may do almost anything with this work of mine, so long as you give me proper credit

Read and outline the ``Process Switches and Alarms'' subsection of the ``Other Types of Instruments'' section of the ``Introduction to Industrial Instrumentation'' chapter in your {\it Lessons In Industrial Instrumentation} textbook.  Note the page numbers where important illustrations, photographs, equations, tables, and other relevant details are found.  Prepare to thoughtfully discuss with your instructor and classmates the concepts and examples explored in this reading.

\vskip 20pt \vbox{\hrule \hbox{\strut \vrule{} {\bf Active reading tip} \vrule} \hrule}

In an ``inverted'' learning environment where assignments such as this substitute for instructor-driven lecture, there is more opportunity for students to share what they have learned with each other.  When you meet with your instructor today to review the material, share the points you found in today's reading that were clear to you, as well as the points you found confusing.  This will stimulate valuable conversation over the text, and prompt everyone to think deeper about it.

\vskip 10pt

\underbar{file i03868}
%(END_QUESTION)





%(BEGIN_ANSWER)


%(END_ANSWER)





%(BEGIN_NOTES)

Process switches = like transmitters, but strictly on/off and not analog.

\vskip 10pt

PSH and PSL on air compressor control system, telling PLC when to start and stop the compressor motor.  PSHH activates only if pressure gets really high!  Like other instruments in a system, the first letter of an instrument's label tag describes the type of process variable being measured and/or controlled.

\vskip 10pt

Alarm switch units actuate off the analog signal from a transmitter, rather than directly sense the process.  This is more economical than using redundant process-sensing switches.

\vskip 10pt

AAH and AAL electronic switch units operating off 4-20 mA signal from AT: gives alarm functionality with just one complex analyzer.  The Moore Industries model SPA is an example of a process alarm switch built to sense a 4-20 mA signal from a transmitter.

\vskip 10pt

LSH and LSL pneumatic switch units operating off 3-15 PSI signal from LT.  Still called ``Level'' switches even though they are actuated by 3-15 PSI signal ``pressure''.

\vskip 10pt

Process alarms triggered by direct process-sensing switches are not dependent on one instrument the way alarms triggered by the output of a process transmitter are!  Must be aware of ``common-cause'' failures when we design an instrument system.

\vskip 10pt

\noindent
Annunciator (a ``latching'' indicator to warn that an event has happened):
\item{} (1) Alarm condition begins
\item{} (2) Light blinks and horn sounds
\item{} (3) Push ``Acknowledge'' button
\item{} (4) Light steady, no horn
\item{} (5) Light goes out when alarm condition clears
\end{itemize}

Panel-mounted annunciators becoming obsolete, being replaced by digital computer recording devices which time-stamp events and provide information on event sequence.








\vskip 20pt \vbox{\hrule \hbox{\strut \vrule{} {\bf Suggestions for Socratic discussion} \vrule} \hrule}

\begin{itemize}
\item{} {\bf This is a good opportuity to emphasize active reading strategies as you check students' comprehension of today's homework, because it will set the pace for your students' homework completion from here on out.  I strongly recommend challenging students to apply the ``Active Reading Tips'' given in this and other questions in today's assignment, making this the primary focus and the instrumentation concepts the secondary focus.}
\item{} What purpose do process switches and alarms serve in control systems where there already exist indicators, recorders, and controllers?
\item{} Explain how it is possible to add alarming capabilities to a control system in two different ways: one by alarming based on the signal output by an existing process transmitter, and another by directly sensing the process variable using process switches.  Also, elaborate on the pros and cons of each approach.
\item{} What would happen in the chlorine disinfection control system if the chlorine analyzer failed with a low signal, with the controller in automatic mode?
\item{} What would happen in the chlorine disinfection control system if the chlorine analyzer failed with a high signal, with the controller in automatic mode?
\item{} What would happen in the steam drum level control system (using alarm switches connected to the output of the pneumatic transmitter) if the level transmitter failed with a low signal, with the level controller in automatic mode?
\item{} What would happen in the steam drum level control system (using independent level-sensing alarm switches) if the level transmitter failed with a low signal, with the level controller in automatic mode?
\item{} Explain how an {\it event recorder} may provide more useful information about a process shut-down than a simple {\it annunciator}.
\end{itemize}


%INDEX% Reading assignment: Lessons In Industrial Instrumentation, Introduction to Industrial Instrumentation

%(END_NOTES)


