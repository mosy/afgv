
%(BEGIN_QUESTION)
% Copyright 2010, Tony R. Kuphaldt, released under the Creative Commons Attribution License (v 1.0)
% This means you may do almost anything with this work of mine, so long as you give me proper credit

An instrument technician finds a great deal on a lathe for his home machine shop.  The only problem is, this lathe has a three-phase motor and his shop only has single-phase electric power.  This technician knows, however, that you can wire most VFDs to input single-phase AC power and output three-phase AC power, so he buys a used VFD and wires it up in this manner to power his lathe.

Not only does this ``fix'' allow him to run his lathe on single-phase power, but it also gives him the ability to vary the lathe's motor speed, and also to suddenly stop it if needed.  Now, this particular VFD is an inexpensive model, and it has no braking resistor connected to it.  Based on this information, identify the likely technique this VFD uses to brake the lathe motor, and identify where the lathe's kinetic energy will be dissipated.

\vskip 20pt \vbox{\hrule \hbox{\strut \vrule{} {\bf Suggestions for Socratic discussion} \vrule} \hrule}

\begin{itemize}
\item{} What are some alternative braking techniques to dynamic braking?  In each of these techniques, where does the grinding wheel's kinetic energy go during the braking process?
\item{} Explain what would happen to a VFD with dynamic braking if the braking resistor failed open.
\item{} Explain what would happen to a VFD with dynamic braking if the braking resistor failed shorted.
\item{} Explain what would happen to a VFD's braking ability if the circuit breaker feeding AC line power to it were to trip (open).
\end{itemize}

\underbar{file i03871}
%(END_QUESTION)





%(BEGIN_ANSWER)


%(END_ANSWER)





%(BEGIN_NOTES)

This VFD most likely uses {\it DC injection} braking to slow down the lathe.  Another possibility is {\it plugging}.

\vskip 10pt

Regenerative braking is unlikely, because this is an inexpensive VFD, and only expensive VFDs typically have this option.

\vskip 10pt

Both DC injection and plugging use the motor itself as the dissipating element.  Thus, the technician will have to be wary of overheating the motor if he uses the braking option a lot.

%INDEX% Final Control Elements, motor: variable frequency drive (DC injection braking)

%(END_NOTES)

