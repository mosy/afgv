
%(BEGIN_QUESTION)
% Copyright 2009, Tony R. Kuphaldt, released under the Creative Commons Attribution License (v 1.0)
% This means you may do almost anything with this work of mine, so long as you give me proper credit

Read and outline the ``4 to 20 mA Analog Current Signals'' section of the ``Analog Electronic Instrumentation'' chapter in your {\it Lessons In Industrial Instrumentation} textbook.  Note the page numbers where important illustrations, photographs, equations, tables, and other relevant details are found.  Prepare to thoughtfully discuss with your instructor and classmates the concepts and examples explored in this reading.

\vskip 20pt \vbox{\hrule \hbox{\strut \vrule{} {\bf Active reading tip} \vrule} \hrule}

A practical strategy for reading any text is to imagine yourself in the position of a teacher who must explain the content of the text to a group of students.  Write your outline in such a way that it would make sense to students encountering this subject for the first time if your outline were used as notes for a teacher's lecture.  Compare your written outline to that of classmates, to see how they chose to explain this same concept.

\vskip 10pt

\underbar{file i03872}
%(END_QUESTION)





%(BEGIN_ANSWER)


%(END_ANSWER)





%(BEGIN_NOTES)

4 to 20 mA is a common analog signal standard: milliamp value proportionately represents a real-world variable (pressure, temperature, etc.).  Control systems usually utilize two 4-20 mA signals: one for PV, one for MV.

% No blank lines allowed between lines of an \halign structure!
% I use comments (%) instead, so that TeX doesn't choke.

$$\vbox{\offinterlineskip
\halign{\strut
\vrule \quad\hfil # \ \hfil & 
\vrule \quad\hfil # \ \hfil \vrule \cr
\noalign{\hrule}
%
% First row
{\bf Current value} & {\bf \% of scale} \cr
%
\noalign{\hrule}
%
% Another row
4 mA & 0\% \cr
%
\noalign{\hrule}
%
% Another row
8 mA & 25\% \cr
%
\noalign{\hrule}
%
% Another row
12 mA & 50\% \cr
%
\noalign{\hrule}
%
% Another row
16 mA & 75\% \cr
%
\noalign{\hrule}
%
% Another row
20 mA & 100\% \cr
%
\noalign{\hrule}
} % End of \halign 
}$$ % End of \vbox

Both 3-15 PSI and 4-20 mA are {\it live zero} standards, because 0\% $\neq$ 0 PSI or 0 mA.

\vskip 10pt

In an analog instrumentation system, all components must be compatibly {\it ranged} in order to maintain the proper representation of measurement all the way through the system.  For example, in the temperature measurement system shown, the millivolt signal range output by the thermocouple needs to generate a 4-20 mA current signal representing the temperature range of 50 to 250 $^{o}$C.  This requires that the temperature transmitter be ``ranged'' by an instrument technician to output the right amount of current for any given millivoltage signal input by the thermocouple.  Each instrument in loop must output a range compatible with the next instrument in the loop (i.e. the output range of one instrument becomes the input range of the next).

\vskip 10pt

PV and MV signals are not identical, but are related to each other when the controller is in automatic mode.  In manual mode, the MV is arbitrarily set by the human operator.










\vskip 20pt \vbox{\hrule \hbox{\strut \vrule{} {\bf Suggestions for Socratic discussion} \vrule} \hrule}

\begin{itemize}
\item{} {\bf This is a good opportuity to emphasize active reading strategies as you check students' comprehension of today's homework, because it will set the pace for your students' homework completion from here on out.  I strongly recommend challenging students to apply the ``Active Reading Tips'' given in this and other questions in today's assignment, making this the primary focus and the instrumentation concepts the secondary focus.}
\item{} Explain what a {\it live zero} signal range is in an instrumentation system.
\item{} Explain the importance of {\it ranging} in an analog instrumentation system such as the temperature measurement system shown in this section of the book.
\item{} Suppose the TT in this system was properly calibrated for a 50 to 250 $^{o}$C range.  How many milliamps would be in this circuit at a sensed temperature of 100 $^{o}$C?  {\it Answer = 8 mA} 
\item{} Suppose the TT in this system was properly calibrated for a 50 to 250 $^{o}$C range.  How many milliamps would be in this circuit at a sensed temperature of 200 $^{o}$C?  {\it Answer = 16 mA} 
\item{} Suppose the TT in this system was properly calibrated for a 50 to 250 $^{o}$C range.  How much voltage would be dropped across the 250 ohm resistor at a sensed temperature of 50 $^{o}$C?  {\it Answer = 1 volt}
\item{} Suppose the TT in this system was properly calibrated for a 50 to 250 $^{o}$C range.  How much voltage would be dropped across the 250 ohm resistor at a sensed temperature of 100 $^{o}$C?  {\it Answer = 2 volts}
\item{} Suppose the TT in this system was properly calibrated for a 50 to 250 $^{o}$C range.  How much voltage would be dropped across the 250 ohm resistor at a sensed temperature of 150 $^{o}$C?  {\it Answer = 3 volts}
\item{} Suppose the TT in this system was properly calibrated for a 50 to 250 $^{o}$C range.  How much voltage would be dropped across the 250 ohm resistor at a sensed temperature of 200 $^{o}$C?  {\it Answer = 4 volts}
\item{} Suppose the TT in this system was properly calibrated for a 50 to 250 $^{o}$C range, but that the indicating meter (TI) had the wrong range on its scale: 100 to 300 $^{o}$C.  What temperature value would the TI indicate when the real temperature was 150 $^{o}$C?  {\it Answer = 200 $^{o}$C}
\item{} Suppose the TT in this system was properly calibrated for a 50 to 250 $^{o}$C range, but that the indicating meter (TI) had the wrong range on its scale: 0 to 200 $^{o}$C.  What temperature value would the TI indicate when the real temperature was 150 $^{o}$C?  {\it Answer = 100 $^{o}$C}
\end{itemize}


%INDEX% Reading assignment: Lessons In Industrial Instrumentation, Analog Electronic Instrumentation (4-20 mA signals)

%(END_NOTES)


