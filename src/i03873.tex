
%(BEGIN_QUESTION)
% Copyright 2009, Tony R. Kuphaldt, released under the Creative Commons Attribution License (v 1.0)
% This means you may do almost anything with this work of mine, so long as you give me proper credit

Read and outline the ``Controller Output Current Loops'' section of the ``Analog Electronic Instrumentation'' chapter in your {\it Lessons In Industrial Instrumentation} textbook.  Note the page numbers where important illustrations, photographs, equations, tables, and other relevant details are found.  Prepare to thoughtfully discuss with your instructor and classmates the concepts and examples explored in this reading.

\vskip 20pt \vbox{\hrule \hbox{\strut \vrule{} {\bf Active reading tip} \vrule} \hrule}

A practical and fun way to actively engage with a text is to imagine yourself in the role of a teacher, who will quiz students on what they learned from reading that same text.  As you write your outline of that text, include some questions of your own that you would ask a student.  This prompts you to think about the text in a different way: to identify the portions you think are most important, to identify concepts that might be more challenging to comprehend, and to visualize what a good understanding of that text would look like embodied in the responses of other students.

\vskip 10pt

\underbar{file i03873}
%(END_QUESTION)





%(BEGIN_ANSWER)


%(END_ANSWER)





%(BEGIN_NOTES)

Controller supplies both electrical power and information to the FCE.  The controller's dependent current source is a true electrical {\it source}, while the final control element or transducer or motor drive connected to the controller functions as an electrical {\it load}.

\vskip 10pt

Dependent current source controls output current according to value of output (MV) inside controller.  Current source ``fights'' to maintain constant current regardless of other circuit changes.

\vskip 10pt

In applications where the FCE is reverse-acting (i.e. 4 mA = 100\% action and 20 mA = 0\% action) we need to configure the controller to have a {\it reverse-indicating} output display.  This is unrelated to reverse control action, which has to do with the controller's algorithm and the need for negative feedback in closed-loop control.







\vskip 20pt \vbox{\hrule \hbox{\strut \vrule{} {\bf Suggestions for Socratic discussion} \vrule} \hrule}

\begin{itemize}
\item{} {\bf This is a good opportuity to emphasize active reading strategies as you check students' comprehension of today's homework, because it will set the pace for your students' homework completion from here on out.  I strongly recommend challenging students to apply the ``Active Reading Tips'' given in this and other questions in today's assignment, making this the primary focus and the instrumentation concepts the secondary focus.}
\item{} Explain how the identify of an electrical component as either a {\it source} or a {\it load} relates to the voltage drop polarity and direction of current.
\item{} Explain the difference between {\it reverse action} in a controller and {\it reverse indication} on that controller's output display.  What factor(s) dictate direct vs. reverse control action, and what factor(s) dictate direct vs. reverse output indication?
\item{} Suppose the two-wire cable connecting the controller to the I/P converter fails open.  Identify all the consequences of this fault (explaining both valve action and electrical properties such as voltage and current at different points in the failed circuit).
\item{} Suppose the two-wire cable connecting the controller to the I/P converter fails shorted.  Identify all the consequences of this fault (explaining both valve action and electrical properties such as voltage and current at different points in the failed circuit).
\item{} Suppose the resistor inside of the motor drive fails open.  Identify all the consequences of this fault (explaining both motor action and electrical properties such as voltage and current at different points in the failed circuit).
\item{} Suppose the resistor inside of the motor drive fails shorted.  Identify all the consequences of this fault (explaining both motor action and electrical properties such as voltage and current at different points in the failed circuit).
\end{itemize}

%INDEX% Reading assignment: Lessons In Industrial Instrumentation, Analog Electronic Instrumentation (controller output current loops)

%(END_NOTES)


