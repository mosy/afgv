
%(BEGIN_QUESTION)
% Copyright 2009, Tony R. Kuphaldt, released under the Creative Commons Attribution License (v 1.0)
% This means you may do almost anything with this work of mine, so long as you give me proper credit

Read and outline the introduction to the ``Relating 4 to 20 mA Current Signals to Instrument Variables'' section of the ``Analog Electronic Instrumentation'' chapter in your {\it Lessons In Industrial Instrumentation} textbook, then work through at least \underbar{two} of the calculation examples shown in the subsections that follow the introduction.

\vskip 10pt

Many students find the subsections entitled ``Graphical Interpretation of Signal Ranges'' and ``Thinking in Terms of Per Unit Quantities'' helpful as alternative approaches to relating signals to instrument variables.

\vskip 20pt \vbox{\hrule \hbox{\strut \vrule{} {\bf Active reading tip} \vrule} \hrule}

One of the distinctive differences between {\it technical} reading and the reading of other document types is the amount of mathematical content contained in the text.  In the interest of reading {\it actively} (i.e. with a fully engaged mind) it is strongly recommended that you pick up your calculator and actually run the calculations shown to you in examples such as those found in this reading assignment.  Do not be content with simply perusing the calculations shown to you in the text, but {\it actually do them yourself}.  The same is true for any algebraic manipulations presented in a text: take advantage of this as a learning opportunity by challenging yourself to do the same manipulations on paper, comparing your results with the text's.

\vskip 10pt


\underbar{file i03874}
%(END_QUESTION)





%(BEGIN_ANSWER)


%(END_ANSWER)





%(BEGIN_NOTES)

$$y = mx + b$$

$$\hbox{Percentage} = {x - \hbox{LRV} \over \hbox{URV} - \hbox{LRV}} = {x - \hbox{LRV} \over \hbox{Span}}$$

When $x$ is signal \% and $y$ is current (mA), $y = {16 \over 100}x + 4$

\vskip 10pt

When $x$ is GPM (0 to 350) and $y$ is current (mA), $y = {16 \over 350}x + 4$

\vskip 10pt

When $x$ is temp (50 to 140 deg) and $y$ is current (mA), $y = {16 \over 90}x - 4.89$

\vskip 10pt

When $x$ is current (mA) and $y$ is pH (4pH to 10pH), $y = {6 \over 16}x + 2.5$

\vskip 10pt

\noindent
{\it Reverse action} is when an instrument's direction of output response is opposite that of its input (i.e. when the input gets larger, the output gets smaller).  Such instruments have a negative slope value ($m$) in their $y = mx + b$ characteristic equations:

\vskip 5pt

When $x$ is pressure (3-15 PSI) and $y$ is reverse current (20-4 mA), $y = -{16 \over 12}x + 24$

\vskip 10pt

When $x$ is ADC counts (3277 to 16384) and $y$ is flow (0 to 700 GPM), $y = {700 \over 13107}x - 175$

\vskip 10pt

Use a number line to convert input quantity into \%, and then that \% into the output quantity.  Subtracting 4 mA from the given mA value yields the ``length'' of the bar on the number line (i.e. how far ``into the span'' this signal is) which when divided by the whole span yields the equivalent percentage.  Multiplying this percentage by the output span tells you how long the bar is on the output number line, and then this value added to the live zero of the output yields the final result.

\vskip 10pt

Use two linear equations to convert input quantity into per unit values (between 0 and 1 inclusive), and then that per unit value into the scaled output quantity:

$$\hbox{Input} = (\hbox{Span}_{input}) (\hbox{Per unit}) + \hbox{LRV}_{input}$$

$$\hbox{Output} = (\hbox{Span}_{output}) (\hbox{Per unit}) + \hbox{LRV}_{output}$$

For a 4-20 mA range:

$$\hbox{mA} = 16(\hbox{\%}) + 4$$






\vskip 20pt \vbox{\hrule \hbox{\strut \vrule{} {\bf Suggestions for Socratic discussion} \vrule} \hrule}

\begin{itemize}
\item{} {\bf This is a good opportuity to emphasize active reading strategies as you check students' comprehension of today's homework, because it will set the pace for your students' homework completion from here on out.  I strongly recommend challenging students to apply the ``Active Reading Tips'' given in this and other questions in today's assignment, making this the primary focus and the instrumentation concepts the secondary focus.}
\item{} Give a step-by-step procedure for determining the $m$ and $b$ values for any linear function with a given graph.
\item{} What does ``per unit'' mean, and how is this concept similar to ``per cent''?
\item{} Explain what the ``rate'' and ``offset'' parameters of the Allen-Bradley {\tt SCL} instruction mean and how they relate to the general slope-intercept linear formula.
\item{} Demonstrate how to perform any of the example scaling problems using per unit calculations rather than developing a custom $y = mx + b$ formula for each application.
\item{} A common misconception is that $b$ in the slope-intercept linear formula must always be equal to the LRV of the output range.  Explain why this is not true, citing a contrary example to prove your point.
\end{itemize}











\vfil \eject

\noindent
{\bf Prep Quiz:}

\vskip 10pt

\noindent
\vbox{\hrule \hbox{\strut \vrule{} {Part A -- calculation} \vrule} \hrule}
Calculate the amount of DC current equivalent to a 25\% signal value in a 4 to 20 mA analog signal range.

\vskip 20pt

\noindent
\vbox{\hrule \hbox{\strut \vrule{} {Part B -- written response} \vrule} \hrule}
Explain the general attendance policy within this program.  In other words, how are absences managed?

\vskip 20pt

\noindent
\vbox{\hrule \hbox{\strut \vrule{} {Part C -- written response} \vrule} \hrule}
You will be doing a lot of studying in this program.  Identify one practical way you can maximize study time outside of the classroom and lab.

\vskip 20pt

{\it Note: your explanations need to be \underbar{complete} and \underbar{clearly written}.  Expressing your ideas clearly and completely is every bit as important as having those ideas correct in your own mind!}













\vfil \eject

\noindent
{\bf Summary Quiz:}

Calculate the equivalent percentage value of a 9 mA DC electrical signal (in a 4 to 20 mA range):

\begin{itemize}
\item{} 9.0\%
\vskip 5pt 
\item{} 46.15\%
\vskip 5pt 
\item{} 37.5\%
\vskip 5pt 
\item{} 31.25\%
\vskip 5pt 
\item{} 18.0\%
\vskip 5pt 
\item{} 41.67\%
\vskip 5pt 
\item{} 35.0\%
\end{itemize}

%INDEX% Reading assignment: Lessons In Industrial Instrumentation, Analog Electronic Instrumentation (4-20 mA calculations)

%(END_NOTES)


