
%(BEGIN_QUESTION)
% Copyright 2009, Tony R. Kuphaldt, released under the Creative Commons Attribution License (v 1.0)
% This means you may do almost anything with this work of mine, so long as you give me proper credit

Read and outline the ``4-Wire (`Self-Powered') Transmitter Current Loops'' section of the ``Analog Electronic Instrumentation'' chapter in your {\it Lessons In Industrial Instrumentation} textbook.  Note the page numbers where important illustrations, photographs, equations, tables, and other relevant details are found.  Prepare to thoughtfully discuss with your instructor and classmates the concepts and examples explored in this reading.

\vskip 20pt \vbox{\hrule \hbox{\strut \vrule{} {\bf Active reading tip} \vrule} \hrule}

A great way to engage with a text is to {\it mark it up} with your own notes and annotations as you read.  Of course, writing an outline of the text in your own words is the ultimate expression of this principle, since outlining is essentially re-creating the author's thoughts rather than just commenting on them.  However, it might not be as apparent that this can be done with {\it diagrams and illustrations} as well.  Identify any graphics within today's assigned reading that you can ``mark up'' with comments and/or symbols of your own for clarity.  Examples include writing notes and labels on mathematical graphs to make them more understandable, and adding current arrows and voltage polarity marks to electrical schematics to clearly show the circuit's operation.

\vskip 10pt

\underbar{file i03875}
%(END_QUESTION)





%(BEGIN_ANSWER)


%(END_ANSWER)





%(BEGIN_NOTES)

4-wire transmitter: two wires for DC power to the transmitter, two wires for 4-20 mA signal.  Transmitter acts as dependent current {\it source}, controller resistor/input acts as a {\it load}.  Transmitter requires a separate source of electrical power to be connected to it.

\vskip 10pt

250 $\Omega$ resistor normally used to convert 4-20 mA current into 1-5 VDC signal for controller to read.








\vskip 20pt \vbox{\hrule \hbox{\strut \vrule{} {\bf Suggestions for Socratic discussion} \vrule} \hrule}

\begin{itemize}
\item{} {\bf This is a good opportuity to emphasize active reading strategies as you check students' comprehension of today's homework, because it will set the pace for your students' homework completion from here on out.  I strongly recommend challenging students to apply the ``Active Reading Tips'' given in this and other questions in today's assignment, making this the primary focus and the instrumentation concepts the secondary focus.}
\item{} Explain how the identify of an electrical component as either a {\it source} or a {\it load} relates to the voltage drop polarity and direction of current.
\item{} Suppose the two-wire cable connecting the transmitter to the controller input terminals fails open.  Identify all the consequences of this fault (explaining both the controller's PV indication and electrical properties such as voltage and current at different points in the failed circuit).
\item{} Suppose the two-wire cable connecting the transmitter to the controller input terminals fails shorted.  Identify all the consequences of this fault (explaining both the controller's PV indication and electrical properties such as voltage and current at different points in the failed circuit).
\item{} Suppose the resistance of the two-wire cable connecting the transmitter to the controller input terminals increases by a factor of 10\%.  Identify all the consequences of this fault (explaining both the controller's PV indication and electrical properties such as voltage and current at different points in the failed circuit).
\item{} Suppose the resistance of the two-wire cable connecting the transmitter to the controller input terminals decreases by a factor of 10\%.  Identify all the consequences of this fault (explaining both the controller's PV indication and electrical properties such as voltage and current at different points in the failed circuit).
\item{} Suppose the resistor at the controller input terminals fails open.  Identify all the consequences of this fault (explaining both the controller's PV indication and electrical properties such as voltage and current at different points in the failed circuit).
\item{} Suppose the DC power source to the transmitter fails with a 0 VDC output.  Identify all the consequences of this fault (explaining both the controller's PV indication and electrical properties such as voltage and current at different points in the failed circuit).
\end{itemize}


%INDEX% Reading assignment: Lessons In Industrial Instrumentation, Analog Electronic Instrumentation (4-wire transmitters)

%(END_NOTES)


