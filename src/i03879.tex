
%(BEGIN_QUESTION)
% Copyright 2009, Tony R. Kuphaldt, released under the Creative Commons Attribution License (v 1.0)
% This means you may do almost anything with this work of mine, so long as you give me proper credit

Read and outline the ``Using Loop Calibrators'' subsection of the ``Troubleshooting Current Loops'' section of the ``Analog Electronic Instrumentation'' chapter in your {\it Lessons In Industrial Instrumentation} textbook.  Note the page numbers where important illustrations, photographs, equations, tables, and other relevant details are found.  Prepare to thoughtfully discuss with your instructor and classmates the concepts and examples explored in this reading.

\vskip 20pt \vbox{\hrule \hbox{\strut \vrule{} {\bf Active reading tip} \vrule} \hrule}

Learning new concepts is easier when you can link the new concept(s) to other concepts you already understand well.  Another active reading strategy is to explicitly make these connections in your outlining of a text.  Examine today's reading assignments to look for applications of concepts you already comprehend, to help make more sense of one or more new concepts.

\vskip 10pt

\underbar{file i03879}
%(END_QUESTION)





%(BEGIN_ANSWER)


%(END_ANSWER)





%(BEGIN_NOTES)

Loop calibrators can source and measure 4-20 mA.

\vskip 10pt

\begin{itemize}
\item{} In {\bf Read} mode, acts as ammeter (passive load)
\item{} In {\bf Source} mode, acts as current source (active source) -- useful for stroking valves and for sourcing 4-20 mA to controllers designed to connect to 4-wire transmitters
\item{} In {\bf Simulate mode}, acts as current regulator (active load) -- useful for simulating transmitter
\end{itemize}











\vskip 20pt \vbox{\hrule \hbox{\strut \vrule{} {\bf Suggestions for Socratic discussion} \vrule} \hrule}

\begin{itemize}
\item{} {\bf This is a good opportuity to emphasize active reading strategies as you check students' comprehension of today's homework, because it will set the pace for your students' homework completion from here on out.  I strongly recommend challenging students to apply the ``Active Reading Tips'' given in this and other questions in today's assignment, making this the primary focus and the instrumentation concepts the secondary focus.}
\item{} What can a loop calibrator do that a multimeter cannot?
\item{} Identify all sources and all loads in each of the example circuits where a loop calibrator is used in conjunction with a 4-20 mA transmitter circuit.
\item{} Referencing the student's loop diagram for their lab, ask where they would connect a loop calibrator to either {\it measure} loop current, {\it source} current to an instrument, or {\it simulate} a 2-wire transmitter.
\item{} Refer to the LIII section on 4-wire transmitters, and identify how to configure and connect a loop calibrator to simulate the function of the 4-wire transmitter in the example diagram.
\item{} Examine the photograph of a Transmation model 1040 loop calibrator and identify how to set it up to do the following:
\begin{itemize}

\item{} Source current
\item{} Simulate a 2-wire transmitter
\end{itemize}
\item{} Examine the 2-wire transmitter diagram shown in the ``Troubleshooting Current Loops with Voltage Measurements'' section of the textbook (the diagram containing the loop-powered indicator and the power source internal to the controller) and ask how to connect a loop calibrator to:
\begin{itemize}

\item{} Simulate the transmitter
\end{itemize}
\end{itemize}







\vfil \eject

\noindent
{\bf Summary Quiz:}

(A time-saving summary quiz idea is to have students go to the lab and demonstrate loop calibrator use on the working loops they've constructed, which also counts for an objective on the lab assignment.)


%INDEX% Reading assignment: Lessons In Industrial Instrumentation, Analog Electronic Instrumentation (loop calibrators)

%(END_NOTES)


