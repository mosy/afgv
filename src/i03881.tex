
%(BEGIN_QUESTION)
% Copyright 2009, Tony R. Kuphaldt, released under the Creative Commons Attribution License (v 1.0)
% This means you may do almost anything with this work of mine, so long as you give me proper credit

Read and outline the ``2-Wire (`Loop-Powered') Transmitter Current Loops'' section of the ``Analog Electronic Instrumentation'' chapter in your {\it Lessons In Industrial Instrumentation} textbook.  Note the page numbers where important illustrations, photographs, equations, tables, and other relevant details are found.  Prepare to thoughtfully discuss with your instructor and classmates the concepts and examples explored in this reading.

\vskip 20pt \vbox{\hrule \hbox{\strut \vrule{} {\bf Active reading tip} \vrule} \hrule}

Well-written technical texts don't just describe {\it what} and {\it how}, but also {\it why}.  These ``why'' explanations are important for you to grasp, and as such they should always be a part of your written outline.  Identify places within today's reading where the rationale for some concept or technique is explained, and show how your outline reflects this.

\vskip 10pt

\underbar{file i03881}
%(END_QUESTION)





%(BEGIN_ANSWER)


%(END_ANSWER)





%(BEGIN_NOTES)

Two wires used to convey both DC power to field instrument and 4-20 mA signal back to the controller.  Transmitter (field device) acts as a dependent current regulator (an electrical {\it load}), while the controller input also acts as an electrical {\it load}.  This type of transmitter circuit requires a series-connected DC voltage {\it source} to work (usually 24 volts), because no other component in the circuit is a real source of electrical power.

\vskip 10pt

Internal circuitry of 2-wire transmitter uses opamp(s) and bypass transistor to shunt enough current to make the total current be what it ought to be.  Internal circuitry must be able to function on less than 4 mA!  Older 10-50 mA standard used to allow more power to be dissipated inside transmitter.

\vskip 10pt

Loop-powered (2-wire) transmitters are necessarily limited in their power consumption, because they must be able to function on a mere 4 mA (minimum) of current and a mere 19 volts (minimum) terminal voltage.  Some power-intensive applications such as chromatographs simply cannot be made loop-powered, because they require more electrical power than this to operate.









\vskip 20pt \vbox{\hrule \hbox{\strut \vrule{} {\bf Suggestions for Socratic discussion} \vrule} \hrule}

\begin{itemize}
\item{} {\bf This is a good opportuity to emphasize active reading strategies as you check students' comprehension of today's homework, because it will set the pace for your students' homework completion from here on out.  I strongly recommend challenging students to apply the ``Active Reading Tips'' given in this and other questions in today's assignment, making this the primary focus and the instrumentation concepts the secondary focus.}
\item{} Explain how the identify of an electrical component as either a {\it source} or a {\it load} relates to the voltage drop polarity and direction of current.
\item{} Explain why there once was a {\it 10-50 mA} current signal standard.
\item{} Suppose the two-wire cable connecting the transmitter to the controller input terminals fails open.  Identify all the consequences of this fault (explaining both the controller's PV indication and electrical properties such as voltage and current at different points in the failed circuit).
\item{} Suppose the two-wire cable connecting the transmitter to the controller input terminals fails shorted.  Identify all the consequences of this fault (explaining both the controller's PV indication and electrical properties such as voltage and current at different points in the failed circuit).
\item{} Suppose the resistance of the two-wire cable connecting the transmitter to the controller input terminals increases by a factor of 10\%.  Identify all the consequences of this fault (explaining both the controller's PV indication and electrical properties such as voltage and current at different points in the failed circuit).
\item{} Suppose the resistance of the two-wire cable connecting the transmitter to the controller input terminals decreases by a factor of 10\%.  Identify all the consequences of this fault (explaining both the controller's PV indication and electrical properties such as voltage and current at different points in the failed circuit).
\item{} Suppose the resistor at the controller input terminals fails open.  Identify all the consequences of this fault (explaining both the controller's PV indication and electrical properties such as voltage and current at different points in the failed circuit).
\item{} Suppose the DC power source to the transmitter fails with a 0 VDC output.  Identify all the consequences of this fault (explaining both the controller's PV indication and electrical properties such as voltage and current at different points in the failed circuit).
\item{} Suppose transmitter senses either an increasing or a decreasing stimulus.  Explain what happens in the simplified internal schematic diagram of the transmitter in response to this changing stimulus.
\item{} Suppose the bypass transistor inside of a 2-wire transmitter fails open.  Identify all the consequences of this fault (explaining both the controller's PV indication and electrical properties such as voltage and current at different points in the failed circuit).
\item{} Suppose the bypass transistor inside of a 2-wire transmitter fails shorted.  Identify all the consequences of this fault (explaining both the controller's PV indication and electrical properties such as voltage and current at different points in the failed circuit).
\end{itemize}






\vfil \eject

\noindent
{\bf Prep Quiz:}

\vskip 10pt

\noindent
\vbox{\hrule \hbox{\strut \vrule{} {Part A -- multiple-choice} \vrule} \hrule}
A ``2-wire'' field-mounted process transmitter receives its electrical power to operate from:

\begin{itemize}
\item{} An electrical power source located near the transmitter
\vskip 5pt
\item{} An AC power supply connected in parallel with the transmitter
\vskip 5pt
\item{} A secondary-cell battery installed inside the transmitter
\vskip 5pt
\item{} A DC power supply connected in series with the transmitter
\vskip 5pt
\item{} Magnetic fields from nearby motors and other equipment
\vskip 5pt
\item{} A DC power supply connected in parallel with the transmitter
\end{itemize}

\vskip 20pt

\noindent
\vbox{\hrule \hbox{\strut \vrule{} {Part B -- written response} \vrule} \hrule}
Explain the general attendance policy within this program.  In other words, how are absences managed?

\vskip 20pt

\noindent
\vbox{\hrule \hbox{\strut \vrule{} {Part C -- written response} \vrule} \hrule}
You will be doing a lot of studying in this program.  Identify one practical way you can maximize study time outside of the classroom and lab.

\vskip 20pt

{\it Note: your explanations need to be \underbar{complete} and \underbar{clearly written}.  Expressing your ideas clearly and completely is every bit as important as having those ideas correct in your own mind!}



%INDEX% Reading assignment: Lessons In Industrial Instrumentation, Analog Electronic Instrumentation (2-wire transmitters)

%(END_NOTES)


