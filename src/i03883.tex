
%(BEGIN_QUESTION)
% Copyright 2009, Tony R. Kuphaldt, released under the Creative Commons Attribution License (v 1.0)
% This means you may do almost anything with this work of mine, so long as you give me proper credit

A pneumatic pressure transmitter has a calibrated range of 60 PSI to 280 PSI, with an output range of 3 to 15 PSI.  Complete the following table of values for this transmitter, assuming perfect calibration (no error):

% No blank lines allowed between lines of an \halign structure!
% I use comments (%) instead, so that TeX doesn't choke.

$$\vbox{\offinterlineskip
\halign{\strut
\vrule \quad\hfil # \ \hfil & 
\vrule \quad\hfil # \ \hfil & 
\vrule \quad\hfil # \ \hfil \vrule \cr
\noalign{\hrule}
%
% First row
Input pressure & Percent of span & Output signal \cr
%
% Another row
applied (PSI) & (\%) & (PSI) \cr
%
\noalign{\hrule}
%
% Another row
 & 9 &  \cr
%
\noalign{\hrule}
%
% Another row
 &  & 5 \cr
%
\noalign{\hrule}
%
% Another row
102 &  &  \cr
%
\noalign{\hrule}
} % End of \halign 
}$$ % End of \vbox


\underbar{file i03883}
%(END_QUESTION)





%(BEGIN_ANSWER)

% No blank lines allowed between lines of an \halign structure!
% I use comments (%) instead, so that TeX doesn't choke.

$$\vbox{\offinterlineskip
\halign{\strut
\vrule \quad\hfil # \ \hfil & 
\vrule \quad\hfil # \ \hfil & 
\vrule \quad\hfil # \ \hfil \vrule \cr
\noalign{\hrule}
%
% First row
Input pressure & Percent of span & Output signal \cr
%
% Another row
applied (PSI) & (\%) & (PSI) \cr
%
\noalign{\hrule}
%
% Another row
{\bf 79.8} & 9 & {\bf 4.08} \cr
%
\noalign{\hrule}
%
% Another row
{\bf 96.67} & {\bf 16.67} & 5 \cr
%
\noalign{\hrule}
%
% Another row
102 & {\bf 19.09} & {\bf 5.291} \cr
%
\noalign{\hrule}
} % End of \halign 
}$$ % End of \vbox

%(END_ANSWER)





%(BEGIN_NOTES)

%(END_NOTES)


