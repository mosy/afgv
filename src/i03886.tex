
%(BEGIN_QUESTION)
% Copyright 2009, Tony R. Kuphaldt, released under the Creative Commons Attribution License (v 1.0)
% This means you may do almost anything with this work of mine, so long as you give me proper credit

Read and outline the ``Process Flow Diagrams'' section of the ``Instrumentation Documents'' chapter in your {\it Lessons In Industrial Instrumentation} textbook.  Note the page numbers where important illustrations, photographs, equations, tables, and other relevant details are found.  Prepare to thoughtfully discuss with your instructor and classmates the concepts and examples explored in this reading.

\vskip 20pt \vbox{\hrule \hbox{\strut \vrule{} {\bf Suggestions for Socratic discussion} \vrule} \hrule}

\begin{itemize}
\item{} Review the tips listed in Question 0 and apply them to this reading assignment.
\end{itemize}

\underbar{file i03886}
%(END_QUESTION)





%(BEGIN_ANSWER)


%(END_ANSWER)





%(BEGIN_NOTES)

Process Flow Diagrams show major pipes and equipment, few instrument details.  Sometimes cannot tell which instruments control what.

\vskip 10pt

In the book's example PFD, we see an evaporator system where a compressor is used to draw vapors off of an evaporator vessel, then send those compressed vapors to a ``knockout drum'' vessel where some of them will condense back into liquid.



\vskip 20pt \vbox{\hrule \hbox{\strut \vrule{} {\bf Suggestions for Socratic discussion} \vrule} \hrule}

\begin{itemize}
\item{} Describe the purpose of a PFD, as contrasted against a P\&ID or a loop diagram
\item{} Identify features of this process that we can tell from the PFD
\item{} Suppose the circuit breaker supplying power to the compressor motor trips.  What effects will this have on variables within this process?
\item{} Suppose the level valve on the evaporator vessel fails shut.  What effects will this have on variables within this process?
\end{itemize}


















\vfil \eject

\noindent
{\bf Prep Quiz:}

\vskip 10pt

\noindent
\vbox{\hrule \hbox{\strut \vrule{} {Part A -- multiple-choice} \vrule} \hrule}
Choose the appropriate sequence of diagrams, in order of least specific (most general) to most specific (most detailed):

\begin{itemize}
\item{} P\&ID, Loop, PFD
\vskip 5pt 
\item{} P\&ID, PFD, Loop
\vskip 5pt 
\item{} Loop, PFD, P\&ID
\vskip 5pt 
\item{} PFD, Loop, P\&ID
\vskip 5pt 
\item{} PFD, P\&ID, Loop
\end{itemize}

\vskip 20pt

\noindent
\vbox{\hrule \hbox{\strut \vrule{} {Part B -- written response} \vrule} \hrule}
Identify where you can locate the Instrumentation program calendar listing all course-specific events and dates.  Note that there are multiple locations where the calendar resides, but you only need to identify one of them!

\vskip 20pt

\noindent
\vbox{\hrule \hbox{\strut \vrule{} {Part C -- written response} \vrule} \hrule}
Identify where you can locate the grade spreadsheet file for all second-year Instrumentation courses.  Note that there are multiple locations where the spreadsheet resides, but you only need to identify one of them!

\vskip 20pt

{\it Note: your explanations need to be \underbar{complete} and \underbar{clearly written}.  Expressing your ideas clearly and completely is every bit as important as having those ideas correct in your own mind!}


%INDEX% Reading assignment: Lessons In Industrial Instrumentation, Instrumentation Documents (PFD's)

%(END_NOTES)


