
%(BEGIN_QUESTION)
% Copyright 2009, Tony R. Kuphaldt, released under the Creative Commons Attribution License (v 1.0)
% This means you may do almost anything with this work of mine, so long as you give me proper credit

Read and outline the ``Process and Instrument Diagrams'' section of the ``Instrumentation Documents'' chapter in your {\it Lessons In Industrial Instrumentation} textbook.  Note the page numbers where important illustrations, photographs, equations, tables, and other relevant details are found.  Prepare to thoughtfully discuss with your instructor and classmates the concepts and examples explored in this reading.

\vskip 20pt \vbox{\hrule \hbox{\strut \vrule{} {\bf Suggestions for Socratic discussion} \vrule} \hrule}

\begin{itemize}
\item{} Review the tips listed in Question 0 and apply them to this reading assignment.
\end{itemize}

\underbar{file i03887}
%(END_QUESTION)





%(BEGIN_ANSWER)


%(END_ANSWER)





%(BEGIN_NOTES)

A P\&ID is a ``zoomed-in'' view of particular section of process compared to a PFD, showing general connections between instruments.  Some instruments appear on a P\&ID that would not appear on a PFD.

\vskip 10pt

All instruments in the same loop share the same number (e.g. FT-42, FIC-42, PDT-42, FV-42).

\vskip 10pt

ISA symbols: lines through middle of bubbles denotes location.

\begin{itemize}
\item{} No line = locate in the field
\item{} Single line = located in the main control room 
\itemitem{} Solid = front
\itemitem{} Dashed = rear
\item{} Double line = located in an auxiliary location
\itemitem{} Solid = front
\itemitem{} Dashed = rear
\end{itemize}

Box around instrument bubbles denotes functions located within the same physical instrument.

\vskip 10pt

No wiring details, instrument configuration data, etc. in a P\&ID.




\vskip 20pt \vbox{\hrule \hbox{\strut \vrule{} {\bf Suggestions for Socratic discussion} \vrule} \hrule}

\begin{itemize}
\item{} Explain how different instrument locations are represented by bubbles in a P\&ID.  Is there a way for you to make logical sense of these symbols so as to remember them better?
\item{} Which temperature transmitter do you think will register the higher temperature, and why?
\item{} Suppose FV-42 fails wide open.  What effect will this fault have on the operation of the compressor?
\end{itemize}

%INDEX% Reading assignment: Lessons In Industrial Instrumentation, Instrumentation Documents (P&ID's)

%(END_NOTES)


