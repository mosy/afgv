
%(BEGIN_QUESTION)
% Copyright 2009, Tony R. Kuphaldt, released under the Creative Commons Attribution License (v 1.0)
% This means you may do almost anything with this work of mine, so long as you give me proper credit

Read and outline the ``Loop Diagrams'' section of the ``Instrumentation Documents'' chapter in your {\it Lessons In Industrial Instrumentation} textbook.  Note the page numbers where important illustrations, photographs, equations, tables, and other relevant details are found.  Prepare to thoughtfully discuss with your instructor and classmates the concepts and examples explored in this reading.

\vskip 20pt \vbox{\hrule \hbox{\strut \vrule{} {\bf Suggestions for Socratic discussion} \vrule} \hrule}

\begin{itemize}
\item{} Review the tips listed in Question 0 and apply them to this reading assignment.
\end{itemize}

\underbar{file i03888}
%(END_QUESTION)





%(BEGIN_ANSWER)


%(END_ANSWER)





%(BEGIN_NOTES)

Some instruments appear on a loop diagram (a.k.a loop sheet) that would not appear on a P\&ID or PFD!

\vskip 10pt

Not quite as detailed as an electronic schematic diagram, but does show all external wiring details (cables, wire colors, terminal blocks, etc.).  These details help in troubleshooting!  All terminals and fluid connection ports on instruments are shown as labeled squares.

\vskip 10pt

Field instruments are always drawn on the left-hand side of the loop diagram, and control-room instruments drawn on the right-hand side.

\vskip 10pt

Instrument input/output ranges helpful in troubleshooting.  The input of one instrument is typically the output of another, which means these ranges should correlate between instruments.

\vskip 10pt

\noindent
Arrows near instrument bubbles denote {\it action}: 

\begin{itemize}
\item{} Up ($\uparrow$) = direct (output increases as input increases)
\item{} Down ($\downarrow$) = reverse (output decreases as input increases).
\end{itemize}

Sometimes a ``backwards'' action is desired for safety reasons.  For example, in this system PDT-42 is reverse-acting, which means it outputs 4 mA when it sees a high differential pressure across the compressor.  This is intentional, because it means any electrical fault causing a low current condition (e.g. open wire) will be interpreted by the control system as a very high differential pressure, which is considered dangerous for the compressor and will therefore prompt the controller to take the safest action.









\vskip 20pt \vbox{\hrule \hbox{\strut \vrule{} {\bf Suggestions for Socratic discussion} \vrule} \hrule}

\begin{itemize}
\item{} Describe how having input and output ranges specified in a loop diagram is helpful for troubleshooting a control system.
\item{} Explain what ``direct'' and ``reverse'' action refer to for an instrument.
\item{} Explain why PDT-42 is configured to be reverse-acting instead of direct-acting as is more common for transmitters.
\item{} Is PDT-42 self-powered or loop-powered?  Explain how you can tell from the loop diagram.
\item{} Where does FT-42 receive its electrical power to operate?  If FT-42 is an electrical load, where is its electrical source?
\item{} Suppose an open circuit develops in cable 23.  What will the controller ``think'' has happened to the compressor?
\item{} Suppose a short circuit develops in cable 23.  What will the controller ``think'' has happened to the compressor?
\item{} Suppose an open circuit develops in cable 22.  What will happen in this system as a result of this fault?
\item{} Suppose a short circuit develops in cable 22.  What will happen in this system as a result of this fault?
\item{} Suppose an open circuit develops in cable 21.  What will the controller ``think'' has happened to the compressor?
\item{} Suppose a short circuit develops in cable 21.  What will the controller ``think'' has happened to the compressor?
\end{itemize}


%INDEX% Reading assignment: Lessons In Industrial Instrumentation, Instrumentation Documents (loop diagrams)

%(END_NOTES)


