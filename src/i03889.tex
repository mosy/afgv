
%(BEGIN_QUESTION)
% Copyright 2009, Tony R. Kuphaldt, released under the Creative Commons Attribution License (v 1.0)
% This means you may do almost anything with this work of mine, so long as you give me proper credit

Read and outline the ``Functional Diagrams'' section of the ``Instrumentation Documents'' chapter in your {\it Lessons In Industrial Instrumentation} textbook.  Note the page numbers where important illustrations, photographs, equations, tables, and other relevant details are found.  Prepare to thoughtfully discuss with your instructor and classmates the concepts and examples explored in this reading.

\vskip 20pt \vbox{\hrule \hbox{\strut \vrule{} {\bf Suggestions for Socratic discussion} \vrule} \hrule}

\begin{itemize}
\item{} Review the tips listed in Question 0 and apply them to this reading assignment.
\end{itemize}

\underbar{file i03889}
%(END_QUESTION)





%(BEGIN_ANSWER)


%(END_ANSWER)





%(BEGIN_NOTES)

Functional diagrams show the flow of information in a control system, not the physical layout, with information flowing from top to bottom (e.g. FT to FIC to FV, top to bottom).  Rectangular blocks are automatic functions ; diamond blocks are human-actuated functions ; dashed lines are discrete (on/off) signals whereas solid lines are continuous signals.

\vskip 10pt

Level of detail shown in a functional diagram may vary from vague to explicit.

\vskip 10pt

\noindent
Meanings of typical symbols in a Functional Diagram (note that there is a later section in the book showing a page with all the Function Diagram symbol definitions):

\begin{itemize}
\item{} Circle = transmitter
\item{} Square/rectangle = automatic function (e.g. PID algorithm)
\item{} Diamond = manual (human-settable) function
\item{} Trapezoid = final control element
\item{} Solid line = continuous signal
\item{} Dashed line = discrete (on/off) signal
\end{itemize}





\vskip 20pt \vbox{\hrule \hbox{\strut \vrule{} {\bf Suggestions for Socratic discussion} \vrule} \hrule}

\begin{itemize}
\item{} Why do functional diagrams exist at all?  What do they tell us that we cannot discern from PFDs, P\&IDs, or loop sheets?
\item{} Identify the meanings of the various geometric shapes found in a Functional Diagram (e.g. circles, squares, diamonds, trapezoids).
\item{} Identify the meanings of the various line types found in a Functional Diagram (e.g. solid, dashed).
\item{} Suppose the flow transmitter shown in the first functional diagram example senses an increased flow.  Assuming a constant controller setpoint value, what action will the controller take as a result?  Can we tell for sure in this diagram, or do we need more information?
\end{itemize}

%INDEX% Reading assignment: Lessons In Industrial Instrumentation, Instrumentation Documents (Functional diagrams)

%(END_NOTES)


