
%(BEGIN_QUESTION)
% Copyright 2009, Tony R. Kuphaldt, released under the Creative Commons Attribution License (v 1.0)
% This means you may do almost anything with this work of mine, so long as you give me proper credit

Read and outline the ``Connections and Wire Terminations'' and ``DIN Rail'' subsections of the ``Electrical Signal and Control Wiring'' section of the ``Instrument Connections'' chapter in your {\it Lessons In Industrial Instrumentation} textbook.  Note the page numbers where important illustrations, photographs, equations, tables, and other relevant details are found.  Prepare to thoughtfully discuss with your instructor and classmates the concepts and examples explored in this reading.

\vskip 20pt \vbox{\hrule \hbox{\strut \vrule{} {\bf Suggestions for Socratic discussion} \vrule} \hrule}

\begin{itemize}
\item{} Review the tips listed in Question 0 and apply them to this reading assignment.
\end{itemize}

\underbar{file i03892}
%(END_QUESTION)





%(BEGIN_ANSWER)


%(END_ANSWER)





%(BEGIN_NOTES)

Electrical wire connections made using soldering, screw terminals, compression lugs, twisting.  

\vskip 10pt

Terminal blocks hold wire ends underneath screws or metal spring clips to make connections.  Crimp-style lugs may be placed on the ends of the wires to make a more rugged tip.  Some terminal block sections are multi-level.  Some terminal block sections contain switches, fuses, breakers, LEDs, etc.  Screw clamp mechanism works well on both solid and stranded wire.  ``Screwless'' terminal blocks use spring clips to ensure a firm connection between the metal bar inside the block and the wire.

\vskip 10pt

Stranded wire will fray if directly pinched by a rotating screw.  Best to crimp a lug (compression terminal) onto the end of a stranded wire and then clamp that lug underneath the screw.  Fork and ring style terminals available.  Compression-style terminals are NEVER to be used on solid-core wire!

\vskip 10pt

Crimping tools used to compress lugs onto wire ends, configured for different lug sizes.

\vskip 10pt

DIN rail is a standard structure used to fasten terminal blocks and other components to flat surfaces.  DIN rail is screwed to a metal panel surface, then terminals and other devices ``clip'' to the rail.  Comes in ``top hat'' and ``G'' styles.

\vskip 10pt

Devices made for DIN rail mounting include terminal blocks, power supplies, relays, converters, etc.  Some terminal blocks ground themselves to the rail!  Terminal labels are also manufactured for DIN rail mounted terminal blocks.



\vskip 20pt \vbox{\hrule \hbox{\strut \vrule{} {\bf Suggestions for Socratic discussion} \vrule} \hrule}

\begin{itemize}
\item{} A very common misconception for many students is to think they can test for a poor wire-to-terminal connection by measuring electrical resistance between the two screw heads of a terminal block.  Explain why such a test tells us {\it absolutely nothing} about the electrical integrity of the connections between the wire ends and the block.
\item{} Examine the photograph of a screw-type terminal block and explain how it makes a firm electrical connection to the wire.
\item{} Examine the photograph of a screwless terminal block and explain how it makes a firm electrical connection to the wire.
\item{} Examine the photograph of a multi-level terminal block and describe what function it performs.
\item{} For which type of wire (solid vs. stranded) should compression-style terminals be applied to the ends?  
\item{} For which type of wire (solid vs. stranded) should compression-style terminals {\it never be} applied to the ends?  
\item{} Identify appropriate methods for terminating a solid-core wire.
\item{} Identify appropriate methods for terminating a stranded wire.
\item{} May a standard pair of pliers be used for crimping compression-style terminals onto the ends of wire, instead of using a special terminal crimping tool?  Explain why or why not.
\item{} How may terminal blocks and other DIN-rail-mounted components be removed from a DIN rail without sliding them down the length of the rail to the very end?
\end{itemize}





%INDEX% Reading assignment: Lessons In Industrial Instrumentation, Instrument Connections

%(END_NOTES)


