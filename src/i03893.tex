
%(BEGIN_QUESTION)
% Copyright 2009, Tony R. Kuphaldt, released under the Creative Commons Attribution License (v 1.0)
% This means you may do almost anything with this work of mine, so long as you give me proper credit

Read and outline the ``Tube and Tube Fittings'' section of the ``Instrumentation Connections'' chapter in your {\it Lessons In Industrial Instrumentation} textbook.  Note the page numbers where important illustrations, photographs, equations, tables, and other relevant details are found.  Prepare to thoughtfully discuss with your instructor and classmates the concepts and examples explored in this reading.

\vskip 20pt \vbox{\hrule \hbox{\strut \vrule{} {\bf Suggestions for Socratic discussion} \vrule} \hrule}

\begin{itemize}
\item{} Review the tips listed in Question 0 and apply them to this reading assignment.
\end{itemize}

\underbar{file i03893}
%(END_QUESTION)





%(BEGIN_ANSWER)


%(END_ANSWER)





%(BEGIN_NOTES)

Tubes are never threaded like pipes are.  Must use connectors to join together.  Walls too thin for threads.  Impulse tube/line is tube used to connect instrument to process.

\vskip 10pt

Compression fittings use cone-shaped ferrule and body to form tight seal.  Force of tightened nut compresses ferrule around tube.  Swagelok 1 inch and smaller: tighten 1-1/4 turns.  Too loose will leak ; too tight will also leak.  Parker fittings have similar rules.

\vskip 10pt

Tubing gauge will fit between nut and body if under-tightened.  A properly-tightening nut will not allow the gauge to fit.  Properly made connections are stronger than the tube itself.

\vskip 10pt

FITTINGS:
\item{} Pipe to tube = {\it connector}
\item{} Tube to tube = {\it union}
\item{} Thru panel = {\it bulkhead}
\item{} Corner = {\it elbow}
\item{} {\it Branch tee} = pipe in, turn to either tube
\item{} {\it Run tee} = pipe in, straight to one tube, turn to the other tube
\item{} {\it Cross} = four ports
\item{} {\it Plug} = blocks fitting
\item{} {\it Cap} = blocks tube
\end{itemize}

Tubes sold in 20 foot sections ; unions used to join together for longer runs.  Offset unions from tube centerlines to make them easier to access.

\vskip 10pt

Vibration loop sometimes made when tube extends between objects vibrating in relation to each other.

\vskip 10pt

Automatic tube-tightening tools:
\item{} Zero-gravity use (on International Space Station!)
\item{} Perfect tightening every time
\item{} Record of each connection stored in memory, time/date stamped!
\item{} Hydraulic tools, too
\end{itemize}









\vskip 20pt \vbox{\hrule \hbox{\strut \vrule{} {\bf Suggestions for Socratic discussion} \vrule} \hrule}

\begin{itemize}
\item{} Explain why one might choose tube versus pipe (or vice-versa) for any particular application.
\item{} Explain how a ``go-no go'' gap gauge works to test the integrity of compression-style tube fittings.  Is such a gauge foolproof in its results, or are there problems that might escape detection using this tool?
\item{} How tight should you tighten the nut on a compression-type tube fitting when {\it re-making} a tube connection (i.e. one that has been swaged previously)?  Explain why.
\item{} Describe what an under-tightened ferrule might look like when inspected.
\item{} Describe what an over-tightened ferrule might look like when inspected.
\item{} Describe some of the professional practices employed when bending and laying out tubing (e.g. offset unions, vibration loops).
\item{} Examine the various tube connector illustrations, and explain the rationale behind their names (e.g. what does ``union'' mean with reference to a tube fitting?).
\item{} Why would one ever choose to use a special tubing tool such as the Aeroswage SX-1 to assemble compression-style tube fittings?
\end{itemize}



%INDEX% Reading assignment: Lessons In Industrial Instrumentation, Instrument Connections (tube fittings)

%(END_NOTES)


