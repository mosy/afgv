
%(BEGIN_QUESTION)
% Copyright 2009, Tony R. Kuphaldt, released under the Creative Commons Attribution License (v 1.0)
% This means you may do almost anything with this work of mine, so long as you give me proper credit

Read and outline the ``Pressure'' subsection of the ``Fluid Mechanics'' section of the ``Physics'' chapter in your {\it Lessons In Industrial Instrumentation} textbook.  Note the page numbers where important illustrations, photographs, equations, tables, and other relevant details are found.  Prepare to thoughtfully discuss with your instructor and classmates the concepts and examples explored in this reading.

\vskip 30pt

A note-taking technique you will find far more productive in your academic reading than mere highlighting or underlining is to write your own {\it outline} of the text you read.  A section of your {\it Lessons In Industrial Instrumentation} textbook called ``Marking Versus Outlining a Text'' describes the technique and the learning benefits that come from practicing it.  This approach is especially useful when the text in question is dense with facts and/or challenging to grasp.  Ask your instructor for help if you would like assistance in applying this proven technique to your own reading.

\vskip 20pt \vbox{\hrule \hbox{\strut \vrule{} {\bf Suggestions for Socratic discussion} \vrule} \hrule}

\begin{itemize}
\item{} Explain how the Conservation of Energy -- one of the most fundamental laws in physics -- applies to levers, hydraulic systems, and electrical transformers.
\item{} Can air be substituted for oil in a hydraulic jack, such as the type used to lift a car's wheel off the ground?  Why or why not?
\item{} Describe some units of measurement for pressure other than the Pascal or the ``pound per square inch''.
\end{itemize}

\underbar{file i03894}
%(END_QUESTION)





%(BEGIN_ANSWER)


%(END_ANSWER)





%(BEGIN_NOTES)

Liquids and gases differ from solids in that they {\it flow} (readily change shape to fill whatever volume is available).  Force exerted on a fluid (liquid or gas) will be transferred in all directions within that fluid.  Gases will compress when a force is exerted on them, but liquids do not yield (appreciably).  Pressure is defined as the amount of force exerted per unit area:

$$P = {F \over A}$$

Metric unit of pressure is the Pascal: one newton of force per square meter of area.  British unit of pressure is the PSI: one pound of force per square foot of area.

\vskip 10pt

Force may be multiplied via a fluid using pistons of differing area.  A small piston exerts pressure on a fluid, which then presses against a larger-area piston to create a larger force, like a fluid lever.  The Conservation of Energy always holds true: when force is multiplied, distance traveled is diminished by the same proportion.  Analogous to an electrical transformer, where AC voltage may be multiplied, but only at the expense of AC current.

Power may be transmitted via fluids as well: hydraulic systems using liquid and pneumatic systems using gas as the fluid medium.  Fluids may also be used to transmit {\it information}, by representing some quantity proportionate to pressure.  A filled-bulb temperature sensor works like this: changes in bulb temperature cause changes in fluid pressure, which is then transmitted far away to a pressure-indicating gauge through narrow tubing.






\vskip 20pt \vbox{\hrule \hbox{\strut \vrule{} {\bf Suggestions for Socratic discussion} \vrule} \hrule}

\begin{itemize}
\item{} Identify the trade-off experienced in a hydraulic jack system with increased force.  In other words, as force is boosted, what other variable becomes diminished?
\item{} Suppose a rubber ball were placed in the fluid within a hydraulic jack.  What would happen to the ball as the fluid pressure in the hydraulic jack were increased?
\item{} Explain what a ``fluid power system'' is, and give a practical example of one.
\item{} Do you suppose it makes any difference whether a ``filled bulb'' thermometer uses a gas or a liquid as the sensing/signaling medium?  Explain your reasoning.
\end{itemize}








\vfil \eject

\noindent
{\bf Prep Quiz:}

The textbook gave an example of how fluid pressure may be used to convey {\it information} in a specific type of measurement system, not just convey {\it energy} as is typically the case with hydraulic and pneumatic fluid power systems.  Describe this fluid-based measurement system and briefly explain how it works.











\vfil \eject

\noindent
{\bf Prep Quiz:}

The formal definition of {\it pressure} is:

\begin{itemize}
\item{} The psychological effect of daily quizzes
\vskip 5pt 
\item{} The amount of energy needed warm water 1 degree F
\vskip 5pt 
\item{} The amount of applied force per unit area
\vskip 5pt 
\item{} The formula for circular area: $A = \pi r^2$
\vskip 5pt 
\item{} The amount of applied force times distance moved
\vskip 5pt 
\item{} The amount of applied force divided by temperature
\vskip 5pt 
\item{} The amount of applied force per unit time
\end{itemize}


%INDEX% Reading assignment: Lessons In Industrial Instrumentation, Fluid mechanics (pressure)

%(END_NOTES)


