
%(BEGIN_QUESTION)
% Copyright 2009, Tony R. Kuphaldt, released under the Creative Commons Attribution License (v 1.0)
% This means you may do almost anything with this work of mine, so long as you give me proper credit

Read and outline the ``Pascal's Principle and hydrostatic pressure'' subsection of the ``Fluid Mechanics'' section of the ``Physics'' chapter in your {\it Lessons In Industrial Instrumentation} textbook.  Note the page numbers where important illustrations, photographs, equations, tables, and other relevant details are found.  Prepare to thoughtfully discuss with your instructor and classmates the concepts and examples explored in this reading.

\vskip 20pt \vbox{\hrule \hbox{\strut \vrule{} {\bf Suggestions for Socratic discussion} \vrule} \hrule}

\begin{itemize}
\item{} Does Pascal's Law apply only to liquids, or to gases as well?
\item{} Explain what is going on with the {\it dimensional analysis} example.
\item{} How would the property(ies) of a fluid have to change in order for Pascal's Law {\it not} to apply anymore?
\item{} Explain how you could calculate the amount of water pressure at the bottom of a dam based on physical measurements of the lake or river held up by the dam as well as the dam itself.
\end{itemize}

\underbar{file i03895}
%(END_QUESTION)





%(BEGIN_ANSWER)


%(END_ANSWER)





%(BEGIN_NOTES)

Pascal's principle is that any {\it change} in pressure applied to a confined fluid will be evenly propagated throughout the fluid's volume.  This does not mean, necessarily, that the amount of static pressure will be the same throughout.  When we have tall columns of liquid, for example, the pressure at the bottom of the column will be greater than the pressure at the top (due to gravity), but any additional pressure added to the column will be evenly distributed throughout the column.

\vskip 10pt

Hydrostatic pressure (pressure created by the weight of a fluid) depends on the fluid column's height and the fluid's density:

$$P = \rho g h$$

$$P = \gamma h$$

As always, units of measurement must agree in these formulae: if height in inches and density in pounds per cubic inch, then pressure will be pounds per square inch (PSI).




\vskip 20pt \vbox{\hrule \hbox{\strut \vrule{} {\bf Suggestions for Socratic discussion} \vrule} \hrule}

\begin{itemize}
\item{} Explain why the vertical water column with two pressure gauges (one at the bottom, and one half-way up) reading differently is {\it not} a violation of Pascal's Principle.
\item{} If the water inside of a vertical column were to freeze into ice, would Pascal's Principle still apply?  Explain why or why not.
\item{} Explain how it is possible for two liquid columns of different area (but equal height) to generate the same amount of hydrostatic pressure.  This is a non-intuitive concept!
\item{} Compare the water pressure exerted at the base of two dams, with equal water levels but vastly different water {\it volumes} contained by their respective reservoirs.
\item{} Will equal-height liquid columns always produce the same amount of hydrostatic pressure?  If not, what other factor(s) affect the amount of pressure generated for any given height?
\item{} Would a column of water carried aboard a space station (zero gravity environment) generate the same hydrostatic pressure as on Earth?  Explain why or why not.
\item{} Would a column of water placed on the surface of Jupiter (much stronger gravity) generate the same hydrostatic pressure as on Earth?  Explain why or why not.
\item{} Would a column of water placed inside an accelerating elevator generate the same hydrostatic pressure as it would on firm ground?  Explain why or why not.
\end{itemize}





\vfil \eject

\noindent
{\bf Prep Quiz:}

Pascal's principle states that any pressure applied to a fluid in an enclosed volume will be evenly propagated throughout that fluid.  Explain {\it why} this is true, in your own words.  If you find it helpful, you may alternatively explain what condition(s) would have to exist in order for Pascal's principle to be {\it invalid}.





\vfil \eject

\noindent
{\bf Prep Quiz:}

Pascal's principle tells us any {\it change} in applied pressure to a confined fluid will be distributed evenly throughout, but it does not say {\it pressure} will be the same throughout all points.  Describe an example of a fluid system where pressures vary at different points even as changes in applied pressure are evenly distributed.  Feel free to sketch an illustration if you find this helpful.



%INDEX% Reading assignment: Lessons In Industrial Instrumentation, Fluid mechanics (Pascal's Principle and hydrostatic pressure)

%(END_NOTES)


