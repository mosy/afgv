
%(BEGIN_QUESTION)
% Copyright 2009, Tony R. Kuphaldt, released under the Creative Commons Attribution License (v 1.0)
% This means you may do almost anything with this work of mine, so long as you give me proper credit

Read and outline the ``Manometers'' subsection of the ``Fluid Mechanics'' section of the ``Physics'' chapter in your {\it Lessons In Industrial Instrumentation} textbook.  Note the page numbers where important illustrations, photographs, equations, tables, and other relevant details are found.  Prepare to thoughtfully discuss with your instructor and classmates the concepts and examples explored in this reading.

\vskip 20pt \vbox{\hrule \hbox{\strut \vrule{} {\bf Suggestions for Socratic discussion} \vrule} \hrule}

\begin{itemize}
\item{} Which is simpler to use, a {\it U-tube} manometer or a {\it well} manometer?  Explain why.
\item{} Identify different manometer designs intended to increase sensitivity, and explain how each of these designs works.
\item{} How large should the {\it well} be in a well-type manometer for us to safely ignore level changes within it?
\item{} Where should one read the liquid column height when using an {\it inclined} manometer?
\item{} One of the problems with manometers as practical measurement tools is their tendency to ``blow out'' their liquid if over-pressured.  Design a manometer that is resistant to ``blow-out'' events, or at least won't spew its contents everywhere when a blow-out happens.
\end{itemize}

\underbar{file i03896}
%(END_QUESTION)





%(BEGIN_ANSWER)


%(END_ANSWER)





%(BEGIN_NOTES)

A manometer is a tube of liquid showing applied fluid (usually gas) pressure by the height change of the liquid column.  The formulae $P = \rho g h$ and $P = \gamma h$ form the basis of manometers, relating hydrostatic pressure to liquid column height.  So long as the liquid's density is constant, the manometer is vertical, and Earth's gravity does not change, a manometer cannot err.  Manometers may be {\it inclined}, in which case the fluid column motion will become amplified for any given amount of pressure, because only the {\it vertical} height matters to hydrostatic pressure.

Pressure in a U-tube manometer is read as the {\it difference in height} between the two liquid columns.

\vskip 10pt

A micromanometer uses a trapped gas bubble between two liquid columns to amplify column motion, using the bubble like a small piston where the two columns are two large pistons moving much less than the smaller one.

\vskip 10pt

Manometers may use mercury instead of water to achieve greater pressure-measurement ranges for a given height, exploiting mercury's superior density (about 13.5 times as dense as water!).

\vskip 10pt

In a well-type manometer, only the height of one liquid column need be measured, because the much larger cross-sectional area of the well makes the liquid inside the level practically motionless.















\vfil \eject

\noindent
{\bf Prep Quiz:}

One way to ``uncalibrate'' a manometer (i.e. make it read pressure inaccurately) is to:

\begin{itemize}
\item{} Increase the diameter of the manometer tubes
\vskip 5pt 
\item{} Apply a different type of gas to the input tube
\vskip 5pt 
\item{} Drain a small amount of liquid out of the manometer
\vskip 5pt 
\item{} Place an elbow fitting on the vent so it vents in a different direction
\vskip 5pt 
\item{} Increase the density of the liquid
\vskip 5pt 
\item{} Decrease the diameter of the manometer tubes
\vskip 5pt 
\item{} Add a small amount of liquid to the manometer
\end{itemize}


%INDEX% Reading assignment: Lessons In Industrial Instrumentation, Fluid mechanics (Manometers)

%(END_NOTES)


