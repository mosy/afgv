
%(BEGIN_QUESTION)
% Copyright 2010, Tony R. Kuphaldt, released under the Creative Commons Attribution License (v 1.0)
% This means you may do almost anything with this work of mine, so long as you give me proper credit

Certain types of microbes (bacteria) facilitate the decomposition of organic matter in such a way that energy may be extracted.  One way is {\it aerobic}, which means the bacteria are supplied with ample amounts of oxygen to metabolize the matter.  The basic chemical reaction may be modeled by the oxidation of glucose (C$_{6}$H$_{12}$O$_{6}$), since this sugar is the base of the starches and cellulose found in plant matter.  When glucose is oxidized (aerobic decomposition), the general (unbalanced) reaction is as follows:

$$\hbox{C}_6\hbox{H}_{12}\hbox{O}_6 + \hbox{O}_2 \to \hbox{CO}_2 + \hbox{H}_2\hbox{O} \hskip 30pt \Delta H = -673 \hbox{ kcal/mol}$$

Balance this chemical reaction to show the proper proportions of carbon dioxide and water vapor to glucose, and identify whether it is endothermic or exothermic.

\vskip 10pt

A different way microbes may ``digest'' organic matter is in the absence of oxygen.  This is called {\it anaerobic} decomposition, and the following (unbalanced) reaction shows glucose being converted into carbon dioxide and methane gases:

$$\hbox{C}_6\hbox{H}_{12}\hbox{O}_6 \to \hbox{CO}_2 + \hbox{CH}_4 \hskip 30pt \Delta H = -34.6 \hbox{ kcal/mol}$$

Balance this chemical reaction as well to show the proper proportions of carbon dioxide and methane to glucose, and identify whether it is endothermic or exothermic.  Then, comment on which of these two reaction products is useful as a fuel (to be burned with oxygen).

\vskip 20pt \vbox{\hrule \hbox{\strut \vrule{} {\bf Suggestions for Socratic discussion} \vrule} \hrule}

\begin{itemize}
\item{} Explain why the negative $\Delta H$ values make sense in light of the fact that both of these reactions are facilitated by {\it bacteria} in dark environments.
\end{itemize}

\underbar{file i03897}
%(END_QUESTION)





%(BEGIN_ANSWER)

$$\hbox{C}_6\hbox{H}_{12}\hbox{O}_6 + 6\hbox{O}_2 \to 6\hbox{CO}_2 + 6\hbox{H}_2\hbox{O} \hskip 30pt \Delta H = -673 \hbox{ kcal/mol ({\it this is exothermic!})}$$

\vskip 10pt

$$\hbox{C}_6\hbox{H}_{12}\hbox{O}_6 \to 3\hbox{CO}_2 + 3\hbox{CH}_4 \hskip 30pt \Delta H = -34.6 \hbox{ kcal/mol ({\it this is mildly exothermic!})}$$


%(END_ANSWER)





%(BEGIN_NOTES)

Data for these chemical reactions came from {\it The Practical Handbook of Compost Engineering} by Roger T. Haug. (1993), pages 102-103 (heats of formation) and pages 123-124 (aerobic vs. anaerobic reactions).  

\vskip 10pt

I had to calculate the heat of formation for glucose (-301.22 kcal/mol) based on Haug's published heat of combustion for glucose of -673 kcal/mol and his published figures for heats of formation for CO$_{2}$ (-94.05 kcal/mol) and water (-68.32 kcal/mol).

%INDEX% Chemistry, stoichiometry: balancing a chemical equation
%INDEX% Process: aerobic decomposition (digestion) of organic matter
%INDEX% Process: anaerobic decomposition (digestion) of organic matter

%(END_NOTES)


