
%(BEGIN_QUESTION)
% Copyright 2009, Tony R. Kuphaldt, released under the Creative Commons Attribution License (v 1.0)
% This means you may do almost anything with this work of mine, so long as you give me proper credit

Read selected portions of the National Transportation Safety Board's Pipeline Accident Report, {\it Pipeline Rupture and Subsequent Fire in Bellingham, Washington, June 10, 1999} (Document NTSB/PAR-02/02 ; PB2002-916502), and answer the following questions:

\vskip 10pt

Page 6 of the report shows a graphical trend of pipeline pressure before, during, and after the rupture.  How high did the pressure spike, in units of PSI?  Do you suppose this was PSIG or PSIA?  Convert this measurement into units of kilopascals (kPa) and into units of bar.  Based on what you see on the trend graph, was this pipeline carrying a gas or a liquid?  How can you tell, from the shape of the trend alone?

\vskip 10pt

Examine the photographs of the ruptured pipeline on page 41 of the report.  Based on what you know about fluid pressure, determine where along the pipeline's interior the force of the pressure was exerted.

\vskip 10pt

Page 57 of the report discusses how the pipeline had been ``hydrostatically tested'' after its original installation.  This means it was pressure-tested with non-moving (static) water.  Why was this detail important to the investigation?

\vskip 20pt \vbox{\hrule \hbox{\strut \vrule{} {\bf Suggestions for Socratic discussion} \vrule} \hrule}

\begin{itemize}
\item{} Is it possible to monitor over-pressure conditions in a pipeline anywhere along the pipe, or must we use a multitude of pressure sensors along the pipeline's length to ensure we monitor pressure at all locations?
\item{} How do you think an over-pressure condition in a pipeline may be prevented?  What sort of devices might act as safety reliefs to ensure a pipeline does not become over-pressured?
\end{itemize}

\underbar{file i03898}
%(END_QUESTION)





%(BEGIN_ANSWER)


%(END_ANSWER)





%(BEGIN_NOTES)

The magnitude of the pressure spike shown on page 6 was 1494 PSIG = 10301 kPa (or 10.301 MPa) = 103.01 bar.  Hydraulic modeling predicted the pressure had reached 1433 PSIG (see page 57).  This is clearly a liquid process, as indicated by the sharp rise and fall of the pressure transient.  Only liquids, with their relatively insiginficant compressibility, could experience pressure transients of such short duration in a pipeline.

\vskip 10pt

The fluid force was exerted equally around the interior surface of the pipe, outward (imagine radial force lines projecting from the pipeline center, out through the pipe walls in all directions away from the centerline).  Just because the rupture occurred in one spot does {\it not} mean pressure was concentrated there!

\vskip 10pt

Hydrostatic testing of this pipeline (to 1820 PSI just after installation) proved its ability to withstand that much pressure without leakage.  Its calculated yield pressure was 2028 PSI.  Since hydraulic modeling of the pressure spike wa calculated to be only 1433 PSI, this proved the weakening of the pipe as a critical factor in the accident.  Had the pipe remained in good condition, it would have been able to withstand the pressure surge without failing.

%INDEX% Reading assignment: NTSB Pipeline Accident Report, Pipeline Rupture and Subsequent Fire in Bellingham

%(END_NOTES)


