
%(BEGIN_QUESTION)
% Copyright 2009, Tony R. Kuphaldt, released under the Creative Commons Attribution License (v 1.0)
% This means you may do almost anything with this work of mine, so long as you give me proper credit

Skim the ``Continuous Pressure Measurement'' chapter in your {\it Lessons In Industrial Instrumentation} textbook to identify several different mechanical technologies for measuring pressure, then briefly describe the operating principle of each one:

\begin{itemize}
\item{} Manometers (identify some of the different types!)
\vskip 5pt
\item{} Bellows
\vskip 5pt
\item{} Diaphragm
\vskip 5pt
\item{} Bourdon tube (identify some of the different types!)
\end{itemize}

\vskip 20pt \vbox{\hrule \hbox{\strut \vrule{} {\bf Suggestions for Socratic discussion} \vrule} \hrule}

\begin{itemize}
\item{} Discuss ideas for ``skimming'' a text to identify key points so you do not have to read the whole thing.
\item{} Explain why a {\it raised well} manometer is virtually blow-out proof.
\item{} Explain how each ``differential'' pressure sensing mechanism works.
\item{} Can a bourdon tube be used to measure vacuum as well as pressure?  Explain why or why not.
\end{itemize}

\underbar{file i03899}
%(END_QUESTION)





%(BEGIN_ANSWER)


%(END_ANSWER)





%(BEGIN_NOTES)

\begin{itemize}
\item{} Manometers (identify some of the different types!): {\it moving column of liquid indicates applied pressure by changing height.  U-tube, inclined, well, and raised-well types.}
\vskip 5pt
\item{} Bellows: {\it ``Accordian'' structure expanding and collapsing with applied pressure.}
\vskip 5pt
\item{} Diaphragm: {\it Thin sheet of material bowing out or in with applied pressure.}
\vskip 5pt
\item{} Bourdon tube (identify some of the different types!): {\it A hollow, bent tube that straightens with applied pressure.  C-shaped, spiral, and helical types.}
\end{itemize}


%INDEX% Reading assignment: Lessons In Industrial Instrumentation, Continuous Pressure Measurement (overview)

%(END_NOTES)


