
%(BEGIN_QUESTION)
% Copyright 2009, Tony R. Kuphaldt, released under the Creative Commons Attribution License (v 1.0)
% This means you may do almost anything with this work of mine, so long as you give me proper credit

Read and outline the ``DP Transmitter Construction and Behavior'' subsection of the ``Differential Pressure Transmitters'' section of the ``Continuous Pressure Measurement'' chapter in your {\it Lessons In Industrial Instrumentation} textbook.  Note the page numbers where important illustrations, photographs, equations, tables, and other relevant details are found.  Prepare to thoughtfully discuss with your instructor and classmates the concepts and examples explored in this reading.

\underbar{file i03900}
%(END_QUESTION)





%(BEGIN_ANSWER)


%(END_ANSWER)





%(BEGIN_NOTES)

DP transmitters have a diaphragm capsule assembly topped by either a mechanism or electronic circuitry to convert the capsule's response into a standard signal.

\vskip 10pt

``High'' and ``Low'' pressure port labels on a DP instrument refer to the direction of the instrument's response to applied pressure, akin to the ``+'' and ``$-$'' labels of an operational amplifier circuit's input terminals.  DP instruments reject common-mode pressure, responding only to differential pressure between the two ports.  A DP instrument's response to pressure at its two ports is therefore analagous to a voltmeter's response to potential at its two test leads: only the {\it difference} between the two inputs is registered by the instrument.

\vskip 10pt

The Maximum Working Pressure (MWP) of a DP transmitter refers to the maximum common-mode pressure it can handle, which is typically far greater than the maximum differential pressure it can measure.







\vskip 20pt \vbox{\hrule \hbox{\strut \vrule{} {\bf Suggestions for Socratic discussion} \vrule} \hrule}

\begin{itemize}
\item{} Explain the difference in instrument response between pressure applied at the ``high'' port and pressure applied at the ``low'' port.
\item{} Based on your knowledge of how a DP instrument is constructed, explain why it cannot distinguish between a pressure applied to the ``H'' port and a vacuum applied to the ``L'' port (with the other port vented).
\item{} Explain what ``common mode rejection'' is and how the operation of a DP instrument mimics that of an electrical voltmeter.
\item{} Describe the difference between {\it maximum working pressure} and {\it maximum DP range}.
\end{itemize}


%INDEX% Reading assignment: Lessons In Industrial Instrumentation, Differential Pressure Transmitters

%(END_NOTES)


