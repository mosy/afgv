
%(BEGIN_QUESTION)
% Copyright 2009, Tony R. Kuphaldt, released under the Creative Commons Attribution License (v 1.0)
% This means you may do almost anything with this work of mine, so long as you give me proper credit

Skim the ``Continuous Pressure Measurement'' chapter in your {\it Lessons In Industrial Instrumentation} textbook to identify different electronic technologies for measuring pressure, then briefly describe the operating principle of each one:

\begin{itemize}
\item{} Strain gauge (electronic sensing)
\vskip 5pt
\item{} Capacitance sensors (electronic sensing)
\vskip 5pt
\item{} Resonant sensors (electronic sensing)
\end{itemize}

\vskip 20pt \vbox{\hrule \hbox{\strut \vrule{} {\bf Suggestions for Socratic discussion} \vrule} \hrule}

\begin{itemize}
\item{} Discuss ideas for ``skimming'' a text to identify key points so you do not have to read the whole thing.
\item{} Explain how each ``differential'' pressure sensor works.
\end{itemize}



\underbar{file i03902}
%(END_QUESTION)





%(BEGIN_ANSWER)


%(END_ANSWER)





%(BEGIN_NOTES)

\begin{itemize}
\item{} Strain gauge (electronic sensing): {\it Strain gauge attached to diaphragm changes resistance with applied pressure.}
\vskip 5pt
\item{} Capacitance sensors (electronic sensing): {\it Diaphragm used as moving plate of capacitor causes capacitance to change with applied pressure.}
\vskip 5pt
\item{} Resonant sensors (electronic sensing): {\it Vibrating element changes frequency with applied stress from diaphragm or other sensing element.}
\end{itemize}


%INDEX% Reading assignment: Lessons In Industrial Instrumentation, Continuous Pressure Measurement (overview)

%(END_NOTES)


