
%(BEGIN_QUESTION)
% Copyright 2009, Tony R. Kuphaldt, released under the Creative Commons Attribution License (v 1.0)
% This means you may do almost anything with this work of mine, so long as you give me proper credit

Read and outline the ``Zero and Span Adjustments (Analog Instruments)'' section of the ``Instrument Calibration'' chapter in your {\it Lessons In Industrial Instrumentation} textbook.  Note the page numbers where important illustrations, photographs, equations, tables, and other relevant details are found.  Prepare to thoughtfully discuss with your instructor and classmates the concepts and examples explored in this reading.

\underbar{file i03903}
%(END_QUESTION)





%(BEGIN_ANSWER)


%(END_ANSWER)





%(BEGIN_NOTES)

The purpose of {\it calibration} is to ensure accurate correspondence between the input and output of an instrument throughout its range.

\vskip 10pt

A ``live zero'' scale is where 0\% of measurement corresponds to a non-zero signal value (e.g. 4 mA at 0\% as opposed to 0 mA at 0\%).  Any linear-responding instrument's behavior may be described by the slope-intercept line formula:

$$y = mx + b$$

The ``zero'' adjustment on an instrument changes the value of $b$, while the ``span'' adjustment on an instrument changes the value of $m$.  Zero adjustments always add or subtract something, while span adjustments always multiply or divide something.

\vskip 10pt

Changes made to an instrument's span adjustment usually affect that instrument's zero setting.  This is called ``interaction.''








\vskip 20pt \vbox{\hrule \hbox{\strut \vrule{} {\bf Suggestions for Socratic discussion} \vrule} \hrule}

\begin{itemize}
\item{} Explain what a {\it live zero} is for an instrument, and why such ranging exists.
\item{} When the {\it zero} adjustment of an analog instrument is set, what kind of physical quantity is actually being manipulated inside the instrument?
\item{} When the {\it span} adjustment of an analog instrument is set, what kind of physical quantity is actually being manipulated inside the instrument?
\item{} Explain why {\it interactive} zero and span adjustments are troublesome.
\item{} Calculate slope and intercept values for a $y = mx + b$ function where $x$ ranges from 200 to 500 degrees F and $y$ ranges from 4 to 20 mA ($m$ = $16 \over 300$ ; $b$ = $-6.667$) 
\item{} Calculate slope and intercept values for a $y = mx + b$ function where $x$ ranges from $-100$ to +100 degrees F and $y$ ranges from 4 to 20 mA ($m$ = $16 \over 200$ ; $b$ = 12) 
\item{} Calculate slope and intercept values for a $y = mx + b$ function where $x$ ranges from 1 pH to 11 pH and $y$ ranges from 4 to 20 mA ($m$ = $16 \over 10$ ; $b$ = 2.4) 
\item{} Calculate slope and intercept values for a $y = mx + b$ function where $x$ ranges from 0 GPM to 350 GPM and $y$ ranges from 1 to 5 VDC ($m$ = $4 \over 350$ ; $b$ = 1) 
\item{} Calculate slope and intercept values for a $y = mx + b$ function where $x$ ranges from 50 PSI to 175 PSI and $y$ ranges from 3 to 15 PSI ($m$ = $12 \over 125$ ; $b$ = $-1.8$) 
\end{itemize}

%INDEX% Reading assignment: Lessons In Industrial Instrumentation, Instrument Calibration

%(END_NOTES)


