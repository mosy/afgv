
%(BEGIN_QUESTION)
% Copyright 2009, Tony R. Kuphaldt, released under the Creative Commons Attribution License (v 1.0)
% This means you may do almost anything with this work of mine, so long as you give me proper credit

Read and outline the ``Damping Adjustments'' section of the ``Instrument Calibration'' chapter in your {\it Lessons In Industrial Instrumentation} textbook.  Note the page numbers where important illustrations, photographs, equations, tables, and other relevant details are found.  Prepare to thoughtfully discuss with your instructor and classmates the concepts and examples explored in this reading.

\underbar{file i03904}
%(END_QUESTION)





%(BEGIN_ANSWER)


%(END_ANSWER)





%(BEGIN_NOTES)

``Damping'' is essentially a low-pass filter function added to an instrument's response.  The purpose of damping is to reduce the amount of process noise reported by the instrument.  Damping implemented in analog electronic instruments by means of RC filter circuits; in digital electronic instruments by programmed algorithms; in pneumatic instruments by volume chambers.

\vskip 10pt

The cutoff frequency of a low-pass RC filter circuit is defined as the point at which only 70.7\% of the input signal amplitude makes it to the output ($f = {1 \over 2 \pi RC}$).

\vskip 10pt

Too much damping introduced to a transmitter causes it to ``lie'' to the control system, possibly creating new problems.  If the controller tuning is aggressive, the real PV may oscillate due to excessive transmitter damping because the control system ``thinks'' the PV is not responding fast enough and therefore takes excessive action to compensate, while in reality the PV is responding and overshoots the setpoint.  The control system trend, since it only reports the damped transmitter signal, will not show the real oscillations!

\vskip 10pt

Damping should be turned off while bench-calibrating, because its presence will only slow down the procedure.

















\vskip 20pt \vbox{\hrule \hbox{\strut \vrule{} {\bf Suggestions for Socratic discussion} \vrule} \hrule}

\begin{itemize}
\item{} Cite practical examples of where transmitter damping might have to be used in an industrial process.
\item{} Explain how you can tell the RC circuit shown in the text is indeed a {\it low-pass filter}.
\item{} Define ``cutoff frequency'' for a first-order filter.
\item{} Explain why it is not a good idea to use excessive transmitter damping.
\item{} Suppose you were handed an analog (non-``smart'') pressure transmitter and asked to empirically determine how much damping it had.  Devise a test for measuring the amount of damping exhibited by any pressure transmitter.
\item{} Imagine a scenario where the transmitter in a flow control loop has been configured for excessive damping, yet the loop controller is still tuned to take aggressive action (e.g. lots of integral action).  How will the loop behave given a setpoint change?  Articulate a ``thought experiment'' modeling the effects of excessive damping on loop stability.  {\it Answer: the loop will overshoot, undershoot, and likely oscillate because of the extra lag time added by the transmitter's damping.  It would be analogous to trying to drive a car at a fixed speed while watching a speedometer that lagged behind in its display of speed.  Worst of all, the operator may not even see this instability on a process trend.}
\end{itemize}


%INDEX% Reading assignment: Lessons In Industrial Instrumentation, Instrument Calibration

%(END_NOTES)


