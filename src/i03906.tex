
%(BEGIN_QUESTION)
% Copyright 2015, Tony R. Kuphaldt, released under the Creative Commons Attribution License (v 1.0)
% This means you may do almost anything with this work of mine, so long as you give me proper credit

Read and outline the ``Calibration Procedures'' section of the ``Instrument Calibration'' chapter in your {\it Lessons In Industrial Instrumentation} textbook.  Note the page numbers where important illustrations, photographs, equations, tables, and other relevant details are found.  Prepare to thoughtfully discuss with your instructor and classmates the concepts and examples explored in this reading.

\underbar{file i03906}
%(END_QUESTION)





%(BEGIN_ANSWER)


%(END_ANSWER)





%(BEGIN_NOTES)

Linear analog instruments may be calibrated by repeatedly applying LRV and URV analog values, and adjusting zero and span to make each one meet.  Due to interaction, this process usually needs to be repeated.  5-point up-and-down tests are often used to more thoroughly check an instrument's calibration than a simple 2-point check.  Make adjustments to an instrument's linearity only if absolutely necessary!

\vskip 10pt

Linear digital instruments may be calibrated with a 2-point application of analog signal values to the input (trimming the ADC), then a 2-point measurement of analog signal values at the output (trimming the DAC).  After that, the range values may be entered arbitrarily.

\vskip 10pt

Nonlinear instruments require calibration at more than 2 points, because it takes more than 2 points to define a curve (whereas 2 points are sufficient to define a line!).  Document how far you moved each adjustment if possible, in case you must ``backtrack'' in your procedure to an earlier condition.

\vskip 10pt

Discrete instruments require two-direction calibration, to ensure proper tripping in the desired direction, and also to verify deadband.  A trick to speed up the procedure is to hold the input value at the desired trip value, then move the instrument's trip point adjustment in the opposite direction to simulate the input moving in the desired direction.









\vskip 20pt \vbox{\hrule \hbox{\strut \vrule{} {\bf Suggestions for Socratic discussion} \vrule} \hrule}

\begin{itemize}
\item{} Explain how the calibration of an analog instrument differs from the calibration of a digital (``smart'') instrument.
\item{} Explain what ``deadband'' means in the context of a process switch.
\item{} Explain the purpose of performing bidirectional (i.e. up {\it and} down) calibration checks on an instrument.
\item{} Can a 2-point calibration check reveal a {\it zero} error in an instrument?  Why or why not?
\item{} Can a 2-point calibration check reveal a {\it span} error in an instrument?  Why or why not?
\item{} Can a 2-point calibration check reveal a {\it linearity} error in an instrument?  Why or why not?
\item{} Can a 2-point calibration check reveal a {\it hysteresis} error in an instrument?  Why or why not?
\item{} Explain why steps 1 through 4 must be repeated when calibrating an analog instrument.
\item{} Describe the advice given for calibrating nonlinear instruments, and explain why it's a good suggestion to heed.
\item{} Explain how {\it deadband} can be a good feature in a process switch.
\item{} Suppose you are tasked with checking the calibration of a pressure switch with a trip point of 15 PSI.  What other information would you need to know about this switch's operation before proceeding to check its calibration, and how would you go about executing the test?
\item{} Describe the ``trick'' given in the ``Discrete Instruments'' subsection, and explain how this may reduce the time necessary to calibrate the trip setting of a discrete instrument.
\end{itemize}


%INDEX% Reading assignment: Lessons In Industrial Instrumentation, Instrument Calibration

%(END_NOTES)


