
%(BEGIN_QUESTION)
% Copyright 2009, Tony R. Kuphaldt, released under the Creative Commons Attribution License (v 1.0)
% This means you may do almost anything with this work of mine, so long as you give me proper credit

Read the ``An Analogy for Calibration versus Ranging'' section of the ``Instrument Calibration'' chapter in your {\it Lessons In Industrial Instrumentation} textbook.  Note the page numbers where important illustrations, photographs, equations, tables, and other relevant details are found.  Prepare to thoughtfully discuss with your instructor and classmates the concepts and examples explored in this reading.

\underbar{file i03907}
%(END_QUESTION)





%(BEGIN_ANSWER)


%(END_ANSWER)





%(BEGIN_NOTES)

Setting a digital alarm clock's wake-up time is analogous to ranging a smart transmitter.  Synchronizing a digital alarm clock's displayed time against an accurate standard (e.g. WWV) is analogous to calibrating a smart transmitter.

\vskip 10pt

Using a wind-up timer as an alarm clock is analagous to analog transmitter ranging: to set a new wake-up time, one must consult the true time and set the timer accordingly.  Thus, the calibration and ranging are one and the same.









\vskip 20pt \vbox{\hrule \hbox{\strut \vrule{} {\bf Suggestions for Socratic discussion} \vrule} \hrule}

\begin{itemize}
\item{} Did anyone find this analogy helpful for understanding the difference between calibration and ranging?
\item{} How do you set your wristwatch and clocks at home?  How accurate is your calibration source for time?
\item{} Under what circumstances might a technician need to re-calibrate a transmitter?
\item{} Under what circumstances might a technician need to re-range a transmitter?
\end{itemize}

%INDEX% Reading assignment: Lessons In Industrial Instrumentation, Instrument Calibration

%(END_NOTES)


