
%(BEGIN_QUESTION)
% Copyright 2009, Tony R. Kuphaldt, released under the Creative Commons Attribution License (v 1.0)
% This means you may do almost anything with this work of mine, so long as you give me proper credit

Read and outline the ``Instrument Turndown'' section of the ``Instrument Calibration'' chapter in your {\it Lessons In Industrial Instrumentation} textbook.  Note the page numbers where important illustrations, photographs, equations, tables, and other relevant details are found.  Prepare to thoughtfully discuss with your instructor and classmates the concepts and examples explored in this reading.

\underbar{file i03909}
%(END_QUESTION)





%(BEGIN_ANSWER)


%(END_ANSWER)





%(BEGIN_NOTES)

{\it Turndown} (also called {\it rangedown}) is the ratio of maximum transmitter span to minimum transmitter span.  Instrument accuracy typically degrades as the instrument's calibrated span is turned down further.

\vskip 10pt

Real-world examples of pressure transmitter turndown (``rangedown''): 

\begin{itemize}
\item{} Rosemount alphaline (analog) 1151 differential pressure transmitters:
\itemitem{} Range code 3 represents a URL of 30 "WC
\itemitem{} Range code 4 represents a URL of 150 "WC
\itemitem{} Range code 5 represents a URL of 750 "WC
\itemitem{} Range code 6 represents a URL of 100 PSI
\itemitem{} Range code 7 represents a URL of 300 PSI
\itemitem{} Range code 8 represents a URL of 1000 PSI
\itemitem{} Range code 9 represents a URL of 3000 PSI
\itemitem{} Range code 0 represents a URL of 6000 PSI
\itemitem{} Output code E (4-20 mA linear) gives a rangedown of 6:1
\itemitem{} Output code J (4-20 mA square-root) gives a rangedown of 6:1
\itemitem{} Output code G (10-50 mA linear) gives a rangedown of 6:1
\itemitem{} Output code L (0.8 to 3.2 V linear) gives a rangedown of 1.1:1
\itemitem{} Output code M (1 to 5 V linear) gives a rangedown of 2:1
\vskip 5pt
\item{} Rosemount 3051CD (differential) and 3051CG (gauge) pressure transmitters:
\itemitem{} Range code 0 represents a URL of 3 "WC
\itemitem{} Range code 1 represents a URL of 25 "WC
\itemitem{} Range code 2 represents a URL of 250 "WC
\itemitem{} Range code 3 represents a URL of 1000 "WC
\itemitem{} Range code 4 represents a URL of 300 PSI
\itemitem{} Range code 5 represents a URL of 2000 PSI
\itemitem{} Range code 0 gives 2:1 rangedown with $\pm$ 0.1\% accuracy
\itemitem{} Range code 1 gives 15:1 rangedown with $\pm$ 0.1\% accuracy
\itemitem{} Range codes 2-5 give 10:1 rangedown with $\pm$ 0.065\% accuracy
\itemitem{} ``High Accuracy'' option gives 5:1 rangedown with $\pm$ 0.04\% accuracy
\vskip 5pt
\item{} Rosemount 3051S differential pressure transmitters:
\itemitem{} Range code 0 represents a URL of 3 "WC
\itemitem{} Range code 1 represents a URL of 25 "WC
\itemitem{} Range code 2 represents a URL of 250 "WC
\itemitem{} Range code 3 represents a URL of 1000 "WC
\itemitem{} Range code 4 represents a URL of 300 PSI
\itemitem{} Range code 5 represents a URL of 2000 PSI
\itemitem{} ``Ultra'' model gives 200:1 rangedown with $\pm$ 0.025\% accuracy
\itemitem{} ``Ultra for Flow'' model gives 200:1 rangedown with $\pm$ 0.04\% accuracy
\itemitem{} ``Classic'' model gives 100:1 rangedown with $\pm$ 0.055\% accuracy
\vskip 5pt
\item{} Yokogawa DPharp model EJX110A differential pressure transmitters:
\itemitem{} Range code L (URL of 40 "WC) gives 100:1 rangedown with $\pm$ 0.04\% accuracy
\itemitem{} Range code M (URL of 400 "WC) gives 200:1 rangedown with $\pm$ 0.04\% accuracy
\itemitem{} Range code H (URL of 2000 "WC) gives 200:1 rangedown with $\pm$ 0.04\% accuracy
\end{itemize}










\filbreak

\vskip 20pt \vbox{\hrule \hbox{\strut \vrule{} {\bf Suggestions for Socratic discussion} \vrule} \hrule}

\begin{itemize}
\item{} Explain the turndown calculation example in the book.  How was the turndown ratio calculated?
\vskip 5pt
\item{} Suppose we intend to use a Rosemount 1151 alphaline pressure transmitter with a range code 7 sensor and output code E (4-20 mA linear) for a process application where the desired range is 30 to 60 PSI.  Is this transmitter suitable for the application?  {\it No, because its rangedown is only 6:1, yielding a minimum span of 50 PSI.}
\vskip 5pt
\item{} Suppose we intend to use a Rosemount 1151 alphaline pressure transmitter with a range code 4 sensor and output code E (4-20 mA linear) for a process application where the desired range is 5 to 20 "WC.  Is this transmitter suitable for the application?  {\it Yes, because its rangedown is 6:1, yielding a minimum span of 25 "WC.}
\vskip 5pt
\item{} Suppose we intend to use a Rosemount 3051CD pressure transmitter with a range code 1 sensor for a process application where the desired range is 15 to 25 "WC.  Is this transmitter suitable for the application?  {\it Yes, because its rangedown is 15:1, yielding a minimum span of 1.667 "WC.}
\vskip 5pt
\item{} Suppose we intend to use a Rosemount 3051CD pressure transmitter with a range code 2 sensor for a process application where the desired range is 2 to 8 "WC.  Is this transmitter suitable for the application?  {\it No, because its rangedown is only 10:1, yielding a minimum span of 25 "WC.}
\vskip 5pt
\item{} Suppose we intend to use a Rosemount 3051S pressure transmitter with a range code 5 sensor and the ``Ultra'' option for a process application where the desired range is 50 to 150 PSI.  Is this transmitter suitable for the application?  {\it Yes, because its rangedown is 200:1, yielding a minimum span of 10 PSI.}
\vskip 5pt
\item{} Suppose we intend to use a Rosemount 3051S pressure transmitter with a range code 3 sensor and the ``Classic'' option for a process application where the desired range is 90 to 95 "WC.  Is this transmitter suitable for the application?  {\it No, because its rangedown is only 100:1, yielding a minimum span of 10 "WC.}
\vskip 5pt
\item{} Suppose we intend to use a Yokogawa EJX110A pressure transmitter with a range code H sensor for a process application where the desired range is 15 to 20 "WC.  Is this transmitter suitable for the application?  {\it No, because its rangedown is only 200:1, yielding a minimum span of 10 "WC.}
\vskip 5pt
\item{} Suppose we intend to use a Yokogawa EJX110A pressure transmitter with a range code L sensor for a process application where the desired range is 2 to 3 "WC.  Is this transmitter suitable for the application?  {\it Yes, because its rangedown is 100:1, yielding a minimum span of 0.4 "WC.}
\end{itemize}















\vfil \eject

\noindent
{\bf Summary Quiz:}

Suppose a DP transmitter has a maximum range of 0 to 1200 inches water column, and you wish to re-calibrate it to measure a pressure range of 0 to 1.5 PSI.  How much turndown must this transmitter have in order to be able to meet your standard?

\begin{itemize}
\item{} At least 5:1
\vskip 5pt 
\item{} At least 10:1
\vskip 5pt 
\item{} At least 20:1
\vskip 5pt 
\item{} At least 30:1
\vskip 5pt 
\item{} At least 50:1
\vskip 5pt 
\item{} At least 100:1
\end{itemize}

%INDEX% Reading assignment: Lessons In Industrial Instrumentation, Instrument Calibration

%(END_NOTES)


