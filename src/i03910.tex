
%(BEGIN_QUESTION)
% Copyright 2009, Tony R. Kuphaldt, released under the Creative Commons Attribution License (v 1.0)
% This means you may do almost anything with this work of mine, so long as you give me proper credit

Read and outline the ``NIST Traceability'' section of the ``Instrument Calibration'' chapter in your {\it Lessons In Industrial Instrumentation} textbook.  Note the page numbers where important illustrations, photographs, equations, tables, and other relevant details are found.  Prepare to thoughtfully discuss with your instructor and classmates the concepts and examples explored in this reading.

\underbar{file i03910}
%(END_QUESTION)





%(BEGIN_ANSWER)


%(END_ANSWER)





%(BEGIN_NOTES)

An instrument's calibration is only as good as the accuracy of the stimulus source it was calibrated against.  In order to perform good calibrations, we must be assured that the standard(s) used for calibration are accurate in themselves.  Metrology is the science of measurement, and the chief standards organization for metrology in the United States is the NIST (National Institute of Standards and Technology).

\vskip 10pt

``Intrinsic'' standards are standards found in nature that (as far as anyone knows) are absolutely constant.  The resonant frequency of a cesium atom nucleus, for example, is the intrinsic standard used by the NIST for the measurement of time.  Intrinsic standards may be reproduced by any suitable laboratory anywhere in the world, which makes them useful as universal calibration standards.  A {\it Josephson Junction} is an example of a machine used as an intrinsic standard for voltage.

\vskip 10pt

NIST ``traceability'' refers to a chain of documentation attesting that an instrument has been calibrated by another instrument whose calibration may be ultimately traced to an intrinsic standard maintained by the NIST.  Thus, NIST traceability is a sort of ``pedigree'' for an instrument's accuracy.









\vskip 20pt \vbox{\hrule \hbox{\strut \vrule{} {\bf Suggestions for Socratic discussion} \vrule} \hrule}

\begin{itemize}
\item{} Explain what an {\it intrinsic calibration standard} is, and the role these standards play in calibration of industrial sensors.
\item{} Explain the average instrument technician's role in maintaining NIST traceability on the job.  What, exactly, does the technician do to ensure this traceability day in and day out?
\item{} Describe a scenario where NIST traceability is violated, without the jobsite instrument technician's knowledge.  In other words, come up with a situation where a field instrument calibrated by an instrument technician is not actually NIST traceable for some reason beyond that technician's knowledge or ability to control.
\end{itemize}












\vfil \eject

\noindent
{\bf Prep Quiz:}

Suppose you begin your new job as an instrument technician, and want to use your fairly new personal multimeter from school to do scheduled instrument calibrations at the job site.  The supervisor stops you and says, ``No, use our NIST-traceable meter instead, because yours is not!'' while handing you an old multimeter that looks as if it survived a war many years ago.  The reason your supervisor wants you to use the beat-up multimeter instead of your newer multimeter is because:

\begin{itemize}
\item{} Your multimeter was not manufactured by the NIST (National Institute of Standards and Technology)
\vskip 5pt 
\item{} The company's meter is already broken in, while yours hasn't passed its wear-in phase
\vskip 5pt 
\item{} The company does not want to assume liability if your multimeter fails on the job
\vskip 5pt 
\item{} Your meter has not been certified accurate by a higher-level calibration standard
\vskip 5pt 
\item{} Your meter might be stolen, and he doesn't want the police tracing it to this company
\vskip 5pt 
\item{} Your supervisor just doesn't like you that much, and this is his way of telling you 
\end{itemize}


%INDEX% Reading assignment: Lessons In Industrial Instrumentation, Instrument Calibration

%(END_NOTES)


