
%(BEGIN_QUESTION)
% Copyright 2009, Tony R. Kuphaldt, released under the Creative Commons Attribution License (v 1.0)
% This means you may do almost anything with this work of mine, so long as you give me proper credit

Read the product manual for the WIKA (brand) ``DELTA-trans'' (model 891-34-2189) differential pressure transmitter, which uses a Hall Effect sensor to generate an electronic output signal from a sensed differential pressure.  Then, answer the following questions:

\vskip 10pt

Explain how an applied pressure is sensed by this DP transmitter, and how the mechanical motion is converted into an electronic signal.  You may need to do some research on ``Hall Effect'' sensors in order to fully answer this question.

\vskip 10pt

Identify how the zero and span adjustments are implemented -- are they mechanical, or electrical?

\vskip 10pt

Identify how the ``high'' and ``low'' ports of this DP instrument are labeled.  Which way does the sensing element move when a fluid pressure is applied to the ``low'' port?

\vskip 10pt

How is this electronic device powered?  Is there a battery that needs to be replaced periodically?

\vskip 20pt \vbox{\hrule \hbox{\strut \vrule{} {\bf Suggestions for Socratic discussion} \vrule} \hrule}

\begin{itemize}
\item{} Cut-away diagrams such as the one shown in the manual for this Wika pressure gauge/transmitter can be confusing for those unaccustomed to interpreting mechanical drawings.  To try explain what the different shadings and ``hatchings'' in this diagram represent, and also how to visualize the mechanism in motion from applied pressure.
\item{} Identify a mechanical change that could be made to this mechanism affecting the {\it zero}.
\item{} Identify a mechanical change that could be made to this mechanism affecting the {\it span}.
\item{} Modify the design of this pressue transmitter to operate differently in some way, perhaps substituting some other sort of sensor in place of the Hall Effect chip.
\item{} Examine the ``Technical Data'' table in this document and explain what the phrase {\it exposed to medium} refers to.
\end{itemize}

\underbar{file i03911}
%(END_QUESTION)





%(BEGIN_ANSWER)


%(END_ANSWER)





%(BEGIN_NOTES)

This pressure transmitter uses a magnet (part \#5) attached to a moving diaphragm, which moves past a Hall Effect sensor on a circuit board (part \#3) to generate an electrical signal.

\vskip 10pt

Zero and span adjustments are electrical potentiometers (part \#6) on a circuit board.

\vskip 10pt

The ``High'' port is labeled with a ``+'' symbol, while the ``Low'' port is labeled with a ``$-$'' signal.  When pressure is applied to the ``Low'' port, the diaphragm moves to the right (as shown in the diagram on page 2).

\vskip 10pt

There is no internal battery for power.  Instead, this transmitter may be wired for 2-wire (loop powered) operation where it acts as a current regulator (4 to 20 mA), or for 3-wire (separate power supply) operation where it acts as a current source (0 to 20 mA).







\vfil \eject

\noindent
{\bf Prep Quiz:}

\vskip 10pt

The WIKA ``DELTA-trans'' differential pressure transmitter uses which principle to generate an electrical signal corresponding to applied pressure:

\begin{itemize}
\item{} Applying stress to a piezoresistive load cell
\vskip 5pt 
\item{} Changing the capacitance between two metal plates
\vskip 5pt 
\item{} Moving an iron core within a transformer winding
\vskip 5pt 
\item{} Changing the amount of light striking an optical sensor
\vskip 5pt 
\item{} Moving a magnet past a magnetic-field sensor
\vskip 5pt 
\item{} Applying different amounts of heat to a thermal sensor
\end{itemize}

%INDEX% Reading assignment: WIKA pressure transmitter product manual

%(END_NOTES)


