
%(BEGIN_QUESTION)
% Copyright 2009, Tony R. Kuphaldt, released under the Creative Commons Attribution License (v 1.0)
% This means you may do almost anything with this work of mine, so long as you give me proper credit

Read and outline the ``Piezoresistive (Strain Gauge) Sensors'' subsection of the ``Electrical Pressure Elements'' section of the ``Continuous Pressure Measurement'' chapter in your {\it Lessons In Industrial Instrumentation} textbook.  Note the page numbers where important illustrations, photographs, equations, tables, and other relevant details are found.  Prepare to thoughtfully discuss with your instructor and classmates the concepts and examples explored in this reading.

\underbar{file i03912}
%(END_QUESTION)





%(BEGIN_ANSWER)


%(END_ANSWER)





%(BEGIN_NOTES)

A {\it strain gauge} is a small device that changes its electrical resistance when stretched or compressed.  When bonded to a sensing diaphragm, this translates applied pressure into an electrical resistance signal.  The strain gauge element itself is typically connected as one arm of a bridge circuit, translating the resistance change into an analog voltage signal, which is then converted into a 4-20 mA signal or digital (fieldbus) signal.

\vskip 10pt

Silicon is often the material of choice for pressure transmitter strain gauges, due to its excellent resistance to fatigue.  Sometimes the diaphragm itself is made of silicon, necessitating a metal isolating diaphragm and fill fluid to transfer the pressure of the process fluid to the silicon diaphragm and avoid direct contact between the process fluid and the (chemically reactive) silicon.









\vskip 20pt \vbox{\hrule \hbox{\strut \vrule{} {\bf Suggestions for Socratic discussion} \vrule} \hrule}

\begin{itemize}
\item{} Explain {\it why} strain gauges change resistance as they are stretched, and as they are compressed.
\item{} Zoom in to the photograph of the strain gauge in the LIII textbook and identify its axis of sensitivity (i.e. which direction of tension or compression along which it responds most sensitively).
\item{} Demonstrate {\it metal fatigue} using a paper clip and relate this to piezoresistive pressure sensors.
\item{} Explain the advantages and disadvantages of silicon versus metal for the strain gauge material.
\item{} What purpose does a {\it fill fluid} serve in a piezoresistive pressure sensor?
\item{} Suppose the type of fill fluid were changed within a pressure sensor, say from a silicon-based liquid to a fluorocarbon-based liquid.  Would the different fill fluid densities affect the calibration of the instrument?  Why or why not?
\item{} Explain why we cannot use air or any other gas as a ``fill fluid'' within a pressure instrument.
\item{} What's the point of replacing a metal sensing diaphragm with a silicon sensing diaphragm if we just have to use a metal isolating diaphragm to protect the silicon diaphragm from chemical attack?
\item{} Explain why it is important that the isolating diaphragm be relatively slack (more flexible) compared to the sensing diaphragm?
\end{itemize}


%INDEX% Reading assignment: Lessons In Industrial Instrumentation, electrical pressure elements

%(END_NOTES)


