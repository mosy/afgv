
%(BEGIN_QUESTION)
% Copyright 2009, Tony R. Kuphaldt, released under the Creative Commons Attribution License (v 1.0)
% This means you may do almost anything with this work of mine, so long as you give me proper credit

Read and outline the ``Differential Capacitance Sensors'' subsection of the ``Electrical Pressure Elements'' section of the ``Continuous Pressure Measurement'' chapter in your {\it Lessons In Industrial Instrumentation} textbook.  Note the page numbers where important illustrations, photographs, equations, tables, and other relevant details are found.  Prepare to thoughtfully discuss with your instructor and classmates the concepts and examples explored in this reading.

\underbar{file i03913}
%(END_QUESTION)





%(BEGIN_ANSWER)


%(END_ANSWER)





%(BEGIN_NOTES)

A differential capacitance sensor uses a metallic diaphragm as a sensing element and also as one plate of a pair of capacitors.  As this diaphragm bows with applied pressure, one capacitance increases (on the side with decreased distance) while the other capacitance decreases (on the side with increased distance) according to the formula $C = {A \epsilon \over d}$.  Silicone fill fluid transfers pressure from two isolating diaphragms, as well as fulfills the function of being the dielectric material between the capacitor plates.

\vskip 10pt

The isolating diaphragms are very flexible, offering little spring resistance against the internal sensing diaphragm.  The metal frame of the cell bounds the motion of the isolating diaphragms so they cannot move past the point where the sensing diaphragm would be over-stressed, thus protecting the sensing diaphragm from damage during an over-pressure event.

\vskip 10pt

The {\it coplanar} transmitter design orients the sensing and isolating diaphragms in different planes to avoid flange bolt stress from affecting the sensing diaphragm.  Stress imposed by these flange bolts otherwise cause pressure measurement errors.

















\vskip 20pt \vbox{\hrule \hbox{\strut \vrule{} {\bf Suggestions for Socratic discussion} \vrule} \hrule}

\begin{itemize}
\item{} Explain what is meant by the term ``resultant force'' when analyzing a differential pressure sensor.
\item{} Explain how capacitance changes with applied pressure.
\item{} Explain how this design of pressure sensor mechanism is protected against overpressure events (i.e. how the sensing diaphragm is protected from over-stress in the even that too much differential pressure is applied to it).
\item{} What advantage does the {\it coplanar} DP sensor design enjoy over the traditional ``monoplanar'' design?
\item{} What purpose does a {\it fill fluid} serve in a differential capacitance pressure sensor?
\item{} Suppose the type of fill fluid were changed within a pressure sensor, say from a silicon-based liquid to a fluorocarbon-based liquid.  Would the different fill fluid densities affect the calibration of the instrument?  Why or why not?
\item{} Explain why we cannot use air or any other gas as a ``fill fluid'' within a pressure instrument.
\item{} What would happen to the response of a differential capacitance cell if the fill fluid leaked out of one side only?
\item{} What would happen to the response of a differential capacitance cell if the fill fluid leaked out of both sides?
\item{} What would happen to the response of a differential capacitance cell if the fill fluid were replaced by one having a greater dielectric permittivity?
\item{} What would happen to the response of a differential capacitance cell if the fill fluid were replaced by one having a lesser dielectric permittivity?
\item{} What would happen to the response of a differential capacitance cell if the sensing diaphragm developed a leak?
\item{} Suppose someone told you that the fill fluid in a differential capacitance sensor isolates the sensing diaphragm from seeing common-mode process pressure, enabling the sensing diaphragm to only experience differential pressure.  Is this correct or not?  Explain your answer in detail.
\end{itemize}


%INDEX% Reading assignment: Lessons In Industrial Instrumentation, electrical pressure elements

%(END_NOTES)


