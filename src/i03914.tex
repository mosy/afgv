
%(BEGIN_QUESTION)
% Copyright 2009, Tony R. Kuphaldt, released under the Creative Commons Attribution License (v 1.0)
% This means you may do almost anything with this work of mine, so long as you give me proper credit

Read and outline the ``Resonant Element Sensors'' subsection of the ``Electrical Pressure Elements'' section of the ``Continuous Pressure Measurement'' chapter in your {\it Lessons In Industrial Instrumentation} textbook.  Note the page numbers where important illustrations, photographs, equations, tables, and other relevant details are found.  Prepare to thoughtfully discuss with your instructor and classmates the concepts and examples explored in this reading.

\underbar{file i03914}
%(END_QUESTION)





%(BEGIN_ANSWER)


%(END_ANSWER)





%(BEGIN_NOTES)

Any tensed spring element (e.g. string on a musical instrument) changes resonant frequency with applied tension.  If a resonating mechanical structure is attached to a diaphragm, pressure changes induce vibration frequency changes.

\vskip 10pt

Foxboro tried manufacturing a transmitter based on this principle, using a tensed metal wire.  Yokogawa perfected the technology using a pair of micro-machined silicon ``resonators'' instead of a single metal wire, which is how its DPharp series of pressure transmitters work.  The frequency shift between these two resonators (as much as $\pm$ 20 kHz with a base frequency of 90 kHz) is interpreted as applied pressure.





\vskip 20pt \vbox{\hrule \hbox{\strut \vrule{} {\bf Suggestions for Socratic discussion} \vrule} \hrule}

\begin{itemize}
\item{} Ask any of your musician classmates what happens to a steel guitar string after it has been re-tightened many times and fatigues.  How might this phenomenon affect a resonant-wire pressure transmitter?
\item{} Demonstrate {\it metal fatigue} and relate this to resonant-element pressure sensors.
\item{} Why are silicon sensors used instead of metal sensors in a resonant sensor instrument?
\item{} Explain how the Yokogawa DPharp design utilizes {\it two} resonant elements to sense differential pressure. 
\item{} Explain what will happen to the resonant frequencies of both resonators in a Yokogawa DPharp transmitter as the ``H'' side pressure increases and the ``L'' side pressure remains the same.
\item{} Explain what will happen to the resonant frequencies of both resonators in a Yokogawa DPharp transmitter as the ``H'' side pressure remains the same and the ``L'' side pressure decreases.
\item{} Explain what will happen to the resonant frequencies of both resonators in a Yokogawa DPharp transmitter as the common-mode pressure at both ``H'' and ``L'' sides increases (but the differential pressure remains the same).
\end{itemize}












\vfil \eject

\noindent
{\bf Summary Quiz:}

\vskip 10pt

Suppose a tensed string vibrates at a frequency of 5.4 kHz:

$$f = {1 \over 2L} \sqrt{F_T \over \mu}$$

\noindent
Where,

$f$ = Resonant frequency of string (Hertz)

$L$ = String length (meters)

$F_T$ = String tension (newtons)

$\mu$ = Unit mass of string (kilograms per meter)

\vskip 10pt

\noindent
Identify how we might {\it double} the string's resonant frequency (i.e. increase it from 5.4 kHz to 10.8 kHz):

\begin{itemize}
\item{} Increase tension by a factor of 4
\vskip 5pt 
\item{} Increase length by a factor of 2
\vskip 5pt 
\item{} Decrease unit mass by a factor of 2
\vskip 5pt 
\item{} Decrease tension by a factor of 2
\vskip 5pt 
\item{} Decrease length by a factor of 4
\vskip 5pt 
\item{} Increase unit mass by a factor of 8
\end{itemize}

%INDEX% Reading assignment: Lessons In Industrial Instrumentation, electrical pressure elements

%(END_NOTES)


