
%(BEGIN_QUESTION)
% Copyright 2011, Tony R. Kuphaldt, released under the Creative Commons Attribution License (v 1.0)
% This means you may do almost anything with this work of mine, so long as you give me proper credit

Research specifications for the Rosemount model 3051S Series ``coplanar'' differential pressure transmitter (model 3051S\_CD), located in the Product Data Sheet document (00813-0100-4801 Revision GA, April 2006).  Then, answer the following questions:

\vskip 10pt

Identify the different ``performance classes'' for this instrument model.  Specifically, identify the percentage accuracy and the ``rangedown'' (otherwise known as ``turndown'') limits for each.

\vskip 10pt

Identify some of the different codes for pressure measurement ranges.  What is the lowest pressure measurement range you can order this instrument in?  What is the highest pressure measurement range?

\vskip 10pt

Identify some of the maximum working pressures (``overpressure limits'') for different range codes.  What consequence(s) might follow exceeding these limits, according to the manual?

\vskip 10pt

Explain why the higher gage pressure ranges are asymmetrical (i.e. why their negative pressure limits are so much less than their positive pressure limits).  All the differential models have symmetrical ranges, so why don't the gage models?

\vskip 10pt

Identify some of the different isolating diaphragm materials available for this instrument.  Explain why the sensing diaphragms don't come in different material types as well.

\vskip 10pt

Identify the sensor fill fluid options available for this ``coplanar'' model.  Note: this is the fill fluid used to fill the transmitter's internal sensor, {\it not} to fill remote-seal capillary tubes and diaphragms (that would be the model 3051S\_L).

\vskip 20pt \vbox{\hrule \hbox{\strut \vrule{} {\bf Suggestions for Socratic discussion} \vrule} \hrule}

\begin{itemize}
\item{} Suppose you had a model 3051S\_CD transmitter with range code 1.  Determine the {\it lower range limit} (LRL), {\it upper range limit} (URL), {\it overpressure limit}, {\it static pressure limit}, and {\it burst pressure limit} for this particular transmitter, explaining how each of these parameters differ from the others in meaning.
\item{} The ``Ultra for Flow'' performance class model has some interesting limitations, necessary for achieving the exceptionally high accuracy advertised.  Identify what some of these limitations are.
\item{} Identify the meaning of the typical model number given on page 30.
\end{itemize}

\underbar{file i03915}
%(END_QUESTION)





%(BEGIN_ANSWER)


%(END_ANSWER)





%(BEGIN_NOTES)

\begin{itemize}
\item{} Rosemount 3051S\_CD and \_CG differential pressure transmitter specifications (exceeding overpressure limits -- shown on page 11 -- may damage the transmitter).  The Functional Specification table shown on page 9 declares range limits for each sensor range code, while the Ordering Information table on page 27 gives the rangedown values for different performance classes:
\begin{itemize}

\item{} Range code 1 represents a URL of 25 "WC, overpressure limit = 2000 PSI
\item{} Range code 2 represents a URL of 250 "WC, overpressure limit = 3626 PSI
\item{} Range code 3 represents a URL of 1000 "WC, overpressure limit = 3626 PSI
\item{} Range code 4 represents a URL of 300 PSI, overpressure limit = 3626 PSI
\item{} Range code 5 represents a URL of 2000 PSI, overpressure limit = 3626 PSI
\item{} ``Ultra'' performance class gives 200:1 rangedown with $\pm$ 0.025\% accuracy
\item{} ``Ultra for Flow'' performance class gives 200:1 rangedown with $\pm$ 0.04\% accuracy (Range Codes 2 and 3 only, and no negative pressure measurement ability!)
\item{} ``Classic'' performance class gives 100:1 rangedown with $\pm$ 0.055\% accuracy
\item{} ``SIS'' (safety-rated) transmitters only allow 10:1 with the exception of range code 0 (3" WC) which is limited to either 2:1 or 5:1 depending on other options.  ({\it footnote on pages 5 and 9})
\end{itemize}
\end{itemize}

\vskip 10pt

Gauge pressure transmitters have asymmetrical ranges (shown on page 9 for range codes 4 and 5) because the maximum vacuum physically possible for any gas is $-14.7$ PSIG.

\vskip 10pt

Process isolating diaphragm materials are shown on page 13 (``Process-Wetted Parts'') and include stainless steel, Hastelloy, Monel, Tantalum, and gold plated Monel and stainless steel.  Sensing diaphragms need not be made of a variety of material types because they never contact the process fluid and therefore are immune from process fluid reactivity.

\vskip 10pt

Only two types of fill fluid are available in the 3051S\_C transmitter model: {\it silicone} and {\it inert halocarbon}, as specified under option L1 (page 29).  The former is for general-purpose use while the latter is for use in processes with highly oxidizing fluids (e.g. pure oxygen).  The reason food-grade fill fluid is not an option for the coplanar transmitter sensor is because sanitary applications demand the use of remote seals to prevent microbial growth that would otherwise occur inside impulse lines connecting the transmitter to the process vessel.  If remote seals must be used in sanitary applications anyway, then the remote seal diaphragms and their capillary tubes will be filled with a food-grade fluid, isolating the sensor's silicone fluid from contact with the process even if both sets of isolating diaphragms were compromised.

\vskip 10pt

Limits for model 3051S\_CD with range code 1:

\begin{itemize}
\item{} Lower Sensor Limit (LSL) = $-25$ inches WC {\it (page 9)}
\item{} Upper Sensor Limit (USL) = +25 inches WC {\it (page 9)}
\item{} Overpressure limit = 2,000 PSI {\it (page 11)}
\item{} Static pressure limit = 0.5 PSIA to 2,000 PSIG {\it (page 11)} 
\item{} Burst pressure limit = 10,000 PSI {\it (page 12)} 
\end{itemize}

Upper and lower sensor limits refer to the maximum differential pressures the sensor can resolve into a measurement signal.  The overpressure limit is the maximum amount of differential pressure the differential capacitance pressure sensor can withstand without suffering damage.  The static pressure limit is the maximum ``common-mode'' pressure the transmitter can withstand during operation with no degradation in performance (e.g. DP measurement accuracy does not suffer).  The burst pressure limit is the amount of static pressure the physical flange and frame of the transmitter can withstand without leaking or rupturing.

%INDEX% Reading assignment: Rosemount pressure transmitter data sheet

%(END_NOTES)


