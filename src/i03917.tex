
%(BEGIN_QUESTION)
% Copyright 2009, Tony R. Kuphaldt, released under the Creative Commons Attribution License (v 1.0)
% This means you may do almost anything with this work of mine, so long as you give me proper credit

Read and outline the ``Bleed (Vent) Fittings'' subsection of the ``Pressure Sensor Accessories'' section of the ``Continuous Pressure Measurement'' chapter in your {\it Lessons In Industrial Instrumentation} textbook.  Note the page numbers where important illustrations, photographs, equations, tables, and other relevant details are found.  Prepare to thoughtfully discuss with your instructor and classmates the concepts and examples explored in this reading.


\underbar{file i03917}
%(END_QUESTION)





%(BEGIN_ANSWER)


%(END_ANSWER)





%(BEGIN_NOTES)

Bleed valve fittings installed in $1 \over 4$ inch NPT ports on transmitter, providing convenient means to vent stored fluid pressure inside the capsule flanges.  These fittings use a screw to maintain pressure on a ball sealing off a hole.  Loosening the screw lets the ball come away from the hole, allowing fluid to escape to atmosphere.

\vskip 10pt

Installing the bleed fitting at a low height allows one to bleed liquid from a gas process.  Installing the bleed fitting at a high height allows one to bleed gas from a liquid process.





\vskip 20pt \vbox{\hrule \hbox{\strut \vrule{} {\bf Suggestions for Socratic discussion} \vrule} \hrule}

\begin{itemize}
\item{} Explain what a ``zero energy state'' is and how we may achieve this through the use of bleed valves or fittings.
\item{} Explain the significance of bleed valve location (up or down) with regard to different process fluids.
\item{} Explain what a ``stinger'' is and what it is used for when calibrating a pressure transmitter.
\item{} Explain why the equalizing valve must be shut when using a ``stinger'' to apply a pressure to the transmitter.
\end{itemize}














\vfil \eject

\noindent
{\bf Prep Quiz:}

What is a ``stinger'' used for on a DP transmitter?

\begin{itemize}
\item{} Forming a ``tee'' connection with a pressure-filled tube
\vskip 5pt 
\item{} Applying a test pressure through one of the bleed fittings
\vskip 5pt 
\item{} Venting high pressure safely to atmosphere 
\vskip 5pt 
\item{} Re-setting the lower-range value (LRV) of the transmitter
\vskip 5pt 
\item{} Re-setting the upper-range value (URV) of the transmitter
\vskip 5pt 
\item{} Playing pranks on fellow instrument technicians
\end{itemize}


%INDEX% Reading assignment: Lessons In Industrial Instrumentation, Pressure sensor accessories (bleed fittings)

%(END_NOTES)


