
%(BEGIN_QUESTION)
% Copyright 2009, Tony R. Kuphaldt, released under the Creative Commons Attribution License (v 1.0)
% This means you may do almost anything with this work of mine, so long as you give me proper credit

Read and outline the ``Pressure Pulsation Damping'' subsection of the ``Pressure Sensor Accessories'' section of the ``Continuous Pressure Measurement'' chapter in your {\it Lessons In Industrial Instrumentation} textbook.  Note the page numbers where important illustrations, photographs, equations, tables, and other relevant details are found.  Prepare to thoughtfully discuss with your instructor and classmates the concepts and examples explored in this reading.


\underbar{file i03918}
%(END_QUESTION)





%(BEGIN_ANSWER)


%(END_ANSWER)





%(BEGIN_NOTES)

Pressure gauges may be oil-filled to help dampen mild pressure pulsations.  For greater-amplitude pulsations, a {\it snubber} is the preferred solution.  Snubbers consist of restrictions placed between the pressure instrument and the process connection, designed to limit the rate of flow between the two which takes place during pulsation cycles.  The restriction of the snubber combined with the volume of the pressure-sensing instrument forms a kind of ``RC'' low-pass filter, dampening the oscillating portion of the pressure so that only the steady (``DC'') portion of the pressure is registered by the instrument.

\vskip 10pt

In order to ensure that the snubber's restriction never plugs with debris, most snubbers operate as part of a filled system, where an isolating diaphragm is placed on the process connection side, with the snubber operating on clean fill fluid only.  If ever this filled system becomes vented or leaks, it is ruined.






\vskip 20pt \vbox{\hrule \hbox{\strut \vrule{} {\bf Suggestions for Socratic discussion} \vrule} \hrule}

\begin{itemize}
\item{} Why use a fill fluid for a pressure snubber, and not just a needle valve pinching off the flow of process fluid?
\item{} Explain why air bubbles would be bad to have inside a fluid fill system.
\item{} What would happen if some of the fill fluid were to leak out of a pressure snubber attached to a bourdon-tube gauge?  Explain your answer in detail.
\item{} Identify some desirable properties of a fill fluid used in a pressure snubber.
\item{} What would happen if the isolating diaphragm of a pressure snubber attached to a bourdon-tube gauge were to develop a leak (e.g. a tear in the metal)?  Explain your answer in detail.
\end{itemize}


%INDEX% Reading assignment: Lessons In Industrial Instrumentation, Pressure sensor accessories (pulsation dampers)

%(END_NOTES)


