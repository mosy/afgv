
%(BEGIN_QUESTION)
% Copyright 2009, Tony R. Kuphaldt, released under the Creative Commons Attribution License (v 1.0)
% This means you may do almost anything with this work of mine, so long as you give me proper credit

Read and outline the ``Remote and Chemical Seals'' subsection of the ``Pressure Sensor Accessories'' section of the ``Continuous Pressure Measurement'' chapter in your {\it Lessons In Industrial Instrumentation} textbook.  Note the page numbers where important illustrations, photographs, equations, tables, and other relevant details are found.  Prepare to thoughtfully discuss with your instructor and classmates the concepts and examples explored in this reading.

\underbar{file i03919}
%(END_QUESTION)





%(BEGIN_ANSWER)


%(END_ANSWER)





%(BEGIN_NOTES)

Isolating diaphragms and liquid-filled capillary tubes between a pressure-sensing instrument and the process fluid are called {\it remote} or {\it chemical} seals.  They are used to prevent the instrument from direct contact with the process fluid, either for hygienic, corrosion, or plugging reasons.  The isolating diaphragm is designed to be ``slack'' (i.e. have little spring tension) in order that the pressure be transmitted with little loss from process fluid to fill fluid.  The capillary tube is very small in diameter, in order to minimize the stored volume of the fill fluid and thereby limit temperature-induced errors caused by fill fluid expansion and contraction.

\vskip 10pt

Filled systems such as this must be kept leak-free, or else severe measurement errors will result.  If air gets into the fill fluid, the result will be akin to an un-bled hydraulic brake system, where full process pressure might not reach the instrument!

\vskip 10pt

Filled impulse lines and filled capillary tubes alike will generate hydrostatic pressure if the process connection point is vertically distanced from the sensing instrument.  Offset pressure may be calculated by the formula $P = \rho g h$ or $P = \gamma h$.  A pressure instrument mounted lower than the process connection point will read too high; an instrument mounted higher than the process connection point will read too low.

\vskip 10pt

Large changes in temperature may cause fill fluid to expand and contract excessively, thus inducing measurement errors.  Viscosity of the fill fluid may cause sluggish pressure response.  The wrong fill fluid might even freeze if ambient conditions become too cold!  Close-coupled seals may mitigate these problems.





\vskip 20pt \vbox{\hrule \hbox{\strut \vrule{} {\bf Suggestions for Socratic discussion} \vrule} \hrule}

\begin{itemize}
\item{} A helpful ``active reading'' technique for technical texts is to work through every mathematical example presented, to ensure you understand the math as you read along.  Apply this technique here, demonstrating how to work through at least one of the calculation examples presented in the textbook.
\item{} What sorts of problems might we encounter in certain processes using impulse lines rather than remote seals?
\item{} Describe what {\it CIP} and {\it SIP} processes are.
\item{} Identify some desirable properties of a fill fluid used in a remote seal system.
\item{} Discuss the elevation error calculation example shown in the book.
\item{} Explain the causes of {\it temperature-induced errors} in remote seal systems.
\item{} Explain the causes of {\it time-delay errors} in remote seal systems.
\item{} Identify factors that would exacerbate {\it time-delay error} in a remote seal system.
\item{} Does the size (area) of a remote seal affect the calibration of the pressure instrument it serves?  {\it The answer to this question is ``no,'' but it is not intuitive for most people.  A good way to prove to yourself that isolating diaphragm size is irrelevant is to perform a thought experiment whereby you ask yourself how much pressure is in the process vessel versus the capillary tube for different isolating diaphragm sizes.  If the diaphragm is sufficiently limp in each case (which it should be), the pressures should be identical.  Therefore, the isolating diaphragm size does not matter.}
\end{itemize}









\vfil \eject

\noindent
{\bf Prep Quiz:}

Why is it important to avoid letting any air bubbles inside the capillary tube of a remote seal system?

\begin{itemize}
\item{} The presence of any bubbles in the fill fluid will cause measurement errors
\vskip 5pt 
\item{} Air bubbles may cause corrosion to form at the instrument's sensing diaphragm
\vskip 5pt 
\item{} Air may chemically react with the fill fluid, causing a fire or explosion hazard
\vskip 5pt 
\item{} The instrument may experience surges of high pressure resulting from the bubbles
\vskip 5pt 
\item{} Bubbles will cause the instrument's response to be damped more than usual
\vskip 5pt 
\item{} Capillary tubes are not designed to transport gases, but only liquids
\end{itemize}


%INDEX% Reading assignment: Lessons In Industrial Instrumentation, Pressure sensor accessories (remote and chemical seals)

%(END_NOTES)


