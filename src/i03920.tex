
%(BEGIN_QUESTION)
% Copyright 2009, Tony R. Kuphaldt, released under the Creative Commons Attribution License (v 1.0)
% This means you may do almost anything with this work of mine, so long as you give me proper credit

Read and outline the ``Filled Impulse Lines'' and ``Purged Impulse Lines'' subsections of the ``Pressure Sensor Accessories'' section of the ``Continuous Pressure Measurement'' chapter in your {\it Lessons In Industrial Instrumentation} textbook.  Note the page numbers where important illustrations, photographs, equations, tables, and other relevant details are found.  Prepare to thoughtfully discuss with your instructor and classmates the concepts and examples explored in this reading.


\underbar{file i03920}
%(END_QUESTION)





%(BEGIN_ANSWER)


%(END_ANSWER)





%(BEGIN_NOTES)

Filling impulse lines with a fluid that is non-reactive with the process and yet stays in the impulse lines (usually by gravity) is another way to isolate pressure-sensing instruments from dangerous or corrosive process fluid.  

\vskip 10pt

A hand pump may be used to re-fill impulse lines with fill fluid.  Block valves prevent instrument over-pressuring during filling operations.  Fill fluid pumping may also be used to unblock plugged process tap connections!

\vskip 10pt

In a purged impulse line system, a continuous flow of purge fluid (either liquid or gas) keeps the impulse lines unclogged.  Again, this purge fluid must be non-reactive with the process fluid.

\vskip 10pt

If the purge flow rate is excessive, positive pressure errors will result.  The flow rate needs to be just high enough to prevent debris from entering the impulse lines and clogging them.

\vskip 10pt

Hydrostatic pressure is generated by purged lines just the same as with filled lines.  In gas-purged systems, however, this effect is negligible.



\vskip 20pt \vbox{\hrule \hbox{\strut \vrule{} {\bf Suggestions for Socratic discussion} \vrule} \hrule}

\begin{itemize}
\item{} Why choose a fill fluid {\it denser} than the process fluid?
\item{} Does the diameter of the impulse line matter for a filled-line system?
\item{} Will there be an elevation (head) error with filled impulse lines like there is for remote seal systems?
\item{} Will there be an elevation (head) error with purged impulse lines like there is for remote seal systems?
\item{} Examining the three instruments shown connected to the wastewater pipe in the ``Remote and Chemical Seals'' section of the textbook, identify the type and direction of calibration error caused by the hydrostatic pressure of each filled line.
\item{} With filled impulse lines, the fill fluid needs to be non-reactive and non-miscible with the process fluid.  Are these criteria also true for purged impulse lines?
\item{} How does one select a good purge flow rate, and how do we typically measure that rate so we know its value?
\item{} Identify ideal characteristics of a purge fluid.
\item{} What might happen in a purged system if the purge flow rate were set too high?
\item{} What might happen in a purged system if the purge flow rate were set too low?
\end{itemize}


%INDEX% Reading assignment: Lessons In Industrial Instrumentation, Pressure sensor accessories (filled and purged impulse lines)

%(END_NOTES)


