
%(BEGIN_QUESTION)
% Copyright 2009, Tony R. Kuphaldt, released under the Creative Commons Attribution License (v 1.0)
% This means you may do almost anything with this work of mine, so long as you give me proper credit

Read the ``1199 Fill Fluid Specifications'' document published by Rosemount (00840-2100-4016, revision AA), and answer the following questions:

\vskip 10pt

In what ways does DC704 fill fluid differ from the general-purpose DC200?

\vskip 10pt

Which fill fluid is most appropriate for high-reactivity applications such as pure oxygen pressure measurement?

\vskip 10pt

Which fill fluid is most common in Rosemount remote seals?

\vskip 10pt

Which fill fluids are appropriate for {\it sanitary} (food processing and pharmaceutical) applications?

\vskip 10pt

Which fill fluid has the most potential to cause calibration errors as a result of large elevation differences between the transmitter and the remote seal(s)?  Explain why.

\vskip 10pt

Which fill fluid has the most potential to exhibit slow response times when measuring fast-changing process pressures?  Explain why.

\vskip 20pt \vbox{\hrule \hbox{\strut \vrule{} {\bf Suggestions for Socratic discussion} \vrule} \hrule}

\begin{itemize}
\item{} Why would we ever care about the reactivity of a fill fluid with the process fluid, since these two fluids are separated from contact with each other by isolating diaphragms?
\item{} For any of the fill fluids with undesirable characteristics (e.g. extremely high viscosity), identify what {\it desirable} characterstics they possess.  What special applications might demand the use of these fluid types?
\end{itemize}

\underbar{file i03923}
%(END_QUESTION)





%(BEGIN_ANSWER)


%(END_ANSWER)





%(BEGIN_NOTES)

In what ways does DC704 fill fluid differ from the general-purpose DC200?  {\it DC704 silicone is denser and much more viscous than DC200.  Its temperature range is shifted higher (-49 to 400 deg F for DC200 ; 32 to 600 deg F for DC704).  Heat-tracing of capillary tubes is recommended for outdoor remote seal applications using DC704 fill fluid.}

\vskip 10pt

Which fill fluid is most appropriate for high-reactivity applications such as pure oxygen pressure measurement?  {\it Inert halocarbon is the best.}

\vskip 10pt

Which fill fluid is most common in Rosemount remote seals?  {\it DC200 is used in over half of all Rosemount remote seal assemblies.}

\vskip 10pt

Which fill fluids are appropriate for {\it sanitary} (food processing and pharmaceutical) applications?  {\it Neobee M20 is a food-grade fluid derived from coconut oil.  Glycerine/water mix is another food-grade alternative, as is propylene gylcol/water.}

\vskip 10pt

Which fill fluid has the most potential to cause calibration errors as a result of large elevation differences between the transmitter and the remote seal(s)?  Explain why.  {\it The inert halocarbon fluid has the greatest specific gravity (1.85) of any fill fluid mentioned in this document, and therefore will generate the greatest amount of hydrostatic pressure for any given height differential in a remote seal system.}

\vskip 10pt

Which fill fluid has the most potential to exhibit slow response times when measuring fast-changing process pressures?  Explain why.  {\it DC705 silicone has the greatest viscosity (175 cs) of any fill fluid mentioned in this document, and therefore will generate the slowest response time.}


%INDEX% Reading assignment: Rosemount 1199 Fill Fluid specifications manual

%(END_NOTES)


