
%(BEGIN_QUESTION)
% Copyright 2009, Tony R. Kuphaldt, released under the Creative Commons Attribution License (v 1.0)
% This means you may do almost anything with this work of mine, so long as you give me proper credit

Read and outline the ``Pilot Valves and Pneumatic Amplifying Relays'' section of the ``Pneumatic Instrumentation'' chapter in your {\it Lessons In Industrial Instrumentation} textbook.  Note the page numbers where important illustrations, photographs, equations, tables, and other relevant details are found.  Prepare to thoughtfully discuss with your instructor and classmates the concepts and examples explored in this reading.


\underbar{file i03925}
%(END_QUESTION)





%(BEGIN_ANSWER)


%(END_ANSWER)





%(BEGIN_NOTES)

If the backpressure of a nozzle in a self-balancing pneumatic instrument is amplified, the responsiveness of the pneumatic instrument will improve.  Pneumatic amplifier serves an analogous function to an electronic transistor: providing a pressure gain for increased sensitivity.

\vskip 10pt

A dual plug/seat mechanism is known as a ``pilot valve'', converting small mechanical motions into pressure changes (like a super-sensitive baffle-nozzle mechanism).  One plug opens up while another closes off, resulting in very responsive pressure changes to rod motion.  Pilot valves may be made in direct- and reverse-acting styles.

\vskip 10pt

Adding a diaphragm to the pilot valve mechanism makes it a ``relay'' instead of just a pilot valve.  Now, a pressure input controls a pressure output, kind of like a pneumatic transistor.  By reversing the seating directions of the two relay plugs, we may switch the action of a relay from direct to reverse.  A direct-acting relay is one where $P_{out}$ rises as $P_{in}$ rises.  A reverse-acting relay is one where $P_{out}$ falls as $P_{in}$ rises.  The gain of a pneumatic relay is the change in output pressure divided by the corresponding change in input pressure (Gain = $\Delta P_{out} \over \Delta P_{in}$).  Whether or not there is a pressure gain, relays definitely provide a {\it volume} gain, which helps any self-balancing pneumatic instrument respond faster to input changes.

Adding a pneumatic relay to a self-balancing pneumatic system makes it more faster (volume gain) and more sensitive to small input changes (pressure gain).  Hydraulic amplifying relays do the same in hydraulic instrumentation systems.

\vskip 10pt

``Bleeding'' relays continually leak some air (e.g. Foxboro M40).  ``Non-bleeding'' relays seal up and do not bleed air (e.g. Fisher).  The distinction is whether or not the supply and vent valves internal to the relay are able to simultaneously close off.





\vskip 20pt \vbox{\hrule \hbox{\strut \vrule{} {\bf Suggestions for Socratic discussion} \vrule} \hrule}

\begin{itemize}
\item{} Small orifices make for highly sensitive baffle/nozzle mechanisms, and we know sensitivity is good.  What is wrong, then, with making these orifices as small as we can make them?
\item{} Explain why the double-seated pilot mechanism is more sensitive than the single-seated pilot mechanism.
\item{} Identify the distinction between a {\it pilot} and a {\it relay} in pneumatic nomenclature.
\item{} Describe the difference between {\it direct} and {\it reverse} action relays.
\item{} Explain the purpose of the two resistors in the equivalent electronic circuit for the pneumatic relay.
\item{} Explain as best you can the operation of the Foxboro pneumatic amplifying relay.
\item{} Explain as best you can the operation of the Fisher pneumatic amplifying relay.
\item{} Explain how {\it bleeding} relays differ in operation and design from {\it non-bleeding} relays.
\item{} What will happen if a leak develops in the diaphragm of a Foxboro amplifying relay?  Explain your answer in detail.
\item{} What will happen if the leaf spring breaks inside of a Foxboro amplifying relay?  Explain your answer in detail.
\item{} What will happen if the coil spring breaks inside of a Fisher amplifying relay?  Explain your answer in detail.
\item{} What will happen if the vent hole plugs inside of a Fisher amplifying relay?  Explain your answer in detail.
\item{} What will happen if a leak develops in the upper diaphragm of a Fisher amplifying relay?  Explain your answer in detail.
\item{} What will happen if a leak develops in the lower diaphragm of a Fisher amplifying relay?  Explain your answer in detail.
\end{itemize}


%INDEX% Reading assignment: Lessons In Industrial Instrumentation, Pneumatic Instrumentation (pilots and relays)

%(END_NOTES)


