
%(BEGIN_QUESTION)
% Copyright 2009, Tony R. Kuphaldt, released under the Creative Commons Attribution License (v 1.0)
% This means you may do almost anything with this work of mine, so long as you give me proper credit

Read and outline the ``Self Balancing Pneumatic Instrument Principles'' section of the ``Pneumatic Instrumentation'' chapter in your {\it Lessons In Industrial Instrumentation} textbook.  Note the page numbers where important illustrations, photographs, equations, tables, and other relevant details are found.  Prepare to thoughtfully discuss with your instructor and classmates the concepts and examples explored in this reading.

\underbar{file i03926}
%(END_QUESTION)





%(BEGIN_ANSWER)


%(END_ANSWER)





%(BEGIN_NOTES)

A balance-beam laboratory scale measures mass by comparing the unknown mass against known mass standards.  It reveals a condition of equality between two masses, nothing more.  To automate an instrument such as this, we must automatically sense a condition of balance, and automatically apply weight to other pan to balance.

\vskip 10pt

First, we apply a baffle/nozzle as a balance detector, indicating when the pointer is in the straight-down position.  Next, we use a bellows to apply force to the balancing pan as a function of nozzle backpressure.  This achieves negative feedback and makes the scale self-balacing.  Finally, we may measure the nozzle pressure as a function of applied mass.

\vskip 10pt

With the bellows in place, the nozzle backpressure becomes proportional to applied weight, not merely a go/no-go indicator for balance.  The pressure gauge may be located in some remote area, connected to the scale by a long tube (i.e. it need not be local to the scale).  The instrument is now a pneumatic mass transmitter.





\vskip 20pt \vbox{\hrule \hbox{\strut \vrule{} {\bf Suggestions for Socratic discussion} \vrule} \hrule}

\begin{itemize}
\item{} Describe a calibration check procedure for a laboratory balance scale.
\item{} Describe in detail the evolution of this self-balancing scale, from the bare scale itself to the full working system with remote indication.
\item{} Would the scale be useful if it were only equipped with a baffle/nozzle mechanism, with no feedback bellows?
\item{} Elaborate on the function of the pressure gauge, both with and without the feedback bellows in the system.
\item{} What would happen to the self-balancing scale if the flapper lever got bent, so it no longer hung straight down from the scale's beam?  Would this affect the accuracy of measurement?  If so, would this be a {\it zero} shift, a {\it span} shift, or a {\it linearity} shift?
\item{} How would the function of this scale be affected if the bellows were made of more flexible material?  Would this affect the accuracy of measurement?  If so, would this be a {\it zero} shift, a {\it span} shift, or a {\it linearity} shift?
\item{} How would the function of this scale be affected if the bellows were replaced by another having a greater area?  Would this affect the accuracy of measurement?  If so, would this be a {\it zero} shift, a {\it span} shift, or a {\it linearity} shift?
\item{} How would the function of this scale be affected if the base of the bellows was moved to a lower position, placing it closer to the balancing pan?  Would this affect the accuracy of measurement?  If so, would this be a {\it zero} shift, a {\it span} shift, or a {\it linearity} shift?
\item{} How would the function of this scale be affected if a small mass were placed on the right-hand pan next to the bellows?  Would this affect the accuracy of measurement?  If so, would this be a {\it zero} shift, a {\it span} shift, or a {\it linearity} shift?
\item{} What happens to the accuracy of the self-balancing scale if the signaling tube connecting the scale to a remote indicator is greatly lengthened?  Does this affect the accuracy of measurement?  If so, would this be a {\it zero} shift, a {\it span} shift, or a {\it linearity} shift?
\item{} What happens to the accuracy of the self-balancing scale if the bellows were manufactured with a slightly larger diameter than it should have?  Would this affect the accuracy of measurement?  If so, would this be a {\it zero} shift, a {\it span} shift, or a {\it linearity} shift?
\end{itemize}


%INDEX% Reading assignment: Lessons In Industrial Instrumentation, Pneumatic Instrumentation (self-balancing systems)

%(END_NOTES)


