
%(BEGIN_QUESTION)
% Copyright 2009, Tony R. Kuphaldt, released under the Creative Commons Attribution License (v 1.0)
% This means you may do almost anything with this work of mine, so long as you give me proper credit

Complete the following table of equivalent pressures:

% No blank lines allowed between lines of an \halign structure!
% I use comments (%) instead, so that TeX doesn't choke.

$$\vbox{\offinterlineskip
\halign{\strut
\vrule \quad\hfil # \ \hfil & 
\vrule \quad\hfil # \ \hfil & 
\vrule \quad\hfil # \ \hfil & 
\vrule \quad\hfil # \ \hfil \vrule \cr
\noalign{\hrule}
%
% First row
\hskip 20pt bar \hskip 20pt & \hskip 20pt PSI \hskip 20pt & \hskip 20pt inches W.C. \hskip 20pt & \hskip 20pt inches mercury \hskip 20pt \cr
%
\noalign{\hrule}
%
% Another row
0.59 &  &  &  \cr
%
\noalign{\hrule}
%
% Another row
  & 4.1 &  &  \cr
%
\noalign{\hrule}
%
% Another row
  &  & 200 &  \cr
%
\noalign{\hrule}
%
% Another row
  &  &  & 35 \cr
%
\noalign{\hrule}
%
% Another row
  &  & 308 &  \cr
%
\noalign{\hrule}
%
% Another row
  &  &  & 105 \cr
%
\noalign{\hrule}
%
% Another row
  & 88 &  &  \cr
%
\noalign{\hrule}
%
% Another row
5.91 &  &  &  \cr
%
\noalign{\hrule}
} % End of \halign 
}$$ % End of \vbox

\vskip 10pt

There is a technique for converting between different units of measurement called ``unity fractions'' which is imperative for students of Instrumentation to master.  For more information on the ``unity fraction'' method of unit conversion, refer to the ``Unity Fractions" subsection of the ``Unit Conversions and Physical Constants'' section of the ``Physics'' chapter in your {\it Lessons In Industrial Instrumentation} textbook.

\vskip 20pt \vbox{\hrule \hbox{\strut \vrule{} {\bf Suggestions for Socratic discussion} \vrule} \hrule}

\begin{itemize}
\item{} Demonstrate how to {\it estimate} numerical answers for these conversion problems without using a calculator.
\end{itemize}

\underbar{file i03927}
%(END_QUESTION)





%(BEGIN_ANSWER)

% No blank lines allowed between lines of an \halign structure!
% I use comments (%) instead, so that TeX doesn't choke.

$$\vbox{\offinterlineskip
\halign{\strut
\vrule \quad\hfil # \ \hfil & 
\vrule \quad\hfil # \ \hfil & 
\vrule \quad\hfil # \ \hfil & 
\vrule \quad\hfil # \ \hfil \vrule \cr
\noalign{\hrule}
%
% First row
\hskip 20pt bar \hskip 20pt & \hskip 20pt PSI \hskip 20pt & \hskip 20pt inches W.C. \hskip 20pt & \hskip 20pt inches mercury \hskip 20pt \cr
%
\noalign{\hrule}
%
% Another row
0.59 & 8.557 & 236.9 & 17.42 \cr
%
\noalign{\hrule}
%
% Another row
0.2827 & 4.1 & 113.5 & 8.348 \cr
%
\noalign{\hrule}
%
% Another row
0.4982 & 7.225 & 200 & 14.71 \cr
%
\noalign{\hrule}
%
% Another row
1.185 & 17.19 & 475.8 & 35 \cr
%
\noalign{\hrule}
%
% Another row
0.7672 & 11.13 & 308 & 22.65 \cr
%
\noalign{\hrule}
%
% Another row
3.556 & 51.57 & 1428 & 105 \cr
%
\noalign{\hrule}
%
% Another row
6.068 & 88 & 2436 & 179.2 \cr
%
\noalign{\hrule}
%
% Another row
5.91 & 85.71 & 2373 & 174.5 \cr
%
\noalign{\hrule}
} % End of \halign 
}$$ % End of \vbox

%(END_ANSWER)





%(BEGIN_NOTES)

%INDEX% Physics, units and conversions: pressure

%(END_NOTES)


