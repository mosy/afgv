
%(BEGIN_QUESTION)
% Copyright 2009, Tony R. Kuphaldt, released under the Creative Commons Attribution License (v 1.0)
% This means you may do almost anything with this work of mine, so long as you give me proper credit

Read and outline the ``Foxboro Model 13A Differential Pressure Transmitter'' subsection of the ``Analysis of Practical Pneumatic Instruments'' section of the ``Pneumatic Instrumentation'' chapter in your {\it Lessons In Industrial Instrumentation} textbook.  Note the page numbers where important illustrations, photographs, equations, tables, and other relevant details are found.  Prepare to thoughtfully discuss with your instructor and classmates the concepts and examples explored in this reading.

\underbar{file i03931}
%(END_QUESTION)





%(BEGIN_ANSWER)


%(END_ANSWER)





%(BEGIN_NOTES)

Process pressure exerts force on capsule.  Force bar transfers motion through flexure to baffle.  If baffle moves closer to nozzle, backpressure is amplified and sent to bellows to restore equilibrium (near-original position ; force-balance).  Pneumatic output pressure becomes an analogue of process fluid pressure at capsule.

\vskip 10pt

Calibration: zero screw adds force to bottom of range bar (zero = add/subtract force).  Range wheel alters lever ratio between capsule and feedback bellows (span = multiply/divide force).






\vskip 20pt \vbox{\hrule \hbox{\strut \vrule{} {\bf Suggestions for Socratic discussion} \vrule} \hrule}

\begin{itemize}
\item{} {\bf Present an actual I/P to students for their inspection and analysis, challenging them to identify the components, principle of operation, and calibration adjustments of the real instrument.}
\item{} Analyze the response of this transmitter to an increased pressure at the ``high'' sensing port.
\item{} Explain how the {\it zero} and {\it span} adjustments work in this instrument.
\item{} Which way should the zero-adjustment spring be tensed to {\it increase} the transmitter's pneumatic output for any given amount of applied process fluid pressure?
\item{} Which way should the range wheel be moved to {\it increase} the transmitter's pneumatic output for any given amount of applied process fluid pressure?
\item{} Is the zero spring in this instrument a {\it tension} spring (i.e. it's being stretched) or a {\it compression} spring (i.e. it's being squished)?  How may we tell?
\item{} What would happen if the capsule were to develop a leak?
\item{} What would happen if the flexure connecting the tops of the force and range bars were to break in half, leaving those two bars disconnected from each other?
\item{} What would happen if the air supply pressure were to increase from 20 PSI to 22 PSI?
\item{} What would happen if the air supply pressure were to decrease from 20 PSI to 18 PSI?
\end{itemize}




\vfil \eject

\noindent
{\bf Prep Quiz:}

In any analog instrument (including pneumatic instruments), {\it zero} and {\it span} adjustments always employ the following principles:

\begin{itemize}
\item{} Zero adjustments add or subtract a quantity; span adjustments square or ``root'' a quantity 
\vskip 5pt
\item{} Zero adjustments add or subtract a quantity; span adjustments multiply or divide a quantity
\vskip 5pt
\item{} Zero adjustments subtract a quantity; span adjustments add a quantity 
\vskip 5pt
\item{} Zero adjustments multiply or divide a quantity; span adjustments add or subtract a quantity 
\vskip 5pt
\item{} Zero adjustments divide a quantity; span adjustments multiply a quantity 
\vskip 5pt
\item{} Zero adjustments add a quantity; span adjustments subtract a quantity 
\end{itemize}




\vfil \eject

\noindent
{\bf Summary Quiz:}

The {\it span} adjustment for a Foxboro model 13A differential pressure transmitter works by:

\begin{itemize}
\item{} Tapping into greater or fewer ``turns'' of wire in the force coil
\vskip 5pt
\item{} Varying the magnetic field strength by moving a bypass ``shunt'' bar
\vskip 5pt
\item{} Stretching or shrinking the effective area of the feedback bellows
\vskip 5pt
\item{} Changing the position of a potentiometer to divide signal voltage
\vskip 5pt
\item{} Moving the position of the fulcrum on a lever to change mechanical advantage
\vskip 5pt
\item{} Altering the tension of a spring by turning a ``preload'' screw
\end{itemize}



%INDEX% Reading assignment: Lessons In Industrial Instrumentation, Pneumatic Instrumentation (Foxboro 13A DP transmitter analysis)

%(END_NOTES)


