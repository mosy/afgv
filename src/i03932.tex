
%(BEGIN_QUESTION)
% Copyright 2009, Tony R. Kuphaldt, released under the Creative Commons Attribution License (v 1.0)
% This means you may do almost anything with this work of mine, so long as you give me proper credit

Read and outline the ``Foxboro Model E69 `I/P' Electro-Pneumatic Transducer'' subsection of the ``Analysis of Practical Pneumatic Instruments'' section of the ``Pneumatic Instrumentation'' chapter in your {\it Lessons In Industrial Instrumentation} textbook.  Note the page numbers where important illustrations, photographs, equations, tables, and other relevant details are found.  Prepare to thoughtfully discuss with your instructor and classmates the concepts and examples explored in this reading.

\vskip 10pt

A video resource you may find helpful for understanding force-balance versus motion-balance I/P converter mechanisms may be found on BTC's YouTube channel ({\tt www.youtube.com/BTCinstrumentation}).

\underbar{file i03932}
%(END_QUESTION)





%(BEGIN_ANSWER)


%(END_ANSWER)





%(BEGIN_NOTES)

4-20 mA current causes baffle to pivot on its axis, advancing the baffle toward the nozzle.  Increasing nozzle backpressure fills the bellows, causing the nozzle to retreat from the baffle (motion-balance).

\vskip 10pt

Calibration: zero screw offsets position of the nozzle (zero = add/subtract motion).  Span adjustment alters effective radius of baffle (span = multiply/divide motion).




\vskip 20pt \vbox{\hrule \hbox{\strut \vrule{} {\bf Suggestions for Socratic discussion} \vrule} \hrule}

\begin{itemize}
\item{} {\bf Present an actual I/P to students for their inspection and analysis, challenging them to identify the components, principle of operation, and calibration adjustments of the real instrument.}
\item{} Analyze the response of this I/P converter to an increased current signal.
\item{} Explain how the {\it zero} and {\it span} adjustments work in this instrument.
\end{itemize}

%INDEX% Reading assignment: Lessons In Industrial Instrumentation, Pneumatic Instrumentation (Foxboro E69 I/P transducer analysis)

%(END_NOTES)


