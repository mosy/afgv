
%(BEGIN_QUESTION)
% Copyright 2009, Tony R. Kuphaldt, released under the Creative Commons Attribution License (v 1.0)
% This means you may do almost anything with this work of mine, so long as you give me proper credit

Read and outline the ``Fisher Model 546 `I/P' Electro-Pneumatic Transducer'' subsection of the ``Analysis of Practical Pneumatic Instruments'' section of the ``Pneumatic Instrumentation'' chapter in your {\it Lessons In Industrial Instrumentation} textbook.  Note the page numbers where important illustrations, photographs, equations, tables, and other relevant details are found.  Prepare to thoughtfully discuss with your instructor and classmates the concepts and examples explored in this reading.

\vskip 10pt

A video resource you may find helpful for understanding force-balance versus motion-balance I/P converter mechanisms may be found on BTC's YouTube channel ({\tt www.youtube.com/BTCinstrumentation}).

\underbar{file i03933}
%(END_QUESTION)





%(BEGIN_ANSWER)


%(END_ANSWER)





%(BEGIN_NOTES)

4-20 mA current magnetizes beam, causing it to torque clockwise (toward the nozzle).  Nozzle backpressure fills bellows to restore equilibrium (near-original position ; force-balance).  Relay provides ~2:1 gain to amplify backpressure into pneumatic output pressure.

\vskip 10pt

Different-sized bellows used to establish different ranges.  Big bellows = 3-15 PSI ; small bellows = 6-30 PSI.

\vskip 10pt

Calibration: zero screw adds force to beam (zero = add/subtract force).  Span adjustment is a shunt plate dividing the magnetic field (span = multiply/divide force).





\vskip 20pt \vbox{\hrule \hbox{\strut \vrule{} {\bf Suggestions for Socratic discussion} \vrule} \hrule}

\begin{itemize}
\item{} {\bf Present an actual I/P to students for their inspection and analysis, challenging them to identify the components, principle of operation, and calibration adjustments of the real instrument.}
\item{} Analyze the response of this I/P converter to an increased current signal.
\item{} Explain how the {\it zero} and {\it span} adjustments work in this instrument.
\item{} Which way should the zero-adjustment spring be tensed to {\it increase} the pneumatic output?
\item{} Which way should the magnetic shunt plate be moved to {\it increase} the pneumatic output?
\item{} What would happen if some of the turns in the electromagnet coil were shorted past?  Would this cause a {\it zero} shift, a {\it span} shift, or a {\it linearity} shift?
\end{itemize}







\vfil \eject

\noindent
{\bf Summary Quiz:}

The {\it span} adjustment for a Fisher model 546 I/P converter works by:

\begin{itemize}
\item{} Tapping into greater or fewer ``turns'' of wire in the force coil
\vskip 5pt
\item{} Varying the magnetic field strength by moving a bypass ``shunt'' bar
\vskip 5pt
\item{} Stretching or shrinking the effective area of the feedback bellows
\vskip 5pt
\item{} Changing the position of a potentiometer to divide signal voltage
\vskip 5pt
\item{} Moving the position of the fulcrum on a lever to change mechanical advantage
\vskip 5pt
\item{} Altering the tension of a spring by turning a ``preload'' screw
\end{itemize}




%INDEX% Reading assignment: Lessons In Industrial Instrumentation, Pneumatic Instrumentation (Fisher 546 I/P transducer analysis)

%(END_NOTES)


