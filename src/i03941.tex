
%(BEGIN_QUESTION)
% Copyright 2009, Tony R. Kuphaldt, released under the Creative Commons Attribution License (v 1.0)
% This means you may do almost anything with this work of mine, so long as you give me proper credit

Skim the ``Continuous Level Measurement'' chapter in your {\it Lessons In Industrial Instrumentation} textbook to specifically answer these questions:

\vskip 10pt

Describe and explain the mathematical relationship between liquid density, liquid depth (or height), and hydrostatic pressure.  How may we exploit this principle to measure liquid level in industrial processes?

\vskip 10pt

Explain what a {\it bubbler} system is, and how it is used to measure liquid level.  Why would anyone opt to use a bubbler instead of directly connecting a pressure sensor to the process vessel?


\vskip 20pt \vbox{\hrule \hbox{\strut \vrule{} {\bf Suggestions for Socratic discussion} \vrule} \hrule}

\begin{itemize}
\item{} Identify different strategies for ``skimming'' a text, as opposed to reading that text closely.  Why do you suppose the ability to quickly scan a text is important in this career?
\item{} You probably noticed a lot of math applied in these sections of the textbook.  Identify some good learning strategies to apply when learning mathematically ``dense'' topics.
\end{itemize}

\underbar{file i03941}
%(END_QUESTION)





%(BEGIN_ANSWER)


%(END_ANSWER)





%(BEGIN_NOTES)

Any vertical column of liquid exerts a hydrostatic pressure proportional to the density of the liquid and to the vertical height of the column:

$$P = \rho g h = \gamma h$$

\noindent
Where,

$P$ = Hydrostatic pressure

$\rho$ = Mass density of fluid in kilograms per cubic meter (metric) or slugs per cubic foot (British)

$g$ = Acceleration of gravity

$\gamma$ = Weight density of fluid in newtons per cubic meter (metric) or pounds per cubic foot (British)

$h$ = Height of vertical fluid column above point of pressure measurement

\vskip 10pt

We may also calculate the pressure exerted by a liquid column by calculating the pressure generated by a water column of the same height, then correcting for the difference in densities using specific gravity as a multiplying factor.

\vskip 10pt

{\it Bubbler} or {\it dip tube} systems use a slow flow of purge gas down a tube submerged in the liquid to transfer the hydrostatic pressure of that liquid to a pressure instrument remotely mounted, sensing the backpressure of the purge gas.

%INDEX% Reading assignment: Lessons In Industrial Instrumentation, Continuous Level Measurement (hydrostatic)

%(END_NOTES)


