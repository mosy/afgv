
%(BEGIN_QUESTION)
% Copyright 2015, Tony R. Kuphaldt, released under the Creative Commons Attribution License (v 1.0)
% This means you may do almost anything with this work of mine, so long as you give me proper credit

Skim the ``Continuous Level Measurement'' chapter in your {\it Lessons In Industrial Instrumentation} textbook to specifically answer these questions:

\vskip 10pt

Describe what {\it Archimedes' Principle} is, and how it may be used to infer the level of a liquid.

\vskip 10pt

Is a ``displacement'' type of liquid level sensor affected by changes in liquid density?  Explain why or why not.  Are they affected by changes in process fluid pressure?  Explain why or why not.


\vskip 20pt \vbox{\hrule \hbox{\strut \vrule{} {\bf Suggestions for Socratic discussion} \vrule} \hrule}

\begin{itemize}
\item{} Identify different strategies for ``skimming'' a text, as opposed to reading that text closely.  Why do you suppose the ability to quickly scan a text is important in this career?
\item{} You probably noticed a lot of math applied in these sections of the textbook.  Identify some good learning strategies to apply when learning mathematically ``dense'' topics.
\end{itemize}

\underbar{file i03942}
%(END_QUESTION)





%(BEGIN_ANSWER)


%(END_ANSWER)





%(BEGIN_NOTES)

Archimedes' Principle states that the buoyant force experienced by a submerged object is equal to the volume of fluid displaced by that object.  If we measure the apparent weight of a submerged object, we may ascertain how deeply that object is submerged in the liquid.

$$F = \gamma V$$

Since density ($\gamma$) is part of the buyoant force formula, it is relevant to the calibration of a buyoancy-type level instrument.


%INDEX% Reading assignment: Lessons In Industrial Instrumentation, Continuous Level Measurement (displacement)

%(END_NOTES)


