
%(BEGIN_QUESTION)
% Copyright 2009, Tony R. Kuphaldt, released under the Creative Commons Attribution License (v 1.0)
% This means you may do almost anything with this work of mine, so long as you give me proper credit

Skim the ``Continuous Level Measurement'' chapter in your {\it Lessons In Industrial Instrumentation} textbook to specifically answer these questions:

\vskip 10pt

Explain how the weight of a vessel may be measured to infer the level of process substance (solid or liquid) inside that vessel.  Describe what sort of device is typically used to sense the vessel's weight.

\vskip 10pt

Describe the special construction of vessels, pipes, and other structures in a weight-based level measurement system.  Why are these special features necessary for accurate level measurement?

\vskip 10pt

Is a ``weight'' type of liquid level sensor affected by changes in liquid density?  Explain why or why not.


\vskip 20pt \vbox{\hrule \hbox{\strut \vrule{} {\bf Suggestions for Socratic discussion} \vrule} \hrule}

\begin{itemize}
\item{} Identify different strategies for ``skimming'' a text, as opposed to reading that text closely.  Why do you suppose the ability to quickly scan a text is important in this career?
\end{itemize}

\underbar{file i03943}
%(END_QUESTION)





%(BEGIN_ANSWER)


%(END_ANSWER)





%(BEGIN_NOTES)

If we mount a vessel on load cells, the weight of that vessel may be used to infer level.  Of course, we must subtract the ``tare weight'' of the vessel itself.  In order for this to work, the vessel must be isolated from mechanical stresses that might otherwise make the vessel appear to be heavier or lighter than it really is.

\vskip 10pt

If the liquid's density changes, it will affect the relationship between weight and height (level).



%INDEX% Reading assignment: Lessons In Industrial Instrumentation, Continuous Level Measurement (weight)

%(END_NOTES)


