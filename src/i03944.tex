
%(BEGIN_QUESTION)
% Copyright 2009, Tony R. Kuphaldt, released under the Creative Commons Attribution License (v 1.0)
% This means you may do almost anything with this work of mine, so long as you give me proper credit

Skim the ``Continuous Level Measurement'' chapter in your {\it Lessons In Industrial Instrumentation} textbook to specifically answer these questions:

\vskip 10pt

Identify three different types of ``echo'' level measurement technologies, and explain how each one works to measure the level of solid or liquid material inside a vessel.

\vskip 10pt

Are ``echo'' sensors affected by changes in process material density?  Explain why or why not.


\vskip 20pt \vbox{\hrule \hbox{\strut \vrule{} {\bf Suggestions for Socratic discussion} \vrule} \hrule}

\begin{itemize}
\item{} Identify different strategies for ``skimming'' a text, as opposed to reading that text closely.  Why do you suppose the ability to quickly scan a text is important in this career?
\end{itemize}

\underbar{file i03944}
%(END_QUESTION)





%(BEGIN_ANSWER)


%(END_ANSWER)





%(BEGIN_NOTES)

\begin{itemize}
\item{} Ultrasonic (sound waves)
\item{} Radar (radio waves)
\item{} Laser (light waves)
\item{} Magnetostrictive (torsion stress wave)
\end{itemize}

Echo-type level instruments are immune to changes in liquid density, unless the wave has to travel through the liquid in which case the liquid's density alters the speed of that wave through the liquid.  The gas or vapor medium above the liquid also has an effect on wave propagation speed, and therefore can affect instrument accuracy.

%INDEX% Reading assignment: Lessons In Industrial Instrumentation, Continuous Level Measurement (echo)

%(END_NOTES)


