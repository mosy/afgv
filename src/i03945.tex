
%(BEGIN_QUESTION)
% Copyright 2009, Tony R. Kuphaldt, released under the Creative Commons Attribution License (v 1.0)
% This means you may do almost anything with this work of mine, so long as you give me proper credit

Skim the ``Continuous Level Measurement'' chapter in your {\it Lessons In Industrial Instrumentation} textbook to specifically answer these questions:

\vskip 10pt

Describe how electrical capacitance maybe used to infer the level of a process material inside a vessel. 

\vskip 10pt

Identify two different categories of capacitance probe, and explain why each type has its purpose in different process applications.

\vskip 10pt

Is a capacitive level sensor affected by changes in liquid density?  Explain why or why not.


\vskip 20pt \vbox{\hrule \hbox{\strut \vrule{} {\bf Suggestions for Socratic discussion} \vrule} \hrule}

\begin{itemize}
\item{} Identify different strategies for ``skimming'' a text, as opposed to reading that text closely.  Why do you suppose the ability to quickly scan a text is important in this career?
\end{itemize}

\underbar{file i03945}
%(END_QUESTION)





%(BEGIN_ANSWER)


%(END_ANSWER)





%(BEGIN_NOTES)

The amount of capacitance between a metal rod and the vessel itself changes as liquid level changes, owing to changes in permittivity and distance:

$$C = {\epsilon A \over d}$$

Two types of capacitive level sensors exist: one for conductive liquids and one for non-conductive liquids.  Capacitive probes designed for conductive liquids are sheathed in insulating material (e.g. plastic).  Capacitive probes designed for non-conducting liquids are bare metal, using the liquid itself as the capacitive dielectric.

\vskip 10pt

So long as changes in liquid density do not alter permittivity, these instruments will be unaffected by liquid density.

%INDEX% Reading assignment: Lessons In Industrial Instrumentation, Continuous Level Measurement (capacitance)

%(END_NOTES)


