
%(BEGIN_QUESTION)
% Copyright 2009, Tony R. Kuphaldt, released under the Creative Commons Attribution License (v 1.0)
% This means you may do almost anything with this work of mine, so long as you give me proper credit

Read and outline the ``Fluid Density Expressions'' subsection of the ``Fluid Mechanics'' section of the ``Physics'' chapter in your {\it Lessons In Industrial Instrumentation} textbook.  Note the page numbers where important illustrations, photographs, equations, tables, and other relevant details are found.  Prepare to thoughtfully discuss with your instructor and classmates the concepts and examples explored in this reading.

\vskip 20pt \vbox{\hrule \hbox{\strut \vrule{} {\bf Suggestions for Socratic discussion} \vrule} \hrule}

\begin{itemize}
\item{} As the specific gravity of a liquid increases, does its ``degrees API'' density value increase or decrease?
\item{} As the specific gravity of a liquid increases, does its ``degrees Twaddell'' density value increase or decrease?
\item{} As the specific gravity of a liquid increases, does its ``degrees Baum\'e (light)'' density value increase or decrease?
\item{} As the specific gravity of a liquid increases, does its ``degrees Baum\'e (heavy)'' density value increase or decrease?
\item{} What value of specific gravity yields a {\it zero} degree API density figure?
\item{} What value of specific gravity yields a {\it zero} degree Twaddell density figure?
\item{} What value of specific gravity yields a {\it zero} degrees Baum\'e (light) density figure?
\item{} What value of specific gravity yields a {\it zero} degrees Baum\'e (heavy) density figure?
\end{itemize}

\underbar{file i03947}
%(END_QUESTION)





%(BEGIN_ANSWER)


%(END_ANSWER)





%(BEGIN_NOTES)

Density is a ratio of mass or weight per unit volume.  This may also be expressed as a ratio to the density of some reference substance, like water (in the case of liquids) or air (in the case of gases), in which case it is referred to as {\it specific gravity}.

\vskip 10pt

{\it Specific volume} is density expressed in reciprocal form: as a ratio of volume to mass.  This is commonly done with steam.

\vskip 10pt

Some industry-specific ``degree'' units exist for the expression of fluid density.

$$\hbox{Degrees API} = {141.5 \over \hbox{Specific gravity}} - 131.5$$

$$\hbox{Degrees Twaddell} = 200 \times (\hbox{Specific gravity} - 1)$$

$$\hbox{Degrees Baum\'e (light)} = {140 \over \hbox{Specific gravity}} - 130$$

$$\hbox{Degrees Baum\'e (heavy)} = 145 - {145 \over \hbox{Specific gravity}}$$

$$\hbox{Degrees Baum\'e (heavy, old Dutch)} = 144 - {144 \over \hbox{Specific gravity}}$$

$$\hbox{Degrees Baum\'e (heavy, Gerlach scale)} = 146.78 - {146.78 \over \hbox{Specific gravity}}$$







\vfil \eject

\noindent
{\bf Summary Quiz:}

Suppose a sample of fuel oil taken at a refinery measures 0.82 specific gravity.  Feel free to use your textbook as a reference, and convert this specific gravity value to degrees API:


\begin{itemize}
\item{} -15.47 $^{o}$API 
\vskip 5pt 
\item{} 304.1 $^{o}$API 
\vskip 5pt 
\item{} 13.21 $^{o}$API 
\vskip 5pt 
\item{} 41.06 $^{o}$API
\vskip 5pt 
\item{} 434.5 $^{o}$API 
\vskip 5pt 
\item{} 39.23 $^{o}$API 
\end{itemize}

%INDEX% Reading assignment: Lessons In Industrial Instrumentation, Fluid mechanics (Fluid density expressions)

%(END_NOTES)


