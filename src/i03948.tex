
%(BEGIN_QUESTION)
% Copyright 2009, Tony R. Kuphaldt, released under the Creative Commons Attribution License (v 1.0)
% This means you may do almost anything with this work of mine, so long as you give me proper credit

Read and outline the ``Pressure of a Fluid Column'' subsection of the ``Hydrostatic Pressure'' section of the ``Continuous Level Measurement'' chapter in your {\it Lessons In Industrial Instrumentation} textbook.  Note the page numbers where important illustrations, photographs, equations, tables, and other relevant details are found.  Prepare to thoughtfully discuss with your instructor and classmates the concepts and examples explored in this reading.

\underbar{file i03948}
%(END_QUESTION)





%(BEGIN_ANSWER)


%(END_ANSWER)





%(BEGIN_NOTES)

The hydrostatic pressure generated by the weight of a fluid column is strictly determined by height and density, not by vessel size or shape:

$$P = \rho g h = \gamma h$$

We may also calculate hydrostatic pressure for any liquid by first calculating the pressure a column of {\it water} the same height would generate, then correcting for the specific gravity of the liquid in question.

\vskip 10pt

If the vessel's cross-sectional area is constant, height will be proportional to volume.  If the liquid density is constant, therefore, hydrostatic pressure will be proportional both to level and to volume.  However, liquid density will have no effect on measurements of {\it total liquid mass}, since any change in density will have complementary effects on density and height.

\vskip 10pt

Special {\it extended diaphragm} pressure transmitters made for close-coupled mounting to the bottom of a tank, eliminating the impulse line normally connecting the tank to the transmitter's ``H'' port.



\vskip 20pt \vbox{\hrule \hbox{\strut \vrule{} {\bf Suggestions for Socratic discussion} \vrule} \hrule}

\begin{itemize}
\item{} {\bf In what ways may a hydrostatic level instrument be ``fooled'' to report a false level measurement?}
\item{} Explain how the two different calculation techniques presented in the textbook work to predict the amount of hydrostatic pressure generated by a fluid column.  Do you find either technique to be easier than the other?
\item{} Explain the textbook's method of converting pressure units of PSF into PSI using unity fractions.
\item{} If the process vessel is cube-shaped, will hydrostatic pressure be proportional to liquid level, liquid volume, or both?
\item{} If the process vessel is a vertical cylinder, will hydrostatic pressure be proportional to liquid level, liquid volume, or both?
\item{} If the process vessel is a horizontal cylinder, will hydrostatic pressure be proportional to liquid level, liquid volume, or both?
\item{} Identify the purpose of an extended-diaphragm level transmitter.
\item{} Explain why a change in fluid density will not affect total mass measurement so long as the vessel's cross-sectional area all along its height remains constant.
\end{itemize}







\vfil \eject

\noindent
{\bf Prep Quiz:}

Calculate the hydrostatic pressure exerted at the bottom of a liquid column 4 feet tall, with the liquid having a weight density ($\gamma$) of 55 pounds per cubic feet.

\begin{itemize}
\item{} 110.7 PSI
\vskip 5pt 
\item{} 220.0 PSI
\vskip 5pt 
\item{} 13.75 PSI
\vskip 5pt 
\item{} 1.733 PSI
\vskip 5pt 
\item{} 1.53 PSI
\vskip 5pt 
\item{} 97.59 PSI
\end{itemize}














\vfil \eject

\noindent
{\bf Prep Quiz:}

Calculate the hydrostatic pressure exerted at the bottom of a water column 4 feet tall ($\gamma$ = 62.4 lb/ft$^{3}$).

\begin{itemize}
\item{} 2995 PSI
\vskip 5pt 
\item{} 249.6 PSI
\vskip 5pt 
\item{} 1.733 PSI
\vskip 5pt 
\item{} 9.02 PSI
\vskip 5pt 
\item{} 110.7 PSI
\vskip 5pt 
\item{} 15.6 PSI
\end{itemize}














\vfil \eject

\noindent
{\bf Prep Quiz:}

Calculate the hydrostatic pressure exerted at the bottom of a liquid column 6 feet tall (specific gravity = 0.8).

\begin{itemize}
\item{} 4.8 PSI
\vskip 5pt 
\item{} 90 PSI
\vskip 5pt 
\item{} 28.3 PSI
\vskip 5pt 
\item{} 2.08 PSI
\vskip 5pt 
\item{} 7.5 PSI
\vskip 5pt 
\item{} 57.6 PSI
\end{itemize}














\vfil \eject

\noindent
{\bf Prep Quiz:}

Calculate the hydrostatic pressure exerted at the bottom of a liquid column 3 feet tall (specific gravity = 1.4).

\begin{itemize}
\item{} 25.7 PSI
\vskip 5pt 
\item{} 50.4 PSI
\vskip 5pt 
\item{} 2.14 PSI
\vskip 5pt 
\item{} 4.2 PSI
\vskip 5pt 
\item{} 24.8 PSI
\vskip 5pt 
\item{} 1.82 PSI
\end{itemize}






%INDEX% Reading assignment: Lessons In Industrial Instrumentation, Continuous Level Measurement (hydrostatic pressure)

%(END_NOTES)


