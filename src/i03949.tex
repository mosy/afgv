
%(BEGIN_QUESTION)
% Copyright 2009, Tony R. Kuphaldt, released under the Creative Commons Attribution License (v 1.0)
% This means you may do almost anything with this work of mine, so long as you give me proper credit

Read and outline the ``Bubbler Systems'' subsection of the ``Hydrostatic Pressure'' section of the ``Continuous Level Measurement'' chapter in your {\it Lessons In Industrial Instrumentation} textbook.  Note the page numbers where important illustrations, photographs, equations, tables, and other relevant details are found.  Prepare to thoughtfully discuss with your instructor and classmates the concepts and examples explored in this reading.

\underbar{file i03949}
%(END_QUESTION)





%(BEGIN_ANSWER)


%(END_ANSWER)





%(BEGIN_NOTES)

Bubbler systems work by slowly bubbling a purge gas out the end of a submerged tube.  As the bubbles escape out the end of this ``dip tube'', the pressure of the purge gas inside the tube will become equal to the hydrostatic pressure of the liquid at the bottom end of that tube.  This allows us to transfer the pressure of that liquid to a remote location where a pressure sensor may measure it (dry).  Either a rotameter or a sightfeed bubbler may be used to monitor the purge gas flow rate and ensure that it is modest.  Too much purge gas flow will cause frictional pressure losses along the length of the purge gas tubing and dip tube, causing positive level measurement errors.  Maintaining a modest flow of purge gas also helps to conserve purge gas, in cases where it is expensive.

\vskip 10pt

Bubbler systems tend to produce slightly varying indications, as the dip tube pressure oscillates with each bubble's escape out the end of the tube.  Such oscillation is usually proportional in magnitude to the diameter of the tube, and in frequency to the purge gas flow rate.  Damping programmed into the transmitter may minimize this effect.








\vskip 20pt \vbox{\hrule \hbox{\strut \vrule{} {\bf Suggestions for Socratic discussion} \vrule} \hrule}

\begin{itemize}
\item{} How can we tell that the purge flow rate for a bubbler system is correct?
\item{} Identify any problem(s) which may result from having a purge flow rate that is too low.
\item{} Identify any problem(s) which may result from having a purge flow rate that is too high.
\item{} Explain why the end of a dip tube is sometimes notched.
\item{} Explain the purpose of a {\it sightfeed bubbler} in a purged level measurement system.
\item{} Identify factors important to the selection of the gas used to purge the bubble tube.
\item{} Explain why the gas pressure in a bubbler system oscillates somewhat.
\end{itemize}
















\vfil \eject

\noindent
{\bf Prep Quiz:}

A {\it bubbler} (or {\it dip tube}) level measurement system most is useful in applications where:

\begin{itemize}
\item{} The process liquid is much denser than water
\vskip 5pt 
\item{} The DP transmitter is pneumatic rather than electronic
\vskip 5pt 
\item{} The process demands absolute level measurement accuracy
\vskip 5pt 
\item{} The process liquid level often changes rapidly
\vskip 5pt 
\item{} The process liquid is extremely corrosive
\vskip 5pt 
\item{} The process liquid is much lighter than water 
\end{itemize}

%INDEX% Reading assignment: Lessons In Industrial Instrumentation, Continuous Level Measurement (hydrostatic pressure -- bubbler)

%(END_NOTES)


