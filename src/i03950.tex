
%(BEGIN_QUESTION)
% Copyright 2009, Tony R. Kuphaldt, released under the Creative Commons Attribution License (v 1.0)
% This means you may do almost anything with this work of mine, so long as you give me proper credit

Read and outline the ``Transmitter Suppression and Elevation'' subsection of the ``Hydrostatic Pressure'' section of the ``Continuous Level Measurement'' chapter in your {\it Lessons In Industrial Instrumentation} textbook.  Note the page numbers where important illustrations, photographs, equations, tables, and other relevant details are found.  Prepare to thoughtfully discuss with your instructor and classmates the concepts and examples explored in this reading.


\underbar{file i03950}
%(END_QUESTION)





%(BEGIN_ANSWER)


%(END_ANSWER)





%(BEGIN_NOTES)

If a pressure sensor is mounted below the 0\% liquid level in a vessel, it will see a positive hydrostatic pressure shifting the zero (LRV) of the measurement range upward.  This is called {\it zero suppression}.

\vskip 10pt

If a pressure sensor is mounted above the 0\% liquid level in a vessel, it will see a negative hydrostatic pressure shifting the zero (LRV) of the measurement range downward.  This is called {\it zero elevation}.  In such installations it is critical that the transmitter be equipped with a remote seal, so that fill fluid does not dribble out of the impulse line during empty-vessel conditions.








\vskip 20pt \vbox{\hrule \hbox{\strut \vrule{} {\bf Suggestions for Socratic discussion} \vrule} \hrule}

\begin{itemize}
\item{} A helpful ``active reading'' technique for technical texts is to work through every mathematical example presented, to ensure you understand the math as you read along.  Apply this technique here, demonstrating how to work through at least one of the calculation examples presented in the textbook.
\item{} Describe a level-measurement scenario where an elevated or suppressed transmitter would have to be used.
\item{} Explain in your own words how to calculate the LRV and URV settings for a hydrostatic level transmitter, based on dimensions given to you in a diagram.
\item{} Explain why remote seals are necessary when the transmitter is {\it elevated} above the process connection point.
\end{itemize}




%INDEX% Reading assignment: Lessons In Industrial Instrumentation, Continuous Level Measurement (hydrostatic pressure -- elevation and suppression)

%(END_NOTES)


