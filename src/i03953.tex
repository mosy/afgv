
%(BEGIN_QUESTION)
% Copyright 2009, Tony R. Kuphaldt, released under the Creative Commons Attribution License (v 1.0)
% This means you may do almost anything with this work of mine, so long as you give me proper credit

Read portions of Product Flier PF11.2:2390 for the Fisher model 2390 and 2500 displacement-type level transmitters (``Level-Trol''), and Instruction Manual ``Form 5729'' for the model 249W level sensor, both published by Fisher, and answer the following questions:

\vskip 10pt

Describe what a {\it cage} is and how it relates to the construction of a displacement-style liquid level transmitter.  Also describe what a {\it cageless} transmitter installation looks like.

\vskip 10pt

Identify a page in at least one of these documents where the torque tube assembly is clearly shown.  Do your best to describe how it functions.

\vskip 10pt

Figure 10 found on page 9 of the Instruction Manual ``Form 5729'' for the model 249W level sensor shows how the displacer connects to the end of the displacer rod.  A small piece of spring-steel wire called a ``cotter spring'' inserts into the displacer end piece to hold the displacer rod's ball into the socket of the end piece.  Explain what could happen if this cotter spring is accidently omitted from the assembly, or is improperly inserted, such that the displacer rod is able to slip out of the end piece.  How, exactly, will this affect the operation of the instrument?

\vskip 20pt \vbox{\hrule \hbox{\strut \vrule{} {\bf Suggestions for Socratic discussion} \vrule} \hrule}

\begin{itemize}
\item{} Suppose the displacer were removed from a Fisher ``Level-Trol'' instrument and replaced with one having the same weight and length but smaller diameter.  How would this change the instrument's response to process liquid level?
\item{} Suppose the displacer were removed from a Fisher ``Level-Trol'' instrument and replaced with one having the same weight and diameter but shorter length.  How would this change the instrument's response to process liquid level?
\item{} Suppose the torque tube were removed from a Fisher ``Level-Trol'' instrument and replaced with one having a stiffer spring constant.  How would this change the instrument's response to process liquid level?
\item{} Displacers are hollow, but filled with enough lead shot to prevent the displacer from ever floating.  Explain how the instrument's response would be affected by a leak developing in the displacer, such that process fluid is able to fill it up.
\item{} Describe a calibration procedure suitable for either type (caged or cageless) ``Level-Trol'' transmitter.
\end{itemize}

\underbar{file i03953}
%(END_QUESTION)





%(BEGIN_ANSWER)


%(END_ANSWER)





%(BEGIN_NOTES)

Caged versus cageless sensors are contrasted on page 3 of the Fisher 2390/2500 product flier, as well as on pages 3 (cageless) versus 4 (caged) of the 249W manual.

\vskip 10pt

Page 6 of the Fisher 2390/2500 product flier shows a clear view of the torque tube, as does page 14 of the Fisher 249W instruction manual.

\vskip 10pt

If the cotter spring is omitted from the assembly and the displacer rod becomes uncoupled from the displacer end piece, the displacer will fall off the end of the rod.  This relieves all the displacer's weight from the rod, making the LevelTrol unit ``think'' there is a very large buoyant force applied to the displacer.  This will be interpreted as a (saturated) high level.  Incidentally, this is a relatively common failure mode for this type of instrument following maintenance, caused by improper assembly.

%INDEX% Reading assignment: Fisher LevelTrol displacer level transmitter and 249W level sensor manuals

%(END_NOTES)


