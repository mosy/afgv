
%(BEGIN_QUESTION)
% Copyright 2011, Tony R. Kuphaldt, released under the Creative Commons Attribution License (v 1.0)
% This means you may do almost anything with this work of mine, so long as you give me proper credit

Read and outline the ``Buoyancy'' subsection of the ``Fluid Mechanics'' section of the ``Physics'' chapter in your {\it Lessons In Industrial Instrumentation} textbook.  Note the page numbers where important illustrations, photographs, equations, tables, and other relevant details are found.  Prepare to thoughtfully discuss with your instructor and classmates the concepts and examples explored in this reading.

\underbar{file i03954}
%(END_QUESTION)





%(BEGIN_ANSWER)


%(END_ANSWER)





%(BEGIN_NOTES)

When an object is immersed in a fluid, the displacement of that fluid exerts an upward force on that object equal to the weight of the volume of fluid displaced.  This force is called {\it buoyant} force, and the principle is called {\it Archimedes' Principle}:

$$F_{buoyant} = \gamma V_{displaced}$$

It holds true for gases (e.g. hot-air and helium balloons) as well as liquids (e.g. ships, submarines).

\vskip 10pt

If we submerged a denser-than-water object in water, we may calculate its specific gravity as follows:

$$\hbox{Specific Gravity } = {m_{dry} \over {m_{dry} - m_{wet}}} = {m_{dry}g \over {m_{dry}g - m_{wet}g}} = {\hbox{Dry weight} \over \hbox{Dry weight} - \hbox{Wet weight}}$$

Hydrometers measure liquid density using a calibrated weight which floats at different heights depending on the density of the sampled liquid.  These are used in industry for the measurement of alcohol concentration, given that pure alcohol is less dense than diluted alcohol.

Multi-ball hygrometers give qualitative measurements of density.  These are commonly used for checking automobile engine antifreeze concentration, as glycol is denser than water.  They are also commonly used to check lead-acid battery charge state, since higher charge is accompanied by a stronger concentration of sulfuric acid in the electrolyte, which makes the electrolyte denser.

\vskip 20pt \vbox{\hrule \hbox{\strut \vrule{} {\bf Suggestions for Socratic discussion} \vrule} \hrule}

\begin{itemize}
\item{} How may we ascertain the amount of cargo held on a ship by examining its waterline?
\item{} What is the criterion for an object to {\it float} in water?
\item{} Legend has it that Archimedes discovered the principle of buoyancy while contemplating how to test a king's crown to see if it was made of pure gold (rather than gold-plated lead or some other lesser metal).  Explain how this could be done using buoyancy.
\item{} Use dimensional analysis to prove that the buoyant force formula $F = \gamma V$ is correct.
\item{} Suppose you ordered a drink at a bar, and wanted to empirically ascertain how strong the alcohol content of that drink was compared to some of the other drinks served there.  Devise an experiment based on buoyancy that could compare the alcohol content of various drinks.
\item{} Explain how a {\it hydrometer} is used to measure the alcohol content of a beverage.
\item{} Explain how a {\it hydrometer} is used to measure the freezing point of antifreeze coolant.
\item{} Explain how a {\it hydrometer} is used to measure the charge of a lead-acid battery.
\end{itemize}

%INDEX% Reading assignment: Lessons In Industrial Instrumentation, Buoyancy

%(END_NOTES)


