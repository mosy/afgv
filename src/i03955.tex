
%(BEGIN_QUESTION)
% Copyright 2009, Tony R. Kuphaldt, released under the Creative Commons Attribution License (v 1.0)
% This means you may do almost anything with this work of mine, so long as you give me proper credit

Read selected sections of the US Chemical Safety and Hazard Investigation Board's report (2005-04-I-TX) of the 2005 BP Texas City oil refinery explosion, and answer the following questions.

\vskip 10pt

Describe in your own words how the situation progressed from the raffinate splitter tower start-up time to the explosion.  How did instrumentation play a part in allowing this accident to happen?

\vskip 10pt

Read pages 320-323 of the report, and identify the following:

\begin{itemize}
\item{} Explain why, based on the report and also what you know about displacer-style level transmitters, why LT-5100 was incapable of registering liquid levels greater than 100\% of its measurement range.
\item{} How was the specific gravity (density) of the process liquid for level transmitter LT-5100 important in the cause of the accident?
\item{} Explain why level transmitter LT-5100 indicated {\it less than 100\% level and falling} even though the transmitter cage was completely full of liquid (``flooded'').
\item{} Describe the role that instrument documentation played in the mis-calibration of level transmitter LT-5100.
\end{itemize}

\vskip 20pt \vbox{\hrule \hbox{\strut \vrule{} {\bf Suggestions for Socratic discussion} \vrule} \hrule}

\begin{itemize}
\item{} An important safety policy at many industrial facilities is something called {\it stop-work authority}, which means any employee has the right to stop work they question as unsafe.  Explain how stop-work authority could have been applied to this particular incident.
\item{} An industry trend since this unfortunate event has been the installation of {\it redundant} level-sensing instruments for critical process vessels.  Identify the characteristics an engineer might consider when choosing redundant level transmitters for a safety application in an oil refinery.
\end{itemize}


\underbar{file i03955}
%(END_QUESTION)





%(BEGIN_ANSWER)


%(END_ANSWER)





%(BEGIN_NOTES)

{\bf Overview:} operators flooded a distillation tower by introducing liquid into it with no exit flow.  Certain instruments and alarms provided false information to the operators about the tower's fill status.  The result was that the tower became completely liquid-filled.  The liquid exited the tower through a set of relief valves, entering a ``blowdown drum'' designed to vent vapors to atmosphere.  This drum also overfilled, sending a liquid geyser of flammable hydrocarbons out the vent stack.  This liquid excursion formed a large vapor cloud which ignited, killing 15 people in nearby trailers.  (Page 21)

\vskip 10pt

As liquid temperature in the raffinate tower increased, the liquid's density decreased.  This caused the displacer-type level transmitter LT-5100 to register a decreasing liquid level even though the tower was over-filling.  (Page 320)

\vskip 10pt

The calibrated specific gravity for LT-5100 was 0.8, and it was verified to register just less than 100\% level when at ambient temperature with a full cage.  Third-party testing showed that the instrument would register 100\% exactly as it should for a liquid with a specific gravity 0f 0.705.  Raffinate at ambient temperature has a specific gravity of only 0.67, demonstrating that the instrument wasn't even properly calibrated for the application to begin with.  (Page 320)

\vskip 10pt

Raffinate has a specific gravity of only 0.55 when at 300 degrees Fahrenheit, the temperature it was at during the over-fill event.  This is only 78\% of the verified full-scale density that LT-5100 responded to.  This meant LT-5100 would only indicate 78\% full when the cage was flooded.  (Page 321)

\vskip 10pt

The instrument data sheet for LT-5100 had not been updated since 1975 when that tower was part of a different process with a different liquid (!).  This accident occurred in 2005 -- a full 30 years later.  Furthermore, the instrument technicians reported this datasheet was not even available to them at all, because they were kept in an engineer's office and not typically given to technicians for calibration work (!).  (Page 322)

\vskip 10pt

LT-5100 had been flagged as troublesome, but work could not be done because the block valves would not shut off completely.  Although the valves were replaced during the turnaround, the transmitter was not serviced prior to unit start-up (!).  (Page 323)

\vskip 10pt

The sightglass (LG-1002 A/B) was too dirty to be of use, and also could not be serviced due to the leaking block valves.  After the block valves were replaced, no one cleaned the sightglass.  (Page 323)

\vskip 20pt

High level switch LSH-5102 had been flagged as failed several times in the two years prior to the event.  An instrument technician replaced part of this switch, but the problems persisted.  Once again, block valves were leaking which prevented earlier repair of this instrument.  Plant records showed closed work orders which contradicted worker reports that the repairs had never been completed.  Post-mortem analysis indicated the level switch had mechanical problems which could not have been repaired without removing it from the vessel (thus requiring functioning block valves).  (Pages 323-324)



%INDEX% Reading assignment: USCSB report on the 2005 BP Texas City refinery explosion

%(END_NOTES)


