
%(BEGIN_QUESTION)
% Copyright 2009, Tony R. Kuphaldt, released under the Creative Commons Attribution License (v 1.0)
% This means you may do almost anything with this work of mine, so long as you give me proper credit

Read and outline the ``Hydrostatic Interface Level Measurement'' subsection of the ``Hydrostatic Pressure'' section of the ``Continuous Level Measurement'' chapter in your {\it Lessons In Industrial Instrumentation} textbook.  Note the page numbers where important illustrations, photographs, equations, tables, and other relevant details are found.  Prepare to thoughtfully discuss with your instructor and classmates the concepts and examples explored in this reading.

\vskip 10pt

Make special note of the ``thought experiment'' problem-solving technique applied to the solution of calibration points for liquid-liquid interface level instruments.

\underbar{file i03957}
%(END_QUESTION)





%(BEGIN_ANSWER)


%(END_ANSWER)





%(BEGIN_NOTES)

In order to apply hydrostatic pressure-sensing instrumentation to the problem of liquid-liquid interface measurement, we must ensure the only variable affecting DP is the interface level, not the total liquid level.  If the tank is vented, we may ensure the total level remains constant by the use of an overflow pipe.  We may also ensure total liquid level does not interfere with the interface measurement by using a compensating leg and keeping its tap location continuously submerged.

\vskip 10pt

A straight-forward way to calculate LRV and URV range points is to perform a pair of ``thought experiments'' for every instrument: one where the liquid-liquid interface is at the LRV point, and another where the liquid-liquid interface is at the URV point.  In each simulation, calculate the total hydrostatic pressure seen by each port on the DP instrument, then subtract those two pressures to arrive at the DP (differential) value.

\vskip 10pt

The {\it span} of differential pressure seen by a DP instrument in a liquid-liquid interface application is always equal to the vertical span multiplied by the {\it difference} in liquid specific gravities.










\vskip 20pt \vbox{\hrule \hbox{\strut \vrule{} {\bf Suggestions for Socratic discussion} \vrule} \hrule}

\begin{itemize}
\item{} Describe the ``thought experiment'' technique of calculating interface level transmitter range values demonstrated in the textbook.
\item{} Explain why the hydrostatic pressure of the light liquid above the upper process connection point ($P = \gamma_2 h_3$) is irrelevant to the calculation of LRV and URV.
\item{} If the compensating leg fill fluid is changed to one having a different density, what kind of calibration error (zero shift, span shift, etc.) will be introduced into the system?
\item{} Suppose all the open-tank process examples shown in this section of the textbook were converted to pressurized tanks.  Would the interface level transmitter have to be recalibrated?  Why or why not?
\item{} Is it possible to use a site-filled wet leg in an interface level measurement system, or must a remote seal always be used?
\end{itemize}










\vfil \eject

\noindent
{\bf Prep Quiz:}

The proper pressure range (LRV and URV) for a hydrostatic interface level instrument depends on the density(ies) of:

\begin{itemize}
\item{} The upper liquid only
\vskip 5pt 
\item{} The lower liquid only
\vskip 5pt 
\item{} Both liquids
\vskip 5pt 
\item{} Neither liquid
\vskip 5pt 
\item{} The vapor above the upper liquid
\end{itemize}




%INDEX% Reading assignment: Lessons In Industrial Instrumentation, Continuous Level Measurement (interface level measurement using hydrostatic pressure)

%(END_NOTES)


