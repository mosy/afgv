
%(BEGIN_QUESTION)
% Copyright 2010, Tony R. Kuphaldt, released under the Creative Commons Attribution License (v 1.0)
% This means you may do almost anything with this work of mine, so long as you give me proper credit

Calculate the line current for a 30 horsepower, single-phase motor operating at a line voltage of 240 VAC.  Assume 100\% efficiency and perfect power factor.

\vskip 10pt

Calculate the line current for a 30 horsepower, three-phase motor operating at a line voltage of 240 VAC.  Assume 100\% efficiency and perfect power factor.

\vskip 10pt

Calculate the line current ratio of one motor to the other, and comment on the practical difference between the two.

\vskip 20pt \vbox{\hrule \hbox{\strut \vrule{} {\bf Suggestions for Socratic discussion} \vrule} \hrule}

\begin{itemize}
\item{} Do we need to know the internal winding configuration of the motor (wye vs. delta) in order to accurately determine current?  If not, when would the winding configuration be necessary for us to know?
\item{} A three-phase motor requires three power conductors, while a single-phase motor requires two power conductors.  Do we realize any cost savings (in copper wire) by going with three-phase motors after all?  How could we quantitatively answer this question?
\end{itemize}

\underbar{file i03958}
%(END_QUESTION)





%(BEGIN_ANSWER)

Single-phase line current = 93.25 amps

\vskip 10pt

Three-phase line current = 53.84 amps

%(END_ANSWER)





%(BEGIN_NOTES)

30 HP = 22.38 kW

$${22.38 \hbox{ kW} \over 240 \hbox{ VAC}} = 93.25 \hbox{ amps}$$

$${22.38 \hbox{ kW} \over \sqrt{3} (240 \hbox{ VAC})} = 53.84 \hbox{ amps}$$

The line current of the three-phase motor is $\sqrt{3}$ times less than the line current of the single-phase motor.  This allows the use of conductors with $\sqrt{3}$ times less metal, which will result in an overall savings of metal since there are only 50\% more (1.5 times) conductors used for the three-phase system.  

%INDEX% Electronics review: AC motor horsepower calculation

%(END_NOTES)

