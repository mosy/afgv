
%(BEGIN_QUESTION)
% Copyright 2009, Tony R. Kuphaldt, released under the Creative Commons Attribution License (v 1.0)
% This means you may do almost anything with this work of mine, so long as you give me proper credit

Read and outline the ``Ultrasonic Level Measurement'' subsection of the ``Echo'' section of the ``Continuous Level Measurement'' chapter in your {\it Lessons In Industrial Instrumentation} textbook.  Note the page numbers where important illustrations, photographs, equations, tables, and other relevant details are found.  Prepare to thoughtfully discuss with your instructor and classmates the concepts and examples explored in this reading.

\underbar{file i03960}
%(END_QUESTION)





%(BEGIN_ANSWER)


%(END_ANSWER)





%(BEGIN_NOTES)

Ultrasonic level instruments work on the principle of firing a pulse of sound energy down at the liquid surface, then timing the echo back to the sensor.  An echo of sound waves is generated when the waves encounter a sudden change in material density (actually, a sudden change in the speed of sound through the media).  Such instruments are typically mounted above the liquid, firing the sound waves through the air space, but it is possible in some circumstances to mount the instrument at the bottom of a vessel and fire sound waves through the liquid.

\vskip 10pt

When sensing the amount of liquid height in the vessel, the measurement is called {\it fillage}.  When sensing the amount of empty space in the vessel, the measurement is called {\it ullage}.

\vskip 10pt

The audio transducer for an ultrasonic instrument is typically a piezoelectric crystal (the ``Level Element'' in the level measurement system).  This would be the ``LE'' (Level Element) in an ultrasonic level sensing system.

\vskip 10pt

In order to achieve good measurement accuracy, the speed of sound through the echo medium (the substance through which the sound wave must propagate in order to travel from the transducer to the interface and back to the transducer) must be known and constant, since changes in sound velocity will affect the echo time just as much as changes in level.

\vskip 10pt

Ultrasonic level instruments may be used to measure the level of solids and not just liquids.  A potential problem here is lateral scattering of the sound energy as well as interpretation of level given an angle of repose.




\vskip 20pt \vbox{\hrule \hbox{\strut \vrule{} {\bf Suggestions for Socratic discussion} \vrule} \hrule}

\begin{itemize}
\item{} {\bf In what ways may an ultrasonic level instrument be ``fooled'' to report a false level measurement?}
\item{} What will happen to the indication of level registered by an ultrasonic instrument if the air temperature increases?
\item{} What will happen to the indication of level registered by an ultrasonic instrument if the air temperature decreases?
\item{} What will happen to the indication of level registered by an ultrasonic instrument if the liquid temperature increases?
\item{} What will happen to the indication of level registered by an ultrasonic instrument if the liquid temperature decreases?
\item{} Will the angle of repose generate a positive or a negative level measurement error when measuring the level of a solid substance?
\end{itemize}



\vfil \eject

\noindent
{\bf Prep Quiz:}

Sound waves reflect strongly whenever they encounter a sudden change in which of the following material properties?

\begin{itemize}
\item{} Resistance
\vskip 5pt 
\item{} Permittivity
\vskip 5pt 
\item{} Opacity
\vskip 5pt 
\item{} Color
\vskip 5pt 
\item{} Density
\vskip 5pt 
\item{} Turbidity
\end{itemize}



%INDEX% Reading assignment: Lessons In Industrial Instrumentation, Continuous Level Measurement (ultrasonic level measurement)

%(END_NOTES)


