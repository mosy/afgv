
%(BEGIN_QUESTION)
% Copyright 2009, Tony R. Kuphaldt, released under the Creative Commons Attribution License (v 1.0)
% This means you may do almost anything with this work of mine, so long as you give me proper credit

Read and outline the ``Radar Level Measurement'' subsection of the ``Echo'' section of the ``Continuous Level Measurement'' chapter in your {\it Lessons In Industrial Instrumentation} textbook.  Note the page numbers where important illustrations, photographs, equations, tables, and other relevant details are found.  Prepare to thoughtfully discuss with your instructor and classmates the concepts and examples explored in this reading.

\underbar{file i03961}
%(END_QUESTION)





%(BEGIN_ANSWER)


%(END_ANSWER)





%(BEGIN_NOTES)

Radar level instruments work in much the same way as ultrasonic level instruments: by measuring the time-of-flight for radio waves to and from the surface whose level is being measured.  A radio wave reflection is generated whenever the wave encounters a sudden change in material permittivity (actually, a sudden change in the speed of light through the media).  Radar instruments are {\it always} mounted above the liquid level, firing their radio energy down at the liquid through a gas or vapor space.

\vskip 10pt

Radar instruments may use a waveguide (probe) to guide the radio energy to and from the liquid, or a simple antenna to transmit and receive radio energy through open space.  Waveguide probe design varies, with single-rod types being more resistant to fouling and coaxial types excelling at maintaining signal strength.

\vskip 10pt

The speed of light is strongly dependent upon dielectric permittivity.  Therefore, level measurement accuracy for a radar-based level instrument also depends on this factor:

$$v = {c \over \sqrt{\epsilon_r}}$$

Permittivity of the gas phase above the liquid is subject to change with changes in gas pressure, gas temperature, and/or gas composition: 

$$\epsilon_r = 1 + (\epsilon_{ref} - 1) {P T_{ref} \over P_{ref} T}$$

We must accurately know the permittivity of whatever substance the radio wave echoes through in order to accurately measure level using a radar instrument.

\vskip 10pt

The strength of the reflected signal depends on the magnitude of the permittivity difference between the two materials at the interface:

$$R = {\left({\sqrt{\epsilon_{r2}} - \sqrt{\epsilon_{r1}}}\right)^2 \over \left(\sqrt{\epsilon_{r2}} + \sqrt{\epsilon_{r1}}\right)^2}$$

For air or other gas media having relative permittivity values near 1, we may use this simpler formula:

$$R = {\left({\sqrt{\epsilon_{r}} - 1}\right)^2 \over \left(\sqrt{\epsilon_{r}} + 1 \right)^2}$$

Liquid-liquid interfaces are also measureable using (guided-wave) radar, if the two liquids' permittivities are substantially different from each other.  If not, the reflection will not be strong enough to reliably detect.  A digital (HART, Fieldbus) radar level transmitter may report multiple levels (gas-liquid, liquid-liquid interface) simultaneously.

\vskip 10pt

Gas-phase errors may be compensated by using a reference probe, sensing changes in gas permittivity in order to correct for time-of-flight echo measurements.

\vskip 10pt

Radar instruments useful for measuring solids, just like ultrasonic instruments.

\vskip 10pt

The ``echo curve'' of a radar instrument shows the received signal strength over time.  Thresholds may be set for detection of interfaces.  A {\it null zone} or {\it hold-off distance} may be set in the instrument, causing it to ignore echoes that are within a certain distance of the probe's beginning (adjacent to the transmitter).

\vskip 10pt

{\it Transition zones} near the beginning and ends of a guided radar probe yield inaccurate results, and so the LRV/URV settings should be established to avoid measurement within these zones.  The type of substance as well as the type of probe will influence the size is these ``unusable'' transition zones.





\vskip 20pt \vbox{\hrule \hbox{\strut \vrule{} {\bf Suggestions for Socratic discussion} \vrule} \hrule}

\begin{itemize}
\item{} {\bf In what ways may a radar level instrument be ``fooled'' to report a false level measurement?}
\item{} After the 2005 Texas City BP refinery explosion, a popular move in the industry has been to replace legacy buoyancy-type level instruments with guided-wave radar level instruments.  Explain why this is being done, based on what you know of that 2005 accident.  Do you think this is a fool-proof strategy, or might a GWR instrument be misled and contribute to another accident?
\item{} What factors should I be aware of when selecting a probe for a GWR?
\item{} Where will the gas-phase effect be more prominent: in a vented vessel or in a pressurized vessel?
\item{} If gas pressure increases, how will this skew a radar instrument's level measurement, all other factors remaining unchanged?
\item{} If gas pressure decreases, how will this skew a radar instrument's level measurement, all other factors remaining unchanged?
\item{} If gas temperature increases, how will this skew a radar instrument's level measurement, all other factors remaining unchanged?
\item{} If gas temperature decreases, how will this skew a radar instrument's level measurement, all other factors remaining unchanged?
\item{} If gas composition changes in such a way that permittivity increases, how will this skew a radar instrument's level measurement, all other factors remaining unchanged?
\item{} If gas composition changes in such a way that permittivity decreases, how will this skew a radar instrument's level measurement, all other factors remaining unchanged?
\end{itemize}





\vfil \eject

\noindent
{\bf Prep Quiz:}

Radar waves reflect strongly whenever they encounter a sudden change in which of the following material properties?

\begin{itemize}
\item{} Resistance
\vskip 5pt 
\item{} Permittivity
\vskip 5pt 
\item{} Opacity
\vskip 5pt 
\item{} Color
\vskip 5pt 
\item{} Hardness
\vskip 5pt 
\item{} Turbidity
\end{itemize}

%INDEX% Reading assignment: Lessons In Industrial Instrumentation, Continuous Level Measurement (radar level measurement)

%(END_NOTES)


