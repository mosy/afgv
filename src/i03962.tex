
%(BEGIN_QUESTION)
% Copyright 2009, Tony R. Kuphaldt, released under the Creative Commons Attribution License (v 1.0)
% This means you may do almost anything with this work of mine, so long as you give me proper credit

Read and outline the ``Magnetostrictive Level Measurement'' subsection of the ``Echo'' section of the ``Continuous Level Measurement'' chapter in your {\it Lessons In Industrial Instrumentation} textbook.  Note the page numbers where important illustrations, photographs, equations, tables, and other relevant details are found.  Prepare to thoughtfully discuss with your instructor and classmates the concepts and examples explored in this reading.

\underbar{file i03962}
%(END_QUESTION)





%(BEGIN_ANSWER)


%(END_ANSWER)





%(BEGIN_NOTES)

A floating magnet surrounding a metal rod (called a ``waveguide'') generates a mechanical twist in the rod at the point of the float's height whenever an electrical interrogation pulse current is sent through the rod.  The time-of-flight of this torsional wave is then used to measure the position of the float along the rod's length.

\vskip 10pt

The speed of sound for this torsional stress wave is quite constant, because the medium it travels through is a metal rod, not a process fluid.  This speed is barely affected by temperature changes, and unchanged by pressure for all practical purposes.  Compositional changes in the process fluid are likewise irrelevant.

\vskip 10pt

Liquid-liquid interfaces are also possible to measure, if the float is made to have just the right amount of buoyancy.  We may simultaneously measure gas-liquid and liquid-liquid interface levels using multiple floats on the same waveguide.

\vskip 10pt

Anything that could cause the float to bind on the waveguide rod will, of course, affect level measurement.







\vskip 20pt \vbox{\hrule \hbox{\strut \vrule{} {\bf Suggestions for Socratic discussion} \vrule} \hrule}

\begin{itemize}
\item{} {\bf In what ways may a magnetostriction level instrument be ``fooled'' to report a false level measurement?}
\item{} If gas pressure increases, how will this skew a magnetostriction instrument's level measurement, all other factors remaining unchanged?
\item{} If gas pressure decreases, how will this skew a magnetostriction instrument's level measurement, all other factors remaining unchanged?
\item{} If temperature increases, how will this skew a magnetostriction instrument's level measurement, all other factors remaining unchanged?
\item{} If temperature decreases, how will this skew a magnetostriction instrument's level measurement, all other factors remaining unchanged?
\item{} If gas composition changes, how will this skew a magnetostriction instrument's level measurement, all other factors remaining unchanged?
\end{itemize}

%INDEX% Reading assignment: Lessons In Industrial Instrumentation, Continuous Level Measurement (magnetostrictive level measurement)

%(END_NOTES)


