
%(BEGIN_QUESTION)
% Copyright 2009, Tony R. Kuphaldt, released under the Creative Commons Attribution License (v 1.0)
% This means you may do almost anything with this work of mine, so long as you give me proper credit

Read and outline the ``Weight'' section of the ``Continuous Level Measurement'' chapter in your {\it Lessons In Industrial Instrumentation} textbook.  Note the page numbers where important illustrations, photographs, equations, tables, and other relevant details are found.  Prepare to thoughtfully discuss with your instructor and classmates the concepts and examples explored in this reading.


\underbar{file i03963}
%(END_QUESTION)





%(BEGIN_ANSWER)


%(END_ANSWER)





%(BEGIN_NOTES)

Total weight $-$ vessel tare weight = weight of product inside vessel.  Weight-based systems naturally give linear, mass measurement.  If the material's density is constant and the vessel's cross-sectional area is constant, level will be proportional to weight.

\vskip 10pt

Load cells are the typical sensing element, with multiple load cells connected together in a ``summing'' circuit to yield total vessel weight.

\vskip 10pt

In order for this type of measurement system to work, there must be no outside forces acting on the vessel.  Pipes must be flexibly coupled to the vessel, and there must be no vibration.  Electrical ground faults can also cause problems, as current passing through a load cell may cause a measurable voltage drop to interfere with the weight measurement.

\vskip 10pt

{\it Hydraulic} load cells use a piston and fluid to translate weight into a fluid pressure to be read by a transmitter.









\vskip 20pt \vbox{\hrule \hbox{\strut \vrule{} {\bf Suggestions for Socratic discussion} \vrule} \hrule}

\begin{itemize}
\item{} {\bf In what ways may a weight-based level instrument be ``fooled'' to report a false level measurement?}
\item{} Identify some of the advantages weight-based level measurement enjoys over other technologies such as radar or hydrostatic pressure.
\item{} Suppose a sealed vessel containing a liquid becomes heated, such that the liquid expands and becomes less dense.  What will a weight-based instrument register as the liquid heats up -- will its signal increase, decrease, or remain the same?
\item{} Suppose a denser-than-normal material is added to a vessel equipped with a weight-based level measurement system.  Will the level transmitter ``think'' the vessel's level is greater than it actually is, less than it actually is, or will it register with good accuracy?
\item{} Suppose the piston diameter in a hydraulic load cell is upgraded in size.  What effect will this change have on the fluid pressure generated, for any given level of material in the vessel?  Will this change constitute a {\it zero} shift, a {\it span} shift, or {\it both}?
\end{itemize}

%INDEX% Reading assignment: Lessons In Industrial Instrumentation, Continuous Level Measurement (weight)

%(END_NOTES)


