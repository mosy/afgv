
%(BEGIN_QUESTION)
% Copyright 2009, Tony R. Kuphaldt, released under the Creative Commons Attribution License (v 1.0)
% This means you may do almost anything with this work of mine, so long as you give me proper credit

Read and outline the ``Radiation'' section of the ``Continuous Level Measurement'' chapter in your {\it Lessons In Industrial Instrumentation} textbook.  Note the page numbers where important illustrations, photographs, equations, tables, and other relevant details are found.  Prepare to thoughtfully discuss with your instructor and classmates the concepts and examples explored in this reading.

\underbar{file i03965}
%(END_QUESTION)





%(BEGIN_ANSWER)


%(END_ANSWER)





%(BEGIN_NOTES)

Nuclear radiation may penetrate a vessel wall, but be attenuated or scattered by the material within the vessel.

\vskip 10pt

Types of radiation typically encountered:

\begin{itemize}
\item{} Alpha particles (helium nuclei, low penetrating power)
\item{} Beta particles (electrons, low penetrating power)
\item{} Gamma rays (electromagnetic waves, great penetrating power -- used in attenuating designs)
\item{} Neutron particles (great penetrating power, attenuated/scattered by hydrogen -- used in backscatter designs)
\end{itemize}

Nuclear {\it sources} are pill-shaped double-wall stainless steel capsules containing a radioactive substance such as Cesium-137 or Cobalt-60.  The half-life of Cesium-137 is about 30 years, whereas the half-life of Cobalt is approximately 5 years (shorter life, but more radioactivity!).

\vskip 10pt

Sources are typically contained in lead-lined boxes equipped with {\it shutters} to turn them on and off.  This shutter may be actuated to perform simple tests of the device.

\vskip 10pt

Radiation detectors typically use Geiger-Muller tubes.  A wire suspended within a metal tube is energized with high voltage.  Inert gas within the tube becomes ionized whenever alpha, beta, or gamma radiation passes through.  This momentary ionization allows a pulse of current between the wire and the tube wall, which may be detected and counted by an electronic circuit.

\vskip 10pt

Neutron radiation is non-ionizing, and so a plain G-M tube will not detect it.  Instead, the detector tube must contain some substance known to react with neutrons to create ionization.  Fission chambers containing Uranium-235 work for this purpose, as do tubes filled with boron trifluoride gas.  In a fission chamber, a neutron captured by a U-235 nucleus causes that nucleus to split and release considerable ionizing radiation.  In a BT3 tube, a neutron captured by a boron nucleus causes that nucleus to transmutate into lithium and multiple ionizing particles.

\vskip 10pt

Sources of measurement error include changes in process density, wall coating, detector drift, and source decay.  NRC licensing is also required to operate such devices on a plant site.  We basically don't use these devices unless we absolutely have to (e.g. for really challenging level measurement applications).









\vskip 20pt \vbox{\hrule \hbox{\strut \vrule{} {\bf Suggestions for Socratic discussion} \vrule} \hrule}

\begin{itemize}
\item{} {\bf In what ways may a radiation-based level instrument be ``fooled'' to report a false level measurement?}
\item{} An important safety policy at many industrial facilities is something called {\it stop-work authority}, which means any employee has the right to stop work they question as unsafe.  Describe a scenario involving radiation-based level measurement where one might invoke stop-work authority.
\item{} As a radioactive source decays (gets weaker), will this introduce a {\it zero} shift, a {\it span} shift, or {\it both} in the level measuring instrument?
\item{} Describe a {\it lock-out/tag-out} procedure suitable for a radioactive source, including means to verify the source radiation has been blocked.
\item{} Explain how an ionization-tube style of radiation detector tube works to detect ionizing radiation such as alpha particles, beta particles, and gamma rays.
\item{} Explain how an ionization-tube style of radiation detector tube works to detect non-ionizing radiation such as neutron particles.
\item{} If the shutter is closed on a source in a through-vessel application, will it simulate a full condition or an empty condition?
\item{} If the shutter is closed on a source in a backscatter application, will it simulate a full condition or an empty condition?
\item{} If the gas leaks out of a radiation detector tube in a through-vessel application, will it simulate a full condition or an empty condition?
\item{} If the gas leaks out of a radiation detector tube in a backscatter application, will it simulate a full condition or an empty condition?
\item{} If the process vessel begins to coat in a through-vessel application, how will this affect the instrument's level measurement?
\item{} If the process vessel begins to coat in a backscatter application, how will this affect the instrument's level measurement?
\end{itemize}









\vfil \eject

\noindent
{\bf Prep Quiz:}

Suppose you are tasked with performing some routine maintenance work on the detector array for a nuclear (radiation) level instrument.  Identify the safest way to minimize exposure to radiation from the Cesium-137 source of this instrument while you are performing the work:

\begin{itemize}
\item{} Wear a radiation-monitoring badge on your body
\vskip 5pt 
\item{} Point the aperture in a direction away from you
\vskip 5pt 
\item{} Close the hand-operated shutter on the source
\vskip 5pt 
\item{} Aim a second Cesium-137 source in the opposite direction
\vskip 5pt 
\item{} De-energize the electrical power to the source
\vskip 5pt 
\item{} Wear garlic cloves around your neck
\end{itemize}

%INDEX% Reading assignment: Lessons In Industrial Instrumentation, Continuous Level Measurement (radiation)

%(END_NOTES)


