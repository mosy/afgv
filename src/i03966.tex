
%(BEGIN_QUESTION)
% Copyright 2015, Tony R. Kuphaldt, released under the Creative Commons Attribution License (v 1.0)
% This means you may do almost anything with this work of mine, so long as you give me proper credit

Read and outline the ``Level Switches'' section of the ``Discrete Process Measurement'' chapter in your {\it Lessons In Industrial Instrumentation} textbook.  Note the page numbers where important illustrations, photographs, equations, tables, and other relevant details are found.  Prepare to thoughtfully discuss with your instructor and classmates the concepts and examples explored in this reading.

\underbar{file i03966}
%(END_QUESTION)





%(BEGIN_ANSWER)


%(END_ANSWER)





%(BEGIN_NOTES)

The ``normal'' or resting status of a level switch is when the level is {\it low}.  Different types of level switch include:

\begin{itemize}
\item{} Float (a floating object rises and falls with liquid level to actuate the switch)
\itemitem{} ``Magnetrol'' level switches detect float position using a magnet and mercury tilt switch
\item{} Tuning fork (vibrating forks vibrate slower when they contact solid or liquid mass)
\item{} Paddle wheel (a slowly-rotating paddle stops when it encounters solid mass)
\itemitem{} ``Bindicator'' level switches use a slow-turning motor and and paddle
\item{} Ultrasonic (sound waves dampened by the presence of liquid or solid)
\item{} Capacitive (presence of level changes capacitance detected by probe)
\item{} Conductive (using the liquid or solid to complete a circuit)
\itemitem{} B/W Controls inductive units close a relay based on magnetic flux exclusion in a secondary circuit
\itemitem{} Primary coil energized by 120 VAC line power
\itemitem{} Secondary coil connected to level probes: when conductive material bridges the probes it completes the secondary circuit which bucks the magnetic field in the core, causing more of that field to seek an easier path through a movable armature, actuating a switch mechanism.
\end{itemize}





\vskip 20pt \vbox{\hrule \hbox{\strut \vrule{} {\bf Suggestions for Socratic discussion} \vrule} \hrule}

\begin{itemize}
\item{} {\bf For each of these level switch types, identify ways in which it might fail or otherwise be ``fooled'' into falsely reporting a certain level condition.}
\item{} Define ``normal'' for a level switch, as the term is used for NO and NC switch contacts.
\end{itemize}

%INDEX% Reading assignment: Lessons In Industrial Instrumentation, Discrete Process Measurement (level switches)

%(END_NOTES)


