
%(BEGIN_QUESTION)
% Copyright 2015, Tony R. Kuphaldt, released under the Creative Commons Attribution License (v 1.0)
% This means you may do almost anything with this work of mine, so long as you give me proper credit

Read selected sections of the National Transportation Safety Board's report (NTSB/PAR-04/02, PB2004-916502 Notation 7666) of the 2003 storage tank explosion and fire in Glenpool, Oklahoma, and answer the following questions.  The ``Accident Narrative'' section of the report (pages 1-6) describes the sequence of events leading up to the explosion.  You will need to figure out where in the document the other details are found.

\vskip 10pt

Answer the following questions:

\begin{itemize}
\item{} What did the NTSB identify as being the most likely cause of the accident, and how could it have been avoided?
\item{} Explain what a {\it floating roof} is, and the purpose it serves in a fuel storage tank.
\item{} Describe what a {\it datum plate} is, and how the stored fuel quantity may be determined by manual tape measurement of liquid level and reference to a {\it strapping table}.
\item{} At what volume and height (level) values did the original strapping table give for liquid contact with the floating roof, and for the point at which the roof would actually float?  How does this compare with the values determines for liquid contact by investigators after the accident?
\item{} Describe what a {\it bonding system} is inside a floating-roof fuel storage tank, and explain the purpose of this system.
\item{} Explain what the filling rate of a floating-roof fuel storage tank has to do with safety, especially at a point when there is not enough fuel in the tank to float the roof.
\end{itemize}


\underbar{file i03968}
%(END_QUESTION)





%(BEGIN_ANSWER)

Hint: the {\it table of contents} is always a good place to begin when searching a technical document for specific information, because it serves as an outline of the document.  The table of contents will help you quickly identify which portion(s) of the document are relevant to the information you are seeking and which are not.

%(END_ANSWER)





%(BEGIN_NOTES)

Tank 11 at ConocoPhillips' Glenpool South Tank Farm facility exploded as it was being filled with diesel fuel sent from the Explorer Pipeline company.  Tank 11 held gasoline earlier that day, and probably had about 55 barrels of gasoline left in the sump of tank 11 prior to filling with diesel.

\vskip 10pt

(From page 9) -- Tank 11 was 109 feet in diameter and 48 feet high, with a conical fixed roof and an internal ``floating'' roof designed to minimize emissions and evaporative losses.  

\vskip 10pt

(From page 10) -- Nominal storage capacity was 80,000 barrels.  Maximum fill rate was 32,000 barrels per hour, and maximum drain rate was 20,000 barrels per hour.  The floating roof rested on legs so that it did not go all the way to the tank's bottom.  10 inch steel pontoons provided buoyancy.  A {\it datum plate} mounted on the south wall of the tank provided a point of reference for manual level measurements.

\vskip 10pt

(From page 12) -- bonding wires connected the floating and fixed roofs to maintain equal electrical potential between them.

\vskip 10pt

Original strapping table values (from page 12):

% No blank lines allowed between lines of an \halign structure!
% I use comments (%) instead, so that TeX doesn't choke.

$$\vbox{\offinterlineskip
\halign{\strut
\vrule \quad\hfil # \ \hfil & 
\vrule \quad\hfil # \ \hfil & 
\vrule \quad\hfil # \ \hfil \vrule \cr
\noalign{\hrule}
%
% First row
{\bf Liquid volume} & {\bf Liquid height} (feet and inches) & {\bf Tank condition} \cr
%
\noalign{\hrule}
%
% Another row
6390 barrels & 3'-0" & {\it Liquid contact with roof} \cr
%
\noalign{\hrule}
%
% Another row
7180 barrels & 3'-6" & {\it Roof floating} \cr
%
\noalign{\hrule}
} % End of \halign 
}$$ % End of \vbox

\vskip 10pt

(From page 13) -- SCADA system equipped with two alarms alerting operators when the floating roof was about to come to rest on the legs because it could no longer float.  These alarms were based on the strapping table values, based on measurements of liquid volume in the tank.  No actual level-sensing device in the tank provided this information.

\vskip 30pt

Revised (post-investigation) strapping table values (from page 13):

% No blank lines allowed between lines of an \halign structure!
% I use comments (%) instead, so that TeX doesn't choke.

$$\vbox{\offinterlineskip
\halign{\strut
\vrule \quad\hfil # \ \hfil & 
\vrule \quad\hfil # \ \hfil & 
\vrule \quad\hfil # \ \hfil \vrule \cr
\noalign{\hrule}
%
% First row
{\bf Liquid volume} & {\bf Liquid height} (feet and inches) & {\bf Tank condition} \cr
%
\noalign{\hrule}
%
% Another row
7498 to 7632 barrels & 3'-8" to 3'-9" & {\it Liquid contact with roof} \cr
%
\noalign{\hrule}
} % End of \halign 
}$$ % End of \vbox

The trend graph shown on page 21 shows how the explosion occurred prior to the liquid level reaching the roof, proving a vapor gap existed which could have sustained an explosion-igniting spark from electrostatic accumulation.

\vskip 10pt

(From page 23) -- API standard RP 2003 states that the rate of fill and discharge into storage vessels should be limited to a maximum pipe velocity of 1 meter per second (3 feet per second) so long as the floating roof is not buoyant.  The purpose of this velocity limit is to minimize the generation of a static electric charge.

\vskip 10pt

(From page 24) -- ConocoPhillips own guide identified diesel fuel as a low-vapor-pressure product with a high risk of static charge accumulation.  For high-vapor-pressure products such as gasoline, the ConocoPhillips guide did not state any maximum fill/empty flow rates because it was assumed the vapor space would be too rich to support combustion.





\vfil \eject

\noindent
{\bf Prep Quiz:}

What did operations personnel do wrong to blow up the diesel storage tank as described in the NTSB Safety Report documenting the 2003 explosion and fire in Glenpool, Oklahoma?

\begin{itemize}
\item{} They failed to purge the tank with nitrogen gas before filling it 
\vskip 5pt 
\item{} They filled up the tank with flammable liquid at too fast a rate
\vskip 5pt 
\item{} They did not lock out a defective electric motor near the tank
\vskip 5pt 
\item{} A maintenance work light was left on inside the vapor space of the tank
\vskip 5pt 
\item{} The incident was the result of a lightning strike -- no one's fault
\vskip 5pt 
\item{} They failed to patch a small leak at the base of the tank
\end{itemize}




\vfil \eject

\noindent
{\bf Prep Quiz:}

Based on the NTSB Safety Report documenting the 2003 explosion and fire in Glenpool, Oklahoma, what is a {\it strapping table}?

\begin{itemize}
\item{} A platform where hoisting straps are attached to heavy objects before lifting
\vskip 5pt 
\item{} A sketch showing the number and placement of anchor straps holding a tank in place
\vskip 5pt 
\item{} A schematic documenting the electrical grounding straps bonded to the floating roof
\vskip 5pt 
\item{} A log chronicling the dates and times when a storage tank is filled with liquid
\vskip 5pt 
\item{} A list of measurements relating liquid level versus stored liquid volume in a tank
\vskip 5pt 
\item{} A log chronicling the dates and times when a storage tank is emptied of liquid
\end{itemize}


%INDEX% Reading assignment: NTSB report on the 2003 Conoco-Phillips Glenpool Oklahoma storage tank explosion and fire

%(END_NOTES)


