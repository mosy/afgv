
%(BEGIN_QUESTION)
% Copyright 2010, Tony R. Kuphaldt, released under the Creative Commons Attribution License (v 1.0)
% This means you may do almost anything with this work of mine, so long as you give me proper credit

Examining a squirrel-cage motor rotor, you see the aluminum bars of the ``squirrel cage'' assembly embedded in what appears to be a mass of iron constituting the bulk of the rotor's mass.  An electrician explains to you that the iron is necessary in the rotor for the stator's magnetic field to act upon.  ``Without the iron there,'' says the electrician, ``the rotor wouldn't spin.''

\vskip 10pt

Explain what is incorrect about the electrician's reasoning.

\vskip 20pt \vbox{\hrule \hbox{\strut \vrule{} {\bf Suggestions for Socratic discussion} \vrule} \hrule}

\begin{itemize}
\item{} This is a very common misconception among students and working technicians alike.  Explain why so many people tend to get this concept wrong.
\item{} One way to disprove an assertion is by demonstrating that the assertion leads to one or more logical absurdities.  This technique is called {\it reductio ad absurdum} (``reducing to an absurdity'').  Apply this technique to the disproof of the assertion that iron is necessary in the rotor of an induction motor in order for that rotor to generate a torque.
\item{} Describe the direction(s) that induced current(s) take in the bars of a squirrel-cage rotor as that rotor experiences a rotating magnetic field from the stator windings of an induction motor.  Refer to a photograph or picture of a squirrel-cage rotor, or point to the rotor bars of a real rotor, as you trace the directions of current with your fingers.
\end{itemize}

\underbar{file i03970}
%(END_QUESTION)





%(BEGIN_ANSWER)

The rotating magnetic field from a three-phase motor stator assembly will exert a torque on {\it any} conductive object.  The magnetic properties of the object within the field are largely irrelevant.  The only reason iron is placed in the rotor is to eliminate what would otherwise be a huge air gap between the stator poles, thereby strengthening the stator's magnetic field for heightened effect.  This stronger magnetic field then acts upon the aluminum ``squirrel cage'' bars to produce more torque than would otherwise be possible.

%(END_ANSWER)





%(BEGIN_NOTES)


%INDEX% Final Control Elements, motor: ``squirrel cage''

%(END_NOTES)

