
%(BEGIN_QUESTION)
% Copyright 2009, Tony R. Kuphaldt, released under the Creative Commons Attribution License (v 1.0)
% This means you may do almost anything with this work of mine, so long as you give me proper credit

Skim the ``Continuous Temperature Measurement'' chapter in your {\it Lessons In Industrial Instrumentation} textbook to specifically answer these questions:

\vskip 10pt

Explain how a {\it bi-metal strip} works as a temperature-sensing device.

\vskip 10pt

Explain how a {\it filled-bulb} system works to sense temperature.

\vskip 20pt \vbox{\hrule \hbox{\strut \vrule{} {\bf Suggestions for Socratic discussion} \vrule} \hrule}

\begin{itemize}
\item{} Identify different strategies for ``skimming'' a text, as opposed to reading that text closely.  Why do you suppose the ability to quickly scan a text is important in this career?
\item{} A common mechanic's ``trick'' for slipping a metal bearing or ring over a metal shaft where the two parts would ordinarily make a very tight fit, is to either heat the ring and/or freeze the shaft prior to assembly.  Explain why this ``trick'' works in light of what you know about coefficients of expansion.
\item{} A ring made of copper tightly fitted over an iron shaft may be more easily removed from the shaft if both are first heated.  However, a copper ring tightly fit over an aluminum shaft will not loosen up when both the ring and shaft are heated.  Referencing the coefficients of thermal expansion for these metals, explain why.  Then, explain how a copper ring fit tightly over an aluminum shaft {\it could} be loosened, if not by heating.
\end{itemize}

\underbar{file i03971}
%(END_QUESTION)





%(BEGIN_ANSWER)


%(END_ANSWER)





%(BEGIN_NOTES)

Bimetallic strips are formed from two different metals, expanding at different rates with increases in temperature.  The differential expansion causes the strip to {\it bend}, thereby physically indicating the temperature of the strip.

Twisted bi-metallic strips will turn when heated, and are used in this manner to drive pointer needles in mechanical thermometers.

\vskip 10pt

Filled systems use the expansion of a fluid in an enclosed volume to indicate increases in temperature.  This expanding fluid may then drive a bellows or other pressure-sensing element to move an instrument pointer.


%INDEX% Reading assignment: Lessons In Industrial Instrumentation, Continuous Temperature Measurement (bimetallic)

%(END_NOTES)


