
%(BEGIN_QUESTION)
% Copyright 2009, Tony R. Kuphaldt, released under the Creative Commons Attribution License (v 1.0)
% This means you may do almost anything with this work of mine, so long as you give me proper credit

Skim the ``Continuous Temperature Measurement'' chapter in your {\it Lessons In Industrial Instrumentation} textbook to specifically answer these questions:

\vskip 10pt

Both {\it thermistors} and {\it RTDs} are electrically-based temperature sensors.  Identify the electrical property of each that changes with temperature.

\vskip 10pt

Explain how you would use common electrical test equipment (such as that found in any electronics workshop or laboratory) to check whether or not a thermistor or an RTD is functional.

\vskip 20pt

{\it Thermocouples} are also electrically-based temperature sensors, but different from RTDs.  Identify the electrical property of a thermocouple that changes with temperature.

\vskip 10pt

Explain how you would use common electrical test equipment (such as that found in any electronics workshop or laboratory) to check whether or not a thermocouple is functional.


\vskip 20pt \vbox{\hrule \hbox{\strut \vrule{} {\bf Suggestions for Socratic discussion} \vrule} \hrule}

\begin{itemize}
\item{} Identify different strategies for ``skimming'' a text, as opposed to reading that text closely.  Why do you suppose the ability to quickly scan a text is important in this career?
\end{itemize}

\underbar{file i03973}
%(END_QUESTION)





%(BEGIN_ANSWER)


%(END_ANSWER)





%(BEGIN_NOTES)

RTDs and thermistors change electrical resistance with temperature.  Thermocouples generate a voltage with temperature.

\vskip 10pt

To test an RTD, measure its resistance with an ohmmeter as temperature changes.

\vskip 10pt

To test a thermocouple, measure its voltage with a voltmeter as temperature changes.


%INDEX% Reading assignment: Lessons In Industrial Instrumentation, Continuous Temperature Measurement (RTDs and thermistors)

%(END_NOTES)


