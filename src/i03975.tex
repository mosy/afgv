
%(BEGIN_QUESTION)
% Copyright 2009, Tony R. Kuphaldt, released under the Creative Commons Attribution License (v 1.0)
% This means you may do almost anything with this work of mine, so long as you give me proper credit

Skim the ``Continuous Temperature Measurement'' chapter in your {\it Lessons In Industrial Instrumentation} textbook to specifically answer these questions:

\vskip 10pt

Explain how the temperature of an object may be sensed {\it optically} (non-contact).

\vskip 10pt

Identify some practical applications of non-contact temperature measurement, as well as some disadvantages (compared to direct-contact methods of temperature measurement).


\vskip 20pt \vbox{\hrule \hbox{\strut \vrule{} {\bf Suggestions for Socratic discussion} \vrule} \hrule}

\begin{itemize}
\item{} Identify different strategies for ``skimming'' a text, as opposed to reading that text closely.  Why do you suppose the ability to quickly scan a text is important in this career?
\end{itemize}

\underbar{file i03975}
%(END_QUESTION)





%(BEGIN_ANSWER)


%(END_ANSWER)





%(BEGIN_NOTES)

Objects warmer than absolute zero emit thermal energy in the form of light (electromagnetic radiation).  Thus, we may infer the temperature of an object by analyzing the light it emits.

\vskip 10pt

The prime advantage of optical temperature measurement is that we do not need to make contact with the object being measured.  Poor accuracy is the primary disadvantage of optical temperature measurement.

\vskip 10pt

Practical applications include temperature measurement of energized electrical power components, where direct contact cannot be made without risk of electric shock or arc blast.

%INDEX% Reading assignment: Lessons In Industrial Instrumentation, Continuous Temperature Measurement (non-contact pyrometers)

%(END_NOTES)


