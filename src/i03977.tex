
%(BEGIN_QUESTION)
% Copyright 2009, Tony R. Kuphaldt, released under the Creative Commons Attribution License (v 1.0)
% This means you may do almost anything with this work of mine, so long as you give me proper credit

Read and outline the ``Heat Versus Temperature'', ``Temperature'', and ``Heat'' subsections of the ``Elementary Thermodynamics'' section of the ``Physics'' chapter in your {\it Lessons In Industrial Instrumentation} textbook.  Note the page numbers where important illustrations, photographs, equations, tables, and other relevant details are found.  Prepare to thoughtfully discuss with your instructor and classmates the concepts and examples explored in this reading.

\underbar{file i03977}
%(END_QUESTION)





%(BEGIN_ANSWER)


%(END_ANSWER)





%(BEGIN_NOTES)

Heat $\neq$ Temperature!

\vskip 10pt

Temperature is the motion of atoms or molecules.  Heat is a transfer of thermal energy.  Heat can only move from a higher-temperature object to a lower-temperature object, never vice-versa.  However, heat transfer can take place with no appreciable rise in temperature (e.g. phase changes and certain chemical reactions).

\vskip 10pt

The reason why heat naturally flows from areas of high temperature to areas of low temperature is due to inter-molecular collisions.  High-temperature objects (i.e. the molecules are moving fast) transfer energy to low-temperature objects (i.e. the molecules are moving slow) because the higher-temperature (high-velocity) molecules lose energy when they collide with lower-temperature (low-velocity) molecules.  Those slower molecules in turn pick up energy from the faster molecules via the same collisions.  Therefore, after all the heat has been transferred, all the molecules will be moving at the same velocity (i.e. all at the same temperature).

\vskip 10pt

For a monatomic gas, temperature is related to average molecular velocity by the following formula:

$${1 \over 2}m\overline{v^2} = {3 \over 2}kT$$

\noindent
Where,

$m$ = Mass of each molecule

$v$ = Velocity of a molecule in the sample

$\overline{v}$ = Average (``mean'') velocity of all molecules in the sample

$\overline{v^2}$ = Mean-squared molecular velocities in the sample

$k$ = Boltzmann's constant (1.38 $\times$ 10$^{-23}$ J / K)

$T$ = Absolute temperature (Kelvin)

\vskip 10pt

The temperature at which all molecular motion ceases is called {\it absolute zero}.  This is equal to -273.15 degrees Celsius, or -459.67 degrees Fahrenheit.  The absolute temperature scales of Kelvin and Rankine were developed to reflect this physical limit, with zero Kelvin or zero degrees Rankine being absolute zero.  The Kelvin scale is merely the Celsius scale zero-shifted by 273.15 degrees.  The Rankine scale is merely the Fahrenheit scale zero-shifted by 459.67 degrees.

Differences of temperature in the Celsius scale will be the same increment in the Kelvin scale.  Likewise, differences of temperature in the Fahrenheit scale will be the same increment in the Rankine scale.

The absolute temperature scales of Kelvin and Rankine are needed for certain thermal calculations.

\vskip 10pt

Heat transfer is literally just the transfer of kinetic energy from fast-moving molecules to slower-moving molecules via collision.  Like a fast-moving ball colliding with a slow-moving ball, the fast-moving ball loses velocity (cools down) while the slow-moving ball gains velocity (heats up).

Heat may be measured in {\it calories} (the amount of energy needed to raise the temperature of 1 gram of water by 1 degree Celsius), or in {\it British Thermal Units} (the amount of energy needed to raise the temperature of 1 pound of water by 1 degree Fahrenheit).  Being units of energy transfer, calories and BTUs may be expressed equivalently in other units of energy measurement: 1 calorie = 4.186 joules, and 1 BTU = 778.2 ft-lb.

Metabolism of food is a process of very slow combustion, where chemical energy in food is converted into energy that an animal's body can use.  For measuring the energy content in food, the unit of the {\it dietary calorie} was invented, which is equal to 1000 regular calories (1 C = 1000 c).





\vskip 20pt \vbox{\hrule \hbox{\strut \vrule{} {\bf Suggestions for Socratic discussion} \vrule} \hrule}

\begin{itemize}
\item{} Explain the distinction between {\it heat} and {\it temperature}, giving a practical example illustrating how the two concepts are not identical.
\item{} Explain what happens when a pan of water is set on top of a hot stove, in thermodynamic terms.
\item{} Explain why heat transfer always happens as the transfer of energy from the hotter object to the colder object, never the vice-versa.
\item{} Explain why absolute zero temperature is impossible to obtain by standard methods of heat transfer.
\end{itemize}






\vfil \eject

\noindent
{\bf Prep Quiz:}

{\it Heat} is most accurately defined as:

\begin{itemize}
\item{} The transfer of thermal energy
\vskip 5pt 
\item{} An object's temperature in Kelvin
\vskip 5pt 
\item{} The velocity of molecular motion
\vskip 5pt 
\item{} The total absence of all molecular motion
\vskip 5pt 
\item{} An object's temperature in degrees Rankine
\vskip 5pt 
\item{} The name of a 1970's funk band
\end{itemize}







\vfil \eject

\noindent
{\bf Prep Quiz:}

{\it Absolute Zero} is the temperature at which:

\begin{itemize}
\item{} The Celsius and Fahrenheit scales are equal
\vskip 5pt 
\item{} All molecular motion in a sample ceases
\vskip 5pt 
\item{} Some substances become superconductors
\vskip 5pt 
\item{} Water turns from a liquid into a solid
\vskip 5pt 
\item{} Propane turns from a gas into a liquid 
\vskip 5pt 
\item{} All quiz scores cease to have meaning
\end{itemize}



%INDEX% Reading assignment: Lessons In Industrial Instrumentation, Elementary Thermodynamics (heat and temperature)

%(END_NOTES)


