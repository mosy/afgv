
%(BEGIN_QUESTION)
% Copyright 2009, Tony R. Kuphaldt, released under the Creative Commons Attribution License (v 1.0)
% This means you may do almost anything with this work of mine, so long as you give me proper credit

Read and outline the ``Specific Heat and Enthalpy'' subsection of the ``Elementary Thermodynamics'' section of the ``Physics'' chapter in your {\it Lessons In Industrial Instrumentation} textbook.  Note the page numbers where important illustrations, photographs, equations, tables, and other relevant details are found.  Prepare to thoughtfully discuss with your instructor and classmates the concepts and examples explored in this reading.

\underbar{file i03979}
%(END_QUESTION)





%(BEGIN_ANSWER)


%(END_ANSWER)





%(BEGIN_NOTES)

1 calorie = amount of heat needed to raise/lower the temperature of 1 gram of water by 1 $^{o}$C.  1 BTU = amount of heat needed to raise/lower the temperature of 1 pound of water by 1 $^{o}$F.

\vskip 10pt

The relationship between heat and temperature for {\it any} substance is given by the following formula:

$$Q = mc \Delta T$$

\noindent
Where,

$Q$ = Heat gained (+) or lost ($-$)

$m$ = Mass of sample
  
$c$ = Specific heat capacity of substance
 
$\Delta T$ = Change in sample temperature from before to after heat gain/loss

\vskip 10pt

Different substances have vastly different heat capacity ($c$) values.  Water is 1, of course.  Lead is 0.031.  Hydrogen gas is 3.41.  Substances having high $c$ values make good convective heat-transfer media.

\vskip 10pt

Water filling up vessels may be used as an analogy for heat capacity, heat, and temperature.  Heat is the volume of water added to the vessel.  Level (height) is the temperature of the sample.  Heat capacity is the width of the vessel: samples with high $c$ are wide vessels, requiring a lot of heat (liquid volume) added in order to increase temperature (liquid level) by any substantial amount.

\vskip 10pt

{\it Enthalpy} is the amount of thermal energy carried by a substance with reference to the freezing point of water.  This figure is useful for calculating how much thermal energy is carried by a fluid in heat transfer systems.  Enthalpy figures for a substance at different temperatures may be subtracted to calculate the amount of heat lost per unit mass as that substance cools from one temperature to another temperature, or the amount of heat gained as is warms from one temperature to another.











\vskip 20pt \vbox{\hrule \hbox{\strut \vrule{} {\bf Suggestions for Socratic discussion} \vrule} \hrule}

\begin{itemize}
\item{} {\bf Explain the concept of enthalpy, in your own words.}
\item{} {\bf Provide a real-life example illustrating the concept of enthalpy.}
\item{} If we need to find a substance for use as a heat-storage medium, should we look for one with a high $c$ value or a low $c$ value?  Explain why.
\item{} If we need to find a fluid substance for use as a cooling medium, should we look for one with a high $c$ value or a low $c$ value?  Explain why.
\item{} Suppose we were designing a solar thermal energy storage system, using a large mass of some substance to store heat gathered via solar collectors over long periods of time.   What substance should we seek for a heat-storage medium, one with a high $c$ value or a low $c$ value?  Explain why.
\item{} Explain analogy of liquids and vessels used to represent heat and heat capacity
\item{} Calculate the enthalpy of water at 200 degrees Fahrenheit.  $H$ = {\bf 168 BTU/lb}
\item{} Calculate the enthalpy of water at 55 degrees Fahrenheit.  $H$ = {\bf 23 BTU/lb}
\item{} Calculate the enthalpy of water at 75 degrees Celsius.  $H$ = {\bf 75 cal/g}
\item{} Calculate the enthalpy of water at 31 degrees Celsius.  $H$ = {\bf 31 cal/g}
\end{itemize}












\vfil \eject

\noindent
{\bf Prep Quiz:}

Calculate the amount of heat necessary to increase the temperature of 3 pounds of water from 45 degrees F to 98 degrees F.

\begin{itemize}
\item{} 53 BTU
\vskip 5pt 
\item{} 159 calories
\vskip 5pt 
\item{} 135 calories
\vskip 5pt 
\item{} 159 BTU
\vskip 5pt 
\item{} 135 BTU
\vskip 5pt 
\item{} 53 calories
\end{itemize}















\vfil \eject

\noindent
{\bf Prep Quiz:}

Calculate the amount of heat necessary to increase the temperature of 5 pounds of water from 60 degrees F to 117 degrees F.

\begin{itemize}
\item{} 585 calories
\vskip 5pt 
\item{} 285 BTU
\vskip 5pt 
\item{} 57 calories
\vskip 5pt 
\item{} 585 BTU
\vskip 5pt 
\item{} 57 BTU
\vskip 5pt 
\item{} 285 calories
\end{itemize}

%INDEX% Reading assignment: Lessons In Industrial Instrumentation, Elementary Thermodynamics (specific heat)

%(END_NOTES)


