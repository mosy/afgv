
%(BEGIN_QUESTION)
% Copyright 2009, Tony R. Kuphaldt, released under the Creative Commons Attribution License (v 1.0)
% This means you may do almost anything with this work of mine, so long as you give me proper credit

Suppose you own a large hot tub holding 700 gallons of water, with water weighing approximately 8.3 pounds per gallon.  Calculate the amount of thermal energy (in units of BTUs) necessary to raise the temperature of the water in the hot tub from ambient (60 degrees Fahrenheit) to 100 degrees Fahrenheit, assuming no heat lost to the surrounding environment in the process.  

\vskip 10pt

Calculate the cost of initially heating this hot tub with propane gas, assuming a propane rate of \$2.20 per gallon, a heating value of 21,700 BTU per pound of propane, and a propane density of 4.2 pounds per gallon.

\vskip 10pt

Calculate the cost of initially heating this hot tub with electricity, assuming an electrical power rate of 8.5 cents per kilowatt-hour.

\vskip 20pt \vbox{\hrule \hbox{\strut \vrule{} {\bf Suggestions for Socratic discussion} \vrule} \hrule}

\begin{itemize}
\item{} Would it be possible to lower the calculated heating cost of this hot tub by insulating it better?  Why or why not?
\item{} What factor(s) affect the continuing energy cost of {\it maintaining} the hot tub at 100 degrees F?  Is it possible, if only in theory, to maintain this temperature with no energy input over time?
\item{} Describe the similarity between thermal energy input versus temperature (in this hot tub example), and {\it mechanical} energy input versus {\it speed} for a moving object such as an automobile.
\item{} Demonstrate how to {\it estimate} numerical answers for this problem without using a calculator.
\end{itemize}

\underbar{file i03980}
%(END_QUESTION)





%(BEGIN_ANSWER)

\noindent
{\bf Partial answer:}

\vskip 10pt

Heat required to warm water from ambient to 100 $^{o}$F = 232,400 BTU

\vskip 10pt

Cost of initially heating the hot tub with electricity = \$5.79

\vskip 10pt

The cost of initially heating the hot tub with propane gas is very nearly the same as with electricity!

%(END_ANSWER)





%(BEGIN_NOTES)

The amount of heat needed to raise the temperature of 5810 pounds of water (700 gallons $\times$ 8.3 pounds per gallon) 40 degrees F (100 $^{o}$F $-$ 60 $^{o}$F) is:

$$Q = mc \Delta T$$

$$Q = (5810 \hbox{ lb})(1)(40^o \hbox{F}) $$

$$Q = \hbox{ 232,400 BTU}$$

This is equivalent to 68.1 kW-hr, which will cost approximately \$5.79 at the stated power rate.

\vskip 10pt

Propane has a fuel value of 21,700 BTU per pound.  This means we would need to burn 10.71 pounds of propane (assuming 100\% efficiency!) to obtain the same heat.  This is equivalent to 2.55 gallons of liquid propane, which at a price of \$2.20 per gallon would cost \$5.61 -- barely cheaper than electricity!  If we were to factor in the typical efficiency of a propane burner and heat exchanger (typically no better than 80\%), electricity becomes the cheaper option.

%INDEX% Physics, heat and temperature: calorimetry problem 

%(END_NOTES)


