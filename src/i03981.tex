
%(BEGIN_QUESTION)
% Copyright 2009, Tony R. Kuphaldt, released under the Creative Commons Attribution License (v 1.0)
% This means you may do almost anything with this work of mine, so long as you give me proper credit

Read and outline the ``Phase Changes'' subsection of the ``Elementary Thermodynamics'' section of the ``Physics'' chapter in your {\it Lessons In Industrial Instrumentation} textbook.  Note the page numbers where important illustrations, photographs, equations, tables, and other relevant details are found.  Prepare to thoughtfully discuss with your instructor and classmates the concepts and examples explored in this reading.

\underbar{file i03981}
%(END_QUESTION)





%(BEGIN_ANSWER)


%(END_ANSWER)





%(BEGIN_NOTES)

Matter has four different phases: solid, liquid, gas, and plasma.  Transitioning from one phase to another generally requires a large exchange of energy.  The energy required per unit mass of substance to transition from one phase to another is called the {\it latent heat} of that substance, in contrast to the {\it specific heat} which is the heat required to merely change its temperature within one phase:

$$Q_{latent} = mL$$

$$Q_{specific} = mc \Delta T$$

Not only do different substances have different latent heat capacity ($L$) values, but also have different values of $L$ for each phase-to-phase transition.  Water, for example, has a latent heat for the liquid-vapor phase change (``latent heat of vaporization'') of 970 BTU per pound, but has a latent heat for the solid-liquid phase change (``latent heat of fusion'') of only 144 BTU per pound.

\vskip 10pt

When a substance exists in a two-phase state (e.g. melting ice, or boiling water), its temperature remains a strict function of pressure.  Boiling water, for example, cannot exist at any temperature other than 212 $^{o}$F (100 $^{o}C$) when at sea-level atmospheric pressure.  So long as pressure remains constant, the temperature of a phase-transitioning substance will be ``locked'' at one value and cannot change until the phase transition is complete.

\vskip 10pt

{\it Enthalpy} is the amount of heat lost per unit mass of a substance as it cools from its given temperature all the way down to the freezing temperature of water.  This figure applies to substances undergoing phase changes in addition to simply cooling within one phase.  A table of figures called a {\it steam table} shows the enthalpy of steam at various temperatures, factoring in the heat released as the steam condenses into water in addition to the heat released by the steam as it approaches the condensation point and the heat released by the water as it cools to freezing temperature.  For this reason, the enthalpy of steam is often referred to as {\it total heat} ($h_g$).

\vskip 10pt

If a vapor is heated to a temperature beyond the boiling point, that vapor is said to be {\it superheated}.

\vskip 10pt

Enthalpy figures may be subtracted in order to calculate heat transferred from one temperature to another, if neither of those temperatures happen to be the freezing point of water (32 $^{o}$F or 0 $^{o}$C).  When calculating heat delivered by a sample of steam, for example, you may subtract the enthalpy of the condensate (the water's temperature minus the freezing temperature) from the original enthalpy of the steam in order to calculate the amount of heat delivered.










\vskip 20pt \vbox{\hrule \hbox{\strut \vrule{} {\bf Suggestions for Socratic discussion} \vrule} \hrule}

\begin{itemize}
\item{} {\bf Consult a steam table and explain how to interpret the values given by it for various pressures and temperatures of steam.}
\item{} Explain why the enthalpy of steam is so much greater than the enthalpy of water.
\item{} Examine the illustration in the textbook showing a liquid-filled vessel equipped with an expansion chamber, and reference this illustration to explain phase changes.  What happens if we add heat to a substance until it changes phase?  What happens if we remove heat from a substance until it changes phase?
\item{} Examine the illustration in the textbook showing a liquid-filled vessel equipped with an expansion chamber, and modify this illustration to illustrate the phase changes of steam to water to ice.  
\item{} Suppose we were designing a solar thermal energy storage system, using a large mass of some substance to store heat gathered via solar collectors over long periods of time.  Would it be better to store this thermal energy in the form of {\it sensible heat} or in the form of {\it latent heat}?  Explain why.
\item{} Suppose we were designing a solar thermal energy storage system, using a large mass of some substance to store heat gathered via solar collectors over long periods of time.  If we wished to exploit the phase change of this substance to store heat, what properties would we look for in selecting a suitable substance?
\end{itemize}












\vfil \eject

\noindent
{\bf Summary Quiz:}

Solar home heating systems often use some form of {\it heat reservoir} to store excess heat captured by the solar collectors during the daytime and release that heat into the home when needed later at night.  A simple medium for heat storage is plain {\it water}, usually contained in large insulated tanks in the home's basement.  During the daytime, the water temperature rises with the collected thermal energy.  During the night, the water temperature falls as thermal energy is extracted to heat the home.

An alternative to using water is to use {\it paraffin wax} as a heat storage medium.  Excess heat gathered by the solar collectors during the day works to melt the wax stored in large insulated tanks.  At night when heat is needed to maintain a comfortable home temperature, heat is extracted from the wax reservoir, causing the wax to re-solidify into solid form.

\vskip 10pt

Explain why a wax-medium heat storage system might be more effective than a water-medium heat storage system, based on your knowledge of thermodynamics.  Be as detailed as you can in your answer!

%INDEX% Reading assignment: Lessons In Industrial Instrumentation, Elementary Thermodynamics (phase changes)

%(END_NOTES)


