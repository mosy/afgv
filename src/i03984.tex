
%(BEGIN_QUESTION)
% Copyright 2009, Tony R. Kuphaldt, released under the Creative Commons Attribution License (v 1.0)
% This means you may do almost anything with this work of mine, so long as you give me proper credit

Read and outline the ``Temperature Coefficient of Resistance ($\alpha$)'' subsection of the ``Thermistors and Resistance Temperature Detectors (RTDs)'' section of the ``Continuous Temperature Measurement'' chapter in your {\it Lessons In Industrial Instrumentation} textbook.  Note the page numbers where important illustrations, photographs, equations, tables, and other relevant details are found.  Prepare to thoughtfully discuss with your instructor and classmates the concepts and examples explored in this reading.


\underbar{file i03984}
%(END_QUESTION)





%(BEGIN_ANSWER)


%(END_ANSWER)





%(BEGIN_NOTES)

An RTD is a temperature sensor made of fine metal wire, the resistance of which may be closely approximated by the following formula:

$$R_T = R_{ref}[1 + \alpha(T - T_{ref})]$$

\noindent
Where,

$R_T$ = Resistance of RTD at given temperature $T$ (ohms)

$R_{ref}$ = Resistance of RTD at the reference temperature $T_{ref}$ (ohms)

$\alpha$ = Temperature coefficient of resistance (ohms per ohm/degree)

\vskip 10pt

RTD resistance values may also be found in {\it tables}, listing temperature and corresponding resistance values.  Interpolation may be used to identify values lying between published numbers.

\vskip 10pt

100 ohms is the most common ``base'' resistance of an RTD, and 0.00385 is the most common $\alpha$ value in industrial use.  10 ohm and 1000 ohm RTDs are also found in industrial use, as are other temperature coefficient values (0.00392 $\Omega$/$\Omega$/$^{o}$C and 0.003902 $\Omega$/$\Omega$/$^{o}$C in particular).











\vskip 20pt \vbox{\hrule \hbox{\strut \vrule{} {\bf Suggestions for Socratic discussion} \vrule} \hrule}

\begin{itemize}
\item{} Do RTDs have {\it positive} or {\it negative} temperature coefficients of resistance?
\item{} Why might one prefer to use an RTD table to look up resistance values when the resistance formula is so easy to use?
\item{} Explain what the {\it Callendar-van Dusen formula} is and how this relates to RTD sensing elements.
\item{} Will a failed-open RTD appear to be at a very low temperature or a very high temperature?
\item{} Will a failed-shorted RTD appear to be at a very low temperature or a very high temperature?
\end{itemize}

%INDEX% Reading assignment: Lessons In Industrial Instrumentation, Continuous Temperature Measurement (RTDs and thermistors)

%(END_NOTES)


