
%(BEGIN_QUESTION)
% Copyright 2009, Tony R. Kuphaldt, released under the Creative Commons Attribution License (v 1.0)
% This means you may do almost anything with this work of mine, so long as you give me proper credit

Read and outline the ``Two-Wire RTD Circuits'', ``Four-Wire RTD Circuits'', ``Three-Wire RTD Circuits'', and ``Proper RTD Sensor Connections'' subsections of the ``Thermistors and Resistance Temperature Detectors (RTDs)'' section of the ``Continuous Temperature Measurement'' chapter in your {\it Lessons In Industrial Instrumentation} textbook.  Note the page numbers where important illustrations, photographs, equations, tables, and other relevant details are found.  Prepare to thoughtfully discuss with your instructor and classmates the concepts and examples explored in this reading.

\underbar{file i03985}
%(END_QUESTION)





%(BEGIN_ANSWER)


%(END_ANSWER)





%(BEGIN_NOTES)

In a 2-wire RTD circuit, wire resistance directly adds the the RTD sensor's resistance to form a larger (total) circuit resistance.  This means wire resistance introduces a {\it positive measurement error}.  For thermistors with high resistance values, the relatively low wire resistance is ``swamped'' and therefore constitutes negligible error.  However, in low-resistance RTD circuits the error may be substantial.

\vskip 10pt

In a 4-wire RTD circuit, two wires carry excitation current to the RTD while two other wires connect the voltage-sensing instrument to the RTD.  The excitation wire resistance is of no consequence because the voltmeter doesn't see that voltage drop, only the drop across the RTD.  The voltmeter's wire resistance is of no consequence because the voltmeter draws negligible current.  Thus, a 4-wire RTD circuit completely eliminates the problem of temperature measurement errors caused by wire resistance.

\vskip 10pt

In a 3-wire RTD circuit, the voltage measurements made at the far end may be used to cancel out effects of wire resistance, if we assume the voltage dropped across each of the current-carrying wires is the same (i.e. those two wires have the same resistance along their respective lengths).  If that assumption is incorrect, we will have a measurement error.

\vskip 10pt

Modern ``smart'' RTD transmitters may be configured for 2-wire, 3-wire, or 4-wire RTDs.  Note that the common connection points shown on the connection diagram always refer to connections made {\it at or within the RTD}, never jumpers at the transmitter terminals!









\vskip 20pt \vbox{\hrule \hbox{\strut \vrule{} {\bf Suggestions for Socratic discussion} \vrule} \hrule}

\begin{itemize}
\item{} {\bf In what ways may an RTD sensor be ``fooled'' to report a false temperature measurement?}
\item{} Explain what the term {\it swamping} refers to, and how this concept applies to 2-wire RTD circuits.
\item{} Does wire resistance make an RTD appear to be colder than it really is, or hotter than it really is?
\item{} Thermistors are popularly used in HVAC temperature-sensing applications.  Why don't we use thermistors instead of RTDs for industrial applications?
\item{} Explain how the ``Kelvin'' or ``4-wire'' resistance measurement technique works to eliminate the effects of wire resistance.
\item{} If one of the excitation wires were to fail open in a 4-wire RTD circuit, would this make the RTD appear very cold or very hot?
\item{} If one of the sense wires were to fail open in a 4-wire RTD circuit, would this make the RTD appear very cold or very hot?
\item{} If the upper current-carrying wire has slightly more resistance than the lower current-carrying wire in the 3-wire dual-voltmeter RTD circuit shown in the texbook, would this make the RTD appear colder than it is or hotter than it is?
\item{} If the lower current-carrying wire has slightly more resistance than the upper current-carrying wire in the 3-wire dual-voltmeter RTD circuit shown in the texbook, would this make the RTD appear colder than it is or hotter than it is?
\item{} If the upper current-carrying wire has slightly more resistance than the lower current-carrying wire in the 3-wire opamp RTD circuit shown in the texbook, would this make the RTD appear colder than it is or hotter than it is?
\item{} If the lower current-carrying wire has slightly more resistance than the upper current-carrying wire in the 3-wire opamp RTD circuit shown in the texbook, would this make the RTD appear colder than it is or hotter than it is?
\item{} Interpret the photograph of a temperature transmitter's terminals, identifying how to connect a 2-wire, 3-wire, and 4-wire RTD to this transmitter.
\item{} Explain why some of the ``incorrect'' connection diagrams will yield inaccurate temperature measurements.
\item{} Does {\it self-heating} result in falsely high temperature measurements or falsely low temperature measurements?
\end{itemize}














\vfil \eject

\noindent
{\bf Prep Quiz:}

{\it Two-wire} RTD circuits suffer from what disadvantage, compared against three- and four-wire RTD circuits?

\begin{itemize}
\item{} Self-heating error, causing the circuit to register a falsely high temperature
\vskip 5pt 
\item{} Susceptibility to sensor damage from mechanical or thermal ``shock'' 
\vskip 5pt 
\item{} Corrosion due to process vapors attacking the wire connections at the sensor
\vskip 5pt 
\item{} Measurement error caused by the inclusion of wire resistance in the RTD circuit
\vskip 5pt 
\item{} Measurement error caused by changes in ambient temperature at the indicating meter
\vskip 5pt 
\item{} Greater costs due to the need for more conductors, larger conduit pipes, etc.
\end{itemize}












\vfil \eject

\noindent
{\bf Prep Quiz:}

The measurement error in a {\it two-wire} RTD circuit resulting from wire resistance will cause the indicating meter to register:

\begin{itemize}
\item{} A temperature that is always too high
\vskip 5pt 
\item{} A temperature that rapidly fluctuates
\vskip 5pt 
\item{} A temperature that is exactly 5 degrees off
\vskip 5pt 
\item{} A temperature that is always too low
\vskip 5pt 
\item{} A temperature that slowly drifts up and down
\end{itemize}

%INDEX% Reading assignment: Lessons In Industrial Instrumentation, Continuous Temperature Measurement (RTDs and thermistors)

%(END_NOTES)


