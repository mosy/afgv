
%(BEGIN_QUESTION)
% Copyright 2009, Tony R. Kuphaldt, released under the Creative Commons Attribution License (v 1.0)
% This means you may do almost anything with this work of mine, so long as you give me proper credit

Read and outline the ``Dissimilar Metal Junctions'' subsection of the ``Thermocouples'' section of the ``Continuous Temperature Measurement'' chapter in your {\it Lessons In Industrial Instrumentation} textbook.  Note the page numbers where important illustrations, photographs, equations, tables, and other relevant details are found.  Prepare to thoughtfully discuss with your instructor and classmates the concepts and examples explored in this reading.


\underbar{file i03986}
%(END_QUESTION)





%(BEGIN_ANSWER)


%(END_ANSWER)





%(BEGIN_NOTES)

Dissimilar-metal junctions produce voltage roughly proportional to temperature.  When we form a complete circuit from dissimilar metals, we inevitably form a {\it pair} of voltage-opposed junctions.  This is why we get a measurement of zero volts when connecting a voltmeter to a room-temperature thermocouple: both the thermocouple junction and the ``reference'' or ``cold'' junction have equal and opposite voltages, producing a net voltage at the meter of zero.

$$V_{meter} = V_{J1} - V_{J2}$$

\noindent
Where,

$V_{J1}$ = Measurement junction voltage

$V_{J2}$ = ``Reference'' or ``cold'' junction voltage

\vskip 10pt

Understanding the principle of this one formula is critical to understanding all thermocouple circuits.  The best way to analyze thermocouple circuits is to continually return to the notion of a simple two-junction voltage loop and run thought experiments to see how it will react under different temperature conditions.


\vskip 20pt \vbox{\hrule \hbox{\strut \vrule{} {\bf Suggestions for Socratic discussion} \vrule} \hrule}

\begin{itemize}
\item{} {\bf In what ways may a thermocouple be ``fooled'' to report a false temperature measurement?}
\item{} Explain why the voltage registered by a meter in the thermocouple circuit fundamentally reflects the {\it difference} in temperature between the measurement and reference junctions.
\item{} Can we avoid the problem of a reference junction by building a voltmeter with dissimilar-metal test leads (e.g. iron for the red lead and copper for the black lead)?
\item{} Why do you measure zero volts when connecting a voltmeter to a room-temperature thermocouple?
\end{itemize}

%INDEX% Reading assignment: Lessons In Industrial Instrumentation, Continuous Temperature Measurement (thermocouples)

%(END_NOTES)


