
%(BEGIN_QUESTION)
% Copyright 2009, Tony R. Kuphaldt, released under the Creative Commons Attribution License (v 1.0)
% This means you may do almost anything with this work of mine, so long as you give me proper credit

Read and outline the ``Thermocouple Types'' subsection of the ``Thermocouples'' section of the ``Continuous Temperature Measurement'' chapter in your {\it Lessons In Industrial Instrumentation} textbook.  Note the page numbers where important illustrations, photographs, equations, tables, and other relevant details are found.  Prepare to thoughtfully discuss with your instructor and classmates the concepts and examples explored in this reading.

\underbar{file i03987}
%(END_QUESTION)





%(BEGIN_ANSWER)


%(END_ANSWER)





%(BEGIN_NOTES)

% No blank lines allowed between lines of an \halign structure!
% I use comments (%) instead, so Tex doesn't choke.

$$\vbox{\offinterlineskip
\halign{\strut
\vrule \quad\hfil # \ \hfil & 
\vrule \quad\hfil # \ \hfil & 
\vrule \quad\hfil # \ \hfil & 
\vrule \quad\hfil # \ \hfil & 
\vrule \quad\hfil # \ \hfil \vrule \cr
\noalign{\hrule}
%
% First row
{\bf Type} & {\bf Positive wire} & {\bf Negative wire} & {\bf Plug} & {\bf Temp. range} \cr
     & {\bf characteristic} & {\bf characteristic} &  & \cr
%
\noalign{\hrule}
%
% Another row
T & Copper (blue) & Constantan (red) & Blue & $-300$ to 700 $^{o}$F \cr
  & {\it yellow colored} & {\it silver colored} &  &  \cr
%
\noalign{\hrule}
%
% Another row
J & Iron (white) & Constantan (red) & Black & 32 to 1400 $^{o}$F \cr
  & {\it magnetic, rusty?} & {\it non-magnetic} &  &  \cr
%
\noalign{\hrule}
%
% Another row
E & Chromel (violet) & Constantan (red) & Violet & 32 to 1600 $^{o}$F \cr
  & {\it shiny finish} & {\it dull finish} &  &  \cr
%
\noalign{\hrule}
%
% Another row
K & Chromel (yellow) & Alumel (red) & Yellow & 32 to 2300 $^{o}$F \cr
  & {\it non-magnetic} & {\it magnetic} &  &  \cr
%
\noalign{\hrule}
%
% Another row
N & Nicrosil (orange) & Nisil (red) & Orange & 32 to 2300 $^{o}$F \cr
%
\noalign{\hrule}
%
% Another row
S & Pt90\% - Rh10\% (black) & Platinum (red) & Green & 32 to 2700 $^{o}$F \cr
%
\noalign{\hrule}
%
% Another row
B & Pt70\% - Rh30\% (grey) & Pt94\% - Rh6\% (red) & Grey & 32 to 3380 $^{o}$F \cr
%
\noalign{\hrule}
} % End of \halign 
}$$ % End of \vbox

The negative wire of every thermocouple is colored red in the North American (US and Canada) color code.

\vskip 10pt

Type J thermocouples cannot withstand oxidizing atmospheres because of the iron wire (it will rust).  Type K thermocouples cannot withstand reducing atmospheres, nor cyanide, nor sulfur.  Type T thermocouples have a low upper temperature limit due to the copper wire they contain.





\vskip 20pt \vbox{\hrule \hbox{\strut \vrule{} {\bf Suggestions for Socratic discussion} \vrule} \hrule}

\begin{itemize}
\item{} Are there variations in thermocouple color codes, or are the colors standardized world-wide?
\item{} Identify some practical applications where you would {\it not} want to use a type J thermocouple
\item{} Identify some practical applications where you would {\it not} want to use a type K thermocouple
\item{} Identify some practical applications where you would {\it not} want to use a type T thermocouple
\end{itemize}

%INDEX% Reading assignment: Lessons In Industrial Instrumentation, Continuous Temperature Measurement (thermocouples)

%(END_NOTES)


