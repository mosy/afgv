
%(BEGIN_QUESTION)
% Copyright 2009, Tony R. Kuphaldt, released under the Creative Commons Attribution License (v 1.0)
% This means you may do almost anything with this work of mine, so long as you give me proper credit

Read and outline the ``Manually Interpreting Thermocouple Voltages'' subsection of the ``Thermocouples'' section of the ``Continuous Temperature Measurement'' chapter in your {\it Lessons In Industrial Instrumentation} textbook.  Note the page numbers where important illustrations, photographs, equations, tables, and other relevant details are found.  Prepare to thoughtfully discuss with your instructor and classmates the concepts and examples explored in this reading.

\underbar{file i03989}
%(END_QUESTION)





%(BEGIN_ANSWER)


%(END_ANSWER)





%(BEGIN_NOTES)

The act of connecting a voltmeter to any thermocouple necessarily creates a reference junction, the polarity of which is opposed to the thermocouple's measurement junction.  Since the meter's indication will naturally be the {\it difference} between these two junctions' voltages, we must determine the voltage produced by the reference junction and {\it add} that voltage to the meter's indicated voltage in order to determine the voltage being produced by the thermocouple (measurement junction):

$$V_{meter} = V_{J1} - V_{J2}$$

$$V_{J1} = V_{meter} + V_{J2}$$

\noindent
Where,

$V_{meter}$ = Voltage indicated by voltmeter connected to thermocouple

$V_{J1}$ = Measurement junction voltage

$V_{J2}$ = ``Reference'' or ``cold'' junction voltage

\vskip 10pt

In order to determine the temperature of the reference junction (J2) so that we may look up its voltage in a thermocouple table, we must use a thermometer or some other ambient temperature measurement device.











\vskip 20pt \vbox{\hrule \hbox{\strut \vrule{} {\bf Suggestions for Socratic discussion} \vrule} \hrule}

\begin{itemize}
\item{} Explain why we cannot simply take the voltmeter's reading and go directly to a thermocouple table to look up the corresponding temperature value when determining the temperature of a thermocouple junction.
\end{itemize}


%INDEX% Reading assignment: Lessons In Industrial Instrumentation, Continuous Temperature Measurement (thermocouples)

%(END_NOTES)


