
%(BEGIN_QUESTION)
% Copyright 2009, Tony R. Kuphaldt, released under the Creative Commons Attribution License (v 1.0)
% This means you may do almost anything with this work of mine, so long as you give me proper credit

Read and outline the ``Reference Junction Compensation'' subsection of the ``Thermocouples'' section of the ``Continuous Temperature Measurement'' chapter in your {\it Lessons In Industrial Instrumentation} textbook.  Note the page numbers where important illustrations, photographs, equations, tables, and other relevant details are found.  Prepare to thoughtfully discuss with your instructor and classmates the concepts and examples explored in this reading.


\underbar{file i03990}
%(END_QUESTION)





%(BEGIN_ANSWER)


%(END_ANSWER)





%(BEGIN_NOTES)

The reference junction in a thermocouple circuit (J2) generates a voltage that subtracts from the measurement junction's voltage (J1) when read by a voltmeter.  Therefore, in order to make the voltmeter read the true measurement junction voltage, we must somehow cancel out the voltage created by J2.  

\vskip 10pt

One way of doing this is to add a third voltage source to the thermocouple circuit ($V_{rjc}$) to {\it compensate} for the reference junction voltage.  This additional voltage source must be equal in value to the voltage of J2, but opposing in polarity.

\vskip 10pt

Electronic ``ice-point'' circuits do exactly this function, using a thermistor or other temperature-sensing element inside to measure ambient temperature and generate an appropriate compensating voltage.









\vskip 20pt \vbox{\hrule \hbox{\strut \vrule{} {\bf Suggestions for Socratic discussion} \vrule} \hrule}

\begin{itemize}
\item{} Many thermocouple transmitters provide the option of turning reference junction compensation {\it off}.  How would this affect the transmitter's ability to read the temperature of the thermocouple?  Can you think of any practical application where we would {\it not} want the transmitter to compensate for the reference junction?
\item{} Why go through the trouble of building an ice-point circuit if we just have to use a different type of temperature sensor within that circuit for it to compensate for the thermocouple's reference junction voltage?  Why not just simplify matters and use this {\it other} sensor type instead of a thermocouple in the process?
\item{} If you were to design and build an ice-point circuit, would that circuit be universal in its application or would it be specific to one type of thermocouple?
\item{} Explain what would happen if someone were to mis-connect an ice-point into a thermocouple loop by placing it ``backwards'' or ``upside-down'' in the circuit.
\item{} How expensive is an ice-point module?  {\tt http:///www.omega.com} is a good resource for identifying prices of temperature measurement equipment.
\end{itemize}











\vfil \eject

\noindent
{\bf Prep Quiz:}

In order to properly compensate for the reference junction in a thermocouple circuit, the compensation voltage ($V_{rjc}$) must be:

\begin{itemize}
\item{} Equal value and opposite polarity to the voltage generated by the measurement junction 
\vskip 5pt 
\item{} Precisely equal to 1.048 millivolts, which is equivalent to room temperature
\vskip 5pt 
\item{} Equal value and opposite polarity to the voltage generated by the reference junction
\vskip 5pt 
\item{} Precisely equal to zero volts, in order to mimic an ice-bath reference
\vskip 5pt 
\item{} Equal value and additive polarity to the voltage generated by the reference junction
\vskip 5pt 
\item{} Equal value and additive polarity to the voltage generated by the measurement junction 
\end{itemize}

%INDEX% Reading assignment: Lessons In Industrial Instrumentation, Continuous Temperature Measurement (thermocouples)

%(END_NOTES)


