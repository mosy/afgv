
%(BEGIN_QUESTION)
% Copyright 2009, Tony R. Kuphaldt, released under the Creative Commons Attribution License (v 1.0)
% This means you may do almost anything with this work of mine, so long as you give me proper credit

Read and outline the ``Law of Intermediate Metals'' subsection of the ``Thermocouples'' section of the ``Continuous Temperature Measurement'' chapter in your {\it Lessons In Industrial Instrumentation} textbook.  Note the page numbers where important illustrations, photographs, equations, tables, and other relevant details are found.  Prepare to thoughtfully discuss with your instructor and classmates the concepts and examples explored in this reading.


\underbar{file i03991}
%(END_QUESTION)





%(BEGIN_ANSWER)


%(END_ANSWER)





%(BEGIN_NOTES)

The {\it Law of Intermediate Metals} states that when three or more metals are joined in a series chain, and all dissimilar metal junctions within that chain are at the same temperature, we may treat that series of junctions as a single dissimilar-metal junction comprised of the outer two metals.  In other words, all ``intermediate'' metals in between the two outer metals are of no consequence if the intermediate junctions are all at the same temperature.

This means when we join a thermocouple of two non-copper wires to an instrument with copper wires, the copper is an intermediate metal, and we may treat the two thermocouple metal-copper junctions at that instrument as a single thermocouple reference junction.  

\vskip 10pt

If the Law of Intermediate Metals were untrue, a loop formed of dissimilar metals (all junctions at equal temperature) would generate electrical power, which would violate the Conservation of Energy (i.e. voltage and current produced with no energy input).

\vskip 10pt

The Law of Intermediate Metals is extremely important for modern semiconductor instruments, where a multitude of dissimilar metals join en route to the silicon integrated circuit measuring thermocouple voltage.  All those junctions may be dismissed as equivalent to a single thermocouple reference junction because they're at the same temperature.





\vskip 20pt \vbox{\hrule \hbox{\strut \vrule{} {\bf Suggestions for Socratic discussion} \vrule} \hrule}

\begin{itemize}
\item{} Explain why the Law of Intermediate Metals is important to us as technicians.  How would thermocouple thermometry be different if this Law were not true?
\item{} Describe the proof given for the Law of Intermediate Metals in the textbook.  What is the basis for the conclusion that a series of junctions at the same temperature is equivalent to a single junction comprised of the outer two metals?
\item{} Looking at the figure showing all the dissimilar metal junctions in a semiconductor instrument application, is the Law of Intermediate Metals invalidated if the junctions become progressively warmer as we get closer to the silicon amplifier?
\end{itemize}

%INDEX% Reading assignment: Lessons In Industrial Instrumentation, Continuous Temperature Measurement (thermocouples)

%(END_NOTES)


