
%(BEGIN_QUESTION)
% Copyright 2009, Tony R. Kuphaldt, released under the Creative Commons Attribution License (v 1.0)
% This means you may do almost anything with this work of mine, so long as you give me proper credit

Read and outline the ``Software Compensation'' subsection of the ``Thermocouples'' section of the ``Continuous Temperature Measurement'' chapter in your {\it Lessons In Industrial Instrumentation} textbook.  Note the page numbers where important illustrations, photographs, equations, tables, and other relevant details are found.  Prepare to thoughtfully discuss with your instructor and classmates the concepts and examples explored in this reading.


\underbar{file i03992}
%(END_QUESTION)





%(BEGIN_ANSWER)


%(END_ANSWER)





%(BEGIN_NOTES)

Hardware compensation is where we add a third voltage source to the thermocouple circuit, canceling out the effects of the reference junction voltage.  Software compensation is when we do this cancellation arithmetically, using a digital computing device equipped with its own ambient temperature sensor.

\vskip 10pt

Software compensation enjoys the advantage of being programmable for different types of thermocouples, so that one instrument may be switched for different thermocouple applications.  With analog thermocouple instrument, the ``ice-point'' compensation as well as the meter's span must be electrically changed in order to work with different thermocouple types.






\vskip 20pt \vbox{\hrule \hbox{\strut \vrule{} {\bf Suggestions for Socratic discussion} \vrule} \hrule}

\begin{itemize}
\item{} Suppose we wished to modify an analog thermocouple indicator (e.g. voltmeter with custom scale) to work with type K thermocouples instead of type J thermocouples.  What exactly would have to be altered in order to re-scale the meter for a different type of thermocouple?
\item{} Suppose we wished to modify an electronic ice-point module to work with type K thermocouples instead of type J thermocouples.  What exactly would have to be altered in order to re-scale the ice point to compensate for a different type of thermocouple reference junction?
\item{} Examine the block diagram of a digital temperature transmitter and explain what is happening inside.
\item{} Examine the block diagram of a digital temperature transmitter and explain what changes when the user switches the thermocouple type parameter.
\end{itemize}








\vfil \eject

\noindent
{\bf Prep Quiz:}

A distinct advantage {\it software} compensation enjoys over {\it hardware} compensation in a thermocouple transmitter is:

\begin{itemize}
\item{} Total compensation for changes in ambient temperature
\vskip 5pt 
\item{} Far greater absolute temperature measurement accuracy 
\vskip 5pt 
\item{} Increased reliability, as there is less hardware to malfunction
\vskip 5pt 
\item{} Eliminates errors due to thermocouple wire resistance
\vskip 5pt 
\item{} Better rejection of noise voltage induced by outside sources
\vskip 5pt 
\item{} The ability to easily adapt to different thermocouple types
\end{itemize}



%INDEX% Reading assignment: Lessons In Industrial Instrumentation, Continuous Temperature Measurement (thermocouples)

%(END_NOTES)


