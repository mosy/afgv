
%(BEGIN_QUESTION)
% Copyright 2009, Tony R. Kuphaldt, released under the Creative Commons Attribution License (v 1.0)
% This means you may do almost anything with this work of mine, so long as you give me proper credit

Read and outline the ``Side-Effects of Reference Junction Compensation'' subsection of the ``Thermocouples'' section of the ``Continuous Temperature Measurement'' chapter in your {\it Lessons In Industrial Instrumentation} textbook.  Note the page numbers where important illustrations, photographs, equations, tables, and other relevant details are found.  Prepare to thoughtfully discuss with your instructor and classmates the concepts and examples explored in this reading.

\underbar{file i03994}
%(END_QUESTION)





%(BEGIN_ANSWER)


%(END_ANSWER)





%(BEGIN_NOTES)

If a thermocouple transmitter's input terminals are directly shorted together, it will register ambient temperature because all it ``sees'' is the reference junction compensation.  This is a handy diagnostic test for a thermocouple instrument!

\vskip 10pt

When attempting to simulate an operating thermocouple to a thermocouple instrument by inputting millivoltage signals, you must ``out-smart'' the instrument's internal reference junction compensation because it still ``thinks'' there is real thermocouple wire (and therefore a reference junction) connected to it for which it must compensate.  Since the effect of reference junction compensation is {\it additive}, you must therefore {\it subtract} the appropriate ambient temperature reference junction voltage from the desired temperature measurement junction voltage you intend to simulate.




\vskip 20pt \vbox{\hrule \hbox{\strut \vrule{} {\bf Suggestions for Socratic discussion} \vrule} \hrule}

\begin{itemize}
\item{} Explain why {\it any} thermocouple transmitter will read ambient temperature if you short the input terminals together.
\item{} Does it matter whether you short the terminals together using copper wire versus a single piece of thermocouple wire?
\item{} Explain why you cannot simply look up the millivoltage for a thermocouple at a particular temperature and input that millivoltage to a thermocouple transmitter to simulate that temperature condition.
\item{} Explain why it might be a good idea to disable the reference junction compensation feature of a temperature transmitter when performing a calibration on it.
\item{} Identify a practical application for a thermocouple temperature transmitter lacking reference junction compensation.
\end{itemize}



\vfil \eject

\noindent
{\bf Prep Quiz:}

If you short-circuit the input terminals of any well-calibrated thermocouple transmitter, it will:

\begin{itemize}
\item{} Register precisely 32 degrees F
\vskip 5pt 
\item{} Register precisely 0 degrees F
\vskip 5pt 
\item{} Register a negative temperature value
\vskip 5pt 
\item{} Register low-scale (LRV) temperature
\vskip 5pt 
\item{} Register ambient air temperature
\vskip 5pt 
\item{} Register full-scale (URV) temperature
\end{itemize}




%INDEX% Reading assignment: Lessons In Industrial Instrumentation, Continuous Temperature Measurement (thermocouples)

%(END_NOTES)


