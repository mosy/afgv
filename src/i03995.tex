
%(BEGIN_QUESTION)
% Copyright 2009, Tony R. Kuphaldt, released under the Creative Commons Attribution License (v 1.0)
% This means you may do almost anything with this work of mine, so long as you give me proper credit

Read and outline the ``Burnout Detection'' subsection of the ``Thermocouples'' section of the ``Continuous Temperature Measurement'' chapter in your {\it Lessons In Industrial Instrumentation} textbook.  Note the page numbers where important illustrations, photographs, equations, tables, and other relevant details are found.  Prepare to thoughtfully discuss with your instructor and classmates the concepts and examples explored in this reading.

\underbar{file i03995}
%(END_QUESTION)





%(BEGIN_ANSWER)


%(END_ANSWER)





%(BEGIN_NOTES)

The direction of saturation that a thermocouple instrument assumes in the event of a failed-open (``burned out'') thermocouple is generally user-selectable.

\vskip 20pt \vbox{\hrule \hbox{\strut \vrule{} {\bf Suggestions for Socratic discussion} \vrule} \hrule}

\begin{itemize}
\item{} Why is it a good thing that temperature transmitters are equipped with burnout detection?
\item{} Why should we care what the burnout mode of the transmitter will be?
\item{} Explain why the resistor shown in the simplified schematic diagram needs to be a very large (mega-ohm) value.  What would happen if this resistor's value were made too low?
\end{itemize}

%INDEX% Reading assignment: Lessons In Industrial Instrumentation, Continuous Temperature Measurement (thermocouples)

%(END_NOTES)


