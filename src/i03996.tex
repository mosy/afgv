
%(BEGIN_QUESTION)
% Copyright 2009, Tony R. Kuphaldt, released under the Creative Commons Attribution License (v 1.0)
% This means you may do almost anything with this work of mine, so long as you give me proper credit

Read pages 10-13 of the Nuclear Regulatory Commission's ``Three Mile Island -- A Report to the Commissioners and to the Public'' for an overview of the nuclear reactor power generating system, and pages 30-31 (Section 5 of the accident narrative) highlighting instrument technicians' roles on the day of the March 28, 1979 accident, then answer the following questions:

\vskip 10pt

The normal operating temperature of water inside the reactor vessel is 575 $^{o}$F, but yet the water does not boil even when the reactor is operating at full power.  Explain how boiling is prevented with such a high operating temperature (see page 10), referencing a {\it steam table} and a {\it phase diagram }if necessary.

\vskip 10pt

On pages 40-41 of the report, it states that measured reactor steam temperatures in excess of 600 degrees F at a measured pressure of 495 PSI were clues that the heat-generating reactor core was ``uncovered'' (no longer fully submerged in water inside the reactor vessel), because these figures imply {\it superheated} steam.  Explain this reasoning, based on what you know about phase changes in general and steam in particular.  Refer to a {\it steam table} and a {\it phase diagram} if necessary.

\vskip 10pt

At the heart of the accident was a stuck-open pressure relief valve called a {\it PORV}.  This open valve allowed cooling water to escape from the primary (reactor) coolant loop, eventually leading to a condition where about half of the reactor core (normally submerged in cooling water) was uncovered.  Instrument technicians were sent to manually measure reactor core temperatures by taking electrical measurements off the ``incore'' thermocouple wires (pp. 30-31 of the Report).  Based on what you know about thermocouples, explain exactly the steps these instrument technicians would have taken to obtain the temperature measurements.

\vskip 10pt

According to the Report, what subsequent action(s) were taken following the instrument technicians' measurements?

\vskip 10pt

Hint: the {\it Socratic Instrumentation} website contains a page where you may download public-domain textbooks, one of which is a set of steam tables published in 1920.  The {\it Fisher Control Valve Handbook} also has a (less comprehensive) set of steam tables in the Appendix section.



\vskip 20pt \vbox{\hrule \hbox{\strut \vrule{} {\bf Suggestions for Socratic discussion} \vrule} \hrule}

\begin{itemize}
\item{} Some of the millivoltage measurements taken of ``incore thermocouples'' by the technicians were ``too low to be believable,'' and this is one of the reasons the high millivoltage measurements were dismissed.  Based on what you know of thermocouples, is there any way a very hot thermocouple could produce unreasonably {\it small} millivoltage signals?
\item{} Identify the ways in which heat is transferred from a nuclear reactor core to do useful work.
\item{} Identify the point on a phase diagram for water where a pressurized-water reactor such as TMI is supposed to operate.
\item{} An insightful paragraph from this report reads, {\it ``Nuclear physics, as all science, is logical, and those instruments are speaking the truth to anyone logical enough to recognize it.  Defeat awaits the individual who even subconsciously ascribes human qualitites like crankiness to functioning meters and thermocouples simply because they are bringing bad news.  The answer in science is never to shoot the messenger.''} (page 27).  Discuss how you might very well encounter such attitudes on the job, and from whom these attitudes might come.
\item{} On pages 42-43 of the report, it documents a ``hydrogen burn'' inside the concrete containment building, fueled by flammable hydrogen gas released by the overheated reactor through the stuck-open PORV.  Explain how people were able to determine this was in fact a real explosion that occurred inside the building, and also identify what they assumed the instruments' readings meant when it first happened (i.e. not an explosion!).
\end{itemize}

\underbar{file i03996}
%(END_QUESTION)





%(BEGIN_ANSWER)

Hint: according to a saturated steam table, the vapor pressure of water near 575 $^{o}$F is approximately 1280 PSIG.

%(END_ANSWER)





%(BEGIN_NOTES)

\noindent
{\bf Pages 10-13}

Boiling inside the primary coolant loop is prevented by maintaining water {\it pressure} at 2155 PSIG -- well above the saturated vapor pressure of approximately 1280 PSIG.  The device responsible for maintaining this high water pressure is the {\it pressurizer}, equipped with electrical heating elements inside to force a sample of primary loop water to boil and maintain a ``steam bubble'' at operating pressure.  These electrical heating elements are hotter than the reactor core, and so boiling only happens inside the pressurizer, not inside the reactor core.

Water in the primary coolant loop transfers heat to a secondary loop of water through a heat exchanger called a {\it steam generator} where water is allowed to boil into steam.  Primary cooling water is circulated by four reactor coolant pumps (two pumps in each of the two primary coolant loops).  One reactor, two coolant loops, four coolant pumps, two steam generators, one steam turbine, one electrical generator.

At the top of the pressurizer is a PORV (Pressure Operated Relief Valve) which lets steam out of the primary coolant loop if the pressure in that loop ever exceeds the lift pressure of that valve.  This PORV is supposed to re-seat when the coolant loop pressure returns to normal.

Secondary loop water is boiled into steam inside the two steam generators, then passes through the turbine to turn the electrical generator, then enters a condenser where it is cooled back into water again.  A third loop of water is used to transfer heat from the condenser to the cooling towers.  ``Condensate polishers'' or ``demineralizers'' are used to remove minerals from the secondary loop water.

An Emergency Core Cooling System (ECCS) to cool the reactor core in emergency conditions consists of high-pressure injection (HPI) pumps to force cool water into the reactor if the primary loop pressure is still high, a set of low-pressure injection (LPI) pumps which can force more water into the reactor if the loop pressure is low (e.g. if there is a large leak in the primary loop).  Both of these injection pumps draw borated water from a Borated Water Storage Tank (BWST).  The LPI pumps may also operate in circulation mode to take primary water to a special heat exchanger designed to remove decay heat from the reactor following a shutdown.  A set of core flood tanks pressurized by gas serve to provide immediate emergency coolant to the core in the intervening time it takes for the HPI/LPI pumps to start up.

The chain of events went as follows: a condensate polisher problem caused the secondary coolant loop to stop, which caused the primary loop to over-pressurize, lifting the PORV.  The PORV failed to shut as it should have, leaking primary coolant from the reactor in a way that confounded operators and ultimately exposed the reactor core itself.

\vskip 10pt

\noindent
{\bf Pages 30-31}

Operators closed block valves at the PORV to stop the loss of primary coolant.  The reactor coolant pumps remain in the off condition, but the HPI pumps are forcing additional water into the reactor.  No cooling action through natural convection is happening because steam and gas bubbles in the system have prevented the primary loop from being an all-liquid system.

A voltmeter spliced into the circuit measuring hot leg temperature shows a temperature reading that is clearly in the superheated range, which proves enough coolant has been lost to expose the reactor core to steam.  A set of thermocouples buried inside the reactor core (the ``incore'' thermocouples) are reading over-range.  Instrument technicians take voltage readings across the leads of several incore thermocouples, finding many to be too low to be believable but finding two that register over 2000 degrees F.  The technicians express their concern that the core has become uncovered.  The instrument engineer ignores the high readings because the low readings cannot be right.  The technicians continue to take additional readings, noting many of them to be far too hot, and log these readings into a book that the instrument engineer ignores.  Soon afterwards the technicians are ordered out of the control room along with all ``non-essential'' personnel, and the log book is not noticed by anyone until weeks later!

The technicians would have connected sensitive voltmeters (probably vacuum-tube voltmeters, given the 1970's vintage of their test equipment) in parallel with the thermocouple wires, noted the millivoltage readings, then added the reference junction millivoltage based on ambient temperature conditions to obtain the measurement junction millivoltage.  This measurement junction millivoltage could then be correlated to reactor core temperature by consulting the appropriate thermocouple table.

\vskip 10pt

\noindent
{\bf Pages 42-43}

At some time there was a pressure transient recorded in the containment building: a chart recorder connected to a pressure transmitter reports the ambient pressure rising sharply from 0 (atmospheric) to 28 PSI and ramping down to zero over a span of about a minute.  Operators dismiss this indication as an ``electrical transient'' rather than recognize it as a sign that hydrogen gas exploded inside the containment building.  To the operators, it was so inconceivable that a 2 million cubic foot building could pressurize and depressurize so rapidly, that they concluded the spike must have been an instrument error.  Two of the shift supervisors concluded that the pressure spike must have been real because it activated the building spray system which requires coincident signals from a 2oo3 pressure sensing system.  For whatever reason, however, neither of these mens' conclusions were successfully communicated to their superiors.

A very insightful quote from operator Ed Frederick explains why the pressure transient was dismissed ({\it emphasis mine}):

\vskip 10pt {\narrower \noindent \baselineskip5pt

``Based on our training, it was impossible . . . if you look back through everybody's training and the FSAR and safety analysis and the building construction, you will not see a paragraph that projects that type of transient.  {\it (It) is so particularly foreign and unbelievable that it has absolutely no significance.}  That's why nobody did anything about it for two days.''

\par} \vskip 10pt

This striking admission underscores the importance of technical personnel {\it understanding the physical principles by which their systems operate} as opposed to merely memorizing procedures.











\vfil \eject

\noindent
{\bf Prep Quiz:}

What prevents water inside the nuclear reactor at Three Mile Island from boiling, even though it operates at an extremely high temperature?

\begin{itemize}
\item{} The water is pressurized to prevent boiling at that temperature
\vskip 5pt 
\item{} High temperatures do not exist long enough to allow boiling
\vskip 5pt 
\item{} Radiation from the reactor core inhibits the formation of steam bubbles
\vskip 5pt 
\item{} Thermocouples act to condense steam bubbles back into water
\vskip 5pt 
\item{} Control rods in the reactor decrease the water's vapor pressure
\vskip 5pt 
\item{} Boric acid added to the water elevates its boiling temperature
\end{itemize}

%INDEX% Physics, heat and temperature: steam table
%INDEX% Reading assignment: Three Mile Island -- A Report to the Commissioners and to the Public

%(END_NOTES)


