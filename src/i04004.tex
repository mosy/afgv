
%(BEGIN_QUESTION)
% Copyright 2009, Tony R. Kuphaldt, released under the Creative Commons Attribution License (v 1.0)
% This means you may do almost anything with this work of mine, so long as you give me proper credit

Read and outline the ``Temperature Switches'' section of the ``Discrete Process Measurement'' chapter in your {\it Lessons In Industrial Instrumentation} textbook.  Note the page numbers where important illustrations, photographs, equations, tables, and other relevant details are found.  Prepare to thoughtfully discuss with your instructor and classmates the concepts and examples explored in this reading.

\underbar{file i04004}
%(END_QUESTION)





%(BEGIN_ANSWER)


%(END_ANSWER)





%(BEGIN_NOTES)

We would expect a NO temperature switch to be ``open'' when it is {\it cold} (below the temperature of its ``trip'' value).  Note that this may or may not be ambient (room) temperature, depending on the setting of the temperature switch.

\vskip 10pt

The {\it deadband} or {\it differential} of a process switch is the hysteresis it exhibits for rising versus falling stimuli.  Switches with adjustable differential require calibration checks where the stimulus is forced to rise and then fall, so that the hysteresis may be accurately measured and adjusted if necessary.

\vskip 10pt

Electronic temperature switch modules input RTD, thermocouple, or 4-20 mA analog signals, and output a switch contact state where the trip point and deadband may be precisely set.









\vskip 20pt \vbox{\hrule \hbox{\strut \vrule{} {\bf Suggestions for Socratic discussion} \vrule} \hrule}

\begin{itemize}
\item{} Define ``normal'' for a temperature switch, as the term is used for NO and NC switch contacts.
\item{} Explain what ``deadband'' or ``differential'' means for a process switch, and why it is important.
\item{} Explain why some deadband is good to have in a process alarm switch.
\item{} Identify some advantages of an {\it electronic} temperature switch over a {\it mechanical} temperature switch.
\item{} Describe a calibration procedure you would use for a temperature switch possessing deadband (e.g. a high-temperature mechanical alarm switch with a trip point of 300 $^{o}$F).
\end{itemize}

%INDEX% Reading assignment: Lessons In Industrial Instrumentation, Temperature Switches

%(END_NOTES)


