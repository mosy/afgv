
%(BEGIN_QUESTION)
% Copyright 2009, Tony R. Kuphaldt, released under the Creative Commons Attribution License (v 1.0)
% This means you may do almost anything with this work of mine, so long as you give me proper credit

Read and outline the ``Temperature Standards'' subsection of the ``Practical Calibration Standards'' section of the ``Instrument Calibration'' chapter in your {\it Lessons In Industrial Instrumentation} textbook.  Note the page numbers where important illustrations, photographs, equations, tables, and other relevant details are found.  Prepare to thoughtfully discuss with your instructor and classmates the concepts and examples explored in this reading.

\underbar{file i04006}
%(END_QUESTION)





%(BEGIN_ANSWER)


%(END_ANSWER)





%(BEGIN_NOTES)

RTD and thermocouple instruments may be calibrated using resistance and voltage standards, like any other electrical instrument.

\vskip 10pt

Actual temperatures may be created in a laboratory environment for testing non-electrical and electrical temperature sensors alike, based on the phase change points of various substances (e.g. water at freezing and boiling).  The concept here is that substances undergoing a phase change are thermodynamically limited in terms of pressure and/or temperature.  Water at 14.7 PSIA, for example, can only boil at 212 degrees F.  Water at its triple point (solid + liquid + vapor) exists only at 0.006 atmospheres pressure and 0.01 degrees Celsius.  The latent heat required to complete the phase change becomes the limit to whether or not ambient temperature can affect the sample's temperature.

\vskip 10pt

Oil bath and sand bath units create uniform temperatures which may be used to test a sensor such as an RTD or thermocouple.  The temperature standard here is another (trusted) sensor placed in the same bath.  Dry-block temperature calibrators work much the same, except the hot object is a block of metal with blind holes drilled in the face to accept sensor probes.

\vskip 10pt

A ``blackbody calibrator'' is an emittive surface used to generate optical rediation that a non-contact pyrometer may sense.











\vskip 20pt \vbox{\hrule \hbox{\strut \vrule{} {\bf Suggestions for Socratic discussion} \vrule} \hrule}

\begin{itemize}
\item{} Identify a case where someone uses a highly precise electronic standard to calibrate an RTD or thermocouple transmitter, but the accuracy of the installed instrument is compromised by some {\it other} factor not accounted for in the electronic calibration procedure.  What are some of these ``other'' factors?
\item{} Reference a {\it phase diagram} for water and identify points on that diagram where water may be used as a temperature standard.
\item{} Suppose an ice-water mixture is being used as a temperature standard for calibration.  If the room temperature control thermostat's setpoint is increased and the room warms up as a result, what will happen to the ice-water mixture?
\item{} Describe what you would have to do in order to set up a {\it triple point} cell for water in our lab.
\item{} Why is a calibrator used for non-contact pyrometers called a {\it blackbody} calibrator?
\item{} Closely examine the photograph of the Fluke 525A laboratory standard, and identify what kinds of temperature-based calibrations we could do with this machine.
\item{} Examine the photograph of the {\it oil bath} temperature calibrator, and explain how you would use one to check the calibration of a temperature instrument.
\item{} Examine the photograph of the {\it dry block} temperature calibrator, and explain how you would use one to check the calibration of a temperature instrument.
\end{itemize}

%INDEX% Reading assignment: Lessons In Industrial Instrumentation, Temperature Calibration Standards

%(END_NOTES)


