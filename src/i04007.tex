
%(BEGIN_QUESTION)
% Copyright 2009, Tony R. Kuphaldt, released under the Creative Commons Attribution License (v 1.0)
% This means you may do almost anything with this work of mine, so long as you give me proper credit

Read and outline the ``Temperature Sensor Accessories'' section of the ``Continuous Temperature Measurement'' chapter in your {\it Lessons In Industrial Instrumentation} textbook.  Note the page numbers where important illustrations, photographs, equations, tables, and other relevant details are found.  Prepare to thoughtfully discuss with your instructor and classmates the concepts and examples explored in this reading.

\underbar{file i04007}
%(END_QUESTION)





%(BEGIN_ANSWER)


%(END_ANSWER)





%(BEGIN_NOTES)

A {\it thermowell} is a pressure-tight ``sock'' allowing a sensor to be removed from a process without creating an open hole in the vessel or pipe.  

\vskip 10pt

The mass of a thermowell will slow down the response time of a sensor inserted into it.  This effect is exacerbated if the sensor is not {\it fully} inserted (i.e. contacting the bottom of the thermowell's blind hole).










\vskip 20pt \vbox{\hrule \hbox{\strut \vrule{} {\bf Suggestions for Socratic discussion} \vrule} \hrule}

\begin{itemize}
\item{} Explain the purpose of a {\it thermowell} in a temperature measurement system.
\item{} Identify potential problems with using thermowells to protect sensing elements such as thermocouples and RTDs.
\item{} Suppose you were tasked with selecting a suitable thermowell for a particular process application.  Identify some of the relevant selection criteria dictating your choice of thermowells.
\end{itemize}



%INDEX% Reading assignment: Lessons In Industrial Instrumentation, Temperature Sensor Accessories

%(END_NOTES)


