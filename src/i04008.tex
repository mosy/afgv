
%(BEGIN_QUESTION)
% Copyright 2009, Tony R. Kuphaldt, released under the Creative Commons Attribution License (v 1.0)
% This means you may do almost anything with this work of mine, so long as you give me proper credit

Read the datasheet for ``The WORM'' flexible temperature sensor marketed by Moore Industries and answer the following questions:

\vskip 10pt

Explain how this sensor design ensures good thermal contact with the thermowell.

\vskip 10pt

Identify some of the sensor types available in this product line.

\vskip 10pt

Note which specific types of thermocouple sensor offered in this sensor respond fastest to changes in process temperature, and explain why.

\vskip 10pt

Note the accuracy specified for RTD sensors, and compare this to the accuracy specifications you've seen for thermocouples.

\underbar{file i04008}
%(END_QUESTION)





%(BEGIN_ANSWER)

The WORM uses a long coil spring to maintain positive pressure contact between the sensor and the bottom of the thermowell.

\vskip 10pt

Available sensor types include type J and K thermocouples, as well as 100 ohm and 1000 ohm platinum RTDs (both having $\alpha$ values of 0.00385) -- shown on page 4.

\vskip 10pt

Fastest thermocouple is the grounded-tip style, with a time constant of 2 seconds (page 4).  By contrast, ungrounded thermocouples exhibit a typical time constant of 4.5 seconds.

\vskip 10pt

RTD accuracy is stated as $\pm$ 0.12\% at 0 $^{o}$C (page 4).  Typical thermocouple accuracies range plus or minus a couple of degrees Celsius.


%(END_ANSWER)





%(BEGIN_NOTES)


%INDEX% Reading assignment: Moore Industries datasheet for ``The WORM'' flexible sensor

%(END_NOTES)


