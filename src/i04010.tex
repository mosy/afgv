
%(BEGIN_QUESTION)
% Copyright 2009, Tony R. Kuphaldt, released under the Creative Commons Attribution License (v 1.0)
% This means you may do almost anything with this work of mine, so long as you give me proper credit

Read portions of the Rosemount datasheet for high-temperature thermocouple assemblies (document 00813-0401-2654) and answer the following questions:

\vskip 10pt

Identify some of the different materials used in the construction of ``protective tubes'' (thermowells) offered by Rosemount for high-temperature measurement applications.

\vskip 10pt

Ceramic thermowells may be damaged by a phenomenon called {\it thermal shock}.  Explain what ``thermal shock'' is, how it may occur during thermocouple installation, and what precaution(s) to take to avoid this.

\vskip 10pt

What are some of the factors to consider when selecting a thermowell material for a particular process application?

\vskip 10pt

Explain what a {\it multipoint gradient} thermocouple assembly is, and what one might be used for.

\vskip 10pt

Explain what {\it limit tolerance class} refers to, according to the DIN EN 60584-2 standard.  Which class has a tighter tolerance, Class 1 or Class 2?

\vskip 20pt \vbox{\hrule \hbox{\strut \vrule{} {\bf Suggestions for Socratic discussion} \vrule} \hrule}

\begin{itemize}
\item{} Why should thermowells in high-temperature applications be inserted {\it vertically} rather than {\it horizontally} whenever possible?
\item{} A section on page 6 of this document explains why type B thermocouples (where both wires are a platinum-rhodium alloy) tend to be more stable over time than type S or R thermocouples which both use pure platinum for one of their wires.  Re-phrase this explanation in your own words, articulating why type B thermocouples are more stable than types R and S.
\end{itemize}

\underbar{file i04010}
%(END_QUESTION)





%(BEGIN_ANSWER)


%(END_ANSWER)





%(BEGIN_NOTES)

Tube materials: ceramic (alumina), stainless steel, Monel, Kanthal, silicon carbide (SiC), Stellite, tantalum, titanium, and Hastelloy.  Table 4 on page 10 shows different thermowell material types.  These seven different materials are also listed in a paragraph on page 4.

\vskip 10pt

Quickly inserting a cold thermocouple into a hot tube may cause thermal shock.  For this reason, new thermocouples should be {\it slowly inserted} into hot ceramic thermowells (page 7).

\vskip 10pt

Resistance to types of chemical attack, and temperature range are the two major considerations for thermowell material.  Table 4 shown on page 10 of this document lists several different thermowell materials and their areas of suitability/unsuitability.

\vskip 10pt

A ``multipoint gradient'' assembly is one where multiple thermocouples fit into a single thermowell, at different depths.  The purpose of this is to provide temperature measurements at different depths within the process fluid, particularly for molten glass applications where temperature control is crucial to product quality.  Pages 30-33.

\vskip 10pt

Tolerance class 2 is ``looser'' (i.e. larger percentage of allowable error) than tolerance class 1 as explained in the last paragraph on page 51.  Table 2 on page 8 shows the different tolerance classes and thermocouple types, where class 1 is $\pm$ 1.5 $^{o}$C and class 2 is $\pm$ 2.5 $^{o}$C.













\vskip 20pt \vbox{\hrule \hbox{\strut \vrule{} {\bf Suggestions for Socratic discussion} \vrule} \hrule}

\begin{itemize}
\item{} Choose a good metal thermowell type for an application containing a high concentration of sulphurous gases ({\bf 1.4749} ; {\bf Kanthal AF 1.4767})
\item{} Choose a good metal thermowell type for an application containing a high concentration of nitrogenous gases ({\bf 1.4841})
\item{} Choose a ceramic thermowell type for an application operating at 1750 $^{o}$C ({\bf Type C 799}, 99.7\% Al$_{2}$O$_{3}$)
\end{itemize}

%INDEX% Reading assignment: Rosemount datasheet for high-temperature thermocouples and thermowells

%(END_NOTES)


