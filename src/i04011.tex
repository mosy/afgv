
%(BEGIN_QUESTION)
% Copyright 2015, Tony R. Kuphaldt, released under the Creative Commons Attribution License (v 1.0)
% This means you may do almost anything with this work of mine, so long as you give me proper credit

Read selected portions of the US Chemical Safety and Hazard Investigation Board's analysis of the 1998 chemical manufacturing incident at the Morton International manufacturing facility in Paterson, New Jersey (Report number 1998-06-I-NJ), and answer the following questions:

\vskip 10pt

Based on the incident summary and key findings presented on pages 1 through 5, summarize how the process is supposed to work and then describe what went wrong to produce the explosion.

\vskip 10pt

A graph on page 31 of the report contrasts {\it heat production} of the chemical reaction versus {\it heat removal} of the process cooling system.  Identify where the ``danger'' point is on this graph, and explain why it is dangerous based on your knowledge of specific heat and heat transfer.

\vskip 10pt

The chemical reactions involved in this process were primarily {\it exothermic}.  Explain what this term means, and why it is important to the cause of this accident.

\vskip 20pt \vbox{\hrule \hbox{\strut \vrule{} {\bf Suggestions for Socratic discussion} \vrule} \hrule}

\begin{itemize}
\item{} Identify some of the discrepancies found in Morton's MSDS datasheet for the Yellow 96 dye product (page 3) and explain how MSDS datasheets (more commonly known now as SDS datasheets) are generally useful.
\item{} Explain why the ``heat removal'' graph is a linear function, based on your knowledge of heat transfer equations.
\item{} Pick any point on the ``heat removal'' graph and explain how the temperature of the kettle will naturally proceed from that temperature, based on a comparison of heat generation versus heat removal rates.  Identify the {\it two} points on this graph where the temperature will settle at an equilibrium value.
\item{} Identify the shape of the ``heat removal'' graph if the dominant mode of heat transfer were {\it radiation} instead of conduction/convection.
\end{itemize}

\underbar{file i04011}
%(END_QUESTION)





%(BEGIN_ANSWER)


%(END_ANSWER)





%(BEGIN_NOTES)

Morton International produced dyes used to color petroleum fuels, one of those dyes being ``Yellow 96.''  This dye was produced by a chemical reaction of two feed materials in a reactor vessel called a {\it kettle}: ortho-nitrochlorobenzene ({\it o}-NCB) and 2-ethylhexylamine (2-EHA).  These feedstocks react exothermically to produce Yellow 96 dye at normal operating temperatures (150 $^{o}$C to 153 $^{o}$C).  At elevated temperatures (195 $^{o}$C), the Yellow 96 dye begins to exothermically decompose.  This two-stage exothermic behavior caused a ``runaway'' chemical reaction that ruptured the pressure safety devices and eventually the process piping itself.  9 employees were injured, 2 of them serious enough to require extended hospitalization.

Three major factors contributed to the event: higher-than-normal starting temperature, heating steam left on for too long, and cooling water not applied soon enough (page 2).

\vskip 10pt

Morton's MSDS incorrectly categorized the reactivity of Yellow 96 dye as 0 (it should have been 1), and placed its boiling temperature at 100 $^{o}$C (it should have been 330 $^{o}$C).  Morton had scaled the process up from a semi-batch operation tested in the lab (where {\it o}-NCB was incrementally added to a full measure of 2-EHA) to a full batch operation where both reactants were mixed in full and then heated (which is more difficult to control).  The process reactor was not equipped with any automatic safety apparatus, and was manually controlled.  Reactor size and batch size were increased with no engineering assessment of safety hazards.

\vskip 10pt

The graph on page 31 shows a point (labeled with the letter ``A'') where heat production from the exothermic reaction exceeds the heat removal capacity of the cooling system.  At this point, temperature will continue to rise unchecked, which is what happened that day to cause the explosion.

\vskip 10pt

Exothermic chemical reactions release energy (usually in the form of heat).  This was first defined in the report within footnote \#2 on page 2.  Fire is a very common example of an exothermic reaction.













\vfil \eject

\noindent
{\bf Prep Quiz:}

Identify a contributing cause for the explosion of a chemical processing reactor at Morton International in 1998:

\begin{itemize}
\item{} Incorrect chemicals introduced into the reactor from the start
\vskip 5pt 
\item{} A pressure relief valve failed in the ``shut'' position
\vskip 5pt 
\item{} Cooling water turned on too late in the processing sequence
\vskip 5pt 
\item{} The reactor safety shutdown system was left in a disabled state
\vskip 5pt 
\item{} An electrical spark from a sensor triggered the explosion
\vskip 5pt 
\item{} A pipe flange had been left loose, leaking volatile chemicals
\end{itemize}

%INDEX% Reading assignment: USCSB Accident Report, Chemical Manufacturing Incident at Morton International in Paterson, New Jersey

%(END_NOTES)


