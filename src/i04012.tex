
%(BEGIN_QUESTION)
% Copyright 2009, Tony R. Kuphaldt, released under the Creative Commons Attribution License (v 1.0)
% This means you may do almost anything with this work of mine, so long as you give me proper credit

Read selected portions of the US Chemical Safety and Hazard Investigation Board's analysis of the 1998 chemical manufacturing incident at the Morton International manufacturing facility in Paterson, New Jersey (Report number 1998-06-I-NJ), and answer the following questions:

\vskip 10pt

Pages 14 and 16 of the report describe the construction of the ``kettle'' batch process used by Morton to produce ``Yellow 96'' dye.  Page 16 in particular shows a simplified P\&ID of the batch process.  Based on what you find in this section of the report, identify and explain all the modes of heat addition to and heat removal from the process vessel.  Also identify all measurement instrumentation for the kettle.

\vskip 10pt

An important factor leading to this event was a failure to heed established {\it Management of Change} (MOC) procedures, as described on page 7, on pages 45-46, and also on pages 57-58.  Explain what ``Management of Change'' refers to and why it is important for process safety.

\vskip 10pt

Supposing the kettle was heated by a flow of saturated steam at 15 PSIG boiler pressure at a mass flow rate of 3 pounds per minute and cooling to become condensate (water) at atmospheric pressure and 200 degrees F, calculate the amount of heat transferred to the kettle by this steam, in units of BTU per minute.  

\vskip 10pt

Hint: the {\it Socratic Instrumentation} website contains a page where you may download public-domain textbooks, one of which is a set of steam tables published in 1920.  The {\it Fisher Control Valve Handbook} also has a (less comprehensive) set of steam tables in the Appendix section.



\vskip 20pt \vbox{\hrule \hbox{\strut \vrule{} {\bf Suggestions for Socratic discussion} \vrule} \hrule}

\begin{itemize}
\item{} An important safety policy at many industrial facilities is something called {\it stop-work authority}, which means any employee has the right to stop work they question as unsafe.  Explain how stop-work authority could have been applied to this particular incident.
\item{} Explain why the transition from a 1000 gallon kettle to a 2000 gallon kettle resulted in diminished heat-removal capacity, based on what you know about thermodynamics.
\item{} If the pressure of the saturated steam used to heat the kettle increased, would this result in more heat delivered to the kettle, less heat delivered to the kettle, or the same amount of heat delivered to the kettle?
\item{} Do you suppose most of the heat transferred in the ``condenser'' took place via {\it specific} (sensible) heat or via {\it latent} heat?
\end{itemize}

\underbar{file i04012}
%(END_QUESTION)





%(BEGIN_ANSWER)

Heat transfer rate to kettle (assuming quantities given here in the question, not anywhere in the USCSB report) is equal to 2987.1 BTU/minute.

%(END_ANSWER)





%(BEGIN_NOTES)

Heat could be added (with steam) through a jacket surrounding the kettle.  Heat was removed by cooling water through that same jacket, as well as via a heat exchanger using water to condense product vapors.  More heat exited the kettle via the mass of hot vapor vented from the kettle and into the thermal oxidizer header.  The only instrumentation on the batch process was kettle temperature, kettle pressure, and cooling water pressure (the latter not shown on the P\&ID).

\vskip 10pt

{\it Management of Change} refers to a re-evaluation of process safety hazards following a change in process equipment or procedures.  Morton scaled up its laboratory prototype process from semi-batch to full-batch production without without an engineering review of the hazards (pp. 4, 6), increasing kettle size from 1000 gallons to 2000 gallons, then increased batch size by 9\% in 1996 (page 7) all without safety engineering review.  The kettle change plus the batch size change resulted in about 10\% less heat-transfer surface area for cooling (page 45).  Half of the batches made after the volume scale-ups exhibited high temperature excursions, compared with only 20\% of the batches made before the alterations (page 58).

\vskip 10pt

Saturated steam tables give the enthalpy of 15 PSIG steam as 1163.7 BTU/lb.  Subtracting the enthalpy of condensate at 200 degrees F (200 $-$ 32 = 168 BTU/lb), we get a difference of 995.7 BTU delivered to the kettle per pound of steam.  With a mass flow rate of 3 pounds per minute, this equates to 2987.1 BTU/min.

%INDEX% Physics, heat and temperature: steam table
%INDEX% Reading assignment: USCSB Accident Report, Chemical Manufacturing Incident at Morton International in Paterson, New Jersey

%(END_NOTES)


