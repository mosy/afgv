
%(BEGIN_QUESTION)
% Copyright 2009, Tony R. Kuphaldt, released under the Creative Commons Attribution License (v 1.0)
% This means you may do almost anything with this work of mine, so long as you give me proper credit

Read portions of the Moore Products model 33 ``Nullmatic'' pneumatic temperature transmitter service manual and answer the following questions:

\vskip 10pt

Identify the upper operating temperature limit of this instrument, and compare this against common thermocouples.

\vskip 10pt

Describe a typical calibration procedure for this transmitter.

\vskip 10pt

Determine whether this is a {\it force-balance} or a {\it motion-balance} instrument, based on an examination of its cut-away diagram.

\vskip 20pt \vbox{\hrule \hbox{\strut \vrule{} {\bf Suggestions for Socratic discussion} \vrule} \hrule}

\begin{itemize}
\item{} Explain how the ``zero'' adjustment works in this temperature transmitter.
\item{} Explain how the span of this temperature transmitter could be altered.
\item{} Identify at least one fault that could cause the pneumatic output of this instrument to saturate low (below 3 PSI)
\item{} Identify at least one fault that could cause the pneumatic output of this instrument to saturate high (above 15 PSI)
\item{} If the ambient temperature of this instrument increases, what effect (if any) will it have on the output signal?
\item{} If you were to classify this filled-bulb instrument by number, which one would it be (e.g. Class I, II, III, V)?
\end{itemize}

\underbar{file i04016}
%(END_QUESTION)





%(BEGIN_ANSWER)

\noindent
{\bf Partial answer:}

\vskip 10pt

This is a direct {\it force-balance} instrument.

%(END_ANSWER)





%(BEGIN_NOTES)

The upper temperature limit for the model 33 transmitter is only 1400 $^{o}$F, with a maximum calibration span of 1000 $^{o}$F (for the ``A'' model).  This is on par with type J thermocouples, and hotter than type T thermocouples.  Most other thermocouple types, however, can measure temperatures greater than 1400 $^{o}$F.

\vskip 10pt

Calibration must be done by placing the sensing bulb inside an actual source of heat (e.g. sand bath, oil bath, dry block) and comparing its response against that of a trusted sensor inside the same heat source.

\vskip 10pt

This is a Class III instrument, with helium being the fill gas of choice.





\vfil \eject

\noindent
{\bf Prep Quiz:}

The Moore Products model 33 Nullmatic temperature transmitter uses which type of sensor to detect process temperature:

\begin{itemize}
\item{} Thermocouple
\vskip 5pt 
\item{} Bimetallic strip
\vskip 5pt 
\item{} RTD
\vskip 5pt 
\item{} Color-changing dye
\vskip 5pt 
\item{} Fluid-filled bulb
\vskip 5pt 
\item{} Pyro, the magical elf
\end{itemize}


%INDEX% Reading assignment: Moore Products service manual for model 33 ``Nullmatic'' pneumatic temperature transmitter

%(END_NOTES)


