
%(BEGIN_QUESTION)
% Copyright 2009, Tony R. Kuphaldt, released under the Creative Commons Attribution License (v 1.0)
% This means you may do almost anything with this work of mine, so long as you give me proper credit

Read and outline the ``Non-Contact Temperature Sensors'' section of the ``Continuous Temperature Measurement'' chapter in your {\it Lessons In Industrial Instrumentation} textbook.  Note the page numbers where important illustrations, photographs, equations, tables, and other relevant details are found.  Prepare to thoughtfully discuss with your instructor and classmates the concepts and examples explored in this reading.

\underbar{file i04017}
%(END_QUESTION)





%(BEGIN_ANSWER)


%(END_ANSWER)





%(BEGIN_NOTES)

Masses above absolute zero temperature emit electromagnetic radiation according to the following formula:

$${dQ \over dt} = e \sigma A T^4$$

\noindent
Where,

$dQ \over dt$ = Radiant heat loss rate (watts)

$e$ = Emissivity factor (unitless, varies with target composition)

$\sigma$ = Stefan-Boltzmann constant (5.67 $\times$ $10^{-8}$ W / m$^{2}$ $\cdot$ K$^{4}$)

$A$ = Surface area (square meters)

$T$ = Absolute temperature (Kelvin)

\vskip 10pt

Temperature detected by this means is strictly limited to the {\it surface} temperature of the object, not the temperature within that object.  This principle applies to thermal imaging as well: the images obtained from an infrared camera only tell you variations in surface temperature, not variations in interior temperature.

\vskip 10pt

Pyrometers use some form of concentrating optics (mirrors, lenses) to concentrate the incident light on a thermopile (a series collection of thermocouple junctions).  The ``hot'' junctions are placed together at the focal point, while the ``cold'' junctions are located where they can sense ambient temperature.

\vskip 10pt

An {\it infrared thermocouple} is a pyrometer where the thermopile voltage is read as though it were a regular thermocouple.  Due to the inherent nonlinearity of non-contact pyrometry, this only works over modest temperature ranges (e.g. spans of 100 $^{o}$F or less).

\vskip 10pt

The fourth power term makes non-contact pyrometry very nonlinear, and therefore limited to relatively narrow ranges of temperature measurement where good accuracy is desired.  If we take the ratio of new versus old temperature (in absolute degrees) and raise that ratio to the fourth power, we should get the approximate ratio of thermopile millivoltages for those two temperatures.

\vskip 10pt

The {\it inverse square law} of radiation states that the intensity of radiation flux drops off proportional to the square law of distance from a point-source, due to simple geometry (the surface area of a sphere increasing with the square of radius).  This means a non-contact pyrometer will suffer from a distance-dependent calibration shift if sensing a thermal point-source.  However, if the object being sensed is not a point-source -- like most objects -- and completely fills the optical viewing field of the pyrometer, there will be absolutely no calibration shift due to distance.  As viewing distance increases, the radiation intensity from any one point on the wall does indeed decrease with the square of distance, but the viewing area has similarly increased with the square of distance, perfectly canceling the effect of the inverse square law.

\vskip 10pt

The {\it field of view} for a non-contact sensor may be expressed as a ratio of distance over field width, or as a viewing angle.

\vskip 10pt

While emissivity ($e$) varies with the chemical composition of the target, factors such as surface texture and shape also have an effect.  The total impact of the target object on radiation received by a pyrometer is called {\it emittance}.  Like emissivity, emittance varies between 1 (perfect ``blackbody'') and 0 (non-radiating).  If the emittance of an object is known, the pyrometer may be calibrated to reflect that factor.

\vskip 10pt

Objects may also reflect and transmit radiation received from other objects (e.g. like a mirror, or clear gas, respectively).

\vskip 10pt

Thermal imaging uses an array of optical sensors to generate color-enhanced images showing the temperature profile of any view.  Red colors typically represent hotter while blue colors represent colder.










\filbreak

\vskip 20pt \vbox{\hrule \hbox{\strut \vrule{} {\bf Suggestions for Socratic discussion} \vrule} \hrule}

\begin{itemize}
\item{} Do optical pyrometers using a thermopile as the sensing element require cold junction compensation (reference junction compensation)?  Why or why not?
\item{} Suppose an optical pyrometer's thermopile outputs 4.6 mV at a target temperature of 500 $^{o}$C.  What will the output voltage be at a temperature of 700 $^{o}$C?  {\bf Answer = 11.55 mV}
\item{} Suppose an optical pyrometer's thermopile outputs 9.1 mV at a target temperature of 800 $^{o}$C.  What will the output voltage be at a temperature of 600 $^{o}$C?  {\bf Answer = 3.988 mV}
\item{} Explain what the {\it inverse square law} means with reference to radiation.
\item{} Suppose you are being exposed to three times the amount of radiation deemed safe for a day's exposure from a small piece of radioactive material placed 5 feet away from you.  How far away must you back up from this source in order to reduce the exposure to the day-safe level?  {\bf Answer = 8.66 feet}
\item{} Explain why the inverse square law may be ignored for practical applications of non-contact pyrometers.
\item{} Identify the condition which must be met for a non-contact pyrometer to read accurately with no regard for distance.
\item{} Explain what a {\it blackbody} is, and what this concept refers to in optical pyrometry.
\item{} Could an optical pyrometer be used to measure the temperature of a nuclear reactor core?  Why or why not?
\item{} Visit {\tt http://www.omega.com} to investigate some of the infrared thermocouples sold by that company.  Specifically, pay attention to the thermocouple type (e.g. J, K, etc.) as well as the linear measurement range of each model.
\item{} Some hand-held infrared pyrometers are equipped with a laser pointer to identify what it is you're sensing.  A common misconception is that you will be measuring only the temperature of whatever is touched by that laser-beam dot.  Explain why this misconception is incorrect, and propose a more accurate conceptual view of what a hand-held infrared pyrometer is sensing.
\item{} Identify some practical uses for an {\it infrared thermal imaging camera}.
\item{} If you have access to an infrared thermometer (or better yet, an IR thermal imaging camera), let your students perform the following experiments:
\itemitem{} Locate studs in an exterior wall by differences in temperature
\itemitem{} Try to measure the temperature of a person's eyeball (not if there is a laser pointer!)
\itemitem{} Try to measure the temperature of a person's eyeball behind glasses (not if there is a laser pointer!)
\itemitem{} Try to measure the temperature of a mirror
\itemitem{} Heat a light-colored object in a microwave oven (not metal!), then place a piece of black electrical tape on that object and try optically measuring temperature on the light-colored surface versus on the black tape's surface.  The two temperatures are in fact the same, but the pyrometer will register differently due to the different emittance values of the light-colored object and the black tape!
\end{itemize}












\vfil \eject

\noindent
{\bf Prep Quiz:}

Suppose an infra-red pyrometer is accurately measuring the temperature of a hot object, and then the hot object moves toward to the pyrometer so that it is only {\it half} as far away as it was before.  How will this change in distance affect the pyrometer's measurement of temperature, assuming the object's actual temperature is the same now as it was before?

\begin{itemize}
\item{} The pyrometer will register {\it one-sixteenth} the absolute temperature as before 
\vskip 5pt 
\item{} The pyrometer will register {\it one-quarter} the absolute temperature as before
\vskip 5pt 
\item{} The pyrometer will register the {\it same} absolute temperature as before 
\vskip 5pt 
\item{} The pyrometer will register {\it twice} the absolute temperature as before 
\vskip 5pt 
\item{} The pyrometer will register {\it four times} the absolute temperature as before 
\vskip 5pt 
\item{} The pyrometer will register {\it sixteen times} the absolute temperature as before 
\end{itemize}

\vfil \eject

\noindent
{\bf Prep Quiz:}

The key to ensuring changes in distance between an optical (non-contact) pyrometer and the hot object it measures do not affect the accuracy of measurement is to:

\begin{itemize}
\item{} Ensure the object will always fill the pyrometer's field of view
\vskip 5pt 
\item{} Choose a pyrometer with the correct range of color detection 
\vskip 5pt 
\item{} Place an infra-red filter on the pyrometer to screen out interference
\vskip 5pt 
\item{} Calibrate the pyrometer for the {\it emissivity} value of that hot object
\vskip 5pt 
\item{} Connect an ``ice-point'' reference junction compensation circuit to it
\vskip 5pt 
\item{} Use a computer to mathematically compensate for changes in distance
\end{itemize}


%INDEX% Reading assignment: Lessons In Industrial Instrumentation, Continuous Temperature Measurement (non-contact sensors)

%(END_NOTES)


