
%(BEGIN_QUESTION)
% Copyright 2009, Tony R. Kuphaldt, released under the Creative Commons Attribution License (v 1.0)
% This means you may do almost anything with this work of mine, so long as you give me proper credit

Skim the ``Continuous Fluid Flow Measurement'' chapter in your {\it Lessons In Industrial Instrumentation} textbook to specifically answer these questions:

\vskip 10pt

Two common styles of ``variable area'' flowmeter are {\it rotameters} and {\it weirs}.  Explain how each one of these flowmeter types functions, and where (exactly) is the flow-path area varying as flow rate increases and decreases.

\vskip 10pt

Describe the difference between a {\it weir} and a {\it flume}, and identify a practical application in industry where either one of these flow elements might be used.


\vskip 20pt \vbox{\hrule \hbox{\strut \vrule{} {\bf Suggestions for Socratic discussion} \vrule} \hrule}

\begin{itemize}
\item{} Identify different strategies for ``skimming'' a text, as opposed to reading that text closely.  Why do you suppose the ability to quickly scan a text is important in this career?
\end{itemize}

\underbar{file i04021}
%(END_QUESTION)





%(BEGIN_ANSWER)


%(END_ANSWER)





%(BEGIN_NOTES)

Rotameters achieve a flow-varying area by using a tapered tube in which a plummet rises and falls.  As the plummet rises, the tube ``opens up'' around it to yield a greater area for fluid to pass by.

\vskip 10pt

Weirs and flumes are open-channel restrictions, where the area fluid flows through increases as fluid height increases in the throat of the device.

\vskip 10pt

Weirs are like dams over which liquid must spill.  They are analogous to orifice plates in pipes.

\vskip 10pt

Fluke are like narrow rapids in a river, through which liquid must rush.  They are analogous to venturi tubes.



%INDEX% Reading assignment: Lessons In Industrial Instrumentation, Continuous Fluid Flow Measurement (variable-area)

%(END_NOTES)


