
%(BEGIN_QUESTION)
% Copyright 2009, Tony R. Kuphaldt, released under the Creative Commons Attribution License (v 1.0)
% This means you may do almost anything with this work of mine, so long as you give me proper credit

Skim the ``Continuous Fluid Flow Measurement'' chapter in your {\it Lessons In Industrial Instrumentation} textbook to specifically answer these questions:

\vskip 10pt

One of the stranger flow-measurement technologies is the {\it Coriolis} flowmeter.  Do your best to explain how this type of flowmeter functions. 


\vskip 20pt \vbox{\hrule \hbox{\strut \vrule{} {\bf Suggestions for Socratic discussion} \vrule} \hrule}

\begin{itemize}
\item{} Identify different strategies for ``skimming'' a text, as opposed to reading that text closely.  Why do you suppose the ability to quickly scan a text is important in this career?
\end{itemize}

\underbar{file i04023}
%(END_QUESTION)





%(BEGIN_ANSWER)


%(END_ANSWER)





%(BEGIN_NOTES)

Coriolis flowmeters pass liquid through a flexible, vibrating tube.  The resonant frequency of this vibrating tube varies with fluid density, while the inertia of the flowing fluid causes the tube to undulate (induces a phase shift).  By sensing the resonant frequency of the tubes' vibrations, the instrument infers fluid density.  By sensing the phase shift between one portion of the tube versus another, the instrument infers mass flow rate.


%INDEX% Reading assignment: Lessons In Industrial Instrumentation, Continuous Fluid Flow Measurement (inertial)

%(END_NOTES)


