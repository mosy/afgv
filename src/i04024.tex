
%(BEGIN_QUESTION)
% Copyright 2009, Tony R. Kuphaldt, released under the Creative Commons Attribution License (v 1.0)
% This means you may do almost anything with this work of mine, so long as you give me proper credit

Skim the ``Continuous Fluid Flow Measurement'' chapter in your {\it Lessons In Industrial Instrumentation} textbook to specifically answer these questions:

\vskip 10pt

A cold day ``feels colder'' if a strong wind is blowing.  Describe how this principle may be exploited to measure the velocity of a fluid flowing through a pipe. 

\vskip 10pt

Based on what you know of modes of heat transfer, is the operating principle of a thermal flowmeter based on {\it conduction}, {\it convection}, or {\it radiation}?


\vskip 20pt \vbox{\hrule \hbox{\strut \vrule{} {\bf Suggestions for Socratic discussion} \vrule} \hrule}

\begin{itemize}
\item{} Identify different strategies for ``skimming'' a text, as opposed to reading that text closely.  Why do you suppose the ability to quickly scan a text is important in this career?
\end{itemize}

\underbar{file i04024}
%(END_QUESTION)





%(BEGIN_ANSWER)


%(END_ANSWER)





%(BEGIN_NOTES)

Thermal flowmeters exploit the phenomenon of ``wind chill'' to infer mass flow rate.  A heated object is inserted into the flow path, and the rate of heat loss from this object (via convection) is measured to infer mass flow rate.


%INDEX% Reading assignment: Lessons In Industrial Instrumentation, Continuous Fluid Flow Measurement (thermal)

%(END_NOTES)


