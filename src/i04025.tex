
%(BEGIN_QUESTION)
% Copyright 2009, Tony R. Kuphaldt, released under the Creative Commons Attribution License (v 1.0)
% This means you may do almost anything with this work of mine, so long as you give me proper credit

Skim the ``Continuous Fluid Flow Measurement'' chapter in your {\it Lessons In Industrial Instrumentation} textbook to specifically answer these questions:

\vskip 10pt

Positive displacement flowmeters have moving parts inside to measure the passage of fluids through them, yet they are different from other flowmeter types using moving parts (such as turbine meters).  Explain what is different about a ``positive displacement'' flowmeter.


\vskip 20pt \vbox{\hrule \hbox{\strut \vrule{} {\bf Suggestions for Socratic discussion} \vrule} \hrule}

\begin{itemize}
\item{} Identify different strategies for ``skimming'' a text, as opposed to reading that text closely.  Why do you suppose the ability to quickly scan a text is important in this career?
\end{itemize}

\underbar{file i04025}
%(END_QUESTION)





%(BEGIN_ANSWER)


%(END_ANSWER)





%(BEGIN_NOTES)

Positive displacement flowmeters are constructed kind of like engines, where a moving mechanism transports a definite volume of fluid per rotation or cycle.  The frequency of this cycling (or the speed of rotation) is proportional to flow rate.  The total number of cycles or turns is proportional to the total volume of fluid passed through the flowmeter.


%INDEX% Reading assignment: Lessons In Industrial Instrumentation, Continuous Fluid Flow Measurement (positive displacement)

%(END_NOTES)


