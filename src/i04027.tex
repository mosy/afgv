
%(BEGIN_QUESTION)
% Copyright 2009, Tony R. Kuphaldt, released under the Creative Commons Attribution License (v 1.0)
% This means you may do almost anything with this work of mine, so long as you give me proper credit

Skim the ``Continuous Fluid Flow Measurement'' chapter in your {\it Lessons In Industrial Instrumentation} textbook to specifically answer these questions:

\vskip 10pt

You have already explored how {\it load cells} may be used to infer the amount of material stored in process vessels.  Explain how these same devices may be used to infer flowrate in and out of the same vessels using the {\it change-of-quantity} technique.

\vskip 10pt

Suppose an instrument technician decides to build a flowmeter for her propane-fueled barbecue, by placing the propane fuel tank on top of an electronic weigh scale (like a standard bathroom scale).  Explain how the technician could take the measurements given by this scale and somehow convert them to register as a {\it flowrate} of propane gas.


\vskip 20pt \vbox{\hrule \hbox{\strut \vrule{} {\bf Suggestions for Socratic discussion} \vrule} \hrule}

\begin{itemize}
\item{} Identify different strategies for ``skimming'' a text, as opposed to reading that text closely.  Why do you suppose the ability to quickly scan a text is important in this career?
\end{itemize}

\underbar{file i04027}
%(END_QUESTION)





%(BEGIN_ANSWER)


%(END_ANSWER)





%(BEGIN_NOTES)

If we measure the weight of a storage vessel over time, we may calculate the rate of weight change and infer net flow in or out of the vessel this way.

\vskip 10pt

If we measured the weight of a propane tank fueling a BBQ, the rate of the tank's weight loss would be equal to the mass flow rate of the propane fuel.

%INDEX% Reading assignment: Lessons In Industrial Instrumentation, Continuous Fluid Flow Measurement (rate-of-change)

%(END_NOTES)


