
%(BEGIN_QUESTION)
% Copyright 2009, Tony R. Kuphaldt, released under the Creative Commons Attribution License (v 1.0)
% This means you may do almost anything with this work of mine, so long as you give me proper credit

Read and outline the ``Work, Energy, and Power'' subsection of the ``Classical Mechanics'' section of the ``Physics'' chapter in your {\it Lessons In Industrial Instrumentation} textbook.  Note the page numbers where important illustrations, photographs, equations, tables, and other relevant details are found.  Prepare to thoughtfully discuss with your instructor and classmates the concepts and examples explored in this reading.

\underbar{file i04028}
%(END_QUESTION)





%(BEGIN_ANSWER)


%(END_ANSWER)





%(BEGIN_NOTES)

{\it Work} is defined as the exertion of energy by a force acting parallel to a motion.  It is typically expressed in a compound unit, of the force times the displacement (e.g. newton-meters or foot-pounds)

$$W = \vec F \cdot \vec x$$

$$W = F x \cos \theta$$

\noindent
Where,

$W$ = Work, in newton-meters (metric) or foot-pounds (British)

$F$ = Force doing the work, in newtons (metric) or pounds (British)

$x$ = Displacement over which the work was done, in meters (metric) or feet (British)

$\theta$ = Angle between force vector and displacement vector

\vskip 10pt

If a crane hoists a weight up into the air, the crane is {\it doing work on the weight}: the crane's force vector is pointing the same direction as the displacement vector (both up), and therefore the work done by the crane is a positive quantity.  The weight, conversely, is {\it having work done on it}: the weight's force vector (pointing down) directly opposes the displacement vector (pointing up), and therefore the work donw by the weight is a negative quantity.  This is analagous to electrical {\it sources} and {\it loads}.

No work is done if the force and displacement vectors are at right angles to each other.

\vskip 10pt

{\it Potential energy} is energy existing in a stored state, having the potential to do useful work (e.g. a tensed spring, a charged capacitor, a suspended weight).  When a weight is lifted against gravity, the work done in lifting that weight gets stored in the weight's suspension:
 
$$E_p = W = Fx$$

$$E_p = mgh \hbox{ \hskip 30pt (if lifted vertically)}$$

\noindent
Where,

$E_p$ = Potential energy in newton-meters (metric) or foot-pounds (British)

$m$ = Mass of object in kilograms (metric) or slugs (British)

$g$ = Acceleration of gravity in meters per second squared (metric) or feet per second squared (British)

$h$ = Height of lift in meters (metric) or feet (British)

\vskip 10pt

If that suspended weight is then released, its stored energy will go into motion as the weight falls down.  In the case of a crane lowering a suspended weight, the stored energy of that weight is typically dissipated in the braking mechanism of the crane's hoist.

\vskip 10pt

In industrial work settings, we must take measures to ensure dangerous potential energy does not release in a way that will cause harm.  This is why we have lock-out and tag-out (``Energy Control'') procedures, including locking strategies allowing a large number of people to ensure the safety locks cannot be removed until every last person's work is finished and it is once again safe to unleash the potential energy.

\vskip 10pt

{\it Kinetic energy} is energy in motion (e.g. a moving bullet, a light wave).  When applied to moving objects, kinetic energy is proportional to the {\it square} of the object's velocity:

$$E_k = {1 \over 2} mv^2$$

\noindent
Where,

$E_k$ = Kinetic energy in joules or newton-meters (metric), or foot-pounds (British)

$m$ = Mass of object in kilograms (metric) or slugs (British)

$v$ = Velocity of mass in meters per second (metric) or feet per second (British)

\vskip 10pt

All forms of energy have equivalent units of measurement, even though those units may be compounded and convoluted.  A ``joule'' of energy is equivalent to a ``newton-meter'' of energy, which is equivalent to a ``kilogram-meter-squared-per-second-squared'' of energy.









\vskip 20pt \vbox{\hrule \hbox{\strut \vrule{} {\bf Suggestions for Socratic discussion} \vrule} \hrule}

\begin{itemize}
\item{} An important safety policy at many industrial facilities is something called {\it stop-work authority}, which means any employee has the right to stop work they question as unsafe.  Describe a scenario involving stored energy where one might invoke stop-work authority.
\item{} How does an {\it escape ramp} work to dissipate the kinetic energy of a runaway freight truck going down a hill?
\item{} How does a liquid-filled {\it crash drum} work to dissipate the kinetic energy of a speeding car as it careens off the road?
\item{} As a kid, shooting BB guns, I learned I could recapture and re-use by BB ammunition by setting up a hanging cloth behind my targets.  The cloth would gently stop the flying BB without deforming it, or being puncured by the BB.  Explain how this worked by appealing to kinetic and potential energy.
\item{} Explain how conductive heat transfer works, knowing that the temperature of an object is an expression of its molecules' kinetic energy.
\item{} Where does the kinetic energy of a bicycle go when the rider applies the brakes?
\item{} Explain what is technically wrong with the following statement: {\it ``This electric winch is really powerful, because it can pull with a force of 20,000 pounds!''}
\item{} Why do hybrid electric cars get such great mileage when driving in stop-and-go traffic, often {\it better} fuel mileage than when the car is driven at constant speed on a highway?
\item{} Suppose someone planning an off-grid home needs to store energy collected by solar panels during the day, to be used at night.  What energy-storage options can you think of?  Be as creative as you can in your answer(s)!
\item{} Analyze the ``Energy Control Procedure'' document photographed in the textbook.  Identify where in this procedure steps are taken to secure sources of potential energy so that they cannot cause harm.
\item{} {\bf INST230 review question:} When a VFD uses {\it DC injection} to slow down an electric motor, where does the kinetic energy goe?
\item{} {\bf INST230 review question:} When a VFD uses {\it dynamic braking} to slow down an electric motor, where does the kinetic energy go?
\item{} {\bf INST230 review question:} When a VFD uses {\it regenerative braking} to slow down an electric motor, where does the kinetic energy go?
\item{} {\bf INST230 review question:} When a VFD uses {\it plugging} to slow down an electric motor, where does the kinetic energy go?
\end{itemize}












\vfil \eject

\noindent
{\bf Prep Quiz:}

An automobile moving at a speed of 30 miles per hour possesses a substantial amount of kinetic energy.  Suppose the automobile's speed increases to 60 miles per hour.  How much kinetic energy does it now possess, compared to before (at the slower speed)?  Note that although there isn't enough information given here to calculate exact energy figures, you can still compare the two energy values in terms of their ratio to each other (e.g. {\it ``at the faster speed the car possesses 1.4 times as much kinetic energy as before}'').












\vfil \eject

\noindent
{\bf Prep Quiz:}

A flatbed truck with a gross weight of 8,000 pounds traveling at 60 MPH possesses a substantial amount of kinetic energy.  Suppose the truck gets loaded up with additional cargo so that its gross (total) weight is now 16,000 pounds and then returns to the highway at the same speed (60 MPH).  How much kinetic energy does it now possess, compared to before when it carried less cargo?  Note that although there isn't enough information given here to calculate exact energy figures, you can still compare the two energy values in terms of their ratio to each other (e.g. {\it ``at the greater gross weight the truck possesses 3.5 times as much kinetic energy as before}'').











\vfil \eject

\noindent
{\bf Prep Quiz:}

A large mass lifted 14 feet vertically by a crane possess a certain amount of potential energy as a result of being lifted.  Suppose the crane lifts this mass an {\it additional} 14 feet up in the air, so that it is now suspended 28 feet above the ground.  How much potential energy does it now possess, compared to before (at the 14 foot height)?  Note that although there isn't enough information given here to calculate exact energy figures, you can still compare the two energy values in terms of their ratio to each other (e.g. {\it ``at the higher elevation the mass possesses 1.8 times as much potential energy as before}'').

%INDEX% Reading assignment: Lessons In Industrial Instrumentation, Physics (work, energy, and power)

%(END_NOTES)


