%(BEGIN_QUESTION)
% Copyright 2009, Tony R. Kuphaldt, released under the Creative Commons Attribution License (v 1.0)
% This means you may do almost anything with this work of mine, so long as you give me proper credit

Calculate the amount of potential energy stored in the mass of an elevator plus its cargo of potatoes (570 pounds) as it lifts up to a height of 10 feet.  Express this potential energy in both British and metric units.  Also, identify which equation is more convenient to use for this calculation: $E_p = Fx$ or $E_p = mgh$?

\vskip 10pt

Calculate the kinetic energy of a bullet with a mass of 150 grains (9.7198 grams) traveling at a velocity of 2820 feet per second, expressing your answer in both British and metric units.

\vskip 20pt \vbox{\hrule \hbox{\strut \vrule{} {\bf Suggestions for Socratic discussion} \vrule} \hrule}

\begin{itemize}
\item{} Demonstrate how to {\it estimate} numerical answers for this problem without using a calculator.
\item{} Describe the various energy-transfer operations taking place when a crane lifts a heavy weight and then sets it down in a different location.
\item{} If a crane lifts a weight 10 feet above the ground, then sets that weight down on top of an object 6 feet above the ground, is energy still conserved?  Explain in detail.
\item{} Identify how to secure a crane in a zero-energy state prior to performing maintenance work on it.
\item{} Which has a greater effect on a moving object's kinetic energy: an increase in mass or an increase in velocity?
\item{} If an automobile doubles its speed, how much longer (distance) will it take to skid to a complete stop if the driver suddenly applies the brakes?
\end{itemize}

\underbar{file i04029}
%(END_QUESTION)





%(BEGIN_ANSWER)

\noindent
{\bf Partial answer:}

\vskip 10pt

Elevator energy = 5700 ft-lb

\vskip 10pt

Bullet energy = 2648.2 ft-lb

%(END_ANSWER)





%(BEGIN_NOTES)

$$E_p = F x = mgh$$

$$\hbox{Elevator energy } = (570 \hbox{ lb})(10 \hbox{ ft}) = 5700 \hbox{ ft-lb} = 7728 \hbox{ joules}$$

\vskip 10pt

In order to calculate the bullet's kinetic energy in British units, we must first convert the bullet's mass value into units of {\it slugs}:

$$\left({9.7198 \hbox{ grams} \over 1}\right) \left({0.031081 \hbox{ slugs} \over 453.5924 \hbox{ grams}}\right) = 6.66019 \times 10^{-4} \hbox{ slugs}$$

Now we may plug this mass value into the kinetic energy formula to solve for the bullet's kinetic energy:

$$E_k = {1 \over 2} m v^2$$

$$\hbox{Bullet energy } = (0.5) (6.66019 \times 10^{-4} \hbox{ slugs}) (2820 \hbox{ ft/s})^2 = 2648.2 \hbox{ ft-lb} = 3590.5 \hbox{ joules}$$

\vskip 10pt

The cargo material (potatoes) is extraneous information, included for the purpose of challenging students to identify whether or not information is relevant to solving a particular problem.

%INDEX% Physics, energy, work, power: potential and kinetic energy

%(END_NOTES)


