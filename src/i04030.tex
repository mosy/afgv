
%(BEGIN_QUESTION)
% Copyright 2009, Tony R. Kuphaldt, released under the Creative Commons Attribution License (v 1.0)
% This means you may do almost anything with this work of mine, so long as you give me proper credit

Read and outline the ``Fluid Viscosity'' subsection of the ``Fluid Mechanics'' section of the ``Physics'' chapter in your {\it Lessons In Industrial Instrumentation} textbook.  Note the page numbers where important illustrations, photographs, equations, tables, and other relevant details are found.  Prepare to thoughtfully discuss with your instructor and classmates the concepts and examples explored in this reading.

\underbar{file i04030}
%(END_QUESTION)





%(BEGIN_ANSWER)


%(END_ANSWER)





%(BEGIN_NOTES)

Viscosity is a measure of a fluid's resistance to {\it shear}.  It may be thought of as a fluid's ``internal friction''.

\vskip 10pt

Absolute (dynamic) viscosity is defined in terms of shear force versus shear velocity, and is symbolized by the Greek letters ``eta'' ($\eta$) or ``mu'' ($\mu$).  The unit of measurement for absolute viscosity is the Pascal-second, or the {\it poise} (0.1 Pascal-seconds).  Water has an absolute viscosity of 1 centipoise (1 cp).

$$\eta = {FL \over Av}$$

\noindent
Where,

$\eta$ = Absolute viscosity (pascal-seconds), also symbolized as $\mu$ 

$F$ = Force (newtons)

$L$ = Film thickness (meters) -- typically {\it much} less than 1 meter for any realistic demonstration!

$A$ = Plate area (square meters)

$v$ = Relative velocity (meters per second)

\vskip 20pt

Kinematic viscosity is defined in terms of absolute viscosity per unit density, and is symbolized by the Greek letter ``nu'' ($\nu$).  The unit of measurement for kinematic viscosity is the {\it stokes}, defined as one poise of absolute viscosity for a mass density of 1 gram per cubic centimeter.  Water has a kinematic viscosity of 1 centistokes.

$$\nu = {\eta \over \rho}$$

\noindent
Where,

$\nu$ = Kinematic viscosity (stokes)

$\eta$ = Absolute viscosity (poise)

$\rho$ = Mass density (grams per cubic centimeter)

\vskip 20pt

The mechanism of liquid viscosity is inter-molecular {\it cohesion}.  Thus, increasing temperature typically reduces the viscosity of liquids.  The mechanism of gas viscosity is inter-molecular {\it collision}.  Thus increasing temperature typically increases the viscosity of gases.

\vskip 10pt

Some fluids -- called non-Newtonian fluids -- exhibit viscosity changes with changes in applied stress.  A mixture of cornstarch and water is one example of a non-Newtonian fluid because it tends to ``solidify'' when stress is applied.











\vskip 20pt \vbox{\hrule \hbox{\strut \vrule{} {\bf Suggestions for Socratic discussion} \vrule} \hrule}

\begin{itemize}
\item{} How may we decrease the viscosity of bunker oil (a dense and viscous hydrocarbon fluid used as fuel for large steam-driven ships) in order to make it flow easier through a pipe?  Suppose the bunker line connecting an oil refinery to the ship's refueling dock is several miles long.
\item{} Suppose we needed to measure the viscosity of several different lubricating oils to rank them against each other.  Devise a test (or a mechanism) which you could use to do this.
\item{} Suppose the thickness of an oil film separating two metal surfaces moving relative to each other (e.g. a machine shaft and a sleeve bearing) increases.  Will the force required to move those surfaces relative to each other increase or decrease?
\item{} Suppose the viscosity of an oil film separating two metal surfaces moving relative to each other (e.g. a machine shaft and a sleeve bearing) increases.  Will the force required to move those surfaces relative to each other increase or decrease?
\item{} Suppose the overlapping area of two metal surfaces moving relative to each other while separated by a film of oil (e.g. a machine shaft and a sleeve bearing) increases.  Will the force required to move those surfaces relative to each other increase or decrease?
\item{} Suppose the relative velocity of two metal surfaces moving relative to each other while separated by a film of oil (e.g. a machine shaft and a sleeve bearing) increases.  Will the force required to move those surfaces relative to each other increase or decrease?
\end{itemize}

%INDEX% Reading assignment: Lessons In Industrial Instrumentation, Fluid Mechanics (viscosity)

%(END_NOTES)


