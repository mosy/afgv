
%(BEGIN_QUESTION)
% Copyright 2009, Tony R. Kuphaldt, released under the Creative Commons Attribution License (v 1.0)
% This means you may do almost anything with this work of mine, so long as you give me proper credit

Les og understrek afgv.pdf/Phusices/Fluid Mechanics/Reynolds Number. Noter sidenummer med viktige illustasjoner, bilder, formler og annen viktig informasjon. Forbered deg på å kunne diskutere emnet fluid viskositet. 


\underbar{file i04031}
%(END_QUESTION)





%(BEGIN_ANSWER)


%(END_ANSWER)





%(BEGIN_NOTES)

Reynolds number is a unitless measure of a fluid's momentum versus its viscosity (a ratio of kinetic forces to viscous forces).  Low Reynolds numbers predict ``viscid'' or {\it laminar} flow, where the paths of fluid molecules travel in straight lines with little or no intersection.  High Reynolds numbers predict ``inviscid'' or {\it turbulent} flow, where fluid molecules move chaotically with lots of intersection and mixing on a microscopic scale.  This kind of micro-turbulent flow is not to be confused with large-scale turbulence such as swirls and eddies caused by piping irregularities.

\vskip 10pt

Turbulent flow is often desired in industrial processes such as heat exchange, mixing, and flow measurement.  Microscopic turbulence has the effect of ``flattening'' the flow profile of fluid exhibiting swirls and other large-scale turbulence from piping shapes -- in other words, micro-turbulence helps condition the flow for accurate flow measurement.

\vskip 10pt

Reynolds values less than 2000 are usually associated with laminar flow.  Values in excess of 10000 are usually associated with turbulent flow.






\vskip 20pt \vbox{\hrule \hbox{\strut \vrule{} {\bf Suggestions for Socratic discussion} \vrule} \hrule}

\begin{itemize}
\item{} Explain how micro-turbulence helps to ``even out'' the flow profile of a fluid through a pipe, whereas viscous flow encourages a bullet-shaped flow profile where fluid in the pipe's center moves a lot faster than fluid near the pipe walls.
\item{} A common misconception among instrumentation professionals is that some flowmeters require long runs of straight pipe in order to ``eliminate turbulence''.  Explain why this statement is only partially true, and provide your own explanation for long straight-length piping that is more technically accurate.
\item{} For a given volumetric flow rate (e.g. gallons per minute), which will exhibit a flatter flow profile: a large-diameter pipe or a small-diameter pipe?
\item{} For a given pipe size, which will exhibit a flatter flow profile: a low rate of flow or a high rate of flow?
\item{} For a given volumetric flow rate (e.g. gallons per minute), which will exhibit less turbulence: a large-diameter pipe or a small-diameter pipe?
\item{} For a given pipe size, which will exhibit less turbulence: a low rate of flow or a high rate of flow?
\item{} What is the danger of building a heat exchanger with internal tubes that are too large?  (Refer to photo of a heat exchanger in the LIII textbook for reference.)
\item{} What value(s) of Reynolds number characterize {\it laminar flow}?
\item{} What value(s) of Reynolds number characterize {\it turbulent flow}?
\end{itemize}

%INDEX% Reading assignment: Lessons In Industrial Instrumentation, Fluid Mechanics (Reynolds number)

%(END_NOTES)


