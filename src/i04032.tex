
%(BEGIN_QUESTION)
% Copyright 2009, Tony R. Kuphaldt, released under the Creative Commons Attribution License (v 1.0)
% This means you may do almost anything with this work of mine, so long as you give me proper credit

Les og understrek afgv.pdf/Phusices/Fluid Mechanics/Law of Continuity. Noter sidenummer med viktige illustasjoner, bilder, formler og annen viktig informasjon. Forbered deg på å kunne diskutere emnet fluid viskositet. 

\underbar{file i04032}
%(END_QUESTION)





%(BEGIN_ANSWER)


%(END_ANSWER)





%(BEGIN_NOTES)

The Law of Mass Conservation dictates that the mass flow rate through a pipe having no leaks or pulsations in the flow stream must remain constant, like current in a series circuit.  This mass flow rate is equal to the mass density of the fluid ($\rho$) multiplied by the cross-sectional area of the pipe ($A$) multiplied by the average velocity of the fluid ($\overline{v}$):

$$W = \rho A \overline{v}$$

In situations where the fluid is ``incompressible'' (i.e. does not exhibit substantial changes in density), the $\rho$ term may be ignored, and constant volumetric flow ($Q$) assumed at all places in a pipe:

$$Q = A \overline{v}$$

This tells us that the velocity of a fluid will be inversely proportional to the cross-sectional area of the pipe: if the pipe gets skinnier, the velocity increases.







\vskip 20pt \vbox{\hrule \hbox{\strut \vrule{} {\bf Suggestions for Socratic discussion} \vrule} \hrule}

\begin{itemize}
\item{} Explain how the Law of Continuity for fluids is a direct consequence of the {\it Law of Mass Conservation}.
\item{} Does the Law of Continuity still hold true if the fluid in question is compressible?
\item{} Explain why all the continuity formulae specify {\it average} velocity rather than absolute velocity. 
\item{} Suppose the Reynolds number of a fluid moving through a pipe was too low to direct measure using a particular flowmeter technology.  How can we re-size the pipe (and flowmeter) to make this flow stream measureable?
\item{} If a pipe reduces from 10" diameter to 4" diameter, what happens to the average velocity of an incompressible fluid flowing through it?  {\it Answer: velocity increases by a factor of 6.25}
\item{} If a pipe reduces from 30" diameter to 18" diameter, what happens to the average velocity of an incompressible fluid flowing through it?  {\it Answer: velocity increases by a factor of 2.78}
\item{} If a pipe expands from 2" diameter to 7" diameter, what happens to the average velocity of an incompressible fluid flowing through it?  {\it Answer: velocity decreases by a factor of 12.25}
\item{} If a pipe expands from 8" diameter to 24" diameter, what happens to the average velocity of an incompressible fluid flowing through it?  {\it Answer: velocity decreases by a factor of 9}
\end{itemize}

%INDEX% Reading assignment: Lessons In Industrial Instrumentation, Fluid Mechanics (Reynolds number)

%(END_NOTES)


