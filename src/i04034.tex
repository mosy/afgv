%(BEGIN_QUESTION)
% Copyright 2009, Tony R. Kuphaldt, released under the Creative Commons Attribution License (v 1.0)
% This means you may do almost anything with this work of mine, so long as you give me proper credit

A flowmeter installed in a DN150 schedule 40 pipe (internal diameter = 154.051 mm) to measure the flow of olive oil requires a Reynolds number of at least 12,500 to function properly.  Calculate the minimum flow rate of oil through this pipe that the flowmeter can measure, assuming the oil's density is 917.9 kg/m³ and its absolute viscosity is 111 centipoise.

\vskip 10pt

Additionally, calculate the minimum average flowing velocity of the olive oil for the flowmeter to properly function.

\vskip 20pt \vbox{\hrule \hbox{\strut \vrule{} {\bf Suggestions for Socratic discussion} \vrule} \hrule}

\begin{itemize}
\item{} How may a piping system be modified to increase the Reynolds number of the flow, for the sake of measuring that flow rate with a flowmeter requiring a high Reynolds number?  Keep in mind that the actual flow rate is fixed here -- we must vary something else to boost Reynolds number for any given flow rate.
\item{} Does the temperature of the liquid have any effect on its Reynolds number?  Why or why not?
\item{} If this pipe were flowing water rather than olive oil, what would the minimum flow rate be to satisfy the Reynolds number requirement of this flowmeter?
\end{itemize}


\underbar{file i04034}
%(END_QUESTION)





%(BEGIN_ANSWER)

\noindent
{\bf Partial answer:}

\vskip 10pt

$Q$ (minimum) = 2900.1 GPM

$$\hbox{Re} = {{3160 G_f Q} \over {D \mu}}$$

\noindent
Where,

Re = Reynolds number (unitless)

$G_f$ = Specific gravity of liquid (unitless)

$Q$ = Flow rate (gallons per minute)

$D$ = Diameter of pipe (inches)

$\mu$ = Absolute viscosity of fluid (centipoise)

3160 = Conversion factor for British units

\vskip 10pt


$$Q = {\hbox{Re} D \mu \over 3160 G_f}$$

$$Q = {(12500) (6.065) (111) \over (3160) \left(57.3 \over 62.4\right)} = 2900.1 \hbox{ GPM}$$

$Q$ (minimum) = 2900.1 GPM = 387.7 ft$^{3}$/min {\it (which is an extraordinarily high flow rate for a 6-inch pipe!)}

\vskip 10pt

$$Q = A \overline{v}$$

$$\overline{v} = {Q \over A} = {387.7 \hbox{ ft}^3\hbox{/min} \over 0.2006 \hbox{ ft}^2} = 1932.35 \hbox{ ft/min}$$

\vskip 10pt

Based on these large figures (over 32 feet per second velocity!), this might not be the best flowmeter technology to use on olive oil!  The relatively high viscosity of the oil makes it difficult to achieve the requisite Reynolds number for good operation.

%(END_ANSWER)





%(BEGIN_NOTES)

$$\hbox{Re} = {{3160 G_f Q} \over {D \mu}}$$

\noindent
Where,

Re = Reynolds number (unitless)

$G_f$ = Specific gravity of liquid (unitless)

$Q$ = Flow rate (gallons per minute)

$D$ = Diameter of pipe (inches)

$\mu$ = Absolute viscosity of fluid (centipoise)

3160 = Conversion factor for British units

\vskip 10pt


$$Q = {\hbox{Re} D \mu \over 3160 G_f}$$

$$Q = {(12500) (6.065) (111) \over (3160) \left(57.3 \over 62.4\right)} = 2900.1 \hbox{ GPM}$$

$Q$ (minimum) = 2900.1 GPM = 387.7 ft$^{3}$/min {\it (which is an extraordinarily high flow rate for a 6-inch pipe!)}

\vskip 10pt

$$Q = A \overline{v}$$

$$\overline{v} = {Q \over A} = {387.7 \hbox{ ft}^3\hbox{/min} \over 0.2006 \hbox{ ft}^2} = 1932.35 \hbox{ ft/min}$$

\vskip 10pt

Based on these large figures (over 32 feet per second velocity!), this might not be the best flowmeter technology to use on olive oil!  The relatively high viscosity of the oil makes it difficult to achieve the requisite Reynolds number for good operation.

%INDEX% Physics, dynamic fluids: Reynolds number

%(END_NOTES)


