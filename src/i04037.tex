
%(BEGIN_QUESTION)
% Copyright 2009, Tony R. Kuphaldt, released under the Creative Commons Attribution License (v 1.0)
% This means you may do almost anything with this work of mine, so long as you give me proper credit

Read and outline the ``Venturi Tubes and Basic Principles'' subsection of the ``Pressure-Based Flowmeters'' section of the ``Continuous Fluid Flow Measurement'' chapter in your {\it Lessons In Industrial Instrumentation} textbook.  Note the page numbers where important illustrations, photographs, equations, tables, and other relevant details are found.  Prepare to thoughtfully discuss with your instructor and classmates the concepts and examples explored in this reading.

\underbar{file i04037}
%(END_QUESTION)





%(BEGIN_ANSWER)


%(END_ANSWER)





%(BEGIN_NOTES)

As a fluid moves through any constriction (venturi tube, orifice, etc.), its velocity accelerates to move through the narrower throat.  This acceleration causes the fluid molecules' kinetic energy to increase.  According to the Conservation of Energy, total energy must remain constant and so as kinetic energy increases the potential energy (static pressure) decreases at the throat of the constriction.  Thus, a constriction creates a differential pressure dependent on the rate of fluid flow through it.

\vskip 10pt

When we manipulate Bernoulli's Equation to solve for flow rate based on differential pressure, we arrive at the following result:

$$Q = \sqrt{2} A_2 {1 \over \sqrt{1 - \left({A_2 \over A_1}\right)^2}} \sqrt{{P_1 - P_2} \over \rho}$$

Simplifying this to a proportionality:

$$Q = k \sqrt{{P_1 - P_2} \over \rho}$$







\vskip 20pt \vbox{\hrule \hbox{\strut \vrule{} {\bf Suggestions for Socratic discussion} \vrule} \hrule}

\begin{itemize}
\item{} What effect will narrowing the constriction have on the $\Delta P$ produced by a venturi tube, assuming all other factors remain constant?
\item{} What effect will increasing the flow rate have on the $\Delta P$ produced by a venturi tube, assuming all other factors remain constant?
\item{} What effect will increasing fluid density have on the $\Delta P$ produced by a venturi tube, assuming all other factors remain constant?
\item{} Identify some of the factors influencing the value of $k$ in the final formula.
\end{itemize}

%INDEX% Reading assignment: Lessons In Industrial Instrumentation, Continuous Fluid Flow Measurement (venturi tubes)

%(END_NOTES)


