
%(BEGIN_QUESTION)
% Copyright 2009, Tony R. Kuphaldt, released under the Creative Commons Attribution License (v 1.0)
% This means you may do almost anything with this work of mine, so long as you give me proper credit

Read and outline the ``Volumetric Flow Calculations'' and ``Mass Flow Calculations'' subsections of the ``Pressure-Based Flowmeters'' section of the ``Continuous Fluid Flow Measurement'' chapter in your {\it Lessons In Industrial Instrumentation} textbook.  Note the page numbers where important illustrations, photographs, equations, tables, and other relevant details are found.  Prepare to thoughtfully discuss with your instructor and classmates the concepts and examples explored in this reading.


\underbar{file i04038}
%(END_QUESTION)





%(BEGIN_ANSWER)


%(END_ANSWER)





%(BEGIN_NOTES)

The constant of proportionality ($k$) in the basic volumetric flow calculation shown below not only accounts for the geometry of the flow element (e.g. venturi tube, orifice plate, pitot tube, etc.) but it may also account for all unit conversions:

$$Q = k \sqrt{{P_1 - P_2} \over \rho}$$

All we need to do is take a known amount of flow ($Q$) with its corresponding pressure drop ($P_1 - P_2$) and solve for $k$ to arrive at a formula we may use to solve for {\it any} combination of flow rates and pressure drops.  The ``$k$ factor'' not only describes the flow element's geometry but also serves as a unit-of-measurement correction factor to harmonize all units of measurement used for flow, differential pressure, and fluid density.

\vskip 10pt

{\it Mass} flow measurement applications are important where the mass of products must be carefully accounted, such as in custody transfer applications.  Substituting $W \over \rho$ for $Q$ in the volumetric flow formula yields a mass flow formula:

$$W = k \sqrt{\rho ({P_1 - P_2})}$$

The value of $k$ is determined from empirical values of $W$ and pressure drop for a particular flow element.







\vskip 20pt \vbox{\hrule \hbox{\strut \vrule{} {\bf Suggestions for Socratic discussion} \vrule} \hrule}

\begin{itemize}
\item{} Normallly, when we perform calculations using Bernoulli's equation, we must pay very close attention to all the units of measurement to ensure they are compatible with one another.  When we use the flow formula with $k$, we don't have to.  Explain how a constant of proportionality ($k$) serves to harmonize otherwise disparate units of measurement in the $\Delta P$ flow formula.
\item{} Identify the units of measurement typical for {\it volumetric} versus {\it mass} flow measurement, and show how these units relate to each other in the formula $W = \rho Q$.
\item{} Explain how the textbook algebraically converts the basic volumetric flow formula $Q = k \sqrt{P_1 - P_2 \over \rho}$ into a {\it mass} flow formula.  The treatment of $\rho$ is particularly interesting here!
\item{} Describe an example of {\it custody transfer} in flow measurement.
\item{} Explain why mass flow measurement is preferred in boiler control systems for steam flow and feedwater flow.
\item{} When an automobile driver purchases gasoline, are they paying based on volumetric measurement or by mass measurement?
\item{} When a bicyclist purchases fuel (food) at a bulk food store, are they paying based on volumetric measurement or by mass measurement?
\end{itemize}






\vfil \eject

\noindent
{\bf Prep Quiz:}

If the flow rate through a venturi tube or orifice plate is {\it doubled}, the resulting pressure difference between the ``high'' and ``low'' pressure taps of that flow element will:

\begin{itemize}
\item{} Decrease by a factor of 2
\vskip 5pt 
\item{} Decrease by a factor of 4
\vskip 5pt 
\item{} Decrease by a factor of 9
\vskip 5pt 
\item{} Increase by a factor of 2
\vskip 5pt 
\item{} Increase by a factor of 4
\vskip 5pt 
\item{} Increase by a factor of 9
\end{itemize}

\vfil \eject

\noindent
{\bf Prep Quiz:}

If the flow rate through a venturi tube or orifice plate is {\it tripled}, the resulting pressure difference between the ``high'' and ``low'' pressure taps of that flow element will:

\begin{itemize}
\item{} Decrease by a factor of 2
\vskip 5pt 
\item{} Decrease by a factor of 4
\vskip 5pt 
\item{} Decrease by a factor of 9
\vskip 5pt 
\item{} Increase by a factor of 2
\vskip 5pt 
\item{} Increase by a factor of 4
\vskip 5pt 
\item{} Increase by a factor of 9
\end{itemize}


%INDEX% Reading assignment: Lessons In Industrial Instrumentation, Continuous Fluid Flow Measurement (volumetric and mass flow calculations)

%(END_NOTES)


