
%(BEGIN_QUESTION)
% Copyright 2009, Tony R. Kuphaldt, released under the Creative Commons Attribution License (v 1.0)
% This means you may do almost anything with this work of mine, so long as you give me proper credit

Read and outline the ``Square-Root Characterization'' subsection of the ``Pressure-Based Flowmeters'' section of the ``Continuous Fluid Flow Measurement'' chapter in your {\it Lessons In Industrial Instrumentation} textbook.  Note the page numbers where important illustrations, photographs, equations, tables, and other relevant details are found.  Prepare to thoughtfully discuss with your instructor and classmates the concepts and examples explored in this reading.

\underbar{file i04039}
%(END_QUESTION)





%(BEGIN_ANSWER)


%(END_ANSWER)





%(BEGIN_NOTES)

The pressure drop generated by an acceleration-style flow element is proportional to the {\it square} of the flow rate.  Flow rate doubles, and pressure quadruples; flow rate triples, and pressure increases by a factor of 9.

\vskip 10pt

In order to achieve a signal that is linear with flow (and not with pressure), we must somehow {\it square-root} the measured pressure signal.  This may be done internally to the transmitter or to the indicating instrument, or it may be done inside an independent ``square root'' relay:

% No blank lines allowed between lines of an \halign structure!
% I use comments (%) instead, so that TeX doesn't choke.

$$\vbox{\offinterlineskip
\halign{\strut
\vrule \quad\hfil # \ \hfil & 
\vrule \quad\hfil # \ \hfil \vrule \cr
\noalign{\hrule}
%
% First row
Input percentage & Output percentage \cr
%
\noalign{\hrule}
%
% Another row
0\% & $\sqrt{0} = 0\%$ \cr
%
\noalign{\hrule}
%
% Another row
25\% & $\sqrt{0.25} = 50\%$ \cr
%
\noalign{\hrule}
%
% Another row
50\% & $\sqrt{0.5} = 70.71\%$ \cr
%
\noalign{\hrule}
%
% Another row
75\% & $\sqrt{0.75} = 86.60\%$ \cr
%
\noalign{\hrule}
%
% Another row
100\% & $\sqrt{1} = 100\%$ \cr
%
\noalign{\hrule}
} % End of \halign 
}$$ % End of \vbox

\vskip 10pt

Indicators with nonlinear scales are also used to simply characterize the output signal of a DP transmitter.  A linear scale if often superimposed on the square-root scale for reference.  This does nothing to linearize the actual signal coming from the transmitter (for the benefit of any other instruments in the loop), but it at least linearizes the reading for any human operator looking at the indicator.

\vskip 10pt

The compression evident at the low end of a square-root scale is not just an artifact of the scale, but rather an important illustration of measurement uncertainty toward the low end of a DP flow transmitter's range.  At low percentages of flow, even tiny errors in pressure measurement translate into relatively large flow measurement errors.  This is why the practical turndown ratio for DP-based flowmeters is approximately 3:1 (i.e. the lowest measurement with trusted accuracy is only one-third of the flow's upper range value).









\vskip 20pt \vbox{\hrule \hbox{\strut \vrule{} {\bf Suggestions for Socratic discussion} \vrule} \hrule}

\begin{itemize}
\item{} Explain why we must take the {\it square root} of the $\Delta P$ signal in order to measure flow rate.
\item{} Explain what a {\it square root extractor} is and what it is used for.
\item{} Explain how we can tell each of the receiver gauges is designed to operate off of a ``live zero'' signal (e.g. {\it 3}-15 PSI, {\it 4}-20 mA) at first glance, without even reading any of the text on the indicator face.
\item{} Closely examine the face of the 4-20 mA nonlinear indicator shown in the textbook, and use a calculator to validate the square-root relationship between numbers on the ``linear'' versus ``flow'' scales.
\item{} Suppose a DP transmitter sensing pressure dropped across a venturi tube has a slight ``zero'' sensor trim error: it reads +1 inch of water column higher than it should at all applied pressures.  Where in the range of flow measurement will this error have the most effect on the accuracy of flow measurement, and why?  Does it matter where the square-root extraction is done in the loop?
\item{} Explain what ``turndown'' means in the context of flowmeters.
\item{} Explain why the ``turndown'' ratio of pressure-based flowmeters is generally so poor.
\end{itemize}



%INDEX% Reading assignment: Lessons In Industrial Instrumentation, Continuous Fluid Flow Measurement (volumetric and mass flow calculations)

%(END_NOTES)


