
%(BEGIN_QUESTION)
% Copyright 2009, Tony R. Kuphaldt, released under the Creative Commons Attribution License (v 1.0)
% This means you may do almost anything with this work of mine, so long as you give me proper credit

Read and outline the ``Orifice Plates'' subsection of the ``Pressure-Based Flowmeters'' section of the ``Continuous Fluid Flow Measurement'' chapter in your {\it Lessons In Industrial Instrumentation} textbook.  Note the page numbers where important illustrations, photographs, equations, tables, and other relevant details are found.  Prepare to thoughtfully discuss with your instructor and classmates the concepts and examples explored in this reading.

\underbar{file i04040}
%(END_QUESTION)





%(BEGIN_ANSWER)


%(END_ANSWER)





%(BEGIN_NOTES)

An orifice plate is simply a metal plate with a hole machined in the middle for flow to pass through.  It is typically sandwiched between two pipe flanges.

\vskip 10pt

The point of maximum constriction in the flow stream is called the {\it vena contracta}.  The ratio of bore diameter to inside pipe diameter is called the {\it beta ratio} ($\beta = {d \over D}$).

\vskip 10pt

Square-edged concentric orifice plates have their bores located in the exact center of the plate.  The square-shaped bore edges allow for bidirectional flow measurement.  Text labeling stamped on the upstream face of an external tab identifies the upstream side of the plate.

\vskip 10pt

Thick orifice plates may have a beveled edge on the downstream face, to minimize contact surface area with the flow stream, and thus minimizing friction between the orifice and the flowing fluid.  These beveled orifice plates obviously are unidirectional.

\vskip 10pt

Eccentric orifice plates have their bore holes located off-center to allow passage of undesired flow components (e.g. bubbles in a liquid stream, where the hole is located high; liquid droplets in a gas stream, where the hole is located low).  Segmental orifice plates have half-round holes oriented either up or down for the same purpose.  Concentric-hole orifice plates may be equipped with vent and/or drain holes to achieve the same end.

\vskip 10pt

Quadrant-edge and conical-entrance orifice plates bevel the {\it upstream} edge of the bore in an attempt to improve measurement accuracy at low Reynolds number values.  The only sure way to tell which way flow should go through such an orifice plate is to pay attention to the text labeling on the tab (text facing upstream).  The increased pressure drop caused by viscous friction on this beveled edge tends to offset the decrease in pressure caused by reduced contraction at the vena contracta for high-viscosity fluids.

\vskip 10pt

Tap locations vary according to pipe size and convention.  Flange taps are very common in the United States.  The downstream tap should be clear of the highly turbulent region following the vena contracta.  Tap holes need to be flush and free of any burrs.

\vskip 10pt

Integral orifice plates mount directly to the DP transmitter, and are often used on small pipes for low flow rates.  Some integral orifice plate manifolds force the fluid to flow {\it through} the flanged body of the DP transmitter.  These are used in cases where the process line size is comparable to the pressure port size on the transmitter.

\vskip 10pt

Orifice plates are best sized by computer software, to account for all the variables related to plate style and accuracy.






\vskip 20pt \vbox{\hrule \hbox{\strut \vrule{} {\bf Suggestions for Socratic discussion} \vrule} \hrule}

\begin{itemize}
\item{} {\bf In what ways may an orifice plate flowmeter be ``fooled'' to report a false flow measurement?}
\item{} How can we tell which way flow should go through an orifice plate?
\item{} Explain the purpose of having a beveled edge on the downstream face of an orifice plate.
\item{} Identify which way to orient an eccentric or segmental orifice plate, for measuring natural gas where some water is present in the flow stream.
\item{} Identify which way to orient an eccentric or segmental orifice plate, for measuring wastewater where some solids are present in the flow stream.
\item{} Identify which way to orient an eccentric or segmental orifice plate, for measuring potable water where some air bubbles are present in the flow stream.
\item{} Identify which way to orient an eccentric or segmental orifice plate, for measuring petroleum oil from an oil well where some sand is present in the flow stream.
\item{} Identify which way to orient an eccentric or segmental orifice plate, for measuring natural gas from a gas well where some sand is present in the flow stream.
\item{} Describe a flow-measurement application where we would use a beveled-edge orifice plate, installing the bevel facing {\it upstream}.
\item{} Describe a flow-measurement application where we would use a beveled-edge orifice plate, installing the bevel facing {\it downstream}.
\item{} Identify some of the alternative tap locations for an orifice plate, and why some might be preferred over others in specific applications.
\item{} Referencing the photograph of the natural gas flowmeter installation shown in the textbook, identify what kind of taps are used in this installation.  Describe what the tap locations would look like if they were different (e.g. radius taps, pipe taps, corner taps, etc.).
\item{} Explain how you could use the block and bypass hand valves to take an integral orifice flowmeter out of service.  How does this procedure differ from that of a standard three-valve DP transmitter manifold?
\end{itemize}


%INDEX% Reading assignment: Lessons In Industrial Instrumentation, Continuous Fluid Flow Measurement (orifice plates)

%(END_NOTES)


