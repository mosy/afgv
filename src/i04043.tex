%(BEGIN_QUESTION)
% Copyright 2009, Tony R. Kuphaldt, released under the Creative Commons Attribution License (v 1.0)
% This means you may do almost anything with this work of mine, so long as you give me proper credit

A horizontal venturi tube at a seawater desalinization plant is sized to produce 11 inches of mercury column (11 "Hg) while flowing 6,000 gallons per minute of sea water (at a density of 1.025 grams per cubic centimeter).

\vskip 10pt

Calculate the differential pressure produced by this same venturi tube at a flow rate of 4,250 GPM, and at a lighter density of 1.01 g/cm$^{3}$.

\vskip 10pt

Assuming a water density of 1.03 g/cm$^{3}$ and a measured differential pressure of 3.1 PSID, calculate the volumetric flow rate through the venturi tube.

\vskip 10pt

Assuming a water density of 1.02 g/cm$^{3}$ and a measured differential pressure of 12 kPaD, calculate the volumetric flow rate through the venturi tube.

\vskip 20pt \vbox{\hrule \hbox{\strut \vrule{} {\bf Suggestions for Socratic discussion} \vrule} \hrule}

\begin{itemize}
\item{} What is the purpose of a ``desalinization'' plant, and where might you expect to find one?
\end{itemize}

\underbar{file i04043}
%(END_QUESTION)





%(BEGIN_ANSWER)

\noindent
{\bf Partial answer:}

\vskip 10pt

$Q$ = 4,533.9 GPM flow rate at 3.1 PSID and 1.03 g/cm$^{3}$

\vskip 10pt

$Q$ = 3,413.7 GPM flow rate at 12 kPaD and 1.02 g/cm$^{3}$

%(END_ANSWER)





%(BEGIN_NOTES)

$$Q = 1831.5 \sqrt{\Delta P \over \rho}$$

\vskip 10pt

$P$ = 5.438 "Hg at 4250 GPM and 1.01 g/cm$^{3}$ density

\vskip 10pt

$Q$ = 4,533.9 GPM flow rate at 3.1 PSID and 1.03 g/cm$^{3}$

\vskip 10pt

$Q$ = 3,413.7 GPM flow rate at 12 kPaD and 1.02 g/cm$^{3}$

\vskip 10pt

The horizontal orientation of the venturi tube is extraneous information, included for the purpose of challenging students to identify whether or not information is relevant to solving a particular problem.

%INDEX% Measurement, flow: simple ``k'' factor equation for flow/pressure correlation

%(END_NOTES)


