
%(BEGIN_QUESTION)
% Copyright 2009, Tony R. Kuphaldt, released under the Creative Commons Attribution License (v 1.0)
% This means you may do almost anything with this work of mine, so long as you give me proper credit

Read and outline the ``High-Accuracy Flow Measurement'' subsection of the ``Pressure-Based Flowmeters'' section of the ``Continuous Fluid Flow Measurement'' chapter in your {\it Lessons In Industrial Instrumentation} textbook.  Note the page numbers where important illustrations, photographs, equations, tables, and other relevant details are found.  Prepare to thoughtfully discuss with your instructor and classmates the concepts and examples explored in this reading.

\underbar{file i04044}
%(END_QUESTION)





%(BEGIN_ANSWER)


%(END_ANSWER)





%(BEGIN_NOTES)

Pressure-based flow elements deviate from ``ideal'' behavior due to a number of factors including compressibility of the fluid, pipe roughness, energy losses due to friction and viscosity, etc.  In order to more accurately predict the flow/pressure relationship for an element such as an orifice plate, we have several other factors included in high-accuracy flow equations.  These include {\it discharge coefficient} ($C$ -- the ratio between true flow rate and theoretical flow rate), {\it gas expansion factor} ($Y$ -- ratio of the gas discharge coefficient to the liquid discharge coefficient for an element), and {\it compressibility factor} ($Z$).

\vskip 10pt

Live parameters such as gas pressure and gas temperature have particular influence on the accuracy of gas flow measurement because they affect the density ($\rho$) of the gas, which of course is an important variable in any DP-based flow formula.  The AGA3 gas flow equation compensates for these variables (as well as $C$, $Y$, and $Z$), and AGA3-compliant gas flow measurement systems must include RTD temperature sensors and absolute pressure sensors in addition to the differential pressure sensor in order to accurately calculate gas flow.  A ``gas flow computer'' takes in these three variables to compute flow rate.  Special {\it multi-variable} transmitters sensing all three of these variables and equipped with the AGA3 gas flow equation may serve the same purpose, computing flow rate and reporting that calculated variable instead of just reporting DP.

\vskip 10pt

For custody transfer (``fiscal'') flow measurement applications, {\it honed meter runs} are used instead of regular pipe.  These precision-machined tubes have mirror-smooth interior surfaces and precisely known dimensions.

\vskip 10pt

In order to improve the inherently poor turndown ratio of orifice-based flowmeters, several honed meter runs may be paralleled through shut-off valves: as flow increases, more valves are opened up to bring additional meter runs on-line, keeping the velocity through all of them within a narrow range where the accuracy is greatest.

\vskip 10pt

Liquid flow measurement applications may also benefit from compensation, but only temperature and not pressure because pressure changes have such a slight effect on liquid density (whereas temperature changes have a much more dramatic effect on liquid density).









\vskip 20pt \vbox{\hrule \hbox{\strut \vrule{} {\bf Suggestions for Socratic discussion} \vrule} \hrule}

\begin{itemize}
\item{} Explain why we must sense DP, absolute pressure, and absolute temperature to perform accurate gas flow measurements when using an orifice plate as the flow element.
\item{} Describe a ``honed meter run'' and explain its significance in flow measurement.
\item{} Explain how ``staging'' is used in a series of honed meter runs to achieve high rangeability.
\item{} What does a {\it pulse} output represent in a multi-variable flow transmitter such as the Yokogawa EJX910?
\item{} Suppose the line pressure of gas flowing through an AGA3 meter run increases, while all the other live variables (DP, temperature) remain constant.  What does the AGA3 equation predict for flow: does flow increase, decrease, or remain the same as before?
\item{} Suppose the line pressure of gas flowing through an AGA3 meter run decreases, while all the other live variables (DP, temperature) remain constant.  What does the AGA3 equation predict for flow: does flow increase, decrease, or remain the same as before?
\item{} Suppose the line temperature of gas flowing through an AGA3 meter run increases, while all the other live variables (DP, temperature) remain constant.  What does the AGA3 equation predict for flow: does flow increase, decrease, or remain the same as before?
\item{} Suppose the line temperature of gas flowing through an AGA3 meter run decreases, while all the other live variables (DP, temperature) remain constant.  What does the AGA3 equation predict for flow: does flow increase, decrease, or remain the same as before?
\item{} Examining the photograph of the integral-orifice Rosemount 3095MV transmitter mounted on the copper line, identify the direction of gas flow.
\end{itemize}










\vfil \eject

\noindent
{\bf Prep Quiz:}

High-accuracy gas flow measurements using orifice plates require which three process variables to be measured?

\begin{itemize}
\item{} Absolute pressure, differential pressure, and absolute temperature
\vskip 5pt 
\item{} Induced voltage, absolute temperature, and gauge pressure
\vskip 5pt 
\item{} Mass flow rate, differential pressure, and ambient (air) temperature
\vskip 5pt 
\item{} Absolute temperature, vortex shedding frequency, and gas viscosity 
\vskip 5pt 
\item{} Coriolis force, vortex shedding frequency, and absolute temperature
\vskip 5pt 
\item{} Gas viscosity, differential pressure, and mass flow rate
\end{itemize}


%INDEX% Reading assignment: Lessons In Industrial Instrumentation, Continuous Fluid Flow Measurement (high-accuracy flow measurement)

%(END_NOTES)


