
%(BEGIN_QUESTION)
% Copyright 2009, Tony R. Kuphaldt, released under the Creative Commons Attribution License (v 1.0)
% This means you may do almost anything with this work of mine, so long as you give me proper credit

Read selected portions of the ``Rosemount 3095MV MultiVariable Mass Flow Transmitter'' reference manual (publication 00809-0100-4716 Revision JA), and answer the following questions:

\vskip 10pt

Page 2-6 (Figure 2-3) shows proper transmitter locations relative to the flow element (e.g. orifice plate).  Identify the proper mounting locations for the transmitter assuming steam service, gas service, and liquid service.

\vskip 10pt

Page 2-7 lists a set of guidelines to follow when installing a DP transmitter in a flow-measuring application for optimum performance.  Explain the rationale behind a few of these guidelines: {\it why} they should be followed.

\vskip 10pt

Identify the type of temperature sensor used to measure the ``flowing temperature'' of the process fluid.  Explain why Rosemount's choice of temperature sensors is ideal for an application such as this.

\vskip 10pt

Write the formula this transmitter uses to calculate mass flow rate, and comment on the variables found within it.  Also, identify the mass flow turndown ratio possible with the ``Ultra for Flow'' high-accuracy option (Code ``U3'' in the part number).

\vskip 20pt \vbox{\hrule \hbox{\strut \vrule{} {\bf Suggestions for Socratic discussion} \vrule} \hrule}

\begin{itemize}
\item{} Where might you find a transmitter such as the Rosemount model 3095 in use?  What types of process applications demand the functionality provided by this instrument?
\item{} According to this document, the rangeability for differential pressure measurement is substantially greater than the rangeability for flow measurement.  Explain the mathematical principle relating these ratios together.
\item{} Should the RTD sensing element be installed upstream or downstream of the orifice plate?  Explain why.
\item{} If the RTD malfunctioned such that it reported a greater temperature than was actually in the pipe, would this lead to a positive flow measurement error or a negative flow measurement error?
\end{itemize}

\underbar{file i04045}
%(END_QUESTION)





%(BEGIN_ANSWER)


%(END_ANSWER)





%(BEGIN_NOTES)

Figure 2-3 on page 206 shows mounting positions for different services: above the pipe for gas, level with the pipe for gas or liquid, below the pipe for steam (with tee fittings for pre-filling the impulse lines).

\vskip 10pt

Some explanations of impulse piping recommendations on page 2-7:

\begin{itemize}
\item{} Keep impulse piping as short as possible: {\it less liquid held in impulse pipes means less hydrostatic pressure; shorter piping means less likelihood of liquid droplets affecting sensed gas pressure}
\item{} Sloped impulse lines: {\it for liquid service this means gas bubbles rise toward the process line where they can escape; for gas service this means liquid droplets fall toward the process line where they can escape.}
\item{} Both impulse lines the same temperature: {\it equal hydrostatic pressures for liquid-filled lines.}
\item{} Fill both impulse lines to the same level with sealing liquid: {\it equal hydrostatic pressures.}
\end{itemize}

\vskip 10pt

The Rosemount model 3095 MV uses a 4-wire RTD to sense process fluid temperature which is ideal given the superior linearity of RTDs (compared to thermocouples and thermistors) and the modest temperature range one would expect from a gas or liquid flowing through an orifice plate.  This is a 100 ohm platinum RTD with an $\alpha$ value equal to 0.00385 (European), as described on page A-7.  Figure 2-11 on page 2-18 shows the 4-wire connection diagram.

\vskip 10pt

Mass flow calculation formula (shown on Appendix page A-5):

$$Q_m = N C_d E Y d^2 \sqrt{\rho \Delta P}$$

\noindent
Where,

$Q_m$ = Mass flow rate (sometimes written as $W$)

$N$ = Unit conversion factor

$C_d$ = Discharge coefficient

$E$ = Velocity of approach factor

$d$ = Orifice bore diameter

$\rho$ = Flowing density of gas

$\Delta P$ = Differential pressure drop

\vskip 10pt

According to the specifications given on page A-5, the ``ultra for flow'' option U3 yields a pressure turndown ratio of 100:1 and a flow turndown ratio of 10:1 (assuming a tolerance of $\pm$ 1\% mass flow error).

%INDEX% Reading assignment: Rosemount 3095MV multivariable transmitter Reference manual

%(END_NOTES)


