
%(BEGIN_QUESTION)
% Copyright 2009, Tony R. Kuphaldt, released under the Creative Commons Attribution License (v 1.0)
% This means you may do almost anything with this work of mine, so long as you give me proper credit

Read selected portions of the Rosemount Orifice Plate Elements manual (``1495 Orifice Plate, 1496 Flange Union, 1497 Meter Section, Installation \& Operation Manual'' publication 00809-0100-4792 Revision AA), and answer the following questions:

\vskip 10pt

Page 3-1 of this manual (Figure 3-1) shows a typical ``flange union'' whereby an orifice plate is sandwiched between two pipe flanges.  Page 4-1 (Figure 4-1) of the manual shows something called a ``meter section,'' which serves a similar purpose.  Identify the difference between a ``meter section'' and a normal ``flange union.''  Explain in your own words why anyone might choose to install a ``meter section'' instead of a regular flange for their orifice plate flowmeter application.

\vskip 10pt

Appendix C shows standard lengths for meter section tubes.  Calculate the typical upstream straight-pipe length (as a ratio of pipe diameters), and also calculate the typical downstream straight-pipe length (as a ratio of pipe diameters), based on the standard lenghts and inside diameters (I.D.) shown in this table.

\vskip 10pt

Appendix A of this manual provides information regarding the proper upstream and downstream straight-pipe requirements for different orifice plate installations.  Based on what you read in this appendix, identify the most ``challenging'' installations for orifice plate elements (i.e. which applications require the longest straight-pipe runs to achieve good accuracy?).

\vskip 10pt

Suppose we have an orifice plate flowmeter installation where there are two pipe elbows (not in the same plane) upstream of the orifice, an orifice plate with a beta ratio of 0.6, and no ``straightening vanes'' installed.  Determine the minimum number of straight-pipe diameters needed upstream and downstream of the orifice plate for good performance in this installation.

\vskip 20pt \vbox{\hrule \hbox{\strut \vrule{} {\bf Suggestions for Socratic discussion} \vrule} \hrule}

\begin{itemize}
\item{} Why is the required straight-pipe length {\it upstream} of an orifice plate greater than the required straight-pipe length {\it downstream} of an orifice plate?
\item{} Does the presence of straightening vanes in an orifice meter run affect only the upstream straight-pipe lengths requirement, or also the downstream straight-pipe length requirement?
\item{} Why do you suppose a pair of pipe elbows in different planes produces a different amount of large-scale turbulence than a pair of pipe elbows in the same plane?
\item{} If the amount of straight pipe between two elbows is increased, will the straight-pipe requirements between those elbows and the orifice plate be affected or not?  Explain why.
\end{itemize}

\underbar{file i04046}
%(END_QUESTION)





%(BEGIN_ANSWER)


%(END_ANSWER)





%(BEGIN_NOTES)

A {\it meter section} is a {\it flange union} with pre-welded straight tubes both upstream and downstream.  The pre-welded tubes help ensure better accuracy than what might be achieved with field-welded pipe, since both the tubes and the welds are more likely to be smooth coming from the factory than assembled in the field.

\vskip 10pt

The standard meter sections listed in Appendix C exhibit a typical upstream straight-pipe length of ten (10) diameters and a downstream straight-pipe length of five (5) diameters.

\vskip 10pt

Regulators and throttling control valves seem to present the greatest level of disturbance, as measured by the necessary upstream straight-pipe run requirements (nearly 45 diameters with a beta ratio of 0.75).

\vskip 10pt

For the double-elbow installation, $D$ upstream (minimum) = 25 diameters and $D$ downstream (minimum) = 4 diameters.









\vfil \eject

\noindent
{\bf Prep Quiz:}

The difference between a {\it meter section} and a {\it flange union} is:

\begin{itemize}
\item{} Meter sections require special DP transmitters; flange unions don't
\vskip 5pt 
\item{} A flange union is made of better-quality metal, for greater strength
\vskip 5pt 
\item{} A flange union yields greater flow-measurement accuracy than a meter section
\vskip 5pt 
\item{} Meter sections use radius taps, while flange unions use flange taps
\vskip 5pt 
\item{} A flange union must be installed horizontally, while a meter section can be vertical
\vskip 5pt 
\item{} A meter section comes with pre-finished pipes attached to the flanges
\end{itemize}

%INDEX% Reading assignment: Rosemount orifice plates manual

%(END_NOTES)


