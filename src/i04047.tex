%(BEGIN_QUESTION)
% Copyright 2009, Tony R. Kuphaldt, released under the Creative Commons Attribution License (v 1.0)
% This means you may do almost anything with this work of mine, so long as you give me proper credit

A pitot tube measuring airspeed on an airplane develops 4.4 inches water column differential pressure at an airspeed of 100 MPH.  Calculate the following:

\begin{itemize}
\item{} Differential pressure at 200 MPH = \underbar{\hskip 50pt}
\vskip 5pt
\item{} Differential pressure at 475 MPH = \underbar{\hskip 50pt}
\vskip 5pt
\item{} Airspeed at 7.3 "W.C. = \underbar{\hskip 50pt}
\vskip 5pt
\item{} Airspeed at 24.1 "W.C. = \underbar{\hskip 50pt}
\end{itemize}

\vskip 20pt \vbox{\hrule \hbox{\strut \vrule{} {\bf Suggestions for Socratic discussion} \vrule} \hrule}

\begin{itemize}
\item{} Why are Pitot tubes ideally suited for airspeed measurement on airplanes?
\end{itemize}

\underbar{file i04047}
%(END_QUESTION)





%(BEGIN_ANSWER)

\begin{itemize}
\item{} Differential pressure at 200 MPH = {\bf 17.6 "W.C.}
\item{} Differential pressure at 475 MPH = {\bf 99.3 "W.C.}
\item{} Airspeed at 7.3 "W.C. = {\bf 128.8 MPH}
\item{} Airspeed at 24.1 "W.C. = {\bf 234 MPH}
\end{itemize}

%(END_ANSWER)





%(BEGIN_NOTES)


%INDEX% Measurement, flow: simple ``k'' factor equation for flow/pressure correlation

%(END_NOTES)


