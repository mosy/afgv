%(BEGIN_QUESTION)
% Copyright 2009, Tony R. Kuphaldt, released under the Creative Commons Attribution License (v 1.0)
% This means you may do almost anything with this work of mine, so long as you give me proper credit

A 5 inch diameter flow nozzle measuring volumetric flowrate of petroleum naphtha ($\gamma$ = 41.5 lb/ft$^{3}$) develops a differential pressure of 75 inches water column at a flow rate of 280 GPM.  Calculate the following:

\begin{itemize}
\item{} Differential pressure at 130 GPM = \underbar{\hskip 50pt}
\vskip 5pt
\item{} Differential pressure at 210 GPM and $\gamma$ = 42.0 lb/ft$^{3}$ = \underbar{\hskip 50pt}
\vskip 5pt
\item{} Flow rate at 40 "W.C. = \underbar{\hskip 50pt}
\vskip 5pt
\item{} Flow rate at 24.1 "W.C. and $\gamma$ = 41.0 lb/ft$^{3}$ = \underbar{\hskip 50pt}
\end{itemize}

\vskip 20pt \vbox{\hrule \hbox{\strut \vrule{} {\bf Suggestions for Socratic discussion} \vrule} \hrule}

\begin{itemize}
\item{} What realistic factors could cause the weight density of a liquid such as naphtha to change from 41.5 lb/ft$^{3}$ to 41.0 or 42.0 lb/ft$^{3}$?
\end{itemize}

\underbar{file i04048}
%(END_QUESTION)





%(BEGIN_ANSWER)

\noindent
{\bf Partial answer:}

\begin{itemize}
\item{} Differential pressure at 210 GPM and $\gamma$ = 42.0 lb/ft$^{3}$ = {\bf 42.70 "W.C.}  
\item{} Flow rate at 40 "W.C. = {\bf 204.5 GPM} 
\end{itemize}



%(END_ANSWER)





%(BEGIN_NOTES)

$$Q = 208.28 \sqrt{\Delta P \over \gamma}$$

\begin{itemize}
\item{} Differential pressure at 130 GPM = {\bf 16.17 "W.C.}
\item{} Differential pressure at 210 GPM and $\gamma$ = 42.0 lb/ft$^{3}$ = {\bf 42.70 "W.C.}  
\item{} Flow rate at 40 "W.C. = {\bf 204.5 GPM} 
\item{} Flow rate at 24.1 "W.C. and $\gamma$ = 41.0 lb/ft$^{3}$ = {\bf 159.7 GPM}
\end{itemize}

\vskip 10pt

The diameter of the flow nozzle is extraneous information, included for the purpose of challenging students to identify whether or not information is relevant to solving a particular problem.










\vfil \eject

\noindent
{\bf Summary Quiz:}

Suppose an orifice plate develops a pressure differential of 100.0 inches water column with a liquid flow rate of 250 GPM going through it.  Calculate the amount of differential pressure this same orifice plate will develop with a liquid flow rate of 170 GPM, assuming all other factors (line pressure, temperature, liquid viscosity, etc.) remain the same.

\begin{itemize}
\item{} 58.92 inches water column 
\vskip 5pt 
\item{} 46.24 inches water column
\vskip 5pt 
\item{} 68.10 inches water column 
\vskip 5pt 
\item{} 100.0 inches water column
\vskip 5pt 
\item{} 63.25 inches water column 
\vskip 5pt 
\item{} 147 inches water column 
\end{itemize}

%INDEX% Measurement, flow: simple ``k'' factor equation for flow/pressure correlation

%(END_NOTES)


