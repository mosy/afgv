%(BEGIN_QUESTION)
% Copyright 2009, Tony R. Kuphaldt, released under the Creative Commons Attribution License (v 1.0)
% This means you may do almost anything with this work of mine, so long as you give me proper credit

A vertical venturi tube measuring mass flowrate of high-pressure steam ($\rho$ = 1.3 lbm/ft$^{3}$) develops a differential pressure of 110 inches water column at a flow rate of 500 pounds per minute.  Calculate the following:

\begin{itemize}
\item{} Differential pressure at 230 lbm/min mass flow = \underbar{\hskip 50pt}
\vskip 5pt
\item{} Differential pressure at 409 lbm/min mass flow and $\rho$ = 1.25 lbm/ft$^{3}$ = \underbar{\hskip 50pt}
\vskip 5pt
\item{} Mass flow rate at 95 "W.C. = \underbar{\hskip 50pt}
\vskip 5pt
\item{} Mass flow rate at 51 "W.C. and $\rho$ = 1.35 lbm/ft$^{3}$ = \underbar{\hskip 50pt}
\end{itemize}

\vskip 20pt \vbox{\hrule \hbox{\strut \vrule{} {\bf Suggestions for Socratic discussion} \vrule} \hrule}

\begin{itemize}
\item{} What significance is there in using the unit of ``pounds mass'' (lbm) instead of ``pounds force'' (lbf) in these quantities?  Is there an appropriate application for using lbf instead?
\item{} What realistic factors could cause the weight density of a vapor such as steam to change from 1.3 lbm/ft$^{3}$ to 1.25 or 1.35 lbm/ft$^{3}$?
\end{itemize}

\underbar{file i04049}
%(END_QUESTION)





%(BEGIN_ANSWER)

\noindent
{\bf Partial answer:}

\begin{itemize}
\item{} Differential pressure at 230 lbm/min mass flow = {\bf 23.28 "W.C.}
\item{} Mass flow rate at 51 "W.C. and $\rho$ = 1.35 lbm/ft$^{3}$ = {\bf 346.9 lbm/min} 
\end{itemize}

%(END_ANSWER)





%(BEGIN_NOTES)

$$W = 41.81 \sqrt{\rho \Delta P}$$

\begin{itemize}
\item{} Differential pressure at 230 lbm/min mass flow = {\bf 23.28 "W.C.}
\item{} Differential pressure at 409 lbm/min mass flow and $\rho$ = 1.25 lbm/ft$^{3}$ = {\bf 76.55 "W.C.}
\item{} Mass flow rate at 95 "W.C. = {\bf 464.7 lbm/min}
\item{} Mass flow rate at 51 "W.C. and $\rho$ = 1.35 lbm/ft$^{3}$ = {\bf 346.9 lbm/min} 
\end{itemize}

\vskip 10pt

The vertical orientation of the venturi tube is extraneous information, included for the purpose of challenging students to identify whether or not information is relevant to solving a particular problem.


%INDEX% Measurement, flow: simple ``k'' factor equation for flow/pressure correlation

%(END_NOTES)


