%(BEGIN_QUESTION)
% Copyright 2009, Tony R. Kuphaldt, released under the Creative Commons Attribution License (v 1.0)
% This means you may do almost anything with this work of mine, so long as you give me proper credit

Suppose you are calibrating a panel-mounted indicator to be used for displaying flow rates, based on the signal coming from a (linear) DP transmitter sensing pressure across a venturi tube.  The DP transmitter is an analog electronic transmitter with a linear characteristic, which means the panel-mounted indicator must be configured for square-root characterization.

Calculate the ideal display value at the following input currents, assuming a calibrated range of 0 to 700 GPM for the indicator:

% No blank lines allowed between lines of an \halign structure!
% I use comments (%) instead, so that TeX doesn't choke.

$$\vbox{\offinterlineskip
\halign{\strut
\vrule \quad\hfil # \ \hfil & 
\vrule \quad\hfil # \ \hfil \vrule \cr
\noalign{\hrule}
%
% First row
Input current & Displayed flow \cr
(mA) & (GPM) \cr
%
\noalign{\hrule}
%
% Another row
4 &  \cr
%
\noalign{\hrule}
%
% Another row
6 &  \cr
%
\noalign{\hrule}
%
% Another row
9.3 &  \cr
%
\noalign{\hrule}
%
% Another row
13 &  \cr
%
\noalign{\hrule}
%
% Another row
14.8 &  \cr
%
\noalign{\hrule}
%
% Another row
20 &  \cr
%
\noalign{\hrule}
} % End of \halign 
}$$ % End of \vbox

\vskip 20pt \vbox{\hrule \hbox{\strut \vrule{} {\bf Suggestions for Socratic discussion} \vrule} \hrule}

\begin{itemize}
\item{} Suppose this analog transmitter were connected to an analog meter to indicate flow.  How could the necessary square-root characterization be performed when all the components are analog and not digital?
\end{itemize}

\underbar{file i04052}
%(END_QUESTION)





%(BEGIN_ANSWER)

\noindent
{\bf Partial answer:}

% No blank lines allowed between lines of an \halign structure!
% I use comments (%) instead, so that TeX doesn't choke.

$$\vbox{\offinterlineskip
\halign{\strut
\vrule \quad\hfil # \ \hfil & 
\vrule \quad\hfil # \ \hfil \vrule \cr
\noalign{\hrule}
%
% First row
Input current & Displayed flow \cr
(mA) & (GPM) \cr
%
\noalign{\hrule}
%
% Another row
4 & 0 \cr
%
\noalign{\hrule}
%
% Another row
6 &  \cr
%
\noalign{\hrule}
%
% Another row
9.3 & 402.9 \cr
%
\noalign{\hrule}
%
% Another row
13 & 525 \cr
%
\noalign{\hrule}
%
% Another row
14.8 &  \cr
%
\noalign{\hrule}
%
% Another row
20 &  \cr
%
\noalign{\hrule}
} % End of \halign 
}$$ % End of \vbox

%(END_ANSWER)





%(BEGIN_NOTES)

% No blank lines allowed between lines of an \halign structure!
% I use comments (%) instead, so that TeX doesn't choke.

$$\vbox{\offinterlineskip
\halign{\strut
\vrule \quad\hfil # \ \hfil & 
\vrule \quad\hfil # \ \hfil \vrule \cr
\noalign{\hrule}
%
% First row
Input current & Displayed flow \cr
(mA) & (GPM) \cr
%
\noalign{\hrule}
%
% Another row
4 & {\bf 0} \cr
%
\noalign{\hrule}
%
% Another row
6 & 247.5 \cr
%
\noalign{\hrule}
%
% Another row
9.3 & {\bf 402.9} \cr
%
\noalign{\hrule}
%
% Another row
13 & {\bf 525} \cr
%
\noalign{\hrule}
%
% Another row
14.8 & 575.1 \cr
%
\noalign{\hrule}
%
% Another row
20 & 700 \cr
%
\noalign{\hrule}
} % End of \halign 
}$$ % End of \vbox

Values shown in bold-faced type are those given to students in the ``Answer'' section.


%INDEX% Measurement, flow: square root characterized indicator
%INDEX% Calibration: table, indicator with square root function

%(END_NOTES)


