
%(BEGIN_QUESTION)
% Copyright 2009, Tony R. Kuphaldt, released under the Creative Commons Attribution License (v 1.0)
% This means you may do almost anything with this work of mine, so long as you give me proper credit

Read and outline the ``Turbine Flowmeters'' subsection of the ``Velocity-Based Flowmeters'' section of the ``Continuous Fluid Flow Measurement'' chapter in your {\it Lessons In Industrial Instrumentation} textbook.  Note the page numbers where important illustrations, photographs, equations, tables, and other relevant details are found.  Prepare to thoughtfully discuss with your instructor and classmates the concepts and examples explored in this reading.

\underbar{file i04055}
%(END_QUESTION)





%(BEGIN_ANSWER)


%(END_ANSWER)





%(BEGIN_NOTES)

Turbine flowmeters use a ``windmill'' turbine in the flow stream to detect fluid flow.  So long as the turbine spins frictionlessly, its blade tip speed will be directly proportional to fluid velocity, making this flowmeter {\it linear} in its response, and giving good turndown (typically 10:1 or better).

\vskip 10pt

Turbine speed is often sensed by a ``pickup'' coil generating a voltage pulse every time a turbine blade passes by the pickup coil.  Mechanical shafts and gears may also be used to transmit the turbine's data to a readable location.  In some designs, a pair of fiber optic cables conducts light to and from the turbine blades, the resulting signal being an optical pulse train.

\vskip 10pt

The mathematical relationship between pulse frequency and volumetric flow rate is a simple proportionality:

$$f = kQ$$

The total amount of fluid volume passed through a turbine flowmeter over any time interval is proportional to the total number of pulses output by the flowmeter.

\vskip 10pt

The American Gas Association Report \#7 (AGA7) standardizes high-accuracy gas flow measurement using turbine flowmeters, and includes compensation for both gas pressure and gas temperature, since these variables both affect gas density and therefore affect the ``standardized'' flow rate of gas for any given velocity.  Turbine flowmeters respond only to the {\it velocity} of the gas, and so any factors affecting gas density will correspondingly affect the measurement of mass flow rate.

\vskip 10pt

Problems faced by turbine flowmeters include ``coasting'' when flow suddenly stops, errors due to friction either at the turbine bearings or due to the fluid's own viscosity slowing down the turbine wheel.  There is a certain ``minimum linear flow'' rate below which a turbine flowmeter will not respond proportionally to flow.






\vskip 20pt \vbox{\hrule \hbox{\strut \vrule{} {\bf Suggestions for Socratic discussion} \vrule} \hrule}

\begin{itemize}
\item{} {\bf In what ways may a turbine flowmeter be ``fooled'' to report a false flow measurement?}
\item{} Explain why turbine flowmeters are linear, based on their operating principle.
\item{} Explain why linear flowmeters exhibit better turndown than pressure-based (square-root) flowmeters.
\item{} Explain how you could test a turbine flowmeter with no test equipment except for a digital multimeter (DMM).
\item{} Explain how you could use the signal from a turbine flowmeter's pickup coil to track the total volume of fluid passed through the meter over a certain timespan.
\item{} Examine the photograph in the textbook of an AGA7 turbine flowmeter installation, and identify the associated instrumentation that makes this a complete flow measurement system suitable for custody transfer of natural gas.
\item{} Examine the photograph in the textbook of a mechanically ``compensated'' turbine flowmeter installation, and explain why temperature and pressure compensation is necessary here.
\item{} Will a swirling flowstream generate a {\it positive} or a {\it negative} flow measurement error in a turbine flowmeter?
\item{} If a turbine flowmeter is forced to measure a fluid that is too viscous, will it register too low or too high?
\item{} If a turbine flowmeter begins to experience bearing friction, will it register too low or too high?
\item{} Suppose the velocity of a gas through a turbine flowmeter remains constant, but the pressure of that gas gradually increases.  Assuming all other factors remain the same, what effect will this change have on the mass flow rate?  Will the turbine meter register this actual rate of gas flow?  Why or why not?
\item{} Suppose the velocity of a gas through a turbine flowmeter remains constant, but the pressure of that gas gradually decreases.  Assuming all other factors remain the same, what effect will this change have on the mass flow rate?  Will the turbine meter register this actual rate of gas flow?  Why or why not?
\item{} Suppose the velocity of a gas through a turbine flowmeter remains constant, but the temperature of that gas gradually increases.  Assuming all other factors remain the same, what effect will this change have on the mass flow rate?  Will the turbine meter register this actual rate of gas flow?  Why or why not?
\item{} Suppose the velocity of a gas through a turbine flowmeter remains constant, but the temperature of that gas gradually decreases.  Assuming all other factors remain the same, what effect will this change have on the mass flow rate?  Will the turbine meter register this actual rate of gas flow?  Why or why not?
\item{} Explain the problem of ``coasting'' in a turbine flowmeter, and propose at least one solution to this problem.
\end{itemize}










\vfil \eject

\noindent
{\bf Prep Quiz:}

{\it Turbine} flowmeters suffer from a unique limitation, not affecting other types of flowmeters.  Identify what this limitation is:

\begin{itemize}
\item{} A square-root extractor is necessary to linearize the output signal
\vskip 5pt 
\item{} It can only be used to measure gas flow streams, not liquid flow streams 
\vskip 5pt 
\item{} Flowmeter indication drops all the way to zero at low flow rates
\vskip 5pt 
\item{} It may only be constructed in very small (less than 1" diameter) pipe sizes
\vskip 5pt 
\item{} It can only be used to measure liquid flow streams, not gas flow streams
\vskip 5pt 
\item{} The meter's indication ``coasts'' a bit when the flow suddenly stops
\vskip 5pt 
\item{} Measurement errors may result if a stray cat finds its way into the pipe
\end{itemize}

%INDEX% Reading assignment: Lessons In Industrial Instrumentation, Continuous Fluid Flow Measurement (turbine flowmeters)

%(END_NOTES)


