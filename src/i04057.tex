%(BEGIN_QUESTION)
% Copyright 2009, Tony R. Kuphaldt, released under the Creative Commons Attribution License (v 1.0)
% This means you may do almost anything with this work of mine, so long as you give me proper credit

Suppose a turbine flowmeter used to measure the flow of natural gas has a ``K factor'' equal to 37.2 pulses per Sl (standard liter).  Calculate the following:

\vskip 10pt

The total amount of gas volume passed through the flowmeter after a digital counter circuit records 2,594,620 pulses.

\vskip 10pt

The flow rate through the meter (i SLM) with the pulse signal having a frequency of 94 Hz.

\vskip 10pt

The amount of time required (in units of hours and minutes) to accumulate 525,000 pulses (on a digital counter circuit) give a steady flow rate of 170 SLM.

\vskip 10pt

Suppose someone entered the wrong K factor value into the digital electronic transmitter connected to the turbine meter's pickup coil.  Would this cause a {\it zero shift}, a {\it span shift}, a {\it linearity error}, or a {\it hysteresis error}?  Explain your reasoning.

\vskip 20pt \vbox{\hrule \hbox{\strut \vrule{} {\bf Suggestions for Socratic discussion} \vrule} \hrule}

\begin{itemize}
\item{} The label ``Standard Cubic Feet'' means one cubic foot of volume with the gas at room temperature and atmospheric (sea-level) pressure.  Explain why we might use the unit of ``Standard Cubic Feet'' to express the flow of a gas through a pipe rather than simple ``Cubic Feet''.
\item{} What advantages does a turbine meter have for measuring natural gas flow that make it well-suited for this application?
\item{} Explain what would be necessary to make a turbine flowmeter register the true {\it mass flow rate} of the fluid rather than just the volumetric flow rate.
\item{} Demonstrate how to {\it estimate} numerical answers for this problem without using a calculator.
\end{itemize}

\underbar{file i04057}
%(END_QUESTION)





%(BEGIN_ANSWER)

\noindent
{\bf Partial answer:}

\vskip 10pt

The amount of time required to accumulate 525,000 pulses (on a digital counter circuit) give a steady flow rate of 170 SLM = {\bf 1 hour, 23 minutes}

%(END_ANSWER)





%(BEGIN_NOTES)

$$\left({2594620 \hbox{ pulses} \over 1} \right) \left( {1 \hbox{ SL} \over 37.2 \hbox{ pulses}} \right) = 69747.8 \hbox{ SL}$$

\vskip 10pt

$$\left( {94 \hbox{ pulses} \over \hbox{sec}} \right) \left( {1 \hbox{ SLM} \over 37.2 \hbox{ pulses}} \right) = 2.5269 \hbox{ SLS} = 151.61 \hbox{ SLM}$$

\vskip 10pt

$$\left( {525000 \hbox{ pulses} \over 1} \right) \left( {1 \hbox{ SLM} \over 37.2 \hbox{ pulses}} \right) \left( {1 \hbox{ min} \over 170 \hbox{ SLM}} \right) = 83.02 \hbox{ minutes}$$

83.02 minutes is equivalent to 1 hour, 23 minutes, 1.02 seconds.

\vskip 10pt

An incorrect K factor entered into the electronic transmitter would result in a {\it span} error, since $k$ is a {\it multiplying} factor in the frequency/flow equation ($f = kQ$), and multiplicative errors are span errors.


%INDEX% Measurement, flow: turbine

%(END_NOTES)


