
%(BEGIN_QUESTION)
% Copyright 2009, Tony R. Kuphaldt, released under the Creative Commons Attribution License (v 1.0)
% This means you may do almost anything with this work of mine, so long as you give me proper credit

Read and outline the ``Vortex Flowmeters'' subsection of the ``Velocity-Based Flowmeters'' section of the ``Continuous Fluid Flow Measurement'' chapter in your {\it Lessons In Industrial Instrumentation} textbook.  Note the page numbers where important illustrations, photographs, equations, tables, and other relevant details are found.  Prepare to thoughtfully discuss with your instructor and classmates the concepts and examples explored in this reading.

\underbar{file i04058}
%(END_QUESTION)





%(BEGIN_ANSWER)


%(END_ANSWER)





%(BEGIN_NOTES)

When fluid moves with a high Reynolds number past a blunt object, alternating vortices form in the wake downstream of that object.  The wavelength of these vortices remains a constant in proportion to the width of the object (the object's width being approximately 17\% of the vortex wavelength), which means the frequency of the vortex shedding is directly proportional to fluid velocity.  A differential pressure sensor mounted downstream of the object picks up this frequency and linearly interprets it as flow rate.

\vskip 10pt

Like turbine flowmeters, the relationship between frequency and volumetric flow is a linear proportionality:

$$f = kQ$$

Also like turbine flowmeters, the total number of pulses accumulated over a time span is proportional to the total volume of fluid passed through the vortex flowmeter.

\vskip 10pt

An important disadvantage of vortex flowmeters is {\it low-flow cutoff}, where the flowmeter's output goes all the way to zero if the flow rate drops below a critical threshold.  This is due to a cessation of vortex shedding, when the Reynolds number of the fluid gets too low: fluid viscosity overwhelms momentum, preventing vortices from forming.









\vskip 20pt \vbox{\hrule \hbox{\strut \vrule{} {\bf Suggestions for Socratic discussion} \vrule} \hrule}

\begin{itemize}
\item{} {\bf In what ways may a vortex flowmeter be ``fooled'' to report a false flow measurement?}
\item{} Thermowells installed in pipes where the flowing velocity is high may suffer from stresses caused by the von K\'arm\'an effect.  Explain why the von K\'arm\'an effect is relevant to thermocouples, which are not even flow-measurement devices.
\item{} Explain why vortex flowmeters are linear, based on their operating principle.
\item{} Explain why vortex flowmeters experience ``low-flow cutoff''.
\item{} Explain how ``low-flow cutoff'' in vortex flowmeters differs from the ``minimum linear flow'' phenomenon experienced by vortex flowmeters.
\item{} Will the cutoff flow rate for a vortex flowmeter be greater, lesser, or the same for liquid molasses compared to water?
\item{} Will the cutoff flow rate for a vortex flowmeter be greater, lesser, or the same for steam at a higher temperature compared to steam at a lower temperature.
\item{} Suppose the velocity of a gas through a vortex flowmeter remains constant, but the pressure of that gas gradually increases.  Assuming all other factors remain the same, what effect will this change have on the mass flow rate?  Will the vortex meter register this actual rate of gas flow?  Why or why not?
\item{} Suppose the velocity of a gas through a vortex flowmeter remains constant, but the pressure of that gas gradually decreases.  Assuming all other factors remain the same, what effect will this change have on the mass flow rate?  Will the vortex meter register this actual rate of gas flow?  Why or why not?
\item{} Suppose the velocity of a gas through a vortex flowmeter remains constant, but the temperature of that gas gradually increases.  Assuming all other factors remain the same, what effect will this change have on the mass flow rate?  Will the vortex meter register this actual rate of gas flow?  Why or why not?
\item{} Suppose the velocity of a gas through a vortex flowmeter remains constant, but the temperature of that gas gradually decreases.  Assuming all other factors remain the same, what effect will this change have on the mass flow rate?  Will the vortex meter register this actual rate of gas flow?  Why or why not?
\end{itemize}











\vfil \eject

\noindent
{\bf Prep Quiz:}

{\it Vortex} flowmeters suffer from a unique limitation, not affecting other types of flowmeters.  Identify what this limitation is:

\begin{itemize}
\item{} A square-root extractor is necessary to linearize the output signal
\vskip 5pt 
\item{} It can only be used to measure gas flow streams, not liquid flow streams 
\vskip 5pt 
\item{} Flowmeter indication drops all the way to zero at low flow rates
\vskip 5pt 
\item{} It may only be constructed in very small (less than 1" diameter) pipe sizes
\vskip 5pt 
\item{} It can only be used to measure liquid flow streams, not gas flow streams
\vskip 5pt 
\item{} The meter's indication ``coasts'' a bit when the flow suddenly stops
\vskip 5pt 
\item{} Measurement errors may result if a stray cat finds its way into the pipe
\end{itemize}


%INDEX% Reading assignment: Lessons In Industrial Instrumentation, Continuous Fluid Flow Measurement (vortex flowmeters)

%(END_NOTES)


