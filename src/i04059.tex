%(BEGIN_QUESTION)
% Copyright 2009, Tony R. Kuphaldt, released under the Creative Commons Attribution License (v 1.0)
% This means you may do almost anything with this work of mine, so long as you give me proper credit

Suppose a vortex flowmeter is used to measure the flow rate of fuel oil into a large combustion boiler.  The vortex meter has a ``K factor'' equal to 10.344 pulses per liter.  Calculate the following:

\vskip 10pt

The sensor frequency at a fuel oil flow rate of 8510 liter per hour.

\vskip 10pt

The total amount of fuel consumed by the boiler after a digital counter circuit records 800,000 pulses.

\vskip 10pt

The fuel oil flow rate (in liter per minute) at a sensor frequency of 35 Hz.

\vskip 10pt

Suppose someone entered the wrong K factor value into the digital electronic transmitter connected to the vortex meter's sensor.  Would this cause a {\it zero shift}, a {\it span shift}, a {\it linearity error}, or a {\it hysteresis error}?  Explain your reasoning.

\vskip 20pt \vbox{\hrule \hbox{\strut \vrule{} {\bf Suggestions for Socratic discussion} \vrule} \hrule}

\begin{itemize}
\item{} Identify how we could set up this vortex flowmeter to record the total amount of fuel oil consumed by the boiler every 24 hours, and then log those values in records for operator reference.
\item{} Explain how you could use simple test equipment to measure the frequency of the signal output by the vortex shedding sensor while the flowmeter was in operation.  Note: some vortex flowmeters provide test points for you to connect electronic test equipment directly to the sensor inside the pipe!
\item{} If the temperature of the fuel oil were to increase slightly, would it affect the vortex flowmeter's measurement accuracy?  Explain why or why not.
\item{} Explain what would be necessary to make a vortex flowmeter register the true {\it mass flow rate} of the fluid rather than just the volumetric flow rate.
\item{} Demonstrate how to {\it estimate} numerical answers for this problem without using a calculator.
\end{itemize}

\underbar{file i04059}
%(END_QUESTION)





%(BEGIN_ANSWER)

\noindent
{\bf Partial answer:}

\vskip 10pt

The total amount of fuel consumed by the boiler after a digital counter circuit records 800,000 pulses = {\bf 77,339.5 liter} 

%(END_ANSWER)





%(BEGIN_NOTES)

$$\left({8510 \hbox{ liter} \over \hbox{hour}} \right) \left( {10.344 \hbox{ pulses} \over \hbox{ liter}} \right) = 88027.44 \hbox{ pulses per hour} = 24.452 \hbox{ Hz}$$

\vskip 10pt

$$\left({800000 \hbox{ pulses} \over 1} \right) \left({ 1 \hbox{ liter} \over 10.344 \hbox{ pulses}} \right) = 77339.5 \hbox{ liter}$$

\vskip 10pt

$$\left({35 \hbox{ pulses} \over \hbox{ sec}} \right) \left({ 1 \hbox{ liter} \over 10.344 \hbox{ pulses}} \right) = 3.3836 \hbox{ liter per second} = 203.02 \hbox{ GPM} = 27.139 \hbox{ ft}^3\hbox{/min}$$

\vskip 10pt

An incorrect K factor entered into the electronic transmitter would result in a {\it span} error, since $k$ is a {\it multiplying} factor in the frequency/flow equation ($f = kQ$), and multiplicative errors are span errors.







\vfil \eject

\noindent
{\bf Summary Quiz:}

A vortex flowmeter installed on a gasoline line has a K factor value of 5.3 pulses per liter.  Calculate the frequency (pulses per second, or Hertz) output by this flowmeter given a gasoline flowrate of 126 liter per minute.

\begin{itemize}
\item{} 11.13 Hz
\vskip 5pt 
\item{} 2.524 Hz
\vskip 5pt 
\item{} 1,426 Hz
\vskip 5pt 
\item{} 23.77 Hz
\vskip 5pt 
\item{} 40,070 Hz
\vskip 5pt 
\item{} 667.8 Hz
\end{itemize}

%INDEX% Measurement, flow: vortex

%(END_NOTES)


