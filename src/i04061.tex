
%(BEGIN_QUESTION)
% Copyright 2009, Tony R. Kuphaldt, released under the Creative Commons Attribution License (v 1.0)
% This means you may do almost anything with this work of mine, so long as you give me proper credit

Read and outline the ``Positive Displacement Flowmeters'' section of the ``Continuous Fluid Flow Measurement'' chapter in your {\it Lessons In Industrial Instrumentation} textbook.  Note the page numbers where important illustrations, photographs, equations, tables, and other relevant details are found.  Prepare to thoughtfully discuss with your instructor and classmates the concepts and examples explored in this reading.

\underbar{file i04061}
%(END_QUESTION)





%(BEGIN_ANSWER)


%(END_ANSWER)





%(BEGIN_NOTES)

Positive displacement flowmeters use an ``engine'' sort of mechanism to shuttle definite volumes of fluid through for each mechanism rotation or cycle.  Some are rotary in nature, while others use pistons, bags, or other elements to measure quantities of fluid passed.  If the mechanism is prevented from moving, fluid cannot move through the flowmeter, in contrast to certain other types of flowmeters such as turbines where cessation of element motion will not impede fluid flow.

\vskip 10pt

These flowmeters are very linear in their response, and totally immune to piping disturbances such as swirls and eddies.  They are, however, susceptible to wear over time due to the requirement of close-fitting moving parts, and for the same reason may be damaged by particulate matter in the flowstream.

\vskip 10pt

If an odometer-style counter is attached to the meter mechanism, it will register the total volume of fluid passed through the meter.





\vskip 20pt \vbox{\hrule \hbox{\strut \vrule{} {\bf Suggestions for Socratic discussion} \vrule} \hrule}

\begin{itemize}
\item{} {\bf In what ways may a positive displacement flowmeter be ``fooled'' to report a false flow measurement?}
\item{} Explain why PD flowmeters are immune to piping disturbances, whereas other flowmeter types (e.g. orifice plate, turbine, etc.) require well-conditioned flow to measure accurately
\item{} Describe how a mechanical device such as a PD flowmeter could be equipped with an electronic output signal representing flow rate.  Be as detailed as you can in describing how the mechanism would have to be modified or augmented.
\item{} Suppose the velocity of a gas through a positive displacement flowmeter remains constant, but the pressure of that gas gradually increases.  Assuming all other factors remain the same, what effect will this change have on the mass flow rate?  Will the positive displacement meter register this actual rate of gas flow?  Why or why not?
\item{} Suppose the velocity of a gas through a positive displacement flowmeter remains constant, but the pressure of that gas gradually decreases.  Assuming all other factors remain the same, what effect will this change have on the mass flow rate?  Will the positive displacement meter register this actual rate of gas flow?  Why or why not?
\item{} Suppose the velocity of a gas through a positive displacement flowmeter remains constant, but the temperature of that gas gradually increases.  Assuming all other factors remain the same, what effect will this change have on the mass flow rate?  Will the positive displacement meter register this actual rate of gas flow?  Why or why not?
\item{} Suppose the velocity of a gas through a positive displacement flowmeter remains constant, but the temperature of that gas gradually decreases.  Assuming all other factors remain the same, what effect will this change have on the mass flow rate?  Will the positive displacement meter register this actual rate of gas flow?  Why or why not?
\end{itemize}


%INDEX% Reading assignment: Lessons In Industrial Instrumentation, Continuous Fluid Flow Measurement (positive displacement flowmeters)

%(END_NOTES)


