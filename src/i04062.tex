%(BEGIN_QUESTION)
% Copyright 2009, Tony R. Kuphaldt, released under the Creative Commons Attribution License (v 1.0)
% This means you may do almost anything with this work of mine, so long as you give me proper credit

A manufacturing facility producing Vaseline jelly uses a positive displacement flowmeter equipped with an electronic ``pickup'' switch to measure the flow of jelly to a storage tank.  This particular flowmeter outputs one pulse for every 20 gallons passing through.  Calculate the following:

\vskip 10pt

The pulse frequency given a flow rate of 310 cubic feet per minute.

\vskip 10pt

The total volume of liquid passed causing an accumulation of 48,522 pulses by an electronic counter circuit connected to the meter's ``pickup'' switch.

\vskip 10pt

The effect on measurement accuracy if the temperature of the Vaseline jelly increases enough to have a substantial effect on viscosity.

\vskip 20pt \vbox{\hrule \hbox{\strut \vrule{} {\bf Suggestions for Socratic discussion} \vrule} \hrule}

\begin{itemize}
\item{} Why do you suppose a positive displacement flowmeter is a good choice for this process fluid application?
\item{} Are positive displacement flowmeters linear or nonlinear?  Explain your answer.
\item{} What would happen if we tried to use an orifice plate to measure the flow rate of Vaseline?
\item{} What would happen if we tried to use a turbine flowmeter to measure the flow rate of Vaseline?
\item{} What would happen if we tried to use a vortex flowmeter to measure the flow rate of Vaseline?
\item{} Demonstrate how to {\it estimate} numerical answers for this problem without using a calculator.
\end{itemize}

\underbar{file i04062}
%(END_QUESTION)





%(BEGIN_ANSWER)

\noindent
{\bf Partial answer:}

\vskip 10pt

The pulse frequency given a flow rate of 310 cubic feet per minute = {\bf 1.932 Hz}

\vskip 10pt

Viscosity is of no concern with a positive-displacement flowmeter!

%(END_ANSWER)





%(BEGIN_NOTES)

$$\left( {310 \hbox{ ft}^3 \over \hbox{min}} \right)  \left( {1728 \hbox{ in}^3 \over 1 \hbox{ ft}^3} \right)  \left({1 \hbox{ gal} \over 231 \hbox{ in}^3}\right)   \left( {1 \hbox{ pulse} \over 20 \hbox{ gal}} \right) = 115.948 \hbox{ pulses per minute} = 1.932 \hbox{ Hz}$$

\vskip 10pt

$$\left({48522 \hbox{ pulses}\over 1}\right) \left({20 \hbox{ gal} \over 1 \hbox{ pulse}}\right) = 970440 \hbox{ gallons}$$

\vskip 10pt

Viscosity is of no concern with a positive-displacement flowmeter, because this type of flowmeter simply shuttles definite volumes of fluid with each pulse with no regard for turbulence or friction.

%INDEX% Measurement, flow: positive displacement

%(END_NOTES)


