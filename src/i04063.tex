
%(BEGIN_QUESTION)
% Copyright 2009, Tony R. Kuphaldt, released under the Creative Commons Attribution License (v 1.0)
% This means you may do almost anything with this work of mine, so long as you give me proper credit

Read pages 2-3 through 2-5 of the ``Rosemount Model 8800C and Model 8800A Smart Vortex Flowmeter'' reference manual (publication 00809-0100-4003 Revision JA), and answer the following questions:

\vskip 10pt

Explain why a vertical pipe orientation is preferred for this type of flowmeter, identifying the proper direction(s) of flow for different process fluids.

\vskip 10pt

Figure 2-2 shows preferred mounting positions for hot pipes -- explain why these positions are preferred to other alternative positions.

\vskip 10pt

Identify the minimum upstream and downstream straight-pipe lengths for this flowmeter. 

\vskip 10pt

Figure 2-9 on page 2-12 shows the bolt-tightening sequence recommended for flange-mounted flowmeter installations.  Examine each of the sequences shown, and explain why the sequence of bolt-tightening matters.  Hint: the exact same principle is involved when tightening lug nuts on a car wheel, and it is called {\it cross-torquing}.

\vskip 20pt \vbox{\hrule \hbox{\strut \vrule{} {\bf Suggestions for Socratic discussion} \vrule} \hrule}

\begin{itemize}
\item{} Explain why the manual recommends you ``install valves downstream of the meter when possible''.
\item{} This manual mentions the option of pressure and temperature compensation for the vortex flowmeter.  Explain why one might choose to apply this type of compensation in a specific process application.  Also, explain why compensating pressure and temperature sensors should be located downstream of the vortex flowmeter rather than upstream.
\item{} Suppose you needed to ``cross-torque'' the bolts on a machine component, but did not have a manual to specify which bolts to torque in what order.  Explain how you could apply a general cross-torquing procedure to {\it any} multi-bolt application.
\item{} Suppose you were asked to build a circuit to interpret the pulse output from this model of vortex flowmeter, blinking an LED on and off with the pulse frequency.  Sketch this circuit, being sure to note which screw terminals on the flowmeter to connect your circuit to.
\end{itemize}

\underbar{file i04063}
%(END_QUESTION)





%(BEGIN_ANSWER)


%(END_ANSWER)





%(BEGIN_NOTES)

Vertical orientation is preferred because this guarantees a full meter body (no gas pockets or condensed liquids off to one side of the tube) and allows equal distribution of solids.  Liquids should flow upward, while gases and vapors may flow in either direction.  Downward liquid flow is possible with the correct piping design (maintaining adequate backpressure to ensure a filled pipe).  The flowmeter should never be installed in a low point in a steam application, to ensure liquid condensate does not pool up in such a low spot.

\vskip 10pt

The electronics ``head'' should be located below the pipe centerline to discourage hot air convecting heat from the pipe to the electronics.  Of course the proximity to {\it other} hot pipes should be taken into consideration as well!

\vskip 10pt

10D upstream and 5D downstream, minimum.  Depending on the nature and severity of an upstream flow disturbance, the K factor may be skewed as much as 5\% even for straight-length pipe sections of 10 diameters to 35 diameters.  (Page 2-5)

\vskip 10pt

{\it Cross-torquing} flange bolts helps to ensure an even ``crush'' of the gasket, and helps to avoid warping the flanges due to uneven bolt pressures.

\vskip 10pt

For more information on this flowmeter's pulse output, refer to pages 2-28 to 2-30.

%INDEX% Reading assignment: Rosemount 8800 vortex flow transmitter Reference manual

%(END_NOTES)


