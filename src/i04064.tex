
%(BEGIN_QUESTION)
% Copyright 2009, Tony R. Kuphaldt, released under the Creative Commons Attribution License (v 1.0)
% This means you may do almost anything with this work of mine, so long as you give me proper credit

Read and outline the ``Magnetic Flowmeters'' subsection of the ``Velocity-Based Flowmeters'' section of the ``Continuous Fluid Flow Measurement'' chapter in your {\it Lessons In Industrial Instrumentation} textbook.  Note the page numbers where important illustrations, photographs, equations, tables, and other relevant details are found.  Prepare to thoughtfully discuss with your instructor and classmates the concepts and examples explored in this reading.

\underbar{file i04064}
%(END_QUESTION)





%(BEGIN_ANSWER)


%(END_ANSWER)





%(BEGIN_NOTES)

Any electrical conductor moving perpendicularly to a magnetic field will experience an induced voltage perpendicular to both the field and the direction of motion.  Conductive liquids moving through a magnetic field do the same thing.  The amount of voltage induced is directly proportional to flow velocity, making this a {\it linear} type of flowmeter:

$$E = Blv = {B d Q \over A} = {4 B Q \over \pi d}$$

$$Q = k {\pi d E \over 4B}$$

\noindent
Where,

$E$ = Motional EMF (volts)

$B$ = Magnetic flux density (Tesla)

$l$ = Length of conductor passing through the magnetic field (meters)

$v$ = Velocity of conductor (meters per second)

$d$ = Diameter of pipe (meters)

$A$ = Cross-sectional flowing area of pipe (square meters)

$Q$ = Flow rate (cubic meters per second)

$k$ = Constant of proportionality

\vskip 10pt

In order for a magnetic flowmeter to properly function, the fluid must be conductive, completely filling the pipe, and the flowtube must be electrically grounded to prevent stray electric currents from interfering with the flow measurement.  Ideally, the flow should be vertical (upward) to ensure full pipe fillage.  If horizontal, the electrodes should be oriented horizontally as well so that random gas bubbles passing through do not break contact with the liquid.

\vskip 10pt

Liquid conductivity does not have a large effect on flow measurement accuracy, so long as the conductivity is arbirtraily high (i.e. the resistivity is very low).  Only certain liquid types such as deionized water, petroluem fuels, and other completely non-conductive liquids are unsuitable for this technology.

\vskip 10pt

Magflow meters are fairly tolerant of flow disturbances, typically only requiring 5 diameters of upstream straight-pipe length and 3 downstream.

\vskip 10pt

Electrical grounding straps are necessary to bypass stray electric currents around the flowtube (and to ground).  If the piping is non-conductive (e.g. plastic), grounding rings must be used to make contact between the grounding straps and the liquid both before and after the flowtube.

\vskip 10pt

Flowtube coils are often energized with AC rather than DC to avoid problems due to ionic polarization.  60 Hz AC excitation can be problematic because the induced EMF will also be 60 Hz, and therefore indistinguishable from 60 Hz noise.  Pulsed ``DC'' flowmeters use square-wave pulses to excite the field coils, so that the induced voltage will be easily distinguishable from any 60 Hz noise.







\filbreak

\vskip 20pt \vbox{\hrule \hbox{\strut \vrule{} {\bf Suggestions for Socratic discussion} \vrule} \hrule}

\begin{itemize}
\item{} {\bf In what ways may a magnetic flowmeter be ``fooled'' to report a false flow measurement?}
\item{} Identify the conditions which must be met in order for a magnetic flowmeter to properly sense fluid flow.
\item{} Identify some distinct advantages magnetic flowmeters enjoy over DP-based, turbine, vortex, and positive displacement.
\item{} Explain how the simple formula for motional EMF is converted into a full flow formula for magnetic flowmeters, following the steps shown in the textbook.
\item{} Explain some of the different methods used to energize the field windings of magnetic flowmeters, and how they attempt to mitigate noise voltages present in the piping.
\item{} Suppose the electrical conductivity of the liquid suddenly increases, while nothing else changes.  Will a magnetic flowmeter register more flow, less flow, or the same amount of flow as before?
\item{} Suppose the velocity of a liquid through a magnetic flowmeter remains constant, but the temperature of that liquid gradually increases.  Assuming all other factors remain the same, what effect will this change have on the mass flow rate?  Will the magnetic flowmeter register this actual rate of liquid flow?  Why or why not?
\item{} Suppose the velocity of a liquid through a magnetic flowmeter remains constant, but the viscosity of that liquid gradually increases.  Assuming all other factors remain the same, what effect will this change have on the mass flow rate?  Will the magnetic flowmeter register this actual rate of liquid flow?  Why or why not?
\item{} Suppose the velocity of a liquid through a magnetic flowmeter remains constant, but the density of that liquid gradually increases.  Assuming all other factors remain the same, what effect will this change have on the mass flow rate?  Will the magnetic flowmeter register this actual rate of liquid flow?  Why or why not?
\item{} Examining the photograph of the large (36" diameter) flowtube shown in the textbook, identify the proper {\it direction} that wastewater should be flowing through it.  Explain the rationale for this flow direction.
\item{} Explain what {\it ionic polarization} is in a magnetic flowmeter, describing what it would do to flow measurement, and also how we combat this effect.
\item{} Describe the difference between ``AC'' magnetic flowmeters and ``DC'' magnetic flowmeters.
\end{itemize}








\vfil \eject

\noindent
{\bf Prep Quiz:}

{\it Magnetic} flowmeters suffer from a unique limitation, not affecting other types of flowmeters.  Identify what this limitation is:

\begin{itemize}
\item{} A square-root extractor is necessary to linearize the output signal
\vskip 5pt 
\item{} It can only be used to measure electrically conductive fluid streams
\vskip 5pt 
\item{} Flowmeter indication drops all the way to zero at low flow rates
\vskip 5pt 
\item{} It may only be constructed in very small (less than 1" diameter) pipe sizes
\vskip 5pt 
\item{} Its calibration accuracy depends on the speed of sound through the fluid
\vskip 5pt 
\item{} The meter's indication ``coasts'' a bit when the flow suddenly stops
\end{itemize}

%INDEX% Reading assignment: Lessons In Industrial Instrumentation, Continuous Fluid Flow Measurement (magnetic flowmeters)

%(END_NOTES)


