
%(BEGIN_QUESTION)
% Copyright 2009, Tony R. Kuphaldt, released under the Creative Commons Attribution License (v 1.0)
% This means you may do almost anything with this work of mine, so long as you give me proper credit

Read pages 2-2 through 2-18 of the ``Rosemount Series 8700 Magnetic Flowmeter Flowtubes'' reference manual (publication 00809-0100-4727 Revision DA), and answer the following questions:

\vskip 10pt

Identify the minimum upstream and downstream straight-pipe runs necessary for reliable flow measurement using one of these magnetic flowmeters.

\vskip 10pt

Two cables connect the remotely-mounted transmitter (``head'') unit to the flowtube.  Identify the purpose of each cable; specifically, what each one connects to inside the flowtube.

\vskip 10pt

Identify the proper direction of process liquid flow when the flowtube is mounted vertically or at an angle, and explain why this is the preferred direction.

\vskip 20pt \vbox{\hrule \hbox{\strut \vrule{} {\bf Suggestions for Socratic discussion} \vrule} \hrule}

\begin{itemize}
\item{} Explain why it is important to \underbar{not} run cables from two different magnetic flow transmitters to their respective flowtube assemblies through the same electrical conduit.
\item{} Explain why cable termination procedures must be strictly adhered to, including not stripping back the cable shield more than half an inch, and also bonding the shield conductors (only) to the flowtube case.
\item{} Comment on the flange bolt torquing sequences shown on page 2-7.  Are these sequences arbitrary, or is there some general principle we should recognize here?
\end{itemize}

\underbar{file i04066}
%(END_QUESTION)





%(BEGIN_ANSWER)


%(END_ANSWER)





%(BEGIN_NOTES)

5D upstream and 2D downstream, minimum (page 2-3).

\vskip 10pt

One cable to energize the flowtube's coils (``coil drive'' cable), and another cable to bring the electrodes' signal back to the transmitter for processing (``electrode'' cable).  (Pages 2-13 through 2-17).

\vskip 10pt

Process liquid flow should always be {\it up} (in the absence of sufficient liquid backpressure), to ensure a completely full tube (pages 2-3 through 2-5).

%INDEX% Reading assignment: Rosemount 8700 magflow tube Reference manual

%(END_NOTES)


