
%(BEGIN_QUESTION)
% Copyright 2009, Tony R. Kuphaldt, released under the Creative Commons Attribution License (v 1.0)
% This means you may do almost anything with this work of mine, so long as you give me proper credit

Read and outline the ``Ultrasonic Flowmeters'' subsection of the ``Velocity-Based Flowmeters'' section of the ``Continuous Fluid Flow Measurement'' chapter in your {\it Lessons In Industrial Instrumentation} textbook.  Note the page numbers where important illustrations, photographs, equations, tables, and other relevant details are found.  Prepare to thoughtfully discuss with your instructor and classmates the concepts and examples explored in this reading.

\underbar{file i04068}
%(END_QUESTION)





%(BEGIN_ANSWER)


%(END_ANSWER)





%(BEGIN_NOTES)

Ultrasonic flowmeters work by passing high-frequency sound wave pulses through the moving fluid.  Doppler flowmeters infer fluid velocity in aereated or dirty liquid flows by comparing the incident and received frequencies when the sound waves strike a bubble or suspended solid in the liquid.  The frequency of the reflected pulse will shift proportional to velocity:

$$\Delta f = {{2vf \cos \theta} \over {c}}$$

$$Q = {{Ac \Delta f} \over {2 f \cos \theta}}$$

Note that the speed of sound through the liquid ($c$) affects the measurement of flow rate.  The speed of sound through any material is a function of its bulk modulus (how compressible it is) and its mass density.  Chemical composition affects bulk modulus, while pressure and temperature can affect density:

$$c = \sqrt{B \over \rho}$$

\vskip 10pt

Transit-time ultrasonic flowmeters require clean flows (either liquid or gas), and infer flow velocity by comparing the propagation time of a sound pulse sent upstream versus one sent downstream:

$$Q = k {t_{up} - t_{down} \over (t_{up})(t_{down})} = {2 k v \over L}$$

Note how the speed of sound through the fluid ($c$) is absent from this formula, telling us that that this flowmeter is immune to changes in the fluid's speed of sound.  The speed of sound can be measured from the transit time values, however, and this has diagnostic value for the flowmeter (to check the condition of its ultrasonic transducers):

$$c = {L \over 2} \left({t_{up} + t_{down} \over (t_{up})(t_{down})}\right)$$

Multipath ultrasonic flowmeters achieve high accuracy by shooting sound wave pulses across the pipe along different paths (``chords''), capturing different portions of the velocity profile.  The velocity measured by each chord may be expressed as a ratio to the average flow velocity, called a {\it velocity ratio}.  Comparisons of velocity ratios near the pipe's middle versus near the pipe's walls gives a quantitative indication of flow profile shape called the {\it profile factor}.  This ratio of inner velocity to outer velocity will change with installation, with flowtube fouling, and also with sensor malfunctions, making it a useful diagnostic indicator.

Another useful diagnostic indicator is the speed of sound registered by each chord, which should be identical for all conditions.  Any significant differences here indicate problems with one or more chord transducers.

\vskip 10pt

High-accuracy gas flow meausurement using multipath transit-time ultrasonic flowmeters is possible, and codified under American Gas Association report \#9 (AGA9).

\vskip 10pt

Some applications allow for ultrasonic transducers to be clamped on the outside of a pipe, for temporary, non-intrusive flow measurement.










\filbreak

\vskip 20pt \vbox{\hrule \hbox{\strut \vrule{} {\bf Suggestions for Socratic discussion} \vrule} \hrule}

\begin{itemize}
\item{} {\bf In what ways may an ultrasonic flowmeter be ``fooled'' to report a false flow measurement?}
\item{} Explain how the simple formula for Doppler frequency shift is converted into a full flow formula for Doppler ultrasonic flowmeters, following the steps shown in the textbook.
\item{} Identify the factors influencing the speed of sound through any substance.
\item{} Identify at least one advantage of Doppler ultrasonic flowmeters over transit-time ultrasonic flowmeters.
\item{} Identify at least one advantage of transit-time ultrasonic flowmeters over Doppler ultrasonic flowmeters.
\item{} Explain why some ultrasonic flowmeters use multiple paths (chords) to sense the flowing velocity of fluid.
\item{} Identify some of the different diagnostic methods employed by multi-path ultrasonic flowmeters to self-check their operation and process conditions.
\item{} Which chord(s) in a multipath ultrasonic flowmeter should exhibit the largest velocity ratio, and why?
\item{} Which chord(s) in a multipath ultrasonic flowmeter should exhibit the smallest velocity ratio, and why?
\item{} Which flow regime (laminar or turbulent) should exhibit the largest profile factor, and why?
\item{} Which flowmeter type -- magnetic or ultrasonic -- is most affected by large-scale turbulence, and why?
\item{} Suppose the velocity of a liquid through a transit-time ultrasonic flowmeter remains constant, but the temperature of that liquid gradually decreases.  Assuming all other factors remain the same, what effect will this change have on the mass flow rate?  Will the ultrasonic flowmeter register this actual rate of liquid flow?  Why or why not?
\item{} Suppose the velocity of a liquid through a transit-time ultrasonic flowmeter remains constant, but the density of that liquid gradually decreases.  Assuming all other factors remain the same, what effect will this change have on the mass flow rate?  Will the ultrasonic flowmeter register this actual rate of liquid flow?  Why or why not?
\item{} Suppose the velocity of a liquid through a magnetic flowmeter remains constant, but the viscosity of that liquid gradually increases.  Assuming all other factors remain the same, what effect will this change have on the mass flow rate?  Will the magnetic flowmeter register this actual rate of liquid flow?  Why or why not?
\item{} Suppose the liquid density through a Doppler ultrasonic flowmeter increases.  What effect (if any at all) will this have on the flowmeter's ability to accurately sense fluid velocity through it?
\item{} Explain how ultrasonic flowmeters may be adapted to perform high-accuracy measurements of natural gas flow.  Are these similar techniques to any other type of flowmeter employed to measure natural gas flow?
\end{itemize}










\vfil \eject

\noindent
{\bf Summary Quiz:}

Which type of flowmeter relies on a constant and known {\it speed of sound} to achieve accurate flow measurement?

\begin{itemize}
\item{} Turbine
\vskip 5pt 
\item{} Positive displacement
\vskip 5pt 
\item{} Transit time
\vskip 5pt 
\item{} Vortex
\vskip 5pt 
\item{} Venturi tube
\vskip 5pt 
\item{} Doppler
\end{itemize}


%INDEX% Reading assignment: Lessons In Industrial Instrumentation, Continuous Fluid Flow Measurement (ultrasonic flowmeters)

%(END_NOTES)


