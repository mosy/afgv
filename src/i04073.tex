
%(BEGIN_QUESTION)
% Copyright 2009, Tony R. Kuphaldt, released under the Creative Commons Attribution License (v 1.0)
% This means you may do almost anything with this work of mine, so long as you give me proper credit

Read and outline the ``Coriolis Flowmeters'' subsection of the ``True Mass Flowmeters'' section of the ``Continuous Fluid Flow Measurement'' chapter in your {\it Lessons In Industrial Instrumentation} textbook.  Note the page numbers where important illustrations, photographs, equations, tables, and other relevant details are found.  Prepare to thoughtfully discuss with your instructor and classmates the concepts and examples explored in this reading.

\underbar{file i04073}
%(END_QUESTION)





%(BEGIN_ANSWER)


%(END_ANSWER)





%(BEGIN_NOTES)

Coriolis flowmeters work by shaking a tube carrying a fluid.  The mass density of that fluid will affect the tubes' resonant frequency, and so frequency becomes an inversely proportional representation of fluid density.  As the fluid moves through the vibrating tubes, its inertia causes the tubes to undulate.  This undulation or twisting motion is directly proportional to mass flow rate.

\vskip 10pt

The Coriolis force is a ``pseudoforce'' in physics: a force that appears to exist when viewed from a non-inertial reference frame.  It arises when a mass moves perpendicular to an axis of rotation.  Rather than rotate, the mass tries to move in a straight line.  When viewed from the rotating reference frame, however, it appears as though the mass tries to follow a curved trajectory.

\vskip 10pt

An example of the Coriolis force is apparent if we were to modify a rotary lawn sprinkler to have the water shoot straight out rather than through angled nozzles.  In such a mechanism, the flowing water's inertia will act to oppose any rotation of the sprinkler tubes.  The more mass flow we have through those tubes, the more it will oppose any external attempt at rotation.

Another example is seen if we try to wave a garden hose back and forth while water flows through it: the intertia of the flowing water makes the hose's end lag, because the water's mass ``tries'' to move in a straight line, and opposes any attempts to change its direction.

\vskip 10pt

A practical Coriolis flowmeter design moves the fluid through a U-shaped tube, shaking the looped end of that tube back and forth.  Inertia of the flowing fluid causes that U-shaped tube to {\it twist}.  The more mass flow rate, the greater the intertial forces, and therefore the greater the amount of twist.  Usually, the flowmeter passes fluid through a pair of balanced tubes, those tubes vibrating 180$^{o}$ out of phase with each other, both to minimize vibration and also to minimize the effects of external vibration on the instrument's operation.  A force coil mounted between the tubes at the curved end shakes them back and forth relative to each other, while a pair of sensing coils measures the amount of twist.  

The frequency of the signals coming from the sensor coils indicate fluid density (higher frequency = less dense), while the phase shift between the two coils' signals represents mass flow rate (more phase shift = more mass flow rate).

\vskip 10pt

The vibrating tubes of a Coriolis flowmeter are matched spring elements, and as such their mechanical properties must be precisely known in order to achieve good measurement accuracy.  For this reason, tubes must be matched to the electronics package of a Coriolis flowmeter from the factory.  If the tubes are changed for whatever reason, the electronics must be re-programmed to match the new tubes.

Temperature affects the spring constant of metal, and therefore Coriolis flowmeters must be equipped with tube temperature sensors to allow the electronics to compensate for changes in tube stiffness resulting from changes in fluid temperature.  This sensing of temperature is also available as a variable (in addition to density and mass flow rate), making the Coriolis flowmeter a multi-variable instrument.

\vskip 10pt

Coriolis flowmeters may infer volumetric flow rate, by dividing the true mass flow ($W$) by the sensed liquid density ($\rho$).

\vskip 10pt

Coriolis flowmeters are linear and highly accurate, and are also completely immune to disturbances in the flow (i.e. no need for straight-runs of pipe upstream or downstream).  The American Gas Association has standardized the use of Coriolis flowmeters for custody transfer of natural gas in their Report \#11 (AGA11), just as they have for orifice plates (AGA3), turbine meters (AGA7), and ultrasonic meters (AGA9).

\vskip 10pt

Disadvantages of Corilios flowmeters include high cost, limited operating temperature range, poor sensitivity to light (non-dense) fluids, and difficulty of in-place cleaning (for sanitary applications).


\filbreak







\vskip 20pt \vbox{\hrule \hbox{\strut \vrule{} {\bf Suggestions for Socratic discussion} \vrule} \hrule}

\begin{itemize}
\item{} {\bf In what ways may a Coriolis flowmeter be ``fooled'' to report a false flow measurement?}
\item{} Describe the operation of a Coriolis flowmeter in the simplest terms possible.
\item{} Examine the illustration of the ``anti-rotating'' lawn sprinkler shown in the book, and describe how this mechanism would react if the water flow rater were suddenly {\it decreased}.
\item{} Examine the illustration of the ``anti-rotating'' lawn sprinkler shown in the book, and describe how this mechanism would react if the water were replaced by alcohol (less dense) at the same volumetric flow rate as before.
\item{} Examine a the cutaway photographs of a Coriolis flowmeter shown in the textbook and describe the functions of the various components.
\item{} Explain why the temperature of the tubes in a Coriolis flowmeter must be monitored.
\item{} Explain why one cannot simply swap electronics and flowtube units between two Coriolis meters without recalibration.
\item{} Suppose the amount of phase shift measured between the two sensing coils of a Coriolis flowmeter suddenly increases, while the frequency remains unchanged.  What does this tell us about the fluid flowing through the flowmeter?
\item{} Suppose the amount of phase shift measured between the two sensing coils of a Coriolis flowmeter suddenly decreases, while the frequency remains unchanged.  What does this tell us about the fluid flowing through the flowmeter?
\item{} Suppose the frequency of the sensing coil signals in a Coriolis flowmeter suddenly decreases.  What does this tell us about the fluid flowing through the flowmeter?
\item{} Suppose the frequency of the sensing coil signals in a Coriolis flowmeter suddenly increases.  What does this tell us about the fluid flowing through the flowmeter?
\item{} Examining the tensed-string resonant frequency formula, identify the effect that decreased temperature will have on the resonant frequency of a fluid-filled tube (assuming constant fluid density).
\item{} Suppose the temperature compensation inside a Coriolis flowmeter fails, such that the flowmeter ``thinks'' the tubes' temperature is constant when it has actually increased.  Assuming the same mass flow rate and fluid density as before, what effect(s) will this discrepancy have on the meter's measurements of fluid flow rate and fluid density?
\end{itemize}









\vfil \eject

\noindent
{\bf Prep Quiz:}

Coriolis flowmeters are actually {\it multi-variable} instruments, measuring which three process variables?

\begin{itemize}
\item{} Temperature, viscosity, and pressure
\vskip 5pt 
\item{} Mass flow, speed of sound, and viscosity
\vskip 5pt 
\item{} Differential pressure, temperature, and flow
\vskip 5pt 
\item{} Mass flow, temperature, and density
\vskip 5pt 
\item{} Opacity, mass flow, and volumetric flow
\vskip 5pt 
\item{} Volumetric flow, pH, and pressure
\end{itemize}

%INDEX% Reading assignment: Lessons In Industrial Instrumentation, Continuous Fluid Flow Measurement (Coriolis flow measurement)

%(END_NOTES)


