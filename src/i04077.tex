
%(BEGIN_QUESTION)
% Copyright 2009, Tony R. Kuphaldt, released under the Creative Commons Attribution License (v 1.0)
% This means you may do almost anything with this work of mine, so long as you give me proper credit

Read and outline the ``Thermal Flowmeters'' subsection of the ``True Mass Flowmeters'' section of the ``Continuous Fluid Flow Measurement'' chapter in your {\it Lessons In Industrial Instrumentation} textbook.  Note the page numbers where important illustrations, photographs, equations, tables, and other relevant details are found.  Prepare to thoughtfully discuss with your instructor and classmates the concepts and examples explored in this reading.

\underbar{file i04077}
%(END_QUESTION)





%(BEGIN_ANSWER)


%(END_ANSWER)





%(BEGIN_NOTES)

Thermal flowmeters exploit the phenomenon of {\it wind chill} (convective cooling of a heated object) to measure true mass flow rate.  The greater the mass flow rate of a cooler fluid past a heated object, the faster that heated object will lose heat due to convection.  A ``hot wire anemometer'' is a simple type of thermal flowmeter, measuring air flow by sensing the temperature of a hot wire, and modulating current through that wire to maintain its temperature steady.  The more air flow past the hot wire, the more current will be necessary to maintain its temperature -- therefore, current becomes proportional to mass air flow rate.

\vskip 10pt

Industrial thermal flowmeters use two temperature sensors: one bonded to the heating element, and the other used to sense the ``ambient'' temperature of the incoming fluid.

\vskip 10pt

{\it Specific heat} is a fluid parameter that affects the accuracy of thermal flowmeters.  Different fluid types have different heat capacities (i.e. different capacities to absorb heat for a given temperature change), and therefore will skew the response of a thermal flowmeter.  This means the chemical composition of the fluid must be stable and known in order to use a thermal mass flowmeter.








\vskip 20pt \vbox{\hrule \hbox{\strut \vrule{} {\bf Suggestions for Socratic discussion} \vrule} \hrule}

\begin{itemize}
\item{} {\bf In what ways may a thermal flowmeter be ``fooled'' to report a false flow measurement?}
\item{} Suppose a thermal flowmeter is measuring the mass flow rate of nitrogen gas, and then suddenly the nitrogen is replaced by hydrogen gas (having a much greater specific heat value).  All other factors being the same as before, will the flowmeter's indication increase, decrease, or remain unchanged?
\item{} Suppose a thermal flowmeter is measuring the mass flow rate of hydrogen gas, and then suddenly the hydrogen is replaced by helium gas (having a lesser specific heat value).  All other factors being the same as before, will the flowmeter's indication increase, decrease, or remain unchanged?
\item{} Suppose a thermal flowmeter is measuring the mass flow rate of oxygen gas, and then suddenly the incoming gas temperature rises.  All other factors being the same as before, will the flowmeter's indication increase, decrease, or remain unchanged?
\item{} Suppose a thermal flowmeter is measuring the mass flow rate of air, and then suddenly the incoming gas temperature falls.  All other factors being the same as before, will the flowmeter's indication increase, decrease, or remain unchanged?
\item{} Suppose a thermal flowmeter is measuring the mass flow rate of some fluid, and then that flow rate drops down to a point where the flow regime changes from turbulent to laminar.  All other factors being the same as before, will the flowmeter still accurately register the true mass flow rate of this fluid?  If not, will the new indication be inflated (greater than it should be) or deflated (less than it should be)?
\item{} Explain how a heat exchanger could (theoretically) be employed as a {\it thermal mass flowmeter}.  What non-flow instrumentation would have to be connected to a heat exchanger to allow it to sense mass flow of some fluid going through it?
\end{itemize}







\vfil \eject

\noindent
{\bf Prep Quiz:}

{\it Thermal} mass flowmeters suffer from a unique limitation, not affecting other types of flowmeters.  Identify what this limitation is:

\begin{itemize}
\item{} A square-root extractor is necessary to linearize the output signal
\vskip 5pt 
\item{} It can only be used to measure electrically conductive fluid streams
\vskip 5pt 
\item{} Its calibration accuracy depends on the fluid's specific heat
\vskip 5pt 
\item{} It may only be constructed in very large (more than 24" diameter) pipe sizes
\vskip 5pt 
\item{} Its calibration accuracy depends on the speed of sound through the fluid
\vskip 5pt 
\item{} The meter's indication ``coasts'' a bit when the flow suddenly stops
\end{itemize}


%INDEX% Reading assignment: Lessons In Industrial Instrumentation, Continuous Fluid Flow Measurement (Thermal flow measurement)

%(END_NOTES)


