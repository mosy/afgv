%(BEGIN_QUESTION)
% Copyright 2009, Tony R. Kuphaldt, released under the Creative Commons Attribution License (v 1.0)
% This means you may do almost anything with this work of mine, so long as you give me proper credit

Perform a ``thought experiment'' where natural gas moves through a thermal mass flowmeter having just one (heated) RTD temperature sensing element.  Explain what happens to the temperature of this element as the gas flow rate increases and decreases, and how the flowmeter's electronics would interpret this temperature change as a change in flow.

\vskip 10pt

Now, perform another ``thought experiment'' where a constant flow of natural gas changes temperature as it moves through a thermal mass flowmeter having just one (heated) RTD temperature sensing element.  Explain what happens to the temperature of this element as the incoming gas increases and decreases in temperature, and how the flowmeter's electronics would interpret this temperature change as a change in flow.

\vskip 10pt

Finally, explain why all thermal flowmeters are built with {\it two} temperature sensors, one heated and one unheated.

\vskip 20pt \vbox{\hrule \hbox{\strut \vrule{} {\bf Suggestions for Socratic discussion} \vrule} \hrule}

\begin{itemize}
\item{} A strong emphasis is placed on performing ``thought experiments'' in this course.  Explain why this is.  What practical benefits might students realize from regular mental exercises such as this?
\item{} Do you think a thermal mass flowmeter would be a good candidate technology for {\it natural gas} flow metering?  Explain why or why not.
\end{itemize}

\underbar{file i04078}
%(END_QUESTION)





%(BEGIN_ANSWER)

As flow increases, temperature decreases.

\vskip 10pt

As incoming temperature increases, sensor temperature increases as well.  This is interpreted to be {\it less} flow.

\vskip 10pt

In order to compensate for the fluid's temperature entering the flowmeter and thus cancel any effects resulting from temperature change, we must have an unheated sensor that detects the fluid's ``ambient'' temperature.

%(END_ANSWER)





%(BEGIN_NOTES)

As flow increases, temperature decreases, because the increased convective cooling steals more heat away from the heated sensor.  This decreased sensor temperature will be (correctly) interpreted as a greater mass flow rate.

\vskip 10pt

As incoming temperature increases, sensor temperature increases as well.  This is interpreted to be {\it less} flow (i.e. less convective heat transfer).  Since there is no ``reference'' temperature sensor, the flowmeter has no way of distinguishing between a change in fluid temperature and a change in fluid flow rate!

\vskip 10pt

In order to compensate for the fluid's temperature entering the flowmeter and thus cancel any effects resulting from temperature change, we must have an unheated sensor that detects the fluid's ``ambient'' temperature.  This way, the flowmeter will be able to compensate for changes in fluid temperature, discriminating between the effects of this temperature change on the sensor versus the effects of different mass flow rates on the sensor.


%INDEX% Measurement, flow: thermal (mass)

%(END_NOTES)


