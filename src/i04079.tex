%(BEGIN_QUESTION)
% Copyright 2009, Tony R. Kuphaldt, released under the Creative Commons Attribution License (v 1.0)
% This means you may do almost anything with this work of mine, so long as you give me proper credit

Calculate the electrical resistance of a 100 ohm RTD ($\alpha$= 0.00385) at the following temperatures:

\begin{itemize}
\item{} T = 120 $^{o}$C ; R = \underbar{\hskip 50pt}
\vskip 10pt 
\item{} T = 390 $^{o}$F ; R = \underbar{\hskip 50pt}
\end{itemize}

\vskip 10pt

Calculate the temperature of a 100 ohm RTD ($\alpha$ = 0.00392) at the following resistances:

\begin{itemize}
\item{} R = 115 $\Omega$ ; T = \underbar{\hskip 50pt}
\vskip 10pt 
\item{} R = 180 $\Omega$ ; T = \underbar{\hskip 50pt}
\end{itemize}

\vskip 20pt \vbox{\hrule \hbox{\strut \vrule{} {\bf Suggestions for Socratic discussion} \vrule} \hrule}

\begin{itemize}
\item{} Identify some advantages RTDs hold over thermocouples.
\item{} Identify some advantages thermocouples hold over RTDs.
\end{itemize}

\underbar{file i04079}
%(END_QUESTION)





%(BEGIN_ANSWER)


%(END_ANSWER)





%(BEGIN_NOTES)

\begin{itemize}
\item{} T = 120 $^{o}$C ; R = \underbar{\bf 146.2 $\Omega$}
\vskip 10pt 
\item{} T = 390 $^{o}$F ; R = \underbar{\bf 176.6 $\Omega$}
\end{itemize}

\vskip 10pt

\begin{itemize}
\item{} R = 115 $\Omega$ ; T = \underbar{\bf 38.27 $^{o}$C}
\vskip 10pt 
\item{} R = 180 $\Omega$ ; T = \underbar{\bf 204.1 $^{o}$C}
\end{itemize}

%INDEX% Measurement, temperature: RTD resistance calibration

%(END_NOTES)


