%(BEGIN_QUESTION)
% Copyright 2009, Tony R. Kuphaldt, released under the Creative Commons Attribution License (v 1.0)
% This means you may do almost anything with this work of mine, so long as you give me proper credit

Suppose both a thermal mass flowmeter and a Coriolis mass flowmeter monitor gas flow going through the exact same pipe.  Normally, the gas flowing through this pipe is pure helium (specific heat $c$ = 1.24 cal/g-K), and the thermal mass flowmeter has been calibrated for helium gas.  Then one fine day an operator places a few shutoff valves in the wrong positions and sends hydrogen gas (specific heat = 3.41 cal/g-K) down the line instead of helium.  

Not knowing that the wrong gas is now flowing through this pipe, the operator adjusts a manual flow control valve to stabilize the flow rate at its normal value, looking at the thermal mass flowmeter's indication as the process variable.

\vskip 10pt

First, explain why the two flowmeters no longer agree with each other (assuming they registered in perfect agreement while sensing the flow of helium gas).

\vskip 10pt

Second, identify whether the Coriolis flowmeter registers {\it more} mass flow than the thermal flowmeter or {\it less} mass flow than the thermal flowmeter.

\vskip 10pt

Finally, identify which of the two flowmeters (if any!) still registers the true mass flow rate with hydrogen going down the line instead of helium.

\vskip 20pt \vbox{\hrule \hbox{\strut \vrule{} {\bf Suggestions for Socratic discussion} \vrule} \hrule}

\begin{itemize}
\item{} Explain what {\it specific heat} means, and give a practical example from everyday life.
\item{} What does this ``thought experiment'' tell us about Coriolis versus thermal mass flowmeters in general?  Which of these flowmeter types do you think costs less?
\item{} Thermal mass flow measurement is used almost universally for intake air flow measurement on automobile engines with electronic controls.  Do you think the same type of problem exists in this application that we saw in our ``thought experiment''?
\item{} Suppose the gas composition does not change (i.e. it is still pure helium), but the line pressure increases.  How will each of these mass flowmeters respond to this one process condition change?
\item{} Suppose the thermal mass flowmeter were replaced with an orifice plate and DP sensor.  Would this solve the problem of discrepancies between flowmeters resulting from fluid composition changes?  Explain why or why not. 
\end{itemize}

\underbar{file i04080}
%(END_QUESTION)





%(BEGIN_ANSWER)


%(END_ANSWER)





%(BEGIN_NOTES)

The two flowmeters will no longer agree with each other, because the thermal meter's calibration has been thrown off by the different specific heat value of hydrogen (by a ratio of 2.75:1 to be exact).

\vskip 10pt

The greater $c$ value for hydrogen gas means the thermal flowmeter registers {\it more} flow than there actually is.  This makes the thermal meter give the greater reading (i.e. the Coriolis flowmeter registers {\bf less} flow than the thermal flowmeter).

\vskip 10pt

The Coriolis flowmeter measures true mass flow regardless of fluid composition, because it operates on the inertial principle of the Coriolis effect.  Only the thermal mass flowmeter will be ``fooled'' by this change in process fluid.









\vfil \eject

\noindent
{\bf Summary Quiz:}

Suppose two businesses use different types of flowmeters to monitor their natural gas consumption: one a simple {\it thermal} flowmeter and the other a simple {\it turbine} flowmeter.  Both flowmeters have been calibrated to read accurately at a standard line pressure.  One day that natural gas line pressure dramatically increases, packing more molecules of natural gas into every cubic foot.  How does this affect the indications of both flowmeters from the perspective of billing (i.e. fairly paying for the natural gas each business consumes)?

\begin{itemize}
\item{} The thermal flowmeter still reads correctly, while the turbine reads lower than it should
\vskip 5pt 
\item{} Both flowmeters read higher than they should 
\vskip 5pt 
\item{} The turbine flowmeter still reads correctly, while the thermal reads higher than it should 
\vskip 5pt 
\item{} The thermal flowmeter still reads correctly, while the turbine reads higher than it should 
\vskip 5pt 
\item{} Both flowmeters read lower than they should
\vskip 5pt 
\item{} Both flowmeters still read correctly
\end{itemize}

%INDEX% Measurement, flow: Coriolis (mass)
%INDEX% Measurement, flow: thermal (mass)

%(END_NOTES)


