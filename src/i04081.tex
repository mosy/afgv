%(BEGIN_QUESTION)
% Copyright 2009, Tony R. Kuphaldt, released under the Creative Commons Attribution License (v 1.0)
% This means you may do almost anything with this work of mine, so long as you give me proper credit

Calculate the mass flow rate of a liquid having a density of 59.3 lbm/ft$^{3}$ flowing through a pipe at a volumetric rate ($Q$) of 1100 GPM. 

\vskip 10pt

$W$ = \underbar{\hskip 50pt} lbm/min

\vskip 10pt

$W$ = \underbar{\hskip 50pt} kg/sec

\vskip 20pt \vbox{\hrule \hbox{\strut \vrule{} {\bf Suggestions for Socratic discussion} \vrule} \hrule}

\begin{itemize}
\item{} Demonstrate how to {\it estimate} numerical answers for this problem without using a calculator.
\item{} Which unit of measurement do you think is best for {\it custody transfer} applications: GPM or lb/min?  Explain your reasoning.
\item{} When expressing mass flow in Imperial measurements, the unit of ``lbm'' is often used.  Why is the letter ``m'' appended to the symbol for pound?  Is there another Imperial unit for mass other than ``lbm''??
\end{itemize}

\underbar{file i04081}
%(END_QUESTION)





%(BEGIN_ANSWER)


%(END_ANSWER)





%(BEGIN_NOTES)

$$W = \rho Q$$

$$\left({1100 \hbox{ gal} \over \hbox{min}}\right) \left(231 \hbox{ in}^3 \over 1 \hbox{ gal}\right) \left(1 \hbox{ ft}^3 \over 1728 \hbox{ in}^3 \right) \left(59.3 \hbox{ lbm} \over \hbox{ft}^3 \right) = 8719.98 \hbox{ lbm/min}$$

\vskip 10pt

Unit conversions from lbm/min to kg/s:

$$\left(8719.98 \hbox{ lbm} \over \hbox{min}\right) \left(1 \hbox{ min} \over 60 \hbox{ s}\right) \left(0.4535924 \hbox{ kg} \over 1 \hbox{ lbm}\right) = 65.922 \hbox{ kg/s}$$


%INDEX% Physics, dynamic fluids: mass flow versus volumetric flow

%(END_NOTES)


