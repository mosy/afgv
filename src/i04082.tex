
%(BEGIN_QUESTION)
% Copyright 2009, Tony R. Kuphaldt, released under the Creative Commons Attribution License (v 1.0)
% This means you may do almost anything with this work of mine, so long as you give me proper credit

Read and outline the ``Weirs and Flumes'' subsection of the ``Variable-Area Flowmeters'' section of the ``Continuous Fluid Flow Measurement'' chapter in your {\it Lessons In Industrial Instrumentation} textbook.  Note the page numbers where important illustrations, photographs, equations, tables, and other relevant details are found.  Prepare to thoughtfully discuss with your instructor and classmates the concepts and examples explored in this reading.

\underbar{file i04082}
%(END_QUESTION)





%(BEGIN_ANSWER)


%(END_ANSWER)





%(BEGIN_NOTES)

Weirs and flumes are both types of {\it variable area} flowmeters, because the amount of area the fluid has to move through the instrument varies with flow rate.  A weir is nothing more than a precision dam which the liquid must spill over to move through.  A flume is a constricted channel.  Weirs are like open-channel orifice plates, while flumes are like open-channel venturi tubes.  One advantage of flumes over weirs is {\it self-cleaning}.  While weirs tend to clog with debris, flumes tend to remain unobstructed.

\vskip 10pt

In either case, the level of water upstream of the flow element is proportional to volumetric flow rate.  This water level may be sensed ultrasonically or by some other means (e.g. float).  Stilling wells may be used to give the level-sensing element a non-moving fluid level to sense.

\vskip 10pt

Weirs are nonlinear (except for the Sutro weir), but their nonlinearity is vastly different from that of an orifice plate or other pressure-based flow element.  The nonlinearity is dependent upon the shape of the notch (V-shaped, rectangular, trapezoidal):

$$Q = 3.33 (L - 0.2H) H^{1.5} \hbox{\hskip 20pt Rectangular weir}$$

$$Q = 3.367 L H^{1.5} \hbox{\hskip 20pt Cippoletti weir}$$

$$Q = 2.48 \left( \tan {\theta \over 2} \right) H^{2.5} \hbox{\hskip 20pt V-notch weir}$$

\vskip 10pt

A popular style of flume called the {\it Parshall flume} is also nonlinear:

$$Q = 0.992 H^{1.547} \hbox{\hskip 20pt 3-inch wide throat Parshall flume}$$

$$Q = 2.06 H^{1.58} \hbox{\hskip 20pt 6-inch wide throat Parshall flume}$$

$$Q = 3.07 H^{1.53} \hbox{\hskip 20pt 9-inch wide throat Parshall flume}$$

$$Q = 4 L H^{1.53} \hbox{\hskip 20pt 1-foot to 8-foot wide throat Parshall flume}$$

$$Q = (3.6875 L + 2.5) H^{1.53} \hbox{\hskip 20pt 10-foot to 50-foot wide throat Parshall flume}$$

\vskip 10pt

The nonlinearity of weirs and flumes actually works to their advantage in terms of {\it turndown}, since the nonlinearity of their mathematical functions is opposite that of orifice plates or venturi tubes.  A weir or flume actually becomes {\it more sensitive} to flow when the flow rate is low.  This yields excellent rangeability (turndown), sometimes as high as 500:1.














\vskip 20pt \vbox{\hrule \hbox{\strut \vrule{} {\bf Suggestions for Socratic discussion} \vrule} \hrule}

\begin{itemize}
\item{} {\bf In what ways may a weir or flume flowmeter be ``fooled'' to report a false flow measurement?}
\item{} Explain why weirs and flumes are categorized as {\it variable-area} flowmeters.
\item{} Are weirs and flumes linear or non-linear in their behavior?
\item{} Explain why weirs and flumes typically have excellent turndown characteristics.
\item{} Why should the downstream edge of a weir have a bevel?  Does this seem reminiscent of another type of flow element you have studied?
\item{} Explain the principle of a {\it stilling well}.  Where else have you seen stilling wells applied in instrumentation?
\item{} Weirs are theoretically more accurate flow-measurement devices than flumes.  Explain, then, why flumes might actually exhibit greater accuracy than weirs in real-life applications.
\end{itemize}

%INDEX% Reading assignment: Lessons In Industrial Instrumentation, Continuous Fluid Flow Measurement (weir flow measurement)

%(END_NOTES)


