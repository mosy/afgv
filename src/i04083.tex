%(BEGIN_QUESTION)
% Copyright 2010, Tony R. Kuphaldt, released under the Creative Commons Attribution License (v 1.0)
% This means you may do almost anything with this work of mine, so long as you give me proper credit

Experiment with different V/Hz (volts per hertz) profile settings for a VFD controlling the speed of an induction AC motor.  Then, operate the motor at different frequencies while measuring line voltage to the motor (between terminals T1-T2, T2-T3, or T1-T3) using a true-RMS AC voltmeter.  Calculate the actual V/Hz ratio at each speed by taking the measured voltage and dividing by the frequency.

$$\hbox{\bf General Purpose setting}$$

% No blank lines allowed between lines of an \halign structure!
% I use comments (%) instead, so that TeX doesn't choke.

$$\vbox{\offinterlineskip
\halign{\strut
\vrule \quad\hfil # \ \hfil & 
\vrule \quad\hfil # \ \hfil & 
\vrule \quad\hfil # \ \hfil \vrule \cr
\noalign{\hrule}
%
% First row
Frequency & Voltage (V RMS) & V/Hz (calculated) \cr
%
\noalign{\hrule}
%
% Another row
10 Hz &  & \cr
%
\noalign{\hrule}
%
% Another row
20 Hz &  & \cr
%
\noalign{\hrule}
%
% Another row
30 Hz &  & \cr
%
\noalign{\hrule}
%
% Another row
40 Hz &  & \cr
%
\noalign{\hrule}
%
% Another row
50 Hz &  & \cr
%
\noalign{\hrule}
%
% Another row
60 Hz &  & \cr
%
\noalign{\hrule}
} % End of \halign 
}$$ % End of \vbox

\vskip 30pt

$$\hbox{\bf High Starting Torque setting}$$

% No blank lines allowed between lines of an \halign structure!
% I use comments (%) instead, so that TeX doesn't choke.

$$\vbox{\offinterlineskip
\halign{\strut
\vrule \quad\hfil # \ \hfil & 
\vrule \quad\hfil # \ \hfil & 
\vrule \quad\hfil # \ \hfil \vrule \cr
\noalign{\hrule}
%
% First row
Frequency & Voltage (V RMS) & V/Hz (calculated) \cr
%
\noalign{\hrule}
%
% Another row
10 Hz &  & \cr
%
\noalign{\hrule}
%
% Another row
20 Hz &  & \cr
%
\noalign{\hrule}
%
% Another row
30 Hz &  & \cr
%
\noalign{\hrule}
%
% Another row
40 Hz &  & \cr
%
\noalign{\hrule}
%
% Another row
50 Hz &  & \cr
%
\noalign{\hrule}
%
% Another row
60 Hz &  & \cr
%
\noalign{\hrule}
} % End of \halign 
}$$ % End of \vbox

Explain in your own words how the VFD achieves a greater starting torque by modifying the V/Hz ratio in the ``high starting torque'' setting.

\underbar{file i04083}
%(END_QUESTION)




%(BEGIN_ANSWER)


%(END_ANSWER)





%(BEGIN_NOTES)

With the Automation Direct GS1 drive, we found the ``pump and fan'' V/Hz curve to be the most linear.

\vfil \eject

\noindent
{\bf Summary Quiz:}

This exercise makes a good summary quiz, seeing that students have recorded the data, calculated V/Hz ratios, and determined what the VFD does to boost torque at low speeds in the ``high starting torque'' profile.

%INDEX% Final Control Elements, motor: variable frequency drive

%(END_NOTES)


