%(BEGIN_QUESTION)
% Copyright 2015, Tony R. Kuphaldt, released under the Creative Commons Attribution License (v 1.0)
% This means you may do almost anything with this work of mine, so long as you give me proper credit

Suppose we need to measure the volumetric flow rate of deionized water (purified by triple-distillation) used as ``make-up'' water for a chemical experiment in a laboratory, from a maximum flow rate of 20 GPM down to a minimum flow rate of 1 GPM.  Identify the most appropriate technologies from this list, and explain why they others will not work:

\begin{itemize}
\item{} Magnetic
\item{} Coriolis
\item{} Pitot tube
\item{} Ultrasonic
\item{} Orifice plate
\item{} Thermal
\item{} Vortex
\item{} Positive displacement
\item{} Pipe elbow
\end{itemize}

\vskip 20pt \vbox{\hrule \hbox{\strut \vrule{} {\bf Suggestions for Socratic discussion} \vrule} \hrule}

\begin{itemize}
\item{} If we needed to measure mass flow rather than volumetric flow, would this change our selection of flowmeter?  Explain why or why not.
\item{} Identify which of these flowmeters are bidirectional, and explain why based on their principles of operation.
\end{itemize}

\underbar{file i04088}
%(END_QUESTION)





%(BEGIN_ANSWER)


%(END_ANSWER)





%(BEGIN_NOTES)

\noindent
These flowmeter technologies will work:

\begin{itemize}
\item{} Coriolis {\it (may be programmed to output volumetric flow signal, inferred from mass flow rate and temperature)}
\item{} Ultrasonic {\it (if using transit-time method)}
\item{} Positive displacement
\end{itemize}

\vskip 10pt

\noindent
These flowmeter technologies will {\bf not} work for this application:

\begin{itemize}
\item{} Magnetic {\it (liquid needs to conduct electricity)}
\item{} Thermal {\it yields mass flow, not volumetric}
\item{} Vortex ({\it depends on how low the low-flow cutoff is})
\item{} Pitot tube {\it insufficient rangeability (turndown)}
\item{} Orifice plate {\it insufficient rangeability (turndown)}
\item{} Pipe elbow {\it poor accuracy under any conditions, as well as insufficient rangeability}
\end{itemize}

\vfil \eject

\noindent
{\bf Summary Quiz:}

A flowmeter is needed to accurately measure the movement of very thick (viscous) oil through a pipe over a 10:1 range.  Select the best flowmeter type for this application:

\begin{itemize}
\item{} Pipe elbow
\vskip 5pt 
\item{} Pitot tube
\vskip 5pt 
\item{} Magnetic
\vskip 5pt 
\item{} Orifice plate
\vskip 5pt 
\item{} Positive displacement
\vskip 5pt 
\item{} Vortex
\end{itemize}


%INDEX% Measurement, flow: comparison of different technologies

%(END_NOTES)


