%(BEGIN_QUESTION)
% Copyright 2009, Tony R. Kuphaldt, released under the Creative Commons Attribution License (v 1.0)
% This means you may do almost anything with this work of mine, so long as you give me proper credit

Read and outline the introduction to the ``Chemistry'' chapter in your {\it Lessons In Industrial Instrumentation} textbook.  Note the page numbers where important illustrations, photographs, equations, tables, and other relevant details are found.  Prepare to thoughtfully discuss with your instructor and classmates the concepts and examples explored in this reading.

\underbar{file i04090}
%(END_QUESTION)





%(BEGIN_ANSWER)


%(END_ANSWER)





%(BEGIN_NOTES)

Chemistry is the study of how atoms join to form molecules.  Any process by which atoms join together or split apart is called a {\it chemical reaction}.  The Conservation of Mass and the Conservation of Energy are two fundamental laws describing these reactions.  Arrangements of electrons within atoms and of atoms forming molecules is based principally on the Conservation of Energy, with electrons seeking patterns of maximum stability.

\vskip 10pt

The energy state of atoms bonded together in a molecule is lower than the combined energy of those same atoms when separated.  Thus, an investment of energy is necessary to break molecules apart into individual atoms, and energy will be released when atoms come together to form molecules.

\vskip 10pt

The mass of all reactants entering a chemical reaction must be equal to the mass of all products exiting that chemical reaction, in accordance with the Conservation of Mass.  In chemical processes, this is known as {\it mas balance}.











\vskip 20pt \vbox{\hrule \hbox{\strut \vrule{} {\bf Suggestions for Socratic discussion} \vrule} \hrule}

\begin{itemize}
\item{} When atoms come together to form molecules, is energy released or is energy absorbed?  Is this an {\it exothermic} or an {\it endothermic} process?
\item{} When molecules split to form individual atoms, is energy released or is energy absorbed?  Is this an {\it exothermic} or an {\it endothermic} process?
\item{} Explain what happens when natural gas (CH$_{4}$) is burned in the presence of oxygen gas.
\item{} Why isn't water flammable?  After all, each molecule of water (H$_{2}$O) contains hydrogen (H) which is highly flammable, and oxygen (O) which accelerates combustion! 
\item{} Explain how the {\it Conservation of Energy} applies to chemistry.  Feel free to illustrate via example.
\item{} Explain how the {\it Conservation of Mass} applies to chemistry.  Feel free to illustrate via example.
\end{itemize}







\vfil \eject

\noindent
{\bf Summary Quiz}

Choose the best statement relating energy exchange to chemical bonds (bonds between atoms):

\begin{itemize}
\item{} Energy input is required both to break bonds and to form bonds
\vskip 5pt
\item{} Chemical bonds are completely unrelated to energy
\vskip 5pt
\item{} Energy input is required to break bonds; energy is released when bonds form
\vskip 5pt
\item{} Energy is released both when bonds break and when bonds form
\vskip 5pt
\item{} Energy is released when bonds break; energy input is required to form bonds
\vskip 5pt
\item{} Combustion is an endothermic process
\end{itemize}


%INDEX% Reading assignment: Lessons In Industrial Instrumentation, Chemistry (basic terms and concepts)

%(END_NOTES)


