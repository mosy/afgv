%(BEGIN_QUESTION)
% Copyright 2009, Tony R. Kuphaldt, released under the Creative Commons Attribution License (v 1.0)
% This means you may do almost anything with this work of mine, so long as you give me proper credit

Read and outline the ``Atomic Theory and Chemical Symbols'' section of the ``Chemistry'' chapter in your {\it Lessons In Industrial Instrumentation} textbook.  Note the page numbers where important illustrations, photographs, equations, tables, and other relevant details are found.  Prepare to thoughtfully discuss with your instructor and classmates the concepts and examples explored in this reading.

\underbar{file i04091}
%(END_QUESTION)





%(BEGIN_ANSWER)


%(END_ANSWER)





%(BEGIN_NOTES)

The elementary particles (electrons, protons, and neutrons) comprising atoms are incredibly light in mass and tiny in size.  Protons and neutrons are bound together in the nucleus of an atom by the {\it strong nuclear force}, which cannot be overcome by any mechanical means (e.g. cutting, rubbing, heating, etc.).  Electrons are bound in orbit around an atom's nucleus by the {\it electromagnetic} force which is much weaker than the strong nuclear force.  Therefore, electrons' positions and states can be altered by mechanical processes.

\vskip 10pt

The number of protons in the nucleus of an atom (the ``atomic number'' of that atom) defines that atom's chemical identity.  Atoms that are electrically balanced will have just as many electrons as they do protons.  The number of neutrons in the nucleus of an atom determines its ``atomic mass'' (the total count of protons + neutrons) but does not affect an atom's chemical identity.  Changes in atomic mass affect certain nuclear properties such as radioactivity.  Atoms with the same atomic number but different atomic mass are called {\it isotopes}. 

\vskip 10pt

Superscripts and subscripts are used to symbolize properties of atoms and molecules.  Left-hand super/subscripts refer to {\it nuclear} quantities while right-hand super/subscripts refer to {\it molecular} quantities.  On the left-hand side of the elemental symbol (which is always a letter or combination of letters), a subscript refers to an atom's atomic number while a superscript refers to an atom's atomic mass.  On the right-hand side of the elemental symbol, a subscript refers to the number of atoms joined together in a molecule while a superscript refers to the molecule's ionic state (electrical charge imbalance).

\vskip 10pt

Chemical formulae describe the numbers of atoms within each molecule of a substance.  A {\it molecular formula} simply shows a total head-count of each element.  A {\it structural formula} attempts to reveal the layout of the molecule by repeating element letters within the formula.  A {\it displayed formula} actually shows a two-dimensional layout of the atoms bonded together as they are in real life (assuming the molecule is laid flat).  A special version of a displayed formula called a {\it line drawing} is used in organic chemistry (the study of carbon-based molecules) where carbon atoms are assumed to lie at each vertex, each line refers to a single-electron bond, and hydrogen atoms are assumed to be bound to carbon atoms wherever possible.  Line drawings are a kind of ``shorthand'' notation useful in organic chemistry where carbon and hydrogen atoms abound.  A {\it compositional formula} shows the average proportions of atoms within a mixture, with subscripts sometimes being non-integer numbers.







\vskip 20pt \vbox{\hrule \hbox{\strut \vrule{} {\bf Suggestions for Socratic discussion} \vrule} \hrule}

\begin{itemize}
\item{} Describe what the {\it strong nuclear force} is, and how it contrasts against the {\it electromagnetic force} in nature.
\item{} What determines the chemical identity of any atom?  What would be necessary in order to change the chemical identity of an atom?
\item{} Define {\it atomic mass}, or {\it atomic weight}, for any atom.  What would be necessary in order to change the atomic mass of an atom?  
\item{} Is it possible for two atoms to have the same atomic mass but different atomic numbers?  Why or why not?
\item{} Is it possible for two atoms to have the same atomic number but different atomic masses?  Why or why not?
\item{} What does the chemical expression $_{2}^{4}$He mean?
\item{} What does the chemical expression $_{12}^{24}$Mg mean?
\item{} What does the chemical expression O$_{2}^{1+}$ mean?
\item{} What does the chemical expression NH$_{3}^{1-}$ mean?
\item{} Describe the differences between a {\it molecular} formula, a {\it structural} formula, a {\it compositional} formula, a {\it displayed} formula (also called an {\it expanded structural} formula), and a {\it line drawing} used to represent molecular forms.
\item{} Explain why chemical engineers sometimes use {\it compositional formulae} in their work. 
\item{} Reference a line drawing and show how each carbon atom in an organic molecule forms exactly {\it four} bonds (i.e. shares four electrons with other atoms in the molecule).
\item{} Which is in a greater energy state: an NH$_{3}$ molecule, or one Nitrogen atom and three Hydrogen atoms all separated from each other?
\item{} Which is in a greater energy state: an H$_{2}$SO$_{4}$ molecule, or two Hydrogen atoms and one Sulfur atom and four Oxygen atoms all separated from each other?
\item{} When Al$_{2}$O$_{3}$ (alumina) is separated into Al (Aluminum) and O (Oxygen), is energy released or is energy absorbed?
\item{} When H (Hydrogen) and F (Flourine) atoms join together to form HF molecules (Hydrogen Fluoride), is energy released or is energy absorbed?
\end{itemize}

%INDEX% Reading assignment: Lessons In Industrial Instrumentation, Chemistry (atomic theory and chemical symbols)

%(END_NOTES)


