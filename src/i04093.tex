%(BEGIN_QUESTION)
% Copyright 2009, Tony R. Kuphaldt, released under the Creative Commons Attribution License (v 1.0)
% This means you may do almost anything with this work of mine, so long as you give me proper credit

Bring the following materials to class to perform an experiment demonstrating the {\it electrolysis} (division of molecules into their atoms by electricity) of water.  You may partner with one or two classmates to bring these materials, and perform the experiment together.  You are also free to perform the experiment before class to get an advance-start on understanding the principles involved:

\begin{itemize}
\item{} Two paper clips
\vskip 5pt
\item{} Small drinking cup (to fill with water)
\vskip 5pt
\item{} At least two ``alligator clip'' jumper wires
\vskip 5pt
\item{} One or more 9-volt batteries
\end{itemize}

Bend the paperclips so they form two electrodes which will dip into water you put in the cup.  Fill the cup with water, and connect the paper-clip electrodes to the 9 volt battery using the ``alligator clip'' jumper wires.  You may use multiple 9-volt batteries connected in series for more voltage (to make the experiment more dramatic).

Closely observe the two energized electrodes.  You should see small bubbles begin to form on their surfaces.  Which electrode generates more bubbles (it should be approximately a 2:1 ratio)?  Which electrode collects hydrogen atoms and which electrode collects negative atoms, based on your knowledge of water being H$_{2}$O and the process of electrolysis splitting water molecules into hydrogen and oxygen atoms?

\vskip 10pt

Identify the transfer of energy in the process of electrolyzing water: is energy being released from the water molecules as they split into hydrogen and oxygen gas, or is energy being absorbed in the process?  To phrase the question differently, is the energy state of a whole water molecule greater than or less than the energy states of the hydrogen and oxygen atoms separated?  Explain your reasoning!

\vskip 10pt

Finally, explain how we could re-combine the hydrogen and oxygen gases to become water once more.  Would this process release energy or absorb energy?  Explain your reasoning!

\vskip 20pt \vbox{\hrule \hbox{\strut \vrule{} {\bf Suggestions for Socratic discussion} \vrule} \hrule}

\begin{itemize}
\item{} Hydrogen gas is highly flammable, and so is a potential fuel for a combustion engine such as that found in an automobile.  Does it make practical sense to produce hydrogen gas with an electrolysis process to generate fuel to power an automobile?  Why or why not?
\item{} Given the fact that hydrogen gas is flammable, and oxygen accelerates combustion of anything that is flammable, why isn't {\it water} a flammable substance?  After all, it's got the fuel and the oxidant built right into each and every H$_{2}$O molecule!
\item{} Is the hydrogen production rate for an electrolytic cell such as this a function of {\it voltage} or a function of {\it current}?  Devise an experiment where you could determine which electrical measure best corresponds with hydrogen production rate.
\end{itemize}

\underbar{file i04093}
%(END_QUESTION)





%(BEGIN_ANSWER)


%(END_ANSWER)





%(BEGIN_NOTES)

The most common ionic state of hydrogen is {\it positive} (H$^{+}$), while the most common ionic state of oxygen is double-negative (O$^{2-}$).  Therefore, hydrogen ions will collect near the negative electrode (the cathode), while oxygen ions will collect near the positive electrode (the anode).

%INDEX% Chemistry, electrolysis of water

%(END_NOTES)


