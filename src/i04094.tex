%(BEGIN_QUESTION)
% Copyright 2009, Tony R. Kuphaldt, released under the Creative Commons Attribution License (v 1.0)
% This means you may do almost anything with this work of mine, so long as you give me proper credit

Read the ``Hydrogen sulfide'' entry in the {\it NIOSH Pocket Guide To Chemical Hazards} (DHHS publication number 2005-149) and answer the following questions:

\vskip 10pt

Write the chemical formula for hydrogen sulfide gas, and identify its constituent elements.

\vskip 10pt

Hydrogen sulfide is a byproduct of petroleum oil refining, and also of anaerobic bacterial decomposition of organic matter (e.g. municipal wastewater and wood pulp production).  Identify the NIOSH recommended exposure limit (REL) and the OSHA Permissible Exposure Limit (PEL) values for this gas.

\vskip 10pt

Determine whether hydrogen sulfide gas is {\it lighter} than air or {\it denser} than air, based on the relative density (RGasD) figure provided in the NIOSH guide.

\vskip 10pt

Identify the type(s) of respirator equipment necessary for protection against high hydrogen sulfide concentrations.

\vskip 20pt \vbox{\hrule \hbox{\strut \vrule{} {\bf Suggestions for Socratic discussion} \vrule} \hrule}

\begin{itemize}
\item{} Identify what ``LEL'' and ``UEL'' refer to, and how these parameters are useful for identifying a compound's flammability.
\item{} Sketch a {\it displayed formula} for hydrogen sulfide.
\item{} Explain the specific hazards hydrogen sulfide might pose to people working in a {\it confined space} where ventilation is limited.
\item{} Identify common sources of hydrogen sulfide gas in the home or workplace.
\item{} When hydrogen sulfide gas combusts (burns) with oxygen, what types of molecules are produced as a result of that combustion?
\item{} Describe a general principle for determining proper respirator equipment: specifically whether {\it filtration} is sufficient or {\it supplied air} is necessary.
\item{} A common unit of measurement for airborne chemical concentration of substances such as hydrogen sulfide is ``parts per million'' or {\it ppm}.  Explain in your own words what ``ppm'' means.
\end{itemize}

\underbar{file i04094}
%(END_QUESTION)





%(BEGIN_ANSWER)


%(END_ANSWER)





%(BEGIN_NOTES)

H$_{2}$S: two atoms of hydrogen and one atom of sulfur per hydrogen sulfide molecule.

\vskip 10pt

REL = 10 ppm ; PEL = 20 ppm (continuous) ; PEL = 50 ppm (10 minute maximum peak value)

\vskip 10pt

Hydrogen sulfide is denser than air, given its specific gravity (RGasD) value of 1.19.

\vskip 10pt

Appropriate respirator gear is that which supplies air to breathe.  Filtration-type respirators are appropriate only if they have an appropriate cartridge (for CO gas) with an end-of-service-life indicator.

\vskip 10pt

\noindent
{\bf Legend for NIOSH respirator types:}

\begin{itemize}
\item{} {\bf F} = full facepiece respirator
\item{} {\bf Qm} = quarter-mask respirator
\item{} {\bf XQ} = except quarter-mask respirator
\item{} {\bf T} = tight-fitting facepiece
\item{} {\bf GmF} = Air-purifying full-facepiece respirator (``gas mask'') with canister
\item{} {\bf Papr} = Powered (fan) air-purifying respirator
\item{} {\bf Sa} = supplied air
\item{} {\bf Scba} = self-contained breathing apparatus
\item{} {\bf Ag} = acid gas cartridge or canister
\item{} {\bf Ov} = organic vapor cartridge or canister
\item{} {\bf S} = chemical cartridge or canister
\item{} {\bf Ccr} = chemical cartridge
\item{} {\bf Cf} = continuous-flow mode
\item{} {\bf Pd, Pp} = pressure-demand or positive-pressure mode
\item{} {\bf Hie} = high-efficiency particulate filter
\end{itemize}













\vfil \eject

\noindent
{\bf Prep Quiz}

Identify the NIOSH recommended exposure limit (REL) for hydrogen sulfide gas, as given in the {\it NIOSH Pocket Guide to Chemical Hazards}:

\begin{itemize}
\item{} 20\%
\vskip 5pt
\item{} 15 ppm
\vskip 5pt
\item{} 20 ppm
\vskip 5pt
\item{} 10\%
\vskip 5pt
\item{} 50\%
\vskip 5pt
\item{} 1.19\%
\vskip 5pt
\item{} 10 ppm
\vskip 5pt
\item{} 50 ppm
\end{itemize}


%INDEX% Reading assignment: NIOSH Chemical Hazard Guide (hydrogen sulfide)

%(END_NOTES)


