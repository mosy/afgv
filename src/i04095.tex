%(BEGIN_QUESTION)
% Copyright 2009, Tony R. Kuphaldt, released under the Creative Commons Attribution License (v 1.0)
% This means you may do almost anything with this work of mine, so long as you give me proper credit

Read the ``Carbon dioxide'' entry in the {\it NIOSH Pocket Guide To Chemical Hazards} (DHHS publication number 2005-149) and answer the following questions:

\vskip 10pt

Write the chemical formula for carbon dioxide gas, and identify its constituent elements.

\vskip 10pt

Determine whether or not carbon dioxide is flammable.

\vskip 10pt

Determine whether carbon dioxide gas is {\it lighter} than air or {\it denser} than air, based on the relative density (RGasD) figure provided in the NIOSH guide.

\vskip 10pt

Identify some of the symptoms of excessive carbon dioxide exposure, and the ``target organs'' of the body it affects.

\vskip 10pt

Identify the type(s) of respirator equipment necessary for protection against high carbon dioxide concentrations.

\vskip 20pt \vbox{\hrule \hbox{\strut \vrule{} {\bf Suggestions for Socratic discussion} \vrule} \hrule}

\begin{itemize}
\item{} Identify what ``LEL'' and ``UEL'' refer to, and how these parameters are useful for identifying a compound's flammability.
\item{} Sketch a {\it displayed formula} for carbon dioxide.
\item{} Explain the specific hazards carbon dioxide might pose to people working in a {\it confined space} where ventilation is limited.
\item{} Identify common sources of carbon dioxide gas in the home or workplace.
\item{} Describe a general principle for determining proper respirator equipment: specifically whether {\it filtration} is sufficient or {\it supplied air} is necessary.
\end{itemize}

\underbar{file i04095}
%(END_QUESTION)





%(BEGIN_ANSWER)


%(END_ANSWER)





%(BEGIN_NOTES)

CO$_{2}$: one carbon atom and two oxygen atoms per carbon dioxide molecule.

\vskip 10pt

Carbon dioxide is {\it not} flammable -- in fact, it is sometimes used as a fire extinguishing agent!

\vskip 10pt

Carbon dioxide is considerably denser than air, given its specific gravity (RGasD) value of 1.53.

\vskip 10pt

Symptoms of ever-exposure include headache, dizziness, restlessness, paresthesia (abnormal or impaired skin sensation), dyspnea (difficulty breathing), sweating, malaise (vague feeling of discomfort), increased heart rate, increased cardiac output, increased blood pressure, coma, asphyxia (unconsciousness or death caused by lack of oxygen), convulsions, and frostbite (if in liquid or solid form).  Target organs are the respiratory system and the cardio-vascular (heart and blood vessels) system.

\vskip 10pt

Appropriate respirator gear is that which supplies air to breathe, as opposed to filtration-type respirators since CO$_{2}$ cannot be filtered.

\vskip 10pt

\noindent
{\bf Legend for NIOSH respirator types:}

\begin{itemize}
\item{} {\bf F} = full facepiece respirator
\item{} {\bf Qm} = quarter-mask respirator
\item{} {\bf XQ} = except quarter-mask respirator
\item{} {\bf T} = tight-fitting facepiece
\item{} {\bf GmF} = Air-purifying full-facepiece respirator (``gas mask'') with canister
\item{} {\bf Papr} = Powered (fan) air-purifying respirator
\item{} {\bf Sa} = supplied air
\item{} {\bf Scba} = self-contained breathing apparatus
\item{} {\bf Ag} = acid gas cartridge or canister
\item{} {\bf Ov} = organic vapor cartridge or canister
\item{} {\bf S} = chemical cartridge or canister
\item{} {\bf Ccr} = chemical cartridge
\item{} {\bf Cf} = continuous-flow mode
\item{} {\bf Pd, Pp} = pressure-demand or positive-pressure mode
\item{} {\bf Hie} = high-efficiency particulate filter
\end{itemize}

%INDEX% Reading assignment: NIOSH Chemical Hazard Guide (carbon dioxide)

%(END_NOTES)


