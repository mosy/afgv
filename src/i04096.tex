%(BEGIN_QUESTION)
% Copyright 2009, Tony R. Kuphaldt, released under the Creative Commons Attribution License (v 1.0)
% This means you may do almost anything with this work of mine, so long as you give me proper credit

Read the ``Carbon monoxide'' entry in the {\it NIOSH Pocket Guide To Chemical Hazards} (DHHS publication number 2005-149) and answer the following questions:

\vskip 10pt

Write the chemical formula for carbon monoxide gas, and identify its constituent elements.

\vskip 10pt

Determine whether or not carbon monoxide is flammable.

\vskip 10pt

Identify the relative density (RGasD) of carbon monoxide, then determine whether this is {\it lighter} than air or {\it heavier} than air.

\vskip 10pt

Identify some of the symptoms of excessive carbon monoxide exposure, and the ``target organs'' of the body it affects.

\vskip 10pt

Identify the type(s) of respirator equipment necessary for protection against high carbon monoxide concentrations.

\vskip 10pt

Identify how similar carbon {\it monoxide} is to carbon {\it dioxide} in its physical properties and effects on the human body.

\vskip 20pt \vbox{\hrule \hbox{\strut \vrule{} {\bf Suggestions for Socratic discussion} \vrule} \hrule}

\begin{itemize}
\item{} Identify what ``LEL'' and ``UEL'' refer to, and how these parameters are useful for identifying a compound's flammability.
\item{} Sketch a {\it displayed formula} for carbon monoxide.
\item{} Carbon monoxide molecules are similar to carbon dioxide molecules in their composition, but have very different chemical and physical properties.  Identify one property in which carbon monoxide differs substantially from carbon dioxide, and try to explain why this is based on the differences in atoms contained within the two molecules.
\item{} Identify some of the specific hazards carbon monoxide might pose to people working in a {\it confined space} where ventilation is limited.
\item{} Identify common sources of carbon monoxide gas in the home or workplace.
\item{} When carbon monoxide gas combusts (burns) with oxygen, what type of molecule is produced as a result of that combustion?
\item{} Describe a general principle for determining proper respirator equipment: specifically whether {\it filtration} is sufficient or {\it supplied air} is necessary.
\end{itemize}

\underbar{file i04096}
%(END_QUESTION)





%(BEGIN_ANSWER)


%(END_ANSWER)





%(BEGIN_NOTES)

CO: one carbon atom and one oxygen atom per carbon monoxide molecule.

\vskip 10pt

Carbon monoxide is flammable, with a lower explosive limit (LEL) of 12.5\% and an upper explosive limit (UEL) of 74\%.

\vskip 10pt

Carbon monoxide is slightly lighter than air, given its specific gravity (RGasD) value of 0.97.

\vskip 10pt

Symptoms of ever-exposure include headache, tachypnea (rapid breathing), nausea, lassitude (weakness or exhaustion), confusion, hallucination, cyanosis (bluish discoloration of skin), depresses S-T segment of electrocardiogram, angina (spasms or painful suffocation), and syncope (loss of consciousness).  Target organs are the cardiovascular system (heart and blood vessels), lungs, blood, and central nervous system.

\vskip 10pt

Appropriate respirator gear is that which supplies air to breathe.  Filtration-type respirators are appropriate only if they have an appropriate cartridge (for CO gas) with an end-of-service-life indicator.

\vskip 10pt

CO and CO$_{2}$ are remarkably different in many ways: flammability, density, and toxicity to name a few.

\vskip 10pt

\noindent
{\bf Legend for NIOSH respirator types:}

\begin{itemize}
\item{} {\bf F} = full facepiece respirator
\item{} {\bf Qm} = quarter-mask respirator
\item{} {\bf XQ} = except quarter-mask respirator
\item{} {\bf T} = tight-fitting facepiece
\item{} {\bf GmF} = Air-purifying full-facepiece respirator (``gas mask'') with canister
\item{} {\bf Papr} = Powered (fan) air-purifying respirator
\item{} {\bf Sa} = supplied air
\item{} {\bf Scba} = self-contained breathing apparatus
\item{} {\bf Ag} = acid gas cartridge or canister
\item{} {\bf Ov} = organic vapor cartridge or canister
\item{} {\bf S} = chemical cartridge or canister
\item{} {\bf Ccr} = chemical cartridge
\item{} {\bf Cf} = continuous-flow mode
\item{} {\bf Pd, Pp} = pressure-demand or positive-pressure mode
\item{} {\bf Hie} = high-efficiency particulate filter
\end{itemize}














\vfil \eject

\noindent
{\bf Prep Quiz}

Identify the relative density (RGasD) of carbon monoxide:

\begin{itemize}
\item{} -313
\vskip 5pt
\item{} 28.0
\vskip 5pt
\item{} 0.97
\vskip 5pt
\item{} 14.01
\vskip 5pt
\item{} 1.03
\vskip 5pt
\item{} 74\%
\vskip 5pt
\item{} 2
\vskip 5pt
\item{} 1.53
\end{itemize}


%INDEX% Reading assignment: NIOSH Chemical Hazard Guide (carbon monoxide)

%(END_NOTES)


