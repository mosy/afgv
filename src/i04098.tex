%(BEGIN_QUESTION)
% Copyright 2009, Tony R. Kuphaldt, released under the Creative Commons Attribution License (v 1.0)
% This means you may do almost anything with this work of mine, so long as you give me proper credit

Determine the number of protons and neutrons in the nucleus of one atom for each of the following elements:

\begin{itemize}
\item{} $^{195}$Pt (Platinum-195) -- Protons = \underbar{\hskip 20pt} ; Neutrons = \underbar{\hskip 20pt}
\vskip 5pt
\item{} $^{75}$As (Arsenic-75) -- Protons = \underbar{\hskip 20pt} ; Neutrons = \underbar{\hskip 20pt}
\vskip 5pt
\item{} $^{235}$U (Uranium-235) -- Protons = \underbar{\hskip 20pt} ; Neutrons = \underbar{\hskip 20pt}
\vskip 5pt
\item{} $^{40}$Ca (Calcium-40) -- Protons = \underbar{\hskip 20pt} ; Neutrons = \underbar{\hskip 20pt}
\end{itemize}

\vskip 20pt \vbox{\hrule \hbox{\strut \vrule{} {\bf Suggestions for Socratic discussion} \vrule} \hrule}

\begin{itemize}
\item{} Which of these particle quantities {\it defines} the element?
\item{} What would be required to {\it transmute} one element into another, say lead into gold?
\item{} Breaking and creating new chemical bonds involves far lower energy levels than transmutating one element into another.  Explain what this fact tells us about the binding force of electrons to the nucleus of an atom versus the binding force between protons and neutrons within the nucleus of an atom.
\item{} Explain what an {\it isotope} is, and how isotope identities relate to these particle numbers in an atomic nucleus.
\end{itemize}

\underbar{file i04098}
%(END_QUESTION)





%(BEGIN_ANSWER)

\begin{itemize}
\item{} $^{195}$Pt (Platinum-195) -- Protons = 78 ; Neutrons = 117
\vskip 5pt
\item{} $^{75}$As (Arsenic-75) -- Protons = 33 ; Neutrons = 42
\vskip 5pt
\item{} $^{235}$U (Uranium-235) -- Protons = 92 ; Neutrons = 143
\vskip 5pt
\item{} $^{40}$Ca (Calcium-40) -- Protons = 20 ; Neutrons = 20
\end{itemize}


%(END_ANSWER)





%(BEGIN_NOTES)

%INDEX% Chemistry, basic principles: periodic table of the elements

%(END_NOTES)


