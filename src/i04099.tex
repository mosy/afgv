%(BEGIN_QUESTION)
% Copyright 2009, Tony R. Kuphaldt, released under the Creative Commons Attribution License (v 1.0)
% This means you may do almost anything with this work of mine, so long as you give me proper credit

Read and outline the ``Electronic Structure'' section of the ``Chemistry'' chapter in your {\it Lessons In Industrial Instrumentation} textbook.  Note the page numbers where important illustrations, photographs, equations, tables, and other relevant details are found.  Prepare to thoughtfully discuss with your instructor and classmates the concepts and examples explored in this reading.

\underbar{file i04099}
%(END_QUESTION)




%(BEGIN_ANSWER)


%(END_ANSWER)





%(BEGIN_NOTES)

Electrons do not orbit nuclei like satellites orbit a plant.  Instead, they exist in ``clouds'' that take on very strange shapes around the nucleus.  No two electrons may share identical quantum states around a nucleus, which means they will never occupy the exact same ``cloud'' as another electron.

\vskip 10pt

Electrons generally fall into the lowest potential energy cloud available around a nucleus.  These energy levels are symbolically represented by {\it shells} (numbered 1 and up; lettered K and up), each shell having {\it subshells} (numbered 0 and up; lettered s, p, d, f).  The higher the shell/subshell, the less bound an electron is to its parent nucleus, and therefore the more easily it is forcibly removed from the atom.

\vskip 10pt

Spectroscopic notation is used to denote the filling of shells and subshells within an element.  Superscript numbers refer to the number of electrons in a subshell, letters refer to the subshells, and numbers refer to the shells.  For brevity, elements are often shown only with the unfilled shells represented, or are shown with all the filled shells as bracketed noble-element symbols.

\vskip 10pt

Shells and subshells may be thought of as analogous to tiers and rows of seats around an amphitheater.  Obviously, the lowest tiers and rows are generally regarded to be the best (being closer to the action), and no two people can occupy the same seat.  Electrons give up energy to ``fall'' into a lower-level shell or subshell, and must be given energy in order to leave it for a higher-level shell or subshell.  Only the electrons residing in incompletely-filled shells play significant roles in chemical bonds, and are called {\it valence} electrons.  The electrons occupying lower-level (completely-filled) shells are so stable they are unlikely to engage in bonds with other atoms.  Noble elements have shells that are completely full of electrons.

\vskip 10pt

Bonds between atoms will occur if their respective valence electrons fall into lower energy states in a bonded condition than they would be in if their respective aroms were unbonded.  This is why atoms release energy when forming bonds, and require an input of energy to break bonds.








\vskip 20pt \vbox{\hrule \hbox{\strut \vrule{} {\bf Suggestions for Socratic discussion} \vrule} \hrule}

\begin{itemize}
\item{} Explain what is wrong about the classic ``satellite'' model of the atom.
\item{} Explain why the arrangement of electrons into {\it shells} and {\it subshells} is an important concept in chemistry.  This may be done by supposing a world in which this were {\it not} true: a world in which electrons assumed random positions around atomic nuclei.
\item{} Describe how the amphitheater analogy helps illustrate the shells and subshells of electrons surrounding the nuclei of atoms.
\item{} Choose any element from the table of electron configurations shown in the textbook, and explain what its spectroscopic notation tells us.
\item{} Choose any element from the table of electron configurations shown in the textbook, and explain what its {\it condensed} spectroscopic notation tells us.
\item{} Identify which electron is more tightly-bound to its parent atom (assume an electrically ``balanced'' atom in each case): an electron in an atom of sodium (Na, $n=11$), or an electron in an atom of argon (Ar, $n=18$).
\item{} Identify which electron is more tightly-bound to its parent atom (assume an electrically ``balanced'' atom in each case): an electron in an atom of potassium (K, $n=19$), or an electron in an atom of krypton (Kr, $n=36$).
\item{} Explain what the term {\it valence} refers to, and why this is an important concept in chemistry.
\item{} Is energy absorbed or released when a chemical bond is formed between two or more atoms?
\item{} Is energy absorbed or released when a chemical bond is broken between two or more atoms?
\item{} Explain how the {\it Conservation of Energy} applies to energy in chemical bonds.
\end{itemize}

%INDEX% Reading assignment: Lessons In Industrial Instrumentation, Chemistry (electronic structure)

%(END_NOTES)


