%(BEGIN_QUESTION)
% Copyright 2009, Tony R. Kuphaldt, released under the Creative Commons Attribution License (v 1.0)
% This means you may do almost anything with this work of mine, so long as you give me proper credit

Read and outline the ``Spectroscopy'' section of the ``Chemistry'' chapter in your {\it Lessons In Industrial Instrumentation} textbook.  Note the page numbers where important illustrations, photographs, equations, tables, and other relevant details are found.  Prepare to thoughtfully discuss with your instructor and classmates the concepts and examples explored in this reading.

\underbar{file i04101}
%(END_QUESTION)




%(BEGIN_ANSWER)


%(END_ANSWER)





%(BEGIN_NOTES)

Red light has a lower frequency (greater wavelength) than blue light.  Waves of light may dislodge electrons from an atom if the energy level is sufficient, because light waves contain electric fields and thus interact with electrically-charged objects such as electrons.  This fact allows us to explore the electronic structure of atoms using light as a tool.  The amount of energy carried by a photon (particle of light) is described by the following formula:

$$E = hf = {hc \over \lambda}$$

If the energy of a photon is equal to the amount of energy needed to boost an electron from a low subshell to a higher-energy subshell in an atom, that photon will dislodge that electron and force it to ``jump'' to that higher energy level.  If and when the electron returns to its former (low-energy) position, it will emit a photon of light having the exact same energy (frequency, wavelength, color) as the jump.

\vskip 10pt

When atoms are exited by an external energy source, they emit photons representative of those atoms' electron configurations.  If those photons' wavelengths are analyzed, we may identify the atoms they came from.  This is called {\it emission spectroscopy}.

When atoms are struck by an external light source, they absorb those photons having energy levels representative of those atoms' electron configurations, and then re-emit photons in different directions.  The result is that the light making its way from the substance will be missing certain wavelengths (the same wavelengths emitted by an excited sample of the same substance).  If those missing wavelengths are analyzed, we may identify the atoms they came from.  This is called {\it absorption spectroscopy}.

\vskip 10pt

Molecules have their own unique emission and absorption signatures just as atoms do.  Thus, we may use light to identify molecules as well as identify atoms.

\vskip 10pt

Certain types of gas, when viewed through an infra-red camera, appear as dark clouds because of their ability to absorb infra-red light.





\filbreak


\vskip 20pt \vbox{\hrule \hbox{\strut \vrule{} {\bf Suggestions for Socratic discussion} \vrule} \hrule}

\begin{itemize}
\item{} Explain why the {\it color} of a light beam is more important than the intensity of that beam with regard to ionizing atoms.
\item{} If I take a multi-color light source and shift its color from red to blue, am I increasing the frequency of the light waves or decreasing the frequency of the light waves?
\item{} If I take a single-color LED and increase the amount of current passing through it to make more light, am I increasing the frequency of the light waves, increasing the number of photons emitted, or both?
\item{} When an electron ``jumps up'' from a lower shell to a higher shell in an atom, is energy absorbed or released by that electron (i.e. is light absorbed or released)?
\item{} When an electron ``jumps down'' from a higher shell to a lower shell in an atom, is energy absorbed or released by that electron (i.e. is light absorbed or released)?
\item{} Which represents a larger ``jump'' for an electron, a ray of red light or a ray or blue light?
\item{} {\bf Use the amphitheater analogy to explain what happens when a photon causes an electron to ``jump'' shells, and conversely what happens when an electron ``jumps'' in such a way as to emit light.}
\item{} Explain why only specific colors of light are able to dislodge electrons from their shells.
\item{} Explain why only specific colors of light are emitted by ``excited'' atoms.
\item{} Explain why the ``Balmer'' series of electron-jumps in a hydrogen atom are associated with {\it visible} light, but the ``Lyman'' series is not.
\item{} Suppose a hydrogen atom is electrically stimulated such that its single electron becomes boosted in energy into the sixth shell ($n = 6$).  Identify some of the different ways in which this electron might return to the ground state, and the relative energies of each emitted photon at every step.
\item{} Examine the emission spectrum of hydrogen gas shown in contrast to a white-light continuous spectrum and also other elements.  How many {\it visible} colors are emitted by hydrogen?  Match each of the visible colors in hydrogen's emission spectrum to the different ``jumps'' of the Balmer series.
\item{} Examine the emission spectrum of hydrogen gas shown in contrast to a white-light continuous spectrum and also other elements.  How many {\it invisible} light frequencies emitted by hydrogen are shown in this spectrum?  Match each of these invisible wavelengths to the different ``jumps'' of the Lyman series.
\item{} Contrast {\it emission} versus {\it absorption} spectroscopy.  Specifically, contrast the light spectra generated by emission versus by absorption.
\item{} Describe how we could use emission spectroscopy to analyze the chemical composition of an unknown gas.
\item{} Describe how we could use absorption spectroscopy to analyze the chemical composition of an unknown gas.
\item{} What happens to the energy of a photon absorbed by an atom?
\item{} Are all the lights in Las Vegas, Nevada truly ``neon'' lights?
\item{} Describe how spectroscopy of the element krypton is actually used as a measurement standard in metrology.
\item{} Explain why molecules have their own spectroscopic signatures, unique from their constituent atoms.  Also, explain why molecular spectra are more complex than elemental spectra.
\item{} A BTC Instrumentation student once described the absorption spectrum of hydrogen as the {\it shadow} image of hydrogen gas.  Explain what this student meant by this description of hydrogen's absorption spectrum.
\item{} Explain why certain gases appear as smoke when viewed through the proper kind of infrared imaging camera.
\item{} Explain how astronomers must wait for a planet to align with a star in order to perform spectroscopic analysis on its atmosphere.
\end{itemize}









\vfil \eject

\noindent
{\bf Prep Quiz:}

Suppose a {\it spectrographic analyzer} displays the light spectrum following an analysis of some chemical sample.  The resulting spectrum is nothing more than a few colored lines against a background of otherwise solid black.  What type of spectroscopy does this analyzer employ?

\begin{itemize}
\item{} Sublimation
\vskip 5pt 
\item{} Multi-variable
\vskip 5pt 
\item{} Emission
\vskip 5pt 
\item{} Diffraction
\vskip 5pt 
\item{} Really expensive
\vskip 5pt 
\item{} Absorption
\end{itemize}

\vfil \eject

\noindent
{\bf Prep Quiz:}

Suppose a {\it spectrographic analyzer} displays the light spectrum following an analysis of some chemical sample.  The resulting spectrum is an unbroken rainbow of colors with just a few black lines of missing color scattered here and there.  What type of spectroscopy does this analyzer employ?

\begin{itemize}
\item{} Sublimation
\vskip 5pt 
\item{} Multi-variable
\vskip 5pt 
\item{} Emission
\vskip 5pt 
\item{} Diffraction
\vskip 5pt 
\item{} Really expensive
\vskip 5pt 
\item{} Absorption
\end{itemize}


%INDEX% Reading assignment: Lessons In Industrial Instrumentation, Chemistry (spectroscopy)

%(END_NOTES)


