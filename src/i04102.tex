%(BEGIN_QUESTION)
% Copyright 2009, Tony R. Kuphaldt, released under the Creative Commons Attribution License (v 1.0)
% This means you may do almost anything with this work of mine, so long as you give me proper credit

Bring a Compact Disk (CD) or a Digital Video Disk (DVD) to class and use the reflective surface as a {\it reflection grating} to display the spectrum of visible light sources (point-sources such as small light bulbs work best!).  You may have to experiment with the angles you hold the disk in order to easily see different spectra, but with practice you should be able to quickly locate the spectrum from any point-source of light.  {\it Safety note: DO NOT ATTEMPT TO VIEW SUNLIGHT OR LASER LIGHT IN THIS WAY, AS YOU MAY EASILY HURT YOUR EYES!}

\vskip 10pt

View the spectrum from several different point-light sources and note how those spectra are similar and how they are different from each other:

\begin{itemize}
\item{} Incandescent light bulb
\vskip 5pt
\item{} Compact fluorescent light bulb
\vskip 5pt
\item{} ``White'' LED (light emitting diode)
\vskip 5pt
\item{} Colored LED (light emitting diode)
\vskip 5pt
\item{} Neon lamp
\end{itemize}

\vskip 10pt

Identify which of these spectra appears to be the most ``continuous'' through all the colors, and which appear the most ``broken.''  Explain what each spectrum reveals about the chemical nature of each light source.

\vskip 20pt \vbox{\hrule \hbox{\strut \vrule{} {\bf Suggestions for Socratic discussion} \vrule} \hrule}

\begin{itemize}
\item{} Astronomers use this basic principle (not actually using {\it CDs}, of course!) to analyze the chemical composition of stars.  Explain how this would work -- what exactly would an astronomer look for in the spectrum of light received from a star?
\end{itemize}

\underbar{file i04102}
%(END_QUESTION)





%(BEGIN_ANSWER)


%(END_ANSWER)





%(BEGIN_NOTES)

The reason CDs and DVDs work well for this purpose is because the closely-spaced tracks on their reflective substrates act as optical gratings.  Specifically, a CD or DVD disk functions as a {\it reflection grating}.

\vskip 10pt

Bringing some cheap spectroscopic gratings to class in addition to CD-ROM disks offers students another way to view optical spectra.

%INDEX% Chemistry, spectroscopy

%(END_NOTES)


