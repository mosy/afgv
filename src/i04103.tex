%(BEGIN_QUESTION)
% Copyright 2009, Tony R. Kuphaldt, released under the Creative Commons Attribution License (v 1.0)
% This means you may do almost anything with this work of mine, so long as you give me proper credit

One of the concerns surrounding depletion of Earth's ozone layer in the upper atmosphere is its effect on ultraviolet light from the sun.  Ozone is a triple-oxygen molecule (O$_{3}$) with some unique chemical characteristics different from regular molecular oxygen (O$_{2}$).  The more the ozone layer depletes, the more ultraviolet light reaches us on the ground.  

\vskip 10pt

Based on your knowledge of spectroscopy, would you classify this relationship between the ozone layer and ``UV'' light as an example of optical {\it emission} or optical {\it absorption}?

\vskip 20pt \vbox{\hrule \hbox{\strut \vrule{} {\bf Suggestions for Socratic discussion} \vrule} \hrule}

\begin{itemize}
\item{} Explain why ozone molecules (O$_{3}$) exhibit this interaction with ultraviolet light, but regular oxygen molecules (O$_{2}$) do not.
\item{} Describe how an ozone-sensing instrument might be constructed based on this principle of UV light absorption.  Identify ways in which such an instrument might be ``fooled'' into giving false readings of ozone gas where none is present.
\item{} Suppose a glass tube containing ozone gas were electrically stimulated by the application of high voltage between two electrodes within that tube.  What sort of light might be emitted by that tube when the electricity is applied?
\end{itemize}

\underbar{file i04103}
%(END_QUESTION)





%(BEGIN_ANSWER)


%(END_ANSWER)





%(BEGIN_NOTES)

This is an example of {\it absorption}, because the UV light is being absorbed by the ozone gas molecules.

%INDEX% Chemistry, spectroscopy

%(END_NOTES)


