%(BEGIN_QUESTION)
% Copyright 2009, Tony R. Kuphaldt, released under the Creative Commons Attribution License (v 1.0)
% This means you may do almost anything with this work of mine, so long as you give me proper credit

One of the elements created in the detonation of a nuclear weapon is strontium-90 ($^{90}$Sr), a highly radioactive substance.  Another radioactive substance -- formerly used as an energy source for illuminated watch dials -- is radium-226 ($^{226}$Ra).  Both of these elements are quite dangerous because they tend to collect in the {\it bones} of people and animals unfortunate enough to ingest them.

Examine the Periodic Table of the Elements to find both strontium and radium, then determine why these two elements tend to collect in the bones of people and animals.

\underbar{file i04104}
%(END_QUESTION)





%(BEGIN_ANSWER)

Hint: bones contain a lot of {\it calcium}.

%(END_ANSWER)





%(BEGIN_NOTES)

Both strontium and radium are Group 2 elements, falling into the same Periodic Table column as calcium (Ca), which means they engage in the same types of chemical reactions as calcium.  When ingested into a living body, both strontium and radium become metabolized the same as calcium, which of course becomes raw material for bones.

\vskip 10pt

Interestingly, iodine consumption is used as a preventative measure against strontium intake because iodine easily bonds with strontium, allowing it to pass unmetabolized through and out of the body.  Identifying this property of iodine is a more advanced concept than simply identifying the fact that strontium and radium are chemically similar to calcium.  One would have to know something about {\it valence} and {\it bonding} in order to make this determination.

%INDEX% Chemistry, periodicity

%(END_NOTES)


