%(BEGIN_QUESTION)
% Copyright 2009, Tony R. Kuphaldt, released under the Creative Commons Attribution License (v 1.0)
% This means you may do almost anything with this work of mine, so long as you give me proper credit

{\it Silicon} is the most common element used in the manufacture of semiconductor electronic devices, due to its {\it valence} and its tendency to form tetrahedral crystals.  Examine the Periodic Table of the Elements to identify other elements that might similarly serve as substrate materials for electronic semiconductors.

\vskip 20pt \vbox{\hrule \hbox{\strut \vrule{} {\bf Suggestions for Socratic discussion} \vrule} \hrule}

\begin{itemize}
\item{} {\it Germanium} used to be used extensively in the early (1950's) electronics industry, but was displaced by silicon as the element of choice because silicon has better high-temperature characteristics.  In fact, this trend of operating temperature versus placement in the periodic table holds true for all the semiconductor-capable elements.  From this trend, qualitatively determine the operating temperature ranges for all the semiconductor-capable elements.
\end{itemize}

\underbar{file i04105}
%(END_QUESTION)





%(BEGIN_ANSWER)


%(END_ANSWER)





%(BEGIN_NOTES)

Silicon is a Group 14 element, as are Germanium, Carbon, Tin, and Lead.  This suggests Group 14 elements are all capable of behaving as semiconductors.  

\vskip 10pt

Germanium has already been used for this purpose (in fact, it was one of the first materials used in this capacity!).  Carbon (in the crystalline form of diamond) theoretically may work the same, but only at very high temperatures.  Tin only functions as a semiconductor at low temperatures.  The higher the atomic number (i.e. the greater the period) in this column, the more easily each element ionizes, and therefore the more easily it behaves as a semiconductor at lower temperatures.  

For each of these Group-14 elements, a tetrahedral {\it allotrope} configuration is necessary for semiconductivity.  Carbon, for example, is not a semiconductor in its allotropic form of {\it graphite}, but it is in its allotropic form of {\it diamond}.


%INDEX% Chemistry, periodicity

%(END_NOTES)


