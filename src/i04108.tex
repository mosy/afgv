%(BEGIN_QUESTION)
% Copyright 2009, Tony R. Kuphaldt, released under the Creative Commons Attribution License (v 1.0)
% This means you may do almost anything with this work of mine, so long as you give me proper credit

Read and outline the ``Balancing Chemical Equations Using Algebra'' subsection of the ``Stoichiometry'' section of the ``Chemistry'' chapter in your {\it Lessons In Industrial Instrumentation} textbook.  Note the page numbers where important illustrations, photographs, equations, tables, and other relevant details are found.  Prepare to thoughtfully discuss with your instructor and classmates the concepts and examples explored in this reading.


\underbar{file i04108}
%(END_QUESTION)




%(BEGIN_ANSWER)


%(END_ANSWER)





%(BEGIN_NOTES)

If we treat the molecular coefficients (multipliers) as variables, we may set up sets of simultaneous equations for all the elements involved in a chemical reaction.  We choose one molecule in the reaction to have a coefficient of 1, then assign variables to all other molecules.  We then write one algebraic equation for each element represented in the chemical equation, each term in this algebraic equation being the product of the molecular coefficient and the number of atoms of that element in each molecule.

\vskip 10pt

Once the simultaneous equations are all set up, you may use substitution or any other technique to solve for the variables' values.  A distinct advantage of the algebraic method is that it works for any chemical formula representation, including ``compositional'' formulae where the subscript multipliers are non-integer.















\vskip 20pt \vbox{\hrule \hbox{\strut \vrule{} {\bf Suggestions for Socratic discussion} \vrule} \hrule}

\begin{itemize}
\item{} Explain why the ``trial and error'' method of equation-balancing would not work well for a compositional formula (non-integer multipliers).
\item{} {\bf Have students explain how each element's mathematical formula is developed from the reaction formula and assigned multiplier variables.}  This is a pattern students need to see for themselves, and the only way they are going to learn how to see patterns like this in prose is to practice doing exactly that!
\item{} Explain why we must set one of the compound quantities at 1 (or any fixed value) in order to solve for the values of the other coefficients when using simultaneous equations to balance chemical reactions.
\item{} Write a series of step-by-step instructions on how to use this method to balance chemical reaction equations.
\end{itemize}

%INDEX% Reading assignment: Lessons In Industrial Instrumentation, Chemistry (stoichiometry)

%(END_NOTES)


