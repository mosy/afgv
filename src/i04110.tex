%(BEGIN_QUESTION)
% Copyright 2009, Tony R. Kuphaldt, released under the Creative Commons Attribution License (v 1.0)
% This means you may do almost anything with this work of mine, so long as you give me proper credit

Read and outline the ``Explosive Limits'' subsection of the ``Classified Areas and Electrical Safety Measures'' section of the ``Process Safety and Instrumentation'' chapter in your {\it Lessons In Industrial Instrumentation} textbook.  Note the page numbers where important illustrations, photographs, equations, tables, and other relevant details are found.  Prepare to thoughtfully discuss with your instructor and classmates the concepts and examples explored in this reading.


\underbar{file i04110}
%(END_QUESTION)




%(BEGIN_ANSWER)


%(END_ANSWER)





%(BEGIN_NOTES)

In order for a fire or explosion to occur, these three criteria must be met: (1) Proper air/fuel ratio, (2) Ignition energy source, and (3) Self-sustaining reactivity.

\vskip 10pt

Too-rich and too-lean mixtures of fuel and air both fail to burn/explode.  LEL is the lean limit, and UEL is the rich limit, for any flammable substance.  Ignition source must have a minimum energy (MIE) in order to trigger combustion.  The {\it ignition curve} for any substance quantifies these limits of mixture and ignition energy, in a ``bathtub'' shape of curve.  These curves typically assume atmospheric oxygen as the oxidizer, at atmospheric pressure and ambient temperature.  Different oxidizers, temperature conditions, pressure conditions, and the presence of catalyst materials will all modify the ignition curve.










\vskip 20pt \vbox{\hrule \hbox{\strut \vrule{} {\bf Suggestions for Socratic discussion} \vrule} \hrule}

\begin{itemize}
\item{} Examine an {\it ignition curve} and explain what defines the limits and asymptotes of the curve.
\item{} Explain what {\it activation energy} is for a chemical reaction, and how this relates to explosive hazards.
\item{} Identify where the {\it activation energy} for any particular combustive reaction is represented on an ignition curve.
\item{} How might the ``ignition curve'' for a gas be modified if a catalyst were introduced to the test chamber?
\item{} How might the ``ignition curve'' for a gas be modified in an atmosphere of pure oxygen rather than air?
\item{} Identify the substance(s) with the greatest explosive potential, from the table of LEL and UEL values shown in the book.
\item{} Identify the substance(s) with the least explosive potential, from the table of LEL and UEL values shown in the book.
\item{} Identify circumstances which could change to alter the LEL, UEL, or MIE values for any particular explosive gas.
\item{} Roughly sketch the ignition curve for acetylene gas, showing the LEL and UEL asymptotes.
\item{} Roughly sketch the ignition curve for ethylene oxide gas, showing the LEL and UEL asymptotes.
\item{} Roughly sketch the ignition curve for propane gas, showing the LEL and UEL asymptotes.
\item{} Roughly sketch the ignition curve for an auto-igniting gas such as silane (SiH$_{4}$).
\item{} Explain why the electrical fuel level sensor in a car's gasoline tank does not pose an explosion hazard.
\item{} Some remote natural gas instrument clusters are 100\% pneumatic, but they operate on the natural gas itself rather than compressed air.  Explain why this does {\it not} pose a fire or explosion hazard inside the pneumatic instruments.
\end{itemize}

%INDEX% Reading assignment: Lessons In Industrial Instrumentation, Chemistry (LEL and UEL explosive limits)

%(END_NOTES)


