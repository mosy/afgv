%(BEGIN_QUESTION)
% Copyright 2009, Tony R. Kuphaldt, released under the Creative Commons Attribution License (v 1.0)
% This means you may do almost anything with this work of mine, so long as you give me proper credit

Read and outline the ``Gas Laws'' subsection of the ``Fluid Mechanics'' section of the ``Physics'' chapter in your {\it Lessons In Industrial Instrumentation} textbook.  Note the page numbers where important illustrations, photographs, equations, tables, and other relevant details are found.  Prepare to thoughtfully discuss with your instructor and classmates the concepts and examples explored in this reading.

\underbar{file i04112}
%(END_QUESTION)




%(BEGIN_ANSWER)

%(END_ANSWER)





%(BEGIN_NOTES)

The {\it Ideal Gas Law} mathematically expresses the relationship between pressure, volume, temperature, and molecule quantity for most any gas:

$$PV = nRT \hskip 30pt \hbox{or} \hskip 30pt PV = NkT$$

\noindent
Where,

$P$ = Absolute pressure (atmospheres)

$V$ = Volume (liters)

$n$ = Gas quantity (moles)

$N$ = Gas quantity (molecules)

$R$ = Universal gas constant (0.0821 L $\cdot$ atm / mol $\cdot$ K)

$k$ = Boltzmann's constant (1.38 $\times$ 10$^{-23}$ J / K)

$T$ = Absolute temperature (K)

\vskip 10pt

The formulae for Charles' Law (constant pressure), Boyles' Law (constant temperature), and Gay-Lussac's Law (constant volume) are all derived from the Ideal Gas Law.

\vskip 10pt

Since gases are mostly composed of empty space between isolated molecules, it doesn't matter much what the molecular species is, which is why this law is universal for gases.  The Ideal Gas Law only applies to gases of modest density (e.g. no crazy temperatures or pressures) and with no phase changes.

\vskip 10pt

To account for real-life gas behavior, a variation of the Ideal Gas Law incorporates another factor ($Z$) called the {\it compressibility} of the gas.  This factor has a maximum value of 1 (a perfect gas), and will be less than one for any real gas:

$$PV = ZnRT$$








\vskip 20pt \vbox{\hrule \hbox{\strut \vrule{} {\bf Suggestions for Socratic discussion} \vrule} \hrule}

\begin{itemize}
\item{} Referencing the Ideal Gas Law, explain what happens to the pressure of a gas when its enclosing volume decreases (e.g. air being compressed inside the cylinder of a compressor).
\item{} Referencing the Ideal Gas Law, explain what happens to the pressure of a gas when its temperature increases (e.g. a Class III filled-bulb temperature sensor).
\item{} Referencing the Ideal Gas Law, explain what happens to the pressure of a gas when more molecules are introduced into a chamber of the same volume (e.g. pumping up a car tire).
\item{} Does the Ideal Gas Law apply to the air filling a pneumatic tire?  Why or why not?
\item{} Does the Ideal Gas Law apply to natural gas inside of a pipeline?  Why or why not?
\item{} Does the Ideal Gas Law apply to petroleum oil inside of a pipeline?  Why or why not?
\item{} Does the Ideal Gas Law apply to the conditions inside a propane (LPG) storage tank?  Why or why not?
\item{} Does the Ideal Gas Law apply to the conditions inside the steam drum of a boiler?  Why or why not?
\item{} Calculate both the mass and the volume occupied by a five moles of methane (CH$_{4}$) gas at a pressure of 1.4 atmospheres and a temperature of 310 Kelvin.  How will your answer be affected if you happen to discover that the compressibility factor of methane is less than 1?
\end{itemize}









\vfil \eject

\noindent
{\bf Summary Quiz:}

Calculate the volume occupied by 415 moles of air at a pressure of 14.7 PSIA and a temperature of -6 $^{o}$C, based on the Ideal Gas Law formula:

$$PV = nRT$$

\noindent
Where,

$P$ = Absolute pressure (atmospheres)

$V$ = Volume (liters)

$n$ = Gas quantity (moles)

$R$ = Universal gas constant (0.0821 L $\cdot$ atm / mol $\cdot$ K)

$T$ = Absolute temperature (K)

\vskip 10pt

\begin{itemize}
\item{} 28.23 liters
\vskip 5pt 
\item{} 9102 liters
\vskip 5pt 
\item{} -204.4 liters
\vskip 5pt 
\item{} 619.2 liters
\vskip 5pt 
\item{} 13.91 liters
\vskip 5pt 
\item{} 6100 liters
\end{itemize}

%INDEX% Reading assignment: Lessons In Industrial Instrumentation, Chemistry (LEL and UEL explosive limits)

%(END_NOTES)


