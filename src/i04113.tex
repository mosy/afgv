%(BEGIN_QUESTION)
% Copyright 2011, Tony R. Kuphaldt, released under the Creative Commons Attribution License (v 1.0)
% This means you may do almost anything with this work of mine, so long as you give me proper credit

Calculate the mass of the following quantities of pure chemical compound.  In each case, feel free to use atomic mass values rounded to the nearest whole number (from a Periodic Table) in your calculations:

\begin{itemize}
\item{} 35.2 moles of alumina Al$_{2}$O$_{3}$ at 25 $^{o}$C
\vskip 10pt
\item{} 10.6 moles of nitroglycerine C$_{3}$H$_{5}$N$_{3}$O$_{9}$ at 77 $^{o}$C
\vskip 10pt
\item{} 3.7 moles of phosgene COCl$_{2}$ at 145 $^{o}$F
\vskip 10pt
\item{} 130 moles of tetraethyl pyrophosphate or ``TEPP'' [(CH$_{3}$CH$_{2}$O)$_{2}$PO]$_{2}$O at $-10$ $^{o}$F
\end{itemize}


\vskip 20pt \vbox{\hrule \hbox{\strut \vrule{} {\bf Suggestions for Socratic discussion} \vrule} \hrule}

\begin{itemize}
\item{} Demonstrate how to {\it estimate} numerical answers for this problem without using a calculator.
\item{} Calculate the amount of heat required to warm 30 moles of water from 20 degrees Celsius to 25 degrees Celsius.
\item{} Calculate the amount of heat required to warm 200 moles of hydrogen gas from 50 degrees Celsius to 70 degrees Celsius.
\end{itemize}

\underbar{file i04113}
%(END_QUESTION)





%(BEGIN_ANSWER)

\begin{itemize}
\item{} 35.2 moles of alumina Al$_{2}$O$_{3}$ at 25 $^{o}$C = {\bf 3590.4 grams} = {\bf 3.5904 kg}
\vskip 10pt
\item{} 10.6 moles of nitroglycerine C$_{3}$H$_{5}$N$_{3}$O$_{9}$ at 77 $^{o}$C = {\bf 2406.2 grams} = {\bf 2.4062 kg}
\vskip 10pt
\item{} 3.7 moles of phosgene COCl$_{2}$ at 145 $^{o}$F = {\bf 366.3 grams} = {\bf 0.3663 kg}
\vskip 10pt
\item{} 130 moles of tetraethyl pyrophosphate or ``TEPP'' [(CH$_{3}$CH$_{2}$O)$_{2}$PO]$_{2}$O at $-10$ $^{o}$F = {\bf 37,700 grams} = {\bf 37.700 kg}
\end{itemize}

%(END_ANSWER)





%(BEGIN_NOTES)

\noindent
35.2 moles of alumina Al$_{2}$O$_{3}$:

$$35.2 \left[(2)(27) + (3)(16) \right] = 3590.4 \hbox{ grams}$$

\vskip 10pt

\noindent
10.6 moles of nitroglycerine C$_{3}$H$_{5}$N$_{3}$O$_{9}$: 

$$10.6 \left[(3)(12) + (5)(1) + (3)(14) + (9)(16) \right] = 2406.2 \hbox{ grams}$$

\vskip 10pt

\noindent
\item{} 3.7 moles of phosgene COCl$_{2}$: 

$$3.7 \left[(1)(12) + (1)(16) + (2)(35.5) \right] = 366.3 \hbox{ grams}$$

\vskip 10pt

\noindent
130 moles of tetraethyl pyrophosphate or ``TEPP'' [(CH$_{3}$CH$_{2}$O)$_{2}$PO]$_{2}$O: 

$$130 \left[ (2) \left[ (2) \left((1)(12) + (3)(1) + (1)(12) + (2)(1) + (16) \right) + (1)(31) + (1)(16) \right] + 16 \right] = 37700 \hbox{ grams}$$

\vskip 10pt

The temperature figures are extraneous information, included for the purpose of challenging students to identify whether or not information is relevant to solving a particular problem.

%INDEX% Chemistry, stoichiometry: moles

%(END_NOTES)


