%(BEGIN_QUESTION)
% Copyright 2011, Tony R. Kuphaldt, released under the Creative Commons Attribution License (v 1.0)
% This means you may do almost anything with this work of mine, so long as you give me proper credit

Research the {\it NIOSH Pocket Guide To Chemical Hazards} (DHHS publication number 2005-149) to answer the following questions:

\vskip 10pt

What does the NIOSH designation {\it IDLH} refer to?

\vskip 10pt

What does the NIOSH designation {\it REL} refer to?

\vskip 10pt

What does the OSHA designation {\it PEL} refer to?  

\vskip 20pt \vbox{\hrule \hbox{\strut \vrule{} {\bf Suggestions for Socratic discussion} \vrule} \hrule}

\begin{itemize}
\item{} Identify some of the criteria used to establish the REL and PEL values.
\item{} Identify some common units of measurement for these exposure limits, citing a real example from the NIOSH pocket guide.
\item{} Which seems the more conservative (safe) exposure limit, the NIOSH {\it REL} or the OSHA {\it PEL}?
\end{itemize}

\underbar{file i04114}
%(END_QUESTION)





%(BEGIN_ANSWER)

%(END_ANSWER)





%(BEGIN_NOTES)

IDLH (``Immediately Dangerous to Life or Health'') values based on possible effects after a 30 minute exposure at that concentration.  The goal of this standard was to set values of exposure such that workers would have enough time to flee to safety without debilitating effects.  When acute toxicity data were unavailable to set an IDLH limit, other criteria were used such as 10\% of the Lower Explosive Limit (LEL).  (pages ix to x)

\vskip 10pt

NIOSH ``REL'' (``Recommended Exposure Limit'') time-weighted average (TWA) values calculated for a 10-hour workday and a 40-hour workweek.  Short-term exposure limit (STEL) values are calculated for a 15 minute time-weighted average exposure.  Ceiling (C) limits should not be endured for any length of time.  (page xi)

\vskip 10pt

OSHA ``PEL'' time-weighted average (TWA) values calculated for an 8-hour workday and a 40-hour workweek.  Short-term (ST) values calculated for a 15 minute time-weighted average exposure.  Ceiling (C) limits should not be endured for any length of time.  (page xii)

\vskip 10pt

NIOSH exposure limits tend to be more conservative than OSHA exposure limits.

%INDEX% Safety, exposure limit: NIOSH immediately dangerous to life or health (IDLH)
%INDEX% Safety, exposure limit: NIOSH recommended exposure limit (REL)
%INDEX% Safety, exposure limit: OSHA permissible exposure limit (PEL)

%(END_NOTES)


