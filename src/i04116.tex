%(BEGIN_QUESTION)
% Copyright 2009, Tony R. Kuphaldt, released under the Creative Commons Attribution License (v 1.0)
% This means you may do almost anything with this work of mine, so long as you give me proper credit

High-temperature combustion causes some of the nitrogen molecules in the combustion air to join with oxygen molecules to form a gaseous pollutant called nitrogen dioxide (NO$_{2}$).  In large fossil-fuel power plants, this pollutant is sometimes ``scrubbed'' from the exhaust gases by injecting a spray of pure ammonia (NH$_{3}$) into the exhaust stream.  The resulting chemical reaction eliminates the pollutant by reducing it to water vapor and pure nitrogen gas:

$$6\hbox{NO}_2 + 8\hbox{NH}_3 \to 12\hbox{H}_2\hbox{O} + 7\hbox{N}_2$$

For every six molecules of nitrogen dioxide, it takes eight molecules of ammonia to reduce it to twelve molecules of water and seven molecules of nitrogen gas.

\vskip 10pt

Calculate the molar quantity of ammonia required to reduce 1350 moles of nitrogen dioxide gas. 

\vskip 10pt

Also, calculate the mass of this nitrogen dioxide quantity and the mass of this ammonia quantity.

\vskip 20pt \vbox{\hrule \hbox{\strut \vrule{} {\bf Suggestions for Socratic discussion} \vrule} \hrule}

\begin{itemize}
\item{} Is there such a thing as a {\it Law of Molar Conservation} as there is for mass and energy conservation? 
\item{} What is so bad about nitrogen dioxide gas that we would try to scrub it from the exhaust?  After all, what has NO$_{2}$ ever done to hurt {\it you?}
\item{} Calculate a {\it mass ratio} of ammonia to nitrogen dioxide to make it easy for anyone to compute the required quantity (mass) of ammonia to neutralize any given quantity (mass) of nitrogen dioxide.
\end{itemize}

\underbar{file i04116}
%(END_QUESTION)





%(BEGIN_ANSWER)

\noindent
{\bf Partial answer:}

\vskip 10pt

$m$ = 30,600 grams of ammonia = 30.6 kg.

%(END_ANSWER)





%(BEGIN_NOTES)

Since the molar ratio of ammonia to nitrogen dioxide is 8:6, the molar quantity of ammonia will be $8 \over 6$ times the molar quantity of the nitrogen dioxide.  For 1350 moles of NO$_{2}$ gas, therefore, we will require 1800 moles of NH$_{3}$ to completely react.

\vskip 10pt

The molecular weight of NO$_{2}$ gas is 14 + 2(16) = 46 amu (i.e. 46 grams per mole).  Therefore:

$$\left( {1350 \hbox{ mol NO}_2 \over 1} \right) \left( {46 \hbox{ g} \over 1 \hbox{ mol NO}_2} \right) = 62100 \hbox{ g} = 62.1 \hbox{ kg}$$

\vskip 10pt

The molecular weight of NH$_{3}$ gas is 14 + 3(1) = 17 amu (i.e. 17 grams per mole).  Therefore:

$$\left( {1800 \hbox{ mol NH}_3 \over 1} \right) \left( {17 \hbox{ g} \over 1 \hbox{ mol NH}_3} \right) = 30600 \hbox{ g} = 30.6 \hbox{ kg}$$

\vskip 10pt

The {\it mass} ratio of NH$_{3}$ to NO$_{2}$ is:

$$\left( 8 \hbox{ mol NH}_3 \over 6 \hbox{ mol NO}_2 \right) \left( 17 \hbox{ g / mol NH}_3 \over 46 \hbox{ g / mol NO}_2 \right) = 0.493$$


%INDEX% Chemistry, stoichiometry: moles

%(END_NOTES)


