%(BEGIN_QUESTION)
% Copyright 2011, Tony R. Kuphaldt, released under the Creative Commons Attribution License (v 1.0)
% This means you may do almost anything with this work of mine, so long as you give me proper credit

Calculate the molar quantity ($n$, in moles) for the following quantities of pure chemical substances.  Feel free to use atomic mass values rounded to the nearest whole number (from a Periodic Table) in your calculations:

\begin{itemize}
\item{} 500 grams of pure iron at 10 $^{o}$C and 1.2 atmospheres
\vskip 10pt
\item{} 1.1 kilograms of pure propane C$_{3}$H$_{8}$ at $-$30 $^{o}$C and 3 atmospheres
\vskip 10pt
\item{} 250 kilograms of naphthalene C$_{10}$H$_{8}$ at 0 $^{o}$C and 45 kPaA
\vskip 10pt
\item{} 71 grams of hexafluoroacetone (CF$_{3}$)$_{2}$CO at 110 $^{o}$F and 50 bar (gauge)
\end{itemize}

\vskip 20pt \vbox{\hrule \hbox{\strut \vrule{} {\bf Suggestions for Socratic discussion} \vrule} \hrule}

\begin{itemize}
\item{} What effects do temperature and pressure have on the mass of a sample?
\item{} Demonstrate how to {\it estimate} numerical answers for this problem without using a calculator.
\item{} Does the {\it phase} of the substance (i.e. gas, liquid, solid) matter in these calculations?  Why or why not?
\end{itemize}

\underbar{file i04118}
%(END_QUESTION)





%(BEGIN_ANSWER)

\noindent
{\bf Partial answer:}

\begin{itemize}
%\item{} 500 grams of pure iron at 10 $^{o}$C and 1.2 atmospheres = {\bf 8.953 moles}
%\vskip 10pt
\item{} 1.1 kilograms of pure propane C$_{3}$H$_{8}$ at $-30$ $^{o}$C and 3 atmospheres = {\bf 25 moles}
\vskip 10pt
%\item{} 250 kilograms of naphthalene C$_{10}$H$_{8}$ at 0 $^{o}$C and 45 kPaA = {\bf 1953 moles}
%\vskip 10pt
\item{} 71 grams of hexafluoroacetone (CF$_{3}$)$_{2}$CO at 110 $^{o}$F and 50 bar (gauge) = {\bf 0.4277 moles}
\end{itemize}

%(END_ANSWER)





%(BEGIN_NOTES)

\noindent
Formula weight of iron = 55.847 AMU (grams per mole), so:

$$\left(500 \hbox{ grams} \over 1 \right) \left(1 \hbox{ mol iron} \over 55.847 \hbox{ grams}\right) = 8.953 \hbox{ moles of iron}$$

\vskip 10pt

\noindent
Formula weight of propane = (3)(12) + (8)(1) = 44 AMU (grams per mole), so:

$$\left(1100 \hbox{ grams} \over 1 \right) \left(1 \hbox{ mol propane} \over 44 \hbox{ grams}\right) = 25 \hbox{ moles of propane}$$

\vskip 10pt

\noindent
Formula weight of naphthalene = (10)(12) + (8)(1) = 128 AMU (grams per mole), so:

$$\left(250000 \hbox{ grams} \over 1 \right) \left(1 \hbox{ mol naphthalene} \over 128 \hbox{ grams}\right) = 1953 \hbox{ moles of naphthalene}$$

\vskip 10pt

\noindent
Formula weight of hexafluoroacetone = $(2) \left[ (1)(12) + (3)(19) \right] + (1)(12) + (1)(16)$ = 166 AMU (grams per mole), so:

$$\left(71 \hbox{ grams} \over 1 \right) \left(1 \hbox{ mol hexafluoroacetone} \over 166 \hbox{ grams}\right) = 0.4277 \hbox{ moles of hexafluoroacetone}$$

\vskip 10pt

The temperature and pressure figures are extraneous information, included for the purpose of challenging students to identify whether or not information is relevant to solving a particular problem.  Temperature and pressure have absolutely no effect whatsoever on a sample's mass, since mass is an intrinsic property of matter.

%INDEX% Chemistry, stoichiometry: moles

%(END_NOTES)


