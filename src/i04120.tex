%(BEGIN_QUESTION)
% Copyright 2011, Tony R. Kuphaldt, released under the Creative Commons Attribution License (v 1.0)
% This means you may do almost anything with this work of mine, so long as you give me proper credit

When methyl alcohol (``methanol'') vapor is mixed with air at sea-level atmospheric pressure and room temperature (25 $^{o}$C), a molecular concentration of 1 ppm (1 molecule of methanol for every 1,000,000 sample molecules) is equivalent to a mass concentration of 1.31 milligrams methanol per cubic meter.

\vskip 10pt

Use the Ideal Gas Law to prove this equivalence for methanol (1 ppm = 1.31 mg/m$^{3}$).

\underbar{file i04120}
%(END_QUESTION)





%(BEGIN_ANSWER)

Assuming ``room temperature'' to be 25 degrees Celsius and sea-level pressure to be 1 atmosphere. we may calculate the number of moles of air molecules in a cubic meter of air as follows:

\vskip 10pt

1 m$^{3}$ = 1000 liters

$$PV = nRT$$

$$n = {PV \over RT}$$

$$n = {(1)(1000) \over (0.0821)(298.15)} = 40.85 \hbox{ moles}$$

A concentration of 1 ppm means 1 molecule out of a million will be methanol, the other 999,999 being air molecules.  Thus, for a methanol vapor concentration of 1 ppm we would expect to find 0.00004058 moles of methyl alcohol in 40.85 moles of sample (1 cubic meter). 

\vskip 10pt

Given methanol's molecular weight of CH$_{4}$O 32.04 grams per mole, this equates to 0.0013089 grams (1.3089 mg) in one cubic meter.

%(END_ANSWER)





%(BEGIN_NOTES)


%INDEX% Chemistry, stoichiometry: moles

%(END_NOTES)


