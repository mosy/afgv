%(BEGIN_QUESTION)
% Copyright 2009, Tony R. Kuphaldt, released under the Creative Commons Attribution License (v 1.0)
% This means you may do almost anything with this work of mine, so long as you give me proper credit

Read and outline the ``Terms and Definitions'' section of the ``Chemistry'' chapter in your {\it Lessons In Industrial Instrumentation} textbook.  Note the page numbers where important illustrations, photographs, equations, tables, and other relevant details are found.  Prepare to thoughtfully discuss with your instructor and classmates the concepts and examples explored in this reading.

\underbar{file i04121}
%(END_QUESTION)




%(BEGIN_ANSWER)


%(END_ANSWER)





%(BEGIN_NOTES)

\begin{itemize}
\item{} {\bf Atom}: the smallest unit of matter that may be isolated by chemical means.
\item{} {\bf Particle}: a part of an atom, separable from the other portions only by levels of energy far in excess of chemical reactions.
\item{} {\bf Proton}: a type of ``elementary'' particle, found in the nucleus of an atom, possessing a positive electrical charge.
\item{} {\bf Neutron}: a type of ``elementary'' particle, found in the nucleus of an atom, possessing no electrical charge, and having nearly the same amount of mass as a proton.
\item{} {\bf Electron}: a type of ``elementary'' particle, found in regions surrounding the nucleus of an atom, possessing a negative electrical charge, and having just a small fraction of the mass of a proton or neutron. 
\item{} {\bf Element}: a substance composed of atoms all sharing the same number of protons in their nuclei (e.g. hydrogen, helium, nitrogen, iron, cesium, fluorine). 
\item{} {\bf Atomic number}: the number of protons in the nucleus of an atom -- this quantity defines the chemical identify of an atom.  
\item{} {\bf Atomic mass} or {\bf Atomic weight}: the total number of elementary particles in the nucleus of an atom (protons + neutrons) -- this quantity defines the vast majority of an atom's mass, since the only other elementary particle (electrons) are so light-weight by comparison to protons and neutrons.  
\item{} {\bf Ion}: an atom or molecule that is not electrically balanced (equal numbers of protons and electrons). 
\itemitem{} {\bf Cation}: a positively-charged ion, called a ``cation'' because it is attracted toward the negative electrode (cathode) immersed in a solution.  
\itemitem{} {\bf Anion}: a negatively-charged ion, called an ``anion'' because it is attracted toward the positive electrode (anode) immersed in a solution.  
\item{} {\bf Isotope}: a variation on the theme of an element -- atoms sharing the same number of protons in their nuclei, but having different numbers of neutrons, are called ``isotopes'' (e.g. uranium-235 versus uranium-238).  
\item{} {\bf Molecule}: the smallest unit of matter composed of two or more atoms joined by electron interaction in a fixed ratio (e.g. water: H$_{2}$O).  The smallest unit of a {\bf compound}.  
\item{} {\bf Compound}: a substance composed of identical molecules (e.g. pure water). 
\item{} {\bf Isomer}: a variation on the theme of a compound -- molecules sharing the same numbers and types of atoms, but having different structural forms, are called ``isomers''.  For example, the sugars glucose and fructose are isomers, both having the same formula C$_{6}$H$_{12}$O$_{6}$ but having disparate structures.  An isomer is to a molecule as an isotope is to an atomic nucleus.  
\item{} {\bf Mixture}: a substance composed of different atoms or molecules not electronically bonded to each other. 
\item{} {\bf Solution}: an homogeneous mixture at the molecular level (different atoms/molecules thoroughly mixed together).  A solution may be a gas, a liquid, or a solid (e.g. air, saltwater, doped silicon).    
\itemitem{} {\bf Solvent}: the majority element or compound in a solution.  Chemists usually consider water to be the {\bf universal solvent}.  
\itemitem{} {\bf Solute}: the minority element or compound in a solution (may be more than one).  
\itemitem{} {\bf Precipitate}: (noun) solute that has ``fallen out of solution'' due to the solution being saturated with that element or compound; (verb) the process of solute separating from the rest of the solution.  (e.g. Mixing too much salt with water results in some of that salt {\bf precipitating} out of the water to form a solid pile at the bottom.)  
\itemitem{} {\bf Supernatant}: the solution remaining above the precipitate.  
\item{} {\bf Suspension}: an heterogeneous mixture where separation occurs due to gravity (e.g. mud).  
\item{} {\bf Colloid} or {\bf Colloidal suspension}: an heterogeneous mixture where separation does not occur (or occurs at a negligible pace) due to gravity (e.g. milk).  
\itemitem{} {\bf Aerosol}: A colloid formed of a solid or liquid substance dispersed in a gas medium. 
\itemitem{} {\bf Foam}: A colloid formed of a gas dispersed in either a liquid or a solid medium. 
\itemitem{} {\bf Emulsion}: A colloid formed of a liquid dispersed in either a liquid or a solid medium. 
\itemitem{} {\bf Sol}: A colloid formed of a solid dispersed in either a liquid or a solid medium. 
\end{itemize}


\filbreak

\vskip 20pt \vbox{\hrule \hbox{\strut \vrule{} {\bf Suggestions for Socratic discussion} \vrule} \hrule}

\begin{itemize}
\item{} Why do you suppose water is often called the ``universal solvent''?
\item{} Provide one practical example of an {\it aerosol}.
\item{} Provide one practical example of a liquid {\it foam}.
\item{} Provide one practical example of a solid {\it foam}.
\item{} Provide one practical example of a liquid {\it emulsion}.
\item{} Provide one practical example of a solid {\it emulsion}.
\item{} Provide one practical example of a liquid {\it sol}.
\item{} Provide one practical example of a solid {\it sol}.
\end{itemize}









\vfil \eject

\noindent
{\bf Prep Quiz}

An ``ion'' is:

\begin{itemize}
\item{} A particle found inside the nucleus of an atom
\vskip 5pt
\item{} An element that does not chemically react with others
\vskip 5pt
\item{} An atom or molecule that is not electrically balanced
\vskip 5pt
\item{} A molecule having a different structure than another
\vskip 5pt
\item{} The minority element or compound in a liquid solution
\vskip 5pt
\item{} An atom having a different number of neutrons than another
\vskip 5pt
\item{} The majority element or compound in a liquid solution
\end{itemize}

%INDEX% Reading assignment: Lessons In Industrial Instrumentation, Chemistry (solutions)

%(END_NOTES)


