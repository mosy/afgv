%(BEGIN_QUESTION)
% Copyright 2009, Tony R. Kuphaldt, released under the Creative Commons Attribution License (v 1.0)
% This means you may do almost anything with this work of mine, so long as you give me proper credit

Read and outline the ``Ions In Liquid Solutions'' section of the ``Chemistry'' chapter in your {\it Lessons In Industrial Instrumentation} textbook.  Note the page numbers where important illustrations, photographs, equations, tables, and other relevant details are found.  Prepare to thoughtfully discuss with your instructor and classmates the concepts and examples explored in this reading.

\underbar{file i04122}
%(END_QUESTION)




%(BEGIN_ANSWER)


%(END_ANSWER)





%(BEGIN_NOTES)

Molecules in liquids often spontaneously split into positive and negative ion pairs.  Molten sodium chloride salt (NaCl) for example exists almost entirely as Na$^{+}$ and Cl$^{-}$ ions.  A very small percentage (0.00000018\%) of pure water molecules do the same (H$^{+}$ and OH$^{-}$).  Thus, molten salt is an excellent conductor of electricity while pure water is a very poor conductor of electricity.

\vskip 10pt

Ion molarity tends to increase with solution temperature.

\vskip 10pt

Ion molarity increases in water when an ionic compound is added to it (e.g. NaCl in H$_{2}$O).












\vskip 20pt \vbox{\hrule \hbox{\strut \vrule{} {\bf Suggestions for Socratic discussion} \vrule} \hrule}

\begin{itemize}
\item{} Describe the difference between an {\it ionic} and a {\it covalent} compound.
\item{} Explain what ``molarity'' means, and why it is an important parameter in solution chemistry.
\item{} Explain why pure water is such a poor conductor of electricity.
\item{} Explain why the concentration of ions in a solution tends to increase as the solution's temperature increases.  What, exactly, is happening at the molecular level when a solution's temperature rises?  {\it Hint: what is the definition of temperature?}
\end{itemize}













\vfil \eject

\noindent
{\bf Prep Quiz}

An ``ionic'' substance such as table salt (NaCl) is characterized by:

\begin{itemize}
\item{} Very high molarity when dissolved in water or other liquid
\vskip 5pt
\item{} Light color -- ionic substances are almost always white
\vskip 5pt
\item{} Dark color -- ionic substances are nearly opaque to light
\vskip 5pt
\item{} Chemical inertness -- no reactivity with other chemicals
\vskip 5pt
\item{} Complete separation into cations and anions in liquid form
\vskip 5pt
\item{} Exothermic tendencies even at room temperature
\end{itemize}












\vfil \eject

\noindent
{\bf Prep Quiz}

A ``covalent'' substance such as methane (CH$_{4}$) is characterized by:

\begin{itemize}
\item{} Dark color -- covalent substances are nearly opaque to light
\vskip 5pt
\item{} Light color -- covalent substances are almost always white
\vskip 5pt
\item{} Poor electrical conductivity when in liquid form
\vskip 5pt
\item{} Chemical inertness -- no reactivity with other chemicals
\vskip 5pt
\item{} Extreme radioactivity, with a very short half-life time
\vskip 5pt
\item{} Good electrical conductivity when in liquid form
\end{itemize}


%INDEX% Reading assignment: Lessons In Industrial Instrumentation, Chemistry (ions in liquid solutions)

%(END_NOTES)


