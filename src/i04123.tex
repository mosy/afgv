%(BEGIN_QUESTION)
% Copyright 2009, Tony R. Kuphaldt, released under the Creative Commons Attribution License (v 1.0)
% This means you may do almost anything with this work of mine, so long as you give me proper credit

Bring the following materials to class to perform an experiment demonstrating conductivity and electrolysis in water.  You may partner with one or two classmates to bring these materials, and perform the experiment together.  You are also free to perform the experiment before class to get an advance-start on understanding the principles involved:

\begin{itemize}
\item{} Two paper clips
\vskip 5pt
\item{} Small drinking cup (to fill with water)
\vskip 5pt
\item{} At least two ``alligator clip'' jumper wires
\vskip 5pt
\item{} One 6-volt or 9-volt battery
\vskip 5pt
\item{} One or more packets of table salt (from the cafeteria)
\vskip 5pt
\item{} Stir-stick (or pencil) to mix salt into the water
\vskip 5pt
\item{} Multimeter capable of measuring current in the milliamp range
\end{itemize}

Bend the paperclips so they form two electrodes which will dip into water you put in the cup.  Place each paperclip on opposite sides of the cup, maximizing the distance between them.  Fill the cup with water, and connect the paper-clip electrodes to the battery and to your multimeter (forming a series circuit so the meter measures {\it current}) using the ``alligator clip'' jumper wires.  Take one measurement of the battery's voltage for reference later (you will need to know the voltage so you can calculate conductance: $G = {I \over V}$).

Document the meter's current measurement indication.  Add approximately one-quarter of the salt to the water and stir it until it completely dissolves, then document the meter's indication and calculate the conductance ($G$).  Repeat this process over and over until you have added all the salt to the water.  {\it Be careful to maintain the same amount of separation between the two paper-clip electrodes when taking measurements, and make absolutely sure they never touch each other!}  Maintaining a constant separation is important for measurement accuracy, and avoiding contact prevents you from blowing a fuse in your meter.

\vskip 10pt

% No blank lines allowed between lines of an \halign structure!
% I use comments (%) instead, so that TeX doesn't choke.

\vbox{\offinterlineskip
\halign{\strut
\vrule \quad\hfil # \ \hfil & 
\vrule \quad\hfil # \ \hfil \vrule \cr
\noalign{\hrule}
%
% First row
{\bf Battery voltage} & \hskip 50pt volts \cr
%
\noalign{\hrule}
} % End of \halign 
} % End of \vbox



% No blank lines allowed between lines of an \halign structure!
% I use comments (%) instead, so that TeX doesn't choke.

$$\vbox{\offinterlineskip
\halign{\strut
\vrule \quad\hfil # \ \hfil & 
\vrule \quad\hfil # \ \hfil & 
\vrule \quad\hfil # \ \hfil \vrule \cr
\noalign{\hrule}
%
% First row
{\bf Salt content} & {\bf Measured current} ($I$) & {\bf Calculated conductance} ($G = {I \over V}$) \cr
%
\noalign{\hrule}
%
% Another row
No salt in the water &  & \cr
%
\noalign{\hrule}
%
% Another row
$\approx$ 25\% of the salt added &  & \cr
%
\noalign{\hrule}
%
% Another row
$\approx$ 50\% of the salt added &  & \cr
%
\noalign{\hrule}
%
% Another row
$\approx$ 75\% of the salt added  &  & \cr
%
\noalign{\hrule}
%
% Another row
100\% of the salt added &  & \cr
%
\noalign{\hrule}
} % End of \halign 
}$$ % End of \vbox

\vskip 10pt

Devise a method of determining when the water has become {\it saturated} with salt, and describe your procedure.  Note: you may require more salt than used in your initial test in order to achieve saturation.

\vskip 10pt

After running this experiment for a short while, you will probably notice a lot of bubbles collecting on the two electrodes.  At least one of these gases will have a very distinctive odor.  Identify which gas this is, and explain why it is being generated based on your knowledge of chemistry.

\underbar{file i04123}
%(END_QUESTION)





%(BEGIN_ANSWER)


%(END_ANSWER)





%(BEGIN_NOTES)

{\it Saturation} will be indicated by a ``leveling off'' of current with additional doses of salt, as well as precipitation of salt observed at the bottom of the cup.

\vskip 10pt

After a while, students should be able to detect the odor of {\it chlorine} gas emitted from the positive electrode, coming from the electrolysis of salt (NaCl).

%INDEX% Chemistry, electrolysis of saltwater

%(END_NOTES)


