%(BEGIN_QUESTION)
% Copyright 2009, Tony R. Kuphaldt, released under the Creative Commons Attribution License (v 1.0)
% This means you may do almost anything with this work of mine, so long as you give me proper credit

Read and outline the introduction, the ``Dissociation and Ionization in Aqueous Solutions'' subsection, and the ``Two-Electrode Conductivity Probes'' subsection of the ``Conductivity Measurement'' section of the ``Continuous Analytical Measurement'' chapter in your {\it Lessons In Industrial Instrumentation} textbook.  Note the page numbers where important illustrations, photographs, equations, tables, and other relevant details are found.  Prepare to thoughtfully discuss with your instructor and classmates the concepts and examples explored in this reading.

\underbar{file i04124}
%(END_QUESTION)




%(BEGIN_ANSWER)


%(END_ANSWER)





%(BEGIN_NOTES)

Electrical conductivity in a liquid solution is a non-specific form of analytical measurement.  Conductivity measurement is useful when the specific ion type doesn't matter, or when the dominant ion type is well-known.

\vskip 10pt

Electrical conductivity only occurs in liquids when liquid molecules ionize into anions ($-$ charge) and cations (+ charge).  Electrical conductivity only occurs in gases when they are heated to very high temperatures to form ions.

\vskip 10pt

A compound is ionically bonded when its constituent atoms are electrostatically attracted to one another due to opposite charges.  A compound is covalently bonded when its constituent atoms mutually share valence electrons.

\vskip 10pt

Pure water contains very few ions (H$^{+}$ and OH$^{-}$), but the concentration of ions in water is easily boosted by adding an {\it electrolyte} which is any ionically-bonded compound such as a salt or a metal.

\vskip 10pt

The amount of electrical conductance ($G$) between two points in a circuit is a ratio of current to voltage:

$$G = {I \over V}$$

This value (in Siemens) is not necessarily useful to use, for it depends as much on the geometry of the electrodes as it does on the ionic concentration of the liquid.  If we are trying to represent the electrical properties of the liquid itself, we need a measurement that accounts for electrode geometry:

$$k = {Gd \over A}$$

\noindent
Where,

$G$ = Conductance, in Siemens (S)

$k$ = Specific conductance (conductivity) of liquid, in Siemens per centimeter (S/cm)

$A$ = Electrode area (each), in square centimeters (cm$^{2}$)

$d$ = Electrode separation distance, in centimeters (cm)

\vskip 10pt


Cell geometry is typically specified as a single value, representing the fraction $d \over A$:

$$\theta = {d \over A}$$

Thus, conductivity is the product of measured conductance and cell geometry:

$$k = G \theta$$

\noindent
Where,

$k$ = Specific conductivity of liquid, in Siemens per centimeter (S/cm)

$G$ = Conductance, in Siemens (S)

$\theta$ = Cell constant, in inverse centimeters (cm$^{-1}$)

\vskip 10pt

Two-electrode conductivity cells suffer from errors to to ``plating'' of the electrodes, which artificially decreases the measured conductivity of the liquid.  Ions and mineral scale accumulated on the plates adds to the measured resistance, making the liquid seem less conductive than it is.  This plating may be minimized by using an AC excitation source, but never completely eliminated.










\filbreak

\vskip 20pt \vbox{\hrule \hbox{\strut \vrule{} {\bf Suggestions for Socratic discussion} \vrule} \hrule}

\begin{itemize}
\item{} {\bf In what ways may a two-wire conductivity probe be ``fooled'' to report a false conductivity measurement?}
\item{} Explain why the addition of table salt to water enhances the conductivity of that solution.
\item{} Explain why conductivity is a {\it non-specific} form of chemical analysis.
\item{} Explain why a measurement of conductance ($G$) is not the same as a measurement of liquid {\it conductivity} ($k$)
\item{} Explain why conductivity ($k$) is measured in units of ``Siemens per centimeter'' rather than just ``Siemens'' as is the case with conductance ($G$).
\item{} Explain why conductivity probes are typically excited by AC rather than DC electric power.
\item{} Suppose the plates in a 2-electrode conductivity cell were separated further apart from each other.  How would this change affect the cell constant ($\theta$)?
\item{} Suppose the plates in a 2-electrode conductivity cell were placed closer together.  How would this change affect the cell constant ($\theta$)?
\item{} Suppose the electrodes in a 2-electrode conductivity cell become fouled over time.  How would this change affect the cell constant ($\theta$)?
\item{} Suppose the excitation current for a 2-electrode conductivity cell were increased.  How would this change affect the measurement of the liquid's conductivity ($k$)?
\item{} Suppose the excitation current for a 2-electrode conductivity cell were decreased.  How would this change affect the measurement of the liquid's conductivity ($k$)?
\end{itemize}


%INDEX% Reading assignment: Lessons In Industrial Instrumentation, Analytical Measurement (two-electrode conductivity probes)

%(END_NOTES)


