%(BEGIN_QUESTION)
% Copyright 2009, Tony R. Kuphaldt, released under the Creative Commons Attribution License (v 1.0)
% This means you may do almost anything with this work of mine, so long as you give me proper credit

Read and outline the ``Four-Electrode Conductivity Probes'' subsection of the ``Conductivity Measurement'' section of the ``Continuous Analytical Measurement'' chapter in your {\it Lessons In Industrial Instrumentation} textbook.  Note the page numbers where important illustrations, photographs, equations, tables, and other relevant details are found.  Prepare to thoughtfully discuss with your instructor and classmates the concepts and examples explored in this reading.

\underbar{file i04125}
%(END_QUESTION)




%(BEGIN_ANSWER)


%(END_ANSWER)





%(BEGIN_NOTES)

A way to escape the problem of conductivity measurement errors resulting from plate fouling is to use a four-wire method of measurement, where two plates carry the excitation current and two other electrodes sense the voltage drop.  This is the same technique used in 4-wire RTD circuits to avoid measurement errors due to wire resistance. 

\vskip 10pt

Some 4-wire conductivity instruments also measure voltage across the current-carrying electrodes, as an indication of probe fouling.  However, so long as the excitation current is not diminished by this fouling, the existence of some fouling will not interfere with the accurate measurement of liquid conductivity.









\vskip 20pt \vbox{\hrule \hbox{\strut \vrule{} {\bf Suggestions for Socratic discussion} \vrule} \hrule}

\begin{itemize}
\item{} {\bf In what ways may a four-wire conductivity probe be ``fooled'' to report a false conductivity measurement?}
\item{} Explain why the ``Kelvin'' method of measuring electrical resistance is popularly used where precision is important, and explain how this general technique is applied to fluid conductivity measurement.
\item{} Explain why 4-electrode conductivity cells are more often used than 2-electrode cells.
\item{} Explain why we need all 4-electrodes in this type of cell.  Why couldn't we just use four wires to connect to a two-electrode cell?
\item{} Identify the purpose of the second voltmeter in a 4-electrode conductivity cell circuit.
\item{} Real conductivity probes are typically energized by AC, not DC.  Explain why this is.
\item{} What would happen to the indication of liquid conductivity if the outer (excitation) electrodes of the 4-wire apparatus shown in the textbook were moved farther apart?
\item{} What would happen to the indication of liquid conductivity if the inner (sense) electrodes of the 4-wire apparatus shown in the textbook were moved farther apart?
\end{itemize}

%INDEX% Reading assignment: Lessons In Industrial Instrumentation, Analytical Measurement (four-electrode conductivity probes)

%(END_NOTES)


