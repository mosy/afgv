%(BEGIN_QUESTION)
% Copyright 2009, Tony R. Kuphaldt, released under the Creative Commons Attribution License (v 1.0)
% This means you may do almost anything with this work of mine, so long as you give me proper credit

Read and outline the ``Electrodeless Conductivity Probes'' subsection of the ``Conductivity Measurement'' section of the ``Continuous Analytical Measurement'' chapter in your {\it Lessons In Industrial Instrumentation} textbook.  Note the page numbers where important illustrations, photographs, equations, tables, and other relevant details are found.  Prepare to thoughtfully discuss with your instructor and classmates the concepts and examples explored in this reading.

\underbar{file i04127}
%(END_QUESTION)




%(BEGIN_ANSWER)


%(END_ANSWER)





%(BEGIN_NOTES)

Electrodeless conductivity probes use electromagnetic induction to measure the conductivity of the liquid solution.  A primary coil is energized by AC, which in turn induces a current through the liquid itself.  This current, in turn, induces an AC voltage in a secondary coil.  The more conductive the liquid solution is, the greater the induced current through the liquid, and therefore the greater the induced voltage in the secondary coil.

\vskip 10pt

These conductivity probes enjoy excellent resistance to fouling since they have no metallic contact with the process liquid.  However, they are generally not sensitive enough to be used in low-conductivity applications such as boiler feedwater measurement.






\vskip 20pt \vbox{\hrule \hbox{\strut \vrule{} {\bf Suggestions for Socratic discussion} \vrule} \hrule}

\begin{itemize}
\item{} {\bf In what ways may an electrodeless conductivity probe be ``fooled'' to report a false conductivity measurement?}
\item{} If you remove an electrodeless conductivity probe from the liquid sample, it does {\it not} act like a simple transformer in dry air.  Explain why this is, since at first glance it would appear that two adjacent inductors should act as a transformer with or without the presence of liquid!
\item{} Identify a practical application where an electrodeless conductivity probe would work well.
\item{} Identify a practical application where an electrodeless conductivity probe would {\it not} work well.
\item{} Devise a ``dry'' test of an electrodeless conductivity probe, to ensure coil integrity.
\item{} Do electrodeless conductivity probes have {\it cell constant} values ($\theta$) like direct-contact probes?  If so, what aspect of the probe's design or construction would change the value of $\theta$?
\item{} What would happen to the indication of liquid conductivity if sensing probe were redesigned such that the number of wire turns in the excitation coil were increased?
\item{} What would happen to the indication of liquid conductivity if sensing probe were redesigned such that the number of wire turns in the sensing coil were increased?
\item{} What would happen to the indication of liquid conductivity if sensing probe were redesigned such that both coils had a larger inner (bore) diameter?
\end{itemize}












\vfil \eject

\noindent
{\bf Summary Quiz}

One of the unique advantages of an ``electrodeless'' conductivity probe is that:

\begin{itemize}
\item{} It is very sensitive, able to detect low levels of conductivity
\vskip 5pt
\item{} It may measure conductivity in ionic as well as covalent solutions
\vskip 5pt
\item{} It uses AC power rather than DC to ``excite'' the probe assembly
\vskip 5pt
\item{} It has a much higher temperature rating than other conductivity probes
\vskip 5pt
\item{} It uses less electrical power than any other type of probe
\vskip 5pt
\item{} It does not suffer from calibration errors due to moderate ``fouling''
\end{itemize}



%INDEX% Reading assignment: Lessons In Industrial Instrumentation, Analytical Measurement (electrodeless conductivity probes)

%(END_NOTES)


