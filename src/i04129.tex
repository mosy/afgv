%(BEGIN_QUESTION)
% Copyright 2009, Tony R. Kuphaldt, released under the Creative Commons Attribution License (v 1.0)
% This means you may do almost anything with this work of mine, so long as you give me proper credit

Read and outline the ``pH'' section of the ``Chemistry'' chapter in your {\it Lessons In Industrial Instrumentation} textbook.  Note the page numbers where important illustrations, photographs, equations, tables, and other relevant details are found.  Prepare to thoughtfully discuss with your instructor and classmates the concepts and examples explored in this reading.

\underbar{file i04129}
%(END_QUESTION)




%(BEGIN_ANSWER)


%(END_ANSWER)





%(BEGIN_NOTES)

pH is a logarithmic scale for expressing the concentration of active hydrogen ions (H$^{+}$) in a liquid solution.  It is mathematically defined as:

$$\hbox{pH} = - \log [\hbox{H}^{+}]$$

\noindent
Where,

pH = pH value of the liquid solution

[H$^{+}$] = Molarity of active hydrogen ions, in moles per liter

\vskip 10pt

Pure water is covalent, and as such experiences little ionization at low temperatures.  The product of hydrogen and hydroxyl molarities in a pure water sample is called the {\it ionization constant} ($K_w$) for water, and it is typically very small.  If the sample of water is pure, the [H$^{+}$] and [OH$^{-}$] molarities will be equal, since there are no other ions in the sample except for those originating from H$_{2}$O.  Thus, a sample of pure water will have a hydrogen ion molarity of $\sqrt{K_w}$.

Since $K_w$ changes with temperature (more ionization as temperature rises), pH of pure water falls as temperature rises.  Pure water has very nearly 7.0 pH at room temperature (25 $^{o}$C).  Pure water is always {\it neutral} with regard to pH, whatever the actual pH value may be, simply because the molarities of [H$^{+}$] and [OH$^{-}$] are equal (balanced).

\vskip 10pt

Any electrolyte compound added to water which generates more H$^{+}$ ions is called an {\it acid}, and it drives the pH value down.  Any electrolyte added to water which generates more OH$^{-}$ ions is called a {\it caustic}, {\it base}, or {\it alkaline}, and it drives the pH value up because the increase of OH$^{-}$ ions causes some of the H$^{+}$ ions to bond with them to form H$_{2}$O, thus reducing the number of active H$^{+}$ ions in solution.

Any electrolyte compound generating neither H$^{+}$ or OH$^{-}$ ions is called a {\it salt}.  When acids and caustics are mixed together, their respective ions bond to form deionized water and some kind of salt.  This is called {\it pH neutralization}, and it is an exothermic process because the formation of stable water and salt molecules represents a decrease in molecular energy states, necessitating a release of energy in their formation.







\vskip 20pt \vbox{\hrule \hbox{\strut \vrule{} {\bf Suggestions for Socratic discussion} \vrule} \hrule}

\begin{itemize}
\item{} Explain why $K_w$ increases with temperature, based on what you know about the nature of temperature.
\item{} Explain why it is a fallacy that pure water always has a pH value of 7.
\item{} Explain what the addition of an {\it acidic} electrolyte does to the molarities of H$^{+}$ and OH$^{-}$ ions in an aqueous solution.
\item{} Explain what the addition of a {\it caustic} or {\it alkaline} electrolyte does to the molarities of H$^{+}$ and OH$^{-}$ ions in an aqueous solution.
\item{} Identify what makes an acid (or a caustic) either ``weak'' or ``strong''.
\item{} Describe the balance-scale analogy for hydrogen and hydroxyl ion molarities in an aqueous solution shown in the textbook.  What necessarily happens when one of these quantities increases?
\item{} Explain how the addition of an alkaline substance to a sample of water causes the pH to rise.  Specifically, how does an alkaline electrolyte influence the number of H$^{+}$ ions in solution when it adds no H$^{+}$ ions to the solution?
\item{} Explain why [H$^{+}$] = $\sqrt{K_w}$ for a pure water sample.
\item{} Is [H$^{+}$] always equal to $\sqrt{K_w}$ for {\it any} liquid sample?  Why or why not?
\item{} Choose some of the entries in the Temperature/$K_w$/pH table shown in the textbook and calculate pH from $K_w$.
\end{itemize}









\vfil \eject

\noindent
{\bf Prep Quiz}

pH is defined as being:

\begin{itemize}
\item{} The logarithm of sodium ion concentration in a liquid solution
\vskip 5pt
\item{} The ionization percentage of a pure water sample
\vskip 5pt
\item{} The degree of hydrogen ion activity in a liquid solution
\vskip 5pt
\item{} The concentration of salt ions in a liquid solution
\vskip 5pt
\item{} The amount of electrical charge carried by each ion in a solution
\end{itemize}


\vfil \eject

\noindent
{\bf Prep Quiz}

A {\it caustic} substance is one:

\begin{itemize}
\item{} with a pH value equal to 7 
\vskip 5pt
\item{} with a pH value less than 7 
\vskip 5pt
\item{} with a pH value equal to 0 
\vskip 5pt
\item{} with a pH value equal to 14 
\vskip 5pt
\item{} with a pH value greater than 7
\end{itemize}


\vfil \eject

\noindent
{\bf Prep Quiz}

An {\it acid} substance is one:

\begin{itemize}
\item{} with a pH value equal to 7 
\vskip 5pt
\item{} with a pH value less than 7 
\vskip 5pt
\item{} with a pH value equal to 0 
\vskip 5pt
\item{} with a pH value equal to 14 
\vskip 5pt
\item{} with a pH value greater than 7
\end{itemize}


%INDEX% Reading assignment: Lessons In Industrial Instrumentation, Chemistry (pH)

%(END_NOTES)


