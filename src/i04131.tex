
%(BEGIN_QUESTION)
% Copyright 2009, Tony R. Kuphaldt, released under the Creative Commons Attribution License (v 1.0)
% This means you may do almost anything with this work of mine, so long as you give me proper credit

Suppose two chemistry students decide to engage in a ``pH battle,'' where one tries to make a bucket of water more acidic by adding hydrochloric acid (HCl), while the other tries to make the same water more alkaline by adding sodium hydroxide (NaOH).

\vskip 10pt

Identify the reaction products resulting from the mixing of these two substances in water, and why you would be wise to stay clear of this experiment.

\vskip 20pt \vbox{\hrule \hbox{\strut \vrule{} {\bf Suggestions for Socratic discussion} \vrule} \hrule}

\begin{itemize}
\item{} What conditions would make this neutralization safer to execute?  What conditions would make it more dangerous?
\item{} Would you expect a pH neutralization process to be {\it endothermic} or {\it exothermic}?  Explain your reasoning.
\item{} Identify the reaction products from mixing calcium hydroxide with sulfuric acid in a similar ``pH battle''.
\item{} Identify the reaction products from mixing hydrofluoric acid with magnesium hydroxide in a similar ``pH battle''.
\end{itemize}

\underbar{file i04131}
%(END_QUESTION)





%(BEGIN_ANSWER)


%(END_ANSWER)





%(BEGIN_NOTES)

The reaction products are salt (NaCl -- regular table salt!) and more water molecules, as well as a possible trip to the local hospital's emergency room due to the hazards of the reactants and the heat released by this particular reaction.

\vskip 10pt

pH neutralization is {\it exothermic}, owing to the formation of stable (low-energy) salt and water molecules.  Any time we form low-energy molecules, it means energy was released.  This is why the neutralization of strong acids and caustics tends to liberate a lot of heat, and can be dangerous!

\vskip 10pt

Other ``pH battle'' reaction products:

\begin{itemize}
\item{} Calcium hydroxide + sulfuric acid $\to$ calcium sulfate (gypsum) + water
\vskip 5pt
\item{} Hydrofluoric acid + magnesium hydroxide $\to$ Magnesium fluoride + water
\end{itemize}


%INDEX% Chemistry, pH: neutralization

%(END_NOTES)


