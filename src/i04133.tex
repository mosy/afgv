%(BEGIN_QUESTION)
% Copyright 2009, Tony R. Kuphaldt, released under the Creative Commons Attribution License (v 1.0)
% This means you may do almost anything with this work of mine, so long as you give me proper credit

Research how to use {\it litmus paper} to measure pH, and prepare to measure the pH values of several different solutions in class.  Plan with some of your classmates to bring some of these substances to class:

\begin{itemize}
\item{} Small drinking cup (no more than one per person necessary)
\vskip 5pt
\item{} Carbonated beverage
\vskip 5pt
\item{} Milk
\vskip 5pt
\item{} Orange juice
\vskip 5pt
\item{} Cranberry juice
\vskip 5pt
\item{} Bottled water
\vskip 5pt
\item{} Water from home
\vskip 5pt
\item{} Water from a public drinking fountain at school
\vskip 5pt
\item{} Antacid pills
\vskip 5pt
\item{} Drain cleaner powder (do not let the powder contact your skin!)
\vskip 5pt
\item{} Vinegar
\end{itemize}

\vskip 10pt

The instructor will take several cups and partially fill them with the liquid samples you bring to class.  Your team's job is to use strips of litmus paper (also supplied by the instructor) to rank each sample in order from {\it highest pH} to {\it lowest pH}.  In order to minimize the ``traffic congestion'' of students at the cups, it is recommended that each team member quickly dip one litmus paper strip into one solution (each of their teammates dipping their strips into different solution cups) and return to their desks for analysis of the paper and to rank the solutions.  When your team has agreed on a ranking, write your ranked results (from greatest to least) on a piece of paper and prepare to present your findings to the class. 

\vskip 20pt \vbox{\hrule \hbox{\strut \vrule{} {\bf Suggestions for Socratic discussion} \vrule} \hrule}

\begin{itemize}
\item{} Identify some of the disadvantages of measuring pH with litmus paper.
\item{} Devise a way in which this colorimetric method of pH sensing could be automated, to provide a 4-20 mA electronic signal representing pH based on the color change of a litmus strip.
\end{itemize}

\underbar{file i04133}
%(END_QUESTION)





%(BEGIN_ANSWER)


%(END_ANSWER)





%(BEGIN_NOTES)

%INDEX% Chemistry, pH: using litmus paper

%(END_NOTES)


