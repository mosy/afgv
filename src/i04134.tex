%(BEGIN_QUESTION)
% Copyright 2009, Tony R. Kuphaldt, released under the Creative Commons Attribution License (v 1.0)
% This means you may do almost anything with this work of mine, so long as you give me proper credit

Read and outline the ``Colorimetric pH Measurement'' subsection of the ``pH Measurement'' section of the ``Continuous Analytical Measurement'' chapter in your {\it Lessons In Industrial Instrumentation} textbook.  Note the page numbers where important illustrations, photographs, equations, tables, and other relevant details are found.  Prepare to thoughtfully discuss with your instructor and classmates the concepts and examples explored in this reading.

\underbar{file i04134}
%(END_QUESTION)




%(BEGIN_ANSWER)


%(END_ANSWER)





%(BEGIN_NOTES)

Some chemical compounds change color with pH, and as such are useful visual indicators of pH.  This is how {\it litmus strips} work, and why hydrangea plants change color with soil pH.

\vskip 10pt

Chemical compounds found in red cabbage also serve as colorimetric pH indicators.  The pigment {\it flavin} in red cabbage is naturally violet in color, but changes to red when pH decreases and changes to green when pH increases.




\vskip 20pt \vbox{\hrule \hbox{\strut \vrule{} {\bf Suggestions for Socratic discussion} \vrule} \hrule}

\begin{itemize}
\item{} {\bf In what ways may a colorimetric pH instrument be ``fooled'' to report a false pH measurement?}
\item{} Is the hydrangea plant pictured in the book located in acidic soil, or alkaline soil?
\item{} A common gardening trick for altering the color of a hydrangea plant is to place old alkaline batteries in the soil near the plant's roots.  What color will this cause the hydrangea to change to?
\item{} Describe how to make your own litmus paper from substances found in a household kitchen.
\item{} Suppose you were tasked with designing a pH-measuring instrument based on color, which operators at a water treatment plant would use to test the pH of samples taken from the filtered drinking water.  Since color can be a subjective variable to different peoples' eyesight, how could you minimize error in this instrument and achieve the best pH measurement possible despite the limitations of human sight?
\end{itemize}

%INDEX% Reading assignment: Lessons In Industrial Instrumentation, Analytical (colorimetric pH)

%(END_NOTES)


