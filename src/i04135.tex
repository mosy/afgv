
%(BEGIN_QUESTION)
% Copyright 2009, Tony R. Kuphaldt, released under the Creative Commons Attribution License (v 1.0)
% This means you may do almost anything with this work of mine, so long as you give me proper credit

Calculate the pH values of the following solutions, given the hydrogen ion concentrations shown.  Based on the pH values you obtain, determine whether each solution is {\it acidic}, {\it alkaline}, or {\it neutral}:

\begin{itemize}
\item{} [H$^{+}$] = 0.0007 $M$ ; pH = \underbar{\hskip 50pt}
\vskip 5pt
\item{} [H$^{+}$] = 0.0000032 $M$ ; pH = \underbar{\hskip 50pt}
\vskip 5pt
\item{} [H$^{+}$] = 0.000000085 $M$ ; pH = \underbar{\hskip 50pt}
\vskip 5pt
\item{} [H$^{+}$] = 0.00000000012 $M$ ; pH = \underbar{\hskip 50pt}
\end{itemize}

\vskip 20pt \vbox{\hrule \hbox{\strut \vrule{} {\bf Suggestions for Socratic discussion} \vrule} \hrule}

\begin{itemize}
\item{} Which of these solutions has the greatest concentration of hydrogen ions, and which of these solutions has the least?
\item{} Demonstrate how to {\it estimate} numerical answers for this problem without using a calculator.
\item{} As hydrogen ion concentration increases in a solution, does the solution become more acidic or more alkaline?
\item{} Does a sample of water become more acidic, more alkaline, or does its pH value remain stable as it is {\it electrolyzed} to produce hydrogen and oxygen gases?
\item{} Explain why we cannot calculate the ionization constant ($K_w$) of the solution from just this information given.
\end{itemize}

\underbar{file i04135}
%(END_QUESTION)





%(BEGIN_ANSWER)

\noindent
{\bf Partial answer:}

\begin{itemize}
\item{} [H$^{+}$] = 0.0000032 $M$ ; pH = \underbar{5.49 pH}
\vskip 5pt
\item{} [H$^{+}$] = 0.00000000012 $M$ ; pH = \underbar{9.92 pH}
\end{itemize}

%(END_ANSWER)





%(BEGIN_NOTES)

$$\hbox{pH} = - \log [\hbox{H}^{+}]$$


\begin{itemize}
\item{} [H$^{+}$] = 0.0007 $M$ ; pH = \underbar{3.15 pH} {\it (acidic)}
\vskip 5pt
\item{} [H$^{+}$] = 0.0000032 $M$ ; pH = \underbar{5.49 pH} {\it (acidic)}
\vskip 5pt
\item{} [H$^{+}$] = 0.000000085 $M$ ; pH = \underbar{7.07 pH} {\it (almost neutral)}
\vskip 5pt
\item{} [H$^{+}$] = 0.00000000012 $M$ ; pH = \underbar{9.92 pH} {\it (alkaline)}
\end{itemize}








\vfil \eject

\noindent
{\bf Summary Quiz}

Calculate the pH of a liquid, given a hydrogen ion molarity of 0.000044 moles per liter ($4.4 \times 10^{-5}$ $M$):

\begin{itemize}
\item{} 7.73 pH
\vskip 5pt
\item{} -3.36 pH
\vskip 5pt
\item{} 3.39 pH
\vskip 5pt
\item{} 10.03 pH
\vskip 5pt
\item{} 4.36 pH
\end{itemize}


%INDEX% Chemistry, pH: molarity calculation

%(END_NOTES)


