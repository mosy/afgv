%(BEGIN_QUESTION)
% Copyright 2009, Tony R. Kuphaldt, released under the Creative Commons Attribution License (v 1.0)
% This means you may do almost anything with this work of mine, so long as you give me proper credit

Read the ``Sodium hydroxide'' (caustic soda) entry in the {\it NIOSH Pocket Guide To Chemical Hazards} (DHHS publication number 2005-149) and answer the following questions:

\vskip 10pt

Write the chemical formula for sodium hydroxide, and identify its constituent elements.  Is this considered an acid, a base, or a salt?

\vskip 10pt

Calculate the molecular weight for this compound, and also the mass that 13 moles of pure sodium hydroxide will have.

\vskip 10pt

Interpret the respirator recommendations given for sodium hydroxide in the NIOSH guide.  Is a filter-style respirator sufficient, or must a person work with a ``supplied air'' apparatus to ensure they do not inhale sodium hydroxide?

\vskip 20pt \vbox{\hrule \hbox{\strut \vrule{} {\bf Suggestions for Socratic discussion} \vrule} \hrule}

\begin{itemize}
\item{} Explain {\it why} a filter-style respirator is or is not sufficient protection against inhalation of this chemical substance, based on what you know about its physical properties.
\item{} Identify why sodium hydroxide powder might be added to drinking water in a water purification process.
\end{itemize}

\underbar{file i04138}
%(END_QUESTION)





%(BEGIN_ANSWER)


%(END_ANSWER)





%(BEGIN_NOTES)

NaOH: sodium, oxygen, and hydrogen.  NaOH is considered a {\it base} because it creates hydroxyl ions (OH$^{-}$) when added to aqueous solutions.  This is also why NaOH is often referred to as {\it caustic soda}.

\vskip 10pt

Molecular weight = 40 (23 + 16 + 1).  13 moles of NaOH will have a mass of 520 grams (0.520 kg).

\vskip 10pt

Sodium hydroxide is an irritant to the human body, and it also reacts exothermically when mixed with water.  The filtration requirements for NaOH are listed as:

\begin{itemize}
\item{} Sa:Cf ({\it Supplied-air with continuous flow})
\item{} 100F ({\it Full-face filter respirator})
\item{} PaprHie ({\it Powered respirator with high-efficiency particulate filter})
\item{} ScbaF ({\it Self-contained breathing apparatus with full-face mask})
\item{} SaF ({\it Supplied-air with full-face mask})
\end{itemize}

Note: the \S{} symbol refers to ``Emergency or planned entry into unknown concentrations or IDLH conditions'' according to page {\tt xx} of the NIOSH Pocket Guide.  The British ``pounds'' symbol refers to the need for eye protection due to the substance in question being an eye irritant or hazard.

%INDEX% Reading assignment: NIOSH Chemical Hazard Guide (sodium hydroxide)

%(END_NOTES)


