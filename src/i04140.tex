%(BEGIN_QUESTION)
% Copyright 2009, Tony R. Kuphaldt, released under the Creative Commons Attribution License (v 1.0)
% This means you may do almost anything with this work of mine, so long as you give me proper credit

Read and outline the ``Potentiometric pH Measurement'' subsection of the ``pH Measurement'' section of the ``Continuous Analytical Measurement'' chapter in your {\it Lessons In Industrial Instrumentation} textbook.  Note the page numbers where important illustrations, photographs, equations, tables, and other relevant details are found.  Prepare to thoughtfully discuss with your instructor and classmates the concepts and examples explored in this reading.

\underbar{file i04140}
%(END_QUESTION)




%(BEGIN_ANSWER)


%(END_ANSWER)





%(BEGIN_NOTES)

A difference of ionic concentration across an ion-permeable membrane will generate a voltage as predicted by the Nernst equation:

$$V = {{R T} \over {nF}} \ln \left({C_1 \over C_2}\right) \hskip 50pt V = {{2.303 R T} \over {nF}} \log \left({C_1 \over C_2}\right)$$

\noindent
Where,

$V$ = Voltage produced across membrane due to ion exchange, in volts (V)

$R$ = Universal gas constant (8.315 J/mol$\cdot$K)

$T$ = Absolute temperature, in Kelvin (K)

$n$ = Number of electrons transferred per ion exchanged (unitless)

$F$ = Faraday constant, in coulombs per mole (96485 C/mol e$^{-}$)

$C_1$ = Concentration of ion in measured solution, in moles per liter of solution ($M$)

$C_2$ = Concentration of ion in reference solution (on other side of membrane), in moles per liter of solution ($M$)

\vskip 10pt

In the case of pH measurement, the membrane is a specially formulated glass bulb designed to be selectively permeable to hydrogen ions (so that the Nernst voltage may be strictly a function of [H$^{+}$] and not of other ions in the solution, filled in the inside with a buffer solution having a stable pH value (typically 7.0 pH).  If the solution outside the bulb has any pH value different from 7.0, the glass will develop a Nernst potential which may be measured using a high-impedance voltmeter.

Knowing this, we may re-write the Nernst equation in such a way that the pH of the solution is directly compared against 7 to create the Nernst potential:

$$V = {{2.303 R T} \over {nF}} \left(7 - \hbox{pH}_1 \right)$$

In order to create a second point of contact with the solution (to form a complete circuit), a second electrode must be inserted into the solution -- a ``reference'' electrode.  Its sole purpose is to create a potential-free point of contact with the liquid.  pH electrodes are often manufactured in {\it combination} form where both the measurement and reference electrodes are part of a single probe unit.

pH probe assemblies may also have ``solution ground'' electrodes and RTD sensors for solution temperature measurement.  Temperature is an important variable because it affects how much Nernst potential will be generated for any solution pH (the $T$ in the Nernst equation).

Glass pH electrodes {\it must} be kept wet at all times, or else they will fail very quickly.  When maintained in a wet state, however, they do wear normally as layers of glass slowly dissolve into the liquid solution.  For this reason, pH probes have a limited shelf and service life.

\vskip 10pt

The voltage produced by the glass electrode is approximately 59 mV per pH unit (away from 7) at room temperature.  Every additional 1 pH unit away from 7 generates another 59 mV (approximately).  Measurement of this voltage is problematic because the glass electrode has such a high impedance.  A voltmeter with exceptionally high input impedance ({\it trillions} of ohms!) must be used.  The high Th\'evenin equivalent resistance of the pH electrode also poses time-delay problems when coupled with the stray capacitance of the cable connecting the pH electrode to the pH transmitter.

Preamplifiers may be used as voltage-followers to boost the current-sourcing ability of a pH probe.  This may take the form of a separate preamplifier unit, or the probe itself may be constructed with a built-in preamplifier.

\vskip 10pt

Temperature affects the amount of voltage generated by an ion-permeable membrane, and as such many pH probes are equipped with an RTD sensor to give the pH instrument the ability to compensate for temperature.  The amount of Nernst potential generated per unit pH is called the {\it slope} of the pH electrode (ideally 59 mV per pH unit at room temperature).  As a probe ages, its slope tends to decrease.  This constitutes a span shift, which is why pH instruments must be routinely calibrated.

Stray millivoltages in the pH electrode circuit will cause a zero shift, and must also be corrected with routine calibration procedures.  The pH value registered by a calibrated instrument with its input shorted (i.e. zero millivolts input) is called the {\it isopotential point}.

\vskip 10pt

Chemical {\it buffer} solutions are used as standards for calibrating pH probes.  Buffers are specially formulated to resist pH changes with slight contamination levels, ensuring a stable pH reference for field calibration.

\vskip 10pt

pH probes must be kept in a clean state in order to properly function.  The fragility of glass pH probes precludes any form of abrasive cleaning technique.  No solid physical contact should ever be made to the glass membrane of a pH probe!  Instead, specific liquid solvents may be sprayed onto the tip of a pH probe to dislodge and dissolve fouling.  Sometimes deionized (distilled) water is sufficient for the task, but for some cases one must choose a different solvent (e.g. methanol for fats, acidic pepsin for proteins, thiourea for sulfides).





\vskip 20pt \vbox{\hrule \hbox{\strut \vrule{} {\bf Suggestions for Socratic discussion} \vrule} \hrule}

\begin{itemize}
\item{} {\bf In what ways may a potentiometric pH instrument be ``fooled'' to report a false pH measurement?}
\item{} Explain the effect of changing the membrane material in a {\it concentration cell}.
\item{} Explain where the voltage is generated in a pH probe assembly.
\item{} Describe what is special about the glass that a pH measuring electrode is made of.
\item{} Explain what would happen if one tried to make a pH measuring electrode out of any material other than this specially-formulated glass.
\item{} Explain why pH electrodes must be kept in a wet state at all times.
\item{} Explain why pH electrodes wear out over time, even if impeccably maintained.
\item{} Explain what a ``combination'' pH electrode is, and how it differs in construction from traditional pH electrodes.
\item{} Identify some different styles of installation for pH electrodes.
\item{} Explain how the pH scale is analagous to the Richter scale used to quantity earthquake strength.
\item{} Describe some of the practical measurement problems posed by the extremely high electrical resistance of the glass electrode.
\item{} Why should pH electrode cables be kept as short as practically possible?
\item{} Why is the Nernst potential 0 mV at a pH value of 7.0?
\item{} Suppose a pH probe preamplifier had a gain other than 1.  Would this create a {\it zero} error, a {\it span} error, a {\it linearity} error, or a {\it hysteresis} error?
\item{} Does it matter whether a preamplifier is installed near the pH probe or near the pH-measuring instrument?  Explain why or why not.
\item{} Explain why temperature compensation is often used with pH electrodes.
\item{} Does glass membrane temperature have a {\it zero} or a {\it span} effect on pH measurement?
\item{} Explain what ``slope'' means for a pH electrode.
\item{} Describe what is unique about a pH {\it buffer} solution.
\item{} Explain why routine calibration is necessary for pH instruments -- specifically, what tends to drift in a pH probe?
\item{} Suppose you disconnected the pH electrode assembly from the input of a pH transmitter and shorted those input terminals together.  What should that transmitter register for pH, assuming it had been well-calibrated on a healthy pH probe?
\end{itemize}






\vfil \eject

\noindent
{\bf Prep Quiz}

A glass pH electrode senses the pH of a liquid solution by:

\begin{itemize}
\item{} Detecting the electrical resistance of the liquid between two points
\vskip 5pt
\item{} Generating a DC voltage proportional to the solution's pH value
\vskip 5pt
\item{} Sensing the frequency of noise voltage present in the solution
\vskip 5pt
\item{} Generating a DC current proportional to the solution's pH value
\vskip 5pt
\item{} Providing an electrical resistance that changes with temperature
\vskip 5pt
\item{} Changing color when it contacts a liquid having a certain pH value
\end{itemize}



\vfil \eject

\noindent
{\bf Prep Quiz}

A change in pH from 8 to 9 represents how large of a change in hydrogen ion concentration?

\begin{itemize}
\item{} A change of 88.9\% ($8 \over 9$)
\vskip 5pt
\item{} A nine-fold change ($9 \times$)
\vskip 5pt
\item{} A two-fold change ($2 \times$)
\vskip 5pt
\item{} A ten-fold change ($10 \times$)
\vskip 5pt
\item{} An eight-fold change ($8 \times$)
\vskip 5pt
\item{} A change of 112.5\% ($9 \over 8$) 
\end{itemize}




\vfil \eject

\noindent
{\bf Summary Quiz}

The input impedance of a pH transmitter must be extremely large because:

\begin{itemize}
\item{} The voltage generated by the probe is so small and prone to interference
\vskip 5pt
\item{} Capacitance in the cable stores an electrical charge which must be drained
\vskip 5pt
\item{} A low-resistance path would present an electrical shock hazard to personnel
\vskip 5pt
\item{} Many industrial solutions have low conductivity and therefore high resistance
\vskip 5pt
\item{} The pH probe is made of glass, and therefore has a large resistance itself
\vskip 5pt
\item{} The manufacturer can charge you more for a pH transmitter
\end{itemize}


%INDEX% Reading assignment: Lessons In Industrial Instrumentation, Analytical (potentiometric pH)

%(END_NOTES)


