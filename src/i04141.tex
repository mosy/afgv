%(BEGIN_QUESTION)
% Copyright 2009, Tony R. Kuphaldt, released under the Creative Commons Attribution License (v 1.0)
% This means you may do almost anything with this work of mine, so long as you give me proper credit

Read and outline the ``Normal Energization States'' subsection of the ``Solenoid valves'' section of the ``Discrete Control Elements'' chapter in your {\it Lessons In Industrial Instrumentation} textbook.  Note the page numbers where important illustrations, photographs, equations, tables, and other relevant details are found.  Prepare to thoughtfully discuss with your instructor and classmates the concepts and examples explored in this reading.

\underbar{file i04141}
%(END_QUESTION)




%(BEGIN_ANSWER)


%(END_ANSWER)





%(BEGIN_NOTES)

Solenoid valves may spend most of their operating lives de-energized, or energized, according to the whim of the system's designer.  This is called the ``normal'' energization state.  Unfortunately, this definition of ``normal'' is completely different from the definition of ``normal'' used with switches and with valve bodies themselves.  For switches and valve bodies, the ``normal'' state is the {\it resting} state (when it receives no actuating stimulus).  However, for solenoid coils, the ``normal'' state is the {\it typical} operating state of the process. 

\vskip 10pt

If a solenoid valve is typically energized in the process (NE), its ``normal'' energization state will be opposite of its ``normal'' valve state.  If a solenoid valve is typically de-energized in the process (NDE), its ``normal'' energization state will be the same as its ``normal'' valve state.

\vskip 10pt

In the turbine trip example, these normally-energized (NE) solenoid valves go to their resting states when the system is in an abnormal condition.  Therefore ``normal'' has two opposite meanings for these valves.

\vskip 10pt

In the air vent door example, the normally-deenergized (NDE) solenoid valve will be in its resting state during typical (``normal'') process conditions.  Therefore ``normal'' has the same meaning for both the solenoid coil and the valve body.





\vskip 20pt \vbox{\hrule \hbox{\strut \vrule{} {\bf Suggestions for Socratic discussion} \vrule} \hrule}

\begin{itemize}
\item{} Identify the example shown in the text where the ``normal'' energization state of a solenoid valve is the same as its ``normal'' valve-body state.
\item{} Identify the example shown in the text where the ``normal'' energization state of a solenoid valve is different from its ``normal'' valve-body state.
\item{} Identify the error in the turbine trip system P\&ID, and explain what would happen if the valve actuator were actually tubed like that.
\item{} In the turbine trip system shown in the book, are the solenoids arranged for a 1oo2 trip or 2oo2 trip?
\item{} Identify the proper MooN notation for the emergency air vent door control system shown in the book.
\end{itemize}

%INDEX% Reading assignment: Lessons In Industrial Instrumentation, Solenoid Valves (normal energization states)

%(END_NOTES)


