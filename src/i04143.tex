
%(BEGIN_QUESTION)
% Copyright 2009, Tony R. Kuphaldt, released under the Creative Commons Attribution License (v 1.0)
% This means you may do almost anything with this work of mine, so long as you give me proper credit

Two instrument technicians are arguing over the suitability of a magnetic flowmeter (``magflow'') in a process line carrying highly concentrated sodium hydroxide (caustic soda).  One technician claims a magflow meter will not work because it requires a lot of ions in the liquid for conductivity, and as we all know caustic solutions have very low hydrogen ion concentrations.  The other technician does not agree.

\vskip 10pt

Which technician is correct, and why?

\vskip 20pt \vbox{\hrule \hbox{\strut \vrule{} {\bf Suggestions for Socratic discussion} \vrule} \hrule}

\begin{itemize}
\item{} If the pH value of an industrial flow stream is liable to change over time, will this affect the suitability of a magflow meter?
\item{} One of the challenges posed by the measurement of some caustic solution flow rates is clogging of the flowmeter by crystals forming in the caustic solution.  Identify specific flowmeter technologies that would be affected more by plugging than others.
\end{itemize}

\underbar{file i04143}
%(END_QUESTION)





%(BEGIN_ANSWER)


%(END_ANSWER)





%(BEGIN_NOTES)

The second technician is correct: even though the H$^{+}$ ion concentration in a caustic solution is low, there are plenty of other ions present (most notably OH$^{-}$ ions in {\it any} caustic solution!).  For sodium hydroxide, there will also be plenty of sodium ions in the solution.  Suffice to say, a magflow meter will work just fine in (hydrated) caustic soda service.

%INDEX% Chemistry, pH: ionic concentration
%INDEX% Measurement, flow: magnetic

%(END_NOTES)


