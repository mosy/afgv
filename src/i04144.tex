%(BEGIN_QUESTION)
% Copyright 2010, Tony R. Kuphaldt, released under the Creative Commons Attribution License (v 1.0)
% This means you may do almost anything with this work of mine, so long as you give me proper credit

One of the technical challenges inherent to potentiometric pH measurement is how to measure a small voltage when the measuring circuit contains an enormously large resistance (the glass bulb of the pH-sensing electrode).  To better understand this electrical principle, we may perform a simple experiment using a voltmeter to measure the voltage of a battery through resistors of various size.

\vskip 10pt

Bring the following materials to class:

\begin{itemize}
\item{} Assortment of resistors from 1 k$\Omega$ to 10 M$\Omega$
\vskip 5pt
\item{} Multimeter 
\vskip 5pt
\item{} At least two ``alligator clip'' jumper wires
\vskip 5pt
\item{} Small battery (any voltage)
\end{itemize}

Use your voltmeter to measure the battery's voltage, inserting different amounts of resistance in series with your meter as you do.  Your first voltage measurement will be with no resistor, connecting the meter's test leads directly to the battery terminals.  This is the ``true'' voltage measurement against which all the others will be compared:

% No blank lines allowed between lines of an \halign structure!
% I use comments (%) instead, so that TeX doesn't choke.

$$\vbox{\offinterlineskip
\halign{\strut
\vrule \quad\hfil # \ \hfil & 
\vrule \quad\hfil # \ \hfil \vrule \cr
\noalign{\hrule}
%
% First row
{\bf Resistor value} & {\bf Measured battery voltage} \cr
%
\noalign{\hrule}
%
% Another row
0 $\Omega$ (no resistor)  &  \cr
%
\noalign{\hrule}
%
% Another row
 &  \cr
%
\noalign{\hrule}
%
% Another row
100 k$\Omega$ &  \cr
%
\noalign{\hrule}
%
% Another row
 &  \cr
%
\noalign{\hrule}
%
% Another row
10 M$\Omega$ &  \cr
%
\noalign{\hrule}
} % End of \halign 
}$$ % End of \vbox

\vskip 10pt

Identify the effect increased resistance has on the voltmeter's indication.

\vskip 10pt

Explain how the results of this experiment demonstrate the voltage-measurement problem inherent to glass pH electrodes.  Also, use your experimental data to determine how much larger than the circuit impedance the voltmeter's internal impedance must be in order to obtain a reasonably accurate voltage measurement.  Note: in order to calculate a ratio, you must know the approximate input impedance of your voltmeter.  Assuming a glass measurement electrode impedance of 300 M$\Omega$, how high must the input impedance of the pH transmitter be?

\vskip 20pt \vbox{\hrule \hbox{\strut \vrule{} {\bf Suggestions for Socratic discussion} \vrule} \hrule}

\begin{itemize}
\item{} How do modern pH measuring instruments reliably measure the millivoltage produced by pH probes, given this problem of glass electrode resistance?
\item{} Explain how the experimental results also confirm the fact that magnetic flowmeters are not greatly affected by changes in liquid conductivity, so long as the liquid's conductivity value does not fall below a certain limit rated by the manufacturer.
\item{} Explain how the traditional ``null-balance'' laboratory technique may be used to measure signal voltages originating from extremely high-resistance sources with negligible error.
\end{itemize}

\underbar{file i04144}
%(END_QUESTION)





%(BEGIN_ANSWER)


%(END_ANSWER)





%(BEGIN_NOTES)

%INDEX% Measurement, analytical: pH

%(END_NOTES)


