
%(BEGIN_QUESTION)
% Copyright 2009, Tony R. Kuphaldt, released under the Creative Commons Attribution License (v 1.0)
% This means you may do almost anything with this work of mine, so long as you give me proper credit

Read relevant portions of the ``Rosemount Model 389 Combination pH/ORP Sensor'' instruction manual (publication PN 51-389, revision 1), and answer the following questions:

\vskip 10pt

Identify the page in this manual describing the buffer calibration procedure for the pH probe.

\vskip 10pt

Identify the calibrated slope values indicating a pH probe in need of replacement.

\vskip 10pt

This manual describes a ``standardization'' procedure in addition to a full ``calibration'' procedure.  Explain the difference between the two procedures.

\vskip 20pt \vbox{\hrule \hbox{\strut \vrule{} {\bf Suggestions for Socratic discussion} \vrule} \hrule}

\begin{itemize}
\item{} Explain the rationale for each step of the two-buffer calibration procedure.
\item{} During a two-buffer calibration procedure, which calibration parameter(s) get adjusted within the pH instrument: {\it zero}, {\it span}, or {\it both}?
\item{} During a standardization procedure, which calibration parameter(s) get adjusted within the pH instrument: {\it zero}, {\it span}, or {\it both}?
\item{} When a technician performs a ``standardization'' on a pH instrument, as opposed to a full calibration, is the technician adjusting the instrument's {\it zero}, its {\it span}, its {\it linearity}, or its {\it hysteresis}? 
\item{} The instructions on page 19 of this manual tell you to ``. . . check the pH buffer manufacturer specifications for millivolt values at various temperatures since it may affect the actual value of the buffer solution mV/pH value.''  This warning is actually referring to \underbar{two} different phenomena: the effect of temperature on the probe's slope, and the effect of temperature on the buffer solution's pH value.  If the pH probe being calibrated is equipped with an RTD temperature sensor, which of these two phenomena is accounted for?
\end{itemize}

\underbar{file i04146}
%(END_QUESTION)





%(BEGIN_ANSWER)


%(END_ANSWER)





%(BEGIN_NOTES)

Calibration instructions are on page 19.

\vskip 10pt

Slope values in the 47 to 49 mV per pH range indicate the need for replacement (page 19).

\vskip 10pt

``Standardization'' is nothing more than a {\it zero shift} to make the transmitter agree with a ``grab sample'' measured by another pH instrument.  Full calibration sets both the zero and span of the pH instrument (page 19).

%INDEX% Reading assignment: Rosemount model 389 pH/ORP probe Reference manual

%(END_NOTES)


