
%(BEGIN_QUESTION)
% Copyright 2009, Tony R. Kuphaldt, released under the Creative Commons Attribution License (v 1.0)
% This means you may do almost anything with this work of mine, so long as you give me proper credit

Suppose a pH probe is immersed in a solution of 98\% water having a hydrogen ion activity of 0.0000032 $M$.  Calculate the pH value of this solution, and also the (ideal) voltage generated by a pH probe assuming a probe resistance of 125 M$\Omega$ and a solution temperature of 30 degrees Celsius.

\vskip 20pt \vbox{\hrule \hbox{\strut \vrule{} {\bf Suggestions for Socratic discussion} \vrule} \hrule}

\begin{itemize}
\item{} Identify factors influencing the voltage output by a glass electrode, other than the pH of the liquid it's immersed in.
\item{} Explain why you chose the value you did for $n$ in the Nernst equation.
\item{} Demonstrate how to {\it estimate} numerical answers for this problem without using a calculator.
\end{itemize}

\underbar{file i04147}
%(END_QUESTION)





%(BEGIN_ANSWER)

\noindent
{\bf Partial answer:}

\vskip 10pt

$V$ = 90.56 mV

%(END_ANSWER)





%(BEGIN_NOTES)

$$V = {{2.303 R T} \over {nF}} \log \left({C_1 \over C_2}\right)$$

$$V = {{(2.303) (8.315) (30 + 273.15)} \over {(1)(96485)}} \log \left({0.0000032 \over 1 \times 10^{-7}}\right)$$

$$V = 90.559 \hbox{ mV}$$

%$$V = {{2.303 R T} \over {nF}} \left(7 - \hbox{pH}_1 \right)$$

\vskip 10pt

$$\hbox{pH} = - \log[\hbox{H}^{+}]$$

$$\hbox{pH} = - \log 0.0000032$$

$$\hbox{pH} = 5.495 \hbox{ pH}$$

\vskip 10pt

The 98\% and 125 M$\Omega$ figures are extraneous information, included for the purpose of challenging students to identify whether or not information is relevant to solving a particular problem.

%INDEX% Chemistry, pH: molarity and Nernst voltage calculation

%(END_NOTES)


