
%(BEGIN_QUESTION)
% Copyright 2009, Tony R. Kuphaldt, released under the Creative Commons Attribution License (v 1.0)
% This means you may do almost anything with this work of mine, so long as you give me proper credit

Calculate the ideal ``slope'' values for a glass pH electrode at the following temperatures:

\begin{itemize}
\item{} $T$ = 15 $^{o}$C ; slope = \underbar{\hskip 50pt} mV/pH
\vskip 5pt
\item{} $T$ = 33 $^{o}$C ; slope = \underbar{\hskip 50pt} mV/pH
\vskip 5pt
\item{} $T$ = 110 $^{o}$F ; slope = \underbar{\hskip 50pt} mV/pH
\vskip 5pt
\item{} $T$ = 40 $^{o}$F ; slope = \underbar{\hskip 50pt} mV/pH
\end{itemize}

\vskip 20pt \vbox{\hrule \hbox{\strut \vrule{} {\bf Suggestions for Socratic discussion} \vrule} \hrule}

\begin{itemize}
\item{} Do changes in a pH instrument's slope represent a {\it zero} shift, a {\it span} shift, or a change in {\it linearity}?
\item{} Demonstrate how to {\it estimate} numerical answers for this problem without using a calculator.
\end{itemize}

\underbar{file i04148}
%(END_QUESTION)





%(BEGIN_ANSWER)

\noindent
{\bf Partial answer:}

\begin{itemize}
\item{} $T$ = 33 $^{o}$C ; slope = \underbar{\bf 60.76} mV/pH
\vskip 5pt
\item{} $T$ = 40 $^{o}$F ; slope = \underbar{\bf 55.09} mV/pH
\end{itemize}

%(END_ANSWER)





%(BEGIN_NOTES)

$$\hbox{Slope} = {{2.303 R T} \over {nF}}$$

$$\hbox{Slope} = {{(2.303) (8.315) T} \over {(1)(96485)}}$$

$$\hbox{Slope} = 0.000198471 T$$

\begin{itemize}
\item{} $T$ = 15 $^{o}$C ; slope = \underbar{\bf 57.19} mV/pH
\vskip 5pt
\item{} $T$ = 33 $^{o}$C ; slope = \underbar{\bf 60.76} mV/pH
\vskip 5pt
\item{} $T$ = 110 $^{o}$F ; slope = \underbar{\bf 62.81} mV/pH
\vskip 5pt
\item{} $T$ = 40 $^{o}$F ; slope = \underbar{\bf 55.09} mV/pH
\end{itemize}

%INDEX% Chemistry, pH: Nernst voltage slope calculations

%(END_NOTES)


