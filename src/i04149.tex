%(BEGIN_QUESTION)
% Copyright 2010, Tony R. Kuphaldt, released under the Creative Commons Attribution License (v 1.0)
% This means you may do almost anything with this work of mine, so long as you give me proper credit

Read and outline the introduction to the ``Chromatography'' section as well as the ``Automated Chromatographs'' subsection of that same section, found within the ``Continuous Analytical Measurement'' chapter in your {\it Lessons In Industrial Instrumentation} textbook.  Note the page numbers where important illustrations, photographs, equations, tables, and other relevant details are found.  Prepare to thoughtfully discuss with your instructor and classmates the concepts and examples explored in this reading.

\underbar{file i04149}
%(END_QUESTION)




%(BEGIN_ANSWER)


%(END_ANSWER)





%(BEGIN_NOTES)

Chromatography consists of forcing a chemical solution to travel through a ``column'' of porous material while carried by a solvent, the result being that some chemical compounds in that solution will travel faster through the column than others.  The {\it time} taken by each compound to propagate through the column becomes the variable used to identify each compound.  A single chromatograph is able to analyze multiple compounds, making it a multi-variable instrument.

\vskip 10pt

The ``secret'' to the flexibility of chromatography is that the detector used to sense the exiting of chemical compounds from the column need not be able to discriminate between lots of different compounds.  It need only tell the difference between carrier and non-carrier compounds.

\vskip 10pt

The output of a chromatograph is a special plot called a {\it chromatogram}, showing ``peaks'' of high detector signal over time.  The timing of these peaks identifies their chemical species, while the height of these peaks is proportional to the detector response (i.e. how much of that species passed by at that time).  Generally speaking, the lightest species exit the column first, while the heaviest species exit last.





\vskip 20pt \vbox{\hrule \hbox{\strut \vrule{} {\bf Suggestions for Socratic discussion} \vrule} \hrule}

\begin{itemize}
\item{} Identify the components and describe the functions of a basic gas chromatograph (GC).
\item{} Is a chromatograph a {\it singularly selective analyzer}, or can it detect {\it multiple} chemical types in a sample?
\item{} Explain how a chromatograph is able to sort one type of chemical substance from another in a solution.
\item{} Identify some characteristics of a good {\it carrier gas} for a GC.
\item{} How might a chromatogram be altered if the GC's carrier gas flow rate were decreased?
\item{} How might a chromatogram be altered if the GC's carrier gas flow rate were increased?
\item{} Is automated chromatography applicable only to gases, or may liquid samples be analyzed in this manner as well?
\end{itemize}








\vfil \eject

\noindent
{\bf Prep Quiz}

The {\it column} inside a chromatograph is:

\begin{itemize}
\item{} A precision switching valve used to re-direct flow
\vskip 5pt
\item{} A very long, thin tube filled with porous material
\vskip 5pt
\item{} The regulating mechanism for carrier gas flow rate
\vskip 5pt
\item{} An electrical heater used to maintain temperature
\vskip 5pt
\item{} A pump used to convey sample gas through the analyzer
\vskip 5pt
\item{} A series of trained hamsters, conditioned to identify different chemical compounds
\end{itemize}









\vfil \eject

\noindent
{\bf Prep Quiz}

Chromatography works on the following principle:

\begin{itemize}
\item{} Detecting the conductivity of a substance as it passes by a heater
\vskip 5pt
\item{} Detecting the ionization of a substance as it passes through a flame
\vskip 5pt
\item{} Sensing the heat convected from a warm element to a moving substance
\vskip 5pt
\item{} Measuring the amount of light absorbed by one substance compared to another
\vskip 5pt
\item{} Timing the retention of a substance as it moves through a porous medium
\vskip 5pt
\item{} Changes of color when one substance chemically reacts with another
\end{itemize}

%INDEX% Reading assignment: Lessons In Industrial Instrumentation, Analytical (chromatography -- intro)

%(END_NOTES)


