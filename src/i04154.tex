
%(BEGIN_QUESTION)
% Copyright 2009, Tony R. Kuphaldt, released under the Creative Commons Attribution License (v 1.0)
% This means you may do almost anything with this work of mine, so long as you give me proper credit

Read the Emerson Gas Chromatograph application note ``BTU Analysis Using a Gas Chromatograph'' (document EPMGC-AN-004 R0608) in its entirety and answer the following questions:

\vskip 10pt

Explain in general terms what a gas chromatograph does to analyze the composition of natural gas fuel.

\vskip 10pt

Identify some of the more prevalent compounds found in natural gas, as listed in this document, interpreting the ``shorthand'' notation used in the industry (C2, IC5, etc.) used to represent different substances.

\vskip 20pt \vbox{\hrule \hbox{\strut \vrule{} {\bf Suggestions for Socratic discussion} \vrule} \hrule}

\begin{itemize}
\item{} Explain the significance of the {\it response factor} described for each chromatogram peak, and how it is calculated.
\item{} Are there any components of natural gas that have zero heating value?  If so, what are they?
\item{} Natural gas pipeline stations employing gas chromatographs for online gas analysis usually also employ AGA3 or similar custody-transfer flow metering technology for precision measurement of natural gas flow.  Explain how the data provided by the chromatograph might also be useful in maintaining optimum accuracy for AGA3 gas flowmeters.
\end{itemize}

\underbar{file i04154}
%(END_QUESTION)





%(BEGIN_ANSWER)


%(END_ANSWER)





%(BEGIN_NOTES)

Page 4 describes how once the GC has calculated the molar quantities for each compound in the gas, the computer multiplies each of those mole values by its respective BTU heat value (pre-programmed into the computer).  This ideal or ``uncorrected'' heating value is then multiplied by the compressibility value of the natural gas to arrive at a ``corrected'' heating value per standard cubic foot (SCF) of natural gas.

\vskip 10pt

Here are some of the more prevalent species found in natural gas (fuzzy table shown on page 4):

\begin{itemize}
\item{} Methane (90\% typical)
\item{} Ethane (4\% typical)
\item{} Carbon dioxide (1.5\% typical)
\item{} Nitrogen (1.5\% typical)
\item{} Propane (0.75\% typical)
\end{itemize}

\vskip 10pt

The ``response factor'' normalizes the chromatogram according to how responsive the detector happens to be for each separated component.  For any given detector technology, the response will not be equivalent for all chemical compounds, and so a sample containing an equal molecular volume of two different substances will most likely {\it not} generate two equivalent-area peaks on the raw chromatogram.  The microprocessor controller must be calibrated according to the relative response of the detector to each component, so the peaks may be scaled according to actual molecular content.

%INDEX% Reading assignment: Emerson Gas Chromatograph Application Note (BTU analysis using a gas chromatograph)

%(END_NOTES)


