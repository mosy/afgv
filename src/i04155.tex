
%(BEGIN_QUESTION)
% Copyright 2010, Tony R. Kuphaldt, released under the Creative Commons Attribution License (v 1.0)
% This means you may do almost anything with this work of mine, so long as you give me proper credit

Read the Emerson Gas Chromatograph application note ``BTU Analysis Using a Gas Chromatograph'' (document EPMGC-AN-004 R0608) in its entirety and answer the following questions:

\vskip 10pt

One of the challenges of chromatography is to perform an analysis of a complex mixture {\it swiftly}.  The gas chromatograph (GC) schematic shown on page 2 of this document achieves fast analysis by using a complex series of valves and columns.  Study the analysis sequence described on this page, and then explain how these multiple columns achieve a faster analysis than if a single column were used.

\vskip 20pt \vbox{\hrule \hbox{\strut \vrule{} {\bf Suggestions for Socratic discussion} \vrule} \hrule}

\begin{itemize}
\item{} The genius of using multiple columns in a chromatograph is the ability to accelerate the separation and detection of certain component in the sample while giving other components the time needed to adequately separate.  Is there some other method we may alternatively use to achieve this end, rather than adding complexity with additional columns and switching valves?
\item{} How does the presence of multiple columns, switching valves, and sequencing affect the order that components appear at the detector?  How would this alter our interpretation of the analyzer's chromatogram?
\item{} Explain why the C6+ ``heavy'' molecules all exit the chromatograph in one bunch, rather than being separated (e.g. C6, C7, C8, etc.) like the lighter molecules.  Hint: it has to do with the ``backflush'' valve and how that routes the C6+ molecules through the columns.
\item{} If we were to modify column 4 for finer separation of the heavy molecules (C6+), in what order would those compounds exit?  What specific modifications would we need to make to column 4 in order to achieve better separation at the end?  Would these alterations speed up or slow down the analysis?
\end{itemize}

\underbar{file i04155}
%(END_QUESTION)





%(BEGIN_ANSWER)


%(END_ANSWER)





%(BEGIN_NOTES)

By using multiple columns and valves, this GC ``traps'' certain compounds and then backflushes them in order to force them to exit the analyzer faster than if all compounds had to travel through one column.  One of the consequences of this analysis strategy is that the chromatogram is not in the sequence one might expect (light species first, heavy species last).

\vskip 10pt

{\it Temperature programming} is also capable of speeding up some portions of the analysis while keeping others slow, but this is generally not as efficient as using multiple columns and switching valves.

\vskip 10pt

The backflush valve (V2) reverses the flow of C6+ ``heavies'' through column 1 which is the only packed column the C6+ molecules ever travel through (column 4 is un-packed and thus performs no separation), so any separation occurring in column 1 is ``un-done'' as the heavy molecules reverse direction, meaning all the ``heavies'' exit the chromatograph as one bunch.  The only separation between C6+ species is what happens in column 4 at the very end.









\vfil \eject

\noindent
{\bf Summary Quiz}

The genius of using multiple columns in a chromatograph is the ability to accelerate the separation and detection of certain component in the sample while giving other components the time needed to adequately separate.  Is there some other method we may alternatively use to achieve this end, rather than adding complexity with additional columns and switching valves?


%INDEX% Reading assignment: Emerson Gas Chromatograph Application Note (BTU analysis using a gas chromatograph)

%(END_NOTES)


