%(BEGIN_QUESTION)
% Copyright 2009, Tony R. Kuphaldt, released under the Creative Commons Attribution License (v 1.0)
% This means you may do almost anything with this work of mine, so long as you give me proper credit

Read and outline the ``Introduction to Optical Analyses'' section of the ``Continuous Analytical Measurement'' chapter in your {\it Lessons In Industrial Instrumentation} textbook.  Note the page numbers where important illustrations, photographs, equations, tables, and other relevant details are found.  Prepare to thoughtfully discuss with your instructor and classmates the concepts and examples explored in this reading.

\underbar{file i04157}
%(END_QUESTION)




%(BEGIN_ANSWER)


%(END_ANSWER)





%(BEGIN_NOTES)

The amount of energy carried by a single photon (particle of light) is given by the following formula:

$$E = hf \hbox{\hskip 50pt or \hskip 50pt} E = {hc \over \lambda}$$

\noindent
Where,

$E$ = Energy carried by a single ``photon'' of light (joules)

$h$ = Planck's constant (6.626 $\times$ $10^{-34}$ joule-seconds)

$f$ = Frequency of light wave (Hz, or 1/seconds)

$c$ = Speed of light in a vacuum ($\approx$ 3 $\times$ $10^8$ meters per second)

$\lambda$ = Wavelength of light (meters)

\vskip 10pt

As a photon strikes an atom, energy is transferred to the electrons of that atom.  If the energy of the photon equals the energy necessary to boost an electron to a higher (available) energy state (i.e. into a higher-energy shell or subshell), the photon's energy will be absorbed in that process.  Electrical current passed through a sample is also able to excite electrons into higher-energy states.  When those electrons return to their original subshell positions, they emit light characteristic to the energy gap of the ``jump'' from the high-energy state to the original state.  Since the available energy states are unique to the electron structure of the atom, the light spectrum will be unique to that element.  Thus, we may pass an electric arc through a gas sample and identify the type of gas it is by the spectrum of light wavelengths emitted.

\vskip 10pt

An alternative analysis method is to shine broad-spectrum light through a gas sample and look for which wavelengths of light are absorbed by the gas.  The atoms of that gas tend to absorb the exact same wavelengths of light that they would emit if excited by an electric current.  The result is an {\it absorption spectrum}, with missing wavelengths unique to the elements present.

\vskip 10pt

Whole molecules also have unique interactions with light (not just atoms), owing to the resonant conditions arising from bonding between atoms in the molecule.  The atoms within molecules can vibrate, twist, and otherwise move relative to each other in ways that match specific wavelengths of light.  Therefore, molecules have their own unique optical signatures which may be analyzed to identify the molecular type.

Diatomic molecules such as hydrogen (H$_{2}$), oxygen (O$_{2}$), and nitrogen (N$_{2}$) exhibit negligible interaction with infra-red light wavelengths, which makes IR light a useful analytical probe when measuring other gases (e.g. CH$_{4}$, CO$_{2}$) mixed with common gases like hydrogen, oxygen, and nitrogen.  An analyzer built to respond only to the optical absorption spectrum of one gas type will be able to pick out that one gas from within a mixture of gases.

\vskip 10pt

The Beer-Lambert Law relates optical absorption to the concentration of light-absorbing compounds in an optical path:

$$A = abc = \log \left({I_0 \over I}\right)$$

\noindent
Where,

$A$ = Absorbance

$a$ = Extinction coefficient for photon-absorbing substance(s) 

$b$ = Path length of light traveling through the sample

$c$ = Concentration of photon-absorbing substance in the sample 

$I_0$ = Intensity of source (incident) light

$I$ = Intensity of received light after passing through the sample

\vskip 10pt






\filbreak

\vskip 20pt \vbox{\hrule \hbox{\strut \vrule{} {\bf Suggestions for Socratic discussion} \vrule} \hrule}

\begin{itemize}
\item{} Describe how we could build an optical analyzer to measure the concentration of carbon dioxide gas in ambient air.  Identify the principle(s) of operation, as well as the construction of the instrument.  How could its sensitivity be maximized?  How could we ensure it was selective to CO$_{2}$ (i.e. that it ignores other species in the air)?
\item{} Explain the difference between an {\it emission spectrum} and an {\it absorption spectrum}.
\item{} Explain the ``shadow'' analogy of optical absorption, in your own words.
\item{} Explain how to intepret an emission spectrum graph for an element such as hydrogen (H$_{2}$).
\item{} Explain how to intepret an absorption spectrum graph for a compound such as methane (CH$_{4}$).
\item{} Explain why the optical spectra of molecular compounds are broader than the individual lines seen in atomic spectra.
\item{} According to the Beer-Lambert Law, what may we do to increase the sensitivity of an optical gas analyzer?
\item{} When a particular wavelength of light is absorbed by a gas, where does that energy go?  It cannot simply disappear, according to the Law of Energy Conservation, so it must go somewhere!
\end{itemize}

%INDEX% Reading assignment: Lessons In Industrial Instrumentation, Analytical (optical)

%(END_NOTES)


