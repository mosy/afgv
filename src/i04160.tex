%(BEGIN_QUESTION)
% Copyright 2009, Tony R. Kuphaldt, released under the Creative Commons Attribution License (v 1.0)
% This means you may do almost anything with this work of mine, so long as you give me proper credit

Read and outline the ``Dispersive Spectroscopy'' subsection of the ``Optical Analyses'' section of the ``Continuous Analytical Measurement'' chapter in your {\it Lessons In Industrial Instrumentation} textbook.  Note the page numbers where important illustrations, photographs, equations, tables, and other relevant details are found.  Prepare to thoughtfully discuss with your instructor and classmates the concepts and examples explored in this reading.

\underbar{file i04160}
%(END_QUESTION)




%(BEGIN_ANSWER)


%(END_ANSWER)





%(BEGIN_NOTES)

One way to analyze the wavelengths of light coming from an optical analyzer is to split that light up into a spectrum using some device such as a prism or a diffraction/reflection grating.  This is called ``optical dispersion'' or ``dispersive spectroscopy''.  This technique was applied to the analysis of sunlight, which led to the discovery that elements within the outer ``atmosphere'' of the sun absorbed certain wavelengths, and therefore allowed scientists to identify chemical compounds on the surface of the sun!

\vskip 10pt

Industrial gas analyzers use this technique by shining light through a sample chamber containing the gas to be analyzed, then dispersing the received light into its constituent wavelengths, and analyzing which wavelengths have been attenuated by the gas.  The patterns of attenuation reveal both the identity(ies) of the gas as well as the relative concentrations.  A computer is usually necessary to analyze the complex spectrum to make these determinations.









\vskip 20pt \vbox{\hrule \hbox{\strut \vrule{} {\bf Suggestions for Socratic discussion} \vrule} \hrule}

\begin{itemize}
\item{} {\bf In what ways may a dispersive optical instrument be ``fooled'' to report a false composition measurement?}
\item{} Explain how dispersive spectroscopy was used to identify chemical elements in the outer layers of the Sun, and later for more distant stars.
\item{} Objects in relative motion to each other -- if that motion is fast enough -- produce a shifting of light color due to the Doppler Effect.  Stars moving away from Earth appear redder than they actually are, and stars approaching Earth appear bluer than they actually are.  Does this nullify optical analysis as a measurement technique for determining the gaseous composition of distant stars?  Why or why not?
\item{} Examine the illustration of a dispersive absorption analyzer and explain how it works to identify and quantify different chemical species.
\item{} Examine the illustration of a dispersive absorption analyzer and explain what would change if the concentration of light-absorbing gas were to increase inside the sample chamber.
\item{} Examine the illustration of a dispersive absorption analyzer and explain what would change if the type of light-absorbing gas were to change inside the sample chamber.
\item{} A BTC Instrumentation student once described absorption spectroscopy as a process where the species of interest was detected by the ``shadow'' it cast.  Explain what this statement means.
\item{} Sketch an illustration of what a dispersive {\it emission} analyzer would look like and explain how it works to identify and quantify different chemical species.
\end{itemize}

%INDEX% Reading assignment: Lessons In Industrial Instrumentation, Analytical (dispersive spectroscopy)

%(END_NOTES)


