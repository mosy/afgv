
%(BEGIN_QUESTION)
% Copyright 2009, Tony R. Kuphaldt, released under the Creative Commons Attribution License (v 1.0)
% This means you may do almost anything with this work of mine, so long as you give me proper credit

Suppose you take a source of {\it white light} and shine it through a {\it diffraction grading} or against a {\it reflection grating}.  Qualitatively describe the spectrum of colors you would see as a result.

\vskip 10pt

Suppose you take a source of {\it white light} and shine it through a transparent chamber holding a strong concentration of a gas, then shine the light exiting that chamber through a {\it diffraction grading} or against a {\it reflection grating}.  Qualitatively describe the spectrum of colors you would see as a result.

\vskip 10pt

Suppose you take a transparent chamber holding a strong concentration of a gas and electrically excite that gas using an arc, then shine the light exiting that chamber through a {\it diffraction grading} or against a {\it reflection grating}.  Qualitatively describe the spectrum of colors you would see as a result.

\vskip 20pt \vbox{\hrule \hbox{\strut \vrule{} {\bf Suggestions for Socratic discussion} \vrule} \hrule}

\begin{itemize}
\item{} Comment on the relationships seen between colors in the spectrum and the chemical identity of the gas.
\item{} Is the sun a perfect source of white light?  Why or why not?
\item{} Devise an instrument to identify the chemical composition of a gas by optical {\it emission}.
\item{} Devise an instrument to identify the chemical composition of a gas by optical {\it absorption}.
\end{itemize}

\underbar{file i04161}
%(END_QUESTION)





%(BEGIN_ANSWER)


%(END_ANSWER)





%(BEGIN_NOTES)

Suppose you take a source of {\it white light} and shine it through a {\it diffraction grading} or against a {\it reflection grating}.  Qualitatively describe the spectrum of colors you would see as a result.  {\bf You will get a full-colored spectrum of light, because all wavelengths are present.}

\vskip 10pt

Suppose you take a source of {\it white light} and shine it through a transparent chamber holding a strong concentration of a gas, then shine the light exiting that chamber through a {\it diffraction grading} or against a {\it reflection grating}.  Qualitatively describe the spectrum of colors you would see as a result.  {\bf You will get a spectrum of light that is missing some specific wavelengths, because those wavelengths have been absorbed by the gas sample.}

\vskip 10pt

Suppose you take a transparent chamber holding a strong concentration of a gas and electrically excite that gas using an arc, then shine the light exiting that chamber through a {\it diffraction grading} or against a {\it reflection grating}.  Qualitatively describe the spectrum of colors you would see as a result.  {\bf You will get just a few unique wavelengths of light (i.e. a very incomplete spectrum), because only those few wavelengths will be emitted by the gas sample.}


%INDEX% Physics, optics: spectroscopy

%(END_NOTES)


