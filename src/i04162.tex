%(BEGIN_QUESTION)
% Copyright 2011, Tony R. Kuphaldt, released under the Creative Commons Attribution License (v 1.0)
% This means you may do almost anything with this work of mine, so long as you give me proper credit

Using an ultraviolet (``black light'') source such as a UV flashlight or a UV LED connected to an appropriate dropping resistor and battery, demonstrate {\it fluorescence} in a darkened classroom with substances you bring to class (see list).  Your instructor will provide the UV light sources.  

\vskip 10pt

\centerline{\bf IMPORTANT SAFETY NOTE: ultraviolet light sources can damage}
\centerline{\bf your eyes, so \underbar{never} look directly into a UV light source!}

\vskip 10pt

Try the following materials for fluorescence:

\begin{itemize}
\item{} Freshly-washed clothes
\item{} Vitamin B-12 pill
\item{} Club soda (tonic water, contains {\it quinine})
\item{} Postage stamps
\item{} Chlorophyll (green, leafy plant)
\item{} Petroleum jelly
\item{} Agate (stone)
\item{} Amber (resin)
\item{} {\it Any others?}
\end{itemize}

Imagine you were tasked with designing an analyzer to detect the presence of any {\it one} of the substances listed above.  Identify ways in which your analyzer could successfully ignore (``reject'') fluorescence from substances {\it other} than the one it is supposed to sense.

\vskip 20pt \vbox{\hrule \hbox{\strut \vrule{} {\bf Suggestions for Socratic discussion} \vrule} \hrule}

\begin{itemize}
\item{} Explain how we may use a digital multimeter to determine the forward voltage drop of an ultraviolet LED, for use in later resistor-sizing calculations.
\item{} Explain how we may calculate the necessary dropping resistor value to keep the LED from burning out when powered by a battery whose voltage exceeds the maximum voltage rating for the LED.
\item{} We know that fluorescence happens when a high-energy photon strikes a molecule and causes a lower-energy photon to be emitted.  Can the reverse happen: a low-energy photon strikes a molecule and emits a higher-energy photon?  Why or why not?
\end{itemize}

\underbar{file i04162}
%(END_QUESTION)





%(BEGIN_ANSWER)


%(END_ANSWER)





%(BEGIN_NOTES)

Ultraviolet LEDs typically have a fairly high forward voltage drop: over 3 volts.  It is important to measure this forward voltage drop prior to building a circuit to power the LED.

\vskip 10pt

\centerline{\bf IMPORTANT SAFETY NOTE: ultraviolet light sources can damage}
\centerline{\bf your eyes, so \underbar{never} look directly into a UV light source!}




\vfil \eject

\noindent
{\bf Prep Quiz}

Suppose you are given an ultraviolet LED that typically operates with a forward voltage drop of 3.7 volts and a forward current of 25 milliamps.  Your only power source, however, is a 12 volt battery.  From this given information, choose the best series resistor to connect to this LED so that it will operate within normal parameters when powered by the 12 volt battery:

\begin{itemize}
\item{} 628 ohms, $1 \over 2$ watt
\vskip 5pt
\item{} 480 ohms, $1 \over 2$ watt
\vskip 5pt
\item{} 148 ohms, $1 \over 4$ watt
\vskip 5pt
\item{} 415 ohms, $1 \over 4$ watt
\vskip 5pt
\item{} 332 ohms, $1 \over 4$ watt
\vskip 5pt
\item{} 130 ohms, 1 watt
\end{itemize}



\vfil \eject

\noindent
{\bf Prep Quiz}

Suppose you are given an ultraviolet LED that typically operates with a forward voltage drop of 3.9 volts and a forward current of 30 milliamps.  Your only power source, however, is a 24 volt DC power supply unit.  From this given information, choose the best series resistor to connect to this LED so that it will operate within normal parameters when powered by the 24 VDC supply:

\begin{itemize}
\item{} 800 ohms, $3 \over 4$ watt
\vskip 5pt
\item{} 130 ohms, $1 \over 8$ watt
\vskip 5pt
\item{} 670 ohms, 1 watt
\vskip 5pt
\item{} 270 ohms, $1 \over 4$ watt
\vskip 5pt
\item{} 205 ohms, $1 \over 4$ watt
\vskip 5pt
\item{} 930 ohms, 1 watt
\end{itemize}



%INDEX% Physics, optics: fluorescence

%(END_NOTES)


