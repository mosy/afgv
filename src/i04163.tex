%(BEGIN_QUESTION)
% Copyright 2009, Tony R. Kuphaldt, released under the Creative Commons Attribution License (v 1.0)
% This means you may do almost anything with this work of mine, so long as you give me proper credit

Read and outline the ``Fluorescence'' section of the ``Continuous Analytical Measurement'' chapter in your {\it Lessons In Industrial Instrumentation} textbook.  Note the page numbers where important illustrations, photographs, equations, tables, and other relevant details are found.  Prepare to thoughtfully discuss with your instructor and classmates the concepts and examples explored in this reading.

\underbar{file i04163}
%(END_QUESTION)




%(BEGIN_ANSWER)


%(END_ANSWER)





%(BEGIN_NOTES)

If a photon ejects a low-level electron from an atom, a higher-level electron may ``fall down'' to fill the vacancy, resulting in a situation where the wavelength emitted by the atom (from the electron falling down from one level to another) is longer than the wavelength of the incident photon that completely ejected the original electron from the atom.  In most cases, this takes the form of ultraviolet light causing materials to emit visible light: a phenomenon called {\it fluorescence}.

\vskip 10pt

Only certain substances fluoresce, and when they do they tend to emit unique colors of visible light.  This makes it possible to identify certain compounds by fluorescence.  Sulfur dioxide (SO$_{2}$) gas, an industrial pollutant, happens to be one of those compounds.

SO$_{2}$ fluorescence analyzers shine UV light on a gas sample, then use a photomultiplier tube to detect any fluorescent light emitted by the gas molecules.

\vskip 10pt

Interference may result from nitric oxide (NO) gas, as well as from polynuclear aromatic hydrocarbon (PAH) compounds, since both these substances fluoresce as well as SO$_{2}$.  In the case of NO, it happens to fluoresce by emitting a different wavelength (color) of visible light, which means we may eliminate the interference by placing an appropriate optical filter in front of the photomultiplier tube window.  In the case of PAH, we must chemically filter those compounds out of the sample gas inlet.









\vskip 20pt \vbox{\hrule \hbox{\strut \vrule{} {\bf Suggestions for Socratic discussion} \vrule} \hrule}

\begin{itemize}
\item{} {\bf In what ways may a fluorescence analyzer instrument be ``fooled'' to report a false composition measurement?}
\item{} Devise an experiment by which we could test an SO$_{2}$ analyzer's ability to reject interference from other gas species.
\item{} Sketch or locate a pictorial representation of an atom (showing electrons residing in different shells and/or subshells), and use that illustration to demonstrate the principle of fluorescence.
\item{} Examine the textbook's illustration of the Balmer and Lyman series of electron jumps for the element hydrogen (in the Chemistry chapter), and demonstrate an example of fluorescence with that element.
\item{} Describe examples of ``interference'' in fluorescence-based gas analyzers, where our ability to accurately measure the gas of interest is compromised by the presence of some other gas type.
\item{} Explain how a photomultiplier tube works to detect light.
\item{} Examine the textbook's photograph of a Thermo Electron model 43 SO$_{2}$ analyzer and explain why the analyzing chamber is shaped like the letter ``T''.
\item{} Explain how you would calibrate an SO$_{2}$ analyzer.
\item{} Explain what would happen to the output of a fluorescence SO$_{s}$ analyzer if its incoming sample was insufficiently filtered and contained some dust.
\item{} Explain what would happen to the output of a fluorescence SO$_{s}$ analyzer if its ``kicker'' unit stopped working.
\item{} Explain why the pressure of the gas being analyzed is an important factor in an optical gas analyzer's calibration.
\item{} Is the calibration error produced by variations in sample gas pressure best characterized as a {\it zero} shift, a {\it span} shift, a {\it linearity} error, or a {\it hysteresis} error?
\end{itemize}




\vfil \eject

\noindent
{\bf Prep Quiz}

An example of fluorescence is:

\begin{itemize}
\item{} When an infra-red photon strikes an atom, releasing an ultraviolet photon
\vskip 5pt
\item{} When an ultraviolet photon strikes an atom, releasing an identical photon
\vskip 5pt
\item{} When an ultraviolet photon strikes an atom, releasing a visible photon
\vskip 5pt
\item{} When a low-energy photon strikes an atom, releasing a high-energy photon
\vskip 5pt
\item{} When an infra-red photon strikes an atom, releasing a visible photon
\end{itemize}





\vfil \eject

\noindent
{\bf Summary Quiz}

Interference with the fluorescent analysis of sulfur dioxide (SO$_{2}$) gas caused by nitric oxide (NO) and polynuclear aromatic hydrocarbons (PAH) is mitigated by the following means:

\begin{itemize}
\item{} Use of a photomultiplier tube, which can only detect the fluorescence of SO$_{2}$
\vskip 5pt
\item{} A flame ionization detector to burn away any PAH molecules and to reduce NO molecules
\vskip 5pt
\item{} Using a ``kicker'' to convert NO molecules into NO$_{2}$ and NO$_{3}$ which don't fluoresce
\vskip 5pt
\item{} Use of span gases that do not contain any NO or PAH molecules to cause interference
\vskip 5pt
\item{} Optical filtering for NO fluorescence, and chemical filtering of PAH molecules
\end{itemize}


%INDEX% Reading assignment: Lessons In Industrial Instrumentation, Analytical (fluorescence)

%(END_NOTES)


