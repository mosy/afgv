%(BEGIN_QUESTION)
% Copyright 2009, Tony R. Kuphaldt, released under the Creative Commons Attribution License (v 1.0)
% This means you may do almost anything with this work of mine, so long as you give me proper credit

Read and outline the ``Chemiluminescence'' section of the ``Continuous Analytical Measurement'' chapter in your {\it Lessons In Industrial Instrumentation} textbook.  Note the page numbers where important illustrations, photographs, equations, tables, and other relevant details are found.  Prepare to thoughtfully discuss with your instructor and classmates the concepts and examples explored in this reading.

\underbar{file i04164}
%(END_QUESTION)




%(BEGIN_ANSWER)


%(END_ANSWER)





%(BEGIN_NOTES)

Chemiluminescence is when an exothermic chemical reaction emits visible light.  Such is the case when nitric oxide (NO) gas reacts with ozone (O$_{3}$) gas.  Chemiluminescent NO gas analyzers generate ozone gas by exciting molecules of oxygen gas, which is then introduced into a dark chamber with the sample gas, and the emitted light measured by a photomultiplier tube.

\vskip 10pt

Only NO gas chemiluminesces, not any of the other nitric oxides (NO$_{x}$).  Normally, this selectivity would be a good thing, but here we're usually interested in measuring all the NO$_{x}$ compounds because they are all pollutants and all mitigated in the same way.  Therefore, NO$_{x}$ analyzers must somehow convert the other oxides of nitrogen into NO gas.  

If we were to do this simply by heating the sample to 1300 $^{o}$F, we would create an interference with ammonia gas, which also forms NO at that high temperature.  So instead we use a molybdenum reactant at a lower temperature (750 $^{o}$F) to do the chemical conversion of NO$_{x}$ $\to$ NO and avoid converting NH$_{3}$ into NO.









\vskip 20pt \vbox{\hrule \hbox{\strut \vrule{} {\bf Suggestions for Socratic discussion} \vrule} \hrule}

\begin{itemize}
\item{} {\bf In what ways may a chemiluminescence analyzer instrument be ``fooled'' to report a false composition measurement?}
\item{} Would the chemiluminescent reaction of NO gas with ozone have a positive $\Delta H$ value or a negative $\Delta H$ value (heat of formation)?
\item{} Describe examples of ``interference'' in chemiluminescence-based gas analyzers, where our ability to accurately measure the gas of interest is compromised by the presence of some other gas type.
\item{} Explain how a photomultiplier tube works to detect light.
\item{} Explain how you would calibrate an NO$_{x}$ analyzer.
\item{} Explain what would happen to the output of a chemiluminescence NO$_{x}$ analyzer if its ``converter'' unit stopped working.
\item{} Explain what would happen to the output of a chemiluminescence NO$_{x}$ analyzer if its ozone generator unit stopped working.
\end{itemize}






\vfil \eject

\noindent
{\bf Prep Quiz}

The purpose of a {\it converter} in a chemiluminescence NO$_{x}$ analyzer is to:

\begin{itemize}
\item{} Convert NO$_{2}$ and NO$_{3}$ into NO gas molecules
\vskip 5pt
\item{} Convert AC power into DC power for the analyzer circuitry
\vskip 5pt
\item{} Convert O$_{2}$ gas molecules into O$_{3}$ gas molecules
\vskip 5pt
\item{} Convert PAH molecules into simpler hydrocarbon molecules
\vskip 5pt
\item{} Prevent PAH molecules from entering the analyzer at all
\vskip 5pt
\item{} Convert CO molecules into CO$_{2}$ molecules
\end{itemize}




\vfil \eject

\noindent
{\bf Summary Quiz}

The problem with simple high-temperature (1300 $^{o}$F) converters in chemiluminescence NO$_{x}$ analyzers is that they:

\begin{itemize}
\item{} Fail to filter out all the PAH molecules, causing measurement interference
\vskip 5pt
\item{} Require continual replenishing of molybdenum metal, which is consumed over time
\vskip 5pt
\item{} Consume too much electrical power to be practical for any real application
\vskip 5pt
\item{} Also convert ammonia (NH$_{3}$) molecules into NO molecules, causing interference
\vskip 5pt
\item{} Deplete the ozone (O$_{3}$) molecules necessary for efficient chemiluminescence
\end{itemize}





\vfil \eject

\noindent
{\bf Summary Quiz}

Balance the chemical equation for a metallic NO$_{x}$ converter converting NO$_{3}$ gas into NO gas:

$$\hbox{NO}_3 + \hbox{Mo} \to \hbox{MoO}_3 + \hbox{NO}$$

$$3\hbox{NO}_3 + 3\hbox{Mo} \to 3\hbox{MoO}_3 + 3\hbox{NO}$$

$$3\hbox{NO}_3 + 2\hbox{Mo} \to 2\hbox{MoO}_3 + 3\hbox{NO}$$

$$3\hbox{NO}_3 + \hbox{Mo} \to \hbox{MoO}_3 + 3\hbox{NO}$$

$$2\hbox{NO}_3 + 3\hbox{Mo} \to 3\hbox{MoO}_3 + 2\hbox{NO}$$

$$2\hbox{NO}_3 + 2\hbox{Mo} \to 2\hbox{MoO}_3 + 2\hbox{NO}$$




\vfil \eject

\noindent
{\bf Summary Quiz}

Suppose the molybdenum becomes completely consumed in the converter unit of a NO$_{x}$ analyzer.  Will this change cause the analyzer to read falsely low, read falsely high, or will it not affect its accuracy at all?


%INDEX% Reading assignment: Lessons In Industrial Instrumentation, Analytical (chemiluminescence)

%(END_NOTES)


