%(BEGIN_QUESTION)
% Copyright 2009, Tony R. Kuphaldt, released under the Creative Commons Attribution License (v 1.0)
% This means you may do almost anything with this work of mine, so long as you give me proper credit

Read the ``Continuous Emissions Monitoring (CEMS) Package With NGA 2000 Analyzers'' datasheet from Rosemount (document PDS 103-101M.A01) and answer the following questions:

\vskip 10pt

Identify the analytical technologies this particular CEMS instrument uses to detect the following chemical species:

\begin{itemize}
\item{} SO$_{2}$: \underbar{\hskip 50pt}
\vskip 5pt
\item{} CO: \underbar{\hskip 50pt}
\vskip 5pt
\item{} CO$_{2}$: \underbar{\hskip 50pt}
\vskip 5pt
\item{} Hydrocarbons: \underbar{\hskip 50pt}
\vskip 5pt
\item{} NO$_{x}$: \underbar{\hskip 50pt}
\vskip 5pt
\item{} O$_{2}$: \underbar{\hskip 50pt}
\end{itemize}

\vskip 10pt

Explain how the ``electrochemical/galvanic'' cell works to sense oxygen.

\vskip 10pt

Part of this analyzer system is a {\it sampling probe} to extract samples of gases from the exhaust stack of a combustion process (such as a large boiler or process heater).  Identify some of this sampling system's functions (i.e. what does the ``sampling probe'' do to the sample to ensure better analysis?).

\vskip 20pt \vbox{\hrule \hbox{\strut \vrule{} {\bf Suggestions for Socratic discussion} \vrule} \hrule}

\begin{itemize}
\item{} What is the purpose for analyzing stack gas composition?
\item{} Which of the analytical sensors in this CEMS instrument contain consumable materials, and which do not?
\item{} Suppose a combustion process was running {\it rich} (i.e. too much fuel per unit of air).  What effects would this condition have on the relative concentrations of CO versus CO$_{2}$ gases in the exhaust?
\item{} Do NO$_{x}$ emissions increase under {\it lean}-burn conditions or under {\it rich}-burn conditions?
\item{} Do hydrocarbon emissions increase under {\it lean}-burn conditions or under {\it rich}-burn conditions?
\item{} How does a ``paramagnetic'' oxygen analyzer work?
\item{} Is the ``electrochemical/galvanic'' sensor {\it potentiometric} (voltage-based) or {\it amperometric} (current-based)?
\end{itemize}

\underbar{file i04165}
%(END_QUESTION)




%(BEGIN_ANSWER)


%(END_ANSWER)





%(BEGIN_NOTES)

Page 3 of this document lists the various measurement technologies used:

\begin{itemize}
\item{} SO$_{2}$: \underbar{\bf ultraviolet (fluorescence)}
\item{} CO: \underbar{\bf NDIR}
\item{} CO$_{2}$: \underbar{\bf NDIR}
\item{} Hydrocarbons: \underbar{\bf flame ionization}
\item{} NO$_{x}$: \underbar{\bf chemiluminescence}
\item{} O$_{2}$: \underbar{{\bf paramagnetic} or {\bf electrochemical}}
\end{itemize}

\vskip 10pt

The ``electrochemical/galvanic'' cell oxidizes lead in a ``micro fuel cell'' to form lead oxide and produce an electric current proportional to the concentration of oxygen in the sample.  Selectivity is ensured by means of a gas diffusion membrane which only passes oxygen into the cell's electrolyte.

\vskip 10pt

Page 5 describes the sampling probe for this analyzer.  The sampling probe filters the sample gas and maintains a high temperature so water vapor does not condense in the sample lines.  An additional ``sample conditioner'' removes this vapor so there is no condensation (and subsequent corrosion) inside the analyzer chambers.









\vskip 20pt \vbox{\hrule \hbox{\strut \vrule{} {\bf Suggestions for Socratic discussion} \vrule} \hrule}

\begin{itemize}
\item{} Explain how you would calibrate this gas analyzer using as few steps as possible.
\item{} Explain why the transport time lag specified in the Gas Sampling Probe is stated as ``variable.''
\end{itemize}



%INDEX% Reading assignment: Rosemount datasheet for ``NGA 2000'' CEMS analyzer

%(END_NOTES)


