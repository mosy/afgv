%(BEGIN_QUESTION)
% Copyright 2009, Tony R. Kuphaldt, released under the Creative Commons Attribution License (v 1.0)
% This means you may do almost anything with this work of mine, so long as you give me proper credit

Calculate the molecular weight of regular air, knowing the molecular weight of nitrogen (N$_{2}$) is 28 amu, the molecular weight of oxygen (O$_{2}$) is 32 amu, the molecular weight of argon (Ar) is 40 amu, and that the percentages of each (by volume, which is practically the same as by molar concentration) are approximately as follows.  Assume an air temperature of 75 degrees Fahrenheit:

\begin{itemize}
\item{} Nitrogen: 78\%
\item{} Oxygen: 21\%
\item{} Argon: 1\%
\end{itemize}

Your calculation will provide a way to relate moles of air to {\it mass}.

\vskip 10pt

Now, suppose an environmental analyzer happens to measure a concentration of sulfur dioxide (SO$_{2}$) in the air of 3 ppm (by volume).  Calculate the mass of sulfur dioxide in every {\it pound} of air.

\vskip 20pt \vbox{\hrule \hbox{\strut \vrule{} {\bf Suggestions for Socratic discussion} \vrule} \hrule}

\begin{itemize}
\item{} Describe some of the problem-solving techniques you could (or did) apply to this question.
\end{itemize}

\underbar{file i04166}
%(END_QUESTION)





%(BEGIN_ANSWER)

\noindent
{\bf Partial answer:}

\vskip 10pt

0.00000663 pounds of sulfur dioxide in every pound of air (6.63 $\times$ 10$^{-6}$ lb of SO$_{2}$ in each lb or air).

%(END_ANSWER)





%(BEGIN_NOTES)

We may calculate the average molecular weight of air by performing a weighted average of the constituent compounds' molecular weights:

\begin{itemize}
\item{} Nitrogen: 78\% = N$_{2}$ = 28 amu
\item{} Oxygen: 21\% = O$_{2}$ = 32 amu
\item{} Argon: 1\% = Ar = 40 amu
\end{itemize}

$$(78\%)(28 \hbox{ amu}) + (21\%)(32 \hbox{ amu}) + (1\%)(40 \hbox{ amu}) = 28.96 \hbox{ amu}$$

\vskip 10pt

A concentration of 3 ppm (by volume) of SO$_{2}$ gas means that for every 1 mole of sample air (28.96 grams of air), there will be $3 \over 1,000,000$ of a mole of SO$_{2}$.  The mass of this tiny quantity of SO$_{2}$ is equal to:

$$\left( {3 \hbox{ mol SO}_2 \over 1000000} \right) \left( {64 \hbox{ g} \over \hbox{mol SO}_2} \right) = 0.000192 \hbox{ grams}$$

We may calculate the mass (in pounds) of SO$_{2}$ gas by doing a proportion with grams of SO$_{2}$ versus air in this 1-mole sample:

$$\left( {0.000192 \hbox{ g SO}_2 \over 28.96 \hbox{ g air}} \right) = \left( {x \hbox{ lb SO}_2 \over 1 \hbox{ lb air} } \right)$$

$$x = 0.00000663 \hbox{ pounds of SO}_2 \hbox{ in 1 lb of air}$$

\vskip 10pt

The 75 degree air temperature is extraneous information, included for the purpose of challenging students to identify whether or not information is relevant to solving a particular problem.

\vskip 10pt

\filbreak

An alternative approach to solving this problem is to calculate the number of moles of air in one pound of air, then calculate $3 \over 1000000$ of that quantity to determine the number of moles of SO$_{2}$, then multiply that quantity by the formula weight of SO$_{2}$ to determine how much SO$_{2}$ mass (in grams) there is in that one-pound air sample.


%INDEX% Chemistry, stoichiometry: moles

%(END_NOTES)


