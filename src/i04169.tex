%(BEGIN_QUESTION)
% Copyright 2011, Tony R. Kuphaldt, released under the Creative Commons Attribution License (v 1.0)
% This means you may do almost anything with this work of mine, so long as you give me proper credit

With NDIR analyzers, the ``fill'' gas used inside the detector is supposed to sensitize the instrument to that same gas in the sample.  For example, an NDIR analyzer having its detector chambers filled completely with CO$_{2}$ gas should exhibit a stronger response to CO$_{2}$ in the sample than to any other infrared light-absorbing gas in the sample.

\vskip 10pt

First, explain why this is so.  What is it, exactly, about the fill gas inside an NDIR detector that makes it respond more strongly to that gas species than to other gas species in the sample?

\vskip 10pt

Second, devise a ``thought experiment'' whereby you could prove that an NDIR instrument was sensitized to one particular gas species.

\underbar{file i04169}
%(END_QUESTION)




%(BEGIN_ANSWER)

The detector in an NDIR analyzer works on pressure: infrared light not absorbed by the sample gas heats gas molecules inside the detector's chambers, exerting pressure in a way that can be converted into an electronic signal.  With Dr. Luft's original design, the gas pressure was sensed by a very thin ``microphone'' diaphragm.  In more modern detectors, the gas pressure causes a flow of gas through a tiny channel which is sensed by a thermal sensor.

The key to understanding NDIR detector sensitivity is to recognize that the gas contained within will only be heated by those specific frequencies of light its molecules absorb.  That is to say, the detector's fill gas will generate pressure only when exposed to wavelengths of light specific to the absorption spectrum of that fill gas type, and not to light of any other wavelengths.

In the given example where our detector was filled completely with CO$_{2}$, pressure would be generated inside the detector only by infrared light wavelengths specific to the absorption patterns of CO$_{2}$ gas.  If any {\it other} light-absorbing gas happened to enter the instrument's sample chamber, the light wavelengths absorbed by that other gas will not heat the CO$_{2}$ gas inside the detector, to the same degree (or even at all!).  Therefore, the detector will be maximally responsive to CO$_{2}$ in the sample, and minimally responsive to any other light-absorbing gases in the sample.

\vskip 10pt

As for the thought experiment, it does us no good to imagine what will happen if CO$_{2}$ enters the sample chamber.  We {\it already} know what that will do, Luft detector or no!  If the goal is to prove that filling the Luft detector {\it sensitizes} the instrument to one particular gas, we must perform a thought experiment where we imagine {\it some different light-absorbing gas} entering the sample chamber, and see for ourselves that this other gas causes little or no response from the Luft detector.

%(END_ANSWER)





%(BEGIN_NOTES)



%INDEX% Measurement, analytical: nondispersive optical

%(END_NOTES)


