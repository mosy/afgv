%(BEGIN_QUESTION)
% Copyright 2009, Tony R. Kuphaldt, released under the Creative Commons Attribution License (v 1.0)
% This means you may do almost anything with this work of mine, so long as you give me proper credit

Read and outline the ``Luft Detectors'' subsection of the ``Non-Dispersive Luft Detector Spectroscopy'' section of the ``Continuous Analytical Measurement'' chapter in your {\it Lessons In Industrial Instrumentation} textbook.  Note the page numbers where important illustrations, photographs, equations, tables, and other relevant details are found.  Prepare to thoughtfully discuss with your instructor and classmates the concepts and examples explored in this reading.

\underbar{file i04171}
%(END_QUESTION)




%(BEGIN_ANSWER)


%(END_ANSWER)





%(BEGIN_NOTES)

A ``Luft'' detector is a gas-filled divided cell sensing pressure differences between two halves of the cell.  The cell is filled with the gas of interest, so that the pressure caused by heating will be maximal for wavelenghts of light absorbed by the process gas we're interested in measuring.  If a gas having a completely different absorption spectrum happens to enter the sample chamber and absorbs light, it will have no effect on the Luft detector because those diminished wavelengths weren't going to heat the (different) gas inside the detector anyway.

\vskip 10pt

Modern Luft detectors dispense with the moving diaphragm in favor of a micro-flow sensor located between two gas-filled chambers.  These are much more tolerant of vibration than the microphone-style (diaphragm) Luft detector.

\vskip 10pt

Luft detectors do not make an NDIR analyzer perfectly selective.  There is still the potential for interference to occur if a gas with an overlapping absorption spectrum with the gas of interest happens to enter the sample chamber.





\vskip 20pt \vbox{\hrule \hbox{\strut \vrule{} {\bf Suggestions for Socratic discussion} \vrule} \hrule}

\begin{itemize}
\item{} {\bf Explain how the Luft detector achieves selectivity}.
\item{} {\bf In what ways may a Luft-detector NDIR instrument be ``fooled'' to report a false composition measurement?}
\item{} Explain what the spectral absorption plots of carbon dioxide and ethane tell us about those two gases' optical characteristics.
\item{} Referring to a Luft detector NDIR analyzer illustration, what would happen if the concentration of sensitizing gas were to increase/decrease inside the sample cell?
\item{} Referring to a Luft detector NDIR analyzer illustration, what would happen if the reference cell were accidently filled with the same gas as is filling the Luft detector?
\item{} Referring to a Luft detector NDIR analyzer illustration, what would happen if the reference cell were accidently filled with a gas that absorbed light, but not the same gas as what is filling the Luft detector?
\item{} Referring to a Luft detector NDIR analyzer illustration, what would happen if the Luft detector were filled with a gas that did not absorb light (e.g. nitrogen)?
\item{} Should the interior walls of a Luft detector be dark or shiny, or does it matter?
\end{itemize}







\vfil \eject

\noindent
{\bf Summary Quiz:}

Identify the purpose of filling the Luft detector with a 100\% concentration of the ``gas of interest'' for an NDIR gas analyzer:

\begin{itemize}
\item{} To sensitize the analyzer to that one species of gas more than all others
\vskip 5pt
\item{} To minimize ``span'' calibration drift due to the source light weakening
\vskip 5pt
\item{} To positively filter out light wavelengths absorbed by interfering species
\vskip 5pt
\item{} To minimize the effects of mechanical vibration on the analyzer's signal
\vskip 5pt
\item{} To minimize ``zero'' calibration drift due to the source light weakening
\end{itemize}


%INDEX% Reading assignment: Lessons In Industrial Instrumentation, Analytical (nondispersive spectroscopy)

%(END_NOTES)


