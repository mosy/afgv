%(BEGIN_QUESTION)
% Copyright 2009, Tony R. Kuphaldt, released under the Creative Commons Attribution License (v 1.0)
% This means you may do almost anything with this work of mine, so long as you give me proper credit

Read and outline the ``Filter Cells'' subsection of the ``Non-Dispersive Luft Detector Spectroscopy'' section of the ``Continuous Analytical Measurement'' chapter in your {\it Lessons In Industrial Instrumentation} textbook.  Note the page numbers where important illustrations, photographs, equations, tables, and other relevant details are found.  Prepare to thoughtfully discuss with your instructor and classmates the concepts and examples explored in this reading.

\underbar{file i04172}
%(END_QUESTION)




%(BEGIN_ANSWER)


%(END_ANSWER)





%(BEGIN_NOTES)

A gas will interfere with an NDIR analyzer's measurement of the gas of interest if it has an absorption spectrum overlapping that of the gas of interest.  One way to combat this interference is to equip the NDIR analyzer with additional gas-filled chambers containing 100\% concentrations of any interfering gas species.  If the filter cells completely block the wavelengths absorbed by the interferent, none will be left to enter the sample chamber, and therefore the presence of that interfering gas inside the sample chamber will have absolutely no effect on the light exiting the chamber.

\vskip 10pt

The only requirement for the gas of interest is that it have at least some portion of its absorption spectrum unique from the interfering species.  If the interferent's absorption spectrum completely overshadows that of the gas of interest, we cannot use filter cells to combat the interference.

\vskip 10pt

Multiple filter cells are fair to use, each pair of filter cells containing 100\% concentrations of the interfering gases.






\vskip 20pt \vbox{\hrule \hbox{\strut \vrule{} {\bf Suggestions for Socratic discussion} \vrule} \hrule}

\begin{itemize}
\item{} {\bf In what ways may an NDIR instrument equipped with filter cells be ``fooled'' to report a false composition measurement?}
\item{} Identify the region of overlap in the absorption spectra of carbon dioxide and acetylene shown in the textbook.
\item{} Suppose we have an (unfiltered) NDIR analyzer with 100\% CO$_{2}$ in the Luft detector cells, and some acetylene enters the sample cell.  How will this instrument respond?
\item{} Could filter cells be used to sensitize an NDIR analyzer to acetylene gas with carbon dioxide as the interferent?
\item{} Sketch the absorption spectra for two gases (one gas of interest, and one interferent gas) where filtering would not work to make the NDIR analyzer selective to the gas of interest.
\item{} Referring to a filter-cell NDIR analyzer illustration, what would happen if the filter gas were to leak out of the cell on the sample side?
\item{} Referring to a filter-cell NDIR analyzer illustration, what would happen if the filter gas were to leak out of the cell on the reference side?
\end{itemize}






\vfil \eject

\noindent
{\bf Summary Quiz:}

Identify the purpose of gas-filled ``filter'' cells in a Luft-type NDIR gas analyzer:

\begin{itemize}
\item{} To sensitize the analyzer to that one species of gas more than all others
\vskip 5pt
\item{} To effectively block light wavelengths absorbed by interfering species
\vskip 5pt
\item{} To minimize ``span'' calibration drift due to the source light weakening
\vskip 5pt
\item{} To minimize the effects of mechanical vibration on the analyzer's signal
\vskip 5pt
\item{} To minimize ``zero'' calibration drift due to the source light weakening
\end{itemize}



%INDEX% Reading assignment: Lessons In Industrial Instrumentation, Analytical (nondispersive spectroscopy)

%(END_NOTES)


