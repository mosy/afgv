
%(BEGIN_QUESTION)
% Copyright 2009, Tony R. Kuphaldt, released under the Creative Commons Attribution License (v 1.0)
% This means you may do almost anything with this work of mine, so long as you give me proper credit

Read and outline the ``Fluid Power Systems'' section of the ``Discrete Control Elements'' chapter in your {\it Lessons In Industrial Instrumentation} textbook.  Note the page numbers where important illustrations, photographs, equations, tables, and other relevant details are found.  Prepare to thoughtfully discuss with your instructor and classmates the concepts and examples explored in this reading.

\underbar{file i04178}
%(END_QUESTION)





%(BEGIN_ANSWER)


%(END_ANSWER)





%(BEGIN_NOTES)

Gas (pneumatic) and liquid (hydraulic) systems convey power using pressurized fluids.  Most operate as discrete (on/off) rather than variable (throttling).  Fluid power diagram symbols are unique.

\vskip 10pt

Spool valve ``box'' symbols show directions of fluid flow between ports.  Interpret these symbols with only one box ``active'' at any time, the boxes shifting from side to side to line up with the ports.  Spool valves always drawn in their ``normal'' (resting) positions.

\vskip 10pt

Shunt pressure regulator controls hydraulic fluid system pressure by bleeding excess pump discharge to the sump.  This is necessary because hydraulic pumps are positive displacement, which mean they {\it must} move fluid somewhere when turned.  If not for the shunt regulator, the pump would ``lock up'' and refuse to turn when no fluid is directed to a load.

\vskip 10pt

Variable-displacement pumps are another way to regulate fluid pressure (without a relief valve).  The pump's displacement per rotation is made variable, to only pump as much volume as is needed to maintain a steady pressure.  This saves energy.

\vskip 10pt

Hydraulic fluid must be maintained clean and cool.  Dirty and/or hot fluid wears out components.  Filters are used to separate particulate matter from hydraulic fluid, and large hydraulic systems use coolers to remove heat from the fluid.

\vskip 10pt

Spool valves use a metal spool fitting tightly within a metal cylinder to direct fluid to different ports in different positions.  The ``spool'' is the narrow portion while the ``land'' is the wider portion which blocks access to the ports.

\vskip 10pt

Fluid power systems tend to be inefficient (wasteful of energy), but have many other advantages including:

\item{} No overloading due to heavy loads
\item{} Cannot produce sparks
\item{} May be operated in wet environments
\end{itemize}

Fluid power systems need to be kept clean.  This means fluid and filter changes for hydraulic, and air filters/dryers for pneumatic.  Two-chamber air dryers use a ``dessicant'' solid to absorb water vapor, one chamber absorbing while the other is ``regenerating'' (drying out the dessicant).  The dryness of compressed air is measured as``dewpoint'': the temperature at which the entrained moisture will condense to form droplets.  The drier the air, the lower the dewpoint temperature.  Pressure Dew Point (PDP) is the dew temperature at system pressure, which is always a greater value than the regular dewpoint which assumes atmospheric pressure.

\vskip 10pt

Water traps found on pressure regulators allow captured water to be drained periodically.  Air taps coming off a main header line should always come off the top of the line, not the bottom, so water condensing inside the line can drain off to some safe location downhill and not be drawn to the instrument(s).  Instrument air lines should be made of non-corroding metals (no iron pipe!).









\vskip 20pt \vbox{\hrule \hbox{\strut \vrule{} {\bf Suggestions for Socratic discussion} \vrule} \hrule}

\begin{itemize}
\item{} Explain how to interpret spool valve symbols in fluid power diagrams.
\item{} Explain how hydraulic fluid pressure is regulated in both shunt regulator and variable-displacement pump systems. 
\item{} Explain why fluids must be kept clean in fluid power systems, and how this is done for both hydraulic and pneumatic systems.
\item{} Identify advantages/disadvantages of fluid-power systems compared to electric motor systems.
\item{} Explain what {\it dewpoint} is, how it is controlled, and why it is important.
\end{itemize}

%INDEX% Reading assignment: Lessons In Industrial Instrumentation, Fluid power systems

%(END_NOTES)


