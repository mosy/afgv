
%(BEGIN_QUESTION)
% Copyright 2012, Tony R. Kuphaldt, released under the Creative Commons Attribution License (v 1.0)
% This means you may do almost anything with this work of mine, so long as you give me proper credit

Read the ``Siemens model 61 Booster Relay'' installation and service manual (document SD61, revision 7; originally a publication of the Moore Products company), and answer the following questions:

\vskip 10pt

Identify how such a ``volume-boosting'' relay is used in conjunction with a pneumatically-actuated control valve.

\vskip 10pt

Examine the cut-away drawing for the model 61F relay (page 7) and explain how it functions.  Describe a ``thought experiment'' where the signal pressure increases, and the relay produces a matching output pressure.

\vskip 10pt

The model 610F has a capability that the other models do not.  Identify what this capability is, by examining the respective drawings (pages 6-8).

\vskip 20pt \vbox{\hrule \hbox{\strut \vrule{} {\bf Suggestions for Socratic discussion} \vrule} \hrule}

\begin{itemize}
\item{} Why would anyone use one of these booster relays in a control valve system?
\item{} Devise a ``thought experiment'' to explain the operation of this pneumatic relay
\item{} The model 610F relay has a special capability -- what is that unique capability?
\end{itemize}

\underbar{file i04181}
%(END_QUESTION)





%(BEGIN_ANSWER)

\noindent
{\bf Partial answer:}

\vskip 10pt

The model 610F has a bias adjustment screw, whereas the other relay models are strictly 1:1 (output pressure = input pressure).

%(END_ANSWER)





%(BEGIN_NOTES)

This is a 1:1 pressure repeater, used to provide a volume gain to a pneumatic signal for faster valve action.  I/P signal goes to booster input, while booster output goes to valve actuator.

\vskip 10pt

The model 610F (shown on page 8) has an {\it adjustable zero}, whereas the other models of Siemens relays are non-adjustable.  The table shown on page 1 also reveals this capability.

%INDEX% Reading assignment: Siemens model 61 volume booster manual

%(END_NOTES)


