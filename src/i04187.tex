
%(BEGIN_QUESTION)
% Copyright 2009, Tony R. Kuphaldt, released under the Creative Commons Attribution License (v 1.0)
% This means you may do almost anything with this work of mine, so long as you give me proper credit

Read and outline the ``Rotary-Stem Valves'' section of the ``Control Valves'' chapter in your {\it Lessons In Industrial Instrumentation} textbook.  Note the page numbers where important illustrations, photographs, equations, tables, and other relevant details are found.  Prepare to thoughtfully discuss with your instructor and classmates the concepts and examples explored in this reading.


\underbar{file i04187}
%(END_QUESTION)




%(BEGIN_ANSWER)


%(END_ANSWER)





%(BEGIN_NOTES)

Ball, butterfly, and disk.  Flow path is almost unobstructed in wide-open position.

\vskip 10pt

BALL VALVE: ball with through-hole rotates.  May be full (simple) or segmented (curved edge on ball for more precise throttling).

\vskip 10pt

BUTTERFLY VALVE: like a stovepipe damper.

\vskip 10pt

DISK VALVE: like an offset butterfly valve, the disk ``camming'' into the seat for tighter shut-off.




\vskip 20pt \vbox{\hrule \hbox{\strut \vrule{} {\bf Suggestions for Socratic discussion} \vrule} \hrule}

\begin{itemize}
\item{} Describe the design differences between various types of rotary control valves.  In particular, how does each design achieve tight shut-off, and how does each design work to throttle (restrict) the flow?
\end{itemize}




%INDEX% Reading assignment: Lessons In Industrial Instrumentation, Control Valves (rotary-stem valves)

%(END_NOTES)


