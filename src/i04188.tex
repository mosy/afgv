
%(BEGIN_QUESTION)
% Copyright 2010, Tony R. Kuphaldt, released under the Creative Commons Attribution License (v 1.0)
% This means you may do almost anything with this work of mine, so long as you give me proper credit

Read selected portions of the Fisher ``Vee-Ball V150, V200, and V300 Rotary Control Valves'' product bulletin (document 51.3:Vee-Ball June 2010) and answer the following questions:

\vskip 10pt

Page 6 of this flier shows illustrations of rotary ball valve bodies.  Examine these illustration, then describe the construction of the ``ball'' used to throttle fluid flow.

\vskip 10pt

Page 7 of this flier shows a photograph of a ``Micro-Notch'' ball which differs in construction from the normal ``ball'' trim used in the Vee-Ball series of rotary valves.  Examine this photograph and describe how this notched ball throttles fluid flow.

\vskip 10pt

Page 5 contains illustrations of multiple {\it seal} designs used to provide tight shut-off when the ball valve is in the fully closed position.  Explain how each seal functions, based on the illustrations.  

\vskip 10pt

Which way should flow go through the ball valve to maximize sealing when the ball is in the full-off position?  Hint: look at the design of each sealing element (page 5), determining how the pressure drop should be oriented to maintain maximum force of the sealing element against the ball face.

\vskip 20pt \vbox{\hrule \hbox{\strut \vrule{} {\bf Suggestions for Socratic discussion} \vrule} \hrule}

\begin{itemize}
\item{} The most troublesome portion of this question for many students is interpreting the {\it seal} illustrations shown on page 5.  One suggestion is to flip between pages to compare the zoomed-in seal illustrations with the entire ball valve illustration shown elsewhere.  This is something you will often do when reading technical documents: flip back and forth between different pages, comparing two or more graphic illustrations to piece them together as one whole image in your mind.  Apply this reading technique to the problem of interpreting the seal design on Fisher Vee-Ball valves, and then explain how those seals function to prevent process fluid leakage past the ball.
\end{itemize}

\underbar{file i04188}
%(END_QUESTION)




%(BEGIN_ANSWER)


%(END_ANSWER)





%(BEGIN_NOTES)

``Ball'' in these valves is only a partial sphere, not a full ball.  Notch in ball edge forms a throttline orifice with seat (page 6).

\vskip 10pt

``Micro-notch'' ball on page 7 is more of a full ball (sphere) shape.  An angled notch forms the throttling orifice.

\vskip 10pt

Seat designs:

\item{} Metal washer held against ball by spring seal
\item{} ``HD'' seal uses long ring held against ball by spring washers
\item{} ``TCM'' seal made out of composite material, not metal
\item{} {\bf Direction of flow for tightest shut-off should press seal toward ball face with DP across valve}
\end{itemize}






%INDEX% Reading assignment: Fisher Vee-Ball rotary control valve product flier

%(END_NOTES)


