
%(BEGIN_QUESTION)
% Copyright 2009, Tony R. Kuphaldt, released under the Creative Commons Attribution License (v 1.0)
% This means you may do almost anything with this work of mine, so long as you give me proper credit

Read selected portions of the Fisher ``Rotary Valve Selection Guide'' (document 40:002 D102550X012) and answer the following questions:

\vskip 10pt

This guide shows several different styles of rotary control valve.  Identify and describe at least four of these styles, explaining how the shapes of their trim differ from one another.

\underbar{file i04189}
%(END_QUESTION)




%(BEGIN_ANSWER)


%(END_ANSWER)





%(BEGIN_NOTES)

\begin{itemize}
\item{} Control-Disk (disk) valve
\item{} Vee-Ball (segmented ball) valve
\item{} High-Performance butterfly valve
\item{} POSI-SEAL High-Performance butterfly valve (bidirectional, bubble-tight seal)
\item{} Cryogenic butterfly valve (suitable for ultra-cold applications)
\item{} Pipeline (full-ball) valve
\item{} Eccentric plug valve (useful for erosive or coking fluids)
\item{} Low-noise, low-cavitation (butterfly) valve (has ``fingers'' around disk edges)
\end{itemize}








\vskip 20pt \vbox{\hrule \hbox{\strut \vrule{} {\bf Suggestions for Socratic discussion} \vrule} \hrule}

\begin{itemize}
\item{} For each of the rotary valve designs cited by students, explain how each one is able to achieve shut-off, and also how each one is able to throttle (restrict) flow.
\end{itemize}

%INDEX% Reading assignment: Fisher Vee-Ball rotary valve selection guide

%(END_NOTES)


