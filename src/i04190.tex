
%(BEGIN_QUESTION)
% Copyright 2009, Tony R. Kuphaldt, released under the Creative Commons Attribution License (v 1.0)
% This means you may do almost anything with this work of mine, so long as you give me proper credit

Read and outline the ``Pneumatic Actuators'' subsection of the ``Control Valve Actuators'' section of the ``Control Valves'' chapter in your {\it Lessons In Industrial Instrumentation} textbook.  Note the page numbers where important illustrations, photographs, equations, tables, and other relevant details are found.  Prepare to thoughtfully discuss with your instructor and classmates the concepts and examples explored in this reading.

\underbar{file i04190}
%(END_QUESTION)




%(BEGIN_ANSWER)


%(END_ANSWER)





%(BEGIN_NOTES)

Pneumatic actuators use air pressure pressing against a diaphragm or a piston to provide actuating force.  Air pressure comes from a pneumatic controller or an I/P converter.  ``Hand jack'' may be installed on valve to manually override its position.  Air line between vibrating valve and stationary support often has ``vibration loop'' to allow flexing.

\vskip 10pt

Block and bypass valves used to route flow around a control valve for emergency or maintenance.

\vskip 10pt

Diaphgram actuators limited in pressure and stroke length.  Piston actuators can take a lot more pressure and have a virtually unlimited stroke length.

\vskip 10pt

Rack-and-pinion piston actuators use dual pistons to generate a rotational motion to move ball or butterfly type valves.

\vskip 10pt

Piston actuators generally exhibit more friction than diaphragm actuators, because there is rubbing action between the piston ring(s) and cylinder wall, whereas with a diaphragm the rubber material simply flexes and rolls.




\vskip 20pt \vbox{\hrule \hbox{\strut \vrule{} {\bf Suggestions for Socratic discussion} \vrule} \hrule}

\begin{itemize}
\item{} What is a ``hand jack'' for an actuator, and when might one be used?
\item{} Suppose the spring inside a pneumatic valve actuator is replaced with another that is stiffer.  How will this affect the operation of the actuator?
\item{} Suppose the vent port on a pneumatic valve actuator becomes plugged so it cannot pass any air.  How will this affect the operation of the actuator?
\item{} Suppose some solid debris enters the inside of a rack/pinion actuator mechanism, perhaps as impurities introduced via the compressed air line.  How might this debris affect the operation of the actuator if it encounters one of the gear teeth?
\item{} Explain why the actuator shown on the NASA valve is so large.
\item{} Explain how a rack-and-pinion actuator works.
\item{} What advantange(s) do diaphragm actuators have over piston?
\end{itemize}








\vfil \eject

\noindent
{\bf Summary Quiz:}

An advantage enjoyed by pneumatic piston actuators over pneumatic diaphragm actuators is:

\begin{itemize}
\item{} Less internal friction than diaphragms
\vskip 5pt 
\item{} Less compressed air consumption
\vskip 5pt 
\item{} Greater force and greater stroke length
\vskip 5pt 
\item{} Longer service life (fewer moving parts)
\vskip 5pt 
\item{} Greater precision in positioning
\vskip 5pt 
\item{} May operate on nitrogen, not just air
\end{itemize}



%INDEX% Reading assignment: Lessons In Industrial Instrumentation, Control Valves (pneumatic actuators)

%(END_NOTES)


