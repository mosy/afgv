
%(BEGIN_QUESTION)
% Copyright 2009, Tony R. Kuphaldt, released under the Creative Commons Attribution License (v 1.0)
% This means you may do almost anything with this work of mine, so long as you give me proper credit

Read and outline the ``Hydraulic Actuators'' and ``Hand (Manual) Actuators'' subsections of the ``Control Valve Actuators'' section of the ``Control Valves'' chapter in your {\it Lessons In Industrial Instrumentation} textbook.  Note the page numbers where important illustrations, photographs, equations, tables, and other relevant details are found.  Prepare to thoughtfully discuss with your instructor and classmates the concepts and examples explored in this reading.

\underbar{file i04191}
%(END_QUESTION)




%(BEGIN_ANSWER)


%(END_ANSWER)





%(BEGIN_NOTES)

Liquid pressure exerted against a piston to produce force.  Hydraulic actuators are easy to ``lock'' into place because the liquid is incompressible.  The hydraulic tubing generally cannot span as long a distance as pneumatic tubing, so the pump(s) need to be close to the valve.

\vskip 10pt

There is no natural energy storage in a hydraulic system as there is for pneumatic, because liquids don't compress the way gases do.

\vskip 10pt

Hand actuators generally take the form of a wheel that moves the valve when turned.  The position of the valve's stem (if it is a sliding stem type of valve) indicates valve position.  Rotary hand valves often have a pointer that tells you what position the valve is in.



\vskip 20pt \vbox{\hrule \hbox{\strut \vrule{} {\bf Suggestions for Socratic discussion} \vrule} \hrule}

\begin{itemize}
\item{} Identify some advantages hydraulic actuating systems enjoy over pneumatic actuators.
\item{} Identify some disadvantages hydraulic actuating systems exhibit compared to pneumatic actuators.
\item{} Typically, the tubing in a hydraulic control system is kept as short and as simple as possible.  Why not have really long, complicated hydraulic tube systems between pump and valve actuator?
\item{} Explain how the large hydraulic ball valve shown in the book is actuated.
\item{} {\bf Explain why the actuator for this large hydraulic ball valve can be so small compared to the valve body itself, whereas with the pneumatically-actuated NASA valve the actuator had to be so much larger than the valve body}.
\item{} What is a ``rising-stem'' valve (hand actuator)?
\end{itemize}




%INDEX% Reading assignment: Lessons In Industrial Instrumentation, Control Valves (pneumatic actuators)

%(END_NOTES)


