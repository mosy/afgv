
%(BEGIN_QUESTION)
% Copyright 2009, Tony R. Kuphaldt, released under the Creative Commons Attribution License (v 1.0)
% This means you may do almost anything with this work of mine, so long as you give me proper credit

Examine the fluid power diagram on page 2 of the Bettis ``Self-Contained Hydraulic Actuator'' service manual (document I-0019) and answer the following questions:

\vskip 10pt

Explain how the hand pump builds hydraulic fluid pressure to actuate the process valve (``line valve''), and how that valve's piston actuator is protected against excessive pressure from pumping the hand pump too many times.

\vskip 10pt

Four spool-type hydraulic valves must be in the proper positions in order to hold pressure on the actuator of the process ``line'' valve.  Identify the necessary positions for each of these spool valves, and also the way in which each one is actuated (e.g. solenoid, hand, fluid pressure).

\vskip 10pt

Explain how you could use the ``Optional Isolation Test Valve'' shown below the pilot to test this system, verifying that the line valve will go to its ``fail'' position under conditions of low process pressure and also under conditions of high process pressure.

\underbar{file i04200}
%(END_QUESTION)




%(BEGIN_ANSWER)


%(END_ANSWER)





%(BEGIN_NOTES)

The High-Pressure relief valve (part \#10) protects the line valve's actuator against excessive hand pump pressure.

\vskip 10pt

The Pressure Pilot (part \#14) must have just the right amount of process line pressure to hold it in its middle position -- either too much or too little will drain low-pressure fluid back to the reservoir.  The normally-closed Solenoid Valve (part \#20) must be maintained in an energized state, or else it too will drain low-pressure fluid to the reservoir.  The Selector Valve (part \#16) must be in the ``Auto'' position or else it too will drain low-pressure fluid to the reservoir.  The Reset Valve (part \#23) is what actually drains high-pressure fluid to the reservoir (through the Pressure Regulator, part \#12) when it loses low-pressure fluid from its pressure actuator.

\vskip 10pt

To test the system for action under conditions of low line pressure, simply shut the process block valve and open the test valve to bleed process pressure from the Pressure Pilot (part \#14).  To test the system for action under conditions of excessive line pressure, shut the process block valve and apply a high test pressure through the test valve to the Pilot by using a pump or other fluid pressure source of appropriate pressure capability.

%INDEX% Reading assignment: Bettis ``Self-Contained Hydraulic Actuator'' service manual

%(END_NOTES)


