
%(BEGIN_QUESTION)
% Copyright 2009, Tony R. Kuphaldt, released under the Creative Commons Attribution License (v 1.0)
% This means you may do almost anything with this work of mine, so long as you give me proper credit

Read and outline the ``Valve Failure Mode'' section of the ``Control Valves'' chapter in your {\it Lessons In Industrial Instrumentation} textbook.  Note the page numbers where important illustrations, photographs, equations, tables, and other relevant details are found.  Prepare to thoughtfully discuss with your instructor and classmates the concepts and examples explored in this reading.

\underbar{file i04201}
%(END_QUESTION)




%(BEGIN_ANSWER)


%(END_ANSWER)





%(BEGIN_NOTES)

Hydraulic and pneumatic valves may be equipped with springs to provide a default position in the event of a fluid power failure.

\vskip 10pt

\noindent 
VALVE ACTUATOR:

\item{} Direct = Air pressure pushes stem down
\item{} Reverse = Air pressure pulls stem up
\end{itemize}


\noindent 
VALVE BODY:

\item{} Direct = Valve opens as stem goes up
\item{} Reverse = Valve closes as stem goes up
\end{itemize}

Direct body + Reverse actuator = ATO (Air To Open) / Fail Closed

\vskip 5pt

Direct body + Direct actuator = ATC (Air To Close) / Fail Open

\vskip 5pt

Reverse body + Reverse actuator = ATC (Air To Close) / Fail Open 

\vskip 5pt

Reverse body + Direct actuator = ATO (Air To Open) / Fail Closed

\vskip 5pt

Reverse-acting valve bodies are rare.

\vskip 10pt

\noindent 
Failure modes drawn with arrows assuming direct-acting bodies (pull stem out to open). 

\item{} Fail Open (FO)
\item{} Fail Closed (FC)
\item{} Fail Locked (FL) -- fluid trapped in actuator to lock in position
\item{} Fail Last / Drift -- piston actuator with no spring -- plug forces cause drift
\end{itemize}

Proper valve failure mode should be made on the basis of process safety.  Case in point: engine coolant valve should fail open (air-to-close).  Other instrument actions in the loop should be chosen to result in safest action in the event of an input signal failure.

\begin{itemize}
\item{} I/P = direct action (loss of mA signal results in loss of air pressure to valve)
\item{} Controller = direct (loss of PV signal results in output signal going down)
\item{} Transmitter = reverse (hotter engine --> less mA -- valve opens more)
\end{itemize}

If valve is air-to-close, controller output display should be reverse-indicating (4 mA reads as wide open) for the operator's sake.


\vskip 20pt \vbox{\hrule \hbox{\strut \vrule{} {\bf Suggestions for Socratic discussion} \vrule} \hrule}

\begin{itemize}
\item{} Identify multiple combinations of actuator and valve body to make an {\it air-to-open} control valve.
\item{} Identify multiple combinations of actuator and valve body to make an {\it air-to-close} control valve.
\item{} Identify some of the different failure modes available for control valves, and match them to particular actuator types (e.g. pneumatic diaphragm, pneumatic piston, hydraulic, electric).
\item{} Identify realistic process applications which would require a specific control valve failure mode (e.g. one where the control valve needs to fail-closed).
\item{} Select appropriate instrument actions (e.g. direct- vs. reverse-acting transmitter, ATC vs. ATO valve) for any particular process in order to maximize the safety of that process.  {\it Hint: refer to processes shown in question file i00788 if you need specific examples.}
\item{} Explain the difference between a controller that is {\it reverse-acting} versus one that is {\it reverse-indicating}.  Is it possible for a controller to be both of these things at once?  Where might we choose to use a reverse-indicating controller?
\end{itemize}









\vfil \eject

\noindent
{\bf Prep Quiz:}

Explain the rationale for selecting a particular ``failure mode'' for a control valve in a process application.  In other words, what criteria/considerations should one consider when choosing the failure mode for a control valve and {\it why} should we consider those factors?  Feel free to cite a specific process example if it helps your explanation.


%INDEX% Reading assignment: Lessons In Industrial Instrumentation, Control Valves (failure modes)

%(END_NOTES)


