
%(BEGIN_QUESTION)
% Copyright 2014, Tony R. Kuphaldt, released under the Creative Commons Attribution License (v 1.0)
% This means you may do almost anything with this work of mine, so long as you give me proper credit

Read selected sections of the US Chemical Safety and Hazard Investigation Board's report (2002-04-I-MO) of the 2002 chlorine release in Festus, Missouri, and answer the following questions.

\vskip 10pt

Pages 29 through 31 of the report give a description of the incident, at a facility where liquid chlorine is transferred from rail cars to smaller containers for sale to industries using chlorine in their processes.  A transfer hose from the rail car to the chlorine process piping burst, enabling liquid chlorine to freely escape.  Explain why the ``Emergency Shutdown'' (ESD) system failed to stop the flow of chlorine as it should have.

\vskip 10pt

Pages 23-28 and 41-44 of the report detail the ESD valves and chlorine piping system.  Examine the diagram on page 42 to determine the leak path of the liquid chlorine.  Note that one of the ESD valves actually managed to completely shut, and another closed off 40\%.  All the other ESD valves remained fully open.

\vskip 10pt

Were these ESD valves fail-open (air-to-close) or fail-closed (air-to-open)?

\vskip 10pt

Examine the ``fault tree'' shown on page 96, and explain the meaning of the {\tt AND} and {\tt OR} gate symbols.  Note the directions of the arrows, which at first appear backwards to anyone familiar with logic gate analysis.  Why do you suppose the arrows are drawn this way?

\vskip 20pt \vbox{\hrule \hbox{\strut \vrule{} {\bf Suggestions for Socratic discussion} \vrule} \hrule}

\begin{itemize}
\item{} What preventive measures could personnel have done at this facility to avoid this accident?  Specifically, how could they have thoroughly tested the emergency shutdown system to ensure it was ready to operate as designed if needed?
\item{} For those who have studied Safety Instrumented Systems (SIS), determine the safety {\it MooN} rating of each ESD valve pair isolating the chlorine header from the tank car (e.g. valves 1 and 5 ; valves 3 and 4) as shown on page 42.  Also determine the safety {\it MooN} rating of the ESD valve system as a whole (i.e. Xoo5) in it ability to prevent releases such as the one that happened.
\item{} Following the ``fault tree'' diagram, identify any factors that could have prevented the accident despite the primary component failure described in the report.
\end{itemize}

\underbar{file i04205}
%(END_QUESTION)





%(BEGIN_ANSWER)


%(END_ANSWER)





%(BEGIN_NOTES)

A hose burst (page 13) venting chlorine to atmosphere.  Multiple emergency shutdown valves (ESD) failed to shut as designed.  ESD valves 2, 4, and 5 were completely open, while ESD valve 3 was 60\% open (page 41).  Only one ESD valve (\#1) did actually close, but it was not enough to stop the leak.  

The reason these ESD valves failed to actuate as they were supposed to is because of corrosion inside the valve trim.  Valves \#3 and \#4 required 40 ft-lbs of torque to actuate (due to the corrosion build-up) while the pneumatic actuators only provided 5 ft-lbs of torque (page 43).  The valve bodies themselves were made of Monel metal, which is highly resistant to attack from corrosion.  The source of the material build-up was determined to be originating from the ``air pad'' and tank car piping, carried to the valves by flowing chlorine gas (page 44).

\vskip 10pt

Chlorine gas reacts with iron to form a protective passivation layer of ferric chloride by the following chemical reaction:

$$3\hbox{Cl}_2 + 2\hbox{Fe} \to 2\hbox{FeCl}_3$$

While this passivation layer normally protects the steel piping from further attack, the presence of any water in the chlorine piping system will produce acids that dissolve this ferric chloride passivation layer and allow continued corrosion.  The chlorine-water reaction producing hydrochloric acid and hypochlorous acid is as follows:

$$\hbox{Cl}_2 + \hbox{H}_2\hbox{O} \to \hbox{HCl} + \hbox{HOCl}$$

A photograph showing this corrosion appears in figure 25 on page 55.

\vskip 10pt

\noindent
Chlorine leak path:

\item{} Air into railcar through ESD valve \#2
\item{} Chlorine out of railcar through ESD \#3 (60\% open) and into chlorine header
\item{} Chlorine out of header through ESD \#5 and into burst hose
\end{itemize}

All ESD valves were air-to-open (fail closed) according to footnote \#4 on page 27.

\vskip 10pt

AND and OR gate symbols refer to conditions needing to be met in order for the accident to occur.  Arrows appear to be backward because the fault tree begins with the accident and proceeds backward to potential causes.

\vskip 10pt

The Mechanical Integrity (MI) program at this facility was not sufficiently detailed to catch problems such as this.  The Center for Chemical Process Safety (CCPS) has guidelines on the testing of ESD valves, saying functional checks need to go all the way to verifying actuation of the final control element: something not done regularly at this facility.  (page 53)

\vskip 10pt

The {\it MooN} rating of each ESD valve pair is 1oo2 for simply isolating the car from the chlorine header.  The {\it MooN} rating of all valves is 5oo5, assuming a self-pressurized chlorine header.

%INDEX% Reading assignment: USCSB report on the 2002 chlorine leak in Festus, Missouri

%(END_NOTES)


