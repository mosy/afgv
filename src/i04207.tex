
%(BEGIN_QUESTION)
% Copyright 2009, Tony R. Kuphaldt, released under the Creative Commons Attribution License (v 1.0)
% This means you may do almost anything with this work of mine, so long as you give me proper credit

Read and outline the introduction to the ``Valve Positioners'' section of the ``Control Valves'' chapter in your {\it Lessons In Industrial Instrumentation} textbook.  Note the page numbers where important illustrations, photographs, equations, tables, and other relevant details are found.  Prepare to thoughtfully discuss with your instructor and classmates the concepts and examples explored in this reading.

\underbar{file i04207}
%(END_QUESTION)





%(BEGIN_ANSWER)


%(END_ANSWER)





%(BEGIN_NOTES)

Forces other than spring or actuator can force valve to undesired positions (e.g. packing friction, process fluid forces on valve plug).  A positioner corrects this by applying enough pressure to the actuator to overcome these other forces.  This makes the valve well-behaved (i.e. responsive to the control signal).  Positioners act as mini-controllers, with stem position being PV, command signal being SP, and air to the actuator being Output.

\vskip 10pt

All valve positioners employ some means of feedback to sense the position of the valve stem.  This may take the form of a linkage, or something more exotic like a magnetic Hall Effect sensor.

\vskip 10pt

Positioners also tend to act as volume boosters, sourcing and venting air faster than an I/P could do alone.

\vskip 10pt

Positioners help ensure tight shutoff by saturating to 0 PSI applied pressure at 0\% signal, not the low bench set value (e.g. 3 PSI) to ensure full seat load (force between the plug and the seat) in the closed position.  An I/P at 4 mA still applies air pressure to a valve ; a positioner at 4 mA is fully saturated at 0 PSI (full spring pressure on seat).

\vskip 10pt

Positioners are absolutely required on double-acting piston actuators with no spring, and also on motor (electric) actuators.  These actuators require power to move in either direction.




\vskip 20pt \vbox{\hrule \hbox{\strut \vrule{} {\bf Suggestions for Socratic discussion} \vrule} \hrule}

\begin{itemize}
\item{} Explain why positioners are needed in some control valve applications.
\item{} Explain how a positioner helps a control valve achieve better shut-off than if it were powered directly by the output of an I/P converter.
\item{} Explain how positioners sense the position of the control valve stem, and why this feature is critically important.
\item{} Suppose the feedback linkage on a valve positioner were to slip so that the positioner ``thought'' the valve was 15\% further open than it actually was.  How would this mis-adjustment affect the actual calibration of the valve in relation to the input (4-20 mA) control signal?
\item{} Suppose a technician tries to make an adjustment to the feedback linkage on a valve positioner with the valve in-service, the positioner ``powered'' with full air supply, and the analog control signal at 50\% (12 mA).  Explain what bad things could happen, and why this might be a safety concern for both the technician and the operators trying to control the process.
\item{} Suppose the feedback linkage connecting a positioner to its valve stem falls off.  How will the control valve behave with this fault?  Explain your answer in detail.
\item{} If the bench-set pressure range on a control valve were adjusted from the normal 3-15 PSI range to a new range of 4-16 PSI, would this affect the positioning of the valve with a positioner installed?  In other words, would 4 mA still put the valve at fully shut and 20 mA still put the valve at fully open?
\item{} Does the presence of a positioner on a control valve make the bench-set of that control valve irrelevant?  Explain why or why not.
\end{itemize}








\vfil \eject

\noindent
{\bf Prep Quiz:}

One of the advantages of using a valve {\it positioner} on a control valve is that it improves the valve's ability to tightly shut off flow.  Explain in detail why this is.








\vfil \eject

\noindent
{\bf Prep Quiz:}

Explain in detail what would happen (and why it would happen) if the {\it feedback linkage} were to become disconnected between a positioner and the valve stem on a functioning control valve.










\vfil \eject

\noindent
{\bf Prep Quiz:}

The purpose of a control valve {\it positioner} is to:

\begin{itemize}
\item{} Keep the valve salespeople in business
\vskip 5pt
\item{} Establish the proper fail-safe mode 
\vskip 5pt
\item{} Eliminate valve stem friction
\vskip 5pt
\item{} Reduce fugitive emissions
\vskip 5pt
\item{} Ensure more precise actuation 
\vskip 5pt
\item{} Prevent cavitation in the valve 
\end{itemize}



\vfil \eject

\noindent
{\bf Prep Quiz:}

The purpose of a control valve {\it positioner} is to:

\begin{itemize}
\item{} Set the proper fail-safe mode 
\vskip 5pt
\item{} Occupy space on the warehouse shelf
\vskip 5pt
\item{} Ensure precise stem actuation 
\vskip 5pt
\item{} Eliminate fugitive emissions
\vskip 5pt
\item{} Hold the valve steady during installation
\vskip 5pt
\item{} Prevent choked flow 
\end{itemize}


%INDEX% Reading assignment: Lessons In Industrial Instrumentation, valve positioners

%(END_NOTES)


