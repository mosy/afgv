
%(BEGIN_QUESTION)
% Copyright 2015, Tony R. Kuphaldt, released under the Creative Commons Attribution License (v 1.0)
% This means you may do almost anything with this work of mine, so long as you give me proper credit

Read selected portions of the Rotork ``AWT range'' actuator manual (document E320E, issue 10/02), and answer the following questions:

\vskip 10pt

Page 3 shows a ``cutaway view'' of the actuator mechanism.  Examine this illustration and identify the locations of the {\it worm gear}, {\it electric motor}, and {\it limit switch assembly}.

\vskip 10pt

Page 5 discusses the switch features of this actuator, including limit (travel) and torque switches.  Identify the types of valves recommended for ``Torque'' versus ``Limit'' seating.

\vskip 10pt

Page 12 shows a schematic diagram for this electric actuator.  Examine this diagram, then explain how the electric motor's direction of rotation is controlled (i.e. what switching occurs to reverse the motor's rotation).

\vskip 10pt

Using the schematic diagram on page 12 as a guide, identify potential faults that could cause the valve to refuse to open (assume C1 is the ``open'' contactor and C2 is the ``close'' contactor), and how you could confirm each one of these potential faults using a multimeter:

\begin{itemize}
\item{} Identify at least one specific problem in the three-phase power contacts to the motor
\item{} Identify at least one specific problem in the contactor coil(s)
\item{} Identify at least one specific problem in the main control circuit board
\end{itemize}

\underbar{file i04224}
%(END_QUESTION)





%(BEGIN_ANSWER)

\noindent
{\bf Partial answer:}

\vskip 10pt

Potential faults causing the valve not to open:

\begin{itemize}
\item{} C1 contact 5-6 failed open; C1 contact 3-4 failed open; C1 contact 1-2 failed open
\item{} C1 coil failed open
\item{} TRIAC output on circuit board to coil C1 failed open (terminal FL9)
\end{itemize}

%(END_ANSWER)





%(BEGIN_NOTES)

Wedge gate and globe valves are examples of valve types best suited for ``torque seating,'' while slide gate and ball valves are examples of valve types best suited for ``limit seating.''

\vskip 10pt

This motor's direction is controlled by a pair of reversing contactors, C1 and C2.




\vskip 20pt \vbox{\hrule \hbox{\strut \vrule{} {\bf Virtual Troubleshooting} \vrule} \hrule}

This question is a good candidate for a ``Virtual Troubleshooting'' exercise.  Presenting the diagram to students, you first imagine in your own mind a particular fault in the system.  Then, you present one or more symptoms of that fault (something noticeable by an operator or other user of the system).  Students then propose various diagnostic tests to perform on this system to identify the nature and location of the fault, as though they were technicians trying to troubleshoot the problem.  Your job is to tell them what the result(s) would be for each of the proposed diagnostic tests, documenting those results where all the students can see.

During and after the exercise, it is good to ask students follow-up questions such as:

\begin{itemize}
\item{} What does the result of the last diagnostic test tell you about the fault?
\item{} Suppose the results of the last diagnostic test were different.  What then would that result tell you about the fault?
\item{} Is the last diagnostic test the best one we could do?
\item{} What would be the ideal order of tests, to diagnose the problem in as few steps as possible?
\end{itemize}


%INDEX% Reading assignment: Limitorque L120 electric valve actuator manual

%(END_NOTES)


