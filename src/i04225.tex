
%(BEGIN_QUESTION)
% Copyright 2009, Tony R. Kuphaldt, released under the Creative Commons Attribution License (v 1.0)
% This means you may do almost anything with this work of mine, so long as you give me proper credit

Read and outline the ``Physics of Energy Dissipation in a Turbulent Fluid Stream'' subsection of the ``Control Valve Sizing'' section of the ``Control Valves'' chapter in your {\it Lessons In Industrial Instrumentation} textbook.  Note the page numbers where important illustrations, photographs, equations, tables, and other relevant details are found.  Prepare to thoughtfully discuss with your instructor and classmates the concepts and examples explored in this reading.

\underbar{file i04225}
%(END_QUESTION)





%(BEGIN_ANSWER)


%(END_ANSWER)





%(BEGIN_NOTES)

Control valves restrict the flow of fluid through them like resistors restrict the flow of electricity.  The mechanism of this restriction is {\it fluid turbulence}.  Turbulence is directly proportional to velocity, and since kinetic energy is proportional to the square of velocity, turbulent restriction is likewise proportional to the square of velocity.  The ``restrictiveness'' of a turbulent flow element may be quantified as such:

$$\hbox{``}R\hbox{''} = {\sqrt{P_1 - P_2} \over Q}$$

Control valve flow capacity ($C_v$) is analagous to electrical {\it conductance}: the reciprocal of resistance.  That is, the greater the value of $C_v$, the more flow it will pass for any given pressure drop.

$$Q = C_v \sqrt{P_1 - P_2 \over G_f}$$

A $C_v$ of 1 means the valve will pass 1 GPM of water through it at 1 PSID pressure drop.  $K_v$ is the metric equivalent: how many cubic meters per hour of water will pass at 1 bar pressure drop.  Both of these factors incorporate unit-conversion coefficients.

\vskip 10pt

Control valve sizing is complex enough to recommend computer software analysis for real-world applications.










\vskip 20pt \vbox{\hrule \hbox{\strut \vrule{} {\bf Suggestions for Socratic discussion} \vrule} \hrule}

\begin{itemize}
\item{} Qualitatively analyze any of the valve flow formulae showing area ($A$), explaining what happens if the area of the valve's port {\it increases} (assuming all other factors remain constant).
\item{} Identify some physical features of a control valve that might affect its $k$ value (turbulent energy dissipation factor).
\item{} Qualitatively analyze any of the valve flow formulae showing area ($A$), explaining what happens if the area of the valve's port {\it decreases by a factor of 2} (assuming all other factors remain constant).
\item{} Qualitatively analyze any of the valve flow formulae shown, explaining what happens to the pressure drop across a valve if the rate of flow through it {\it increases} (assuming all other factors remain constant).
\item{} Can a control valve be used as a flow-measurement device, representing flow through it in the form of a differential pressure?  In other words, is it possible to measure the pressure drop across a control valve and mathematically derive a rate of flow through the valve from that information alone?
\end{itemize}

%INDEX% Reading assignment: Lessons In Industrial Instrumentation, control valves (energy dissipation)

%(END_NOTES)


