
%(BEGIN_QUESTION)
% Copyright 2009, Tony R. Kuphaldt, released under the Creative Commons Attribution License (v 1.0)
% This means you may do almost anything with this work of mine, so long as you give me proper credit

Read and outline the ``Relative Flow Capacity'' subsection of the ``Control Valve Sizing'' section of the ``Control Valves'' chapter in your {\it Lessons In Industrial Instrumentation} textbook.  Note the page numbers where important illustrations, photographs, equations, tables, and other relevant details are found.  Prepare to thoughtfully discuss with your instructor and classmates the concepts and examples explored in this reading.

\underbar{file i04227}
%(END_QUESTION)





%(BEGIN_ANSWER)


%(END_ANSWER)





%(BEGIN_NOTES)

Some valve designs (e.g. globe) generate more fluid turbulence than others; thus, they have smaller $C_v$ values than other designs for the same pipe size.  Butterfly valves, for example, have much higher flow capacities than globe valves for the same pipe size because they present the flowing fluid with much less turbulence (fewer twists and turns in the flow path).

\vskip 10pt

Relative flow capacity ($C_d$) relates $C_d$ to pipe diameter for any given type of control valve design.  The basic concept is that $C_v$ should scale linearly with pipe area (with the square of pipe diameter), and so we should be able to classify any particular design of valve's flowing ability as a fixed proportion to pipe size:

$$C_d = {C_v \over d^2}$$

 
% No blank lines allowed between lines of an \halign structure!
% I use comments (%) instead, so Tex doesn't choke.

$$\vbox{\offinterlineskip
\halign{\strut
\vrule \quad\hfil # \ \hfil & 
\vrule \quad\hfil # \ \hfil \vrule \cr
\noalign{\hrule}
%
% First row
{\bf Valve design type} & $C_d$ \cr
%
\noalign{\hrule}
%
% Another row
Single-port globe valve, ported plug & 9.5 \cr
%
\noalign{\hrule}
%
% Another row
Single-port globe valve, contoured plug & 11 \cr
%
\noalign{\hrule}
%
% Another row
Single-port globe valve, characterized cage & 15 \cr
%
\noalign{\hrule}
%
% Another row
Double-port globe valve, ported plug & 12.5 \cr
%
\noalign{\hrule}
%
% Another row
Double-port globe valve, contoured plug & 13 \cr
%
\noalign{\hrule}
%
% Another row
Rotary ball valve, segmented & 25 \cr
%
\noalign{\hrule}
%
% Another row
Rotary ball valve, standard port (diameter $\approx$ 0.8$d$) & 30 \cr
%
\noalign{\hrule}
%
% Another row
Rotary butterfly valve, 60$^{o}$, no offset seat & 17.5 \cr
%
\noalign{\hrule}
%
% Another row
Rotary butterfly valve, 90$^{o}$, offset seat & 29 \cr
%
\noalign{\hrule}
%
% Another row
Rotary butterfly valve, 90$^{o}$, no offset seat & 40 \cr
%
\noalign{\hrule}
} % End of \halign 
}$$ % End of \vbox

To calculate approximate flow capacity ($C_v$) multiply the $C_d$ factor for that valve's type by the square of its pipe diameter in inches:

$$C_v \approx d^2 C_d$$









\vskip 20pt \vbox{\hrule \hbox{\strut \vrule{} {\bf Suggestions for Socratic discussion} \vrule} \hrule}

\begin{itemize}
\item{} Explain what $C_d$ means, in your own words.
\item{} Explain the practical significance of $C_d$.  In other words, why should we care?
\item{} Explain why different styles of control valve have different $C_d$ ratings, and why this is important when selecting a control valve for a particular application.
\item{} Why do double-ported globe valves exhibit greater $C_d$ values than single-ported globe valves?
\end{itemize}

%INDEX% Reading assignment: Lessons In Industrial Instrumentation, control valves (relative flow capacity)

%(END_NOTES)


