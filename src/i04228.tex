
%(BEGIN_QUESTION)
% Copyright 2014, Tony R. Kuphaldt, released under the Creative Commons Attribution License (v 1.0)
% This means you may do almost anything with this work of mine, so long as you give me proper credit

Stephanie is considering converting her car from an internal combustion engine to purely electric drive (i.e. an electric motor and batteries).  In order to begin planning this conversion, she must determine how much horsepower is required of the electric motor to sustain freeway driving speeds.  Unfortunately Stephanie has no way to directly measure the power output of her car's engine in order to experimentally determine horsepower under different driving conditions.

\vskip 10pt

One day Stephanie is driving home from school and notices some stopped cars well ahead of her.  She shifts her car's transmission into neutral and coasts a little while before braking to a stop behind the last stopped car.  As her car is coasting, Stephanie has an epiphany: the rate at which her car is slowing down should be indicative of how much power is required to maintain its speed when the transmission is in gear, because the forces slowing the car down as it coasts are the same forces the engine must overcome to maintain cruising speed.  

\vskip 10pt

With this principle in mind, Stephanie tries an experiment.  At the next available opportunity while driving at 70 miles per hour, she intentionally shifts her car's transmission into neutral and times how long it takes to coast down to 65 miles per hour, after which she re-engages the transmission and speeds back up to regular highway velocity.  Here is the data from her experiment, plus other data she knows about her car:  

\begin{itemize}
\item{} {\bf Time so slow down from 70 MPH to 65 MPH} = 4.4 seconds
\item{} {\bf Fuel tank level} = 75\% full
\item{} {\bf Ambient air temperature} = 69 $^{o}$F
\item{} {\bf Engine temperature} = 205 $^{o}$F
\item{} {\bf Curb weight} = 3170 lbs
\item{} {\bf Engine speed at 70 MPH} = 2300 RPM
\item{} {\bf Music playing on the stereo} = {\it Autobahn} by Kraftwerk
\end{itemize}

Based on this data, calculate the approximate horsepower required to maintain Stephanie's car at a speed of 70 MPH.  Stephanie will use this calculated value to help choose an appropriate motor size for her electric conversion.

\vfil

\underbar{file i04228}
\eject
%(END_QUESTION)





%(BEGIN_ANSWER)

This is a graded question -- no answers or hints given!

%(END_ANSWER)





%(BEGIN_NOTES)

The principle in Stephanie's epiphany is that her car possesses a certain amount of kinetic energy at 70 MPH and a lesser amount of kinetic energy at 65 MPH.  The difference between these two kinetic energy values is the amount of energy the car's engine would have normally had output to overcome losses (e.g. rolling friction, air friction, hill climbing) at cruising speed.  This energy difference, divided by the amount of time it took to lose that kinetic energy, yields an average value for {\it power} representing the power draw of those losses at cruising speed.

\vskip 10pt

First, calculating kinetic energy at 70 MPH with velocity and mass in the appropriate English units:

$$v = \left(70 \hbox{ miles} \over \hbox{hour} \right) \left(5280 \hbox{ ft} \over 1 \hbox{ mile}\right) \left(1 \hbox{ hour} \over 3600 \hbox{ seconds}\right) = 102.67 \hbox{ feet per second}$$

$$m = {W \over g} = {3170 \hbox{ lb} \over 32.174 \hbox{ ft/s}^2} = 98.53 \hbox{ slugs}$$

$$E_k = {1 \over 2} mv^2 = {1 \over 2} (98.53 \hbox{ slugs}) (102.67 \hbox{ ft/s})^2 = 519257.9 \hbox{ ft-lb}$$

\vskip 10pt

Next, calculating kinetic energy at 65 MPH with velocity and mass in the appropriate English units:

$$v = \left(65 \hbox{ miles} \over \hbox{hour} \right) \left(5280 \hbox{ ft} \over 1 \hbox{ mile}\right) \left(1 \hbox{ hour} \over 3600 \hbox{ seconds}\right) = 95.33 \hbox{ feet per second}$$

$$E_k = {1 \over 2} mv^2 = {1 \over 2} (98.53 \hbox{ slugs}) (95.33 \hbox{ ft/s})^2 = 447727.5 \hbox{ ft-lb}$$

\vskip 10pt

Now we may calculate the amount of kinetic energy lost during the coasting period from 70 MPH to 65 MPH:

$$E_{loss} = 519257.9 \hbox{ ft-lb} - 447727.5 \hbox{ ft-lb} = 71530.4 \hbox{ ft-lb}$$

\vskip 10pt

Average power is calculated by dividing the change in energy by the time period over which that change occurred:

$$P = {E_{loss} \over t} = {71530.4 \hbox{ ft-lb} \over 4.4 \hbox{ s}} = 16256.9 \hbox{ ft-lb/s}$$

\vskip 10pt

Now all that is left is a units conversion into horsepower:

$$\left(16256.9 \hbox{ ft-lb/s} \over 1 \right) \left(1 \hbox{ HP} \over 550 \hbox{ ft-lb/s}\right) = 29.56 \hbox{ HP}$$

\vskip 10pt

Therefore, Stephanie's electric conversion will require an electric motor capable of outputting about 30 horsepower continuously to maintain highway cruising speed under those same conditions the experiment was conducted (e.g. wind speed, road grade, etc.).

%INDEX% Physics, energy, work, power

%(END_NOTES)




