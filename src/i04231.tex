
%(BEGIN_QUESTION)
% Copyright 2009, Tony R. Kuphaldt, released under the Creative Commons Attribution License (v 1.0)
% This means you may do almost anything with this work of mine, so long as you give me proper credit

Read and outline the introduction to the ``Variable-Speed Motor Controls'' chapter as well as the ``Metering Pumps'' section of that same chapter in your {\it Lessons In Industrial Instrumentation} textbook.  Note the page numbers where important illustrations, photographs, equations, tables, and other relevant details are found.  Prepare to thoughtfully discuss with your instructor and classmates the concepts and examples explored in this reading.

\vskip 20pt \vbox{\hrule \hbox{\strut \vrule{} {\bf Suggestions for Socratic discussion} \vrule} \hrule}

\begin{itemize}
\item{} When programming an electronic motor ``drive'' to control the speed of that motor, it is vitally important for that drive to be properly configured with the motor's nameplate data (e.g. rated voltage, rated current, maximum speed, etc.).  Explain why this is so important to get right, and what could happen if these ``base'' parameters are misconfigured.
\end{itemize}

\underbar{file i04231}
%(END_QUESTION)





%(BEGIN_ANSWER)


%(END_ANSWER)





%(BEGIN_NOTES)

Controlling fluid flow by varying speed of pump (instead of throttling with a valve) saves energy!

\vskip 10pt

VFD technology able to make AC induction motors do almost any task.  AC motors much more reliable than DC motors due to lack of wearing parts.

\vskip 10pt

Variable-speed motor controls can also ``soft-start'' machines (instead of always applying full power).  Some can do regenerative braking to recover kinetic energy.

\vskip 10pt

\noindent
Disadvantanges of variable-speed motor control as compared to control valves:

\item{} Valves act faster
\item{} Valves provide tight shutoff of flow
\item{} Valves dissipate energy (which is sometimes needed in a system)
\item{} Split-ranging can be difficult to achieve with motors
\item{} Limited fail-safe options with motors
\item{} Motor might not exist!
\end{itemize}










\vfil \eject

\noindent
{\bf Prep Quiz:}

Identify which of the following statements is {\it not} true about using variable-speed electric motors as final control elements (instead of control valves):

\begin{itemize}
\item{} Variable-speed motors can provide the useful feature of ``soft-start''
\vskip 5pt 
\item{} Variable-speed pumps and fans use less total energy than constant-speed pumps and fans
\vskip 5pt 
\item{} Variable-speed motors may extend the operating life of the machines they turn
\vskip 5pt 
\item{} Variable-speed motors reduce the operating vibration of the machines they turn
\vskip 5pt 
\item{} Variable-speed pumps and fans are able to start and stop flows faster than valves
\vskip 5pt 
\item{} Variable-speed motors are often able to ``recover'' energy when slowing down
\end{itemize}


%INDEX% Reading assignment: Lessons In Industrial Instrumentation, physics (rotational motion)

%(END_NOTES)


