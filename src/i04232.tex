
%(BEGIN_QUESTION)
% Copyright 2009, Tony R. Kuphaldt, released under the Creative Commons Attribution License (v 1.0)
% This means you may do almost anything with this work of mine, so long as you give me proper credit

Read and outline the ``Flanged Pipe Fittings'' subsection of the ``Pipe and Pipe Fittings'' section of the ``Instrument Connections'' chapter in your {\it Lessons In Industrial Instrumentation} textbook.  Note the page numbers where important illustrations, photographs, equations, tables, and other relevant details are found.  Prepare to thoughtfully discuss with your instructor and classmates the concepts and examples explored in this reading.

\underbar{file i04232}
%(END_QUESTION)





%(BEGIN_ANSWER)


%(END_ANSWER)





%(BEGIN_NOTES)

Flanged pipe connections are used to bolt pipe sections together.  A gasket is sandwiched between the flange faces to prevent leakage of process fluid.  Install the gasket by first installing lower half of bolts, insert gasket, and then install the upper half of bolts.

\vskip 10pt

``Raised-Face'' (RF) flanges use concentric grooves on the face of each flange to form a pressure-tight seal against the crushable gasket material.  ``Ring Type Joint'' (RTJ) flanges use a metal ring set inside a groove in the flange to seal high pressures.

\vskip 10pt

Pressure classes defined by ANSI originally referred to the pressure ratings of pipe flanges carrying steam at saturated temperature.  With modern metallurgy, those same dimensional standards are now able to withstand much greater pressures.  The integrity of a flanged joint will be maintained only if both flanges are of the same pressure class.

\vskip 10pt

Be sure to {\it cross-torque} flange bolts to ensure even application of force on the flange gasket.  Use torque wrenches to ensure proper torque application, or use {\it Rotabolt} fasteners which directly sense bolt stretch.  When loosening flange bolts, be sure to loosen those bolts facing away from you, so that in the event there is a rupture of stored pressure in the line, the fluid sprays away from you and not toward you!

\vskip 10pt

Flange {\it blinds} are plates of metal sandwiched between flanges to seal off flanged joints for semi-permanent blockage of pipes.  Spectacle blinds may be rotated for ``open'' or ``blinded'' service.










\vskip 20pt \vbox{\hrule \hbox{\strut \vrule{} {\bf Suggestions for Socratic discussion} \vrule} \hrule}

\begin{itemize}
\item{} Explain how a flanged pipe joint seals against fluid leakage.
\item{} Describe the difference between an {\it RF} flanged joint and an {\it RTJ} flanged joint.
\item{} Explain the importance of {\it cross-torquing} flange bolts when tightening them.
\item{} Explain the purpose and construction of a flange {\it blind}.
\item{} Explain how to identify the status of a {\it spectacle blind} by its outward appearance.
\item{} Explain why it is safest to loosen the flange bolts on the {\it opposite} side of the pipe from where you are standing when disassembling pipe flanges that were previously in service.
\end{itemize}


















\vfil \eject

\noindent
{\bf Summary Quiz:}

A control valve with a 300\# ANSI ``pressure class'' rating means:

\begin{itemize}
\item{} It has a greater pressure rating than a 600\# class valve 
\vskip 5pt 
\item{} The average fluid pressure should not exceed 300 PSIA
\vskip 5pt 
\item{} It has the same pressure rating as a 150\# class valve 
\vskip 5pt 
\item{} The valve will burst if the fluid pressure exceeds 300 PSIG
\vskip 5pt 
\item{} The control valve assembly weighs approximately 300 pounds
\vskip 5pt 
\item{} It has a greater pressure rating than a 150\# class valve
\end{itemize}


%INDEX% Reading assignment: Lessons In Industrial Instrumentation, pipe flanges

%(END_NOTES)


