
%(BEGIN_QUESTION)
% Copyright 2009, Tony R. Kuphaldt, released under the Creative Commons Attribution License (v 1.0)
% This means you may do almost anything with this work of mine, so long as you give me proper credit

Read and outline the ``Gas Valve Sizing'' subsection of the ``Control Valve Sizing'' section of the ``Control Valves'' chapter in your {\it Lessons In Industrial Instrumentation} textbook.  Note the page numbers where important illustrations, photographs, equations, tables, and other relevant details are found.  Prepare to thoughtfully discuss with your instructor and classmates the concepts and examples explored in this reading.

\underbar{file i04233}
%(END_QUESTION)





%(BEGIN_ANSWER)


%(END_ANSWER)





%(BEGIN_NOTES)

Gases are compressible (unlike liquids) which means the density ($\rho$) of a gas will change as it flows through a control valve and experiences different pressures.  As such, valve size calculations for gas service differ substantially from liquid valve size calculations.  A relatively simple gas valve size formula appears here:

$$Q = 963 \> C_v \sqrt{{\Delta P (P_1 + P_2)} \over {G_g T}}$$

Note the inclusion of a pressure sum term ($P_1 + P_2$) as well as absolute temperature ($T$), in order to account for changes in gas density.

\vskip 10pt

If the gas flow happens to approach or exceed the speed of sound, different formulae must be used.  Once again, valve sizing software is recommended for real-world applications, to account for the multitude of variables one encounters in gas control valve applications.






\vskip 20pt \vbox{\hrule \hbox{\strut \vrule{} {\bf Suggestions for Socratic discussion} \vrule} \hrule}

\begin{itemize}
\item{} Explain why gas valve sizing is more complex than liquid valve sizing.
\item{} Identify the meaning of the term ``critical flow'' with reference to control valves.
\end{itemize}







\vfil \eject

\noindent
{\bf Prep Quiz:}

The reason gas valve sizing calculations are more complicated than liquid valve sizing calculations is because:

\begin{itemize}
\item{} Liquids typically run at lower temperatures than gases
\vskip 5pt
\item{} The speed of sound is slower through gases than liquids
\vskip 5pt
\item{} Packing designs must be completely changed for gas service
\vskip 5pt
\item{} It is more difficult to achieve tight shut-off with gases
\vskip 5pt
\item{} Gases are compressible but liquids are incompressible
\vskip 5pt
\item{} Cavitation can occur with gases but not with liquids
\end{itemize}

%INDEX% Reading assignment: Lessons In Industrial Instrumentation, gas valve sizing

%(END_NOTES)


