
%(BEGIN_QUESTION)
% Copyright 2009, Tony R. Kuphaldt, released under the Creative Commons Attribution License (v 1.0)
% This means you may do almost anything with this work of mine, so long as you give me proper credit

Read and outline the ``Self-Operated Valves'' subsection of the ``Control Valve Actuators'' section of the ``Control Valves'' chapter in your {\it Lessons In Industrial Instrumentation} textbook.  Note the page numbers where important illustrations, photographs, equations, tables, and other relevant details are found.  Prepare to thoughtfully discuss with your instructor and classmates the concepts and examples explored in this reading.

\underbar{file i04234}
%(END_QUESTION)





%(BEGIN_ANSWER)


%(END_ANSWER)





%(BEGIN_NOTES)

Self-operated control valves use the process fluid to actuate the valve mechanism.  Self-operated pressure regulators are a good example of this kind of valve.

\vskip 10pt

Pressure regulators may be spring-loaded or pilot loaded.  Spring-loaded regulators use a compressed spring to establish the setpoint pressure of the regulating mechanism.  Pilot-loaded regulators use gas pressure from another (smaller, spring-loaded) regulator to establish the setpoint pressure.  The advantage of pilot loading is that the setpoint of the main regulator may be changed at will, and even from a distance.

\vskip 10pt

Irrigation control solenoid valves are another example of a pilot-operated valve, where a small electric solenoid valve controls water pressure applied to the actuator of a larger valve.  This minimizes how large the electric solenoid coil needs to be in order to do the job, by leveraging the pressure of the controlled water to actuate the main valve mechanism.

\vskip 10pt

Pressure relief valves (PRVs) and pressure safety valves (PSVs) are another example of self-operated valves, where the pressure of the process fluid provides the actuating force.




\vskip 20pt \vbox{\hrule \hbox{\strut \vrule{} {\bf Suggestions for Socratic discussion} \vrule} \hrule}

\begin{itemize}
\item{} Explain how the design and function of a {\it pilot-loaded} pressure regulator differs from that of a simple {\it spring-loaded} pressure regulator.
\item{} Explain how to take a {\it pilot-loaded} pressure regulator and somehow connect it to an electronic loop controller so that it functions as the final control element in a feedback loop.
\item{} Describe the purpose of a {\it PRV} or {\it PSV} in a process.
\item{} Perform a ``thought experiment'' where the self-operated, spring-loaded pressure regulator maintains a regulated output pressure despite increased gas demand.
\item{} Perform a ``thought experiment'' where the self-operated, spring-loaded pressure regulator maintains a regulated output pressure despite increased gas supply pressure.
\item{} Perform a ``thought experiment'' where the self-operated, pilot-loaded pressure regulator is faced with an increased loading pressure from an operator adjusting the pilot.
\item{} Identify the effect(s) of the feedback tube plugging up in the self-operated, spring-loaded pressure regulator.
\item{} Identify the effect(s) of the spring breaking in the self-operated, spring-loaded pressure regulator.
\item{} Identify the effect(s) of the pilot vent plugging up in the self-operated, pilot-loaded pressure regulator.
\end{itemize}

%INDEX% Reading assignment: Lessons In Industrial Instrumentation, gas valve sizing

%(END_NOTES)


