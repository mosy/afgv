
%(BEGIN_QUESTION)
% Copyright 2009, Tony R. Kuphaldt, released under the Creative Commons Attribution License (v 1.0)
% This means you may do almost anything with this work of mine, so long as you give me proper credit

Read and outline the ``Inherent versus Installed Characteristics'' subsection of the ``Control Valve Characterization'' section of the ``Control Valves'' chapter in your {\it Lessons In Industrial Instrumentation} textbook.  Note the page numbers where important illustrations, photographs, equations, tables, and other relevant details are found.  Prepare to thoughtfully discuss with your instructor and classmates the concepts and examples explored in this reading.

\underbar{file i04235}
%(END_QUESTION)





%(BEGIN_ANSWER)


%(END_ANSWER)





%(BEGIN_NOTES)

If the differential pressure across a control valve ($P_1 - P_2$) is held constant, the rate of fluid flow through that valve will be directly proportional to its $C_v$ at any stem position.  If a control valve is designed to exhibit a $C_v$ that varies linearly with stem position (a ``linear'' valve), that valve will exhibit a linear flow rate to stem position relationship for a constant pressure drop.

\vskip 10pt

Most real-life control valve applications, however, do {\it not} provide a constant differential pressure across the valve.  Generally, the pressure drop across a valve decreases as flow rate increases.  The result of this is that the flow rate exhibited by a control valve in a variable-pressure application ``droops'' away from the inherently linear trend one would expect to see in a constant-pressure scenario.  The more the valve is opened, the less pressure it has to work with, and the flow -- although greater than it was at lower stem positions -- will not be as great as it would be at the same stem position with a constant pressure drop.

\vskip 10pt

The behavior of a control valve under constant-pressure conditions is called its {\it inherent characteristic}, while the behavior of a control valve under real-life (variable-pressure) conditions is called its {\it installed characteristic}.






\vskip 20pt \vbox{\hrule \hbox{\strut \vrule{} {\bf Suggestions for Socratic discussion} \vrule} \hrule}

\begin{itemize}
\item{} Explain why control valves do not behave the same when installed in a real process as they do in a laboratory where the differential pressure may be maintained at a constant value.
\item{} Does the installed characteristic of a control valve result in more or less flow through the valve than its inherent characteristic?  Explain your answer in detail.
\item{} How can a process be modified to provide the most constant pressure drop possible to a control valve?
\end{itemize}







\vfil \eject

\noindent
{\bf Prep Quiz:}

The main reason control valves typically do not exhibit the same response in a working process as they do when tested in a laboratory is because:

\begin{itemize}
\item{} The valve packing must be adjusted tighter in a real process
\vskip 5pt 
\item{} The pressure drop across the valve varies in a real process
\vskip 5pt 
\item{} Laboratories use relatively low fluid pressures for testing
\vskip 5pt 
\item{} Technicians usually don't install control valves correctly
\vskip 5pt 
\item{} The process fluid density it usually not the same as water
\vskip 5pt 
\item{} Laboratories use higher instrument air pressures than industry
\end{itemize}



\vfil \eject

\noindent
{\bf Prep Quiz:}

The main reason control valves typically do not exhibit the same response in a working process as they do when tested in a laboratory is because:

\begin{itemize}
\item{} Real process fluids are colder than laboratory test fluids
\vskip 5pt 
\item{} Laboratory tests do not produce cavitation as in real life
\vskip 5pt 
\item{} The pressure drop across the valve varies in a real process 
\vskip 5pt 
\item{} Technicians usually don't install control valves correctly 
\vskip 5pt 
\item{} Flow in a laboratory setting is usually laminar, not turbulent
\vskip 5pt 
\item{} Real process fluids are hotter than laboratory test fluids 
\end{itemize}


%INDEX% Reading assignment: Lessons In Industrial Instrumentation, control valve characterization (inherent, vs. installed characteristics)

%(END_NOTES)


