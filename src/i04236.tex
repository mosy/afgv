
%(BEGIN_QUESTION)
% Copyright 2009, Tony R. Kuphaldt, released under the Creative Commons Attribution License (v 1.0)
% This means you may do almost anything with this work of mine, so long as you give me proper credit

Read and outline the ``Control Valve Performance with Constant Pressure'' subsection of the ``Control Valve Characterization'' section of the ``Control Valves'' chapter in your {\it Lessons In Industrial Instrumentation} textbook.  Note the page numbers where important illustrations, photographs, equations, tables, and other relevant details are found.  Prepare to thoughtfully discuss with your instructor and classmates the concepts and examples explored in this reading.

\underbar{file i04236}
%(END_QUESTION)





%(BEGIN_ANSWER)


%(END_ANSWER)





%(BEGIN_NOTES)

In a scenario where a control valve sees a constant differential pressure (such as when it drains water at the base of a dam where the upstream water level is constant and the downstream is open air), the flow rate through that valve will be proportional to its $C_v$.  If the valve has a linear inherent characteristic, the flow rate will be proportional to stem position.

\vskip 10pt

We may plot a set of curves for any control valve showing flow rate versus pressure drop at various stem positions (one curve plotted per stem position).  These curves will be non-linear because of the physics of the pressure-versus-turbulent-flow relationship, not anything peculiar about the valve itself.  A ``load line'' showing differential pressure available to the valve is superimposed on these characteristic curves, and the points of intersection between those curves and the load line tells us how much flow will go through the valve at each of those stem positions.







\vskip 20pt \vbox{\hrule \hbox{\strut \vrule{} {\bf Suggestions for Socratic discussion} \vrule} \hrule}

\begin{itemize}
\item{} Explain how we may use the characteristic curves for a control valve, and the load line (or load curve) for a piping system, to predict flow rates for various valve stem positions.
\item{} Explain how the load line will be affected if the water height on the upstream side of the dam increases.
\item{} Explain how the load line will be affected if the water height on the upstream side of the dam decreases.
\item{} Explain how the characteristic curves will be affected if the valve's trim is reduced to yield a decreased $C_{v(max)}$.
\item{} Explain how the characteristic curves will be affected if the valve's trim is expanded to yield an increased $C_{v(max)}$.
\item{} {\it Advanced:} Explain how the characteristic curves and load line will be affected if the fluid's density substantially changes.
\end{itemize}

%INDEX% Reading assignment: Lessons In Industrial Instrumentation, control valve characterization (performance with constant pressure)

%(END_NOTES)


