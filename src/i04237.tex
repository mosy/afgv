
%(BEGIN_QUESTION)
% Copyright 2009, Tony R. Kuphaldt, released under the Creative Commons Attribution License (v 1.0)
% This means you may do almost anything with this work of mine, so long as you give me proper credit

Read and outline the ``Control Valve Performance with Varying Pressure'' subsection of the ``Control Valve Characterization'' section of the ``Control Valves'' chapter in your {\it Lessons In Industrial Instrumentation} textbook.  Note the page numbers where important illustrations, photographs, equations, tables, and other relevant details are found.  Prepare to thoughtfully discuss with your instructor and classmates the concepts and examples explored in this reading.

\underbar{file i04237}
%(END_QUESTION)





%(BEGIN_ANSWER)


%(END_ANSWER)





%(BEGIN_NOTES)

If the piping system is such that the differential pressure available to the control valve decreases with increased flow rate, the valve will not flow as much as it would with constant differential pressure (especially at the top end).  This effect is manifested on a graph of characteristic curves by a curved (rather than linear and vertical) {\it load line}.  As the load line curves to yield less pressure with greater flow, the intersection points between the valve's characteristic curves and the load line are such that the valve's flow rate ``droops'' at greater stem positions, yielding diminishing returns for ever-larger stem positions.

\vskip 10pt

Frictional losses in piping is just one source of flow-varying pressure at a control valve.  Pressure dropped by heat exchangers, filters, and even the discharge pressure of pumps all vary with flow rate, conspiring to make the load line anything but straight and vertical.

\vskip 10pt

The ``drooping'' installed characteristic of a control valve not only results in diminished maximum flow, but also in varying sensitivity throughout the valve's controlling range.  Close to the fully shut position, when pressure drop across the valve is high, the valve is sensitive (yielding large changes in flow for small changes in stem position).  When close to fully open, when pressure drop across the valve is low, the valve is insensitive (yielding smaller changes in flow for the same changes in stem position).  This variable sensitivity can result in odd control behavior in a closed-loop system, with the loop being less stable at low flow rates and being sluggish (less responsive) at high flow rates.






\vskip 20pt \vbox{\hrule \hbox{\strut \vrule{} {\bf Suggestions for Socratic discussion} \vrule} \hrule}

\begin{itemize}
\item{} Explain how we may use the characteristic curves for a control valve, and the load line (or load curve) for a piping system, to predict flow rates for various valve stem positions.
\item{} Explain how the load line will be affected if the water height on the upstream side of the dam increases.
\item{} Explain how the load line will be affected if the water height on the upstream side of the dam decreases.
\item{} Explain how the characteristic curves will be affected if the valve's trim is reduced to yield a decreased $C_{v(max)}$.
\item{} Explain how the characteristic curves will be affected if the valve's trim is expanded to yield an increased $C_{v(max)}$.
\item{} {\it Advanced:} Explain how the characteristic curves and load line will be affected if the fluid's density substantially changes.
\end{itemize}


%INDEX% Reading assignment: Lessons In Industrial Instrumentation, control valve characterization (performance with varying pressure)

%(END_NOTES)


