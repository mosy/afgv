
%(BEGIN_QUESTION)
% Copyright 2010, Tony R. Kuphaldt, released under the Creative Commons Attribution License (v 1.0)
% This means you may do almost anything with this work of mine, so long as you give me proper credit

Read selected portions of the National Transportation Safety Board's Pipeline Accident Report, {\it Pipeline Rupture and Subsequent Fire in Bellingham, Washington, June 10, 1999} (Document NTSB/PAR-02/02 ; PB2002-916502) and answer the following questions:

\vskip 10pt

Pages 31-32 of the report describe the pressure relief system at the Bayview terminal.  Identify the ANSI pressure class of the piping components at the Bayview terminal, and their maximum pressure rating at ambient temperature.  Also discuss what a ``piston stop'' is and how it affects the operation of a control valve.

\vskip 10pt

Pages 33-37 describe the pressure relief valve RV-1919, which was found to be malfunctioning at the time of the accident.  Describe how this pressure relief valve is supposed to work, and what the accident investigators found when they examined the valve.  Additional information on this valve is found on pages 45-46 and 69 of the report.

\vskip 10pt

Explain the difference(s) between a {\it pressure relief valve} such as RV-1919 and a {\it pressure control valve} such as a Fisher E-body sliding-stem globe valve.  What purpose does each valve serve in a process system?  Can one be used in place of the other if necessary?  

\vskip 20pt \vbox{\hrule \hbox{\strut \vrule{} {\bf Suggestions for Socratic discussion} \vrule} \hrule}

\begin{itemize}
\item{} What role did instrument technicians play in this accident?
\item{} Explain how the instrument technician could have done a better job than he did, possibly averting the disaster.
\item{} Explain why the actual pressure rating of the 300\# class pipes and components greatly exceeds 300 PSI.  What, exactly, does the ``300\#'' rating mean?
\item{} Explain why you think the results obtained testing the relief valve pilot mechanisms differed from the ``dynamic'' (flowing) test results of the relief valves.
\item{} Explain how and why {\it X-ray photography} was used as part of the forensic analysis following this accident.
\end{itemize}

\underbar{file i04240}
%(END_QUESTION)





%(BEGIN_ANSWER)


%(END_ANSWER)





%(BEGIN_NOTES)

ANSI 300\# pressure class components rated at 740 PSIG inside the Bayview terminal facility (all outlet valves).  ANSI 600\# pressure class for the inlet control valves (CV-1904, CV-1902, CV-1903, and CV-1907 at the ``Ferndale receiver'').  (Page 31)

A ``piston stop'' is a part of a regulating valve mechanism that may be set to prevent the valve from completely closing if set in the ``extended'' position.  Olympic Pipeline desired their outlet control valves to have extended stops (i.e. to always remain at least partially open) but that the inlet valves should be capable of stopping flow.  It was discovered after the Bayview terminal came on-line that the inlet valve stops were extended to prevent full closure but that the outlet valves could completely close, which is precisely opposite of Olympic's design specs.  Olympic, however, never remedied this state of affairs!  (Pages 31-32)

\vskip 10pt

RV-1919 was found to have a pilot mechanism with a faulty setting.  The pilot was supposed to be upgraded with high-pressure components, but instead was equipped with low-pressure (70-180 PSIG rated) components which did not allow proper adjustment to the design lift pressure of 740 PSIG (the design pressure of the ANSI 300\# components).  Failing to consult the manual for this valve, a mechanic decided to compress the spring inside the pilot until it was nearly ``solid'' to achieve a higher lifting pressure, which unfortunately made the relief valve's operation unpredictable.  A Fisher employee faxed a note to Hoffman Instruments saying how it takes more than a new spring to switch the pilot to the high-pressure range, and the Hoffman employee says he did pass this information on to an Olympic mechanic (though he did not recall who that was).

Subsequent tests on the pilot mechanism for RV-1919 showed it ``lifting'' at 440 PSIG, but X-ray photographs of the spring inside (page 46) showed it to be fully compressed, and therefore subject to wide variation in actual lift pressure during service.  Later tests on the same valve removed from the pipeline resulted in more than 800 PSI required to ``lift'' it (page 69 of the report).  Dynamic (flowing) tests performed on all of Olympic Pipeline's relief valves proved their actual lift pressures exceeding design setpoints by 8 PSIG to 130 PSIG (page 37)!

Interestingly, the operating setpoint for RV-1919 and RV-1923 (intended by the pipeling company to be 740 PSIG) exceeded the manufacturer's maximum pressure setpoint of 650 PSIG by a substantial margin.  It appears as though the pipeline company's engineers had decided to set the lift pressures of the relief valves to the maximum working pressure for the 300\# class pipes without regard for the relief valve's specifications!

\vskip 10pt

Pressure relief valves serve distinctly different purposes from pressure control valves.  The former are safety devices (not control devices), and they are always self-actuated unlike the latter.  The two valve types are in no way interchangeable with one another!


\vfil \eject

\noindent
{\bf Prep Quiz:}

Pages 31 through 32 of the National Transportation Safety Board's Pipeline Accident Report, {\it Pipeline Rupture and Subsequent Fire in Bellingham, Washington, June 10, 1999} (Document NTSB/PAR-02/02 ; PB2002-916502) documents the pressure class of the piping components at the Bayview terminal.  The ``ANSI'' pressure class rating of the Bayview terminal components were:

\begin{itemize}
\item{} ANSI 400\# for inlet valves, ANSI 150\# for outlet valves
\vskip 5pt 
\item{} ANSI 600\# for inlet valves, ANSI 300\# for outlet valves
\vskip 5pt 
\item{} ANSI 150\# for inlet valves, ANSI 600\# for outlet valves
\vskip 5pt 
\item{} ANSI 1500\# for inlet valves, ANSI 600\# for outlet valves
\vskip 5pt 
\item{} ANSI 700\# for inlet valves, ANSI 1500\# for outlet valves
\vskip 5pt 
\item{} ANSI 900\# for inlet valves, ANSI 300\# for outlet valves
\end{itemize}

%INDEX% Reading assignment: NTSB Pipeline Accident Report, Pipeline Rupture and Subsequent Fire in Bellingham

%(END_NOTES)


