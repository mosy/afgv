
%(BEGIN_QUESTION)
% Copyright 2016, Tony R. Kuphaldt, released under the Creative Commons Attribution License (v 1.0)
% This means you may do almost anything with this work of mine, so long as you give me proper credit

Read and outline the ``Valve Seat Leakage'' section of the ``Control Valves'' chapter in your {\it Lessons In Industrial Instrumentation} textbook.  Note the page numbers where important illustrations, photographs, equations, tables, and other relevant details are found.  Prepare to thoughtfully discuss with your instructor and classmates the concepts and examples explored in this reading.

\vskip 20pt \vbox{\hrule \hbox{\strut \vrule{} {\bf Further exploration . . . (optional)} \vrule} \hrule}

For more insight on this topic, consider this passage written in scientific terms as it might appear in a ``whitepaper'' document:

\vskip 10pt {\narrower \noindent \baselineskip5pt

An important parameter in the mitigation of seat leakage is that of {\it seat load}.  This is typically expressed as the quotient of seating force between the plug and the seat versus the mean seat joint circumference, usually in the range of 20 to 1000 pounds per linear inch.  Sufficient seat load causes the plug and seat to yield at their mating surfaces, forming a leak-tight interface.  Insufficient seat load fails to produce the requisite deformation for tight sealing, while excessive seat load may stress said components beyond their elastic limits.

\par} \vskip 10pt

Try interpreting and re-expressing the meaning of this passage in simpler terms.  What is it trying to say, and how does it relate to the assigned reading on valve seat leakage in your textbook?

\vskip 10pt

\underbar{file i04241}
%(END_QUESTION)





%(BEGIN_ANSWER)


%(END_ANSWER)





%(BEGIN_NOTES)

Some applications demand that a control valve be able to tightly shut off, others don't.  Shutoff class specified as a Roman numeral (I is worst, VI is tightest).  ``Bubble-tight'' shutoff refers to Class VI, where only a limited number of gas bubbles per minute are allowed to leak past the shut valve.





\vskip 20pt \vbox{\hrule \hbox{\strut \vrule{} {\bf Suggestions for Socratic discussion} \vrule} \hrule}

\begin{itemize}
\item{} Describe what a ``soft seat'' is for a control valve.
\item{} Identify a control valve type that is notoriously poor at tight shut-off, and explain why.
\item{} Examine the photograph of the valve test bench shown in the textbook, and explain the purpose of the various components.
\item{} Explain how you could check a control valve for Class VI (``bubble tight'') shut-off without a fancy test bench as shown in the textbook photograph.  In other words, devise a test that doesn't require any special equipment.
\end{itemize}

%INDEX% Reading assignment: Lessons In Industrial Instrumentation, control valve problems (seat leakage)
%INDEX% Reading assignment: "Further Exploration" (vertical text), paraphrasing from a Fisher valve monograph on seat load for control valves

%(END_NOTES)


