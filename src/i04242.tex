
%(BEGIN_QUESTION)
% Copyright 2009, Tony R. Kuphaldt, released under the Creative Commons Attribution License (v 1.0)
% This means you may do almost anything with this work of mine, so long as you give me proper credit

Read and outline the ``Mechanical Friction'' subsection of the ``Control Valve Problems'' section of the ``Control Valves'' chapter in your {\it Lessons In Industrial Instrumentation} textbook.  Note the page numbers where important illustrations, photographs, equations, tables, and other relevant details are found.  Prepare to thoughtfully discuss with your instructor and classmates the concepts and examples explored in this reading.

\underbar{file i04242}
%(END_QUESTION)





%(BEGIN_ANSWER)


%(END_ANSWER)





%(BEGIN_NOTES)

Friction between the stem and the packing inside a control valve impedes the motion of valve to some degree.  {\it Static friction} is the friction between stationary objects.  {\it Dynamic friction} is the friction between moving objects, and is always a lesser value than static friction.

\vskip 10pt

In a pneumatic valve actuator, the combination of static and dynamic friction, plus the effects of changing actuator volume on air pressure, makes it so friction often results in ``jerky;; valve motion.  This is called a ``slip-stick'' cycle, also known as ``stiction''.

\vskip 10pt

The amount of applied pressure change to a pneumatic actuator necessary to {\it reverse} the actuator's motion is proportional to {\it twice} the amount of static friction.  That is to say, the amount of pressure change needed to reverse the direction of a pneumatic actuator is usually twice as large as the amount of pressure change needed to continue the actuator's motion in the same direction after friction has halted its travel.

\vskip 10pt

Valve packing needs to be lubricated to experience a long life.  {\it Packing lubricators} are special devices installed on valve bonnets built to facilitate the injection of lubricating grease into the packing of a control valve.

\vskip 10pt

The ``valve signature'' diagnostic graph provided by a digital valve positioner is useful for measuring valve friction.  The amount of vertical gap separating the ``opening'' and ``closing'' traces on a valve signature is proportional to the amount of friction inside the valve.  Jaggedness at the closing end of the valve signature indicates a rough fit between the plug and seat.






\vskip 20pt \vbox{\hrule \hbox{\strut \vrule{} {\bf Suggestions for Socratic discussion} \vrule} \hrule}

\begin{itemize}
\item{} Explain the difference between {\it static} friction and {\it dynamic} friction, using a practical example (not the one described in the textbook!) to illustrate.
\item{} When driving in winter weather, it is inadvisable to lock up the brakes when trying to stop.  Explain why, appealing to static versus dynamic friction (between the tires and the snowy/icy ground).
\item{} Explain the ``slip-stick'' cycle for a pneumatic valve.
\item{} Explain how you could measure the amount of packing friction in a control valve (in units of pounds force) using nothing but an air pressure regulator, a compressed air supply, a collection of tubing and tube fittings, and a precision pressure gauge.
\item{} Explain how the valve signature graph would change if packing friction were increased.
\end{itemize}








\vfil \eject

\noindent
{\bf Prep Quiz:}

For a control valve, the phrase ``slip-stick'' refers to:

\begin{itemize}
\item{} The valve moving in a ``jerky'' fashion rather than smoothly
\vskip 5pt 
\item{} Liquid turning into vapor at the valve's vena contracta
\vskip 5pt 
\item{} Constant flow rate despite decreased downstream pressure 
\vskip 5pt 
\item{} Vapor bubbles condensing into liquid downstream of the valve
\vskip 5pt 
\item{} The appearance of valve trim after being corroded by fluids
\vskip 5pt 
\item{} A special tool used to lubricate a dry valve packing
\end{itemize}


%INDEX% Reading assignment: Lessons In Industrial Instrumentation, control valve problems (mechanical friction)

%(END_NOTES)


