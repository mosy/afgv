
%(BEGIN_QUESTION)
% Copyright 2009, Tony R. Kuphaldt, released under the Creative Commons Attribution License (v 1.0)
% This means you may do almost anything with this work of mine, so long as you give me proper credit

Read and outline the ``Flashing'' subsection of the ``Control Valve Problems'' section of the ``Control Valves'' chapter in your {\it Lessons In Industrial Instrumentation} textbook.  Note the page numbers where important illustrations, photographs, equations, tables, and other relevant details are found.  Prepare to thoughtfully discuss with your instructor and classmates the concepts and examples explored in this reading.

\underbar{file i04243}
%(END_QUESTION)





%(BEGIN_ANSWER)


%(END_ANSWER)





%(BEGIN_NOTES)

Fluid pressure in a control valve drops as that fluid passes through the valve trim's constriction and picks up speed.  The fluid pressure then (mostly) recovers when the fluid enters a wider spot and slows down again.  If this pressure falls below the liquid's vapor pressure, the process liquid will ``flash'' into vapor (i.e. boil).  

\vskip 10pt

Flashing ``chokes'' fluid flow through the valve, and it also erodes parts by accelerating any non-flashed liquid material droplets.  Flashing produces a hissing sound reminiscent of sand flowing through the valve.

\vskip 10pt

The {\it pressure recovery factor} ($F_L$) of a control valve expresses its total pressure drop in relation to the maximum pressure drop from inlet to {\it vena contracta} (the point of minimum fluid pressure inside the valve):

$$F_L = \sqrt{P_1 - P_2 \over P_1 - P_{vc}}$$

The greater the pressure recovery from vena contracta to downstream ($P_2$), the less the pressure recovery factor.  Valves exhibiting a high pressure recovery (i.e. a {\it low} pressure recovery factor) are most prone to flashing because their vena contracta pressures are less, all other factors being equal.

\vskip 10pt

Butterfly valves are more prone to flashing than globe valves, because globe valves dissipate their energy at multiple points, whereas butterfly valves dissipate almost all their energy at one point.  Having to dissipate all energy at a single position inside the valve means that position's $P_{vc}$ must be lower and therefore more conducive to flashing.







\vskip 20pt \vbox{\hrule \hbox{\strut \vrule{} {\bf Suggestions for Socratic discussion} \vrule} \hrule}

\begin{itemize}
\item{} Explain what ``flashing'' is and how it may be mitigated.
\item{} Interpret pressure/position diagrams for valves, explaining why pressure changes at different points.
\item{} Qualitatively analyze the pressure recovery factor formula by noting which of the three valves shown in the textbook illustration (each one with its own graph of pressure at different points in the valve body) exhibits the greatest and the least $F_L$ values.
\item{} Explain why some types of control valve have different $F_L$ values than other types of control valve.
\item{} Suppose we had a control valve that was flashing, and we wished to mitigate that flashing by changing the temperature of the process liquid.  Would we want to make the liquid hotter or colder?  Explain why.
\item{} Suppose we had a control valve that was flashing, and we wished to mitigate that flashing by changing the line pressure of the process liquid.  Would we want to operate the valve at a higher line pressure or a lower line pressure?  Explain why.
\end{itemize}








\vfil \eject

\noindent
{\bf Prep Quiz:}

Explain in detail what ``flashing'' means when it happens inside of a control valve.  


%INDEX% Reading assignment: Lessons In Industrial Instrumentation, control valve problems (flashing)

%(END_NOTES)


