
%(BEGIN_QUESTION)
% Copyright 2016, Tony R. Kuphaldt, released under the Creative Commons Attribution License (v 1.0)
% This means you may do almost anything with this work of mine, so long as you give me proper credit

Read and outline the ``Cavitation'' subsection of the ``Control Valve Problems'' section of the ``Control Valves'' chapter in your {\it Lessons In Industrial Instrumentation} textbook.  Note the page numbers where important illustrations, photographs, equations, tables, and other relevant details are found.  Prepare to thoughtfully discuss with your instructor and classmates the concepts and examples explored in this reading.

\vskip 20pt \vbox{\hrule \hbox{\strut \vrule{} {\bf Further exploration . . . (optional)} \vrule} \hrule}

For more insight on this topic, watch the video ``Challenge At Glen Canyon Dam'' chronicling the diagnosis and rectification of a major cavitation problem in one of the spillway tubes at a hydroelectric dam.  This was an example of cavitation on a massive scale!  The video may be downloaded from the {\it Socratic Instrumentation} website, located in the ``Video Resources'' section.

\underbar{file i04244}
%(END_QUESTION)





%(BEGIN_ANSWER)


%(END_ANSWER)





%(BEGIN_NOTES)

If fluid pressure inside a control valve recovers to the point where flashed vapor re-condenses back into liquid, cavitation results.  Condensing vapor bubble collapses to produce a jet of liquid with incredible pressure (+200,000 PSI), damaging valve components.

\vskip 10pt

Cavitation literally blasts pieces of metal away inside the valve.  Certain valve designs attempt to minimize cavitation damage by relocating the zone of cavitation to a place where the damage will have minimum impact to the seating surfaces.  Cavitation produces a crackling sound reminiscent of rocks flowing through the valve!  

\vskip 10pt

Cavitation accelerates corrosion by blasting away the passivation layer normally protecting the metal from further attack.  The complementary actions of cavitation and corrosion are sometimes referred to under one term {\it cavitation corrosion}.

\vskip 10pt

\noindent
To combat cavitation:

\begin{itemize}
\item{} (1) Prevent flashing in the first place (i.e. prevent $P_{vc}$ from ever becoming less than the liquid's vapor pressure)
\item{} (2) Cushion the cavitating vapor bubbles with non-condensing gas bubbles
\item{} (3) Prolong flashing (don't let pressure rise to form cavitation)
\end{itemize}

\vskip 10pt

Cavitation-control valve trim works by dropping pressure in stages rather than all at once.  This follows cavitation abatement method \#1 (above).











\vskip 20pt \vbox{\hrule \hbox{\strut \vrule{} {\bf Suggestions for Socratic discussion} \vrule} \hrule}

\begin{itemize}
\item{} Interpret pressure/position diagrams for valves, explaining why pressure changes at different points.
\item{} Explain what ``cavitation'' is and identify multiple methods of mitigation.
\item{} Explain how cavitation accelerates the process of chemical erosion in a control valve.
\item{} Explain the rationale for the downward flow direction in a Fisher Micro-Flat valve.
\item{} Can cavitation happen in a natural gas pipeline?  If so, identify the specific conditions necessary for this to occur.
\end{itemize}













\vfil \eject

\noindent
{\bf Prep Quiz:}

For a control valve, the term ``cavitation'' refers to:

\begin{itemize}
\item{} The valve moving in a ``jerky'' fashion rather than smoothly
\vskip 5pt 
\item{} Liquid turning into vapor at the valve's vena contracta
\vskip 5pt 
\item{} Constant flow rate despite decreased downstream pressure 
\vskip 5pt 
\item{} Vapor bubbles condensing into liquid downstream of the valve
\vskip 5pt 
\item{} The appearance of valve trim after being corroded by fluids
\vskip 5pt 
\item{} A special tool used to lubricate a dry valve packing
\end{itemize}

%INDEX% Reading assignment: Lessons In Industrial Instrumentation, control valve problems (cavitation)

%(END_NOTES)


