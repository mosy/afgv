
%(BEGIN_QUESTION)
% Copyright 2009, Tony R. Kuphaldt, released under the Creative Commons Attribution License (v 1.0)
% This means you may do almost anything with this work of mine, so long as you give me proper credit

Read and outline the ``Choked Flow'' and ``Noise'' subsections of the ``Control Valve Problems'' section of the ``Control Valves'' chapter in your {\it Lessons In Industrial Instrumentation} textbook.  Note the page numbers where important illustrations, photographs, equations, tables, and other relevant details are found.  Prepare to thoughtfully discuss with your instructor and classmates the concepts and examples explored in this reading.

\underbar{file i04245}
%(END_QUESTION)





%(BEGIN_ANSWER)


%(END_ANSWER)





%(BEGIN_NOTES)

{\it Choking} in a control valve is when the fluid flow rate remains steady despite decreases in downstream pressure ($P_2$).  Ideally, flow through a valve is a function of {\it differential} pressure across the valve, but this is no longer true when choking begins.

\vskip 10pt

In gas service, choking occurs when the speed of the gas through the valve reaches the speed of sound for that gas.  Pascal's principle holds that changes in fluid pressure propagate to all regions of that fluid, but this happens at the speed of sound, and so when the fluid speed equals or exceeds the speed of sound, downstream pressure changes cannot propagate fast enough backwards against the flow to have any effect at the vena contracta.

\vskip 10pt

Choking in gas service is likely to happen when $P_{vc} < {1 \over 2} P_1$.  Choking in liquid service happens whenever flashing occurs (i.e. $P_{vc} \leq P_{vapor}$).

\vskip 10pt

Gas choking applied in {\it critical velocity nozzles} designed to produce sonic flow velocities to guarantee a set flow rate of gas through.  Downstream pressure changes have no effect on flow through a sonic nozzle.

\vskip 20pt

Fluid turbulence causes noise to emanate from a control valve.  Special trim may be used to mitigate this noise, designed to shift the frequency of the noise to a range outside of human hearing.  Labyrinth-style trim mitigates noise by directing the fluid through a series of sharp turns, dissipating fluid energy without incurring high velocities.







\vskip 20pt \vbox{\hrule \hbox{\strut \vrule{} {\bf Suggestions for Socratic discussion} \vrule} \hrule}

\begin{itemize}
\item{} Define ``choked flow'' for a control valve.
\item{} Explain the mechanism of choked flow in gas services, appealing to the speed of sound through that gas.
\item{} Explain how a Fisher ``Whisper'' valve trim reduces fluid noise.
\item{} Identify the rule-of-thumb criterion for predicting choked flow in a gas valve.
\item{} If a control valve is experiencing choked flow in its wide-open position, is there any way to increase the flow rate, or is that flow rate absolutely limited?
\item{} Explain what a {\it critical velocity nozzle} is and what it is used for.
\end{itemize}












\vfil \eject

\noindent
{\bf Prep Quiz:}

For a control valve, the phrase ``choked flow'' refers to:

\begin{itemize}
\item{} The valve moving in a ``jerky'' fashion rather than smoothly
\vskip 5pt 
\item{} Liquid turning into vapor at the valve's vena contracta
\vskip 5pt 
\item{} Constant flow rate despite decreased downstream pressure 
\vskip 5pt 
\item{} Vapor bubbles condensing into liquid downstream of the valve
\vskip 5pt 
\item{} The appearance of valve trim after being corroded by fluids
\vskip 5pt 
\item{} A special tool used to lubricate a dry valve packing
\end{itemize}


%INDEX% Reading assignment: Lessons In Industrial Instrumentation, control valve problems (choked flow)
%INDEX% Reading assignment: Lessons In Industrial Instrumentation, control valve problems (noise)

%(END_NOTES)


