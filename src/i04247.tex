
%(BEGIN_QUESTION)
% Copyright 2009, Tony R. Kuphaldt, released under the Creative Commons Attribution License (v 1.0)
% This means you may do almost anything with this work of mine, so long as you give me proper credit

Read and outline the ``Erosion'' and ``Chemical Attack'' subsections of the ``Control Valve Problems'' section of the ``Control Valves'' chapter in your {\it Lessons In Industrial Instrumentation} textbook.  Note the page numbers where important illustrations, photographs, equations, tables, and other relevant details are found.  Prepare to thoughtfully discuss with your instructor and classmates the concepts and examples explored in this reading.

\underbar{file i04247}
%(END_QUESTION)





%(BEGIN_ANSWER)


%(END_ANSWER)





%(BEGIN_NOTES)

Erosion is when valve parts wear due to solid particles or wet steam passing through.  The only mitigating technique applicable is to use really hard valve trim materials (e.g. ceramic).

\vskip 10pt

Small holes in valve become larger as a result of erosion.

\vskip 20pt

Corrosion is when chemicals react with valve component materials.  Cavitation greatly accelerates erosion by stripping away the protective ``passivation layer'' that normally forms on metal.  The combination of cavitation and corrosion is sometimes referred to by the clever term {\it cavitation corrosion}.





\vskip 20pt \vbox{\hrule \hbox{\strut \vrule{} {\bf Suggestions for Socratic discussion} \vrule} \hrule}

\begin{itemize}
\item{} Explain why {\it wet} steam is destructive to control valve components.
\item{} Explain why cavitation accelerates the process of chemical corrosion.
\end{itemize}



%INDEX% Reading assignment: Lessons In Industrial Instrumentation, control valve problems (erosion)
%INDEX% Reading assignment: Lessons In Industrial Instrumentation, control valve problems (chemical attack)

%(END_NOTES)


