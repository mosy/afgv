
%(BEGIN_QUESTION)
% Copyright 2015, Tony R. Kuphaldt, released under the Creative Commons Attribution License (v 1.0)
% This means you may do almost anything with this work of mine, so long as you give me proper credit

Read the Fisher ``Design CAV4 Control Valve'' product flier (document 51.2:CAV4, April 2007), and answer the following questions:

\vskip 10pt

Explain how the unique construction of this sliding-stem valve trim helps prevent cavitation in liquid service.  In particular, note how the design differs from some other anti-cavitation valve designs.

\vskip 10pt

This particular control valve has been designed for extremely high pressure drops.  Just how great are the pressure drops (in units of PSID) this valve is designed to handle, and why is this parameter relevant to the subject of cavitation?

\vskip 10pt

This valve design intentionally separates the {\it throttling} and {\it seating} portions of the valve trim.  Explain why it has been designed this way, and what might happen to the valve over time if it were designed more conventionally (with throttling and seating performed by the same portion of the trim).

\vskip 10pt

This bulletin provides a formula on page 6 useful for selecting the particular cavitation-control valve design to use for different process applications ($A_r = {{\Delta P} \over {P_1 - P_v}}$).  Identify whether a {\it low} value for {\it application ratio} $A_r$ or a {\it high} value for $A_r$ is ``better'' with regard to minimal cavitation.

\vskip 30pt

Read the Fisher ``Vee-Ball Design V150, V200, and V300 Noise Attenuator'' product flier (document 51.3:Vee-Ball(S2), July 2006), and answer the following questions:

\vskip 10pt

Explain how the unique construction of this ball valve trim helps prevent cavitation in liquid service.

\vskip 10pt

Based on the valve characteristics graph shown in figure 4 (page 6), is this trim {\it quick-opening}, {\it linear}, or {\it equal-percentage}?

\underbar{file i04248}
%(END_QUESTION)





%(BEGIN_ANSWER)


%(END_ANSWER)





%(BEGIN_NOTES)

Fisher CaviTrol 4 (CAV4) trim is designed for liquid applications where cavitation is a problem, typically when pressure drop from upstream to downstream exceeds 3000 PSID.  Boiler feedwater recirculation is one such application (page 1).

\vskip 10pt

Standard anti-cavitation cage trim doesn't work well with pressure drops exceeding 3000 PSI, because leakage between the cage and plug just creates high-speed jets.  The CAV4 trim design throttles with separate components downstream (below) seating components.  Initial stages drop more pressure per stage than the last stages, the last being more prone to cavitation.  (page 5, figures 3 and 4)

Another feature of the CAV4 trim is that each successive pressure-dropping stage exhibits a greater flowing area, which means the last stage drops considerably less pressure than the first.  This helps avoid flashing by reducing the pressure recovery (increasing pressure recovery factor $F_L$) of that last stage and letting the higher-pressure stages do most of the pressure-reducing work (where their pressure is high enough to preclude flashing).  Figure 5 on page 6 shows how CAV4 trim is better than anti-cavitation control valves with equal-drop stages.

\vskip 10pt

The {\it application ratio }$A_r$ parameter calculated on page 6 of this manual is a ratio of total pressure drop ($P_1 - P_2$) to the difference between upstream pressure and vapor pressure of the liquid.  If the valve drops as much or more pressure than that required to fall below the vapor pressure (i.e. $A_r \geq 1$), then the valve is guaranteed to flash.   Values of $A_r$ less than 1 may or may not cavitate, depending on how great the total pressure drop is.

$$A_r = {P_1 - P_2 \over P_1 - P_{v}}$$

\vskip 20pt

Vee-Ball noise attenuator useful for noise reduction and for cavitation control.  ``Honeycomb'' structure dissipates fluid energy so not all the energy loss must occur at the edge of the throttling ball.

\vskip 10pt

Graph on page 6 clearly shows an {\it equal-percentage} characteristic.







\vskip 20pt \vbox{\hrule \hbox{\strut \vrule{} {\bf Suggestions for Socratic discussion} \vrule} \hrule}

\begin{itemize}
\item{} Explain why the fact that pressure drop across the last stage is less than for the first stage is important to avoiding cavitation (graph on page 6 of Fisher manual).
\item{} Explain why boiler feedwater recirculation control (i.e. a control valve recirculating boiler feedwater from the discharge of the feedwater pump back around to the pump's suction port) is a particularly troublesome process application. 
\end{itemize}





%INDEX% Reading assignment: Fisher cavitation-control valve manuals

%(END_NOTES)


