
%(BEGIN_QUESTION)
% Copyright 2009, Tony R. Kuphaldt, released under the Creative Commons Attribution License (v 1.0)
% This means you may do almost anything with this work of mine, so long as you give me proper credit

Read ``Case Number 1'' (pages 1-4) of the US Chemical Safety and Hazard Investigation Board's safety bulletin on ``Management of Change'' (Bulletin number 2001-04-SB) discussing the 1998 coker fire at the Equilon refinery in Anacortes (Washington), and answer the following questions:

\vskip 10pt

Explain in general terms what happened in the Coker unit of the refinery following a power outage, that led to this accident.  Specifically, identify how temperature measurement played a crucial role in the decision to drain the coke drum.

\vskip 10pt

Identify how the temperature sensors would have had to be built differently in order to provide better information to operations about the status of the coke drum under these abnormal conditions.  Explain why their existing design was adequate to measure temperature under normal operating conditions.

\vskip 10pt

Page 2 describes how a similar incident (though not lethal) occurred in 1996.  Describe the follow-up to that incident, and how better ``Management of Change'' (MOC) procedures might have prevented the 1998 disaster.

\vskip 10pt

A significant number of industrial accidents may be traced back to insufficient or non-existent MOC policies.  Identify some of the reasons MOC might not be adequately applied at an industrial facility, and what you as a technician might be able to do to increase the likelihood MOC procedures are respected.

\vskip 20pt \vbox{\hrule \hbox{\strut \vrule{} {\bf Suggestions for Socratic discussion} \vrule} \hrule}

\begin{itemize}
\item{} An important safety policy at many industrial facilities is something called {\it stop-work authority}, which means any employee has the right to stop work they question as unsafe.  Explain how stop-work authority could have been applied to this particular incident.
\item{} An ironic hazard within many chemical processing operations is that of a {\it power failure}.  This seems odd at first, because we typically associate the lack of electrical power with safety (lock-out, tag-out) rather than with danger.  However, a chemical processing operation may become very dangerous when electrical service is disrupted.  Try to generalize why this is, based on this particular case at the Equilon refinery and/or on any industrial accident examples you may be familiar with.
\item{} The report notes that ``heat transfer calculations'' would have shown a much longer cooling time necessary than what operators assumed.  Identify which heat transfer equations might apply to this type of scenario, identifying the important variables dictating rate of cooling for the coke drum.
\end{itemize}

\underbar{file i04253}
%(END_QUESTION)





%(BEGIN_ANSWER)


%(END_ANSWER)





%(BEGIN_NOTES)

A 2-hour power outage led to a 10-hour steam outage, causing coking operations to be shut down with one drum only 7\% full.  37 hours later, operations attempted to drain the drum, not realizing how hot the material inside was.  Operators assumed that heat losses to ambient air would have provided sufficient cooling to ensure the drum would be safe to open.  The hot hydrocarbon compounds ignited upon release, causing a major fire and 6 fatalities.

\vskip 10pt

The temperature probes only measured drum skin temperature, not the temperature inside the drum.  This is why they only registered 230 $^{o}$F while the still-liquid mass inside the drum was in fact much hotter.  The outer regions of the mass had cooled and solidified, which is what the probes registered.  Operators, feeling the outside of the piping was cool, decided (wrongly) that the contents inside the drum were cool enough to safely drain.

In order for the temperature probes to have accurately measured temperature inside the core of the drum, they would have had to extend further in.  During normal operation, extended depth was not necessary because the entirety of the drum's contents would be at a fairly uniform temperature.  Only after an extended outage would a situation such as this develop, where the mass cooled and solidified on the outside, leaving a hot core of liquid insulated within.  

Extended temperature probes would probably not be practical, because they would become damaged by the violent flow of hot oil into the drum, and also drilling operations removing solidified coke from the drum.  By necessity of longevity, the probes would have to be kept close to the wall of the drum where they would be safe.

\vskip 10pt

In a previous incident involving the emptying of partially-filled coke drums (in 1996), an internal investigation team recommended procedures be written for emptying partially-filled drums.  No report was written, however, leading to the condition in 1998 where operations had to guess what to do.  During the 1996 incident, water had been injected into the coking drum prior to opening, with the result being a lot of material spewed out of the drum by steam pressure when the drum was opened.  Water injection was decided to be too risky to attempt (!) in the 1998 incident, and so no water was injected for cooling effect that latter time.

\vskip 10pt

MOC procedures slow down the decision-making and upgrade processes at a facility, which translates into (potentially) lost production and therefore less profit.  As an instrument technician with a knowledge of physics and chemistry and measurement technologies relevant to process safety, you stand in a critical role in your ability to both comprehend and communicate dangers to other people at the facility.

%INDEX% Reading assignment: USCSB Safety Bulletin, ``Management of Change'' at two facilities including the Equilon refinery in Anacortes, Washington

%(END_NOTES)


