
%(BEGIN_QUESTION)
% Copyright 2015, Tony R. Kuphaldt, released under the Creative Commons Attribution License (v 1.0)
% This means you may do almost anything with this work of mine, so long as you give me proper credit

Read and outline the ``Basic Feedback Control Principles'' section of the ``Closed-Loop Control'' chapter in your {\it Lessons In Industrial Instrumentation} textbook.  Note the page numbers where important illustrations, photographs, equations, tables, and other relevant details are found.  Prepare to thoughtfully discuss with your instructor and classmates the concepts and examples explored in this reading.

\underbar{file i04254}
%(END_QUESTION)





%(BEGIN_ANSWER)


%(END_ANSWER)





%(BEGIN_NOTES)

The ``process'' is the physical system or variable that we wish to control.  A ``transmitter'' device senses the process variable and reports that as a signal to a control and/or display device.  A ``final control element'' such as a control valve is there to influence the process variable by varying some ``manipulated variable'' in the process.  

\vskip 10pt

The ``controller'' links the transmitter to the final control element, creating a closed feedback loop.  ``Negative feedback'' means the controller takes opposing action to any load change.  ``Loads'' are any ``wild'' or uncontrolled variables that affect the process variable we're trying to maintain at setpoint.  Loads are what make control systems necessary.

\vskip 10pt

Examples of feedback control systems include:

\begin{itemize}
\item{} Heat exchanger temperature controls
\item{} Autopilot controls on airplanes
\item{} Cruise controls on automobiles
\item{} Wastewater disinfection controls
\end{itemize}

\vskip 10pt

A control {\it algorithm} is the mathematical relationship between the process variable, the setpoint, and the automatic output (manipulated variable) of the controller.  The so-called ``PID'' algorithm is the most common one used for feedback control.









\vskip 20pt \vbox{\hrule \hbox{\strut \vrule{} {\bf Suggestions for Socratic discussion} \vrule} \hrule}

\begin{itemize}
\item{} Define terms such as {\it process variable}, {\it setpoint}, {\it manipulated variable}, {\it final control element}, {\it load}, and {\it algorithm}.
\item{} Give examples of some common feedback control systems and briefly describe how they work.
\item{} Explain what a shell-and-tube heat exchanger is and how it works.
\item{} Suppose the analytical transmitter (AT) in the chlorine disinfection process were to fail with a {\it low} signal.  How would the control system respond to this fault, and what would happen to the actual chlorine concentration?
\item{} Suppose the analytical transmitter (AT) in the chlorine disinfection process were to fail with a {\it high} signal.  How would the control system respond to this fault, and what would happen to the actual chlorine concentration?
\item{} Suppose the control valve in the chlorine disinfection process were to fail in the fully {\it closed} position.  How would the control system respond to this fault, and what would happen to the actual chlorine concentration?
\item{} Suppose the control valve in the chlorine disinfection process were to fail in the fully {\it open} position.  How would the control system respond to this fault, and what would happen to the actual chlorine concentration?
\end{itemize}











\vfil \eject

\noindent
{\bf Prep Quiz:}

The type of feedback used in a closed-loop control system is called ``negative feedback'' because:

\begin{itemize}
\item{} The electrical polarities of the PV and Output signals are opposite each other
\vskip 5pt 
\item{} The electrical polarity of the control signal is negative with respect to ground
\vskip 5pt 
\item{} It only functions when the process variable has a range extending below 0\%
\vskip 5pt 
\item{} The control system always responds in a direction opposite the disturbance
\vskip 5pt 
\item{} The control system naturally tries to shut the system down (negative production)
\vskip 5pt 
\item{} It makes the human operators feel depressed and generally bad about themselves
\end{itemize}












\vfil \eject

\noindent
{\bf Prep Quiz:}

A {\it heat exchanger} is a device used to:

\begin{itemize}
\item{} Convert potential energy into kinetic energy
\vskip 5pt 
\item{} Facilitate a chemical reaction between two fluids
\vskip 5pt 
\item{} Convert kinetic energy into potential energy
\vskip 5pt 
\item{} Control the pressure of a moving fluid stream
\vskip 5pt 
\item{} Transfer thermal energy between two different fluids
\vskip 5pt 
\item{} Control the flow rate of a moving fluid stream
\end{itemize}


%INDEX% Reading assignment: Lessons In Industrial Instrumentation, closed-loop control (basic principles)

%(END_NOTES)


