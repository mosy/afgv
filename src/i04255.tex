
%(BEGIN_QUESTION)
% Copyright 2009, Tony R. Kuphaldt, released under the Creative Commons Attribution License (v 1.0)
% This means you may do almost anything with this work of mine, so long as you give me proper credit

Read and outline the ``On/Off Control'' section of the ``Closed-Loop Control'' chapter in your {\it Lessons In Industrial Instrumentation} textbook.  Note the page numbers where important illustrations, photographs, equations, tables, and other relevant details are found.  Prepare to thoughtfully discuss with your instructor and classmates the concepts and examples explored in this reading.

\underbar{file i04255}
%(END_QUESTION)





%(BEGIN_ANSWER)


%(END_ANSWER)





%(BEGIN_NOTES)

``On/off'' control is sometimes referred to as ``Bang-Bang'' control: purely on or off at the final control element, forcing the process variable to oscillate between two setpoints.  A more sophisticated term for this is ``differential gap'' control.  Most home thermostats are like this, but this is unsuitable for a wide range of industrial processes.





\vskip 20pt \vbox{\hrule \hbox{\strut \vrule{} {\bf Suggestions for Socratic discussion} \vrule} \hrule}

\begin{itemize}
\item{} Explain why this form of control may not be suitable for an industrial process, giving a specific example to illustrate.
\item{} Explain why it would probably be a bad idea to set the LSP and USP values very close together to each other in a process controlled by an on/off controller.
\item{} Identify the criteria one might consider when deciding where to set the LSP and USP values.
\end{itemize}

%INDEX% Reading assignment: Lessons In Industrial Instrumentation, closed-loop control (on/off control)

%(END_NOTES)


