
%(BEGIN_QUESTION)
% Copyright 2009, Tony R. Kuphaldt, released under the Creative Commons Attribution License (v 1.0)
% This means you may do almost anything with this work of mine, so long as you give me proper credit

Read and outline the ``Proportional-Only Control'' section of the ``Closed-Loop Control'' chapter in your {\it Lessons In Industrial Instrumentation} textbook.  Note the page numbers where important illustrations, photographs, equations, tables, and other relevant details are found.  Prepare to thoughtfully discuss with your instructor and classmates the concepts and examples explored in this reading.

\underbar{file i04256}
%(END_QUESTION)





%(BEGIN_ANSWER)


%(END_ANSWER)





%(BEGIN_NOTES)

The algorithm for a proportional controller is as follows:

$$m = K_p e + b$$

\noindent
Where,

$m$ = Output

$K_p$ = Controller gain (how "aggressive" the controller will be)

$e$ = Error (PV $-$ SP or SP $-$ PV)

$b$ = Bias

\vskip 10pt

It may be helpful to mark the inputs of a proportional controller with ``+'' and ``$-$'' symbols, to distinguish the relative effects of PV versus SP on the output.  The ``action'' of a controller is conventionally defined by the effect of the PV on output, therefore:

$$m = K_p(\hbox{PV} - \hbox{SP}) + b \hskip 20pt \hbox{Direct-acting}$$

$$m = K_p(\hbox{SP} - \hbox{PV}) + b \hskip 20pt \hbox{Reverse-acting}$$

The necessary controller action is dictated by the transmitter, final control element, and process.  The heat exchanger temperature control system required a reverse-acting controller because its control valve was signal-to-open.

\vskip 10pt

The {\it gain} of a controller ($K_p$) is the ratio of output change to input change ($\Delta m \over \Delta e$).  Bias ($b$) is simply the output value of a controller when it has no error.  

\vskip 10pt

If gain is infinite, you get on/off control.  Too much gain results in oscillations.  Too little gain results in sluggish response.  Some ``overshoot'' is normal for a properly-tuned loop.

\vskip 10pt

Controller agressiveness also expressed in terms of ``proportional band:'' the amount the error must change in order to cause a 100\% change in controller output.  Proportional band is the reciprocal of gain, and is always expressed as a {\it percentage} value whereas gain is always expressed as a unitless number:

$$\hbox{Proportional Band} = {1 \over \hbox{Gain}}$$
  
Gain of 20 = PB of 5\%.  Gain of 5 is PB of 20\%.







\vskip 20pt \vbox{\hrule \hbox{\strut \vrule{} {\bf Suggestions for Socratic discussion} \vrule} \hrule}

\begin{itemize}
\item{} Explain the difference between a {\it direct-acting} controller and a {\it reverse-acting} controller.
\item{} Why do proportional controllers have a {\it bias} term in their characteristic equations?
\item{} Examine the mechanical proportional control mechanism (for liquid level) shown in the textbook, and identify how to change the {\it gain} value of this controller.  Also, identify how to change the {\it bias} value of this controller.
\item{} Explain how on/off control is just a {\it limiting case} of proportional control.
\item{} Define ``proportional band'' in your own words.
\item{} Explain how one determines the proper action (direct vs. reverse) for a proportional controller.
\end{itemize}



%INDEX% Reading assignment: Lessons In Industrial Instrumentation, closed-loop control (proportional-only control)

%(END_NOTES)


