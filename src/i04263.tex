
%(BEGIN_QUESTION)
% Copyright 2015, Tony R. Kuphaldt, released under the Creative Commons Attribution License (v 1.0)
% This means you may do almost anything with this work of mine, so long as you give me proper credit

\vbox{\hrule \hbox{\strut \vrule{} {\bf Desktop Process exercise} \vrule} \hrule}

\noindent
Configure the controller as follows (for ``proportional-only'' control):

\begin{itemize}
\item{} Control action = {\it reverse}
\item{} Gain = 1 (Proportional Band = 100\%)
\item{} Reset (Integral) = {\it minimum effect} = {\it 100+ minutes/repeat} = {\it 0 repeats/minute}
\item{} Rate (Derivative) = {\it minimum effect} = {\it 0 minutes} 
\end{itemize}

Check to see that the controller is able to function in automatic mode (adjusting motor speed as you adjust the setpoint value).  Now, you are set to experiment with the effect of different ``gain'' values in the PID algorithm.  You may access the ``gain'' parameter by entering the controller's {\it tuning} function.

\vskip 10pt

Try setting the ``gain'' value to a number significantly less than 1, then changing the setpoint (SP) value several times to observe the system's response.  If you have a data acquisition (DAQ) unit connected to measure controller PV and output signal values, note the relationship between the two graphs plotted on the computer display following each setpoint change.

\vskip 10pt

Now try setting the ``gain'' value to a number significantly greater than 1, changing the setpoint value again and again to observe the system's response.

\vskip 10pt

\noindent
Answer the following questions:

\begin{itemize}
\item{} Which gain settings result in the swiftest response from the motor?
\vskip 5pt 
\item{} Which gain settings result in the most sluggish response from the motor?
\vskip 5pt 
\item{} Are there any gain setting values that result in {\it oscillation} of motor speed?
\vskip 5pt 
\item{} Do you notice any {\it proportional-only offset}?
\vskip 5pt 
\item{} Determine the ``optimal'' gain setting for your process resulting in swift response and minimum offset without too much oscillation.
\end{itemize}

\vskip 20pt \vbox{\hrule \hbox{\strut \vrule{} {\bf Suggestions for Socratic discussion} \vrule} \hrule}

\begin{itemize}
\item{} Generalizing to all proportional controllers, explain the effect of decreasing the controller gain value further and further.
\item{} Generalizing to all proportional controllers, explain the effect of increasing the controller gain value further and further.
\end{itemize}

\underbar{file i04263}
%(END_QUESTION)





%(BEGIN_ANSWER)


%(END_ANSWER)





%(BEGIN_NOTES)

{\bf Lesson:} finding the right gain for a proportional-only controller is a compromise between minimum proportional-only offset and loop stability.





\vfil \eject

\noindent
{\bf Summary Quiz:}

(An alternative to a summary quiz is to have students demonstrate their Desktop Process units operating in automatic mode)

%INDEX% Desktop Process: experimenting with gain settings

%(END_NOTES)


