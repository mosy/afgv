
%(BEGIN_QUESTION)
% Copyright 2009, Tony R. Kuphaldt, released under the Creative Commons Attribution License (v 1.0)
% This means you may do almost anything with this work of mine, so long as you give me proper credit

Read and outline the ``Proportional Control Action'' subsection of the ``Pneumatic PID Controllers'' section of the ``Closed-Loop Control'' chapter in your {\it Lessons In Industrial Instrumentation} textbook.  Note the page numbers where important illustrations, photographs, equations, tables, and other relevant details are found.  Prepare to thoughtfully discuss with your instructor and classmates the concepts and examples explored in this reading.

\underbar{file i04264}
%(END_QUESTION)





%(BEGIN_ANSWER)


%(END_ANSWER)





%(BEGIN_NOTES)

In force-balance pneumatic controllers, input bellows (one for PV, one for SP) exert opposing forces on a beam.  The output bellows exerts a counter-acting force on that beam to keep it immobile.  The pneumatic controller shown on the first page is {\it direct-acting}, since an increase in PV signal causes an increase in output signal.  Switching to reverse action is as simple as swapping air line connections to the PV and SP bellows.

Gain adjustment is possible in a force-balance controller by changing bellows sizes, or moving the position of the fulcrum.  Moving the fulcrum toward the output bellows increases the gain because the output bellows now must fight harder to make it balance (i.e. more output change for the same input change).

\vskip 10pt

Motion-balance controllers also exist, where an output bellows moves in such a way to counter-act the motion cause by PV and SP input bellows.  The beam tilts as a result, the output bellows motion acting to maintain a constant beam/nozzle gap as the input bellows move.

Gain adjustment is possible in the motion-balance controller by moving the nozzle along the beam's length.  Moving the nozzle toward the output bellows decreases the gain because the output bellows doesn't have to move as far to make it balance (i.e. less output change for the same input change).










\vskip 20pt \vbox{\hrule \hbox{\strut \vrule{} {\bf Suggestions for Socratic discussion} \vrule} \hrule}

\begin{itemize}
\item{} Explain what we would need to do to reverse the control action in one of the pneumatic controller mechanisms shown in the textbook.
\item{} Explain why the changes in gain achieved by repositioning elements in the force-balance and motion-balance mechanisms are opposite of each other (i.e. moving the fulcrum toward the output bellows in a force-balance mechanism increases gain, while moving the nozzle toward the output bellows in a motion-balance mechanism decreases gain).
\item{} What would happen in the controller if the nozzle plugged, and why?
\item{} What would happen in the controller if the restriction plugged, and why?
\item{} Suppose a mechanical adjustment is made to the mechanism (you choose).  What effect(s) will this have on the controller's behavior?
\item{} Suppose one of the bellows (you choose) is changed for another of a different size.  What effect(s) will this have on the controller's behavior?
\item{} Suppose someone mangles the lever and bends it (either up or down -- you choose).  What effect(s) will this have on the controller's behavior?
\end{itemize}

%INDEX% Reading assignment: Lessons In Industrial Instrumentation, closed-loop control (pneumatic PID controllers)

%(END_NOTES)


