
%(BEGIN_QUESTION)
% Copyright 2009, Tony R. Kuphaldt, released under the Creative Commons Attribution License (v 1.0)
% This means you may do almost anything with this work of mine, so long as you give me proper credit

Read and outline the introduction to the ``Analog Electronic PID Controllers'' section of the ``Closed-Loop Control'' chapter in your {\it Lessons In Industrial Instrumentation} textbook.  Note the page numbers where important illustrations, photographs, equations, tables, and other relevant details are found.  Prepare to thoughtfully discuss with your instructor and classmates the concepts and examples explored in this reading.

\underbar{file i04266}
%(END_QUESTION)





%(BEGIN_ANSWER)


%(END_ANSWER)





%(BEGIN_NOTES)

Analog electronic PID controllers are more reliable and faster than digital electronic PID controllers, but their only practical edge over digital technology is speed now that digital controllers are quite reliable.  Most analog designs were based on opamps.








\vskip 20pt \vbox{\hrule \hbox{\strut \vrule{} {\bf Suggestions for Socratic discussion} \vrule} \hrule}

\begin{itemize}
\item{} Why use an {\it analog} controller in the 21st century?
\end{itemize}










\vfil \eject

\noindent
{\bf Prep Quiz:}

In an age of inexpensive and reliable digital electronic controller hardware, an analog electronic control circuit is something of a dinosaur.  Identify one practical reason why someone might choose to use an analog electronic controller instead of a digital electronic controller to control a process.

%INDEX% Reading assignment: Lessons In Industrial Instrumentation, closed-loop control (analog electronic PID controllers, intro)

%(END_NOTES)


