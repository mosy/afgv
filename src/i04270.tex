
%(BEGIN_QUESTION)
% Copyright 2009, Tony R. Kuphaldt, released under the Creative Commons Attribution License (v 1.0)
% This means you may do almost anything with this work of mine, so long as you give me proper credit

Read and outline the ``Introduction to Calculus'' section of the ``Calculus'' chapter in your {\it Lessons In Industrial Instrumentation} textbook.  Note the page numbers where important illustrations, photographs, equations, tables, and other relevant details are found.  Prepare to thoughtfully discuss with your instructor and classmates the concepts and examples explored in this reading.

\underbar{file i04270}
%(END_QUESTION)





%(BEGIN_ANSWER)


%(END_ANSWER)





%(BEGIN_NOTES)

The upper-case Greek letter ``delta'' ($\Delta$) represents a {\it difference} in the value of some variable.  This difference could be a change in the variable's value over time, over space, or as it relates to some other changing variable.

For example, $\Delta T$ could represent a change in temperature ($T$) over a period of time, a difference in temperature between two separate locations, etc.

\vskip 10pt

When successive values of a changing variable lie very close to one another, the difference is denoted either by a lower-case letter ``d'' or by the lower-case Greek letter ``delta'' ($\delta$) rather than by $\Delta$.  Thus, $dT$ (or $\delta T$) could represent the change in temperature from one moment in time to the next, or the change in temperature from one location to the next location immediately adjacent.  These small differences are called {\it differentials}.

\vskip 10pt

The sum of successive differences leads to one large difference, as noted by the following notation (here using differences in temperature, $T$):

$$\Delta T_{total} = \sum_{n=9:45}^{10:32} \Delta T_n = \hbox{Total temperature rise over time, from 9:45 to 10:32}$$

The summation of successive differentials is a continuous process rather than discrete, as noted by the following notation (using differentials of temperature):

$$\Delta T_{total} = \int_{9:45}^{10:32} dT = \hbox{Total temperature rise over time, from 9:45 to 10:32}$$

This long ``S''-shaped symbol refers to the process of {\it integration}.










\vskip 20pt \vbox{\hrule \hbox{\strut \vrule{} {\bf Suggestions for Socratic discussion} \vrule} \hrule}

\begin{itemize}
\item{} This reading assignment covers some very fundamental principles, and as such students' active reading of the text should be scutinized.  Are they taking comprehensive notes?  Are they expressing concepts in their own terms?  Your Socratic discussions with students should mirror the points listed in Question 0.
\item{} Examine the summation shown in the textbook for furnace temperature taken over 1-minute intervals, and interpret each of the symbols in that mathematical expression:  
\begin{itemize}

\item{} What does the lower number ($n$ = 9:45) represent?
\item{} What does the upper number (10:32) represent?
\item{} What does $n$ represent?
\item{} What does $\Delta T_n$ represent?
\item{} What does $\Delta T_{total}$ represent?
\end{itemize}
\item{} Examine the integral shown in the textbook for furnace temperature, and interpret each of the symbols in that mathematical expression:  
\begin{itemize}

\item{} What does the lower number (9:45) represent?
\item{} What does the upper number (10:32) represent?
\item{} What does $dT$ represent?
\item{} What does $\Delta T_{total}$ represent?
\end{itemize}
\end{itemize}










\vfil \eject

\noindent
{\bf Prep Quiz:}

Explain in your own words the distinction between the following mathematical symbols.  In other words, what do the $\Delta$ and $d$ symbols {\it mean} with reference to $x$?  Be as detailed and specific as you can.

$$\Delta x \hskip 50pt dx$$

%INDEX% Reading assignment: Lessons In Industrial Instrumentation, calculus (introduction)

%(END_NOTES)


