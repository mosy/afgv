
%(BEGIN_QUESTION)
% Copyright 2009, Tony R. Kuphaldt, released under the Creative Commons Attribution License (v 1.0)
% This means you may do almost anything with this work of mine, so long as you give me proper credit

Read and outline the ``The Concept of Differentiation'' section of the ``Calculus'' chapter in your {\it Lessons In Industrial Instrumentation} textbook.  Note the page numbers where important illustrations, photographs, equations, tables, and other relevant details are found.  Prepare to thoughtfully discuss with your instructor and classmates the concepts and examples explored in this reading.

\underbar{file i04272}
%(END_QUESTION)





%(BEGIN_ANSWER)


%(END_ANSWER)





%(BEGIN_NOTES)

We may infer the flow rate of propane out of a propane tank fueling a torch by successively measuring the tank's mass at discrete intervals in time, the quotient having units of mass per unit time (e.g. kilogram per minute):

$$\overline{W} = {\Delta m \over \Delta t}$$

If the tank's mass is plotted on a graph with time being the independent variable, the {\it slope} of this trend at any point will represent the flow rate of the gas.  High gas flow rates will be steeply-pitched on the mass graph, while low gas flow rates will be shallow.

\vskip 10pt

If we were to sample the tank's mass at an infinitely rapid pace, we would be calculating the quotient of {\it differentials} of mass and time rather than {\it differences} of mass and time.  Such a ratio is called a {\it derivative}, and in this case it represents the instantaneous flow rate of propane at any point in time:

$$W = {dm \over dt} = [\hbox{m/s}]$$

\vskip 10pt

{\it Velocity} is the derivative of position and time:

$$v = {dx \over dt} = [\hbox{m/s}^2]$$

\vskip 10pt

{\it Acceleration} is the derivative of velocity and time:

$$a = {dv \over dt} = {d^2x \over dt^2} = [\hbox{m/s}^2]$$

\vskip 10pt

Non-temporal quantities such as amplifier gain may also be expressed as a derivative:

$$\hbox{Gain} = {dV_{out} \over dV_{in}}$$











\vskip 20pt \vbox{\hrule \hbox{\strut \vrule{} {\bf Suggestions for Socratic discussion} \vrule} \hrule}

\begin{itemize}
\item{} This reading assignment covers some very fundamental principles, and as such students' active reading of the text should be scutinized.  Are they taking comprehensive notes?  Are they expressing concepts in their own terms?  Your Socratic discussions with students should mirror the points listed in Question 0.
\item{} How will the graph of propane weight over time change if someone closes the shut-off valve?
\item{} How will the graph of propane weight over time change if the hose ruptures?
\item{} How will the graph of propane weight over time change if external propane is pumped into the tank through the hose?
\end{itemize}

%INDEX% Reading assignment: Lessons In Industrial Instrumentation, calculus (the concept of differentiation)

%(END_NOTES)


