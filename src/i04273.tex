
%(BEGIN_QUESTION)
% Copyright 2009, Tony R. Kuphaldt, released under the Creative Commons Attribution License (v 1.0)
% This means you may do almost anything with this work of mine, so long as you give me proper credit

Read and outline Chapter 3 of the book {\it Calculus Made Easy} by Sylvanus P. Thompson, titled ``On Relative Growings''.  Note the page numbers where important illustrations, photographs, equations, tables, and other relevant details are found.  Prepare to thoughtfully discuss with your instructor and classmates the concepts and examples explored in this reading.

\underbar{file i04273}
%(END_QUESTION)





%(BEGIN_ANSWER)


%(END_ANSWER)





%(BEGIN_NOTES)

Constants versus variables: {\it constants} are usually represented by the first letters of the alphabet ($a$, $b$, $c$), while {\it variables} usually represented by the last ($x$, $y$, $z$).  Variables' values are often dependent upon other variables' values.  In the case of similar triangles, lengthening one side by a differential ($x + dx$) results in the other side lengths increasing by their own differentials ($y + dy$).  In the case of a similar triangle, the derivative $dy \over dx$ is a constant, being fixed by the trigonometric function of a fixed angle.

In the case of a triangle with a fixed hypotenuse (the ladder against a wall), changes in $x$ will also cause changes in $y$, but not at a fixed ratio.  Moving the base of the ladder out from the wall by one inch ($dx$ = 1 inch) results in the ladder's height against the wall changing by -0.11 inches ($dy$ = -0.11 inches), so that $dy \over dx$ = -0.11 when $x$ = 19 inches and $y$ = 180 inches.

\vskip 10pt

Derivatives only make sense when the variables ($x$, $y$) are related to each other as they are in these triangle examples.  Otherwise, $dy \over dx$ has no meaning whatsoever.

Equations where two related variables are expressed in such a way that neither variable is by itself on one side are called {\it implicit functions}.  Examples include ${y \over x} = \tan 30^o$ and $x^2 + 3 = 2y - 7$.  If an equation is in such a form where one variable is by itself on one side and everything else is on the other side, it is called an {\it explicit function}, where the lone variable is {\it dependent} and any variable(s) on the other side are {\it independent}.  An example of this is $u = x^2 \sin \theta$, where $u$ is dependent and both $x$ and $\theta$ are independent.

Variable dependency is often shown by {\it function} notation.  $F(x, y, z)$ tells us that $x$, $y$, and $z$ are all variables implicitly related to each other.  $y = F(x,z)$ tells us that $y$ is a dependent variable to $x$ and $z$.

\vskip 10pt

Ratios of differentials such as $dy \over dx$ are known as {\it differential coefficients} (or {\it derivatives} in modern parlance).  The process of finding the value of a derivative is called {\it differentiation}.

\vskip 10pt

It should be noted that $dx$ does not mean ``$d$ times $x$''.  Instead, $d$ means ``an element of'' or ``a bit of''.  If we have a function of $y$ in terms of $x$ written as $y = F(x)$ and we differentiate that function, we may express the result as $d \left(F(x)\right) \over dx$ of we may alternatively write $F'(x)$.











\vskip 20pt \vbox{\hrule \hbox{\strut \vrule{} {\bf Suggestions for Socratic discussion} \vrule} \hrule}

\begin{itemize}
\item{} This reading assignment covers some very fundamental principles, and as such students' active reading of the text should be scutinized.  Are they taking comprehensive notes?  Are they expressing concepts in their own terms?  Your Socratic discussions with students should mirror the points listed in Question 0.
\end{itemize}










\vfil \eject

\noindent
{\bf Prep Quiz:}

Identify each of the variables in this explicit function as being either {\it independent} or {\it dependent}:

$$V = IR$$











\vfil \eject

\noindent
{\bf Prep Quiz:}

Identify each of the variables in this explicit function as being either {\it independent} or {\it dependent}:

$$P = {V^2 \over R}$$


%INDEX% Reading assignment: Calculus Made Easy, chapter 2 (``On Relative Growings'')

%(END_NOTES)


