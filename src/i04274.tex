
%(BEGIN_QUESTION)
% Copyright 2009, Tony R. Kuphaldt, released under the Creative Commons Attribution License (v 1.0)
% This means you may do almost anything with this work of mine, so long as you give me proper credit

Read and outline the ``The Concept of Integration'' section of the ``Calculus'' chapter in your {\it Lessons In Industrial Instrumentation} textbook.  Note the page numbers where important illustrations, photographs, equations, tables, and other relevant details are found.  Prepare to thoughtfully discuss with your instructor and classmates the concepts and examples explored in this reading.

\underbar{file i04274}
%(END_QUESTION)





%(BEGIN_ANSWER)


%(END_ANSWER)





%(BEGIN_NOTES)

Average mass flow rate ($\overline{W}$), when multiplied by an interval of time ($\Delta t$) yields a total change in mass ($\Delta m$):

$$\Delta m = \overline{W} \Delta t$$

This process of multiplication, when viewed on a graph of mass flow rate plotted with respect to time, is the {\it area} between the plotted function and the $x$ axis.  If the mass flow rate is not steady but changes over time, we can still calculate the area underneath the graph and interpret that area as change in mass.  For several rectangular sections of constant flow on a graph, the total change in mass is the summation of each rectangle's area.  For eight such periods, the shorthand notation is thus:

$$\Delta m = \sum_{n=1}^8 W_n \> \Delta t_n$$

Each $W_n \Delta t_n$ represents the mass change in just one of the rectangular areas, while the summation represents the total mass change.

\vskip 10pt

For applications where the mass flow function is a curve rather than a stepped line, we may closely approximate the total mass change by using a series of very narrow rectangles (i.e. small $\Delta t$ intervals) and summing those rectangular areas.  This is called a {\it Riemann sum}.  If the time intervals are infinitesimal, they become time {\it differentials} ($dt$) and we may sum them using the integration symbol ($\int$) rather than the discrete summation symbol ($\sum$):

$$\Delta m = \sum_{n=0}^x W \> \Delta t_n \hskip 30pt \hbox{Summing discrete quantities of } W \Delta t$$

$$\Delta m = \int_0^x W \> dt \hskip 30pt \hbox{Summing continuous quantities of } W \> dt$$

Since the result of either calculation is $\Delta m$ (change in mass), we need to know the initial mass quantity in the fuel tank ($m_0$) to calculate how much mass if left at the end of the integration period (at time $x$): 

$$m_x = \int_0^x W \> dt + m_0$$

The odometer on a car's dashboard display is an example of integration, integrating speed and time to calculate total distance traveled ($x = \int v \> dt + x_0$).










\vskip 20pt \vbox{\hrule \hbox{\strut \vrule{} {\bf Suggestions for Socratic discussion} \vrule} \hrule}

\begin{itemize}
\item{} This reading assignment covers some very fundamental principles, and as such students' active reading of the text should be scutinized.  Are they taking comprehensive notes?  Are they expressing concepts in their own terms?  Your Socratic discussions with students should mirror the points listed in Question 0.
\item{} Examine the summation shown in the textbook for propane flow taken over discrete time intervals, and interpret each of the symbols in that mathematical expression:  
\begin{itemize}

\item{} What does the lower number ($n = 0$) represent?
\item{} What does the upper number ($x$) represent?
\item{} What does $W$ represent?
\item{} What does $n$ represent?
\item{} What does $\Delta t_n$ represent?
\item{} What does $\Delta m$ represent?
\end{itemize}
\item{} Examine the integral shown in the textbook for propane flow, and interpret each of the symbols in that mathematical expression:  
\begin{itemize}

\item{} What does the lower number (0) represent?
\item{} What does the upper number ($x$) represent?
\item{} What does $W$ represent?
\item{} What does $dt$ represent?
\item{} What does $\Delta m$ represent?
\end{itemize}
\item{} Examine either the summation or the integral shown in the textbook for propane flow, and identify appropriate units of measurement used for each variable ($m$, $W$, $t$).
\end{itemize}










\vfil \eject

\noindent
{\bf Prep Quiz:}

The graphical interpretation of {\it integration} is:

\begin{itemize}
\item{} The area enclosed by the graph
\vskip 5pt 
\item{} The curvature of the graph
\vskip 5pt 
\item{} The slope of the graph
\vskip 5pt 
\item{} The total length of the graph
\vskip 5pt 
\item{} The maximum height of the graph
\vskip 5pt 
\item{} The minimum height of the graph
\end{itemize}






\vfil \eject

\noindent
{\bf Prep Quiz:}

Explain in your own words the distinction between the following mathematical symbols.  In other words, what do the $\Sigma$ and $\int$ symbols {\it mean}?  Be as detailed and specific as you can.

$$\Sigma \hskip 50pt \int$$

%INDEX% Reading assignment: Lessons In Industrial Instrumentation, calculus (the concept of integration)

%(END_NOTES)


