
%(BEGIN_QUESTION)
% Copyright 2009, Tony R. Kuphaldt, released under the Creative Commons Attribution License (v 1.0)
% This means you may do almost anything with this work of mine, so long as you give me proper credit

Read and outline the ``How Derivatives and Integrals Relate to One Another'' section of the ``Calculus'' chapter in your {\it Lessons In Industrial Instrumentation} textbook.  Note the page numbers where important illustrations, photographs, equations, tables, and other relevant details are found.  Prepare to thoughtfully discuss with your instructor and classmates the concepts and examples explored in this reading.

\underbar{file i04275}
%(END_QUESTION)





%(BEGIN_ANSWER)


%(END_ANSWER)





%(BEGIN_NOTES)

Differentiation is a quotient of differences: an expression of one variable's rate of change compared to another variable.  Integration is a sum of products: an expression of the accumulation of the product of two variables over some specified interval.  These two mathematical operations are inverse to one another: one ``un-does'' the other.

\vskip 10pt

Position ($x$), velocity ($v$), and acceleration ($a$) are all related quantities through differentiation and integration with respect to time. 

$$v = {dx \over dt} \hskip 30pt a = {dv \over dt}$$

$$v = \int a \> dt \hskip 30pt x = \int v \> dt$$

\vskip 10pt

The {\it Fundamental Theorem of Calculus} demonstrates the inversely complementary nature of integration and differentiation:

$${d \over dx} \left[ \int_a^b f(x) \> dx \right] = f(x)$$









\vskip 20pt \vbox{\hrule \hbox{\strut \vrule{} {\bf Suggestions for Socratic discussion} \vrule} \hrule}

\begin{itemize}
\item{} This reading assignment covers some very fundamental principles, and as such students' active reading of the text should be scutinized.  Are they taking comprehensive notes?  Are they expressing concepts in their own terms?  Your Socratic discussions with students should mirror the points listed in Question 0.
\item{} Compare and contrast differentiation versus integration, noting how the two functions are flip-sides of the same coin in almost all regards.
\item{} Explain what a {\it tangent line} is, in your own words.
\item{} Identify potential units of measurement for $y$, $x$, and $dy \over dx$ in the differentiation example.
\item{} Identify potential units of measurement for $y$, $x$, and $\int y \> dx$ in the integration example.
\item{} Interpret the Fundamental Theorem of Calculus, explaining what that formula is saying in your own words.
\item{} View the flip-book animation of differentiation and integration contained in your textbook, and explain in your own terms what is going on.
\end{itemize}








\vfil \eject

\noindent
{\bf Summary Quiz:}

We know that {\it integration} and {\it differentiation} are inverse operations: one un-does the other.  Specifically, which arithmetic functions are associated with each of these calculus operations?

\begin{itemize}
\item{} Integration is square-root followed by addition ; differentiation is squaring followed by subtraction
\vskip 10pt 
\item{} Integration is multiplication followed by addition ; differentiation is subtraction followed by division
\vskip 10pt 
\item{} Integration is multiplication followed by division ; differentiation is addition followed by subtraction
\vskip 10pt 
\item{} Integration is addition followed by square-root ; differentiation is multiplication followed by subtraction
\vskip 10pt 
\item{} Integration is addition followed by subtraction ; differentiation is multiplication followed by division
\vskip 10pt 
\item{} Integration is addition followed by multiplication ; differentiation is division followed by square-root
\end{itemize}



%INDEX% Reading assignment: Lessons In Industrial Instrumentation, calculus (derivatives and integrals)

%(END_NOTES)


