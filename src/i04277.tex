
%(BEGIN_QUESTION)
% Copyright 2009, Tony R. Kuphaldt, released under the Creative Commons Attribution License (v 1.0)
% This means you may do almost anything with this work of mine, so long as you give me proper credit

Read and outline the ``Numerical Differentiation'' section of the ``Calculus'' chapter in your {\it Lessons In Industrial Instrumentation} textbook.  Note the page numbers where important illustrations, photographs, equations, tables, and other relevant details are found.  Prepare to thoughtfully discuss with your instructor and classmates the concepts and examples explored in this reading.

\underbar{file i04277}
%(END_QUESTION)





%(BEGIN_ANSWER)


%(END_ANSWER)





%(BEGIN_NOTES)

{\it Differentiation} is the act of finding one variable's rate of change compared to a related variable.  This may be the rate of change of some variable over time, or in relation to a non-temporal variable.  Arithmetically, differentiation is a process of {\it subtraction} (finding the change in variables) followed by {\it division} (calculating a ratio of one change to another).

\vskip 10pt

Pressure rate-of-change is a good indicator of a pipeline rupture.  A computer sampling every so often to calculate rates of pressure change over time is calculating $\Delta P \over \Delta t$, which is not as precise as calculating true derivatives (slopes of tangent lines).  A typical differentiating algorithm is shown in ``pseudocode'' as a repeating loop of instructions: subtraction followed by division followed by updating of variables and repetition.

Analog opamp circuits are capable of performing true (instantaneous) differentiation by generating an output voltage proportional to the amount of current through a capacitor subjected to an input voltage.  Capacitor current is directly related to the time-derivative of voltage ($I = C {dV \over dt}$).  True differentiation, however, may not be desired in a real-life application because it will respond aggressively to any noise or other ``jitters'' in the input signal.  Any sudden change in voltage, even if small in magnitude, will cause a differentiator circuit to respond with a high output if the time duration of that transient is very short.

\vskip 10pt

An example of non-temporal differentiation includes calculating the sensitivity of a baffle/nozzle mechanism (rate of change of pressure signal versus motion, $dP \over dx$).





\vskip 20pt \vbox{\hrule \hbox{\strut \vrule{} {\bf Suggestions for Socratic discussion} \vrule} \hrule}

\begin{itemize}
\item{} Describe the process of differentiating real-world data in arithmetic terms.  In other words, what arithmetic operations must one perform, in what order, are necessary to calculate rates of change from tabulated data?
\item{} Explain why rates of pressure change are so important in pipeline operations.
\item{} Explain why a digital computer sampling pipeline pressures over time can never perfectly calculate the derivative (rate of change) of pressure with respect to time, but rather can only manage an approximation of the derivative.
\item{} Explain what a {\it tangent line} is, and how one would choose ``convenient points'' on a graph to plot one and to measure its slope.
\item{} Explain how the ``pseudocode'' algorithm works to calculate derivatives.  In particular, explain how this computer program is able to distinguish between current values of pressure and time versus past values of pressure and time.
\item{} Explain why a simple opamp circuit is able to perform time-differentiation more accurately than the best digital computer ever can.
\item{} Explain why true differentiation may not be what we really desire for practical rate-of-change signal processing.
\end{itemize}







\vfil \eject

\noindent
{\bf Prep Quiz:}

We generally sketch a {\it tangent line} as part of the process for graphically:

\begin{itemize}
\item{} Summing the accumulated areas of many rectangles
\vskip 5pt 
\item{} Determining the boundaries of the interval of integration
\vskip 5pt 
\item{} Calculating the ratio of hypotenuse length over adjacent length
\vskip 5pt 
\item{} Finding the center of an arbitrarily-drawn circle
\vskip 5pt 
\item{} Calculating the rate of change of one variable to another
\vskip 5pt 
\item{} Calculating the difference (change) between two points
\end{itemize}


\vfil \eject

\noindent
{\bf Prep Quiz:}

A {\it tangent line} is a line that:

\begin{itemize}
\item{} Is used to calculate the area bound by a curve
\vskip 5pt 
\item{} Forms the hypotenuse of a right triangle
\vskip 5pt 
\item{} Is only one unit measure in length
\vskip 5pt 
\item{} Matches the slope of a curve at some specified point
\vskip 5pt 
\item{} Is the ratio of the opposite over the adjacent sides
\vskip 5pt 
\item{} Crosses a curve perpendicularly at some specified point
\end{itemize}



%INDEX% Reading assignment: Lessons In Industrial Instrumentation, calculus (numerical differentiation)

%(END_NOTES)


