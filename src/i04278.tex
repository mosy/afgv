
%(BEGIN_QUESTION)
% Copyright 2009, Tony R. Kuphaldt, released under the Creative Commons Attribution License (v 1.0)
% This means you may do almost anything with this work of mine, so long as you give me proper credit

Read and outline the ``Numerical Integration'' section of the ``Calculus'' chapter in your {\it Lessons In Industrial Instrumentation} textbook.  Note the page numbers where important illustrations, photographs, equations, tables, and other relevant details are found.  Prepare to thoughtfully discuss with your instructor and classmates the concepts and examples explored in this reading.

\underbar{file i04278}
%(END_QUESTION)





%(BEGIN_ANSWER)


%(END_ANSWER)





%(BEGIN_NOTES)

{\it Integration} is the act of finding the accumulation of one variable multiplied by a related variable.  This may be in relation to time, or in relation to some non-temporal variable.  Arithmetically, integration is a process of {\it multiplication} (finding the product of two variables) followed by {\it addition} (calculating an accumulated sum, or a running total).

\vskip 10pt

Calculating total volume or mass based on volumetric or mass flow measurements is a practical application of numerical integration.  A computer sampling every so often to calculate change in volume or mass based on a flow signal is called an {\it integrator}.  Since flow rate is defined as the rate of change of either volume or mass per unit time, then multiplying such a signal by small increments in time ($dt$) and accumulating those products will yield a figure of how much volume or mass has passed by in the specified interval.  An odometer is an example of an integrator, integrating speed over time to yield distance.  A weighfeeder works like a liquid flow totalizer, only taking the mass rate signal of dry material and calculating total mass passed by.  A typical integrating algorithm is shown in ``pseudocode'' as a repeating loop of instructions: multiplication followed by addition followed by updating of the time variable and repetition.

Integrators ignore noise, regardless of their sample speed.  Noise ``jitters'' tend to cancel each other out, the positive spikes being about the same magnitude and frequency as the negative spikes.

\vskip 10pt

An example of non-temporal integration includes calculating work done (force acting over a displacement).  The interval we specify for an integration is very important, since it defines whether the differential will have a positive or a negative sign.  Integrating over an interval of zero always yields zero.  Calculating the potential energy of a drawn bow is an example of integration, where we calculate the area enclosed by the bow's {\it force-draw curve}.  Compound bows, with their very nonlinear froce-draw curves, store far more energy (far greater enclosed area under the curve) for the same holding force than a traditional longbow.

Another example of using integration to calculate work is in determining the amount of work done by the piston of a heat engine.  The piston moves to compress a gas, which is then suddenly heated by combustion to produce a higher pressure, and then the piston moves the other way to expand that heated gas.  The difference between the integral (shaded area) of compression versus the larger integral (shaded area) of expansion is the amount of net work done by the piston during each cycle.  If we sketch the pressure/volume graph as a closed loop to show both compression and expansion, the net work will be proportional to the area enclosed within that loop.




\vskip 20pt \vbox{\hrule \hbox{\strut \vrule{} {\bf Suggestions for Socratic discussion} \vrule} \hrule}

\begin{itemize}
\item{} Describe the process of integrating real-world data in arithmetic terms.  In other words, what arithmetic operations must one perform, in what order, are necessary to calculate accumulations from tabulated data?
\item{} Explain why integration is so useful in industrial measurement applications.
\item{} Explain how the Foxboro model 14 pneumatic integrator works, and how this ingenious mechanism is analogous to the odometer in an automobile.
\item{} For those who have studied pressure-based flow measurement, explain how the Foxboro model 14 pneumatic integrator performed {\it square root extraction} in addition to time-integration.
\item{} Explain how the ``pseudocode'' algorithm works to calculate integrals.
\item{} Explain why integration does not suffer from the presence of {\it noise} in the signal the way that differentiation does.
\item{} Explain why the net amount of work done in a mechanical system is guaranteed to be zero when the limits of integration are equal to each other.
\item{} Explain how integration applies to the calculation of work done by a piston-and-cylinder heat engine.
\end{itemize}



%INDEX% Reading assignment: Lessons In Industrial Instrumentation, calculus (numerical integration)

%(END_NOTES)


