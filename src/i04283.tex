
%(BEGIN_QUESTION)
% Copyright 2009, Tony R. Kuphaldt, released under the Creative Commons Attribution License (v 1.0)
% This means you may do almost anything with this work of mine, so long as you give me proper credit

Read and outline the ``Proportional-Only Offset'' section of the ``Closed-Loop Control'' chapter in your {\it Lessons In Industrial Instrumentation} textbook.  Note the page numbers where important illustrations, photographs, equations, tables, and other relevant details are found.  Prepare to thoughtfully discuss with your instructor and classmates the concepts and examples explored in this reading.

\underbar{file i04283}
%(END_QUESTION)





%(BEGIN_ANSWER)


%(END_ANSWER)





%(BEGIN_NOTES)

It is the job of any loop controller to counter-act {\it loads}, which are defined as uncontrolled variables affecting the process variable we're trying to maintain at setpoint.

\vskip 10pt

Thought experiment where we predict the response of a propotional-only controller to decreased inlet temp:

\item{} Outlet temperature drops
\item{} Controller senses drop in PV, opens up steam valve more
\item{} More steam (more heat) to exchanger
\item{} Temperature stabilizes, but at a value less than before
\item{} P action cannot re-establish PV=SP because that would put steam valve back to original position
\item{} This deviation (PV $-$ SP) is called ``offset,'' or ``droop''
\end{itemize}

``Droop'' is misleading: offset can go other way, too.  Example: inlet temp rises!

\vskip 10pt

Operational amplifiers also require slight offset (differential input voltage) to generate any non-zero output voltage, since $V_{out} = A_V(V_{in(+)} - V_{in(-)})$.

\vskip 10pt

Increasing the controller's gain results in less offset, because the PV doesn't have to deviate as much in order to generate the necessary change in output to stabilize.  Too much gain, though, and the process will oscillate!

\vskip 10pt

A better solution is ``integral'' (reset) action!







\vskip 20pt \vbox{\hrule \hbox{\strut \vrule{} {\bf Suggestions for Socratic discussion} \vrule} \hrule}

\begin{itemize}
\item{} Give your own example of proportional-only offset other than a heat exchanger or a cruise control system in a car.
\itemitem{} Room temperature control on a cold day with people opening doors and windows
\itemitem{} Room temperature control on a hot day with people opening doors and windows
\itemitem{} Liquid level control in a vessel with a wild incoming flow
\itemitem{} Liquid flow control with a wild pump pressure
\itemitem{} pH control with a wild influent composition
\item{} Explain in your own words {\it why} an offset develops between PV and SP in a proportional-only control system experiencing a load change.
\item{} Explain in your own words {\it why} an offset develops between PV and SP in a proportional-only control system experiencing a setpoint change.
\item{} If the gain of a proportional-only controller is increased, will offset increase or decrease?  Explain your answer.
\end{itemize}













\vfil \eject

\noindent
{\bf Prep Quiz:}

The definition of a {\it load} with regard to process control loops is:

\begin{itemize}
\item{} The time lag between a change in output and a change seen in the PV
\vskip 5pt 
\item{} A variable affecting the PV, that is itself unregulated by the control system
\vskip 5pt 
\item{} A drain of energy on a system, causing it to operate inefficiently
\vskip 5pt 
\item{} A device that dissipates energy in a circuit, as opposed to sourcing energy to the circuit
\vskip 5pt 
\item{} The value at which the control system attempts to stabilize the PV over time
\vskip 5pt 
\item{} The multiplication factor of a process, measured from output to input
\end{itemize}








\vfil \eject

\noindent
{\bf Prep Quiz:}

The definition of {\it proportional-only offset} is:

\begin{itemize}
\item{} A persistent error between PV and SP which proportional action cannot eliminate
\vskip 5pt 
\item{} The time between successive updates (scans) of a digital proportional-only controller
\vskip 5pt 
\item{} The amount of valve stem offset required when a controller is proportional-only
\vskip 5pt 
\item{} The mathematical difference between a controller's PV and output signals
\vskip 5pt 
\item{} The amount you need to adjust the SP value to achieve stability over time
\vskip 5pt 
\item{} The time required for a proportional-only controller to recover from a load change
\end{itemize}


%INDEX% Reading assignment: Lessons In Industrial Instrumentation, closed-loop control (proportional-only offset)

%(END_NOTES)


