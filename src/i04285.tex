
%(BEGIN_QUESTION)
% Copyright 2009, Tony R. Kuphaldt, released under the Creative Commons Attribution License (v 1.0)
% This means you may do almost anything with this work of mine, so long as you give me proper credit

\vbox{\hrule \hbox{\strut \vrule{} {\bf Desktop Process exercise} \vrule} \hrule}

\noindent
Configure the controller as follows (for ``proportional-only'' control):

\begin{itemize}
\item{} Control action = {\it reverse}
\item{} Gain = {\it set to whatever value yields optimal control}
\item{} Reset (Integral) = {\it minimum effect} = {\it 100+ minutes/repeat} = {\it 0 repeats/minute}
\item{} Rate (Derivative) = {\it minimum effect} = {\it 0 minutes} 
\end{itemize}

After determining the ``ideal'' gain setting value for your process controller yielding the swiftest possible response to setpoint changes without oscillation, place the controller in manual mode and adjust the output until the process variable is approximately 50\%.  Now, place the controller back into automatic mode.  Thanks to the ``setpoint tracking'' feature programmed into most digital electronic controllers, the SP should be precisely equal to the PV (i.e. no error).

\vskip 10pt

Now, increase the setpoint by 10\% and watch closely to observe the new value that the process variable settles at.  Does this new PV value exactly match the new SP value?  Try increasing the SP by 10\% again and re-observe the PV's new value once it settles.  Is the error greater or less than before?

Explain how {\it proportional-only offset} (otherwise known as ``droop'') accounts for the error you are seeing in this experiment.

\vskip 10pt

Try demonstrating proportional-only offset by moving the SP to values less than original (below 50\%).  How does the amount of offset compare to the offset exhibited with higher SP values?

\vskip 20pt \vbox{\hrule \hbox{\strut \vrule{} {\bf Suggestions for Socratic discussion} \vrule} \hrule}

\begin{itemize}
\item{} Do your best to explain to your teammates {\it why} an offset develops between PV and SP in a proportional-only controller.
\item{} Explain why proportional-only offset is a concern to us at all.  Can you think of an application where such offset would be intolerable?  Can you think of an application where it would not matter at all?
\end{itemize}

\underbar{file i04285}
%(END_QUESTION)





%(BEGIN_ANSWER)


%(END_ANSWER)





%(BEGIN_NOTES)

{\bf Lesson:} proportional-only offset is directly proportional to how far setpoint is moved away from equilibrium (where PV = SP).


%INDEX% Desktop Process: automatic control of motor speed (demonstrating P-only offset)

%(END_NOTES)


