
%(BEGIN_QUESTION)
% Copyright 2009, Tony R. Kuphaldt, released under the Creative Commons Attribution License (v 1.0)
% This means you may do almost anything with this work of mine, so long as you give me proper credit

\vbox{\hrule \hbox{\strut \vrule{} {\bf Desktop Process exercise} \vrule} \hrule}

\noindent
Configure the controller as follows (for ``proportional-only'' control):

\begin{itemize}
\item{} Control action = {\it reverse}
\item{} Gain = {\it set to whatever value yields optimal control}
\item{} Reset (Integral) = {\it minimum effect} = {\it 100+ minutes/repeat} = {\it 0 repeats/minute}
\item{} Rate (Derivative) = {\it minimum effect} = {\it 0 minutes} 
\end{itemize}

In this exercise, you will experiment with the effect of different controller gain (proportional band) settings on the amount of proportional-only offset developed.  First, place the controller in manual mode and adjust the output until the process variable is approximately 50\%.  Now, place the controller back into automatic mode.  Thanks to the ``setpoint tracking'' feature programmed into most digital electronic controllers, the SP should be precisely equal to the PV (i.e. no error).

\vskip 10pt

Decrease the controller's gain setting (increase the proportional band) by a factor of two.  Increase the setpoint by 10\% and watch closely to observe the new value that the process variable settles at.  Note the amount of proportional-only offset exhibited by the controller.  Use manual mode once again to stabilize the process variable at approximately 50\% and switch back to automatic mode.

\vskip 10pt

Decrease the controller's gain setting (increase the proportional band) once more by a factor of two.  Increase the setpoint by 10\% and watch closely to observe the new value that the process variable settles at.  Note the amount of proportional-only offset exhibited by the controller.  

\vskip 10pt

\vskip 20pt \vbox{\hrule \hbox{\strut \vrule{} {\bf Suggestions for Socratic discussion} \vrule} \hrule}

\begin{itemize}
\item{} Generalizing to all proportional controllers, explain the effect of a decreased controller gain on the amount of proportional offset that develops for any given process conditions.
\item{} Generalizing to all proportional controllers, explain the effect of an increased controller gain on the amount of proportional offset that develops for any given process conditions.
\end{itemize}

\underbar{file i04287}
%(END_QUESTION)





%(BEGIN_ANSWER)


%(END_ANSWER)





%(BEGIN_NOTES)

{\bf Lesson:} proportional-only offset is inversely proportional to controller gain.

%INDEX% Desktop Process: automatic control of motor speed (effect of gain on P-only offset)

%(END_NOTES)


