
%(BEGIN_QUESTION)
% Copyright 2009, Tony R. Kuphaldt, released under the Creative Commons Attribution License (v 1.0)
% This means you may do almost anything with this work of mine, so long as you give me proper credit

In their seminal 1942 paper {\it Optimum Settings for Automatic Controllers}, J.G. Ziegler and N.B. Nichols describe the compromise that must be struck when adjusting the gain (``sensitivity'') of a controller having only proportional action:

\vskip 10pt {\narrower \noindent \baselineskip5pt

\noindent
``The rational adjustment of proportional-response sensitivity is then simply a matter of balancing the two evils of offset and amplitude ratio.'' (page 761)

\par} \vskip 10pt

Ziegler and Nichols used the phrase ``amplitude ratio'' to describe the severity of oscillations following a sudden change in setpoint or load.  The ``amplitude ratio'' of an oscillation was a measure of each successive peak's height compared to the previous peak.  A large amplitude ratio therefore referred to oscillations requiring many cycles to dampen, while a small amplitude ratio referred to oscillations dampening in very short order.

\vskip 10pt

Describe this balancing act between the ``two evils'' of offset and oscillation while adjusting the gain setting on a process controller, making reference to your own experiences of adjusting gain settings on process controllers.

\vskip 20pt \vbox{\hrule \hbox{\strut \vrule{} {\bf Suggestions for Socratic discussion} \vrule} \hrule}

\begin{itemize}
\item{} Being that the solution to proportional-only offset is to use {\it integral} action in the loop controller in addition to proportional action, why do you suppose Ziegler and Nichols even bothered to suggest finding a compromise between low gain and high gain?  Why not just suggest the use of integral action as a universal solution for the ``two evils'' of offset and amplitude ratio?
\end{itemize}

\underbar{file i04288}
%(END_QUESTION)





%(BEGIN_ANSWER)


%(END_ANSWER)





%(BEGIN_NOTES)

Too little gain and offset becomes severe.  Too much gain and oscillations become severe.  The ``balancing'' necessary between these twin evils should be obvious.

\vskip 10pt

In case anyone asks, the reason this quote is found on page {\it 761} is not that Ziegler and Nichols' paper is hundreds of pages long.  It is because their ten-page paper was published in a larger journal, ``Transactions of the American Society of Mechanical Engineers''.











\vfil \eject

\noindent
{\bf Summary Quiz:}

Excessive gain programmed into a loop controller will cause the loop to behave in what way?

\begin{itemize}
\item{} The PV will respond too slowly to changes in SP
\vskip 5pt 
\item{} The output of the controller will remain at zero
\vskip 5pt 
\item{} The proportional-only offset (``droop'') will be large
\vskip 5pt 
\item{} The dead time of the process will be excessive
\vskip 5pt 
\item{} The control valve will exhibit too much ``stiction''
\vskip 5pt 
\item{} The PV will tend to oscillate rather than hold steady
\end{itemize}

%INDEX% Control, proportional: proportional-only offset

%(END_NOTES)


