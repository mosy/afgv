
%(BEGIN_QUESTION)
% Copyright 2009, Tony R. Kuphaldt, released under the Creative Commons Attribution License (v 1.0)
% This means you may do almost anything with this work of mine, so long as you give me proper credit

I deres banebrytende artikkel fra 1942, {\it Optimum Settings for Automatic Controllers}, beskriver J.G. Ziegler og N.B. Nichols kompromisset som må inngås når man justerer forsterkningen ("følsomheten") til en regulator som kun har proporsjonalvirkning:

\vskip 10pt {\narrower \noindent \baselineskip5pt

\noindent
"The rational adjustment of proportional-response sensitivity is then simply a matter of balancing the two evils of offset and amplitude ratio." (side 761)

\par} \vskip 10pt

Ziegler og Nichols brukte uttrykket "amplitude ratio" (amplitudeforhold) for å beskrive alvorlighetsgraden av svingninger etter en plutselig endring i settpunkt eller belastning. "Amplitudeforholdet" til en svingning var et mål på hver påfølgende topps høyde sammenlignet med den forrige toppen. Et stort amplitudeforhold refererte derfor til svingninger som krevde mange sykluser for å dempes, mens et lite amplitudeforhold refererte til svingninger som dempet seg veldig raskt.

\vskip 10pt

Beskriv denne balansegangen mellom de "to onder" avvik (offset) og svingning (oscillation) når man justerer forsterkningsinnstillingen på en prosessregulator, gjerne ved å referere til dine egne erfaringer med å justere forsterkning på prosessregulatorer.

\vskip 20pt \vbox{\hrule \hbox{\strut \vrule{} {\bf Forslag til sokratisk diskusjon} \vrule} \hrule}

\begin{itemize}
\item{} Gitt at løsningen på proporsjonal-avvik er å bruke {\it integral}-virkning i sløyferegulatoren i tillegg til proporsjonalvirkning, hvorfor tror du Ziegler og Nichols i det hele tatt gadd å foreslå å finne et kompromiss mellom lav og høy forsterkning? Hvorfor ikke bare foreslå bruk av integralvirkning som en universell løsning for de "to onder" avvik og amplitudeforhold?
\end{itemize}

\underbar{file i04288}
%(END_QUESTION)





%(BEGIN_ANSWER)


%(END_ANSWER)





%(BEGIN_NOTES)

For liten forsterkning fører til stort avvik. For mye forsterkning fører til alvorlige svingninger. Den nødvendige "balanseringen" mellom disse tvilling-ondene burde være åpenbar.

\vskip 10pt

I tilfelle noen spør, grunnen til at dette sitatet finnes på side {\it 761} er ikke at Ziegler og Nichols' artikkel er hundrevis av sider lang. Det er fordi deres ti-siders artikkel ble publisert i et større tidsskrift, "Transactions of the American Society of Mechanical Engineers".











\vfil \eject

\noindent
{\bf Summary Quiz:}

Overdreven forsterkning programmert inn i en sløyferegulator vil føre til at sløyfen oppfører seg på hvilken måte?

\begin{itemize}
\item{} PV vil reagere for sakte på endringer i SP
\vskip 5pt 
\item{} Utgangen fra regulatoren vil forbli på null
\vskip 5pt 
\item{} Det stasjonære avviket (P-avvik / "droop") vil være stort
\vskip 5pt 
\item{} Dødtiden i prosessen vil bli overdreven
\vskip 5pt 
\item{} Kontrollventilen vil utvise for mye "stiction" (friksjon)
\vskip 5pt 
\item{} PV vil ha en tendens til å svinge (oscillere) i stedet for å holde seg stabil
\end{itemize}

%INDEX% Control, proportional: proportional-only offset

%(END_NOTES)
