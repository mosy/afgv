
%(BEGIN_QUESTION)
% Copyright 2009, Tony R. Kuphaldt, released under the Creative Commons Attribution License (v 1.0)
% This means you may do almost anything with this work of mine, so long as you give me proper credit

\vbox{\hrule \hbox{\strut \vrule{} {\bf Desktop Process exercise} \vrule} \hrule}

\noindent
Configure the controller as follows:

\begin{itemize}
\item{} Control action = {\it reverse}
\item{} Gain = {\it very low} = {\it very large proportional band value}
\item{} Reset (Integral) = {\it set to whatever value yields optimal control}
\item{} Rate (Derivative) = {\it minimum effect} = {\it 0 minutes} 
\end{itemize}

Get the process into a condition where the PV is holding at approximately 50\% in automatic mode, then turn off the electric power to the final control element (e.g. the power to the motor drive).  Observe the response of the controller's output to the sudden change in PV.  Describe what you see here in terms of {\it integral windup}.

\vskip 10pt

After the controller's output has ``wound up'' in response to the halted process, turn the power back on to the final control element and observe the control response.  Does the PV immediately settle at its proper setpoint value?  Is there ``overshoot'' or ``undershoot'' of the PV in relation to SP following the power-up?  Determine whether or not changes in the reset (integral) setting will improve the quality of control following a power outage.

\vskip 20pt \vbox{\hrule \hbox{\strut \vrule{} {\bf Suggestions for Socratic discussion} \vrule} \hrule}

\begin{itemize}
\item{} Will the controller's integral action ``wind up'' while in manual mode?  Explain why or why not.
\end{itemize}

\underbar{file i04291}
%(END_QUESTION)





%(BEGIN_ANSWER)


%(END_ANSWER)





%(BEGIN_NOTES)

{\bf Lesson:} forcing integral (reset) wind-up to occur and observing its effects.

%INDEX% Desktop Process: automatic control of motor speed (demonstration of integral windup)

%(END_NOTES)


