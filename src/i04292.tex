
%(BEGIN_QUESTION)
% Copyright 2009, Tony R. Kuphaldt, released under the Creative Commons Attribution License (v 1.0)
% This means you may do almost anything with this work of mine, so long as you give me proper credit

\vbox{\hrule \hbox{\strut \vrule{} {\bf Desktop Process exercise} \vrule} \hrule}

Tune the controller in your Desktop Process first using proportional action only (reset and rate settings at minimum effect), testing the control quality by observing the graph produced by the data acquisition software.  Your goal is quick response to setpoint changes with minimal oscillation of the PV.

\vskip 10pt

After determining a reasonable value for the controller's gain setting, incorporate some reset (integral) action in order to eliminate offset following a setpoint or load change.  Once again, your goal is quick response with minimal oscillation and overshoot of the PV.

\vskip 10pt

Record these gain and reset settings for future use.

\underbar{file i04292}
%(END_QUESTION)





%(BEGIN_ANSWER)


%(END_ANSWER)





%(BEGIN_NOTES)

{\bf Lesson:} experimenting with proportional+integral control, contrasting this against proportional-only control.  The difference in control quality should be obvious once good P+I values are chosen.








\vfil \eject

\noindent
{\bf Summary Quiz:}

(An alternative to a summary quiz is to have students demonstrate their Desktop Process units operating in automatic mode with P and I actions set for good control behavior)



%INDEX% Desktop Process: automatic control of motor speed (simple P+I tuning)

%(END_NOTES)


