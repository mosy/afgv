
%(BEGIN_QUESTION)
% Copyright 2009, Tony R. Kuphaldt, released under the Creative Commons Attribution License (v 1.0)
% This means you may do almost anything with this work of mine, so long as you give me proper credit

Suppose an operator wishes to have a computer display the total amount of water consumed by a process for each week of continuous operation.  A flowmeter presently exists on the water pipe measuring the flow rate of water into the process.  Identify whether the computer will need to {\it differentiate} or {\it integrate} the flowmeter signal to calculate a total water quantity for the operator.  Provide example units of measurement for the flowmeter and for the total water quantity. 

\vskip 10pt

Suppose an operator wishes to have a computer signal an alarm if a chemical reactor heats or cools too quickly.  A thermocouple presently exists on the reactor vessel to measure process temperature.  Identify whether the computer will need to {\it differentiate} or {\it integrate} the thermocouple signal in order to activate the alarm.  Provide example units of measurement for the thermocouple and for the temperature alarm setpoint.

\vskip 20pt \vbox{\hrule \hbox{\strut \vrule{} {\bf Suggestions for Socratic discussion} \vrule} \hrule}

\begin{itemize}
\item{} Mathematically express each of these computations, using proper calculus notation.
\item{} Explain how {\it units of measurement} are especially in determining when to apply differentiation, versus when to apply integration.
\item{} If you chose British units of measurement for your answer, identify some suitable metric units of measurement instead.
\item{} If you chose metric units of measurement for your answer, identify some suitable British units of measurement instead.
\end{itemize}

\underbar{file i04294}
%(END_QUESTION)





%(BEGIN_ANSWER)

\noindent
{\bf Partial answer:}

\vskip 10pt

Totalizing water flow = {\it integration}.  Water flow rate in {\it gallons per minute}; water quantity in {\it gallons}.

%(END_ANSWER)





%(BEGIN_NOTES)

Temperature rate alarm = {\it differentiation}.  Temperature in {\it degrees Celsius}; temperature rate in {\it degrees Celsius per hour}.

%INDEX% Mathematics, calculus: integral and derivative related to process measurement

%(END_NOTES)


