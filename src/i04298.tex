
%(BEGIN_QUESTION)
% Copyright 2009, Tony R. Kuphaldt, released under the Creative Commons Attribution License (v 1.0)
% This means you may do almost anything with this work of mine, so long as you give me proper credit

Read and outline the ``Integral Control Action'' subsection of the ``Pneumatic PID Controllers'' section of the ``Closed-Loop Control'' chapter in your {\it Lessons In Industrial Instrumentation} textbook.  Note the page numbers where important illustrations, photographs, equations, tables, and other relevant details are found.  Prepare to thoughtfully discuss with your instructor and classmates the concepts and examples explored in this reading.

\underbar{file i04298}
%(END_QUESTION)





%(BEGIN_ANSWER)


%(END_ANSWER)





%(BEGIN_NOTES)

Adding a ``reset bellows'' opposing the proportional feedback bellows in a pneumatic controller mechanism, fed air from the output line through a needle valve restrictor, adds integral action to the controller's behavior.  If ever there is an imbalance between PV and SP, there must exist an imbalance between these two other bellows (feedback and reset), with the fast-changing proportional feedback bellows having to ``stay ahead'' of the time-delayed reset bellows.  Thus, for a constant error, the output will ramp at a constant rate.





\vskip 20pt \vbox{\hrule \hbox{\strut \vrule{} {\bf Suggestions for Socratic discussion} \vrule} \hrule}

\begin{itemize}
\item{} Explain in detail the ``thought experiment'' describing how integral action works in a simple pneumatic controller mechanism.
\item{} Explain how the pneumatic controller mechanism would respond if the reset restrictor valve were suddenly shut off while an error persisted between PV and SP.
\item{} Explain how the pneumatic controller mechanism would respond if the reset restrictor valve were suddenly opened wide while an error persisted between PV and SP.
\item{} Add a capacity tank to the controller mechanism shown in the book -- the tank being connected to the reset bellows -- and ask what effect(s) the addition of this tank will have on the controller's behavior.  ({\it Integral action slows down!})
\item{} Add a capacity tank to the controller mechanism shown in the book -- the tank being connected to the output (proportional) bellows -- and ask what effect(s) the addition of this tank will have on the controller's behavior.  ({\it No effect!})
\item{} An advanced question to pose to students who have studied laminar versus turbulent flow regimes is ``why is it important that the restrictor valves in a pneumatic controller be designed to support {\it laminar} flow rather than {\it turbulent} flow?  The answer is that the relationship between pressure drop and flow rate will be linear for laminar flow, but quadratic for turbulent flow.  If the valves' flow regimes were turbulent, a doubling of error (PV $-$ SP) would {\it not} result in a doubling of integration rate, but rather would only result in the output ramping $\sqrt{2}$ times faster.
\end{itemize}














\vfil \eject

\noindent
{\bf Prep Quiz:}

If we {\it open up} the ``reset valve'' by a small amount in the pneumatic controller shown in the book, the integral action will:

\begin{itemize}
\item{} Slow down (i.e. greater $\tau_i$)
\vskip 5pt 
\item{} Completely stop (i.e. $\tau_i = \infty$)
\vskip 5pt 
\item{} Speed up (i.e. lesser $\tau_i$)
\vskip 5pt 
\item{} Immediately saturate the output (i.e. $\tau_i = 0$)
\vskip 5pt 
\item{} Not change at all
\end{itemize}














\vfil \eject

\noindent
{\bf Prep Quiz:}

If we {\it close down} the ``reset valve'' by a small amount in the pneumatic controller shown in the book, the integral action will:

\begin{itemize}
\item{} Slow down (i.e. greater $\tau_i$)
\vskip 5pt 
\item{} Completely stop (i.e. $\tau_i = \infty$)
\vskip 5pt 
\item{} Speed up (i.e. lesser $\tau_i$)
\vskip 5pt 
\item{} Immediately saturate the output (i.e. $\tau_i = 0$)
\vskip 5pt 
\item{} Not change at all
\end{itemize}













\vfil \eject

\noindent
{\bf Prep Quiz:}

Suppose a technician is bench-testing a loop controller, setting the setpoint to 40\% and sending a 50\% process variable signal (12 mA) to its input.  With this constant error between SP and PV, how will the integral action of the controller react?

\begin{itemize}
\item{} The output signal will {\it ramp} at an ever-increasing rate (like a curved ``ski jump'' on a graph)
\vskip 5pt 
\item{} The output signal will {\it step down} by an amount equal to the error 
\vskip 5pt 
\item{} The output signal will hold steady at 50\% (like a straight and level line on a graph)
\vskip 5pt 
\item{} The output signal will {\it step up} by an amount equal to the error
\vskip 5pt 
\item{} The output signal will {\it ramp} at a constant rate (like a straight slope on a graph)
\vskip 5pt 
\item{} The output signal will {\it oscillate} back and forth (like a sine wave on a graph)
\end{itemize}


%INDEX% Reading assignment: Lessons In Industrial Instrumentation, closed-loop control (pneumatic PID controllers, integral control action)

%(END_NOTES)


