
%(BEGIN_QUESTION)
% Copyright 2009, Tony R. Kuphaldt, released under the Creative Commons Attribution License (v 1.0)
% This means you may do almost anything with this work of mine, so long as you give me proper credit

Read and outline the ``Summary of PID Control Terms'' section of the ``Closed-Loop Control'' chapter in your {\it Lessons In Industrial Instrumentation} textbook.  Note the page numbers where important illustrations, photographs, equations, tables, and other relevant details are found.  Prepare to thoughtfully discuss with your instructor and classmates the concepts and examples explored in this reading.

\vskip 20pt \vbox{\hrule \hbox{\strut \vrule{} {\bf Suggestions for Socratic discussion} \vrule} \hrule}

\begin{itemize}
\item{} Which of the three PID control actions (P, I, or D) acts {\it on the future}?
\item{} Which of the three PID control actions (P, I, or D) acts {\it on the present}?
\item{} Which of the three PID control actions (P, I, or D) acts {\it on the past}?
\end{itemize}

\underbar{file i04299}
%(END_QUESTION)





%(BEGIN_ANSWER)


%(END_ANSWER)





%(BEGIN_NOTES)

P action is {\it punctuality} -- it operates immediately.  Equal to Delta-Out / Delta-In.
P action is where error tells the valve how far to move.

\vskip 10pt

I action is {\it impatience} -- it operates continually when error != 0.  Equal to output velocity / input error.
I action is where error tells the valve how fast to move.

\vskip 10pt

D action is {\it discretion} -- it works to minimize fast changes in PV.  Equal to output offset / input velocity.
D action is where error's ramp rate tells valve how far to move.






\vskip 20pt \vbox{\hrule \hbox{\strut \vrule{} {\bf Suggestions for Socratic discussion} \vrule} \hrule}

\begin{itemize}
\item{} Summarize the three control actions (P, I, and D) in your own words.
\end{itemize}




%INDEX% Reading assignment: Lessons In Industrial Instrumentation, closed-loop control (PID term summary)

%(END_NOTES)


