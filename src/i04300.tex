
%(BEGIN_QUESTION)
% Copyright 2009, Tony R. Kuphaldt, released under the Creative Commons Attribution License (v 1.0)
% This means you may do almost anything with this work of mine, so long as you give me proper credit

Read and outline the ``P, I, and D responses graphed'' section of the ``Closed-Loop Control'' chapter in your {\it Lessons In Industrial Instrumentation} textbook.  Note the page numbers where important illustrations, photographs, equations, tables, and other relevant details are found.  Prepare to thoughtfully discuss with your instructor and classmates the concepts and examples explored in this reading.

\underbar{file i04300}
%(END_QUESTION)





%(BEGIN_ANSWER)


%(END_ANSWER)





%(BEGIN_NOTES)

\vskip 20pt \vbox{\hrule \hbox{\strut \vrule{} {\bf Suggestions for Socratic discussion} \vrule} \hrule}

\begin{itemize}
\item{} Explain {\it why} the output trend has the shape that it does, for different examples shown in this section.
\item{} {\it Instructor: propose an alteration in the PV's shape or duration for any of the control actions graphed in this section of the book, and then ask students to predict how the output trend will change as a result.}
\item{} {\it Instructor: propose a different shape or duration for the output trend on any of the control actions graphed in this section of the book, and then ask students to predict how the PV must differ to produce that output trend.}
\end{itemize}


%INDEX% Reading assignment: Lessons In Industrial Instrumentation, closed-loop control (P, I, and D responses graphed)

%(END_NOTES)


