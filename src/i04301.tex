
%(BEGIN_QUESTION)
% Copyright 2016, Tony R. Kuphaldt, released under the Creative Commons Attribution License (v 1.0)
% This means you may do almost anything with this work of mine, so long as you give me proper credit

Read and outline the ``Derivative (Rate) Control'' section of the ``Closed-Loop Control'' chapter in your {\it Lessons In Industrial Instrumentation} textbook.  Note the page numbers where important illustrations, photographs, equations, tables, and other relevant details are found.  Prepare to thoughtfully discuss with your instructor and classmates the concepts and examples explored in this reading.

\vskip 20pt \vbox{\hrule \hbox{\strut \vrule{} {\bf Further exploration . . . (optional)} \vrule} \hrule}

The now-famous paper on PID controller tuning written by Ziegler and Nichols in 1942 contains many useful insights into behavior of the basic PID control algrithm.  Here is one of them:

\vskip 10pt {\narrower \noindent \baselineskip5pt

Pre-act response does not replace automatic reset response since it ceases to act when the pen becomes stationary.  However, while reset increases period of oscillation and decreases stability, the effect of pre-act is just the opposite.  On the debit side for pre-act lies only the increased difficulty of adjusting three responses instead of two, but the use of the basic unit, pre-act time, allows the setting to be determined from the period of oscillation.

\par} \vskip 10pt

Describe what this passage is saying, in your own words.  Do you see any unfamiliar terms, which you may determine the meaning of from context?  What insight(s) can you gather about the use of the integral control mode, as well as the derivative control mode?

\vskip 10pt


\underbar{file i04301}
%(END_QUESTION)





%(BEGIN_ANSWER)


%(END_ANSWER)





%(BEGIN_NOTES)

Derivative action works to limit how fast the error will change.  Fast rates of PV/SP change cause the output to compensate in response, taking a ``cautious'' approach.

\vskip 10pt

Derivative control action is useful for slow-responding processes.  Also called ``pre-act''.

\vskip 10pt

Some controllers only have P + D actions (and no I).  Some controller algorithms calculate derivative action on error, others on PV only (to avoid output spikes when SP suddenly changes).  Judicious use of D action allows for more-aggressive P and I actions with less overshoot.

\vskip 10pt

D action ``goes wild'' when faced with a noisy signal (large rates-of-change), and should be used sparingly.






\vskip 20pt \vbox{\hrule \hbox{\strut \vrule{} {\bf Suggestions for Socratic discussion} \vrule} \hrule}

\begin{itemize}
\item{} Identify a practical application for derivative control action in an industrial process.
\item{} Explain why derivative action is sometimes referred to as {\it pre-act}.
\item{} Explain why some process controllers calculate derivative action based on rates-of-change of {\it PV} rather than rates-of-change of {\it error}.
\item{} Explain why a noisy PV signal is incompatible with derivative control action.
\end{itemize}


















\vfil \eject

\noindent
{\bf Prep Quiz:}

Derivative (rate) action in a process loop controller responds directly to:

\begin{itemize}
\item{} The accumulated product of error (PV $-$ SP) and time
\vskip 5pt 
\item{} The amount of ``pre-act'' inherent to the process
\vskip 5pt 
\item{} How hot the process temperature is at any given moment 
\vskip 5pt 
\item{} The frequency that the PV oscillates
\vskip 5pt 
\item{} The rate that the PV (or error) changes over time
\vskip 5pt 
\item{} Direct operator control in manual mode
\end{itemize}



\vfil \eject

\noindent
{\bf Prep Quiz:}

Derivative (rate) action in a process loop controller responds very poorly to:

\begin{itemize}
\item{} High-frequency ``noise'' on the PV signal
\vskip 5pt 
\item{} Small process variable values
\vskip 5pt 
\item{} Sudden changes in setpoint (SP) value
\vskip 5pt 
\item{} Substantial offset between PV and SP
\vskip 5pt 
\item{} Large process variable values 
\vskip 5pt 
\item{} Adjustments made by an operator in manual mode
\end{itemize}


%INDEX% Reading assignment: Lessons In Industrial Instrumentation, closed-loop control (Derivative (rate) control)

%(END_NOTES)


