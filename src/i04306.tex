
%(BEGIN_QUESTION)
% Copyright 2009, Tony R. Kuphaldt, released under the Creative Commons Attribution License (v 1.0)
% This means you may do almost anything with this work of mine, so long as you give me proper credit

\vbox{\hrule \hbox{\strut \vrule{} {\bf Desktop Process exercise} \vrule} \hrule}

Configure your Desktop Process for proportional-only control, where there is negligible integral or derivative control action.  Experiment with different ``gain'' values until reasonably good control is obtained from the process (i.e. fast response to setpoint changes with minimal ``overshoot,'' good recovery from load changes).  Record the ``optimum'' gain setting you find for your process, for future reference.

\vskip 10pt

Identify and demonstrate some of the disadvantages of proportional-only control.

\vskip 20pt \vbox{\hrule \hbox{\strut \vrule{} {\bf Suggestions for Socratic discussion} \vrule} \hrule}

\begin{itemize}
\item{} Does your controller use {\it gain} or {\it proportional band} as the unit for Proportional action?
\item{} Does your controller use {\it repeats per minute}, {\it repeats per second}, {\it minutes per repeat}, or {\it seconds per repeat} as the unit for Integral action?  In each case, what would be considered a suitable value to yield ``negligible'' action?
\item{} Does your controller use {\it minutes} or {\it seconds} as the unit for Derivative action?  In each case, what would be considered a suitable value to yield ``negligible'' action?
\end{itemize}

\underbar{file i04306}
%(END_QUESTION)





%(BEGIN_ANSWER)


%(END_ANSWER)





%(BEGIN_NOTES)

{\bf Lesson:} finding the right gain for a proportional-only controller is a compromise between crisp response and loop stability.  This exercise also forces students to think about what constitutes ``negligible'' I and D actions.


%INDEX% Desktop Process: P-only control

%(END_NOTES)


