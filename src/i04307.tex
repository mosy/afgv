
%(BEGIN_QUESTION)
% Copyright 2009, Tony R. Kuphaldt, released under the Creative Commons Attribution License (v 1.0)
% This means you may do almost anything with this work of mine, so long as you give me proper credit

\vbox{\hrule \hbox{\strut \vrule{} {\bf Desktop Process exercise} \vrule} \hrule}

Configure your Desktop Process for proportional-plus-integral (P+I) control, where there is negligible derivative control action.  Experiment with different ``gain'' and ``reset'' tuning parameter values until reasonably good control is obtained from the process (i.e. fast response to setpoint changes with minimal ``overshoot,'' good recovery from load changes).  Record the ``optimum'' P and I settings you find for your process, for future reference.

\vskip 10pt

Identify and demonstrate how the addition of integral control action to proportional control action overcomes some of the limitations of proportional-only control.

\vskip 20pt \vbox{\hrule \hbox{\strut \vrule{} {\bf Suggestions for Socratic discussion} \vrule} \hrule}

\begin{itemize}
\item{} Does your controller use {\it repeats per time} or {\it time per repeat} as the unit for Integral action?  What is the specific unit of time used (minutes, seconds)?
\item{} Calculate the number of seconds per repeat of integral action you ended up using for good P+I control of the Desktop Process motor speed.  How does this amount of time compare with the amount of time the motor physically takes to achieve a new speed value following a manual step-change in the controller's output?
\end{itemize}

\underbar{file i04307}
%(END_QUESTION)





%(BEGIN_ANSWER)


%(END_ANSWER)





%(BEGIN_NOTES)

{\bf Lesson:} finding the right integral action for a P+I controller.  This is also a good review of why integral control action is useful.  Another important lesson is how the inclusion of integral control action may allow different amounts of gain than if the controller were P-only.


%INDEX% Desktop Process: P+I control

%(END_NOTES)


