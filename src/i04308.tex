
%(BEGIN_QUESTION)
% Copyright 2009, Tony R. Kuphaldt, released under the Creative Commons Attribution License (v 1.0)
% This means you may do almost anything with this work of mine, so long as you give me proper credit

\vbox{\hrule \hbox{\strut \vrule{} {\bf Desktop Process exercise} \vrule} \hrule}

Configure your Desktop Process for proportional-plus-derivative (P+D) control, where there is negligible integral control action.  Experiment with different ``gain'' and ``rate'' tuning parameter values until reasonably good control is obtained from the process (i.e. fast response to setpoint changes with minimal ``overshoot,'' good recovery from load changes).  Record the ``optimum'' P and D settings you find for your process, for future reference.

\vskip 10pt

Identify and demonstrate how the addition of derivative control action to proportional control action overcomes some of the limitations of proportional-only control.

\vskip 20pt \vbox{\hrule \hbox{\strut \vrule{} {\bf Suggestions for Socratic discussion} \vrule} \hrule}

\begin{itemize}
\item{} Does your controller use {\it minutes} or {\it seconds} as the unit for Derivative action?
\item{} What happens when you configure a loop controller with too much derivative action?  How is this effect different from (or similar to) what happens when a controller has too much gain, or too much integral action?
\item{} Does derivative control action work to eliminate offset like integral action does?  How can you tell?
\end{itemize}

\underbar{file i04308}
%(END_QUESTION)





%(BEGIN_ANSWER)


%(END_ANSWER)





%(BEGIN_NOTES)

{\bf Lesson:} finding the right derivative action for a P+D controller.  This is also a good way to see whether or not derivative control action is useful for a particular process.  Another important lesson is how the inclusion of derivative control action may allow different amounts of gain than if the controller were P-only.


%INDEX% Desktop Process: P+D control

%(END_NOTES)


