
%(BEGIN_QUESTION)
% Copyright 2009, Tony R. Kuphaldt, released under the Creative Commons Attribution License (v 1.0)
% This means you may do almost anything with this work of mine, so long as you give me proper credit

\vbox{\hrule \hbox{\strut \vrule{} {\bf Desktop Process exercise} \vrule} \hrule}

Configure your Desktop Process for full proportional-plus-integral-plus-derivative (PID) control.  Experiment with different ``gain,'' ``reset,'' and ``rate'' tuning parameter values until reasonably good control is obtained from the process (i.e. fast response to setpoint changes with minimal ``overshoot,'' good recovery from load changes).  Record the ``optimum'' P, I, and D settings you find for your process, for future reference.

\vskip 10pt

Compare the optimum PID tuning parameter values you arrived at compared to those of your classmates.

\vskip 20pt \vbox{\hrule \hbox{\strut \vrule{} {\bf Suggestions for Socratic discussion} \vrule} \hrule}

\begin{itemize}
\item{} How do the P, I, and D settings (when all used together to achieve optimum control) compare to the P setting by itself found to yield optimum proportional-only control, or the P and I settings found to yield optimum PI control, or the P and D settings found to yield optimum PD control?
\end{itemize}

\underbar{file i04309}
%(END_QUESTION)





%(BEGIN_ANSWER)


%(END_ANSWER)





%(BEGIN_NOTES)

{\bf Lesson:} finding the right integral and derivative actions for a PID controller.  Another important lesson is how the inclusion of integral and derivative control actions may allow different amounts of gain than if the controller were P-only.








\vfil \eject

\noindent
{\bf Summary Quiz:}

(An alternative to a summary quiz is to have students demonstrate their Desktop Process units operating in automatic mode with P, I, and D actions set for good control behavior)



%INDEX% Desktop Process: P+I+D control

%(END_NOTES)


