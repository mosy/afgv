
%(BEGIN_QUESTION)
% Copyright 2009, Tony R. Kuphaldt, released under the Creative Commons Attribution License (v 1.0)
% This means you may do almost anything with this work of mine, so long as you give me proper credit

Read and outline the ``Full-PID Circuit Design'' subsection of the ``Analog Electronic PID Controllers'' section of the ``Closed-Loop Control'' chapter in your {\it Lessons In Industrial Instrumentation} textbook.  Note the page numbers where important illustrations, photographs, equations, tables, and other relevant details are found.  Prepare to thoughtfully discuss with your instructor and classmates the concepts and examples explored in this reading.

\underbar{file i04311}
%(END_QUESTION)





%(BEGIN_ANSWER)


%(END_ANSWER)





%(BEGIN_NOTES)

Two analog controller circuits are shown in this section, one implementing the ``ideal'' PID equation and the other implementing the ``series'' PID equation.  The ``series'' PID controller circuit has fewer components.


$$m = K_p \left(e + {1 \over \tau_i} \int e \> dt + \tau_d {de \over dt} \right) \hbox{\hskip 50pt {\bf Ideal PID equation}}$$


$$m = K_p \left[ \left({\tau_d \over \tau_i} + 1 \right) e + {1 \over \tau_i} \int e \> dt + \tau_d {de \over dt} \right] \hbox{\hskip 25pt {\bf Series} or {\bf Interacting PID equation}}$$







\vskip 20pt \vbox{\hrule \hbox{\strut \vrule{} {\bf Suggestions for Socratic discussion} \vrule} \hrule}

\begin{itemize}
\item{} Identify which way the potentiometer wiper should be moved to make (P, I, D) action more aggressive in one of the controller circuits, and also explain {\it why}.
\item{} Predict the effect(s) of various component value changes or component faults in the full PID controller circuit shown in the textbook.
\item{} Explain why neither PID equation shown in the text contains a bias ($b$) term.
\end{itemize}



%INDEX% Reading assignment: Lessons In Industrial Instrumentation, closed-loop control (PID analog electronic control)

%(END_NOTES)


