
%(BEGIN_QUESTION)
% Copyright 2009, Tony R. Kuphaldt, released under the Creative Commons Attribution License (v 1.0)
% This means you may do almost anything with this work of mine, so long as you give me proper credit

Read and outline the ``Instrument Identification Tags'' section of the ``Instrumentation Documents'' chapter in your {\it Lessons In Industrial Instrumentation} textbook.  Note the page numbers where important illustrations, photographs, equations, tables, and other relevant details are found.  Prepare to thoughtfully discuss with your instructor and classmates the concepts and examples explored in this reading.

\vskip 20pt \vbox{\hrule \hbox{\strut \vrule{} {\bf Suggestions for Socratic discussion} \vrule} \hrule}

\begin{itemize}
\item{} Review the tips listed in Question 0 and apply them to this reading assignment.
\end{itemize}

\underbar{file i04312}
%(END_QUESTION)





%(BEGIN_ANSWER)


%(END_ANSWER)





%(BEGIN_NOTES)

ISA standard 5.1 defines tags: letters for function, number for loop ID:

\begin{itemize}
\item{} FC-135 (flow controller in loop 135)
\item{} 12-FC-135 (flow controller in loop 135, in unit area 12)
\end{itemize}

First letter of tag for each instrument in a loop must be the same, based on variable measured by transmitter.  Second letter may be a modifier to that first letter (PD.. = differential pressure).  Remaining letters describe function of instrument (``PT'' = pressure transmitter ; ``PI'' = pressure indicator).  If multiple letters are used for functions, first letter is passive (manual) while second letter is active (automatic).

\vskip 10pt

User-defined letters (e.g. B, C, D, G, M, N, O) are used more than once in the same diagram.  Unclassified letter (X) used only once (or a very limited number of times) in the same diagram.








\vskip 20pt \vbox{\hrule \hbox{\strut \vrule{} {\bf Suggestions for Socratic discussion} \vrule} \hrule}

\begin{itemize}
\item{} Explain how the instrument tag names shown in your lab project loop diagram conform to the ISA 5.1 standard.
\item{} What criterion dictates the {\it first letter} of any ISA-standard instrument tag in a control loop?
\item{} Referencing your own loop diagram, identify some alternative letterings that could be applied to each instrument (e.g. PC or PRC instead of PIC).
\item{} Where might you choose to use an {\it unclassified} letter to represent an instrument's process variable as opposed to a {\it user-defined} letter?
\end{itemize}









\vfil \eject

\noindent
{\bf Prep Quiz:}

\vskip 10pt

\noindent
\vbox{\hrule \hbox{\strut \vrule{} {Part A -- multiple-choice} \vrule} \hrule}
Identify the proper ISA ``tag'' for a {\it flow controller} that also {\it indicates} the process variable to human operators:

\begin{itemize}
\item{} FII
\vskip 5pt 
\item{} FCC
\vskip 5pt 
\item{} FQC
\vskip 5pt 
\item{} FIC
\vskip 5pt 
\item{} FKC
\end{itemize}

\vskip 20pt

\noindent
\vbox{\hrule \hbox{\strut \vrule{} {Part B -- written response} \vrule} \hrule}
Identify where you can locate the Instrumentation program calendar listing all course-specific events and dates.  Note that there are multiple locations where the calendar resides, but you only need to identify one of them!

\vskip 20pt

\noindent
\vbox{\hrule \hbox{\strut \vrule{} {Part C -- written response} \vrule} \hrule}
Identify where you can locate the grade spreadsheet file for all second-year Instrumentation courses.  Note that there are multiple locations where the spreadsheet resides, but you only need to identify one of them!

\vskip 20pt

{\it Note: your explanations need to be \underbar{complete} and \underbar{clearly written}.  Expressing your ideas clearly and completely is every bit as important as having those ideas correct in your own mind!}


%INDEX% Reading assignment: Lessons In Industrial Instrumentation, Instrumentation Documents (tag names)

%(END_NOTES)


