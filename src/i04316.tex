
%(BEGIN_QUESTION)
% Copyright 2009, Tony R. Kuphaldt, released under the Creative Commons Attribution License (v 1.0)
% This means you may do almost anything with this work of mine, so long as you give me proper credit

Read and outline the introduction and the ``Self-Regulating Processes'' subsections of the ``Process Characteristics'' section of the ``Process Dynamics and PID Controller Tuning'' chapter in your {\it Lessons In Industrial Instrumentation} textbook.  Note the page numbers where important illustrations, photographs, equations, tables, and other relevant details are found.  Prepare to thoughtfully discuss with your instructor and classmates the concepts and examples explored in this reading.

\underbar{file i04316}
%(END_QUESTION)





%(BEGIN_ANSWER)


%(END_ANSWER)





%(BEGIN_NOTES)

The three major process characteristic types are {\it self-regulating}, {\it integrating}, and {\it runaway}.  Each of these process types has its own unique PID tuning requirements, and so it is best to identify the type of process you're tuning the controller for prior to making any tuning adjustments.

Process characteristics are best probed by performing {\it open-loop} tests, where the loop controller is placed in {\it manual} mode and step-changes made to the output.  The response of the process variable to these manual-mode step-changes reveal important process characteristics.  If the controller is left in automatic mode, the PV's response will be an amalgam of the process characterisics and the controller's tuning, which is \underbar{not} what we're looking for.  


\vskip 10pt


If a flow-control valve is suddenly moved, flow tends to self-stabilize at a new value.

\vskip 10pt

Definition of a self-regulating process: the PV will self-stabilize whenever the FCE or load changes.  This may happen fast or slow.  These processes exhibit a unique PV value for every FCE value.  This means a unique FCE value will be required to achieve any different PV.

\begin{itemize}
\item{} P-only controllers will experience offset when trying to control a self-regulating process.
\item{} We can minimize offset with more controller gain, but too much makes it oscillate.
\item{} Integral controller action is needed to eliminate offset in a self-regulating process.
\item{} Self-regulating processes naturally stabilize following a change in MV or load.
\end{itemize}

\vskip 10pt

\noindent
{\bf Summary:}

\item{} Integral controller action is absolutely required to achieve PV=SP for self-reg processes!
\item{} Faster integral control action eliminates error quicker, but too much may cause oscillation.
\item{} Amount of integral action tolerable depends on the amount of time lag in process.
\end{itemize}











\vskip 20pt \vbox{\hrule \hbox{\strut \vrule{} {\bf Suggestions for Socratic discussion} \vrule} \hrule}

\begin{itemize}
\item{} Explain why a loop controller must be placed in {\it manual} mode (i.e. ``open-loop'') to test the characteristics of the process.
\item{} Explain how we can tell the controller is in manual mode solely based on an examination of the trend graph.
\item{} Explain what a self-regulating process is and how it reacts.
\item{} Explain why proportional-only controller action experiences offset following SP changes when controlling a self-regulating process.
\item{} Explain why integral control action is necessary to eliminate offset in a self-regulating process.
\item{} Examine the open-loop trend shown for the flow control system, and determine whether we will need to configure the controller for direct or reverse action.
\end{itemize}




















\vfil \eject

\noindent
{\bf Prep Quiz:}

A common industrial example of a {\it self-regulating} process with very rapid response time is:

\begin{itemize}
\item{} Temperature control
\vskip 5pt 
\item{} Liquid flow control
\vskip 5pt 
\item{} Solids level control
\vskip 5pt 
\item{} Liquid level control
\vskip 5pt 
\item{} Gas pressure control
\vskip 5pt 
\item{} Chemical reaction control
\end{itemize}

%INDEX% Reading assignment: Lessons In Industrial Instrumentation, process characteristics (self-regulating)

%(END_NOTES)


