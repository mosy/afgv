
%(BEGIN_QUESTION)
% Copyright 2009, Tony R. Kuphaldt, released under the Creative Commons Attribution License (v 1.0)
% This means you may do almost anything with this work of mine, so long as you give me proper credit

Read and outline the ``Runaway Processes'' subsection of the ``Process Characteristics'' section of the ``Process Dynamics and PID Controller Tuning'' chapter in your {\it Lessons In Industrial Instrumentation} textbook.  Note the page numbers where important illustrations, photographs, equations, tables, and other relevant details are found.  Prepare to thoughtfully discuss with your instructor and classmates the concepts and examples explored in this reading.

\underbar{file i04318}
%(END_QUESTION)





%(BEGIN_ANSWER)


%(END_ANSWER)





%(BEGIN_NOTES)

Definition of a runaway process: then the PV {\it accelerates} following FCE or load changes, naturally avoiding any condition of equilibrium.

\vskip 10pt

A Segway transporter is an example of this, being a kind of {\it inverted pendulum} -- the textbook example of a runaway process.  Some exothermic chemical reactions are runaway as well.  Runaway processes are characterized by intrinsic {\it positive} feedback, and are the most challenging to control.

\vskip 10pt

Runaway processes become self-regulating if enough natural negative feedback is added to it (e.g. water-moderated fission reactor).  Nuclear fission inherently runaway (chain reaction).  Water coolant acts to stabilize reaction by reducing the multiplication factor as it gets hotter (less dense).  Chernobyl reactor an example of runaway characteristic (in some operating conditions).

\vskip 10pt

Runaway processes require {\it derivative} controller action to stabilize.  Integral action is also necessary to compensate for load changes, as in other process types (self-regulating and integrating).

\vskip 10pt


\noindent
{\bf Summary:}

\item{} Runaway processes naturally accelerate following a change in MV or load.
\item{} Accelerating characteristic caused by natural positive feedback.
\item{} Derivative controller action is necessary to stabilize a runaway process -- proportional and integral controller actions are insufficient on their own.
\item{} Some integral controller action is necessary to compensate for load changes in a runaway process.
\item{} Runaway may become self regulating if enough natural negative feedback is present (e.g. water-moderated fission reactor).
\end{itemize}
















\vskip 20pt \vbox{\hrule \hbox{\strut \vrule{} {\bf Suggestions for Socratic discussion} \vrule} \hrule}

\begin{itemize}
\item{} Explain why a loop controller must be placed in {\it manual} mode to test the characteristics of the process.
\item{} Explain how we can tell the controller is in manual mode solely based on an examination of the trend graph.
\item{} Explain why derivative controller action is needed to stabilize a runaway process.  What is it about derivative action that is so useful here?
\item{} Examine the open-loop trend shown for the inverted pendulum process, and determine whether we will need to configure the controller for direct or reverse action.
\item{} Explain why nuclear fission is an inherently runaway process.
\item{} Explain how pressurized water fission reactors achieve self-regulation.
\end{itemize}



















\vfil \eject

\noindent
{\bf Prep Quiz:}

A classic example of a {\it runaway} process is:

\begin{itemize}
\item{} Motor speed 
\vskip 5pt 
\item{} Furnace temperature
\vskip 5pt 
\item{} Liquid level
\vskip 5pt 
\item{} Boiling liquid temperature
\vskip 5pt 
\item{} Inverted pendulum balance
\vskip 5pt 
\item{} Endothermic chemical reaction
\end{itemize}


%INDEX% Reading assignment: Lessons In Industrial Instrumentation, process characteristics (runaway)

%(END_NOTES)


