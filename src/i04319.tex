
%(BEGIN_QUESTION)
% Copyright 2009, Tony R. Kuphaldt, released under the Creative Commons Attribution License (v 1.0)
% This means you may do almost anything with this work of mine, so long as you give me proper credit

Read and outline the ``Steady-State Process Gain'' subsection of the ``Process Characteristics'' section of the ``Process Dynamics and PID Controller Tuning'' chapter in your {\it Lessons In Industrial Instrumentation} textbook.  Note the page numbers where important illustrations, photographs, equations, tables, and other relevant details are found.  Prepare to thoughtfully discuss with your instructor and classmates the concepts and examples explored in this reading.

\underbar{file i04319}
%(END_QUESTION)





%(BEGIN_ANSWER)


%(END_ANSWER)





%(BEGIN_NOTES)

Every component within a feedback control loop has its own gain, because each component in a feedback loop has its own output and its own input, with a ratio relating change of one to change of the other:

$$\hbox{Gain} = {\Delta \hbox{Output} \over \Delta \hbox{Input}}$$

\vskip 10pt

To measure the gain of a process, you place the loop controller in manual mode, then step-change the final control element (FCE), seeing how far the PV changes following this FCE change.

$$\hbox{Process gain} = {\Delta \hbox{PV} \over \Delta \hbox{Controller output}}$$

Process gain is affected by gain (span) of transmitter, and also by the gain ($C_v$) of the control valve.

\vskip 10pt

If the process gain changes, {\it all} actions of controller need to be changed as well to compensate.  This is why the Ideal and Series PID equations are preferable to the Parallel equation: changing $K_p$ changes all actions proportionately with a single controller adjustment.

\vskip 10pt

Process gains may not be consistent over a wide range of FCE values.  Inconsistent gains may be compensated for by using control valve with different characteristic (equal-percent, linear, etc.).








\vskip 20pt \vbox{\hrule \hbox{\strut \vrule{} {\bf Suggestions for Socratic discussion} \vrule} \hrule}

\begin{itemize}
\item{} If a control valve's $C_v$ is decreased, what needs to happen to the loop controller's aggressiveness in order to maintain robust control quality?  How exactly can we decrease the $C_v$ of a control valve?
\item{} If the measurement span of a process transmitter is increased (e.g. 0-100 PSI becomes 0-300 PSI), what needs to happen to the loop controller's aggressiveness in order to maintain robust control quality? 
\item{} If gain is defined as {\it output} change over {\it input} change, then why when we perform an open-loop test of process gain do we divide {\it PV} change by {\it Output} change?
\item{} How do we identify {\it variable} (inconsistent) process gain?  What is problematic about variable process gain?  How do we tune a controller for a process like this?
\end{itemize}







%INDEX% Reading assignment: Lessons In Industrial Instrumentation, process characteristics (process gain)

%(END_NOTES)


