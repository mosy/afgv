
%(BEGIN_QUESTION)
% Copyright 2009, Tony R. Kuphaldt, released under the Creative Commons Attribution License (v 1.0)
% This means you may do almost anything with this work of mine, so long as you give me proper credit

In order to explore different process characteristics, it is useful to have access to loop simulation software you can run on your own personal computer.  A ``loop simulation'' program mimics the behavior of a real process, allowing you to make tuning changes to a PID controller and see the results in a trend graph.  

Some programs exist (for free download or streaming simulation) on the Internet allowing you to do this from your own personal computer.  Research some of the available software options and try downloading at least one of them.

\vskip 10pt

Be sure to bring your portable computer to class with you -- ideally with the software already installed -- for today's classroom activities!

\underbar{file i04320}
%(END_QUESTION)





%(BEGIN_ANSWER)

One of the better simulations I've found is from {\it Dex Automation}.  You can run the free Java application streaming on the website or you can pay a small fee for a stand-alone executable to run offline.

%(END_ANSWER)





%(BEGIN_NOTES)




\vfil \eject

\noindent
{\bf Prep Quiz:}

Provide the name of (at least one of) the loop simulation software packages you found when you researched free loop simulation software on the Internet.

%INDEX% Control, process characteristics: computer simulation software

%(END_NOTES)

