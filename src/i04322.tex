
%(BEGIN_QUESTION)
% Copyright 2009, Tony R. Kuphaldt, released under the Creative Commons Attribution License (v 1.0)
% This means you may do almost anything with this work of mine, so long as you give me proper credit

Read and outline the ``Lag Time'' subsection of the ``Process Characteristics'' section of the ``Process Dynamics and PID Controller Tuning'' chapter in your {\it Lessons In Industrial Instrumentation} textbook.  Note the page numbers where important illustrations, photographs, equations, tables, and other relevant details are found.  Prepare to thoughtfully discuss with your instructor and classmates the concepts and examples explored in this reading.

\underbar{file i04322}
%(END_QUESTION)





%(BEGIN_ANSWER)


%(END_ANSWER)





%(BEGIN_NOTES)

Low-pass RC filter circuits re-shape a square-wave input to have a ``sawtooth'' shape, because the charging and discharging of the capacitor forces the output signal to ``lag'' behind the input.  Process instruments such as pressure transmitters do this, too.  Self-regulating processes also exhibit the same ``sawtooth'' tendency when stimulated in a square-wave fashion due to intrinsic lag.  

Any physical system exhibiting this ``settling'' behavior over time may be described by a differential equation, where the rate of change of one variable over time is directly related to the instantaneous value of that same variable.  Newton's Cooling Law is an example of a first-order differential equation, describing the rate of cooling of a hot object:

$${dT \over dt} = -k (T - T_{ambient})$$

From this differential equation we may derive a solution where the system exhibits a {\it time constant} ($\tau$), describing the amount of time necessary for the process variable to change 63.2\% of the way toward its final value:

$$T = \left( T_{initial} - T_{final} \right) \left( e^{-{t \over \tau}} \right) + T_{final}$$

This is precisely the same concept as time constant in RC and LR circuits, where voltage and current values grow and decay over time according to an inverse exponential function.

\vskip 10pt

{\it Lag time} is the process control synonym for {\it time constant}, meaning the exact same thing.  While the 63.2\% change definition for time constant applies to the response to a step-change (square wave), the response of the same system to a linear ramping input gives more meaning to the word ``lag.''  When stimulated with a linear ramping input, a first-order lag system's output signal will lag behind by $\tau$ time.












\vskip 20pt \vbox{\hrule \hbox{\strut \vrule{} {\bf Suggestions for Socratic discussion} \vrule} \hrule}

\begin{itemize}
\item{} Interpret Newton's Cooling Law equation, explaining what each term in the equation represents.
\item{} If the hot object in Newton's Cooling Law were to be better insulated, what would change in the differential equation?
\item{} If the mass in Newton's Cooling Law were to start out at a higher temperature, what would change in the differential equation?
\item{} Explain how ``time constant'' is defined in the response of a system to a step-change input (i.e. 63.2\%).
\item{} Explain how ``lag time'' is defined in the response of a system to a ramping input.
\item{} {\bf Describe a procedure for measuring the amount of lag time in a control loop.}
\end{itemize}














\vfil \eject

\noindent
{\bf Prep Quiz:}

Calculate the time constant ($\tau$) of an RC circuit having a resistance of 10 k$\Omega$ and a capacitance of 33 $\mu$F:

\begin{itemize}
\item{} 303 seconds
\vskip 5pt 
\item{} 19.8 seconds
\vskip 5pt 
\item{} 33 milli-seconds
\vskip 5pt 
\item{} 0.33 seconds
\vskip 5pt 
\item{} 3.3 nano-seconds
\vskip 5pt 
\item{} 33 micro-seconds
\end{itemize}











\vfil \eject

\noindent
{\bf Prep Quiz:}

Calculate the time constant ($\tau$) of an RC circuit having a resistance of 1 k$\Omega$ and a capacitance of 27 $\mu$F:

\begin{itemize}
\item{} 2.7 micro-seconds
\vskip 5pt 
\item{} 27 micro-seconds
\vskip 5pt 
\item{} 2.7 milli-seconds
\vskip 5pt 
\item{} 0.27 seconds
\vskip 5pt 
\item{} 27 milli-seconds
\vskip 5pt 
\item{} 27 nano-seconds
\end{itemize}


%INDEX% Reading assignment: Lessons In Industrial Instrumentation, process characteristics (lag time)

%(END_NOTES)


