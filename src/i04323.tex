
%(BEGIN_QUESTION)
% Copyright 2009, Tony R. Kuphaldt, released under the Creative Commons Attribution License (v 1.0)
% This means you may do almost anything with this work of mine, so long as you give me proper credit

Read and outline the ``Multiple Lags (Orders)'' subsection of the ``Process Characteristics'' section of the ``Process Dynamics and PID Controller Tuning'' chapter in your {\it Lessons In Industrial Instrumentation} textbook.  Note the page numbers where important illustrations, photographs, equations, tables, and other relevant details are found.  Prepare to thoughtfully discuss with your instructor and classmates the concepts and examples explored in this reading.

\underbar{file i04323}
%(END_QUESTION)





%(BEGIN_ANSWER)


%(END_ANSWER)





%(BEGIN_NOTES)

A potato placed in an oven illustrates multiple orders of lag.  The oven air itself is a first-order lag process.  The potato adds another order of lag, forming a second-order process from the perspective of the heating element.

Second-order lags are fundamentally different from first-order lags, not just longer in time.  The characteristic shape of a first-order lag response to a stepped input shows a crisp beginning and an asymptotic ending.  The characteristic shape of a second-order lag response shows a slow beginning and an asymptotic ending.  Also, if the input of a second-order lag function is a true square wave (stepping both directions), the output will have rounded peaks which continue a bit before reversing direction rather than sharp peaks which reverse as soon as the input stimulus reverses.  This explains why second-order lag processes tend to overshoot setpoint more than first-order lag processes.

\vskip 10pt

Multiple orders of lag may be intrinsic to the process itself (e.g. cascaded tanks draining into the next tank), or may be introduced by the instruments (e.g. transmitter damping, control valve sluggishness).

\vskip 10pt

The more lags in a process (i.e. the greater the ``order''), the more disconnected the process variable becomes from the controller output over time.  This manifests itself as {\it phase shift} when the PV and output are oscillating.  The negative feedback essential to any feedback control system is 180$^{o}$, but if enough process lags accumulate to add another 180$^{o}$ of shift, the total loop phase shift will become 360$^{o}$ and self-sustaining oscillations may result.  This principle is exploited in RC phase shift oscillator circuits, where multiple RC phase-shift networks are used to add 180$^{o}$ of shift to an amplifier that is already inverting (exhibits its own 180$^{o}$ phase shift).  The fact that you cannot make an RC oscillator with just one RC phase-shift network is proof that multiple orders of lag are more detrimental to control stability than single lags.

Purely first-order lag processes will not self-oscillate even with enormous amounts of controller gain, because they lack the phase shift necessary to regenerate their own oscillations.












\vskip 20pt \vbox{\hrule \hbox{\strut \vrule{} {\bf Suggestions for Socratic discussion} \vrule} \hrule}

\begin{itemize}
\item{} Explain how we may be able to distinguish a multiple-order lag process from a first-order lag process based on trend graphs.
\item{} Explain how instruments may add multiple orders of lag to a process that is inherently first-order.
\item{} Explain why multiple orders of lag are detrimental to feedback control.
\item{} Explain why one order of lag in a process -- no matter how long -- is better for feedback control than multiple orders of lag.
\item{} Explain what the {\it Barkhausen criterion} is, and how this relates to loop stability, phase shift, and gain.
\item{} Identify some practical sources of lag (especially multiple orders of lag) in a process control loop.
\end{itemize}















\vfil \eject

\noindent
{\bf Prep Quiz:}

Identify which of these process characteristics is most susceptible to oscillation when controlled by negative feedback:

\begin{itemize}
\item{} No lag time whatsoever
\vskip 5pt 
\item{} A short, first-order lag
\vskip 5pt 
\item{} A long, first-order lag
\vskip 5pt 
\item{} A short, second-order lag
\vskip 5pt 
\item{} A long, second-order lag
\end{itemize}

%INDEX% Reading assignment: Lessons In Industrial Instrumentation, process characteristics (multiple lags)

%(END_NOTES)


