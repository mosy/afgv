
%(BEGIN_QUESTION)
% Copyright 2009, Tony R. Kuphaldt, released under the Creative Commons Attribution License (v 1.0)
% This means you may do almost anything with this work of mine, so long as you give me proper credit

Read and outline the ``Hysteresis'' subsection of the ``Process Characteristics'' section of the ``Process Dynamics and PID Controller Tuning'' chapter in your {\it Lessons In Industrial Instrumentation} textbook.  Note the page numbers where important illustrations, photographs, equations, tables, and other relevant details are found.  Prepare to thoughtfully discuss with your instructor and classmates the concepts and examples explored in this reading.

\underbar{file i04325}
%(END_QUESTION)





%(BEGIN_ANSWER)


%(END_ANSWER)





%(BEGIN_NOTES)

{\it Hysteresis} is a lack of response to a change in direction.  A sticky control valve is an excellent example of hysteresis: the valve does not immediately change direction when the signal changes direction.  Hysteresis may be detected by performing open-loop (manual mode) step-changes that reverse direction.

\vskip 10pt

Hysteresis manifests as dead time to a closed-loop feedback process.  When the MV signal reverses direction and the PV fails to respond at all, it is seen as a ``dead'' process to the controller.  The amount of dead time resulting from hysteresis is a function of how far the MV signal moves.

\vskip 10pt

Integral control action will continually ``hunt'' to try to stabilize the PV at SP when there is hysteresis in the loop.  When caused by a sticky control valve, this is known as a {\it slip-stick cycle}.  In a self-regulating process, this cycle is characterized by a square-wave PV trend and a triangle-wave output trend.  In an integrating process, both the PV and the output oscillations are triangular in shape.  Slip-stick cycles cannot be eliminated through tuning (unless one competely eliminates integral controller action in a self-regulating process), but may only be corrected by eliminating the hysteresis in the valve.










\vskip 20pt \vbox{\hrule \hbox{\strut \vrule{} {\bf Suggestions for Socratic discussion} \vrule} \hrule}

\begin{itemize}
\item{} Define ``hysteresis'' in general terms.
\item{} Identify some practical sources of hysteresis in a process control loop.
\item{} Describe a procedure for measuring the amount of hysteresis in a control loop.
\item{} Explain what a {\it slip-stick cycle} is, and what causes it to happen.
\item{} Explain why a {\it slip-stick cycle} cannot be completely eliminated by controller tuning in a self-regulating process, without removing integral action entirely.
\item{} Explain why a {\it slip-stick cycle} cannot be completely eliminated by controller tuning in an integrating process at all.
\end{itemize}



%INDEX% Reading assignment: Lessons In Industrial Instrumentation, process characteristics (hysteresis)

%(END_NOTES)


