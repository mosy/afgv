
%(BEGIN_QUESTION)
% Copyright 2009, Tony R. Kuphaldt, released under the Creative Commons Attribution License (v 1.0)
% This means you may do almost anything with this work of mine, so long as you give me proper credit

Use loop simulation software on a personal computer to simulate the effects of an ``ultimate gain'' test on a particular process loop.  The idea here is to set the loop controller for proportional-only action (no integral or derivative actions) and incrementally increase the controller gain until the loop sustains sinusoidal oscillations.  The controller gain value at which the loop self-oscillates is called the {\it ultimate gain} ($K_u$).  The period of those oscillations is called the {\it ultimate period} ($P_u$).

\vskip 10pt

Select a loop from those offered in the software's library, and record the results of this test.

\vskip 10pt

Ultimate gain of loop ($K_u$) = \underbar{\hskip 50pt} 

\vskip 10pt

Ultimate period of loop ($P_u$) = \underbar{\hskip 50pt} 

\vskip 10pt

Now, explain how we may use these measured values to determine good P, I, and D settings for a loop controller.


\vskip 20pt \vbox{\hrule \hbox{\strut \vrule{} {\bf Suggestions for Socratic discussion} \vrule} \hrule}

\begin{itemize}
\item{} Explain the procedure you would follow to perform a closed-loop test like this.
\item{} Explain why it is important to avoid a {\it limit cycle} when performing a closed-loop test.
\item{} How practical do you think this tuning procedure is?  Explain your answer in detail.
\item{} In the hilarious book {\it How to Become an Instrument Engineer -- The Making of a Prima Donna}, authors Greg McMillan and Stanley Weiner have an entire chapter called ``How To Tune Controllers'' in which they strongly recommend the Ziegler-Nichols closed-loop method.  Benefits of this method cited by the authors include (1) the inclusion of all nonlinearities and controller characteristics in the test, (2) no need for complex interpretation of the trend graph, (3) no need to switch controller modes (auto to manual and vice-versa), (4) applicability to integrating and runaway processes, and (5) more tolerance of intermittent disturbances during the test.  Comment on these advantages, especially as they compare against the Ziegler-Nichols open-loop method.
\end{itemize}

\underbar{file i04327}
%(END_QUESTION)





%(BEGIN_ANSWER)


%(END_ANSWER)





%(BEGIN_NOTES)

The measured $K_u$ and $P_u$ values may be plugged into the formulae given by Ziegler and Nichols to calculate $K_p$, $\tau_i$, and/or $\tau_d$:

\vskip 10pt

\noindent
{\bf For a P-only controller:}

$$K_p = 0.5 K_u$$

\vskip 10pt

\noindent
{\bf For a PI controller:}

$$K_p = 0.45 K_u \hskip 30pt \tau_i = {P_u \over 1.2}$$

\vskip 10pt

\noindent
{\bf For a PID controller:}

$$K_p = 0.6 K_u \hskip 30pt \tau_i = {P_u \over 2} \hskip 30pt \tau_d = {P_u \over 8}$$

%INDEX% Control, process characteristics: computer simulation software (lag and dead times)

%(END_NOTES)

