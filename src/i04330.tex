
%(BEGIN_QUESTION)
% Copyright 2009, Tony R. Kuphaldt, released under the Creative Commons Attribution License (v 1.0)
% This means you may do almost anything with this work of mine, so long as you give me proper credit

Read and outline all subsections in the ``Heuristic PID Tuning Procedures'' section of the ``Process Dynamics and PID Controller Tuning'' chapter in your {\it Lessons In Industrial Instrumentation} textbook.  Note the page numbers where important illustrations, photographs, equations, tables, and other relevant details are found.  Prepare to thoughtfully discuss with your instructor and classmates the concepts and examples explored in this reading.

\vskip 10pt

In particular, you should write your own step-by-step instructions for implementing a heuristic tuning method, so you will have a concise reference to apply to later loop tuning challenges:

\begin{itemize}
\item{} 
\vskip 10pt
\item{} 
\vskip 10pt
\item{} 
\vskip 10pt
\item{} 
\end{itemize}


\underbar{file i04330}
%(END_QUESTION)





%(BEGIN_ANSWER)


%(END_ANSWER)





%(BEGIN_NOTES)

``Heuristic'' tuning of a PID loop means to adjust the P, I, and/or D settings of a controller in some methodical fashion as its closed-loop response is noted on a trend.  Rather than relying on formulae to determine optimum tuning parameter values, heuristic methods rely on the comprehension of the instrument technician analyzing the trend and deciding whether each action is too aggressive or not aggressive enough.

\vskip 10pt

\noindent
A simple heuristic procedure is as follows:

\item{} Determine important process characteristics (self-reg vs. integrating vs. runaway; dead time; lag time and order; instrument problems) from an open-loop test.
\item{} Identify any problematic characteristics (e.g. noise)
\item{} Choose the best dominant controller action (P or I)
\item{} Set all PID parameters to minimum effect
\item{} Increase dominant parameter until best control is achieved
\item{} Increase aggressiveness of other actions as needed
\item{} Identify excessive action by phase shift analysis between PV and Output
\item{} Repeat last steps as often as needed
\end{itemize}

\vskip 10pt

Uses and pitfalls of P, I, and D actions listed.  Recommendations given based on shape of open-loop PV response to Output step-change.

\vskip 10pt

The amount of phase shift seen between an oscillating PV and its controller Output is useful in determining the dominant action of the controller.  If the phase shift is at or near zero (180$^{o}$ for a reverse-acting controller), then the dominant action is Proportional.  If the phase shift is lagging (Output lags behind PV), then the dominant action is Integral.  If the phase shift is leading (Output leads PV), then the dominant action is Derivative.  Note that ``lead'' and ``lag'' must be defined relative to the controller action: noting which way the Output wave is shifted from the PV wave compared to what it should be given the controller's direction of action.  For a direct-acting controller, we compare like peaks.  For a reverse-acting controller, we compare opposite peaks (e.g. positive peak of PV versus negative peak of Outuput).














\vskip 20pt \vbox{\hrule \hbox{\strut \vrule{} {\bf Suggestions for Socratic discussion} \vrule} \hrule}

\begin{itemize}
\item{} Describe the primitive PID tuning technique I learned when I was in school, then describe a better heuristic approach to PID tuning than that.
\item{} Explain the rationale for some of the General Rules following the open-loop process dynamics table.
\item{} {\bf Explain the purpose of each and every step in your bulleted tuning instructions.}
\item{} Describe how we may use {\it phase shift} analysis to discern which action of a controller (P, I, or D) is causing oscillations.
\item{} Explain what {\it porpoising} is in a control loop, and why it is always a bad thing.
\item{} Explain why porposing cannot be caused by excessive integral action, but only by excessive proportional or derivative action.
\end{itemize}

%INDEX% Reading assignment: Lessons In Industrial Instrumentation, PID tuning (heuristic)

%(END_NOTES)


