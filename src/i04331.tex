
%(BEGIN_QUESTION)
% Copyright 2009, Tony R. Kuphaldt, released under the Creative Commons Attribution License (v 1.0)
% This means you may do almost anything with this work of mine, so long as you give me proper credit

Read and outline the ``Identifying Operational Needs'' subsection of the ``Before You Tune . . .'' section of the ``Process Dynamics and PID Controller Tuning'' chapter in your {\it Lessons In Industrial Instrumentation} textbook.  Note the page numbers where important illustrations, photographs, equations, tables, and other relevant details are found.  Prepare to thoughtfully discuss with your instructor and classmates the concepts and examples explored in this reading.

\underbar{file i04331}
%(END_QUESTION)





%(BEGIN_ANSWER)


%(END_ANSWER)





%(BEGIN_NOTES)

Is tight control to setpoint essential?  Which is more important: reduced error over time or quick response to loads?  Priorities which you should rank:

\begin{itemize}
\item{} Minimum change in PV with load changes (use P and/or D)
\item{} Fast response to SP changes (use P and/or D)
\item{} Minimum oscillation (temper P\&I, use D)
\item{} Minimum error over time (must use I!)
\item{} Minimum valve velocity (temper P\&D, use I)
\end{itemize}

Multi-effect evaporator level control is a good example where we should minimize control valve velocity even at the expense of larger (and/or longer) setpoint deviations.  Sudden changes in the out-flow rate of one evaporator will cause the downstream evaporator(s) to experience fluctuating loads.






\vskip 20pt \vbox{\hrule \hbox{\strut \vrule{} {\bf Suggestions for Socratic discussion} \vrule} \hrule}

\begin{itemize}
\item{} Explain why crisp response to setpoint and load changes might {\it not} be the optimal response of a loop controller.  Give at least one practical example in industry.
\end{itemize}


%INDEX% Reading assignment: Lessons In Industrial Instrumentation, PID tuning (identifying operational needs)

%(END_NOTES)


