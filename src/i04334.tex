
%(BEGIN_QUESTION)
% Copyright 2009, Tony R. Kuphaldt, released under the Creative Commons Attribution License (v 1.0)
% This means you may do almost anything with this work of mine, so long as you give me proper credit

Read and outline the ``Identifying the Problem(s)'' subsection of the ``Before You Tune . . .'' section of the ``Process Dynamics and PID Controller Tuning'' chapter in your {\it Lessons In Industrial Instrumentation} textbook.  Note the page numbers where important illustrations, photographs, equations, tables, and other relevant details are found.  Prepare to thoughtfully discuss with your instructor and classmates the concepts and examples explored in this reading.

\underbar{file i04334}
%(END_QUESTION)





%(BEGIN_ANSWER)


%(END_ANSWER)





%(BEGIN_NOTES)

Always try to determine how long the reported problem has existed.  If it's a recent problem in a previously healthy system, you may narrow the probability of faults to those which can suddenly occur.  On the other hand, a problem that's been present since system construction may very well be a design flaw or some other serious error.

\vskip 10pt

Look for process problems revealed in an open-loop test (controller in manual mode, stepping the control element and monitoring PV response):

\begin{itemize}
\item{} Noise in the PV signal
\item{} Control element hysteresis (e.g. valve ``stiction'')
\item{} Process characteristic (e.g. self-regulating, integrating, or runaway)
\item{} Process gain
\item{} Process lag magnitude and order
\item{} Process dead time
\end{itemize}

\vskip 10pt

If no trend recorder is available in the process, you may connect a data acquisition (DAQ) device to sense instrument signal voltages and plot them on a computer screen.






\vskip 20pt \vbox{\hrule \hbox{\strut \vrule{} {\bf Suggestions for Socratic discussion} \vrule} \hrule}

\begin{itemize}
\item{} Describe procedures by which you could positively detect each of the listed loop problems (e.g. valve stiction, uneven process gain)
\end{itemize}







\vfil \eject

\noindent
{\bf Prep Quiz:}

Identify a loop problem that absolutely cannot be avoided or compensated for by merely tuning the loop controller:

\begin{itemize}
\item{} An operator with a very bad attitude 
\vskip 5pt 
\item{} Excessive friction in the control valve packing
\vskip 5pt 
\item{} Instability caused by transmitter re-ranging
\vskip 5pt 
\item{} Instability caused by control valve re-sizing
\vskip 5pt 
\item{} Very high process (loop) gain
\vskip 5pt 
\item{} Very low process (loop) gain
\end{itemize}


\vfil \eject

\noindent
{\bf Prep Quiz:}

Identify a process characteristic that can {\it not} be determined by examining the process response to a step-change in output (with the controller in manual mode):

\begin{itemize}
\item{} Process gain
\vskip 5pt 
\item{} Valve stiction
\vskip 5pt 
\item{} Dead time
\vskip 5pt 
\item{} Transmitter calibration
\vskip 5pt 
\item{} Lag time
\vskip 5pt 
\item{} Self-regulating or integrating behavior
\end{itemize}

%INDEX% Reading assignment: Lessons In Industrial Instrumentation, PID tuning (identifying the problems)

%(END_NOTES)


