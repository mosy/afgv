
%(BEGIN_QUESTION)
% Copyright 2009, Tony R. Kuphaldt, released under the Creative Commons Attribution License (v 1.0)
% This means you may do almost anything with this work of mine, so long as you give me proper credit

Read and outline the ``Supervisory Control'' section of the ``Basic Process Control Strategies'' chapter in your {\it Lessons In Industrial Instrumentation} textbook.  Note the page numbers where important illustrations, photographs, equations, tables, and other relevant details are found.  Prepare to thoughtfully discuss with your instructor and classmates the concepts and examples explored in this reading.

\underbar{file i04335}
%(END_QUESTION)





%(BEGIN_ANSWER)


%(END_ANSWER)





%(BEGIN_NOTES)

Definition of supervisory control: {\it when a computer adjusts the setpoint of a loop controller according to some plan or schedule}.  This is often referred to as a {\it remote setpoint} as opposed to a local setpoint set by a human operator.

\vskip 10pt

An example of supervisory control is {\it ramp and soak} used in heat-treating furnaces.  Some chemical processing industries use powerful computers running sophisticated mathematical models to provide supervisory setpoint control for the ``base'' or ``regulatory'' PID loop controllers.  These supervisory control strategies may work to optimize product quality or even profit based on market conditions.







\vskip 20pt \vbox{\hrule \hbox{\strut \vrule{} {\bf Suggestions for Socratic discussion} \vrule} \hrule}

\begin{itemize}
\item{} What is the point of having PID loop controllers at all if there is going to be a powerful computer sending them remote setpoints?  Why not just have the powerful computer do the PID algorithms as well?
\item{} Explain what is meant by the phrase ``remote setpoint'' as compared to ``local setpoint''.
\item{} How might a remote setpoint be added to the process your team has built in the lab?
\item{} Describe some practical applications of supervisory control in industry.
\end{itemize}


%INDEX% Reading assignment: Lessons In Industrial Instrumentation, basic control strategies (supervisory)

%(END_NOTES)


