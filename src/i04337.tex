
%(BEGIN_QUESTION)
% Copyright 2015, Tony R. Kuphaldt, released under the Creative Commons Attribution License (v 1.0)
% This means you may do almost anything with this work of mine, so long as you give me proper credit

Read and outline the ``Ratio Control'' section of the ``Basic Process Control Strategies'' chapter in your {\it Lessons In Industrial Instrumentation} textbook.  Note the page numbers where important illustrations, photographs, equations, tables, and other relevant details are found.  Prepare to thoughtfully discuss with your instructor and classmates the concepts and examples explored in this reading.

\underbar{file i04337}
%(END_QUESTION)





%(BEGIN_ANSWER)


%(END_ANSWER)





%(BEGIN_NOTES)

Adjusting the flow of water in a shower using separate ``hot'' and ``cold'' water valves is an exercise in ratio control.  Valves may be mechanically linked together to form a ratio control system, as is commonly done on gas burner systems to adjust air/fuel ratio.

\vskip 10pt

Automated ratio control systems use a control loop to force a ``captive'' flow to match (ratio) an uncontrolled or ``wild'' flow.  The PV of the wild loop ends up becoming the SP of the captive loop.  The addition of a ratio station (multiplying relay) allows the ratio between two variables to be changed on the fly.

\vskip 10pt

Hydrocarbon reforming is an example of a ratio control process, whereby the flow rates of hydrocarbon gas (typically methane) and steam must be held in precise ratio in order for the endothermic hydrogen-producing reaction to take place.  In the example shown, methane flow is ``wild'' (albeit controlled by a simple loop) and steam is ``captive'' to a setpoint formed by the methane flow signal passed through a multiplying relay.  Ratio typically operated a bit lean (more steam than theoretically necessary) to mitigate coking in the reactor tubes.  This ratio may be made automatically adjustable by the action of an analyzer feeding signals back to the multiplying relay.








\vskip 20pt \vbox{\hrule \hbox{\strut \vrule{} {\bf Suggestions for Socratic discussion} \vrule} \hrule}

\begin{itemize}
\item{} Explain how {\it ratio} control differs from {\it cascade} control.
\item{} Locate the cam described in the text near the photograph of the mechanical linkage controlling the air/fuel ratio on a gas burner, and describe how it may be configured by a mechanic to optimize the air/fuel ratio for that burner.
\item{} Annotate the PID controller in the base/pigment ratio control system with ``+'' and ``$-$'' symbols to show directions of action for each of its two inputs.
\item{} Suppose the base and pigment flow transmitters in the paint mixing system (shown in the textbook) began with identical calibrated ranges (0 to 20 GPM), but then you were asked to re-range these transmitters to yield a 1.5:1 pigment:base mixing ratio.  What, exactly, would you alter in that system?
\item{} Annotate the PID controllers in the hydrocarbon reforming ratio control system with ``+'' and ``$-$'' symbols to show directions of action for their inputs.
\item{} Describe the consequences of improper methane/steam ratio mixing in the reforming process.
\item{} Explain why true ``mass flow'' transmitters are required in the steam reforming process, rather than volumetric flowmeters.
\item{} Identify the condition(s) that would require an adjustment of the steam/methane ratio in the reforming process.
\item{} Propose various instrument faults in any of the ratio control systems shown in the textbook, and have students explain the effects of those faults assuming automatic mode in the controller.
\end{itemize}








\vfil \eject

\noindent
{\bf Prep Quiz:}

The {\it ratio control} strategy used in hydrocarbon reforming processes balances the ratio of:

\begin{itemize}
\item{} Methane flow rate to steam flow rate
\vskip 5pt 
\item{} Air pressure to fuel gas pressure
\vskip 5pt 
\item{} Air temperature to steam temperature
\vskip 5pt 
\item{} Air flow rate to methane flow rate
\vskip 5pt 
\item{} Steam flow rate to air flow rate
\vskip 5pt 
\item{} Water pressure to air flow rate
\end{itemize}


\vfil \eject

\noindent
{\bf Prep Quiz:}

The difference between a {\it wild} variable and a {\it captive} variable in a ratio control system is:

\begin{itemize}
\item{} The captive variable is the one connected to the SP input of the controller
\vskip 5pt 
\item{} The wild variable is always the one measured by a turbine flowmeter
\vskip 5pt 
\item{} The wild variable is the one connected to the PV input of the controller
\vskip 5pt 
\item{} The captive variable always has a faster response time than the wild variable
\vskip 5pt 
\item{} The captive variable is the one being controlled to match the wild variable
\vskip 5pt 
\item{} The wild variable always has a faster response time than the captive variable
\end{itemize}


%INDEX% Reading assignment: Lessons In Industrial Instrumentation, basic control strategies (ratio)

%(END_NOTES)


