
%(BEGIN_QUESTION)
% Copyright 2015, Tony R. Kuphaldt, released under the Creative Commons Attribution License (v 1.0)
% This means you may do almost anything with this work of mine, so long as you give me proper credit

Read and outline the introduction and the ``Load Compensation'' subsection of the ``Feedforward Control'' section of the ``Basic Process Control Strategies'' chapter in your {\it Lessons In Industrial Instrumentation} textbook.  Note the page numbers where important illustrations, photographs, equations, tables, and other relevant details are found.  Prepare to thoughtfully discuss with your instructor and classmates the concepts and examples explored in this reading.

\underbar{file i04338}
%(END_QUESTION)





%(BEGIN_ANSWER)


%(END_ANSWER)





%(BEGIN_NOTES)

Feedback control systems are justified by the presence of loads in a system (e.g. hills, wind, etc. for a cruise control system).  Feedback is inherently reactive, not proactive -- it can never {\it anticipate} load changes.  Therefore, even the best feedback control system must deviate from setpoint, however briefly, in order to compensate for a changing load.

\vskip 10pt

Feedforward control monitors loads and adjusts FCE based on loads, to proactively compensate.  Ideally, the PV need not be measured at all!  Supervision analogy: information fed to supervisor in advance of some change turns what would ordinarily be a crisis into a non-event.  For a cruise control system, feedforward control would consist of topographical map data, wind speed data, etc. for the cruise control to input in order to properly adjust the throttle.

\vskip 10pt

Pure feedforward control is impractical, and so feedforward is usually used in conjunction with feedback control (``feedforward with trim'').  Level control system monitoring three (load) flows is an example of this fact: perfect control assumes no evaporation of liquid and absolutely no leaks, besides perfect transmitter calibrations!  Feedforward control is really just ratio control in disguise: balancing loads with compensation to achieve mass or energy balance.

\vskip 10pt

Feedforward control cannot oscillate, although it is possible to overcompensate with feedforward and thereby create deviations (in the opposite direction from normal) from setpoint following load changes!

\vskip 10pt

Three-element boiler feedwater control (steam flow is wild variable, added to LIC output to be RSP of FIC, controlling feedwater flow in) is an example of feedforward with trim.  Ideally, the mass balance of steam out versus water in holds the water level constant in the drum.

Feedforward heat exchanger control: ``energy demand'' of cold oil calculated from oil flow rate and differential temperature.  This adds to TIC output to become signal driving steam valve.  If energy demand changes, steam valve moves without need of correction from the TIC.

\vskip 10pt

Feedforward control action may be proportioned using a ``gain and bias'' function block placed in the feedforward signal path.  We may ``tune'' the gain and bias values of this function block by first placing the feedback controller in manual mode (or set its gain to 0\%) and then introducing load changes to see whether the feedforward action is too aggressive or not aggressive enough.








\vskip 20pt \vbox{\hrule \hbox{\strut \vrule{} {\bf Suggestions for Socratic discussion} \vrule} \hrule}

\begin{itemize}
\item{} {\bf Identify ways in which feedforward control could be added to the process you're working on in lab.  Identify all loads on your process variable, as well as specific instruments to measure each one.}
\item{} {\bf After identifying a feedforward control strategy for the process you're working on in lab, describe a step-by-step procedure by which you could ``tune'' the gain and bias of that feedforward function.}
\item{} Refer to the introductory portion of the Feedforward section (in the LIII textbook), and explain the meaning of the sign taped to the blue control panel.
\item{} Explain how the control strategy of feed{\it forward} differs from feed{\it back}.
\item{} Explain why pure feedforward control is not practical, but must always be augmented by feedback control.
\item{} Explain why feedforward control can never produce oscillations in the process variable the way feedback control can.
\item{} Explain how the physics principles of {\it energy conservation} and {\it mass conservation} are applied in feedforward control schemes.
\item{} Explain how three-element boiler feedwater control functions.
\item{} Referencing the diagram of a three-element boiler feedwater control system, identify which portions of the system implement feedforward control and which portions of the system implement feedback control.
\item{} Explain the purpose of the multiplying function block in the heat exchanger feedforward control system.
\item{} Referencing the diagram of a heat exchanger temperature control system, identify which portions of the system implement feedforward control and which portions of the system implement feedback control.
\item{} Referencing the diagram of a heat exchanger temperature control system, explain why the subtractor function is used in the calculation of the ``energy demand'' signal.
\end{itemize}












\vfil \eject

\noindent
{\bf Prep Quiz:}

The basic concept of feedforward control is to:

\begin{itemize}
\item{} Manipulate a process based on the PV's rate of change over time
\vskip 5pt 
\item{} Monitor process loads and make corrections based on those load values
\vskip 5pt 
\item{} Maintain the process variable at some fixed ratio to another variable
\vskip 5pt 
\item{} Let the output of one controller be the setpoint for another controller
\vskip 5pt 
\item{} Adjust the setpoint of a controller according to a timed schedule
\vskip 5pt 
\item{} Manipulate a process based on the amount and the duration of error (PV $-$ SP)
\end{itemize}







\vfil \eject

\noindent
{\bf Summary Quiz:}

Pure feedforward control (i.e. a control strategy that only uses feedforward, and no feedback) is usually not practical, because:

\begin{itemize}
\item{} Feedforward systems tend to oscillate if not corrected by feedback
\vskip 5pt 
\item{} The derivative (rate) action in a feedforward system cannot tolerate any noise
\vskip 5pt 
\item{} Quick-opening control valve characterization is required to make it work
\vskip 5pt 
\item{} Feedforward systems tend to suffer from integral (reset) windup
\vskip 5pt 
\item{} It is impossible to perfectly account for every single load in a process
\vskip 5pt 
\item{} There is no way to adjust gain in a purely feedforward system
\end{itemize}



%INDEX% Reading assignment: Lessons In Industrial Instrumentation, basic control strategies (feedforward)

%(END_NOTES)


