
%(BEGIN_QUESTION)
% Copyright 2009, Tony R. Kuphaldt, released under the Creative Commons Attribution License (v 1.0)
% This means you may do almost anything with this work of mine, so long as you give me proper credit

Read and outline the ``Dead Time Compensation'' subsection of the ``Feedforward with Dynamic Compensation'' section of the ``Basic Process Control Strategies'' chapter in your {\it Lessons In Industrial Instrumentation} textbook.  Note the page numbers where important illustrations, photographs, equations, tables, and other relevant details are found.  Prepare to thoughtfully discuss with your instructor and classmates the concepts and examples explored in this reading.

\underbar{file i04341}
%(END_QUESTION)





%(BEGIN_ANSWER)


%(END_ANSWER)





%(BEGIN_NOTES)

Process example: lime automatically added to water treatment mixing tank to counteract the low pH of flocculant added to the water.  pH transmitter, controller, and lime screw feeder form a feedback loop.  From the pH controller's perspective, the flocculant feed rate is a {\it load} which must be compensated for.  In order to better deal with this load, feedforward control may be implemented using the flocculant hand controller signal as a load variable to preemptively change lime feed rate before pH ever drifts off setpoint.

\vskip 10pt

If the two conveyor belts have equal transport delays (dead times), all will work well with the feedforward loop.  If, however, these dead times are unequal, the feedforward action will not arrive at the mixing vessel at the right time to counter the change in flocculant rate.  If the lime conveyor delay is shorter than the flocculant conveyor delay, the feedforward action will occur too soon and the effect will be temporary over-compensation.  If the lime conveyor delay is longer than the flocculant conveyor delay, the feedforward action will occur too late and the effect will be temporary under-compensation.  If equalizing the physical conveyor speeds is not practical, we can effectively equalize them by inserting a dead time function into one of the signal paths.  The insertion of time functions into a feedforward signal path is called {\it dynamic compensation}.

\vskip 10pt

It is important to install any dynamic compensation functions in a place where they will only affect the feedforward signal, and not the feedback (trim) control.  This ensures the feedback loop will not see any unnecessary time delays.










\vskip 20pt \vbox{\hrule \hbox{\strut \vrule{} {\bf Suggestions for Socratic discussion} \vrule} \hrule}

\begin{itemize}
\item{} Explain why {\it dynamic compensation} is sometimes useful or even necessary in a feedforward control system.
\item{} Describe how you could test a simple feedforward control system to tell whether or not it needed some form of dynamic compensation added to it.  In other words, {\it how can we tell when dynamic compensation is necessary?}
\item{} Why is the dead time function purposely inserted {\it before} the summing function rather than {\it after} the summing function?
\item{} Why should we never install a dead time function between the transmitter and a feedback controller?
\item{} Explain what will happen if the pH transmitter in the flocculant/lime system fails with a low signal.
\item{} Explain what will happen if the pH transmitter in the flocculant/lime system fails with a high signal.
\end{itemize}


%INDEX% Reading assignment: Lessons In Industrial Instrumentation, basic control strategies (feedforward w/ dead time compensation)

%(END_NOTES)


