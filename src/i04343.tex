
%(BEGIN_QUESTION)
% Copyright 2009, Tony R. Kuphaldt, released under the Creative Commons Attribution License (v 1.0)
% This means you may do almost anything with this work of mine, so long as you give me proper credit

Read and outline the ``Lead/Lag and Dead Time Function Blocks'' subsection of the ``Feedforward with Dynamic Compensation'' section of the ``Basic Process Control Strategies'' chapter in your {\it Lessons In Industrial Instrumentation} textbook.  Note the page numbers where important illustrations, photographs, equations, tables, and other relevant details are found.  Prepare to thoughtfully discuss with your instructor and classmates the concepts and examples explored in this reading.

\underbar{file i04343}
%(END_QUESTION)





%(BEGIN_ANSWER)


%(END_ANSWER)





%(BEGIN_NOTES)

Dead time, lag, and lead functions are all easily implemented in digital control systems.  Dead time is done by using a shift register.  Analog resistor-capacitor (RC) networks easily implement lead and lag functions, even though these functions are more typically implemented digitally.

\vskip 10pt

If $\tau_{lag} = \tau_{leadg}$, the lead/lag function block does not alter the signal at all.

If $\tau_{lag} > \tau_{lead}$, the lead/lag function block acts as a lag function.

If $\tau_{lag} < \tau_{lead}$, the lead/lag function block acts as a lead function.

The ratio $\tau_{lead} \over \tau_{lag}$ determines the amplitude of the initial ``step'' to a square-wave input.  $\tau_{lag}$ always detemines the decay time following this initial step.

\vskip 10pt

An analog ``lead'' RC network is added to 10$\times$ oscilloscope probes in order to cancel out the natural lag function caused by the cable's capacitance.  Over-compensation (too much lead) results in a square wave appearing with ``surges'' at each rising and falling edge.  Under-compensation (too much lag, not enough lead) results in a square wave appearing with rounded corners.  If lead and lag times are exactly matched, the square wave appears perfectly square.






\vskip 20pt \vbox{\hrule \hbox{\strut \vrule{} {\bf Suggestions for Socratic discussion} \vrule} \hrule}

\begin{itemize}
\item{} Explain why ``compensation'' is necessary in $\times$10 oscilloscope probes, and how that compensation is achieved.
\item{} How will a lead/lag function respond if the lead time and lag time values are equal?
\item{} How will a lead/lag function respond if the lead time value exceeds the lag time value?
\item{} How will a lead/lag function respond if the lag time value exceeds the lead time value?
\item{} Which parameter in a lead/lag function block defines the ``decay'' time of the function?
\item{} Examine the set of graphed responses showing a lead/lag function with different $\tau_{lead}$-to-$\tau_{lag}$ ratios, at the end of this section in the textbook.  Identify where we might use each of these responses in a feedforward control strategy (i.e. which process characteristics would dictate the settings of $\tau_{lead}$ and $\tau_{lag}$).
\end{itemize}






\vfil \eject

\noindent
{\bf Prep Quiz:}

The purpose for {\it dynamic compensation} in a feedforward control system is to:

\begin{itemize}
\item{} Prevent unwanted cycling (multiple oscillations) in the feedforward signal loop
\vskip 5pt 
\item{} Compensate for differences in lag between load variables and manipulated variables
\vskip 5pt 
\item{} Compensate for inaccuracies in control due to transmitter mis-calibration
\vskip 5pt 
\item{} Simplify the task of ``tuning'' the gain and bias values in a feedforward system
\vskip 5pt 
\item{} Eliminate ``stiction'' in the control valve that would otherwise compromise control
\end{itemize}



\vfil \eject

\noindent
{\bf Prep Quiz:}

The purpose of a $\times$10 probe for an oscilloscope is to:

\begin{itemize}
\item{} Allow the oscilloscope to measure electrical capacitance instead of just voltage 
\vskip 5pt 
\item{} To slow down the waveform as it is displayed on the oscilloscope's screen
\vskip 5pt 
\item{} Be able to measure voltage signals larger than the oscilloscope can directly input
\vskip 5pt 
\item{} Allow the oscilloscope to measure electrical resistance instead of just voltage
\vskip 5pt 
\item{} To amplify the signal before it is sent to the oscilloscope's input connector
\vskip 5pt 
\item{} To speed up the waveform as it is displayed on the oscilloscope's screen 
\end{itemize}

%INDEX% Reading assignment: Lessons In Industrial Instrumentation, basic control strategies (lead/lag and dead time function blocks)

%(END_NOTES)


