
%(BEGIN_QUESTION)
% Copyright 2010, Tony R. Kuphaldt, released under the Creative Commons Attribution License (v 1.0)
% This means you may do almost anything with this work of mine, so long as you give me proper credit

Sketch arrows next to each of the two solenoid valves showing the directions of air flow in the energized (E) and de-energized (D) states, assuming the process control valve is supposed to be open in regular operation and close if both of the solenoid valves ``trip'' (i.e. 2oo2 to trip):

$$\includegraphics[width=15.5cm]{i04356x01.eps}$$

\underbar{file i04356}
%(END_QUESTION)





%(BEGIN_ANSWER)

$$\includegraphics[width=15.5cm]{i04356x02.eps}$$

%(END_ANSWER)





%(BEGIN_NOTES)

\vskip 20pt \vbox{\hrule \hbox{\strut \vrule{} {\bf Virtual Trip-testing} \vrule} \hrule}

This question is a good candidate for a ``Virtual Trip-testing'' exercise.  Presenting the diagram to students, you pose an assignment whereby students must figure out how to test some component of this system to check that it will operate as intended to shut down the system in an abnormal (trip) condition, with some realistic limitation (e.g. power cannot be shut off to the load).  Students then propose various methods for executing the test.  Your job is to determine whether or not their proposed tests will achieve the desired result(s).

During and after the exercise, it is good to ask students follow-up questions such as:

\begin{itemize}
\item{} Where might our planned test strategy go wrong?  In other words, what thing(s) might happen to foil our test, either to invalidate the results or to not honor the stated limitation(s)?
\item{} Suppose the limitation were different.  How would this affect our ability to carry out the test?
\item{} Is the last test strategy best one we could execute?
\end{itemize}



%INDEX% Final Control Elements, valve: fail-safe solenoids

%(END_NOTES)


