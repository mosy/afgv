
%(BEGIN_QUESTION)
% Copyright 2010, Tony R. Kuphaldt, released under the Creative Commons Attribution License (v 1.0)
% This means you may do almost anything with this work of mine, so long as you give me proper credit

Read and outline the introduction to the ``Digital Data Acquisition and Networks'' chapter in your {\it Lessons In Industrial Instrumentation} textbook.  Note the page numbers where important illustrations, photographs, equations, tables, and other relevant details are found.  Prepare to thoughtfully discuss with your instructor and classmates the concepts and examples explored in this reading.

\underbar{file i04394}
%(END_QUESTION)





%(BEGIN_ANSWER)


%(END_ANSWER)





%(BEGIN_NOTES)

Digital technology has greatly expanded industrial instrumentation capability.  ``Data acquisition'' = measuring and recording process data.  

\vskip 10pt

Analog signaling = one channel of data per cable.  Digital signaling = limitless channels of data per cable (the only practical constraint being the time it takes to convey all that data).  This is especially well-suited to multivariable instruments such as Coriolis flowmeters, being able to report multiple measurements over a single pair of wires.

\vskip 10pt

HART is a hybrid of analog + digital: digital signals superimposed on the same two wires that carry the 4-20 mA analog signal, but very slowly.  Lots of other digital signal standards (``fieldbuses'') in industry.  DCS (``Distributed Control System'') systems usually provide graphical representations of digital instruments in an easy-to-navigate format.

\vskip 10pt

``SCADA'' = Supervisory Control and Data Acquisition.  Basically a DCS writ large: spread out over a large geographic area.  Powerline grids, pipelines, etc.  Remote Terminal Units (RTUs) gather data and communicate to central Main Terminal Units (MTUs).  Telemetry = one-way (simplex) communication.  SCADA = two-way communication.

\vskip 10pt

Early powerline systems used ``powerline carrier'' systems to convey data along power conductors, kind of like how HART conveys digital data over an analog 4-20 mA line.








\vskip 20pt \vbox{\hrule \hbox{\strut \vrule{} {\bf Suggestions for Socratic discussion} \vrule} \hrule}

\begin{itemize}
\item{} Explain the difference between a ``telemetry'' system and a ``SCADA'' system, citing practical applications of each.
\item{} What kind of telemetry or SCADA system would you like to try building in our lab?  How about at home?
\item{} Explain how ``power line carrier'' technology is similar to HART digital technology.
\end{itemize}











\vfil \eject

\noindent
{\bf Prep Quiz:}

Describe an advantage of {\it digital} data communication over {\it analog} data communication.  Be as specific as you can, and feel free to cite a realistic application if it helps your description.












\vfil \eject

\noindent
{\bf Prep Quiz:}

Describe a practical example of a {\it SCADA} system.  Be as specific as you can, and feel free to cite a realistic application if it helps your description. 




%INDEX% Reading assignment: Lessons In Industrial Instrumentation, Digital data and networks (intro)

%(END_NOTES)

