
%(BEGIN_QUESTION)
% Copyright 2010, Tony R. Kuphaldt, released under the Creative Commons Attribution License (v 1.0)
% This means you may do almost anything with this work of mine, so long as you give me proper credit

Read and outline the ``Analog-Digital Conversion'' section of the ``Digital Data Acquisition and Network'' chapter in your {\it Lessons In Industrial Instrumentation} textbook.  Note the page numbers where important illustrations, photographs, equations, tables, and other relevant details are found.  Prepare to thoughtfully discuss with your instructor and classmates the concepts and examples explored in this reading.

\underbar{file i04396}
%(END_QUESTION)





%(BEGIN_ANSWER)


%(END_ANSWER)





%(BEGIN_NOTES)

``Resolution'' = smallest increment of analog signal representable (1 bit's worth).  12-bit converter has 4095 ``counts'' and therefore a resolution equal to ${1 \over 4095}$ of the analog span (if 0 to 10 VDC, resolution = 2.442 mV).

$${V_{in} \over V_{fullscale}} = {\hbox{Counts} \over 2^n - 1}$$

Converting between analog value and digital ``counts'' works exactly the same as translating between input and output quantities for any linear analog instrument's range!

\vskip 10pt

Uncertainty of analog value corresponding to a particular count value is the ``quantization error'' of the converter.  The more bits the converter has, the less this uncertainty.

\vskip 10pt

Converters sample at discrete moments in time.  We must sample much faster than the analog frequency (according to Nyquist's theorem, at least twice as fast as the signal frequency of interest), or else we lose detail.  Too-slow sample rates are equivalent to dead time in the system.  We may also suffer from {\it aliasing} if the sample rate is too slow: example showing red dots (sample points) forming a red wave that is far slower than the actual blue wave (analog signal).  We may combat aliasing by placing a low-pass filter on the input of teh ADC, blocking too-fast signals from even entering. 

\vskip 10pt

Aliasing can even happen within a digital system, if different parts of it scan the same signal at different rates!









\vskip 20pt \vbox{\hrule \hbox{\strut \vrule{} {\bf Suggestions for Socratic discussion} \vrule} \hrule}

\begin{itemize}
\item{} Explain what {\it quantization error} is in an analog-digital converter.
\item{} Calculate the resolution of a 14-bit ADC with a 0 to 5 VDC analog range (5 V / ($2^{14}$ $-$ 1) = 305.19 $\mu$V)
\item{} Calculate the resolution of a 10-bit ADC with a 0 to 10 VDC analog range (10 V / ($2^{10}$ $-$ 1) = 9.775 mV)
\item{} Calculate the resolution of an 8-bit ADC with a 0 to 22 mA analog range (22 mA / ($2^{8}$ $-$ 1) = 86.27 $\mu$A)
\item{} Suppose we needed to digitize some analog quantity precisely enough to resolve down to $\pm$1\%.  How many bits (minimum) would the ADC need to have? (7 bits)
\item{} Suppose we needed to digitize some analog quantity precisely enough to resolve down to $\pm$0.1\%.  How many bits (minimum) would the ADC need to have? (10 bits)
\item{} Suppose we needed to digitize some analog quantity precisely enough to resolve down to $\pm$0.01\%.  How many bits (minimum) would the ADC need to have? (14 bits)
\item{} Explain what {\it aliasing} is, and how to avoid it in a digital system.
\end{itemize}












\vfil \eject

\noindent
{\bf Prep Quiz:}

Explain what {\it aliasing} is in an analog-to-digital conversion process.  Be as specific as you can, and feel free to cite a realistic application if it helps your description. 


%INDEX% Reading assignment: Lessons In Industrial Instrumentation, Digital data and networks (analog-digital conversion)

%(END_NOTES)

