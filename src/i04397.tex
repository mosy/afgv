
%(BEGIN_QUESTION)
% Copyright 2010, Tony R. Kuphaldt, released under the Creative Commons Attribution License (v 1.0)
% This means you may do almost anything with this work of mine, so long as you give me proper credit

Read and outline the ``Digital Representation of Text'' section of the ``Digital Data Acquisition and Network'' chapter in your {\it Lessons In Industrial Instrumentation} textbook.  Note the page numbers where important illustrations, photographs, equations, tables, and other relevant details are found.  Prepare to thoughtfully discuss with your instructor and classmates the concepts and examples explored in this reading.

\underbar{file i04397}
%(END_QUESTION)





%(BEGIN_ANSWER)


%(END_ANSWER)





%(BEGIN_NOTES)

Morse Code: used bit patterns of varying length to represent all English letters (one case only, no upper- and lower-case) and numbers 0-9.  Later Baudot code used in teletype machines were consistent 5 bits in length (32 possible combinations).  Two special ``shift'' characters (designated as ``letters'' and ``figures'') overloaded the other 30 characters with other meanings.

\vskip 10pt

EBCDIC invented by IBM in 1962, used 8-bit serial data frames for 256 possible character codes.  All English letters (upper- and lower-case) plus numbers and special ``control characters'' used for printer control.

\vskip 10pt

ASCII invented in 1963, used 7-bit serial data frames for 128 possible character codes.  Still used today for nearly all English text encoding!

% No blank lines allowed between lines of an \halign structure!
% I use comments (%) instead, so that TeX doesn't choke.

$$\vbox{\offinterlineskip
\halign{\strut
\vrule \quad\hfil # \ \hfil & 
\vrule \quad\hfil # \ \hfil & 
\vrule \quad\hfil # \ \hfil & 
\vrule \quad\hfil # \ \hfil & 
\vrule \quad\hfil # \ \hfil & 
\vrule \quad\hfil # \ \hfil & 
\vrule \quad\hfil # \ \hfil & 
\vrule \quad\hfil # \ \hfil & 
\vrule \quad\hfil # \ \hfil \vrule \cr
\noalign{\hrule}
%
% First row
$\downarrow$ LSB / MSB  & 000 & 001 & 010 & 011 & 100 & 101 & 110 & 111 \cr
%
\noalign{\hrule}
%
% Another row
0000 & NUL & DLE & SP & 0 & @ & P & ` & p \cr
%
\noalign{\hrule}
%
% Another row
0001 & SOH & DC1 & ! & 1 & A & Q & a & q \cr
%
\noalign{\hrule}
%
% Another row
0010 & STX & DC2 & " & 2 & B & R & b & r \cr
%
\noalign{\hrule}
%
% Another row
0011 & ETX & DC3 & \# & 3 & C & S & c & s \cr
%
\noalign{\hrule}
%
% Another row
0100 & EOT & DC4 & \$ & 4 & D & T & d & t \cr
%
\noalign{\hrule}
%
% Another row
0101 & ENQ & NAK & \% & 5 & E & U & e & u \cr
%
\noalign{\hrule}
%
% Another row
0110 & ACK & SYN & \& & 6 & F & V & f & v \cr
%
\noalign{\hrule}
%
% Another row
0111 & BEL & ETB & ' & 7 & G & W & g & w \cr
%
\noalign{\hrule}
%
% Another row
1000 & BS & CAN & ( & 8 & H & X & h & x \cr
%
\noalign{\hrule}
%
% Another row
1001 & HT & EM & ) & 9 & I & Y & i & y \cr
%
\noalign{\hrule}
%
% Another row
1010 & LF & SUB & * & : & J & Z & j & z \cr
%
\noalign{\hrule}
%
% Another row
1011 & VT & ESC & + & ; & K & $[$ & k & $\{$ \cr
%
\noalign{\hrule}
%
% Another row
1100 & FF & FS & , & $<$ & L & $\backslash$ & l & $|$ \cr
%
\noalign{\hrule}
%
% Another row
1101 & CR & GS & - & $=$ & M & $]$ & m & $\}$ \cr
%
\noalign{\hrule}
%
% Another row
1110 & SO & RS & . & $>$ & N & \^{} & n & \~{} \cr
%
\noalign{\hrule}
%
% Another row
1111 & SI & US & / & ? & O & \_ & o & DEL \cr
%
\noalign{\hrule}
} % End of \halign 
}$$ % End of \vbox

Unicode uses 16 bits per character, for a total number of characters equal to 65,536.  Encompasses both ASCII and EBCDIC as sub-sets within the Unicode character set.









\vskip 20pt \vbox{\hrule \hbox{\strut \vrule{} {\bf Suggestions for Socratic discussion} \vrule} \hrule}

\begin{itemize}
\item{} {\bf Spell your first name in ASCII codes.  Feel free to use hexadecimal rather than binary for each character.}
\item{} Explain why Morse Code is considered a ``self-compressing'' encoding scheme for text, while ASCII and EBCDIC are not.
\item{} Compare and contrast the EBCDIC versus ASCII codes.
\item{} Compare and contrast Unicode versus ASCII.
\item{} Why do you suppose there is no table of Unicode characters included in the textbook?  Is the author just stubborn?
\end{itemize}


%INDEX% Reading assignment: Lessons In Industrial Instrumentation, Digital data and networks (textual data)

%(END_NOTES)

