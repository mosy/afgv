
%(BEGIN_QUESTION)
% Copyright 2010, Tony R. Kuphaldt, released under the Creative Commons Attribution License (v 1.0)
% This means you may do almost anything with this work of mine, so long as you give me proper credit

Read and outline the introduction of the ``Digital Data Communication Theory'' section of the ``Digital Data Acquisition and Networks'' chapter in your {\it Lessons In Industrial Instrumentation} textbook.  Note the page numbers where important illustrations, photographs, equations, tables, and other relevant details are found.  Prepare to thoughtfully discuss with your instructor and classmates the concepts and examples explored in this reading.

\underbar{file i04398}
%(END_QUESTION)





%(BEGIN_ANSWER)


%(END_ANSWER)





%(BEGIN_NOTES)

Digital communication allows a multitude of signals to be conveyed over a single channel, albeit time-delayed.  Analog communication is virtually instantaneous, but limited to one signal per channel.

\vskip 10pt

Digital communication enjoys far better noise immunity than analog signaling, albeit with finite resolution.  Analog communication has infinite resolution, but is sensitive to noise. 

\vskip 10pt

``Serial'' digital communication means sending and receiving one bit at a time, as opposed to ``parallel'' requiring many wires to communicate all bits at once.












\vskip 20pt \vbox{\hrule \hbox{\strut \vrule{} {\bf Suggestions for Socratic discussion} \vrule} \hrule}

\begin{itemize}
\item{} Contrast analog versus digital data communication, explaining how each one has its own unique advantages and disadvantages.
\item{} Describe a practical application where the limited resolution of a digital system would be revealed, especially in contrast to a similar analog system.
\item{} Describe a practical application where the time delay of a digital system would be revealed, especially in contrast to a similar analog system.
\end{itemize}












\vfil \eject

\noindent
{\bf Prep Quiz:}

Explain in detail why digital communication systems are generally able to tolerate more {\it noise} than analog communication systems.  Feel free to sketch a diagram or to cite a practical example if it helps you explain the principle.


%INDEX% Reading assignment: Lessons In Industrial Instrumentation, Digital data and networks (communication intro)

%(END_NOTES)

