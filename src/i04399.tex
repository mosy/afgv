
%(BEGIN_QUESTION)
% Copyright 2010, Tony R. Kuphaldt, released under the Creative Commons Attribution License (v 1.0)
% This means you may do almost anything with this work of mine, so long as you give me proper credit

Read and outline the ``Serial Communication Principles'' subsection of the ``Digital Data Communication Theory'' section of the ``Digital Data Acquisition and Networks'' chapter in your {\it Lessons In Industrial Instrumentation} textbook.  Note the page numbers where important illustrations, photographs, equations, tables, and other relevant details are found.  Prepare to thoughtfully discuss with your instructor and classmates the concepts and examples explored in this reading.

\underbar{file i04399}
%(END_QUESTION)





%(BEGIN_ANSWER)


%(END_ANSWER)





%(BEGIN_NOTES)

Telegraphs were early digital data communication systems, relying on the standard of Morse Code to encode alphanumeric characters as serial bit patterns.  In Morse code as well as in other serial data code systems, an agreed-upon standard exists for representing each different character in the alphabet by means of electrical pulses.

\vskip 10pt

``Delimiting'' is necessary to define where one serial bit pattern ends and another begins, in order to make sense of a long string of bits.  Otherwise, interpreting these bit patterns as characters becomes impossible.  Morse and Continental codes used pauses as delimiting marks between characters.  Later teletype machines used special synchronization codes to denote the same for the receiving machine(s).  Modern asynchronous systems use ``start'' and ``stop'' bits to denote the same.










\vskip 20pt \vbox{\hrule \hbox{\strut \vrule{} {\bf Suggestions for Socratic discussion} \vrule} \hrule}

\begin{itemize}
\item{} Design a simple telegraph circuit two people could use to communicate information across the span of an industrial plant site.
\item{} Explain the importance of {\it standard codes} in a serial data communication system.
\item{} Explain the importance of {\it delineation} in a serial data communication system.
\end{itemize}











\vfil \eject

\noindent
{\bf Prep Quiz:}

An important concept in serial data communication is {\it delimiting}.  Explain what this term means, using a practical example to illustrate your point.


%INDEX% Reading assignment: Lessons In Industrial Instrumentation, Digital data and networks (serial data)

%(END_NOTES)

