
%(BEGIN_QUESTION)
% Copyright 2010, Tony R. Kuphaldt, released under the Creative Commons Attribution License (v 1.0)
% This means you may do almost anything with this work of mine, so long as you give me proper credit

Read and outline the ``Physical Encoding of Bits'' subsection of the ``Digital Data Communication Theory'' section of the ``Digital Data Acquisition and Networks'' chapter in your {\it Lessons In Industrial Instrumentation} textbook.  Note the page numbers where important illustrations, photographs, equations, tables, and other relevant details are found.  Prepare to thoughtfully discuss with your instructor and classmates the concepts and examples explored in this reading.

\underbar{file i04400}
%(END_QUESTION)





%(BEGIN_ANSWER)


%(END_ANSWER)





%(BEGIN_NOTES)

Non-Return to Zero (NRZ) encoding uses different voltage levels to denote ``marks'' and ``spaces'' (1 and 0, respectively).  Interestingly a mark (1) is a low voltage and a space (0) is a high voltage in the RS-232 standard. 

\vskip 10pt

Manchester encoding uses rising and falling edges of a square wave to denote 1's and 0's (respectively).  This makes a Manchester-encoded waveform self-clocking (able to encode the clock speed along with the data).  Identifying the real signal transitions versus reversals is the only challenge with Manchester encoding, and this may be done by identifying the widest state, corresponding to the half-period of the clock frequency.

\vskip 10pt

Frequency Shift-Key (FSK) encoding uses a pair of frequencies to represent 1's and 0's (1200 Hz and 2200 Hz, respectively, in the Bell 202 / HART standard).








\vskip 20pt \vbox{\hrule \hbox{\strut \vrule{} {\bf Suggestions for Socratic discussion} \vrule} \hrule}

\begin{itemize}
\item{} Explain what ``mark'' and ``space'' refer to in an NRZ encoding scheme.
\item{} Explain how Manchester encoding works, and how it differs from NRZ encoding.
\item{} Explain how to interpret the ``mark'' and ``space'' states of a Manchester-encoded signal.  Hint: how do you first identify the clock period of the signal?
\item{} Suppose you were analyzing a serial data waveform using a digital storage oscilloscope, allowing you to ``capture'' a segment of the signal and analyze it at your own pace.  If you had the oscilloscope accidently connected backwards (probe and ground clip to the opposite serial cable wires), would it affect your interpretation of the digital data?  Assume a Manchester-encoded signal and no problems with ground references.
\item{} Explain how FSK encoding works, and how it differs from NRZ and from Manchester encoding.
\item{} Manchester encoding was developed for use in magnetic data storage media such as magnetic disks and magnetic tape.  Based on what you know of these media (especially how the magnetic states are written to and read from the media), explain why Mahcnester works particularly well as an encoding scheme.
\end{itemize}












\vfil \eject

\noindent
{\bf Prep Quiz:}

The textbook describes three different methods for electrically encoding serial data bits: {\it NRZ} (Non-Return-to-Zero), {\it Manchester}, and {\it FSK} (Frequency Shift Keying).  Choose any one of these encoding methods and explain how it represents a binary ``1'' state versus a binary ``0'' state.



%INDEX% Reading assignment: Lessons In Industrial Instrumentation, Digital data and networks (bit encoding)

%(END_NOTES)

