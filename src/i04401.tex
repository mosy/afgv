
%(BEGIN_QUESTION)
% Copyright 2010, Tony R. Kuphaldt, released under the Creative Commons Attribution License (v 1.0)
% This means you may do almost anything with this work of mine, so long as you give me proper credit

Read and outline the ``Communication Speed'' subsection of the ``Digital Data Communication Theory'' section of the ``Digital Data Acquisition and Networks'' chapter in your {\it Lessons In Industrial Instrumentation} textbook.  Note the page numbers where important illustrations, photographs, equations, tables, and other relevant details are found.  Prepare to thoughtfully discuss with your instructor and classmates the concepts and examples explored in this reading.

\underbar{file i04401}
%(END_QUESTION)





%(BEGIN_ANSWER)


%(END_ANSWER)





%(BEGIN_NOTES)

In a serial communication system, transmitter and receiver must be configured to operate at the same speed, in order to synchronize the communication of bits.  Some standards have fixed speeds (e.g. FOUNDATION Fieldbus at 31.25 bps), while others may be user-set (e.g. EIA/TIA-232 anywhere from 300 bps to over 115 kbps).

\vskip 10pt

``Baud'' not the same thing as bps.  Bits per second is self-explanatory, while ``baud'' refers to the actual frequency of the signal waveform (Hz).  In some schemes (e.g. NRZ) these are equivalent.  In others, (e.g. Manchester) they are not.











\vskip 20pt \vbox{\hrule \hbox{\strut \vrule{} {\bf Suggestions for Socratic discussion} \vrule} \hrule}

\begin{itemize}
\item{} Describe the distinction between ``bit rate'' and ``baud'' for a serial communication system.  Give at least one practical example where the two terms mean different things.
\item{} Explain why it is theoretically possible to arbitrarily increase the data transmission speed in a serial data network using NRZ or Manchester encoding, but there will be a definite speed limit for any comparable system using FSK encoding.
\item{} What physical limitations do you suppose there are to bit rate in a digital data communication system?  Why, for example, are RS-232 serial networks limited to top speeds in the hundreds of kilobits per second, while Ethernet can go as high as 1 {\it billion} bits per second?
\end{itemize}





%INDEX% Reading assignment: Lessons In Industrial Instrumentation, Digital data and networks (communication speed)

%(END_NOTES)

