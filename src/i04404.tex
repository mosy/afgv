
%(BEGIN_QUESTION)
% Copyright 2010, Tony R. Kuphaldt, released under the Creative Commons Attribution License (v 1.0)
% This means you may do almost anything with this work of mine, so long as you give me proper credit

Read and outline ``The OSI Reference Model'' subsection of the ``Digital Data Communication Theory'' section of the ``Digital Data Acquisition and Networks'' chapter in your {\it Lessons In Industrial Instrumentation} textbook.  Note the page numbers where important illustrations, photographs, equations, tables, and other relevant details are found.  Prepare to thoughtfully discuss with your instructor and classmates the concepts and examples explored in this reading.

\underbar{file i04404}
%(END_QUESTION)





%(BEGIN_ANSWER)


%(END_ANSWER)





%(BEGIN_NOTES)

The OSI Reference model helps make sense of data communication standards by delineating different ``layers'' of functionality.  Most communication standards do not address all 7 layers of the OSI model.

\vskip 10pt

Examples of low-level standards include Ethernet (layers 1 and 2) and RS-232 (layer 1).  These standards focus on the ``nuts and bolts'' of data transfer: voltage levels, frame configuration, etc.).  An example of a high-level standard is Modbus, which is concerned only with the meanings of commands and device addressing, not how the signals actually get sent from one device to another.







\vskip 20pt \vbox{\hrule \hbox{\strut \vrule{} {\bf Suggestions for Socratic discussion} \vrule} \hrule}

\begin{itemize}
\item{} Explain what the OSI Reference Model is good for.
\item{} Give an example of a common network standard that is relevant only to some of the OSI Reference Model's layers.
\end{itemize}

%INDEX% Reading assignment: Lessons In Industrial Instrumentation, Digital data and networks (OSI reference model)

%(END_NOTES)

