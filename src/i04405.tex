
%(BEGIN_QUESTION)
% Copyright 2010, Tony R. Kuphaldt, released under the Creative Commons Attribution License (v 1.0)
% This means you may do almost anything with this work of mine, so long as you give me proper credit

Convert the following 16-bit unsigned integer values into decimal (note the use of hexadecimal notation rather than direct binary):

\begin{itemize}
\item{} 5A24 (hexadecimal) = \underbar{\hskip 50pt} (decimal)
\vskip 10pt
\item{} FFFF (hexadecimal) = \underbar{\hskip 50pt} (decimal)
\vskip 10pt
\item{} E109 (hexadecimal) = \underbar{\hskip 50pt} (decimal)
\end{itemize}

\vskip 10pt

Convert the following 16-bit signed integer values into decimal (note the use of hexadecimal notation rather than direct binary):

\begin{itemize}
\item{} 3B11 (hexadecimal) = \underbar{\hskip 50pt} (decimal)
\vskip 10pt
\item{} C9D0 (hexadecimal) = \underbar{\hskip 50pt} (decimal)
\vskip 10pt
\item{} FFFF (hexadecimal) = \underbar{\hskip 50pt} (decimal)
\end{itemize}

\vskip 20pt \vbox{\hrule \hbox{\strut \vrule{} {\bf Suggestions for Socratic discussion} \vrule} \hrule}

\begin{itemize}
\item{} Outline the procedure(s) you used to perform these conversions.
\item{} Is there any way to tell if any of the signed values is positive or negative just by examining the hexadecimal characters (i.e. without translating into binary first)?
\item{} A powerful problem-solving technique is to simplify the problem so that it is easier to solve, then use that solution as a starting point for the final solution of the given (complex) problem.  Try laying out the place-weights for a {\it 4-bit signed integer binary number} and then figuring out the bit combinations that would give you the greatest possible positive value, the greatest possible negative value, 0, $-1$, and any other arbitrary values between.  Seeing how a short (4-bit) signed integer works helps you see how larger (e.g. 16-bit) signed integers are constructed from 1 and 0 bits.
\end{itemize}

\underbar{file i04405}
%(END_QUESTION)





%(BEGIN_ANSWER)

\noindent
{\bf Partial answer:}

\vskip 10pt

Unsigned:
\item{} FFFF (hexadecimal) = \underbar{\bf 65535} (decimal)
\end{itemize}

\vskip 10pt

Signed:
\item{} FFFF (hexadecimal) = \underbar{\bf $-$1} (decimal)
\end{itemize}


%(END_ANSWER)





%(BEGIN_NOTES)

Unsigned:
\item{} 5A24 (hexadecimal) = \underbar{\bf 23076} (decimal)
\vskip 10pt
\item{} FFFF (hexadecimal) = \underbar{\bf 65535} (decimal)
\vskip 10pt
\item{} E109 (hexadecimal) = \underbar{\bf 57609} (decimal)
\end{itemize}

\vskip 10pt

Signed:
\item{} 3B11 (hexadecimal) = \underbar{\bf +15121} (decimal)
\vskip 10pt
\item{} C9D0 (hexadecimal) = \underbar{\bf $-$13872} (decimal)
\vskip 10pt
\item{} FFFF (hexadecimal) = \underbar{\bf $-$1} (decimal)
\end{itemize}







\vfil \eject

\noindent
{\bf Summary Quiz:}

Express the following 16-bit unsigned value (shown here in hexadecimal notation) in decimal form:

$$\hbox{2E17}$$

\begin{itemize}
\item{} 81
\vskip 5pt 
\item{} 250
\vskip 5pt 
\item{} 4585
\vskip 5pt 
\item{} 4119
\vskip 5pt 
\item{} 8455
\vskip 5pt 
\item{} 11799
\end{itemize}

%INDEX% Digital number format: signed and unsigned integer conversions

%(END_NOTES)

