
%(BEGIN_QUESTION)
% Copyright 2010, Tony R. Kuphaldt, released under the Creative Commons Attribution License (v 1.0)
% This means you may do almost anything with this work of mine, so long as you give me proper credit

Read and outline the ``Transmission Lines'' section of the ``AC Electricity'' chapter in your {\it Lessons In Industrial Instrumentation} textbook.  Note the page numbers where important illustrations, photographs, equations, tables, and other relevant details are found.  Prepare to thoughtfully discuss with your instructor and classmates the concepts and examples explored in this reading.

\vskip 30pt

Note: a video demonstration of a transmission line being terminated both properly and improperly may be found on the BTC Instrumentation YouTube channel (search for a video named ``Reflected Waves on a Cable'').  In this video, an oscillscope is used to demonstrate the reflected signal caused by improper termination of a spooled cable.

\underbar{file i04409}
%(END_QUESTION)





%(BEGIN_ANSWER)


%(END_ANSWER)





%(BEGIN_NOTES)

An electrical signal pulse applied to the end of a long cable causes the cable's reactive characteristics (capacitance, inductance) to charge with energy.  During this ``charging'' time, the cable acts like a load with a definite resistance value (the cable's characteristic impedance).  This impedance value is determined by the cross-sectional geometry of the cable, and not the cable's length!

\vskip 10pt

{\bf Open line test}: voltage pulse propagates down the cable, reflects off the end with the same polarity, leaving a cable with full source voltage and no current when the pulse reaches the source (i.e. cable finally acts as an open).  First step in oscilloscope waveform is time period while pulse is traveling down and back to the source.  The higher step is after the cable is charged and behaves as an open (full source voltage).  The last half-step is the cable ``discharging'' back to its original state.

\vskip 10pt

{\bf Shorted line test}: voltage pulse propagates down the cable, reflects off the end with inverted polarity, leaving a cable with zero voltage and full current when the pulse reaches the source (i.e. cable finally acts as a short).  First pulse in the oscilloscope waveform is time period while pulse is traveling down and back to the source.  The step down is after the cable behaves as a short (zero source voltage).  The reverse pulse is the cable ``discharging'' back to its original state.

\vskip 10pt

{\bf Properly terminated line test}: voltage pulse propagates down the cable, with no reflection because the terminating resistance absorbs it -- the line looks infinitely long from the perspective of the source.  Oscilloscope shows a relatively clean square wave, because the terminated cable always presents the same impedance to the source.  Elimination of echoes through the use of termination resistors is important, otherwise the reflected signals may interfere with new signals sent into the line.

\vskip 10pt

Termination resistors are usually built into point-to-point communication devices.  For multidrop networks, termination resistors must be placed manually at the correct locations.

\vskip 10pt

Any discontinuities in the cable (deformations or insulation damage) will cause signal reflections due to the sudden change in impedance.

\vskip 10pt

Velocity factor of a cable tells us the fraction of light speed that pulses will travel along its length, and is determined by the dielectric permittivity of the conductors' insulation.

\vskip 10pt

Signals will lose energy as they propagate down the length of a cable.  This energy loss usually represented in decibels per 100 meters.








\filbreak


\vskip 20pt \vbox{\hrule \hbox{\strut \vrule{} {\bf Suggestions for Socratic discussion} \vrule} \hrule}

\begin{itemize}
\item{} Explain why the author's early attempts to measure 50 ohms of resistance anywhere on a section of ``50 ohm'' coaxial cable were doomed to failure.
\item{} Suppose a long cable were cut so as to be shorter.  How would this affect its electrical characteristics as a transmission line?
\item{} Explain what is happening at each point in the ``stepped'' pulse waveform shown on the oscilloscope screen (open-ended transmission line).
\item{} Explain what is happening at each point in the ``stepped'' pulse waveform shown on the oscilloscope screen (shorted transmission line).
\item{} If the cable were shortened, what would change about the waveform shown on the oscilloscope screen (open-ended transmission line)?
\item{} If the cable were shortened, what would change about the waveform shown on the oscilloscope screen (shorted transmission line)?
\item{} What exactly does a termination resistor do when installed at the end of a transmission line?
\item{} Identify what we could alter about a cable to increase its characteristic impedance.
\item{} Identify what we could alter about a cable to decrease its characteristic impedance.
\item{} Identify what we could alter about a cable to increase its velocity factor.
\item{} Identify what we could alter about a cable to decrease its velocity factor.
\item{} Why do we care about characteristic impedance in digital network cabling?  What possible bad effects might result from reflected signals?
\item{} For those who have studied level measurement, explain how a guided-wave radar (GWR) level transmitter's waveguide acts as a transmission line.
\item{} Describe the meaning of {\it velocity factor} in a cable and how it impacts reflected signals.
\item{} What accounts for power lost in a transmission line?  In other words, why don't transmission lines exhibit 100\% power transfer from beginning to end?
\end{itemize}












\vfil \eject

\noindent
{\bf Prep Quiz:}

Explain the purpose for connecting a {\it terminating} resistor to the end of a transmission line.  Be thorough and complete in your answer, giving a full explanation and not just a cursory statement:











\vfil \eject

\noindent
{\bf Prep Quiz:}

Explain what is meant by the statement, ``A {\it terminating} resistor connected to the end of a transmission line has the effect of making the transmission line appear infinitely long.''  Be thorough and complete in your answer, giving a full explanation and not just a cursory statement: 



%INDEX% Reading assignment: Lessons In Industrial Instrumentation, AC Electricity (transmission lines)

%(END_NOTES)

