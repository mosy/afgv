
%(BEGIN_QUESTION)
% Copyright 2014, Tony R. Kuphaldt, released under the Creative Commons Attribution License (v 1.0)
% This means you may do almost anything with this work of mine, so long as you give me proper credit

Read and outline the ``Signal Coupling and Cable Separation'' subsection of the ``Electrical Signal and Control Wiring'' section of the ``Instrument Connections'' chapter in your {\it Lessons In Industrial Instrumentation} textbook.  Note the page numbers where important illustrations, photographs, equations, tables, and other relevant details are found.  Prepare to thoughtfully discuss with your instructor and classmates the concepts and examples explored in this reading.

\vskip 30pt

A note-taking technique you will find far more productive in your academic reading than mere highlighting or underlining is to write your own {\it outline} of the text you read.  A section of your {\it Lessons In Industrial Instrumentation} textbook called ``Marking Versus Outlining a Text'' describes the technique and the learning benefits that come from practicing it.  This approach is especially useful when the text in question is dense with facts and/or challenging to grasp.  Ask your instructor for help if you would like assistance in applying this proven technique to your own reading.

\underbar{file i04412}
%(END_QUESTION)





%(BEGIN_ANSWER)


%(END_ANSWER)





%(BEGIN_NOTES)

Conductors near each other may experience ``coupling'' of signals from one to the other, either through electric fields (capacitance) or magnetic fields (mutual inductance).  Both coupling mechanisms work with AC only, and is proportional to the magnitude of the signal (voltage for capacitive and current for inductive) and also its frequency.  Reduce coupling through cable separation (don't put power and signal cables near each other!).  When separation is not possible, make cables cross perpendicularly instead of parallel (less area for capacitance, incorrect angle for mutual inductance).

\vskip 10pt

Digital signals far more tolerance of noise than analog signals.  Any amount of noise in an analog signal corrupts the data, but in a digital signal the noise has to be severe enough that a 1 gets mistaken for a 0, or vice-versa.









\vskip 20pt \vbox{\hrule \hbox{\strut \vrule{} {\bf Suggestions for Socratic discussion} \vrule} \hrule}

\begin{itemize}
\item{} Does the type of electrical insulation used on signal conductors affect coupling with power conductors?  Why or why not?
\item{} Explain why analog signals are more susceptible to corruption by noise than digital signals.
\item{} Explain why perpendicular-crossing wires experience minimal capacitive coupling.
\item{} Explain why perpendicular-crossing wires experience minimal inductive coupling.
\end{itemize}






\vfil \eject

\noindent
{\bf Prep Quiz:}

The primary reason for separating power and signal cables when wiring a control system is to:

\begin{itemize}
\item{} Minimize the ``noise'' imposed on the signals from the power wiring
\vskip 5pt 
\item{} Reduce the potential for electric shock hazard or arc flash/blast
\vskip 5pt 
\item{} Reduce expense, because putting all cables into one conduit costs more money
\vskip 5pt 
\item{} Make wiring more organized so it is easier to troubleshoot later
\vskip 5pt 
\item{} Minimize parasitic power losses resulting from capacitive or inductive effects
\vskip 5pt 
\item{} Increase the speed (bandwidth) of the digital signals in the system
\end{itemize}

%INDEX% Reading assignment: Lessons In Industrial Instrumentation, instrument connections (signal coupling and cable separation)

%(END_NOTES)

