
%(BEGIN_QUESTION)
% Copyright 2010, Tony R. Kuphaldt, released under the Creative Commons Attribution License (v 1.0)
% This means you may do almost anything with this work of mine, so long as you give me proper credit

Read and outline the ``Electric Field (Capacitive) De-coupling'' subsection of the ``Electrical Signal and Control Wiring'' section of the ``Instrument Connections'' chapter in your {\it Lessons In Industrial Instrumentation} textbook.  Note the page numbers where important illustrations, photographs, equations, tables, and other relevant details are found.  Prepare to thoughtfully discuss with your instructor and classmates the concepts and examples explored in this reading.

\underbar{file i04413}
%(END_QUESTION)





%(BEGIN_ANSWER)


%(END_ANSWER)





%(BEGIN_NOTES)

Electric fields exist between surfaces of differing electric potential.  Therefore, electric fields cannot exist within a solid conductor (because all points are electrically common and therefore equipotential), but only within insulating media.  By the same token, electric fields cannot span the diameter of a hollow conductor because all interior surfaces of that hollow conductor are equipotential.  If a hollow conductor is grounded, no external electric field can penetrate it, thus ``shielding'' any conductors within.  

\vskip 10pt

Shielded cable works on this principle: surround the conductors with a conductive foil or braid (called the ``shield''), then connect that sheath to earth ground to provide all external electric fields a termination point on the outer surface of the shield.  The shield conductor of a cable should only be grounded at one end, to avoid ground loop currents resulting from differences in earth potential across long distances!

\vskip 10pt

Capacitively coupled noise is common-mode in nature: affecting all conductors within a cable relatively the same.  Thus, differential signaling helps avoid problems even without shielding.







\vskip 20pt \vbox{\hrule \hbox{\strut \vrule{} {\bf Suggestions for Socratic discussion} \vrule} \hrule}

\begin{itemize}
\item{} Explain why substantial DC electric fields cannot exist within a solid conductor.
\item{} Explain why substantial DC electric fields cannot span the diameter of a hollow conductor.
\item{} Examining the last schematic diagram illustrating differential signaling, identify where an AC voltmeter {\it could be connected} between to measure AC noise voltage.
\item{} Explain how {\it ground loops} are formed, and why they are bad.
\item{} What will happen if the shield conductor of a shielded cable is left ungrounded?
\end{itemize}







\vfil \eject

\noindent
{\bf Prep Quiz:}

Signal cables may be made more resistant to interference from external electric fields by:

\begin{itemize}
\item{} Grounding the shield conductor at one end of the cable
\vskip 5pt 
\item{} Using spring-type terminal blocks instead of screw-type
\vskip 5pt 
\item{} Twisting wire pairs together along their entire length
\vskip 5pt 
\item{} Looping the cable through a ferrite magnetic core
\vskip 5pt 
\item{} Grounding the shield conductor at both ends of the cable
\vskip 5pt 
\item{} Coiling excess cable length around a metal pipe or pole
\end{itemize}

%INDEX% Reading assignment: Lessons In Industrial Instrumentation, instrument connections (electric field de-coupling)

%(END_NOTES)

