
%(BEGIN_QUESTION)
% Copyright 2010, Tony R. Kuphaldt, released under the Creative Commons Attribution License (v 1.0)
% This means you may do almost anything with this work of mine, so long as you give me proper credit

Read and outline the ``Magnetic Field (Inductive) De-coupling'' subsection of the ``Electrical Signal and Control Wiring'' section of the ``Instrument Connections'' chapter in your {\it Lessons In Industrial Instrumentation} textbook.  Note the page numbers where important illustrations, photographs, equations, tables, and other relevant details are found.  Prepare to thoughtfully discuss with your instructor and classmates the concepts and examples explored in this reading.

\underbar{file i04414}
%(END_QUESTION)





%(BEGIN_ANSWER)


%(END_ANSWER)





%(BEGIN_NOTES)

Magnetic fields {\it loop} rather than terminate as electric fields do.  Thus, you cannot stop a magnetic field, but can only re-direct its path.  Mu-metal shields help redirect magnetic fields around sensitive circuits and devices by providing very low-reluctance pathways for magnetic flux to flow.

\vskip 10pt

Capacitive coupling tends to be common-mode, but inductive coupling is differential (pairs of wires in a cable acting as a loop to form an inductor to pick up interference).  The natural loop formed by two conductors in a cable provide a ``coil'' for induced voltage to form.

\vskip 10pt

One trick to minimize magnetically induced noise in a two-wire cable is to twist those two conductors so that the circuit ``appears'' to be a long series of {\it opposed} loops whose induced noise voltages cancel each other out.  In this way, a magnetic field can still penetrate a cable, but have little or no noise-inducing effect.  This is why instrumentation cables are generally manufactured as {\it twisted, shielded pairs}: the twisting guards against magnetically induced noise, while the shielding guards against electrically induced noise.









\vskip 20pt \vbox{\hrule \hbox{\strut \vrule{} {\bf Suggestions for Socratic discussion} \vrule} \hrule}

\begin{itemize}
\item{} Explain why magnetic fields are so much more difficult to shield than electric fields.
\item{} Explain how and why twisting the conductors within a cable helps guard against magnetically induced noise.
\item{} Explain what Lenz's Law is, and describe a practical application of it.
\end{itemize}











\vfil \eject

\noindent
{\bf Prep Quiz:}

Signal cables may be made more resistant to interference from external magnetic fields by:

\begin{itemize}
\item{} Grounding the shield conductor at one end of the cable
\vskip 5pt 
\item{} Using spring-type terminal blocks instead of screw-type
\vskip 5pt 
\item{} Twisting wire pairs together along their entire length
\vskip 5pt 
\item{} Manufacturing the wires from aluminum rather than copper
\vskip 5pt 
\item{} Grounding the shield conductor at both ends of the cable
\vskip 5pt 
\item{} Coiling excess cable length around a metal pipe or pole
\end{itemize}

%INDEX% Reading assignment: Lessons In Industrial Instrumentation, instrument connections (magnetic field de-coupling)

%(END_NOTES)

