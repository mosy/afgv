
%(BEGIN_QUESTION)
% Copyright 2010, Tony R. Kuphaldt, released under the Creative Commons Attribution License (v 1.0)
% This means you may do almost anything with this work of mine, so long as you give me proper credit

Read and outline the ``High-Frequency Signal Cables'' subsection of the ``Electrical Signal and Control Wiring'' section of the ``Instrument Connections'' chapter in your {\it Lessons In Industrial Instrumentation} textbook.  Note the page numbers where important illustrations, photographs, equations, tables, and other relevant details are found.  Prepare to thoughtfully discuss with your instructor and classmates the concepts and examples explored in this reading.

\underbar{file i04415}
%(END_QUESTION)





%(BEGIN_ANSWER)


%(END_ANSWER)





%(BEGIN_NOTES)

Electrical signals ``echo'' in long cables.  In order to deal with these reflected signals is to either absorb them at the end(s) of a cable, or reduce the rate of communication until they lose their effectiveness to corrupt the signals.  Fieldbus networks use the former approach (terminating resistors installed at cable ends), while HART uses the latter approach (very slow bit rate).  Improperly terminated and/or installed cables may cause problems for high-frequency signals that were never seen with low-frequency and DC signals.

\vskip 10pt

A time-domain reflectometer (TDR) is a test instrument used to check a cable for reflections and other high-frequency problems.









\vskip 20pt \vbox{\hrule \hbox{\strut \vrule{} {\bf Suggestions for Socratic discussion} \vrule} \hrule}

\begin{itemize}
\item{} Explain how Fieldbus digital instrument networks deal with the problem of reflected signals in the cabling.
\item{} Explain how HART digital instrument networks deal with the problem of reflected signals in the cabling.
\item{} Identify additional selection and installation criteria for high-speed digital signal cables that are not present for analog (e.g. 4-20 mA) signal cables.
\end{itemize}










\vfil \eject

\noindent
{\bf Summary Quiz:}

To avoid data-corruption problems in digital network wiring lacking termination resistors, one must:

\begin{itemize}
\item{} Disable parity checking on all devices
\vskip 5pt 
\item{} Ground both ends of the cable's shield
\vskip 5pt 
\item{} Use null-modem cables for all connections
\vskip 5pt 
\item{} Run the cables through {\it metal} conduit
\vskip 5pt 
\item{} Limit the bit rate (speed) of the data
\vskip 5pt 
\item{} Route the cable alongside power conductors
\end{itemize}


%INDEX% Reading assignment: Lessons In Industrial Instrumentation, instrument connections (high-frequency cables)

%(END_NOTES)

