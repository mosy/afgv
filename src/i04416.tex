
%(BEGIN_QUESTION)
% Copyright 2010, Tony R. Kuphaldt, released under the Creative Commons Attribution License (v 1.0)
% This means you may do almost anything with this work of mine, so long as you give me proper credit

Read and outline the ``EIA/TIA-232'' subsection of the ``EIA/TIA-232, 422, and 485 Networks'' section of the ``Digital Data Acquisition and Networks'' chapter in your {\it Lessons In Industrial Instrumentation} textbook.  Note the page numbers where important illustrations, photographs, equations, tables, and other relevant details are found.  Prepare to thoughtfully discuss with your instructor and classmates the concepts and examples explored in this reading.

\underbar{file i04416}
%(END_QUESTION)





%(BEGIN_ANSWER)


%(END_ANSWER)





%(BEGIN_NOTES)

RS-232 is an ``unbalanced'' style of network, with all signal voltages referenced to a common ground terminal.  $-3$ volts (or greater) is a ``mark'' (1) while +3 volts or greater is a ``space'' (0) as interpreted by the receiver.  Transmitters are supposed to generate at least $\pm$ 5 volts for a 2 volt noise margin.

\vskip 10pt

Essential DE-9 connector pins for a DTE (e.g. personal computer, other device at end of 232 link).  For a DCE (e.g. modem), reverse assignments of pins 2 and 3:

\item{$(2)$} Receive
\item{$(3)$} Transmit
\item{$(5)$} Signal ground

\vskip 10pt

A ``null modem'' cable swapping pins 2 and 3 is necessary to connect a DTE to a DTE, or a DCE to a DCE.

\vskip 10pt

RS-232 limited in data rate and distance (19.2 kbps at 50 feet).  Maximum cable distance is an inverse function of data transmission speed: the faster you wish to communicate data, the less distance you can do it over.









\vskip 20pt \vbox{\hrule \hbox{\strut \vrule{} {\bf Suggestions for Socratic discussion} \vrule} \hrule}

\begin{itemize}
\item{} Explain what ``noise margin'' is in a signal communication standard, and how it is calculated.
\item{} Assuming an unshielded, untwisted pair cable, is an RS-232 digital signal susceptible to electric field interference, magnetic field interference, or both?
\item{} Describe the differences between DTE and DCE devices, and why this matters to us.
\item{} Explain what a ``null modem'' cable does in an RS-232 network.
\item{} If a DTE were connected to another DTE using a plain cable (not a ``null modem'' cable), what would happen?
\item{} Referencing the diagram in the textbook showing two DTE devices connected together through two DCE devices and a telephone line, ask students to identify the effects of any (one) particular conductor in that system failing open.
\item{} Identify some of the fundamental factors limiting bit rate in an EIA/TIA-232 communication channel.  What is going on, at an electrical circuit level, that prevents such a network from communicating at arbitrarily high speeds?
\end{itemize}






 

\vfil \eject

\noindent
{\bf Prep Quiz:}

Explain the purpose of a {\it null modem} cable in an EIA/TIA-232 serial data system.  Be as specific as you can in your answer, and feel free to cite a realistic application if it helps your explanation.



%INDEX% Reading assignment: Lessons In Industrial Instrumentation, RS-232 networks

%(END_NOTES)

