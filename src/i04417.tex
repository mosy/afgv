
%(BEGIN_QUESTION)
% Copyright 2010, Tony R. Kuphaldt, released under the Creative Commons Attribution License (v 1.0)
% This means you may do almost anything with this work of mine, so long as you give me proper credit

Read and outline the ``EIA/TIA-422 and EIA/TIA-485'' subsection of the ``EIA/TIA-232, 422, and 485 Networks'' section of the ``Digital Data Acquisition and Networks'' chapter in your {\it Lessons In Industrial Instrumentation} textbook.  Note the page numbers where important illustrations, photographs, equations, tables, and other relevant details are found.  Prepare to thoughtfully discuss with your instructor and classmates the concepts and examples explored in this reading.

\underbar{file i04417}
%(END_QUESTION)





%(BEGIN_ANSWER)


%(END_ANSWER)





%(BEGIN_NOTES)

RS-422 and RS-485 use differential signaling rather than single-ended signaling like RS-232.  This reduces the effects of common-mode noise.  RS-422 is simplex, while RS-485 is duplex.  Four wires required for full-duplex RS-485 ; two wires sufficient for half-duplex RS-485.  A ground wire should be connected between devices to limit common-mode voltage to reasonable levels.

\vskip 10pt

RS-232 is an ``unbalanced'' style of network, with all signal voltages referenced to a common ground terminal.  $-0.2$ volts (or greater) is a ``mark'' (1) while +0.2 volts or greater is a ``space'' (0) as interpreted by the receiver.  RS-485 transmitters are supposed to generate at least $\pm$ 1.5 volts for a 1.3 volt noise margin.  RS-422 transmitters supposed to generate at least $\pm$ 2 volts for a 1.8 volt noise margin.

\vskip 10pt

RS-422/485 not as limited in data rate and distance as RS-232 (200 kbps at 100 feet with no terminators; 1 Mbps at 100 feet with terminators).

\vskip 10pt

No standardization among manufacturers as to how to label pins on RS-422 or RS-485 device. + and $-$ common, A and B also common.  Fortunately, wrong connections between RS-422/485 terminals will not cause harm to the devices!

\vskip 10pt

Connecting termination resistors to an RS-485 network may mess up the biasing (resistors inside devices are now loaded).











\vskip 20pt \vbox{\hrule \hbox{\strut \vrule{} {\bf Suggestions for Socratic discussion} \vrule} \hrule}

\begin{itemize}
\item{} Explain what ``noise margin'' is in a signal communication standard, and how it is calculated.
\item{} Assuming an unshielded, untwisted pair cable, is an RS-485 digital signal susceptible to electric field interference, magnetic field interference, or both?
\item{} What does it mean to say that RS-422 and RS-485 are OSI layer-1 standards?
\item{} How may you predict whether or not an RS-422 or RS-485 network will require termination resistors?
\item{} Explain why RS-422 and RS-485 are able to get away with such low signaling voltages compared to RS-232.
\item{} Explain how {\it biasing resistors} work in an RS-485 system.
\end{itemize}











\vfil \eject

\noindent
{\bf Prep Quiz:}

The advantage of {\it differential signaling} such as that used by EIA/TIA-422 and -485 networks is:

\begin{itemize}
\item{} All data communicated this way will be securely encrypted
\vskip 5pt 
\item{} Fewer electronic components are necessary to build the devices
\vskip 5pt 
\item{} There is no need to synchronize bit rates between transmitter and receiver
\vskip 5pt 
\item{} Smaller-gauge wire may be used, thus saving money in cabling
\vskip 5pt 
\item{} Errors in the data are easier to detect this way
\vskip 5pt 
\item{} Increased noise immunity over single-ended network standards
\end{itemize}

%INDEX% Reading assignment: Lessons In Industrial Instrumentation, RS-422 and RS-485 networks

%(END_NOTES)

