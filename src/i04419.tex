
%(BEGIN_QUESTION)
% Copyright 2010, Tony R. Kuphaldt, released under the Creative Commons Attribution License (v 1.0)
% This means you may do almost anything with this work of mine, so long as you give me proper credit

Read and outline the introduction to the ``Ethernet Networks'' section of the ``Digital Data Acquisition and Networks'' chapter in your {\it Lessons In Industrial Instrumentation} textbook.  Note the page numbers where important illustrations, photographs, equations, tables, and other relevant details are found.  Prepare to thoughtfully discuss with your instructor and classmates the concepts and examples explored in this reading.

\underbar{file i04419}
%(END_QUESTION)





%(BEGIN_ANSWER)


%(END_ANSWER)





%(BEGIN_NOTES)

Ethernet was invented by Bob Metcalf in 1973, and operated at 2.94 Mbps.  His unique invention was CSMA/CD bus arbitration, allowing all arbitration intelligence to reside in the devices themselves instead of some centralized ``traffic cop'' device.  Ethernet is layer 1 and 2 only.  ``MAC'' addresses specified for each device, to allow multiple devices to coexist on a common broadcast network.

\vskip 10pt

IEEE 802.3 is the name of the official Ethernet standard now.  Sub-standards of the 802.3 standard now exist, covering different speeds of Ethernet as well as different signal media (e.g. optical vs. electrical).











\vskip 20pt \vbox{\hrule \hbox{\strut \vrule{} {\bf Suggestions for Socratic discussion} \vrule} \hrule}

\begin{itemize}
\item{} Explain what was unique about Bob Metcalfe's ``Ethernet'' at the time he invented it.
\item{} Explain how the CSMA/CD arbitration protocol works.
\item{} Contrast CSMA/CD against other CSMA protocols, such as CSMA/BA and CSMA/CA.
\item{} Describe what ``coaxial'' cable is, and how it differs from other cable types you've used.
\end{itemize}









\vfil \eject

\noindent
{\bf Prep Quiz:}

Ethernet uses which method of channel arbitration?

\begin{itemize}
\item{} Token passing
\vskip 5pt 
\item{} CSMA/BA (bitwise arbitration)
\vskip 5pt 
\item{} Master/Slave
\vskip 5pt 
\item{} CSMA/CA (collision avoidance)
\vskip 5pt 
\item{} Jabbering
\vskip 5pt 
\item{} CSMA/CD (collision detection)
\vskip 5pt 
\item{} TDMA
\end{itemize}






\vfil \eject

\noindent
{\bf Prep Quiz:}

A {\it collision} in an Ethernet network is defined as:

\begin{itemize}
\item{} When two or more devices try to transmit at the same time
\vskip 5pt 
\item{} When the ``master'' node in the network fails
\vskip 5pt 
\item{} When one device transmits non-stop, ``hogging'' the network
\vskip 5pt 
\item{} An error detected in the transmission of a data packet
\vskip 5pt 
\item{} When a device completely fails while in mid-transmission
\vskip 5pt 
\item{} When devices of different speeds (10/100 Mbps) attempt to communicate
\end{itemize}



%INDEX% Reading assignment: Lessons In Industrial Instrumentation, Ethernet (intro)

%(END_NOTES)

