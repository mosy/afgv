
%(BEGIN_QUESTION)
% Copyright 2010, Tony R. Kuphaldt, released under the Creative Commons Attribution License (v 1.0)
% This means you may do almost anything with this work of mine, so long as you give me proper credit

Read and outline the ``Repeaters (Hubs)'' subsection of the ``Ethernet Networks'' section of the ``Digital Data Acquisition and Networks'' chapter in your {\it Lessons In Industrial Instrumentation} textbook.  Note the page numbers where important illustrations, photographs, equations, tables, and other relevant details are found.  Prepare to thoughtfully discuss with your instructor and classmates the concepts and examples explored in this reading.

\underbar{file i04420}
%(END_QUESTION)





%(BEGIN_ANSWER)


%(END_ANSWER)





%(BEGIN_NOTES)

Bob Metcalf's original Ethernet was a completely passive network, with tee connectors on a coaxial cable, and termination resistors at each far end.  Signal strength degraded at each tee connection.

\vskip 10pt

Repeaters (``hubs'') re-amplify the digital signals for superior strength, and also self-terminate all cables plugged into them.  Hubs are layer-1 devices only, and function as DCEs (whereas personal computers would be Ethernet DTEs).

\vskip 10pt

Cascaded hubs form a single collision domain, because they act to form one big broadcast network (where all devices sense all transmissions).








\vskip 20pt \vbox{\hrule \hbox{\strut \vrule{} {\bf Suggestions for Socratic discussion} \vrule} \hrule}

\begin{itemize}
\item{} Explain why Metcalfe's original Ethernet design using coaxial cable and tee connectors was problematic.  Specifically, identify points where such a network could experience problems or fail outright.
\item{} Explain why two terminating resistors were required at the far ends of the coaxial cable network in Metcalfe's original Ethernet design, but not at the ends of any stubs to that main trunk.
\item{} Explain why no terminating resistors need to be connected to any Ethernet cables when they are plugged into a {\it hub}.
\item{} Explain why you cannot ``ping'' a hub, only devices connected to a hub.
\item{} Describe what a ``collision domain'' is in an Ethernet network.  Does this concept extend to CSMA/BA or CSMA/CA networks as well?  Explain why or why not.
\item{} Describe what a ``collision domain'' is in an Ethernet network.  Does this concept extend to Master-Slave or token-passing or TDMA networks as well?  Explain why or why not.
\item{} Identify advantages of Ethernet over other channel arbitration methods that avoid collision.
\end{itemize}

%INDEX% Reading assignment: Lessons In Industrial Instrumentation, Ethernet (repeaters)

%(END_NOTES)

