
%(BEGIN_QUESTION)
% Copyright 2010, Tony R. Kuphaldt, released under the Creative Commons Attribution License (v 1.0)
% This means you may do almost anything with this work of mine, so long as you give me proper credit

Read and outline the ``Ethernet Cabling'' subsection of the ``Ethernet Networks'' section of the ``Digital Data Acquisition and Networks'' chapter in your {\it Lessons In Industrial Instrumentation} textbook.  Note the page numbers where important illustrations, photographs, equations, tables, and other relevant details are found.  Prepare to thoughtfully discuss with your instructor and classmates the concepts and examples explored in this reading.

\underbar{file i04421}
%(END_QUESTION)





%(BEGIN_ANSWER)


%(END_ANSWER)





%(BEGIN_NOTES)

Standard Ethernet (cat 5) cabling is unshielded, twisted pair (UTP), using RJ-45 connectors to terminate.  For 10 Mbps and 100 Mbps Ethernet, only 4 wires out of 8 in the cat 5 cable are used!  1000 Mbps Ethernet uses all 8 conductors in the cable.

\vskip 10pt

UTP cable introduced separate TX and RX wire pairs, as opposed to Metcalf's original coaxial cabling where only two conductors carried all signals.  This means UTP cables are made in ``straight'' as well as ``crossover'' styles.  Straight cables used for DTE-DCE connections.  Crossover cables used for DTE-DTE and DCE-DCE connections.  Some hubs provide manual crossover switching, some provide none, some provide automatic crossover switching.

\vskip 10pt

Ethernet hubs typically have LED indicators to show cable continuity, data activity, and collisions.






\vskip 20pt \vbox{\hrule \hbox{\strut \vrule{} {\bf Suggestions for Socratic discussion} \vrule} \hrule}

\begin{itemize}
\item{} Does Ethernet use single-ended (unbalanced) or differential (balanced) voltage signals to communicate?  How can we tell by examining the pin assignments on RJ-45 connectors?
\item{} Explain why Ethernet cables use twisted-pairs, but not (necessarily) shielded pairs.  {\it Hint: Ethernet uses differential signaling (like RS-485), not ground-referenced signaling (like RS-232).}
\item{} In UTP Ethernet cable, the twist rates of the different pairs are actually different from each other.  Explain why this is.
\item{} Could Metcalfe's original coaxial Ethernet cable concept support full-duplex communication?  Why or why not?
\item{} Can UTP Ethernet cable support full-duplex communication?  Why or why not?
\item{} Explain what a ``crossover'' cable is, and where one should be used.
\item{} Devise a multimeter test for determining whether or not a cable with RJ-45 plugs on either end is a ``straight'' cable or a ``crossover'' cable.
\item{} Explain the purpose of the ``Normal/Uplink'' pushbutton on the Netgear Ethernet hub shown in the photo.
\item{} Identify some of the indicator functions of LEDs found on the face of a typical hub.
\end{itemize}

%INDEX% Reading assignment: Lessons In Industrial Instrumentation, Ethernet (cabling)

%(END_NOTES)

