
%(BEGIN_QUESTION)
% Copyright 2014, Tony R. Kuphaldt, released under the Creative Commons Attribution License (v 1.0)
% This means you may do almost anything with this work of mine, so long as you give me proper credit

Read and outline the ``Switching Hubs'' subsection of the ``Ethernet Networks'' section of the ``Digital Data Acquisition and Networks'' chapter in your {\it Lessons In Industrial Instrumentation} textbook.  Note the page numbers where important illustrations, photographs, equations, tables, and other relevant details are found.  Prepare to thoughtfully discuss with your instructor and classmates the concepts and examples explored in this reading.

\vskip 30pt

Note: a video demonstration of the difference between an Ethernet switch and an Ethernet hub (repeater) may be viewed on the BTC Instrumentation YouTube channel (search for a video named ``Ethernet hubs versus switches'').  In this video, we see a potential problem that can arise if you move cables from port to port on an Ethernet switch.

\underbar{file i04422}
%(END_QUESTION)





%(BEGIN_ANSWER)


%(END_ANSWER)





%(BEGIN_NOTES)

Switching hubs reduce traffic in network segments by monitoring MAC addresses in data packets, then routing packets only to those ports with the receiving addresses.  This also creates separate ``collision domains'' to help reduce collisions.  Switching hubs are layer-2 devices (as well as operating on layer 1 to amplify the signals).

\vskip 10pt

Switches are superior to simple (repeating) hubs in almost every way.  The cost difference is slight.








\vskip 20pt \vbox{\hrule \hbox{\strut \vrule{} {\bf Suggestions for Socratic discussion} \vrule} \hrule}

\begin{itemize}
\item{} Explain how a ``switch'' differs from a ``hub'' in Ethernet networks.
\item{} Explain how switches help segregate collision domains in an Ethernet network, and how this improves communication throughout the network.
\item{} Describe a condition where two devices collide with each other, but do not prevent two other devices (in another collision domain) from successfully communicating with each other.
\item{} Create your own human analogy for explaining collision ``domains'' in any CSMA/CD network, using it to explain the advantage of using switches rather than simple repeating hubs.
\item{} Can switches be used in CSMA/BA networks?  Why or why not?
\item{} Can switches be used in CSMA/CA networks?  Why or why not?
\end{itemize}










\vfil \eject

\noindent
{\bf Summary Quiz:}

An Ethernet {\it switch} (also called a {\it switching hub}) differs from a regular hub (repeater) in that it:

\begin{itemize}
\item{} Operates at about 10 times the speed of a regular (repeating) hub
\vskip 5pt 
\item{} Directs Ethernet packets only to the port where the destination device attaches
\vskip 5pt 
\item{} Provides connection points for optical fiber lines instead of just copper wires
\vskip 5pt 
\item{} Uses a switching power supply for increased energy efficiency and smaller space
\vskip 5pt 
\item{} Screens Ethernet packets for data containing computer virus code
\vskip 5pt 
\item{} Requires the use of crossover cables to connect to other Ethernet devices
\end{itemize}


%INDEX% Reading assignment: Lessons In Industrial Instrumentation, Ethernet (switching hubs)

%(END_NOTES)

