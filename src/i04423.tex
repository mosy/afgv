
%(BEGIN_QUESTION)
% Copyright 2010, Tony R. Kuphaldt, released under the Creative Commons Attribution License (v 1.0)
% This means you may do almost anything with this work of mine, so long as you give me proper credit

Data sent in {\it synchronous} serial networks is very efficient: each bit in the datastream's bandwidth is used to transfer data and nothing else.  Data sent in {\it asynchronous} serial networks is less efficient, because in each data frame there are ``extra'' bits necessary for the {\it start}, {\it stop}, and {\it parity} signals.

\vskip 10pt

Calculate the amount of time required to transfer 400 kilobytes of data in a synchronous serial network operating at 56 kilobits per second.

\vskip 10pt

Calculate the amount of time required to transfer 400 kilobytes of data in an asynchronous serial network operating at 56 kilobits per second, with 7 data bits, 1 start bit, 2 stop bits, and 1 parity bit.

\vskip 20pt \vbox{\hrule \hbox{\strut \vrule{} {\bf Suggestions for Socratic discussion} \vrule} \hrule}

\begin{itemize}
\item{} Which type of serial network is more commonly used in long-haul applications, such as digital communications trunks between major cities in the United States?
\item{} Are there any advantages to the slower type of serial network?
\end{itemize}

\underbar{file i04423}
%(END_QUESTION)





%(BEGIN_ANSWER)

\noindent
{\bf Partial answer:}

\vskip 10pt

57.1 seconds of time required for the synchronous network.

%(END_ANSWER)





%(BEGIN_NOTES)

The time required in the synchronous network is easy to calculate.  Simply take the number of total bits and divide by the bit rate:

$${400 \hbox{ kbytes} \over 56 \hbox{ kbit/sec}} = {3200 \hbox{ kbits} \over 56 \hbox{ kbit/sec}} = 57.143 \hbox{ sec}$$  


In the asynchronous network, however, 11 bits must be communicated for every 7 bits of data, lengthening the transfer time by a factor of $11 \over 7$.

$$(57.143 \hbox{ sec}) \left(11 \over 7 \right) = 89.796 \hbox{ sec}$$  

%INDEX% Networking, serial data: transfer time

%(END_NOTES)

