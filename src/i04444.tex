
%(BEGIN_QUESTION)
% Copyright 2010, Tony R. Kuphaldt, released under the Creative Commons Attribution License (v 1.0)
% This means you may do almost anything with this work of mine, so long as you give me proper credit

Read and outline the introduction to the ``Internet Protocol (IP)'' section of the ``Digital Data Acquisition and Networks'' chapter in your {\it Lessons In Industrial Instrumentation} textbook.  Note the page numbers where important illustrations, photographs, equations, tables, and other relevant details are found.  Prepare to thoughtfully discuss with your instructor and classmates the concepts and examples explored in this reading.

\underbar{file i04444}
%(END_QUESTION)





%(BEGIN_ANSWER)


%(END_ANSWER)





%(BEGIN_NOTES)

Internet Protocol (IP) is a communication standard allowing data to be transmitted over virtually any kind of physical network.  IP specifies how ``packets'' of date may be routed across alternate lines of communication, which makes possible large, heterogenous communication networks.

\vskip 10pt

IP relies on other, higher-level protocols such as TCP to divide blocks of digital data into small enough pieces to be communicated as packets and to organize the reassembly of those pieces at the receiving end.  This is why IP is usually found in conjunction with TCP (which does these important functions), as the TCP/IP protocol.





\vskip 20pt \vbox{\hrule \hbox{\strut \vrule{} {\bf Suggestions for Socratic discussion} \vrule} \hrule}

\begin{itemize}
\item{} Explain what Internet Protocol (IP) does, in your own words.
\item{} Explain what it means for a digital message to be ``packetized'' and sent over a wide-ranging network as ``packets'' instead of one continuous message.
\item{} Explain what it means to ``interleave'' different sets of data over one network, and how the ``packetization'' of data enabled by Internet Protocol (IP) makes this possible.
\item{} Given that the internet was the invention of DARPA (a government agency charged with defense-related technology), what values and philosophies do you suppose guided the development of Internet Protocol (IP)?
\item{} Explain how IP can work with lower-level protocols such as Ethernet or RS-485 to form a complete system for transferring data.
\item{} What type(s) of data comprise an IP {\it packet}?
\end{itemize}

%INDEX% Reading assignment: Lessons In Industrial Instrumentation, Digital data and networks (Internet protocol)

%(END_NOTES)

