
%(BEGIN_QUESTION)
% Copyright 2010, Tony R. Kuphaldt, released under the Creative Commons Attribution License (v 1.0)
% This means you may do almost anything with this work of mine, so long as you give me proper credit

Read and outline the ``IP Addresses'' subsection of the ``Internet Protocol (IP)'' section of the ``Digital Data Acquisition and Networks'' chapter in your {\it Lessons In Industrial Instrumentation} textbook.  Note the page numbers where important illustrations, photographs, equations, tables, and other relevant details are found.  Prepare to thoughtfully discuss with your instructor and classmates the concepts and examples explored in this reading.

\underbar{file i04445}
%(END_QUESTION)





%(BEGIN_ANSWER)


%(END_ANSWER)





%(BEGIN_NOTES)

IP is strictly a layer-3 (OSI model) standard, not dealing with voltage levels, bit rates, arbitration methods, etc.  IP strictly deals with the routing of packets across a heterogenous network.  DCE devices that help route IP packets are called ``routers.''

\vskip 10pt

IPv4 addresses are 32 bits in length, and usually expressed as four decimal octets (e.g. 169.254.5.100).  Each device in an IP link must have its own unique IP address.

\vskip 10pt

IP addresses are necessary to identify nodes on an IP network.  These addresses are separate from lower-level (e.g. MAC) addresses, and allow data navigation across wide networks spanning different communication technologies.  IP addresses are to Ethernet MAC addresses as postal addresses are to Social Security numbers: they serve different purposes.  The existence of IP addresses makes packet routing possible even when Ethernet is not part of the communication link.

\vskip 10pt

The {\tt ping} command is a command-line diagnostic tool that works to check communication on an IP network by sending very short messages to specified IP addresses and displaying whether or not the attempt(s) were successful.  Successful pings verify cable integrity, IP functionality for the sending and receiving devices, and same subnet.

\vskip 10pt

IPv4 addresses limited to 32 bits ($2^{32}$ total addresses), which is limited by modern standards.  DHCP is a protocol used to {\it temporarily} assign IP addresses to computers connected to the internet.  ICANN assigns IP address ranges to businesses for internet use.  Certain ranges of addresses are ``private'' and not to be used on the internet:

\begin{itemize}
\item{} {\tt 10.0.0.0} to {\tt 10.255.255.255}
\item{} {\tt 172.16.0.0} to {\tt 172.31.255.255}
\item{} {\tt 192.168.0.0} to {\tt 192.168.255.255}
\end{itemize}

Every IP device has its own ``loopback'' IP address (127.0.0.1) which goes nowhere in the real world, but is useful to ping in order to determine if the device is IP-enabled.  Some computer software uses the loopback address for internal functions (e.g. X-windows for UNIX operating systems).










\vskip 20pt \vbox{\hrule \hbox{\strut \vrule{} {\bf Suggestions for Socratic discussion} \vrule} \hrule}

\begin{itemize}
\item{} Convert an arbitrary IPv4 address (given by the instructor in four-octet ``dotted decimal'' form) into binary form.
\item{} Is it possible to ``ping'' a digital device connected to your computer via RS-232?  Why or why not?
\item{} Is it possible to ``ping'' a digital device connected to your computer via Ethernet?  Why or why not?
\item{} Explain the rationale for DHCP -- why assign temporary IP addresses to users' computers on the internet?
\item{} If you have immediate access to a personal computer, try pinging some popular website (e.g. www.google.com) and see the results!
\end{itemize}










\vfil \eject

\noindent
{\bf Prep Quiz:}

Explain the purpose of Internet Protocol (IP).  Be thorough and complete in your answer, giving a full explanation and not just a cursory statement.


%INDEX% Reading assignment: Lessons In Industrial Instrumentation, Digital data and networks (IP addresses)

%(END_NOTES)

