
%(BEGIN_QUESTION)
% Copyright 2010, Tony R. Kuphaldt, released under the Creative Commons Attribution License (v 1.0)
% This means you may do almost anything with this work of mine, so long as you give me proper credit

Read and outline the ``Subnetworks and Subnet Masks'' subsection of the ``Internet Protocol (IP)'' section of the ``Digital Data Acquisition and Networks'' chapter in your {\it Lessons In Industrial Instrumentation} textbook.  Note the page numbers where important illustrations, photographs, equations, tables, and other relevant details are found.  Prepare to thoughtfully discuss with your instructor and classmates the concepts and examples explored in this reading.

\underbar{file i04446}
%(END_QUESTION)





%(BEGIN_ANSWER)


%(END_ANSWER)





%(BEGIN_NOTES)

Subnetwork {\it masks} define which portions of a device's IP address must match another device's IP address in order to communicate.  ``255'' means eight 1 bits, or that all bits in that octet define the subnet.  Subnetting is useful to manage high-traffic loads on large network systems, by segregating communications to distinct areas within the wider network.

\vskip 10pt

Devices on different subnets cannot communicate with each other (without special entries in their routing tables permitting this).  A device trying to ping another device on a different subnet returns a ``Destination host unreachable'' error.  A device with a wider subnet trying to ping a device with a narrower subnet will be allowed to send the ping packet but will not be able to receive the reply, hence a ``Request timed out'' error.

\vskip 10pt

Modern subnet notation uses a slash mark followed by a single number specifying the number of IP address bits defining the mask, from MSB on down.  For example, 255.255.255.0 would be /24 (the first 24 bits define the subnet).

\vskip 10pt

You can ``broadcast ping'' all devices on a subnet by pinging ``.255'' for all fields not in the subnet (e.g. 192.168.255.255).







\vskip 20pt \vbox{\hrule \hbox{\strut \vrule{} {\bf Suggestions for Socratic discussion} \vrule} \hrule}

\begin{itemize}
\item{} What is the purpose of subnetting in an IP system?
\item{} Explain why the two computers received different error messages when they had subnets of differing scope.
\item{} Explain what 169.254.5.10/24 means.
\item{} Identify the broadcast ping address for subnet 192.168.
\item{} Identify the broadcast ping address for subnet 169.254.7.
\end{itemize}


%INDEX% Reading assignment: Lessons In Industrial Instrumentation, Digital data and networks (subnetworks)

%(END_NOTES)

