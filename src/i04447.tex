
%(BEGIN_QUESTION)
% Copyright 2010, Tony R. Kuphaldt, released under the Creative Commons Attribution License (v 1.0)
% This means you may do almost anything with this work of mine, so long as you give me proper credit

Read and outline the ``Command-Line Diagnostic Utilities'' subsection of the ``Internet Protocol (IP)'' section of the ``Digital Data Acquisition and Networks'' chapter in your {\it Lessons In Industrial Instrumentation} textbook.  Note the page numbers where important illustrations, photographs, equations, tables, and other relevant details are found.  Prepare to thoughtfully discuss with your instructor and classmates the concepts and examples explored in this reading.

\underbar{file i04447}
%(END_QUESTION)





%(BEGIN_ANSWER)


%(END_ANSWER)





%(BEGIN_NOTES)

{\tt ipconfig} ({\tt ifconfig} in UNIX) useful for viewing and setting IP information on a PC.

\vskip 10pt

{\tt nslookup} shows domain name associated with an IP address.

\vskip 10pt

{\tt tracert} ({\tt traceroute} on UNIX systems) traces the path taken by a test packet from one location to another on the internet.







\vskip 20pt \vbox{\hrule \hbox{\strut \vrule{} {\bf Suggestions for Socratic discussion} \vrule} \hrule}

\begin{itemize}
\item{} Demonstrate the use of the {\tt ipconfig} command on a Microsoft Windows PC.
\item{} Demonstrate the use of the {\tt nslookup} command on a Microsoft Windows PC.
\item{} Demonstrate the use of the {\tt tracert} command on a Microsoft Windows PC.
\item{} Identify the meaning of ``HWaddr'' in the {\tt ifconfig} display shown in the book.
\item{} How do IPv6 addresses differ from IPv4 addresses?
\item{} Explain why we do not get the exact same route displayed when we run {\tt tracert} ({\tt traceroute} in UNIX) on some remote IP address.
\end{itemize}











\vfil \eject

\noindent
{\bf Summary Quiz:}

Describe a practical application for the {\tt tracert} command.  Be thorough and complete in your answer, giving a full description and not just a cursory statement.


%INDEX% Reading assignment: Lessons In Industrial Instrumentation, Digital data and networks (command-line diagnostic utilities)

%(END_NOTES)

