
%(BEGIN_QUESTION)
% Copyright 2010, Tony R. Kuphaldt, released under the Creative Commons Attribution License (v 1.0)
% This means you may do almost anything with this work of mine, so long as you give me proper credit

Suppose an ammeter inserted between test points {\bf E} and {\bf C} registers 6 mA in this series-parallel circuit:

$$\includegraphics[width=15.5cm]{i04456x01.eps}$$

Identify the likelihood of each specified fault for this circuit.  Consider each fault one at a time (i.e. no coincidental faults), determining whether or not each fault could independently account for {\it all} measurements and symptoms in this circuit.

% No blank lines allowed between lines of an \halign structure!
% I use comments (%) instead, so that TeX doesn't choke.

$$\vbox{\offinterlineskip
\halign{\strut
\vrule \quad\hfil # \ \hfil & 
\vrule \quad\hfil # \ \hfil & 
\vrule \quad\hfil # \ \hfil \vrule \cr
\noalign{\hrule}
%
% First row
{\bf Fault} & {\bf Possible} & {\bf Impossible} \cr
%
\noalign{\hrule}
%
% Another row
$R_1$ failed open &  &  \cr
%
\noalign{\hrule}
%
% Another row
$R_2$ failed open &  &  \cr
%
\noalign{\hrule}
%
% Another row
$R_3$ failed open &  &  \cr
%
\noalign{\hrule}
%
% Another row
$R_1$ failed shorted &  &  \cr
%
\noalign{\hrule}
%
% Another row
$R_2$ failed shorted &  &  \cr
%
\noalign{\hrule}
%
% Another row
$R_3$ failed shorted &  &  \cr
%
\noalign{\hrule}
%
% Another row
Current source dead &  &  \cr
%
\noalign{\hrule}
} % End of \halign 
}$$ % End of \vbox

Finally, identify the {\it next} diagnostic test or measurement you would make on this system.  Explain how the result(s) of this next test or measurement help further identify the location and/or nature of the fault.

\underbar{file i04456}
%(END_QUESTION)





%(BEGIN_ANSWER)

The 6 mA current emanating from the current source is supposed to split up into equal parts (3 mA) through resistors $R_2$ and $R_3$.  The fact that we measure a full 6 mA in the E-C wire tells us the splitting is not happening as it should.  Instead of splitting, we seem to have all the current passing through $R_3$ with none of it passing through $R_2$. 

Only two faults could explain this occurring, and that is an open $R_2$ (preventing current through it) or a shorted $R_3$ (diverting all current away from $R_2$):

% No blank lines allowed between lines of an \halign structure!
% I use comments (%) instead, so that TeX doesn't choke.

$$\vbox{\offinterlineskip
\halign{\strut
\vrule \quad\hfil # \ \hfil & 
\vrule \quad\hfil # \ \hfil & 
\vrule \quad\hfil # \ \hfil \vrule \cr
\noalign{\hrule}
%
% First row
{\bf Fault} & {\bf Possible} & {\bf Impossible} \cr
%
\noalign{\hrule}
%
% Another row
$R_1$ failed open &  & $\surd$ \cr
%
\noalign{\hrule}
%
% Another row
$R_2$ failed open & $\surd$ &  \cr
%
\noalign{\hrule}
%
% Another row
$R_3$ failed open &  & $\surd$ \cr
%
\noalign{\hrule}
%
% Another row
$R_1$ failed shorted &  & $\surd$ \cr
%
\noalign{\hrule}
%
% Another row
$R_2$ failed shorted &  & $\surd$ \cr
%
\noalign{\hrule}
%
% Another row
$R_3$ failed shorted & $\surd$ &  \cr
%
\noalign{\hrule}
%
% Another row
Current source dead &  & $\surd$ \cr
%
\noalign{\hrule}
} % End of \halign 
}$$ % End of \vbox

Measuring voltage between points C and D would indicate whether the fault was an open or a short.  If open ($R_2$), the voltage should be 6 volts.  If shorted ($R_3$), the voltage should be zero volts.

%(END_ANSWER)





%(BEGIN_NOTES)


%INDEX% Troubleshooting review: electric circuits

%(END_NOTES)

