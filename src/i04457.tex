
%(BEGIN_QUESTION)
% Copyright 2007, Tony R. Kuphaldt, released under the Creative Commons Attribution License (v 1.0)
% This means you may do almost anything with this work of mine, so long as you give me proper credit

Suppose we have three PID controllers, each implementing a different equation:

$$m = K_p e + {1 \over \tau_i} \int e \> dt + \tau_d {de \over dt} + b \hbox{\hskip 50pt {\bf Parallel}}$$

\vskip 10pt

$$m = K_p \left( e + {1 \over \tau_i} \int e \> dt + \tau_d {de \over dt} \right) + b \hbox{\hskip 50pt {\bf Ideal}}$$

\vskip 10pt

$$m = K_p \left[ \left({\tau_d \over \tau_i} + 1 \right) e + {1 \over \tau_i} \int e \> dt + \tau_d {de \over dt} \right] + b \hbox{\hskip 25pt {\bf Series}}$$

Furthermore, suppose we set the PID tuning constants of each controller identically: a gain ($K_p$) of 0.8, a reset time constant ($\tau_i$) of 0.1 minutes per repeat, and a rate time constant ($\tau_d$) of 2 minutes.  Due to the different equations, the three controllers will each act differently despite the identical tuning constants.

Rank these three controllers according to how aggressive their respective proportional, integral, and derivative actions are by writing ``Parallel,'' ``Ideal,'' and ``Series'' in the following table.  If two equations are equally aggressive for the same type of action, write both names in the same space within the table:

% No blank lines allowed between lines of an \halign structure!
% I use comments (%) instead, so that TeX doesn't choke.

$$\vbox{\offinterlineskip
\halign{\strut
\vrule \quad\hfil # \ \hfil & 
\vrule \quad\hfil # \ \hfil & 
\vrule \quad\hfil # \ \hfil \vrule \cr
\noalign{\hrule}
%
% First row
Action & Least aggressive response & Most aggressive response \cr
%
\noalign{\hrule}
%
% Another row
 &  &  \cr
Proportional &  &  \cr
 &  &  \cr
%
\noalign{\hrule}
%
% Another row
 &  &  \cr
Integral &  &  \cr
 &  &  \cr
%
\noalign{\hrule}
%
% Another row
 &  &  \cr
Derivative &  &  \cr
 &  &  \cr
%
\noalign{\hrule}
} % End of \halign 
}$$ % End of \vbox

For example, if you think the Ideal equation will respond with the most aggressive proportional action given these tuning constants, and that both the Parallel and Series equations will be equally responsive (but less than the Ideal), then you would fill in the first row of the table like this:

% No blank lines allowed between lines of an \halign structure!
% I use comments (%) instead, so that TeX doesn't choke.

$$\vbox{\offinterlineskip
\halign{\strut
\vrule \quad\hfil # \ \hfil & 
\vrule \quad\hfil # \ \hfil & 
\vrule \quad\hfil # \ \hfil \vrule \cr
\noalign{\hrule}
%
% First row
Action & Least aggressive response & Most aggressive response \cr
%
\noalign{\hrule}
%
% Another row
 &  &  \cr
Proportional & {\bf Parallel} and {\bf Series} & {\bf Ideal} \cr
 &  &  \cr
%
\noalign{\hrule}
} % End of \halign 
}$$ % End of \vbox


\underbar{file i04457}
%(END_QUESTION)





%(BEGIN_ANSWER)

I recommend deducting 1 point for each equation name improperly placed in the table.

% No blank lines allowed between lines of an \halign structure!
% I use comments (%) instead, so that TeX doesn't choke.

$$\vbox{\offinterlineskip
\halign{\strut
\vrule \quad\hfil # \ \hfil & 
\vrule \quad\hfil # \ \hfil & 
\vrule \quad\hfil # \ \hfil \vrule \cr
\noalign{\hrule}
%
% First row
Action & Least aggressive response & Most aggressive response \cr
%
\noalign{\hrule}
%
% Another row
Proportional & {\bf Parallel} and {\bf Ideal} & {\bf Series} \cr
%
\noalign{\hrule}
%
% Another row
Integral & {\bf Ideal} and {\bf Series} & {\bf Parallel}  \cr
%
\noalign{\hrule}
%
% Another row
Derivative & {\bf Ideal} and {\bf Series} & {\bf Parallel} \cr
%
\noalign{\hrule}
} % End of \halign 
}$$ % End of \vbox

%(END_ANSWER)





%(BEGIN_NOTES)

{\bf This question is intended for exams only and not worksheets!}.

%(END_NOTES)


