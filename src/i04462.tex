
%(BEGIN_QUESTION)
% Copyright 2015, Tony R. Kuphaldt, released under the Creative Commons Attribution License (v 1.0)
% This means you may do almost anything with this work of mine, so long as you give me proper credit

Read and outline the ``Basic Concept of HART'' subsection of the ``HART Digital/Analog Hybrid Standard'' section of the ``Digital Data Acquisition and Networks'' chapter in your {\it Lessons In Industrial Instrumentation} textbook.  Note the page numbers where important illustrations, photographs, equations, tables, and other relevant details are found.  Prepare to thoughtfully discuss with your instructor and classmates the concepts and examples explored in this reading.

\underbar{file i04462}
%(END_QUESTION)





%(BEGIN_ANSWER)


%(END_ANSWER)





%(BEGIN_NOTES)

HART superimposes AC signals on the 4-20 mA loop-powered DC circuit to allow digital communication to occur simultaneously with analog communication.  Technicians can configure HART instruments remotely by connecting anywhere in parallel with the transmitter!  Smart transmitters can report diagnostics, alarms, etc.

\vskip 10pt

A HART communicator (or PC with modem) is needed to communicate on a HART network.  HART is very slow (1200 bps) by design.  This means there will generally be no more than two messages communicated per second.

\vskip 10pt

More modern digital instrumentation standards such as Fieldbus are more capable, but HART remains the most popular at this time.  WirelessHART part of the version 7 HART standard.  No more 1200 bps speed limit!








\vskip 20pt \vbox{\hrule \hbox{\strut \vrule{} {\bf Suggestions for Socratic discussion} \vrule} \hrule}

\begin{itemize}
\item{} Explain why the HART protocol was invented for field instruments.
\item{} Explain why HART networks do not require terminating resistors, even when the cable length is quite long.
\item{} When Rosemount first developed HART technology, it sold ``retrofit'' kits to existing customers, so they could swap out the old analog circuitry of their model 1151 pressure transmitters and replace it with new HART circuitry.  Explain why this was a smart business strategy, as opposed to only offering the new HART technology in brand-new transmitter models.
\item{} Explain what a ``DD'' file is, and what purpose it serves in a HART digital system.
\end{itemize}






\vfil \eject

\noindent
{\bf Prep Quiz:}

HART is described as a {\it hybrid} data standard.  Identify what this word means, in this context:

\begin{itemize}
\item{} Both analog and digital data are conveyed over the same wire pair
\vskip 5pt 
\item{} The HART standard was invented by cross-pollination of devices
\vskip 5pt 
\item{} HART uses batteries to store energy and increase system efficiency
\vskip 5pt 
\item{} Both high-speed and low-speed data communication rates are supported
\vskip 5pt 
\item{} Either Manchester or FSK encoding is supported in the HART standard
\vskip 5pt 
\item{} HART devices may use either IPv4 or IPv6 network addressing
\end{itemize}

%INDEX% Reading assignment: Lessons In Industrial Instrumentation, Digital data and networks (HART basic concepts)

%(END_NOTES)

