
%(BEGIN_QUESTION)
% Copyright 2015, Tony R. Kuphaldt, released under the Creative Commons Attribution License (v 1.0)
% This means you may do almost anything with this work of mine, so long as you give me proper credit

Read and outline the ``HART physical layer'' subsection of the ``HART Digital/Analog Hybrid Standard'' section of the ``Digital Data Acquisition and Networks'' chapter in your {\it Lessons In Industrial Instrumentation} textbook.  Note the page numbers where important illustrations, photographs, equations, tables, and other relevant details are found.  Prepare to thoughtfully discuss with your instructor and classmates the concepts and examples explored in this reading.

\underbar{file i04463}
%(END_QUESTION)





%(BEGIN_ANSWER)


%(END_ANSWER)





%(BEGIN_NOTES)

HART standard developed to work over existing wiring, of long length and unknown characteristic impedance.  Thus, its data rate was designed to be very slow (1200 bps) to avoid signal reflection problems on unterminated cabling.  FSK modulation of digital data (1 mA P-P AC cycle at 1200 Hz and two cycles at 2200 Hz for 1 and 0, respectively).  HART slave devices inject AC current signals, HART master devices inject AC voltage signals.  Both read AC voltage signals.

\vskip 10pt

Loop resistance is necessary to prevent HART AC signals from being ``shorted out'' by the DC power supply.  This resistance may be anything between 250 ohms and 1100 ohms.  HART communicator may be connected in parallel with loop resistor just as well as it may be connected in parallel with the HART field instrument.









\vskip 20pt \vbox{\hrule \hbox{\strut \vrule{} {\bf Suggestions for Socratic discussion} \vrule} \hrule}

\begin{itemize}
\item{} Explain why the bit rate of a HART instrument is so terribly slow.  Why does it have to be that way?
\item{} Identify which OSI layer(s) are relevant to the HART standard.
\item{} Explain what the {\it Superposition Theorem} is for circuits, and how it is useful in analyzing HART circuits.
\item{} Explain why HART communication is impossible if a DC voltage source is connected directly in parallel with the terminals of a HART instrument.
\item{} Explain why there is a lower limit to the amount of loop resistance in a HART network.
\item{} Explain why there is an upper limit to the amount of loop resistance in a HART network.
\item{} Explain how a HART communicator is able to function when connected directly in parallel with the loop resistor rather than parallel with the HART instrument.
\end{itemize}












\vfil \eject

\noindent
{\bf Prep Quiz:}

The FSK frequencies used in the HART standard are:

\begin{itemize}
\item{} 1.2 Hz and 2.2 Hz
\vskip 5pt 
\item{} 120 Hz and 220 Hz
\vskip 5pt 
\item{} 1.2 kHz and 2.2 kHz 
\vskip 5pt 
\item{} 12 kHz and 22 kHz
\vskip 5pt 
\item{} 120 kHz and 220 kHz
\vskip 5pt 
\item{} 1.2 MHz and 2.2 MHz
\end{itemize}

%INDEX% Reading assignment: Lessons In Industrial Instrumentation, Digital data and networks (HART physical layer)

%(END_NOTES)

