
%(BEGIN_QUESTION)
% Copyright 2010, Tony R. Kuphaldt, released under the Creative Commons Attribution License (v 1.0)
% This means you may do almost anything with this work of mine, so long as you give me proper credit

Read and outline the ``HART Multidrop Mode'' subsection of the ``HART Digital/Analog Hybrid Standard'' section of the ``Digital Data Acquisition and Networks'' chapter in your {\it Lessons In Industrial Instrumentation} textbook.  Note the page numbers where important illustrations, photographs, equations, tables, and other relevant details are found.  Prepare to thoughtfully discuss with your instructor and classmates the concepts and examples explored in this reading.

\underbar{file i04464}
%(END_QUESTION)





%(BEGIN_ANSWER)


%(END_ANSWER)





%(BEGIN_NOTES)

In HART multidrop mode, all HART devices work digitally with no 4-20 mA analog signaling.  Multiple devices connected in parallel along the same two wires, taking turns digitally ``talking'' with the master device.  If device address is 0, it works in analog/digital hybrid mode (normal).  If device address is non-zero, it functions as a multidrop (digital only) device.  Addresses 1 through 15 valid, and each instrument in a multidrop network must have its own unique address number.

\vskip 10pt

Multidrop mode is rarely used, because it is terribly slow.










\vskip 20pt \vbox{\hrule \hbox{\strut \vrule{} {\bf Suggestions for Socratic discussion} \vrule} \hrule}

\begin{itemize}
\item{} What type of channel arbitration is used in a HART multidrop system (e.g. master/slave, token-passing, TDMA, CSMA, etc.)?
\item{} Describe what would happen if {\it two} HART communicators were connected simultaneously to the same HART network, explaining why.
\item{} Given that valid HART addresses range from 0 to 15, how many binary bits are used to specify the address of a HART device?
\item{} Which OSI layers are involved with multidrop HART communication?
\item{} Can you use the {\tt ping} command to test network connectivity with a HART multidrop instrument?  Why or why not?
\end{itemize}









\vfil \eject

\noindent
{\bf Summary Quiz:}

While the HART standard supports a ``multidrop'' mode where multiple HART devices share the same pair of wires for full-digital communication, this arrangement is rarely used for process {\it control} purposes.  Other digital instrumentation network standards (e.g. FOUNDATION Fieldbus) are generally used instead.  Identify why:

\begin{itemize}
\item{} Multidrop mode means you cannot access device diagnostic data anymore
\vskip 5pt 
\item{} Radio interference becomes a major problem for multidropped HART devices
\vskip 5pt 
\item{} Too unreliable, because all devices are ``lost'' if that single cable breaks
\vskip 5pt 
\item{} Total network power consumption is excessive when using multidrop HART
\vskip 5pt 
\item{} The slow data rate of HART impedes data transfer on multidropped device networks
\vskip 5pt 
\item{} HART multidrop networks are limited to a total of (only) 5 field devices
\end{itemize}

%INDEX% Reading assignment: Lessons In Industrial Instrumentation, Digital data and networks (HART multidrop)

%(END_NOTES)

