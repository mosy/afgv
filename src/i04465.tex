
%(BEGIN_QUESTION)
% Copyright 2010, Tony R. Kuphaldt, released under the Creative Commons Attribution License (v 1.0)
% This means you may do almost anything with this work of mine, so long as you give me proper credit

Read and outline the ``HART Multi-Variable Transmitters and Burst Mode'' subsection of the ``HART Digital/Analog Hybrid Standard'' section of the ``Digital Data Acquisition and Networks'' chapter in your {\it Lessons In Industrial Instrumentation} textbook.  Note the page numbers where important illustrations, photographs, equations, tables, and other relevant details are found.  Prepare to thoughtfully discuss with your instructor and classmates the concepts and examples explored in this reading.

\underbar{file i04465}
%(END_QUESTION)





%(BEGIN_ANSWER)


%(END_ANSWER)





%(BEGIN_NOTES)

Some process transmitters naturally measure more than one variable.  HART communication allows a way for such transmitters to report them all, one in analog form and the rest in digital form.

\vskip 10pt

The Rosemount model 333 Tri-Loop demux is able to take up to three channels of digital data from a HART device and output three independent 4-20 mA analog signals for an analog control system to interpret.  Doing this requires the transmitter be in ``burst'' mode where it reports all its data without waiting to be polled by a master device.  Burst mode results in about three HART telegrams per second, as opposed to about two per second for polled operation.  Even this burst mode is pitifully slow for most control applications.









\vskip 20pt \vbox{\hrule \hbox{\strut \vrule{} {\bf Suggestions for Socratic discussion} \vrule} \hrule}

\begin{itemize}
\item{} Cite an example of a multi-variable instrument. 
\item{} Describe how ``burst mode'' differs from regular HART communication.
\item{} Can multiple HART devices be operated in ``burst mode'' when installed on a multidrop network?  Why or why not?
\item{} HART instruments may be configured for ``burst mode'' in regular applications where a HART master device (e.g. communicator) is connected to the instrument and polling it.  In such applications, is ``burst mode'' necessary?  Would it be detrimental in any way?
\end{itemize}


%INDEX% Reading assignment: Lessons In Industrial Instrumentation, Digital data and networks (HART multivariable transmitters)

%(END_NOTES)

