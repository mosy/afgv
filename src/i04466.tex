
%(BEGIN_QUESTION)
% Copyright 2010, Tony R. Kuphaldt, released under the Creative Commons Attribution License (v 1.0)
% This means you may do almost anything with this work of mine, so long as you give me proper credit

Identify the practical purpose of each of these diagnostic utility programs, all accessible through the ``command line'' environment of a personal computer:

\begin{itemize}
\item{} {\tt ping}
\vskip 10pt
\item{} {\tt ipconfig} or {\tt ifconfig}
\vskip 10pt
\item{} {\tt netstat}
\vskip 10pt
\item{} {\tt tracert} or {\tt traceroute}
\vskip 10pt
\item{} {\tt nslookup}
\end{itemize}

Be sure to bring your portable computer to class and experiment with each of these commands!

\underbar{file i04466}
%(END_QUESTION)





%(BEGIN_ANSWER)

\begin{itemize}
\item{} {\tt ping} tests for the presence of an IP-enabled device on a network
\vskip 10pt
\item{} {\tt ipconfig} or {\tt ifconfig} shows the IP configuration data for a computer (IP address, subnet mask, etc.)
\vskip 10pt
\item{} {\tt netstat} displays the TCP and UDP port connection statuses for a computer
\vskip 10pt
\item{} {\tt tracert} or {\tt traceroute} traces the route taken by a data packet over the Internet from source to destination
\vskip 10pt
\item{} {\tt nslookup} shows the DNS ``name'' for a computer on the Internet
\end{itemize}


%(END_ANSWER)





%(BEGIN_NOTES)


%INDEX% Networking, diagnostic tools

%(END_NOTES)

