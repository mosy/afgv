
%(BEGIN_QUESTION)
% Copyright 2015, Tony R. Kuphaldt, released under the Creative Commons Attribution License (v 1.0)
% This means you may do almost anything with this work of mine, so long as you give me proper credit

Read and outline the introduction and the ``Modbus Data Frames'' subsection of the ``Modbus'' section of the ``Digital Data Acquisition and Networks'' chapter in your {\it Lessons In Industrial Instrumentation} textbook.  Note the page numbers where important illustrations, photographs, equations, tables, and other relevant details are found.  Prepare to thoughtfully discuss with your instructor and classmates the concepts and examples explored in this reading.

\underbar{file i04467}
%(END_QUESTION)





%(BEGIN_ANSWER)


%(END_ANSWER)





%(BEGIN_NOTES)

Modbus is a communication standard specifying how data may be exchanged between control devices.

\vskip 10pt

Modbus has two basic serial modes: ASCII and RTU.  In ASCII modes, all Modbus commands transmitted as ASCII-readable characters, making the Modbus data stream readable by a human technician monitoring with a terminal program.  In RTU mode, data transmitted directly in binary format, making RTU mode nearly twice as fast as ASCII mode.

Modbus data frames may be encapsulated in other network packets (e.g. as data in a TCP segment, in the case of Modbus/TCP).








\vskip 20pt \vbox{\hrule \hbox{\strut \vrule{} {\bf Suggestions for Socratic discussion} \vrule} \hrule}

\begin{itemize}
\item{} Modbus was developed by the first PLC company: Modicon, Inc.  Explain why a data transfer protocol such as Modbus might be useful for programmable logic controllers.
\item{} What type of channel arbitration is used in a Modbus system (e.g. master/slave, token-passing, CSMA, etc.)?
\item{} Identify which OSI layer(s) are relevant to the Modbus standard.
\item{} Identify any advantage(s) of Modbus ASCII over Modbus RTU.
\item{} Identify any advantage(s) of Modbus RTU over Modbus ASCII.
\end{itemize}

%INDEX% Reading assignment: Lessons In Industrial Instrumentation, Digital data and networks (Modbus data frames)

%(END_NOTES)

