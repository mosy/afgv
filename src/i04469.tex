
%(BEGIN_QUESTION)
% Copyright 2010, Tony R. Kuphaldt, released under the Creative Commons Attribution License (v 1.0)
% This means you may do almost anything with this work of mine, so long as you give me proper credit

Read Appendix C of the Allen-Bradley ``PowerFlex 4 Adjustable Frequency AC Drive user manual'' (document FRN 5.xx), and answer the following questions:

\vskip 10pt

Describe what a VFD (Variable Frequency Drive) is useful for.  What, exactly, does it do in a control system?

\vskip 10pt

What does this section have to say about the ``+'' and ``$-$'' wires for Modbus RS-485 devices?

\vskip 10pt

Identify some of the commands for the AC motor drive accessible as individual bits in register 8192.

\vskip 10pt

Identify the register within the AC motor drive holding the frequency ``reference'' (command) value.  This is the numerical value commanding the motor how fast to spin.  Is this numerical value specified in integer, fixed-point, or floating-point format?

\vskip 10pt

Does this AC drive accept Modbus commands in RTU format, ASCII format, or either?

\vskip 20pt \vbox{\hrule \hbox{\strut \vrule{} {\bf Suggestions for Socratic discussion} \vrule} \hrule}

\begin{itemize}
\item{} Based on your reading of this manual, is there any danger in accidently reversing the Modbus (RS-485) wire connections?
\item{} The wiring diagram on page C-1 shows a 120 ohm termination resistor installed at the cable end.  Can you think of any application where you might wish to use a different-value termination resistor on this network cable (i.e. something other than 120 $\Omega$)?
\item{} Page 1-9 of this manual describes a ``reflected wave problem'' that may manifest on long lengths of motor cable between the drive and the motor.  Based on the description and the table of figures shown on that page, what does this problem consist of?
\item{} Identify some of the error codes generated by this VFD which may be read via Modbus (held in register 8449).
\item{} The network wiring diagram shown in figure C.1 shows an interesting method of grounding the shield conductor in each cable.  Interpret this diagram, and then elaborate on alternative methods of shield grounding which would also work.
\end{itemize}

\underbar{file i04469}
%(END_QUESTION)





%(BEGIN_ANSWER)


%(END_ANSWER)





%(BEGIN_NOTES)

The purpose of any VFD is to supply power to an induction motor in such a way as to control the speed of that motor.

\vskip 10pt

Page C-1: apparently, there is no standard for how to label the ``+'' and ``$-$'' terminals on a serial Modbus device, and so the user is encouraged to swap those wire connections as a diagnostic step if they encounter trouble!  

\vskip 10pt

Page C-3: register 8192 contains bits for starting the motor, stopping the motor, jogging the motor, acceleration and deceleration limits, specifying the frequency of the motor, etc.

\vskip 10pt

Page C-4: register 8193 holds the frequency reference value in fixed-point format (XXX.X) where the least-significant digit is tenths of a Hertz.

\vskip 10pt

Page C-1 and 3-24: Modbus RTU format only, not ASCII!

%INDEX% Reading assignment: Allen-Bradley PowerFlex 4 VFD user manual

%(END_NOTES)

