
%(BEGIN_QUESTION)
% Copyright 2010, Tony R. Kuphaldt, released under the Creative Commons Attribution License (v 1.0)
% This means you may do almost anything with this work of mine, so long as you give me proper credit

Read pages 4-46 through 5-8 of the Automation Direct ``GS1 Series Drives user manual'' (document GS1-M), and answer the following questions:

\vskip 10pt

Describe what a VFD (Variable Frequency Drive) is useful for.  What, exactly, does it do in a control system?

\vskip 10pt

Identify the purpose of the ``FA-ISONET'' device referenced on pages 5-6 and 5-7.

\vskip 10pt

Identify some of the status bits readable in register 48450 (2001 hex).

\vskip 10pt

Identify the register within the AC motor drive holding the speed ``reference'' (command) value.  This is the numerical value commanding the motor how fast to spin.  Is this numerical value specified in integer, fixed-point, or floating-point format?

\vskip 10pt

Does this AC drive accept Modbus commands in RTU format, ASCII format, or either?

\vskip 20pt \vbox{\hrule \hbox{\strut \vrule{} {\bf Suggestions for Socratic discussion} \vrule} \hrule}

\begin{itemize}
\item{} Identify which layer of the OSI model the FA-ISONET device operates on.
\item{} Can the Modbus bit rate of this VFD be arbitrarily set, or is it fixed at one communication speed?
\item{} Which variant of Modbus is more efficient from the standpoint of maximum data transfer in minimum time: ASCII or RTU?
\item{} Identify which register(s) within the VFD you would have to write data into via Modbus in order to command the drive to ``Run'' and to ``Stop''.
\item{} Explain the significance of the speed reference value residing in a register address beginning with ``4'' within the Modbus addressing scheme.
\end{itemize}

\underbar{file i04470}
%(END_QUESTION)





%(BEGIN_ANSWER)


%(END_ANSWER)





%(BEGIN_NOTES)

The purpose of any VFD is to supply power to an induction motor in such a way as to control the speed of that motor.

\vskip 10pt

Page 5-6: the FA-ISONET device converts between RS-232 (ground-referenced) signals and RS-485 (differential) signals.  The FS-ISONET happens to use 4-wire RS-485 signaling, where the transmit ad receive pairs are independent of each other.  In order to communicate with the VFD which uses a 2-wire RS-485 connection, some jumper wires must be added to parallel the ``+'' and ``$-$'' terminals of the transmit and receive pairs.

\vskip 10pt

Page 5-4: table describing each bit in this 16-bit read-only word.  Bits include stop, standby, running, jog, direction, source of motor frequency, etc.

\vskip 10pt

Page 4-49: register 42331 holds the frequency reference value in fixed-point format (XXX.X), and register 48451 holds the same value in read-only fashion (page 5-5).  The manual is not as explicit in declaring this as a fixed-point value as the Allen-Bradley PowerFlex 4 manual is, but the position of the decimal point reveals that it is.  {\it Note: page 4-49 also shows registers you must write to in order to start, stop, reverse, etc.}

\vskip 10pt

Page 5-16: communicates in either Modbus ASCII or Modbus RTU format (user-selectable)!  Pages 5-17 through 5-21 show the structure of the serial data frames and Modbus messages.

\vskip 10pt

A well-organized table of communication parameters appears on pages 5-2 through 5-3.


%INDEX% Reading assignment: Automation Direct GS1 series VFD user manual

%(END_NOTES)

