
%(BEGIN_QUESTION)
% Copyright 2015, Tony R. Kuphaldt, released under the Creative Commons Attribution License (v 1.0)
% This means you may do almost anything with this work of mine, so long as you give me proper credit

Read and outline the ``AC Induction Motors'' subsection of the ``On/Off Electric Motor Control Circuits'' section of the ``Discrete Control Elements'' chapter in your {\it Lessons In Industrial Instrumentation} textbook.  Note the page numbers where important illustrations, photographs, equations, tables, and other relevant details are found.  Prepare to thoughtfully discuss with your instructor and classmates the concepts and examples explored in this reading.

\underbar{file i04492}
%(END_QUESTION)





%(BEGIN_ANSWER)


%(END_ANSWER)





%(BEGIN_NOTES)

A {\it rotating magnetic field} may be produced by energizing sets of offset stator windings with polyphase AC current.  This field mimics the magnetized rotor inside the generator producing the polyphase power.

\vskip 10pt

An AC motor with a magnetized rotor will spin at synchronous speed, in lock-step with the rotating magnetic field.  This is called a {\it synchronous} motor.  An AC motor with a non-magnetized but conductive rotor will spin at slightly less than synchronous speed, and is called an {\it induction} motor because the rotor's torque is produced by induced currents in the rotor and the effects of Lenz's Law.  If the rotor of an induction motor ever reached synchronous speed, no torque would develop because induction would cease (no relative speed between rotating field and rotor).  If the rotor is forced over-speed, it will output power into the line (induction generator)!

Squirrel-cage induction motors use short-circuited aluminum bars in an iron rotor.  A model for an induction motor is a transformer with a short-circuited, movable secondary winding.  Motors draw a lot of current (``inrush'') when initially started, but the current tapers to normal levels once up to speed.

\vskip 10pt

Reverse a three-phase motor's direction by swapping any two lines.

\vskip 10pt

Single-phase AC power cannot create a truly rotating magnetic field, only a pulsing magnetic field.  In order to have a definite rotation necessary to start the motor, there must be multiple phases involved.  Auxiliary ``start'' windings with phase-shift capacitors (and a disconnecting speed switch) are one way to get a single-phase AC motor started.  Induction motors must start as polyphase machines, but may continue to run as single-phase machines.  Shaded-pole motors use special coils installed at the corners of the pole pieces to magnetically generate a lagging phase shift and thereby cause a definite rotation of the field.

If a three-phase motor is ``single-phased'' due to a fault, it cannot start by itself but it may continue to run once started.









\filbreak

\vskip 20pt \vbox{\hrule \hbox{\strut \vrule{} {\bf Suggestions for Socratic discussion} \vrule} \hrule}

\begin{itemize}
\item{} Explain how a rotating magnetic field is produced inside of an induction motor, referencing the multiple images of a motor stator's magnetic fields (in the textbook) if necessary.
\item{} Explain why an induction motor's rotor does not spin at full synchronous speed, but lags slightly behind the speed of the rotating magnetic field.
\item{} Describe the construction of a ``squirrel-cage'' induction motor.
\item{} Explain what ``inrush'' current is for an electric motor.
\item{} Describe what will happen if an induction motor is physically over-sped (i.e. spun faster than the synchronous speed of the rotating magnetic field) by some mechanical power source.
\item{} Explain how single-phase induction motors differ from three-phase induction motors.
\item{} Explain what it means to ``single-phase'' a three-phase induction motor, and why this is a bad thing.
\end{itemize}










\vfil \eject

\noindent
{\bf Prep Quiz:}

Explain in your own words how a {\it rotating magnetic field} is created inside an AC induction motor.











\vfil \eject

\noindent
{\bf Prep Quiz:}

Define {\it inrush current} for an AC induction motor, in your own words.










\vfil \eject

\noindent
{\bf Prep Quiz:}

Describe the difference between an {\it induction} motor versus a {\it synchronous} motor, in your own words.  Note that there is more than one way to correctly differentiate these two types of machines.









\vfil \eject

\noindent
{\bf Prep Quiz:}

The principal quality that the rotor of any AC induction motor must possess in order to spin at all is:

\begin{itemize}
\item{} Air-cooling capability
\vskip 5pt 
\item{} Ferromagnetism (iron core)
\vskip 5pt 
\item{} Electrical conductivity
\vskip 5pt 
\item{} Water-cooling capability
\vskip 5pt 
\item{} Piezoelectricity
\vskip 5pt 
\item{} Permanent magnetism
\end{itemize}






\vfil \eject

\noindent
{\bf Summary Quiz:}

A squirrel-cage induction motor's rotor spins slower than synchronous speed because:

\begin{itemize}
\item{} A power factor less than 1 prevents perfect synchronization
\vskip 5pt 
\item{} There must be a speed difference in order for induction to occur
\vskip 5pt 
\item{} The wire windings are not superconducting (i.e. they have resistance)
\vskip 5pt 
\item{} The magnetic field rotates too quickly for any mechanism to follow it
\vskip 5pt 
\item{} The rotor is a permanent magnet, with fixed north and south poles
\vskip 5pt 
\item{} Everyone knows squirrels simply cannot run that fast
\end{itemize}

%INDEX% Reading assignment: Lessons In Industrial Instrumentation, AC induction motors

%(END_NOTES)

