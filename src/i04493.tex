
%(BEGIN_QUESTION)
% Copyright 2010, Tony R. Kuphaldt, released under the Creative Commons Attribution License (v 1.0)
% This means you may do almost anything with this work of mine, so long as you give me proper credit

You task is to work in a team to disassemble a small AC induction ``squirrel cage'' electric motor.  Identify the following components of the motor once disassembled:

\begin{itemize}
\item{} Rotor
\vskip 10pt
\item{} Stator 
\vskip 10pt
\item{} Stator windings
\vskip 10pt
\item{} ``Squirrel-cage'' rotor bars
\vskip 10pt
\item{} Shorting rings (on either end of ``squirrel-cage'' rotor)
\vskip 10pt
\item{} Bearings
\vskip 10pt
\item{} Keyway (on motor shaft)
\vskip 10pt
\item{} Power terminals
\end{itemize}

Either with the motor assembled or disassembled, use your multimeter to measure terminal-to-terminal winding resistance.  How many ohms do you read from T1 to T2, or from T2 to T3, or from T1 to T3?  Should these resistance measurements be the same or different from one another?

\vskip 10pt

Either with the motor assembled or disassembled, use your multimeter to measure terminal-to-frame winding resistance.  How many ohms do you read from T1 to frame, or from T2 to frame, or from T1 to frame?  Should these resistance measurements be the same or different from one another?

\vskip 10pt

Feel free to photograph the disassembled motor with a digital camera for your own future reference.  Reassemble the motor (ensuring the shaft still spins freely) when done.



\vskip 20pt \vbox{\hrule \hbox{\strut \vrule{} {\bf Suggestions for Socratic discussion} \vrule} \hrule}

\begin{itemize}
\item{} Explain the purpose of the shorting rings on the rotor.  Why must the rotor bars be electrically shorted to each other?
\item{} What might happen to the motor if one of the rotor bar connections to a shorting ring happened to open?
\item{} Does your AC induction motor use a {\it starting capacitor}?  If so, why?  If not, why not?
\end{itemize}

\underbar{file i04493}
%(END_QUESTION)





%(BEGIN_ANSWER)

Terminal-to-terminal resistance should be very low, since you are merely measuring the DC resistance of the stator windings.  Terminal-to-frame resistance, however, should be infinite because the insulation of the stator windings should prevent any electrical contact with the motor frame.

%(END_ANSWER)





%(BEGIN_NOTES)

%INDEX% Final Control Elements, motor: ``squirrel cage''

%(END_NOTES)

