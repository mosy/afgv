
%(BEGIN_QUESTION)
% Copyright 2010, Tony R. Kuphaldt, released under the Creative Commons Attribution License (v 1.0)
% This means you may do almost anything with this work of mine, so long as you give me proper credit

Read and outline the ``Motor Protection'' subsection of the ``On/Off Electric Motor Control Circuits'' section of the ``Discrete Control Elements'' chapter in your {\it Lessons In Industrial Instrumentation} textbook.  Note the page numbers where important illustrations, photographs, equations, tables, and other relevant details are found.  Prepare to thoughtfully discuss with your instructor and classmates the concepts and examples explored in this reading.

\underbar{file i04496}
%(END_QUESTION)





%(BEGIN_ANSWER)


%(END_ANSWER)





%(BEGIN_NOTES)

Fuses and circuit breakers exist to protect conductors from damage from overcurrent.  Overload heaters exist to protect the motor itself from damage due to overcurrent (overloading).  Overload assembly + contactor = ``motor starter''.

\vskip 10pt

Solder-pot overload ``heaters'' become warm with motor current.  If current is excessive for too long, the solder melts and releases a switch contact, which opens and interrupts power to the contactor coil.  Solder pot takes time to cool down, preventing reset and immediate re-starting of an overheated motor.  Thus, the solder pot acts as a thermal model of the motor.

\vskip 10pt

Overload heaters only function well when their ambient temperature is similar to that of the motor.  

\vskip 10pt

We may measure voltage drop across an overload heater in order to qualitatively assess line current.

\vskip 10pt

{\it Protective relays} using RTDs to sense motor temperature directly may be used to protect motors against overload conditions.  Such relays also use potential transformers (PTs) and current transformers (CTs) to sense line power conditions to the device being protected.  A ``zero sequence'' CT wraps around all three motor lines to sense ground fault currents.










\vskip 20pt \vbox{\hrule \hbox{\strut \vrule{} {\bf Suggestions for Socratic discussion} \vrule} \hrule}

\begin{itemize}
\item{} Explain how overload heaters protect an electric motor against an overloading condition.
\item{} Explain the operation of a ``solder-pot'' overload heater mechanism.
\item{} Explain how overload heaters differ in operation and in purpose from a {\it fuse}. 
\item{} Explain how overload heaters differ in operation and in purpose from a {\it circuit breaker}. 
\item{} Describe a practical scenario where the current ratings of fuses or of a circuit breaker feeding power to a motor far exceeds that of the overload heaters installed in series with the motor.
\item{} Describe a practical scenario where an overload heater might fail to properly protect an electric motor against a true overload condition.
\item{} Suppose you discover an electrician pointing a cooling fan at a motor starter (at the contactor/overload assembly) on a warm day.  Why would anyone do this?  Is this a safe thing to do?  Why or why not???
\item{} Explain the purpose of the multimeter ``trick'' shown in the book, and describe what kind of fault this trick would help diagnose.
\item{} Identify a more sophisticated type of motor protection device than an overload heater.
\item{} What is a {\it protective relay}, and where might we find such a device used?
\item{} Describe the purpose of instrument transmformers such as current transformers (CTs) and potential transformers (PTs) in a protective relay system.
\item{} Explain the purpose of a ``zero sequence'' CT in a protective relay system.
\end{itemize}








\vfil \eject

\noindent
{\bf Prep Quiz:}

A {\it thermal overload heater} in a motor control circuit works by:

\begin{itemize}
\item{} Warming up enough to actuate a switch which stops the motor
\vskip 5pt 
\item{} Sensing motor winding temperature to guard against overheating
\vskip 5pt 
\item{} Using motor current to pre-heat the air and prevent condensation
\vskip 5pt 
\item{} Burning open to interrupt motor current (like a fuse)
\vskip 5pt 
\item{} Actuating the circuit breaker feeding power to the motor
\vskip 5pt 
\item{} Adds heat to the stator winding to prevent damage fom ice
\end{itemize}












\vfil \eject

\noindent
{\bf Prep Quiz:}

A {\it protective relay} is:

\begin{itemize}
\item{} A synonym for a motor contactor -- it simply controls power to a motor
\vskip 5pt 
\item{} A name for a vibration switch designed to trip if a motor becomes imbalanced
\vskip 5pt 
\item{} A device designed to detect an abnormal condition in an electrical circuit
\vskip 5pt 
\item{} A synonym for an overload heater -- it stops the motor if line current is high
\vskip 5pt 
\item{} An electromechanical device designed to sound an alarm when triggered
\vskip 5pt 
\item{} A control relay sealed so as to be protected against dust and dirt
\end{itemize}


%INDEX% Reading assignment: Lessons In Industrial Instrumentation, Motor protection

%(END_NOTES)

