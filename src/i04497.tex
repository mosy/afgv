
%(BEGIN_QUESTION)
% Copyright 2010, Tony R. Kuphaldt, released under the Creative Commons Attribution License (v 1.0)
% This means you may do almost anything with this work of mine, so long as you give me proper credit

Read and outline the ``Motor Control Circuit Wiring'' subsection of the ``On/Off Electric Motor Control Circuits'' section of the ``Discrete Control Elements'' chapter in your {\it Lessons In Industrial Instrumentation} textbook.  Note the page numbers where important illustrations, photographs, equations, tables, and other relevant details are found.  Prepare to thoughtfully discuss with your instructor and classmates the concepts and examples explored in this reading.

\underbar{file i04497}
%(END_QUESTION)





%(BEGIN_ANSWER)


%(END_ANSWER)





%(BEGIN_NOTES)

Control power transformer steps 480 VAC motor power down to 120 VAC for the contactor coil and associated switch circuitry.  Normally-closed ``OL'' contact in series with contactor coil to de-energize it if any of the overload heaters gets too warm.  The entire assemblage of contactor, overload heater, transformer, etc. is called a {\it bucket}.

\vskip 10pt

If momentary pushbuttons used to start and stop motor, Start will be NO while Stop is NC.  A NO auxiliary contact on the contactor will be connected in parallel with the Start pushbutton to function as a {\it seal-in} to keep the contactor energized when the Start pushbutton is released.  A PLC may alternatively be used to provide this latching function.

\vskip 10pt

Reversing motor control circuits use two contactors to swap lines to make the motor change direction.  Interlocking auxiliary contacts prevent simultaneous energization of the Forward and Reverse contactors, which would result in a direct line-to-line short!  Contactors may also be mechanically interlocked, such that one contactor physically cannot actuate while the other contactor is actuated.

\vskip 10pt

Digital motor buckets both monitor and command motor operation via networks such as DeviceNet.











\vskip 20pt \vbox{\hrule \hbox{\strut \vrule{} {\bf Suggestions for Socratic discussion} \vrule} \hrule}

\begin{itemize}
\item{} Refer to the diagram of a motor control ``bucket'' shown in the book and describe how the overload heaters are supposed to function to protect the motor.
\item{} Refer to the diagram of a motor control ``bucket'' shown in the book and describe how the latching start/stop circuit is supposed to function.
\item{} Explain how a reversing motor control circuit manages to reverse the rotational direction of a three-phase electric motor.
\item{} Explain the purpose of {\it interlocking} in a reversing motor starter, and how this important feature may be implemented in more than one way.
\item{} Explain what will happen if the ``Forward'' and ``Reverse'' pushbuttons are simultaneously pressed in a reversing motor control circuit.
\item{} Explain what would happen if the ``Forward'' and ``Reverse'' contactors were to somehow simultaneously close.
\item{} Imagine a particular fault in one of the motor control circuit diagrams shown in the book, and have students identify all consequences of that fault.
\item{} Describe how a ``smart'' motor control interface (such as the Square D ``Motor Logic Plus'') unit works to control an electric motor digitally.
\item{} Examine the photograph showing a bucket equipped with a Square D ``Motor Logic Plus'' controller unit, and identify where the motor's overload protection is located.
\end{itemize}











\vfil \eject

\noindent
{\bf Prep Quiz:}

A {\it seal-in contact} in a motor control circuit works to:

\begin{itemize}
\item{} Lock out electric power so the motor is safe to work on
\vskip 5pt 
\item{} Seal (prevent) dust from getting to the interior of the motor
\vskip 5pt 
\item{} Ensure the motor will not be damaged by overload conditions
\vskip 5pt 
\item{} Stop the motor whenever an emergency condition is sensed
\vskip 5pt 
\item{} Prevent the ``Stop'' button from actuating during normal operation
\vskip 5pt 
\item{} Latch the motor in the ``run'' state after releasing the ``Start'' button
\end{itemize}


%INDEX% Reading assignment: Lessons In Industrial Instrumentation, Motor control circuit wiring

%(END_NOTES)

