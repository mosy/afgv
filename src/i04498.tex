
%(BEGIN_QUESTION)
% Copyright 2010, Tony R. Kuphaldt, released under the Creative Commons Attribution License (v 1.0)
% This means you may do almost anything with this work of mine, so long as you give me proper credit

Read and outline the ``AC Motor Speed Control'' section of the ``Variable-Speed Motor Controls'' chapter in your {\it Lessons In Industrial Instrumentation} textbook.  Note the page numbers where important illustrations, photographs, equations, tables, and other relevant details are found.  Prepare to thoughtfully discuss with your instructor and classmates the concepts and examples explored in this reading.

\underbar{file i04498}
%(END_QUESTION)





%(BEGIN_ANSWER)


%(END_ANSWER)





%(BEGIN_NOTES)

AC motors work by creating a rotating magnetic field in stator windings, by energizing them with polyphase AC power.  If the rotor is magnetized, it follows the rotating field in lock-step (synchronous operation).  If the rotor is merely conductive, induced currents caused by the rotating magnetic field cause the rotor to be dragged along at a speed slightly less than synchronous (induction operation).  Speed of rotating magnetic field defined by this formula:

$$S = {120 f \over n}$$

\noindent
Where,

$S$ = Speed in RPM

$f$ = Frequency in Hz

$n$ = Number of poles (2 minimum)

\vskip 10pt

The more poles, the more AC cycles required to complete a full revolution, and thus the slower the magnetic field's rotating speed.  If we vary the frequency of the AC power, we may likewise adjust the magnetic field's rotating speed, and do so continuously.  Such variable-frequency drive (VFD) control gives precise speed control, energy savings, and reduced machine wear.  VFDs may also serve as electronic brakes, dissipating mechanical energy from the load.

\vskip 10pt

VFDs internally consist of a rectifier section to turn AC into DC, a filter section to ``smooth'' the DC power, and finally an inverter section using transistors to switch DC into AC.  The AC sine wave is synthesized using rapid pulse-width modulation.  PWM allows the drive to finely control frequency and voltage independently.

AC motor speed is a function of frequency, while torque is a function of current.  In order to maintain reasonably stable torque, a VFD must vary voltage to the motor along with frequency, since the motor's inductive reactance will vary with frequency.  This is called the volts-per-hertz ratio.

\vskip 10pt

Since AC induction motor speed is limited by the applied frequency, there is no need for speed feedback when controlling AC motors using VFDs.

\vskip 10pt

VFDs broadcast lots of radio-frequency interference (RFI) which may cause problems with other equipment.  VFD power conductors must {\it never} be routed near signal wiring, or else interference will result!











\filbreak

\vskip 20pt \vbox{\hrule \hbox{\strut \vrule{} {\bf Suggestions for Socratic discussion} \vrule} \hrule}

\begin{itemize}
\item{} Identify {\it two} different means of changing AC induction or synchronous motor speed.
\item{} Identify some advantages of using a VFD to control induction motor speed, as opposed to full-power operation.
\item{} What does it mean when a VFD causes an electric motor to {\it brake}?
\item{} Explain how a VFD works, referencing the simple schematic diagram shown in the book (showing rectifier, filter, and inverter circuit sections).
\item{} Explain how a VFD may be powered by single-phase AC (rather than 3-phase AC), referencing the simple schematic diagram shown in the book.
\item{} Explain how the operation of a VFD will be affected if one of its rectifying diodes fails open.
\item{} Explain how the operation of a VFD will be affected if one of its rectifying diodes fails shorted.
\item{} Explain how the operation of a VFD will be affected if one of its filtering capacitors fails open.
\item{} Explain how the operation of a VFD will be affected if one of its filtering capacitors fails shorted.
\item{} Explain how the operation of a VFD will be affected if one of its inverter transistors fails open.
\item{} Explain how the operation of a VFD will be affected if one of its inverter transistors fails shorted.
\item{} Explain how a VFD uses pulse-width modulation to synthesize sine wave power to an electric motor.
\item{} Explain why the {\it voltage} applied to an AC induction motor must be varied commensurate with frequency, rather than maintain a constant voltage to the motor as frequency varies.
\item{} Explain how a VFD uses pulse-width modulation to vary voltage to the motor, since the DC bus voltage is typically fixed.
\item{} Explain why DC motor speed control systems require a tachogenerator or other speed sensor for feedback, while AC motor speed control does not.
\end{itemize}














\vfil \eject

\noindent
{\bf Prep Quiz:}

The one variable {\it most} influential on the speed of an AC induction motor is:

\begin{itemize}
\item{} Frequency of the applied AC power
\vskip 5pt 
\item{} Voltage of the applied AC power
\vskip 5pt 
\item{} Temperature of the motor windings
\vskip 5pt 
\item{} Current of the applied AC power
\vskip 5pt 
\item{} Inductance of the power conductors
\vskip 5pt 
\item{} Power factor of the AC system
\end{itemize}


\vfil \eject

\noindent
{\bf Summary Quiz:}

Supposing a particular induction motor has a full-load speed of 1720 RPM at 60 Hz, calculate the necessary frequency to make it spin 475 RPM.

\begin{itemize}
\item{} 475 Hz
\vskip 5pt 
\item{} 14.19 Hz
\vskip 5pt 
\item{} 16.57 Hz
\vskip 5pt 
\item{} 39.58 Hz
\vskip 5pt 
\item{} 217.3 Hz
\vskip 5pt 
\item{} 14.65 Hz
\end{itemize}

%INDEX% Reading assignment: Lessons In Industrial Instrumentation, AC motor speed control

%(END_NOTES)

