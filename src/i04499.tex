
%(BEGIN_QUESTION)
% Copyright 2010, Tony R. Kuphaldt, released under the Creative Commons Attribution License (v 1.0)
% This means you may do almost anything with this work of mine, so long as you give me proper credit

You task is to work in a team to disassemble a contactor (and overload heater block) used for starting an AC induction motor.  Identify the following components of the device once disassembled:

\begin{itemize}
\item{} Moving contacts
\vskip 10pt
\item{} Stationary contacts
\vskip 10pt
\item{} Arc shields
\vskip 10pt
\item{} Armature (moving iron piece)
\vskip 10pt
\item{} Overload heaters 
\vskip 10pt
\item{} Coil terminals
\vskip 10pt
\item{} Contact voltage, current, and/or horsepower ratings
\end{itemize}

Feel free to photograph the disassembled contactor with a digital camera for your own future reference.  Reassemble the contactor (ensuring the armature still moves freely) when done.  Be sure to bring appropriate tools to class for this exercise (e.g. phillips and slotted screwdrivers, multimeter).

\vskip 20pt \vbox{\hrule \hbox{\strut \vrule{} {\bf Suggestions for Socratic discussion} \vrule} \hrule}

\begin{itemize}
\item{} Identify potential points of failure inside the contactor you are examining.  For each proposed fault, identify the effect(s) of that fault on the contactor's operation.
\item{} Explain how the arc shields function, not just to serve as a barrier between the arcing contacts and any nearby people, but also how the shields prevent a phase-to-phase arc from developing between adjacent contacts.
\item{} Describe a procedure you might use to clean the contact surfaces inside a motor contactor if they were to become dirty or pitted with use.
\item{} Demonstrate how you could use a multimeter to verify proper contact operation inside a contactor.
\end{itemize}

\underbar{file i04499}
%(END_QUESTION)





%(BEGIN_ANSWER)


%(END_ANSWER)





%(BEGIN_NOTES)

%INDEX% Final Control Elements, motor contactor: disassembly and reassembly

%(END_NOTES)

