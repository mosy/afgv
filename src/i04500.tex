
%(BEGIN_QUESTION)
% Copyright 2010, Tony R. Kuphaldt, released under the Creative Commons Attribution License (v 1.0)
% This means you may do almost anything with this work of mine, so long as you give me proper credit

Read and outline the ``Relay Control Systems'' chapter in your {\it Lessons In Industrial Instrumentation} textbook in its entirety.  Note the page numbers where important illustrations, photographs, equations, tables, and other relevant details are found.  Prepare to thoughtfully discuss with your instructor and classmates the concepts and examples explored in this reading.

\underbar{file i04500}
%(END_QUESTION)





%(BEGIN_ANSWER)


%(END_ANSWER)





%(BEGIN_NOTES)

AND function = series switch contacts ; OR function = parallel switch contacts ; NOT function = NC switch contact.

\vskip 10pt

SPST relay has one moving pole and one stationary contact (either NO or NC).  SPDT relay has one moving pole and two stationary contacts (NO and NC).  DPDT relay has two isolated poles, each with its own pair of stationary contacts (two NO, two NC).  Form A = NO ; Form B = NC ; Form C = NO+NC.

Different symbols used to represent relay coils (zig-zag, coily wire, or box with slash mark).  NO and NC contacts represented by different symbols as well.

\vskip 10pt

Ladder diagrams use vertical lines to show power wires, horizontal lines to show parallel circuit branches.  Each independent wire in a ladder diagram bears a unique number: common numbers mean those wires are supposed to be electrically common to each other.

``Normal'' status for a contact means when it is at rest.  For relay contacts, this means when the coil is unpowered.

\vskip 10pt

Use of arrow and X symbols to represent conductivity and non-conductivity, respectively.  This will help avoid confusion with symbols denoting {\it normally} open and closed, which is how switches are always drawn in diagrams.

\vskip 10pt
PLCs programmed in ladder diagram format use ``NO'' and ``NC'' virtual switch contacts to interpret bit states of input terminals.  Virtual ``power'' sent through rungs in the program ``energize'' virtual relay coils, which then write bit states to registers in the PLC's memory.









\vskip 20pt \vbox{\hrule \hbox{\strut \vrule{} {\bf Suggestions for Socratic discussion} \vrule} \hrule}

\begin{itemize}
\item{} Explain what is meant by the word ``normal'' in the context of switches.  For example, what is a {\it normally-closed} switch?
\item{} Describe what a ``Form A'' switch contact is.
\item{} Describe what a ``Form B'' switch contact is.
\item{} Describe what a ``Form C'' switch contact is.
\item{} Describe what a ``SPST'' switch contact is.
\item{} Describe what a ``SPDT'' switch contact is.
\item{} Describe what a ``DPST'' switch contact is.
\item{} Describe what a ``DPDT'' switch contact is.
\item{} Interpret the relay pinout diagrams shown in the book.
\item{} In the pressure alarm relay circuit shown, what would happen if the wire broke between the pressure switch contacts and the relay coil?
\item{} In the pressure alarm relay circuit shown, what would happen if the wire broke between the relay contact and the alarm lamp?
\item{} Explain how the simple PLC pressure alarm system shown in the book works.
\item{} Under what condition(s) will the virtual contact labeled {\tt I:0/2} be colored green in the PLC programming editor display?
\item{} Under what condition(s) will the virtual contact labeled {\tt I:0/2} be uncolored in the PLC programming editor display?
\item{} Under what condition(s) will the virtual contact labeled {\tt B3:0/0} be colored green in the PLC programming editor display?
\item{} Under what condition(s) will the virtual contact labeled {\tt B3:0/0} be uncolored in the PLC programming editor display?
\item{} Under what condition(s) will the real-world output on the PLC be energized?
\end{itemize}












\vfil \eject

\noindent
{\bf Prep Quiz:}

The acronym {\it DPDT} refers to:

\begin{itemize}
\item{} Two parallel normally-open contacts
\vskip 5pt 
\item{} Two series form-C contacts
\vskip 5pt 
\item{} Two sets of form-C contacts
\vskip 5pt 
\item{} Two parallel normally-closed contacts 
\vskip 5pt 
\item{} A form-A contact followed by a form-B contact
\vskip 5pt 
\item{} Don't Push, Don't Tell 
\end{itemize}

%INDEX% Reading assignment: Lessons In Industrial Instrumentation, Relay Control Systems

%(END_NOTES)

