
%(BEGIN_QUESTION)
% Copyright 2010, Tony R. Kuphaldt, released under the Creative Commons Attribution License (v 1.0)
% This means you may do almost anything with this work of mine, so long as you give me proper credit

Read and outline the ```Normal' Status of a Switch'' section of the ``Discrete Process Measurement'' chapter in your {\it Lessons In Industrial Instrumentation} textbook.  Note the page numbers where important illustrations, photographs, equations, tables, and other relevant details are found.  Prepare to thoughtfully discuss with your instructor and classmates the concepts and examples explored in this reading.

\vskip 30pt

Note: this is a subject of much confusion for students, especially with regard to {\it process} switches such as pressure, level, temperature, and flow switches.  A special practice worksheet has been made for students on this very subject called ``Process Switches and Switch Circuits'' available on the Socratic Instrumentation website.

Given the importance of this topic as well as the confusing meaning of the word ``normal'' when describing switch contacts, this reading exercise is an excellent opportunity for you to practice active reading strategies.  In particular, \underbar{you should write your own outline of this textbook section}, expressing all the major thoughts in your own words so that you will have a firmer grasp of these important concepts.  This should be your goal for all ``Read and outline . . .'' assignments, in order to maximize your learning.

\underbar{file i04501}
%(END_QUESTION)





%(BEGIN_ANSWER)


%(END_ANSWER)





%(BEGIN_NOTES)

The term ``normal'' is often misunderstood for switch contacts.  A switch in its ``normal'' status is a switch that is not being stimulated (i.e. it is {\it at rest}).  Switch symbols are always shown in schematic diagrams in their normal (resting) states.

Normally-open contacts (NO) are often referred to as {\it form A} contacts.  Normally-closed contacts (NC) are often referred to as {\it form B} contacts.

\vskip 10pt

A normally-closed (NC) low-flow switch will be held open by the typical process operating conditions, and will return to its ``normal'' (resting) status only when the process conditions become abnormal (no flow).  Here is where ``normal'' gets confusing: for the switch, it's defined as the resting state; but for the process it may be defined in a completely different way.  ``Normal'' for a switch is defined by the manufacturer, who has no idea how you are going to use this switch in your process.

\vskip 10pt

\noindent
``Normal'' switch states:

\begin{itemize}
\item{} {\bf Pushbutton:} not being pressed
\item{} {\bf Pressure:} no pressure applied
\item{} {\bf Temperature:} cold
\item{} {\bf Flow:} no flow
\item{} {\bf Proximity:} target far away
\end{itemize}

\vskip 10pt

Process switch symbols drawn so that an upward motion of mobile element corresponds to increasing stimulus.

\vskip 10pt 

NC contacts will close when stimulus is less than threshold; will open when stimulus exceeds threshold.  NO contacts will open when stimulus is less than theshold; will close when stimulus exceeds threshold.  Switch status is a function of its ``normal'' design as well as its present stimulus value.  We may examine its present status and compare with its ``normal'' design to qualitatively determine its stimulus value (e.g. looking at PLC input LEDs and determining process conditions from lit/unlit states).







\vskip 20pt \vbox{\hrule \hbox{\strut \vrule{} {\bf Suggestions for Socratic discussion} \vrule} \hrule}

\begin{itemize}
\item{} Explain why there is so much confusion surrounding the term ``normal'' for process switches.
\item{} Identify at least one fault that could prevent the LED from illuminating (regardless of process variable conditions) in the example circuit shown in the reading.
\item{} Think of an example where the ``normal'' status of a switch is not the same as its {\it typical operating} status in a process, besides the low-flow alarm switch example given in the reading.
\item{} Explain how we may determine the stimulus applied to each switch in the PLC example shown in the reading, based on terminal connections and PLC input LED statuses.
\item{} Suppose one of the process switches connected to the PLC in the book did not change state (at the PLC input) when stimulated and un-stimulated.  Identify what type of failure (open vs. shorted) would cause this to happen, for each of the switches shown.
\end{itemize}











\vfil \eject

\noindent
{\bf Prep Quiz:}

Choose the best sentence describing this circuit's behavior:

$$\includegraphics[width=15.5cm]{i04501x01.eps}$$

\begin{itemize}
\item{} The lamp will remain energized at all times regardless of the temperature
\vskip 5pt 
\item{} The lamp will de-energize when the temperature switch is hot
\vskip 5pt 
\item{} The lamp will remain de-energized at all times regardless of the temperature
\vskip 5pt 
\item{} The lamp will energize when the temperature switch is cold
\vskip 5pt 
\item{} The circuit will burst into flame when the temperature switch is cold
\end{itemize}












\vfil \eject

\noindent
{\bf Prep Quiz:}

Choose the best sentence describing this circuit's behavior:

$$\includegraphics[width=15.5cm]{i04501x02.eps}$$

\begin{itemize}
\item{} The lamp will remain energized at all times regardless of the temperature 
\vskip 5pt 
\item{} The lamp will de-energize when the temperature switch is hot
\vskip 5pt 
\item{} The lamp will remain de-energized at all times regardless of the temperature 
\vskip 5pt 
\item{} The lamp will energize when the temperature switch is hot
\vskip 5pt 
\item{} The circuit will create an arc flash when the temperature switch is hot
\end{itemize}













\vfil \eject

\noindent
{\bf Prep Quiz:}

Choose the best sentence describing this circuit's behavior:

$$\includegraphics[width=15.5cm]{i04501x03.eps}$$

\begin{itemize}
\item{} The lamp will remain de-energized at all times regardless of the flow 
\vskip 5pt 
\item{} The lamp will energize when the fluid flow rate is low
\vskip 5pt 
\item{} The lamp will energize when the fluid flow rate is high
\vskip 5pt 
\item{} The lamp will remain energized at all times regardless of the flow 
\vskip 5pt 
\item{} The circuit will create an arc blast when the flow rate is high
\end{itemize}


%INDEX% Reading assignment: Lessons In Industrial Instrumentation, switch contact status

%(END_NOTES)

