
%(BEGIN_QUESTION)
% Copyright 2010, Tony R. Kuphaldt, released under the Creative Commons Attribution License (v 1.0)
% This means you may do almost anything with this work of mine, so long as you give me proper credit

Read and outline the ``Hand Switches'' and ``Limit Switches'' sections of the ``Discrete Process Measurement'' chapter in your {\it Lessons In Industrial Instrumentation} textbook.  Note the page numbers where important illustrations, photographs, equations, tables, and other relevant details are found.  Prepare to thoughtfully discuss with your instructor and classmates the concepts and examples explored in this reading.

\underbar{file i04502}
%(END_QUESTION)





%(BEGIN_ANSWER)


%(END_ANSWER)





%(BEGIN_NOTES)

Hand switches operated by hand, may contain ``stacks'' of contact blocks for a variety of NO/NC options.

\vskip 10pt

Limit switches detect the motion of a machine part or other object (e.g. car door switch).  Will be in its ``normal'' (resting) state when nothing is touching it.  May have roller-tipped lever for low-friction contact with moving part.  Form-C styles are typical, with a ``Com'' terminal, ``NC'' terminal, and ``NO'' terminal.  Applications include control valve stem position, machine tool platen position, etc.







\vskip 20pt \vbox{\hrule \hbox{\strut \vrule{} {\bf Suggestions for Socratic discussion} \vrule} \hrule}

\begin{itemize}
\item{} Describe the construction of a ``stackable'' switch mechanism.
\item{} Describe what a {\it Form A} switch contact is.
\item{} Describe what a {\it Form B} switch contact is.
\item{} Describe what a {\it Form C} switch contact is.
\item{} How could one determine the NO/NC status of a pushbutton switch's contacts if the terminals were not labeled, and the plastic case of the switch opaque so you could not visually inspect the contacts working?
\end{itemize}

%INDEX% Reading assignment: Lessons In Industrial Instrumentation, hand and limit switches

%(END_NOTES)

