
%(BEGIN_QUESTION)
% Copyright 2010, Tony R. Kuphaldt, released under the Creative Commons Attribution License (v 1.0)
% This means you may do almost anything with this work of mine, so long as you give me proper credit

Using a terminal strip to organize all wire connections, construct a circuit to turn on a DC load (e.g. lamp, relay coil) using a {\bf proximity switch} as the sensor.  The instructor will provide all necessary components to you during class time.  Be sure to bring appropriate tools to class for this exercise (e.g. phillips and slotted screwdrivers, multimeter).

\vskip 20pt \vbox{\hrule \hbox{\strut \vrule{} {\bf Suggestions for Socratic discussion} \vrule} \hrule}

\begin{itemize}
\item{} A problem-solving technique useful for constructing circuits is to {\it sketch a schematic diagram of the intended circuit} before making a single connection.  This important step not only helps you to identify potential problems before they arise, but is also useful when constructing circuits as a team because it prompts all team members to exchange ideas and ask questions before committing to a plan of action.
\item{} Is your proximity switch NO or NC?  How can you tell?
\item{} Is your proximity switch sourcing or sinking?  How can you tell?
\item{} What are some of the advantages that proximity switches have over traditional direct-contact limit switches?
\item{} What are some good applications where we could use proximity switches in industry?
\item{} Suppose you needed to make a DC proximity switch (with transistor output) switch power to an {\it AC} load.  How could you accomplish this function, since the proximity switch can only handle DC, not AC?
\end{itemize}

\underbar{file i04504}
%(END_QUESTION)





%(BEGIN_ANSWER)


%(END_ANSWER)





%(BEGIN_NOTES)

%INDEX% Switch, proximity

%(END_NOTES)

