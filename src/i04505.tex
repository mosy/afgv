
%(BEGIN_QUESTION)
% Copyright 2010, Tony R. Kuphaldt, released under the Creative Commons Attribution License (v 1.0)
% This means you may do almost anything with this work of mine, so long as you give me proper credit

Using a terminal strip to organize all wire connections, construct a circuit to turn on a DC load (e.g. lamp, relay coil) using an industrial-style {\bf pushbutton switch}.  Experiment with both the {\it normally-open} contacts and the {\it normally-closed} contacts (assuming the pushbutton switch has both).  The instructor will provide all necessary components to you during class time.

\vskip 20pt \vbox{\hrule \hbox{\strut \vrule{} {\bf Suggestions for Socratic discussion} \vrule} \hrule}

\begin{itemize}
\item{} A problem-solving technique useful for constructing circuits is to {\it sketch a schematic diagram of the intended circuit} before making a single connection.  This important step not only helps you to identify potential problems before they arise, but is also useful when constructing circuits as a team because it prompts all team members to exchange ideas and ask questions before committing to a plan of action.
\item{} How could one determine the NO/NC status of a pushbutton switch's contacts if the terminals were not labeled, and the plastic case of the switch opaque so you could not visually inspect the contacts working?
\item{} Switch contacts usually have greater AC current ratings than DC current ratings -- why do you think this is?
\end{itemize}

\underbar{file i04505}
%(END_QUESTION)





%(BEGIN_ANSWER)


%(END_ANSWER)





%(BEGIN_NOTES)

%INDEX% Switch, pushbutton

%(END_NOTES)

