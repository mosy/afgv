
%(BEGIN_QUESTION)
% Copyright 2015, Tony R. Kuphaldt, released under the Creative Commons Attribution License (v 1.0)
% This means you may do almost anything with this work of mine, so long as you give me proper credit

Read and outline the ``Human-Machine Interfaces'' section of the ``Programmable Logic Controllers'' chapter in your {\it Lessons In Industrial Instrumentation} textbook.  Note the page numbers where important illustrations, photographs, equations, tables, and other relevant details are found.  Prepare to thoughtfully discuss with your instructor and classmates the concepts and examples explored in this reading.

\underbar{file i04526}
%(END_QUESTION)





%(BEGIN_ANSWER)


%(END_ANSWER)





%(BEGIN_NOTES)

Human-Machine Interfaces (HMIs) allow operations personell to view and control certain data within a PLC, without allowing too much access (e.g. the ability to alter the control program).  HMIs may take the form of ordinary personal computers running special HMI software, or they may be single-purpose devices designed for panel-mounting.

\vskip 10pt

HMIs communicate data to and from a PLC by means of digital networks (e.g. Ethernet).

\vskip 10pt

A {\it tag name database} within the HMI associates memory locations within the PLC with arbitrary names used as variable identifies within the HMI's graphical program.  This database also specifies each variable's numerical type, read/write privilege, which PLC is being accessed, etc.  It is important that an HMI panel is not allowed to write data to the PLC that should be written by some other means, such as the case of a discrete input on the PLC which is updated by real-world I/O states.  The general rule is: {\it never allow more than one element to write to any single data point!}

A ``discrete'' variable is one that can only be 0 or 1.  ``Integer'' variables represent whole-number quantities, the ``signed'' variety also being able to represent negative counts.  ``Floating point'' variables are for analog quantities in the real world (e.g. motor speed and temperature).  An ``ASCII'' variable is one where binary data will represent text characters.

\vskip 10pt

Although tag names are arbitrary, there is wisdom in defining a naming standard, so that variables may be more easily located and grouped by human programmers.  This is especially true when tag names are grouped in alphabetical order: the starting character of each tag name is the primary grouping factor.

One important detail is to avoid space characters in tag names, because some systems treat the space character with special significance.  Use hyphens or underscore characters instead!

\vskip 10pt

Some HMI units provide data archival, server, alarm, and/or trending capability.  Some HMI units even have the ability to function as PLCs in their own right, complete with I/O points and control programming capability!











\vskip 20pt \vbox{\hrule \hbox{\strut \vrule{} {\bf Suggestions for Socratic discussion} \vrule} \hrule}

\begin{itemize}
\item{} Describe what the purpose of an HMI is, and how one might be used in conjunction with a PLC in an industrial system.
\item{} Explain why the ``pushbutton'' tags in the displayed HMI tagname database are {\it read-only} rather than {\it read-write}.
\item{} In the book, the rule ``never allow more than one element to write to any data point''.  Explain the rationale behind this rule, and describe a case where this rule is violated and what possible consequences may result.
\item{} Explain why a naming convention for tags in an HMI's database is important for large-scale programming projects.
\end{itemize}










\vfil \eject

\noindent
{\bf Prep Quiz:}

Today's reading assignment on Human-Machine Interfaces mentions a very common error students tend to make when first learning how to interface HMIs with PLCs.  Describe this error, in your own words.

%INDEX% Reading assignment: Lessons In Industrial Instrumentation, Programmable Logic Controllers (HMIs)

%(END_NOTES)

