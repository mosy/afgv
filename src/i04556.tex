
%(BEGIN_QUESTION)
% Copyright 2010, Tony R. Kuphaldt, released under the Creative Commons Attribution License (v 1.0)
% This means you may do almost anything with this work of mine, so long as you give me proper credit

Read and outline the introductory and ``FF Design Philosophy'' sections of the ``FOUNDATION Fieldbus Instrumentation'' chapter in your {\it Lessons In Industrial Instrumentation} textbook.  Note the page numbers where important illustrations, photographs, equations, tables, and other relevant details are found.  Prepare to thoughtfully discuss with your instructor and classmates the concepts and examples explored in this reading.

\underbar{file i04556}
%(END_QUESTION)





%(BEGIN_ANSWER)


%(END_ANSWER)





%(BEGIN_NOTES)

In a typical analog signaling DCS-based control system, each instrument has its own dedicated 4-20 mA signal cable to the DCS, and all control decisions (e.g. PID) are made within the processor of the DCS.  AI, PID, and AO software function blocks within the DCS manage all the signal processing.

\vskip 10pt

In a digital bus system such as Profibus, instruments may be trunked along common cables, but all control decisions still take place within the DCS.  Coupling devices serve as junctions in the digital cabling, with trunking saving lots of wire compared to analog (4-20 mA) signaling.  In a bus system such as this, all the analog/digital scaling takes place in the field instruments themselves rather than the DCS, and thus the AI and AO software function blocks reside in those instruments.  PID and other control-type function blocks reside in the DCS.

\vskip 10pt

FOUNDATION Fieldbus also makes use of trunked digital signal cabling, but the big difference is that all control decisions may be made in the field instruments themselves rather than the DCS!  This means the DCS need only provide supervisory and maintenance functions rather than control.  All software function blocks related to scaling and control reside in the field instruments (although the DCS may be programmed to perform control functions as well).

\vskip 10pt

Two different network types for FOUNDATION Fieldbus: H1 and HSE (formerly known as H2).  H1 was intended for field instruments, while HSE is intended as a ``backbone'' connection between DCS nodes.








\vskip 20pt \vbox{\hrule \hbox{\strut \vrule{} {\bf Suggestions for Socratic discussion} \vrule} \hrule}

\begin{itemize}
\item{} Define the word ``distributed'' as it is used in the acronym {\it DCS}.  How does this word find meaning in an FF system?
\item{} Identify the effects of a transmitter cable failing open in a DCS with analog signaling.
\item{} Identify the effects of a trunk cable failing open in a DCS with Profibus signaling.
\item{} Identify the effects of a trunk cable failing open in a DCS with FF signaling.
\item{} Why might a designer choose Profibus PA over 4-20 mA analog signaling?
\item{} Why might a designer choose Profibus PA over FF signaling?
\item{} An important function of an ``AI'' function block is signal {\it scaling}.  Explain what ``scaling'' is and why it is important.
\end{itemize}









\vfil \eject

\noindent
{\bf Prep Quiz:}

A capability {\it unique} to FOUNDATION Fieldbus instruments is:

\begin{itemize}
\item{} Multiple field instruments may share a common signal cable
\vskip 5pt 
\item{} Faster data communication rate than exhibited by HART
\vskip 5pt 
\item{} Remote configuration of instruments from the host system
\vskip 5pt 
\item{} Control decisions may be made within the field instruments
\vskip 5pt 
\item{} Transmission of multiple variables over one signal cable
\vskip 5pt 
\item{} Transmitters contain self-diagnostic ability to detect faults
\end{itemize}

%INDEX% Reading assignment: Lessons In Industrial Instrumentation, FOUNDATION Fieldbus (FF design philosophy)

%(END_NOTES)

