
%(BEGIN_QUESTION)
% Copyright 2010, Tony R. Kuphaldt, released under the Creative Commons Attribution License (v 1.0)
% This means you may do almost anything with this work of mine, so long as you give me proper credit

Read pages 31 through 34 of the ``Wiring Guidelines for FOUNDATION Fieldbus Transmitters'' document published by Foxboro (MI 020-360, July 1999), and answer the following questions:

\vskip 10pt

How many different control loops exist in this process, and how many wire pairs were originally necessary to convey all the 4-20 mA analog signals to and from the DCS?

\vskip 10pt

How many wire pairs are necessary to do the same control tasks using Fieldbus instruments, assuming a single H1 segment?

\vskip 10pt

Explain why the manager's original plan of using original wiring to create a single FF network was not practical, and then describe some alternative solutions.

\vskip 20pt \vbox{\hrule \hbox{\strut \vrule{} {\bf Suggestions for Socratic discussion} \vrule} \hrule}

\begin{itemize}
\item{} Would you consider the proposed Fieldbus wiring topology in the blend chest process to be daisy-chain, bus/spur, or chicken foot?
\item{} What do you suppose the term ``home run'' means with reference to the multi-pair cable in this tutorial?
\item{} In the proposal using five wire pairs to connect the FF devices to the DCS, how much voltage would be dropped along the length of each wire pair?  Assuming a 24 VDC power source at the DCS, how much voltage would be left at the terminals of each FF device?
\item{} In the proposal using five separate H1 networks, how many FF interface cards would be required at the DCS?  Assuming eight channels per analog I/O card, how many analog cards would have been needed with the old 4-20 mA analog system?
\end{itemize}

\underbar{file i04558}
%(END_QUESTION)





%(BEGIN_ANSWER)

Thirteen wire pairs, servicing six control loops

\vskip 10pt

Just a single wire pair would (ideally) suffice for these six loops

\vskip 10pt

A single H1 segment is not feasible in this case since the ``home run'' wire gauge is too small and cannot carry the total Fieldbus device current without suffering excessive voltage drop along its length.

\vskip 10pt

Installation of a repeater at the field (junction box) end would reduce ``home run'' cable current and voltage drop to minimal levels.  Alternatively, multiple wire pairs in the ``home run'' cable could be used to create multiple H1 networks, joining together to the DCS through ``bridge'' devices at the other end.

%(END_ANSWER)





%(BEGIN_NOTES)

\filbreak \vskip 20pt \vbox{\hrule \hbox{\strut \vrule{} {\bf Virtual Troubleshooting} \vrule} \hrule}

\noindent
{\bf Predicting the effect of a given fault:} present each of the following faults to the students, one at a time, having them comment on all the effects each fault would produce.

\begin{itemize}
\item{} Blend chest level transmitter (LT) failing with a low signal
\item{} TiO2 flow transmitter (FT) failing with a high signal
\item{} Broke flow transmitter (FT) failing with a low signal
\end{itemize}


\vskip 10pt


\noindent
{\bf Identifying possible/impossible faults:} present symptoms to the students and then have them determine whether or not a series of suggested faults could account for all the symptoms, explaining {\it why} or {\it why not} for each proposed fault:

\begin{itemize}
\item{} Symptom: {\it }
\item{}  -- {\bf Yes/No}
\item{}  -- {\bf Yes/No}
\item{}  -- {\bf Yes/No}
\end{itemize}


\vskip 10pt


\noindent
{\bf Determining the utility of given diagnostic tests:} present symptoms to the students and then propose the following diagnostic tests one by one.  Students rate the value of each test, determining whether or not it would give useful information (i.e. tell us something we don't already know).  Students determine what different results for each test would indicate about the fault, if anything:

\begin{itemize}
\item{} Symptom: {\it }
\item{}  -- {\bf Yes/No}
\item{}  -- {\bf Yes/No}
\end{itemize}


\vskip 10pt


\noindent
{\bf Diagnosing a fault based on given symptoms:} imagine the ??? fails ??? in this system (don't reveal the fault to students!).  Present the operator's observation(s) to the students, have them consider possible faults and diagnostic strategies, and then tell them the results of tests they propose based on the following symptoms, until they have properly identified the nature and location of the fault:

\begin{itemize}
\item{} Operator observation: {\it }
\item{} 
\item{} 
\end{itemize}







\vfil \eject

\noindent
{\bf Summary Quiz:}

In the Foxboro ``Wiring Guidelines for FOUNDATION Fieldbus Transmitters'' document, what was impractical with the first retrofit proposal where all thirteen Fieldbus instruments would be serviced by one of the pairs in the multi-pair ``home run'' cable?

\begin{itemize}
\item{} It is impossible to operate more than ten Fieldbus devices on one H1 segment
\vskip 5pt 
\item{} This would have exceeded the allowable cable length for a Fieldbus H1 segment
\vskip 5pt 
\item{} The wiring was in poor condition and could not be used again for any purpose
\vskip 5pt 
\item{} There would have been no suitable place to locate the termination resistor
\vskip 5pt 
\item{} The wire pairs were not individually shielded, and therefore would incur noise
\vskip 5pt 
\item{} The total current would have dropped too much voltage along the length of wire
\end{itemize}


%INDEX% Reading assignment: Foxboro Fieldbus wiring guidelines (paper mill retrofit case)

%(END_NOTES)


