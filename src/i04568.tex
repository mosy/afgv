
%(BEGIN_QUESTION)
% Copyright 2010, Tony R. Kuphaldt, released under the Creative Commons Attribution License (v 1.0)
% This means you may do almost anything with this work of mine, so long as you give me proper credit

Read and outline the ``H1 FF Device Configuration and Commissioning'' section of the ``FOUNDATION Fieldbus Instrumentation'' chapter in your {\it Lessons In Industrial Instrumentation} textbook.  Note the page numbers where important illustrations, photographs, equations, tables, and other relevant details are found.  Prepare to thoughtfully discuss with your instructor and classmates the concepts and examples explored in this reading.

\underbar{file i04568}
%(END_QUESTION)





%(BEGIN_ANSWER)


%(END_ANSWER)





%(BEGIN_NOTES)

Fieldbus instruments require digital ``driver'' files to be installed in a host system to tell that host ``what to do with'' the instrument once it's been commissioned.  The data is stored in the following files:

\begin{itemize}
\item{} Device Description (DD) file, ending in extension {\tt .sym} (ASCII text format)
\item{} Device Description (DD) file, ending in extension {\tt .ffo} (binary format)
\item{} Capability file, ending in extension {\tt .cff} (ASCII text format)
\item{} Value file, ending in extension {\tt .cff} (ASCII text format)
\end{itemize}

DD and Capability files are available from the fieldbus.org website and also from instrument manufacturers as free downloads.  Value files are generated by the host system upon commissioning, storing user parameters such as tag name unique to that installation.  All these files follow programming conventions specified by the DDL (Device Description Language).

DD and Capability files stored on the host system must be at least as new as the instruments to be used.  This is analogous to having new ``driver'' software installed on a personal computer in order for that computer to successfully work with a peripheral device (e.g. printer).

\vskip 10pt

Commissioning:

\begin{itemize}
\item{} Connect FF instrument to H1 network segment
\item{} Appears under ``decommissioned'' devices if new
\item{} Create a ``placeholder'' tag with appropriate ISA tagname and characteristics for the device to be commissioned.  H1 segment address assigned at this time, as well as backup LAS assignment.
\item{} Choose to ``commission'' the new tag, selecting the desired instrument from a list of decommissioned devices.
\item{} Choose whether or not to ``reconcile'' any differences between the placeholder and the actual device, depending on whether you wish to keep the parameters in the placeholder, or keep the parameters currently in the device.
\item{} Download new changes to field nodes of the host system
\item{} Available function blocks may be seen nested within the device icon
\end{itemize}

\vskip 10pt

Calibration and ranging synonymous for analog instruments, separate functions for digital (smart and FF) instruments.  Calibration ensures accuracy compared to a known standard.  Ranging defines the 0\% and 100\% end-points.  The analogy of an digital alarm clock is helpful here: setting the clock time is {\it calibration}, while setting the wake-up time is {\it ranging}.

Calibration (input trim) parameters contained in a FF instrument's transducer (XD) function block.  Place instrument into OOS mode, apply standard calibration values, and set the {\tt Cal\_Point\_Lo} and {\tt Cal\_Point\_Hi} parameters accordingly.

Range parameters contained in a FF instrument's analog input (AI) or analog output (AO) block.  The AI block contains these important parameters:

\begin{itemize}
\item{} {\tt Channel} = defines which transducer channel to read (the instrument may have multiple channels for multiple variables)
\item{} {\tt L\_type} = linearization type (direct, indirect, or indirect square root)
\item{} {\tt XD\_scale} = range of sensor measurement
\item{} {\tt OUT\_scale} = range of published measurement
\end{itemize}

Set L\_type for Direct when you wish the transmitter to directly publish the raw transducer measurement.  Set L\_type to Indirect when you wish to scale the measurement into another range and/or another unit.  Ethanol storage tank example: transducer measurement of PSI scaled into published measurement of feet (liquid height).  Liquid flow example: transducer measurement of inches WC scaled into published measurement of gallons per minute (flow rate).











\filbreak

\vskip 20pt \vbox{\hrule \hbox{\strut \vrule{} {\bf Suggestions for Socratic discussion} \vrule} \hrule}

\begin{itemize}
\item{} What does it mean to say that a computer file is ``ASCII-encoded''?
\item{} Describe the difference between the {\it Capability} and {\it Value} files for an instrument in a FF system.  Which one comes from the manufacturer, and which one is generated on-site by the user during commissioning?  What kinds of data would you expect to find within each of these ASCII files?
\item{} Briefly describe the procedure you must follow to commission a FF instrument on an Emerson DeltaV DCS.
\item{} During the commissioning procedure for a new FF device on the Emerson DeltaV DCS, the technician is asked whether or not to {\it reconcile} the parameter differences between the device and its placeholder in the DeltaV heirarchy.  Explain what this means, and why you should (or should not) choose the ``reconcile'' option.
\item{} What do blue triangle symbols represent in the DeltaV Explorer heirarchy?
\item{} Explain the difference between {\it calibration} and {\it ranging} for a FF instrument.
\item{} Explain how ranging a FF instrument is fundamentally different from ranging either an analog instrument or a ``smart'' (i.e. HART) instrument with 4-20 mA output.
\item{} Many people confuse ``Direct'' and ``Indirect'' {\tt L\_Type} settings for ``Direct'' and ``Reverse'' instrument {\it action}.  This is not the case, but can you think of a way to make a FF instrument reverse-acting by clever configuration of the {\tt XD\_Scale} and/or {\tt OUT\_Scale} ranges?
\end{itemize}












\vfil \eject

\noindent
{\bf Prep Quiz:}

Suppose a FOUNDATION Fieldbus level transmitter senses the height of liquid inside of a tank (where an empty tank registers as 0 feet and a full tank registers as 12.5 feet) and scales this sensed level into a {\it volume} measurement representing the amount of liquid stored inside the vessel (empty = 0 gallons and full = 3000 gallons).  If the vessel gets replaced by one having the same height but a larger diameter, such that the same amount of liquid height now represents a greater volume of liquid stored inside, what must be done to the transmitter?

\begin{itemize}
\item{} The transmitter must be re-calibrated (re-trimmed) and re-ranged.
\vskip 5pt 
\item{} The transmitter need only be re-calibrated (re-trimmed).
\vskip 5pt 
\item{} The transmitter need only be re-ranged.
\vskip 5pt 
\item{} The transmitter must be completely replaced.
\vskip 5pt 
\item{} Nothing about the transmitter needs to be changed.
\end{itemize}

%INDEX% Reading assignment: Lessons In Industrial Instrumentation, FOUNDATION Fieldbus (FF device configuration and commissioning)

%(END_NOTES)

