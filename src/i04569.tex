
%(BEGIN_QUESTION)
% Copyright 2010, Tony R. Kuphaldt, released under the Creative Commons Attribution License (v 1.0)
% This means you may do almost anything with this work of mine, so long as you give me proper credit

Read and outline the ``H1 FF Segment Troubleshooting'' section of the ``FOUNDATION Fieldbus Instrumentation'' chapter in your {\it Lessons In Industrial Instrumentation} textbook.  Note the page numbers where important illustrations, photographs, equations, tables, and other relevant details are found.  Prepare to thoughtfully discuss with your instructor and classmates the concepts and examples explored in this reading.

\underbar{file i04569}
%(END_QUESTION)





%(BEGIN_ANSWER)


%(END_ANSWER)





%(BEGIN_NOTES)

Always use coupling devices that have built-in short circuit protection!!  It is well worth the extra cost.

\vskip 10pt

H1 cable conductors should register having no continuity between them or with the shield.

\vskip 10pt

Normal FF H1 signal strength should be between 700 mV P-P and 350 mV P-P.  Excessive signal strength suggests missing terminator(s), while inadequate signal strength suggests too many.  Noise should be 50 mV P-P or less.

\vskip 10pt

When using an oscilloscope to check signals and noise on an H1 network, you must avoid grounding any of the conductors under test.  The scope should therefore be in {\it differential} mode (display sum of channels A and B, with channel B inverted).

\vskip 10pt

Excessive message re-transmissions are a high-level indication of Fieldbus H1 network trouble, though it is non-specific as to the cause.






\vskip 20pt \vbox{\hrule \hbox{\strut \vrule{} {\bf Suggestions for Socratic discussion} \vrule} \hrule}

\begin{itemize}
\item{} Explain why an ohmmeter might register an increasing amount of resistance between the two conductors of a FF segment cable.
\item{} Referencing a schematic diagram of a Fieldbus H1 segment, explain how you would implement each of the cable resistance tests described in the text.
\item{} An electrical test tool called an {\it insulation tester}, also referred to as a ``Megger'' in honor of one of the early manufacturers of this tool, is essentially a high-voltage ohmmeter used to check for the presence of unwanted connections between insulated conductors, and between those conductors and earth ground.  Explain how careless use of a Megger could actually {\it destroy} FF instrumentation and its associated DCS interface cards.
\item{} If one desired to use a ``Megger'' to test FF H1 cable, describe a safe testing procedure which would eliminate any potential of harming FF instrumentation or other network devices.
\item{} Devise a testing strategy to confirm the current-limiting feature of a {\it coupling device} equipped with short-circuit protection.  Assume the H1 segment has not been commissioned yet (i.e. no active FF devices, DCS not programmed, etc.).
\item{} Explain how you could use a digital multimeter (DMM) to perform rudimentary signal strength and noise tests on an active H1 network segment.
\item{} Explain how the ``differential'' input mode of an oscilloscope works, and why it is important to use this mode when measuring FF signals on an H1 segment.
\item{} What is a {\it message retransmission} in an H1 network, and how would you check this statistic in an operating FF system?
\end{itemize}








\vfil \eject

\noindent
{\bf Summary Quiz:}

Identify a peak-to-peak signal voltage value likely to be found in a FOUNDATION Fieldbus H1 network missing one of its terminator resistors:

\begin{itemize}
\item{} 0 millivolts
\vskip 5pt 
\item{} 130 millivolts
\vskip 5pt 
\item{} 220 millivolts
\vskip 5pt 
\item{} 306 millivolts
\vskip 5pt 
\item{} 530 millivolts
\vskip 5pt 
\item{} 870 millivolts
\end{itemize}

%INDEX% Reading assignment: Lessons In Industrial Instrumentation, FOUNDATION Fieldbus (H1 FF segment troubleshooting)

%(END_NOTES)

