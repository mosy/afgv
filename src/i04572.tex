
%(BEGIN_QUESTION)
% Copyright 2010, Tony R. Kuphaldt, released under the Creative Commons Attribution License (v 1.0)
% This means you may do almost anything with this work of mine, so long as you give me proper credit

Suppose you wished the LAS in an H1 Fieldbus segment to skip addresses 55 through 233, inclusive.  Determine the necessary FUN and NUN values to make this happen, and also determine whether or not there might be any unintended consequences of setting FUN and NUN values in a segment.

\vskip 20pt \vbox{\hrule \hbox{\strut \vrule{} {\bf Suggestions for Socratic discussion} \vrule} \hrule}

\begin{itemize}
\item{} Explain why FUN and NUN values even exist in Fieldbus systems.  What possible use or benefit are they to the operation of the system?
\item{} Show how it is possible to develop a solution method for this type of problem by {\it simplifying the original problem}.  Instead of skipping addresses 55 through 233, what address range could you imagine skipping which would make the FUN and NUN values so simple as to be obvious?  How then will this simpler example help you in calculating FUN and NUN values for {\it any} given address range?
\end{itemize}

\underbar{file i04572}
%(END_QUESTION)





%(BEGIN_ANSWER)

FUN = 55 ; NUN = 179

%(END_ANSWER)





%(BEGIN_NOTES)

The danger is that if any new device with an address between 55 and 233 is added to a network, it will never be discovered by the LAS because no Probe Node tokens will be issued to any address within that range!

%INDEX% Fieldbus, FOUNDATION (H1): Device addresses
%INDEX% Fieldbus, FOUNDATION (H1): FUN and NUN addresses

%(END_NOTES)


