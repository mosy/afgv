
%(BEGIN_QUESTION)
% Copyright 2010, Tony R. Kuphaldt, released under the Creative Commons Attribution License (v 1.0)
% This means you may do almost anything with this work of mine, so long as you give me proper credit

On the job, an instrument technician is in the process of troubleshooting a problem in a newly commissioned and operating H1 Fieldbus segment.  Deciding to check for the presence of termination resistors in the segment (in case one or both of them has been forgotten), he connects his ohmmeter across the ``Fieldbus'' terminals of the nearest transmitter connected on that segment and expects to measure 100 ohms.

\vskip 10pt

Identify what is wrong with this technician's diagnostic strategy.

\vskip 20pt \vbox{\hrule \hbox{\strut \vrule{} {\bf Suggestions for Socratic discussion} \vrule} \hrule}

\begin{itemize}
\item{} If an ohmmeter is not the correct tool to check terminator resistance, what is?
\item{} Suppose an H1 segment was lacking a terminator resistor at its far end.  Describe how this lack of a terminator could cause network communication problems in that segment, referencing CD and PT messages in your explanation.
\end{itemize}

\underbar{file i04573}
%(END_QUESTION)





%(BEGIN_ANSWER)

There is definitely more than one thing wrong here!

%(END_ANSWER)





%(BEGIN_NOTES)

\noindent
Errors in this technician's strategy:

\begin{itemize}
\item{} You cannot measure resistance with an ohmmeter on a live circuit!
\item{} As a proper H1 segment requires {\it two} termination resistors, the proper resistance value as seen from any point in the network with a DC tester should be 50 ohms, not 100 ohms (i.e. two 100 ohm resistances in parallel).
\item{} The termination resistors are capacitively coupled, so the ohmmeter would not read it anyway!!!
\end{itemize}

%INDEX% Fieldbus, FOUNDATION (H1): segment troubleshooting

%(END_NOTES)

