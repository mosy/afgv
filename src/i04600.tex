
%(BEGIN_QUESTION)
% Copyright 2011, Tony R. Kuphaldt, released under the Creative Commons Attribution License (v 1.0)
% This means you may do almost anything with this work of mine, so long as you give me proper credit

Calculate the RF link budget (the margin between minimum required transmitting power and actual transmitting power) for a FreeWave FGR2 industrial data radio (900 MHz) transmitting at 5 mW, assuming the following parameters:

\begin{itemize}
\item{} BER (bit error rate) = $10^{-4}$
\item{} Distance = 8.5 miles 
\item{} EAN0900YC Yagi transmitting antenna 
\item{} ASC0202SN transmitting cable (20 feet long) 
\item{} EAN0906YA Yagi receiving antenna 
\item{} ASC0302SN receiving cable (30 feet long) 
\end{itemize}

\vskip 20pt \vbox{\hrule \hbox{\strut \vrule{} {\bf Suggestions for Socratic discussion} \vrule} \hrule}

\begin{itemize}
\item{} Explain the purpose for having a {\it budget} or {\it margin} of difference between the minimum required transmitting power and the actual transmitting power.  Why should we need more than what is minimally calculated, especially when considering that excess transmitting power may cause problems for neighboring electronic systems?
\end{itemize}

\underbar{file i04600}
%(END_QUESTION)





%(BEGIN_ANSWER)

$P_{tx}$ (minimum) = $-11.696$ dBm
 
\vskip 10pt

With the transmitting power set to 5 milliwatts (6.99 dBm), the RF link budget -- the difference between this actual transmitting power and the bare-minimum required to get the job done -- is 18.69 dB.

%(END_ANSWER)





%(BEGIN_NOTES)

Consulting the datasheet for this radio unit, we see that its sensitivity (the minimum amount of RF signal power needed at its input connector in order to reliably interpret that signal) is $-110$ dBm at the specified bit error rate of $10^{-4}$.  This means that the transmitter must output enough power to deliver $-110$ dBm of power to the receiver, accounting for all losses in the path (path loss, antenna losses/gains, cable losses, etc.).

\vskip 10pt

First, we will calculate the path loss for a 900 kHz signal over 8.5 miles of empty space, starting with a calculation of wavelength ($\lambda$) and distance ($D$) in meters:

$$\lambda = {v \over f} = {2.9979 \times 10^8 \over 900 \times 10^6} = 0.3331 \hbox{ m}$$

$$\left(8.5 \hbox{ mi} \over 1 \right) \left(5280 \hbox{ ft} \over 1 \hbox{ mi} \right) \left(12 \hbox{ in} \over 1 \hbox{ ft} \right) \left(2.54 \hbox{ cm} \over 1 \hbox{ in} \right) \left(1 \hbox{ m} \over 100 \hbox{ cm} \right) = 13679.42 \hbox{ m}$$

$$L_{path} = -20 \log \left(4 \pi D \over \lambda \right)$$

$$L_{path} = -20 \log \left(4 \pi (13679.42) \over 0.3331 \right) = -114.254 \hbox{ dB}$$

\vskip 10pt

Path loss is not the only loss in this system, however.  We must also contend with power losses through each of the specified cables (one at the transmitter and one at the receiver).  From the FreeWave cable datasheet, we see that a model ASC0202SN transmitting cable (20 feet long) suffers a loss of $-1.7$ dB at 900 MHz, while a model ASC0302SN receiving cable (30 feet long) suffers a loss of $-2.5$ dB at 900 MHz.

\vskip 10pt

Also from FreeWave datasheets, we see that a model EAN0900YC Yagi transmitting antenna exhibits a gain of +12.15 dBi, while a model EAN0906YA Yagi receiving antenna exhibits a gain of +8 dBi.

\vskip 10pt

Tallying all these gains and losses:

% No blank lines allowed between lines of an \halign structure!
% I use comments (%) instead, so that TeX doesn't choke.

$$\vbox{\offinterlineskip
\halign{\strut
\vrule \quad\hfil # \ \hfil & 
\vrule \quad\hfil # \ \hfil \vrule \cr
\noalign{\hrule}
%
% First row
{\bf Gain or Loss} & {\bf Decibel value} \cr
%
\noalign{\hrule}
%
% Another row
Transmitter cable loss & $-1.7$ dB \cr
%
\noalign{\hrule}
%
% Another row
Transmitter antenna gain & +12.15 dBi \cr
%
\noalign{\hrule}
%
% Another row
Path loss & $-114.254$ dB \cr
%
\noalign{\hrule}
%
% Another row
Receiver antenna gain & +8 dBi \cr
%
\noalign{\hrule}
%
% Another row
Receiver cable loss & $-2.5$ dB \cr
%
\noalign{\hrule}
%
% Another row
$G_{total} + L_{total}$ & {\bf $-$98.304 dB} \cr
%
\noalign{\hrule}
} % End of \halign 
}$$ % End of \vbox

The link budget formula solves for minimum transmitter power given minimum receiver power and all gains/losses in the path:

$$P_{tx(min)} = P_{rx(min)} - (G_{total} + L_{total})$$

$$P_{tx(min)} = -110 \hbox{ dBm} - (-98.304 \hbox{ dB}) = -11.696 \hbox{ dBm}$$

This equates to a minimum transmitter power output of 0.0677 milliwatts, which is quite a bit less than the transmitter's minimum power output of 5 milliwatts (+6.9897 dBm), giving a margin of 18.686 dB for fade and other losses at minimum power.  At the transmitter's maximum power output of 1 watt (+30 dBm), the margin increases to 41.696 dB.

%INDEX% Electronics review, RF link budget calculation

%(END_NOTES)


