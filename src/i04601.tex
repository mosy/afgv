
%(BEGIN_QUESTION)
% Copyright 2011, Tony R. Kuphaldt, released under the Creative Commons Attribution License (v 1.0)
% This means you may do almost anything with this work of mine, so long as you give me proper credit

Calculate the maximum width of the (first) Fresnel zone between two {\sl Wireless}HART instruments separated by a distance of 50 feet.

\vskip 20pt \vbox{\hrule \hbox{\strut \vrule{} {\bf Suggestions for Socratic discussion} \vrule} \hrule}

\begin{itemize}
\item{} Why is the width of the first Fresnel zone important to us?
\item{} Is the fact that this is a {\sl Wireless}HART system relevant to the solution?  Explain why or why not.
\end{itemize}

\underbar{file i04601}
%(END_QUESTION)





%(BEGIN_ANSWER)

The Fresnel zone is 1.38 meters (4.527 feet) wide at its ``fattest'' point between the two {\sl Wireless}HART instruments.

%(END_ANSWER)





%(BEGIN_NOTES)

$$r = \sqrt{{n \lambda d_1 d_2} \over D}$$

\noindent
Where,

$r$ = Radius of Fresnel zone at the point of interest

$n$ = Fresnel zone number (an integer value, with 1 representing the first zone)

$d_1$ = Distance between one antenna and the point of interest

$d_2$ = Distance between the other antenna and the point of interest

$D$ = Distance between both antennas

\vskip 10pt

The width of the Fresnel zone, of course, will be twice the radius.

$$\lambda = {c \over f} = {2.9979 \times 10^8 \over 2.4 \times 10^9} = 0.1249 \hbox{ m}$$

$$D = 50 \hbox{ ft} = 15.24 \hbox{ m}$$

Since we are interested in the maximum width of the Fresnel zone, we know this will be at the mid-way point between the two antennas ($d_1 = d_2$ = 7.62 m).

$$r = \sqrt{{(1) (0.1249) (7.62) (7.62)} \over 15.24}$$

$$r = 0.68987 \hbox{ m}$$

The maximum width, therefore, will be 1.3797 meters, or 4.527 feet (twice the radius at this mid-point).

%INDEX% Electronics review, Fresnel zones for radio links
%INDEX% Networking, WirelessHART

%(END_NOTES)


