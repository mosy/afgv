
%(BEGIN_QUESTION)
% Copyright 2010, Tony R. Kuphaldt, released under the Creative Commons Attribution License (v 1.0)
% This means you may do almost anything with this work of mine, so long as you give me proper credit

Read and outline the ``Protective Measures'' subsection of the ``Classified Areas and Electrical Safety Measures'' section of the ``Process Safety and Instrumentation'' chapter in your {\it Lessons In Industrial Instrumentation} textbook.  Note the page numbers where important illustrations, photographs, equations, tables, and other relevant details are found.  Prepare to thoughtfully discuss with your instructor and classmates the concepts and examples explored in this reading.

\underbar{file i04640}
%(END_QUESTION)





%(BEGIN_ANSWER)


%(END_ANSWER)





%(BEGIN_NOTES)

Measures to discourage electrical devices from triggering fires and explosions in classified areas include (1) containing the explosion once it occurs inside an electrical enclosure, (2) purge the enclosure with a non-flammable gas, (3) encapsulate the device so no flammable mixtures can reach the circuitry, (4) limit total energy below the MIE.

\vskip 10pt

NEMA 7 and 8 electrical enclosures are built very strong to withstand internal explosions, with the gap between the door and enclosure frame acting as a safe gap (MESG) to cool exploding gases so that they won't trigger an explosion outside.

\vskip 10pt

Enclosures may be actively purged with instrument air or some other non-flammable atmosphere in order to expel any flammable mixtures from outside.

\vskip 10pt

Devices may be completely sealed off from the surrounding atmosphere in order to absolutely prevent explosive mixtures from reaching the energy.  Hermetically-sealed mercury switches are an example of this.

\vskip 10pt

Intrisincally safe circuits are those where even a circuit fault cannot deliver enough energy to the field to initiate combustion (NEC article 504) because that energy is limited.  By contrast, a ``nonincendive'' circuit (NEC article 500) is one incapable of intiating combustion under normal operating conditions (not including circuit faults).

IS barrier circuits use resistors and zener diodes to prevent ignition-capable levels of voltage or current from reaching field devices, even if those field devices fail.  The series resistor limits fault current, while the parallel zener diode limits fault voltage.  Without an IS barrier in place, a 4-20 mA circuit may very well be non-incendive, but it would not be intrinsically safe.

\vskip 10pt

For power-generating sensors (e.g. thermocouples), intrinsic safety is met only if the voltage and current requirements fall beneath the limits for a ``simple apparatus'' (1.5 volts or less, 100 mA or less, 25 mW or less).  Passive loads such as lamps may also be deemed simple apparatus if their power dissipation is no more than 1.3 watts.

\vskip 10pt

NEC article 504 also specifies wiring practices for intrinsically safe circuits, where the wiring must be separated from other (non-IS) wiring by at least 50 mm, and that conductors must be restrined from contact even if they fall out of their respective terminal blocks.











\filbreak

\vskip 20pt \vbox{\hrule \hbox{\strut \vrule{} {\bf Suggestions for Socratic discussion} \vrule} \hrule}

\begin{itemize}
\item{} Identify multiple means of guarding against the possibility of an explosion, used in industrial applications.
\item{} Explain how NEMA 7 or 8 enclosures related to measurements of MESG for an explosive mixture.
\item{} Explain the difference between {\it nonincendive} and {\it intrinsically safe} circuits as defined by the NEC (Article 504).  Note that ``nonincendive'' circuits are not approved for Division 1 locations, while ``intrinsically safe'' circuits are.
\item{} Explain what a ``simple apparatus'' is as defined by the NEC (Article 504).
\item{} Do you think an intrinsic safety barrier will prevent HART signal communication with a ``smart'' transmitter?  Why or why not?
\item{} If I have a 4-20 mA loop-powered pressure transmitter rated for intrinsically safe installations, does it need a safety barrier in the loop circuit or will it be intrinsically safe all on its own?  Explain why or why not.
\item{} If I am installing a 4-20 mA loop-powered pressure transmitter inside of a NEMA 7 or NEMA 8 enclosure, does it need a safety barrier in the loop circuit?  Explain why or why not.
\item{} How should wiring for intrinsically safe circuits be treated differently than for other circuits?  
\item{} Can IS circuit wiring be run in the same conduit, cable tray, or raceway as non-IS wiring?
\end{itemize}













\vfil \eject

\noindent
{\bf Summary Quiz:}

Identify the distinction between an {\it intrinsically safe} (IS) circuit versus a {\it nonincendive} circuit:

\begin{itemize}
\item{} Nonincendive circuits are approved for use in all Division 1 locations, whereas IS circuits are not
\vskip 5pt 
\item{} A nonincendive circuit cannot trigger an explosion even when faulted, whereas an IS circuit can
\vskip 5pt 
\item{} Nonincendive circuits use hermetically-sealed switch contacts and are therefore safe
\vskip 5pt 
\item{} IS circuits are approved for Class III locations, whereas nonincendive circuits are not
\vskip 5pt 
\item{} An IS circuit cannot trigger an explosion even when faulted, whereas a nonincendive circuit can
\vskip 5pt 
\item{} There is no difference -- ``nonincendive'' and ``IS'' mean the exact same thing
\end{itemize}



%INDEX% Reading assignment: Lessons In Industrial Instrumentation, process safety (classified areas, protective measures)

%(END_NOTES)

