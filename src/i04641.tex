
%(BEGIN_QUESTION)
% Copyright 2010, Tony R. Kuphaldt, released under the Creative Commons Attribution License (v 1.0)
% This means you may do almost anything with this work of mine, so long as you give me proper credit

Read and outline the ``Classified Area Taxonomy'' subsection of the ``Classified Areas and Electrical Safety Measures'' section of the ``Process Safety and Instrumentation'' chapter in your {\it Lessons In Industrial Instrumentation} textbook.  Note the page numbers where important illustrations, photographs, equations, tables, and other relevant details are found.  Prepare to thoughtfully discuss with your instructor and classmates the concepts and examples explored in this reading.

\underbar{file i04641}
%(END_QUESTION)





%(BEGIN_ANSWER)


%(END_ANSWER)





%(BEGIN_NOTES)

Article 500 of the National Electrical Code (NEC) specifies ``classes'' and ``divisions'' of explosive hazards in industry.  Articles 505 and 506 specify ``zones'' representing similar categories of hazards.

\vskip 10pt

Class = type of hazard (I = gas or vapor, II = dust, III = fiber).  Division = likelihood of hazard (1 = exists under normal operating conditions, 2 = exists infrequently or during abnormal conditions).

\vskip 10pt

Zones 0, 1, and 2 = gas or vapor.  Zone 0 is where hazard is normal, zone 1 is where hazard is infrequent but possible under normal operating conditions, and zone 2 is where hazard is unlikely and transient.  Zones 20, 21, and 22 have similar meaning, but for dust and fiber hazards (analogous to Class II and III lumped together).

\vskip 10pt

Areas near refueling stations may be Division 1 or 2 depending on distance from vapor source.

\vskip 10pt

NEC article 500 subdivides Class I and II hazards into {\it groups} according to ignition criteria including ``maximum experimental safe gap'' (MESG) and ``minimum ignition current ratio'' (MICR).  MESG is the maximum gap between two hollow hemispheres containing an explosive mixture that will contain the flame of an explosion.  MICR is the ratio of ignition current for the hazard in question compared to the ignition current for an ideal mix of methane and air.  A low value in either case means more hazard.

\vskip 10pt

Class I substances are grouped as such:

% No blank lines allowed between lines of an \halign structure!
% I use comments (%) instead, so that TeX doesn't choke.

$$\vbox{\offinterlineskip
\halign{\strut
\vrule \quad\hfil # \ \hfil & 
\vrule \quad\hfil # \ \hfil & 
\vrule \quad\hfil # \ \hfil & 
\vrule \quad\hfil # \ \hfil \vrule \cr
\noalign{\hrule}
%
% First row
{\bf Group} & {\bf Typical substance} & {\bf Safe gap} & {\bf Ignition current} \cr
%
\noalign{\hrule}
%
% Another row
A & Acetylene &  &  \cr
%
\noalign{\hrule}
%
% Another row
B & Hydrogen & MESG $\leq$ 0.45 mm & MICR $\leq$ 0.40 \cr
%
\noalign{\hrule}
%
% Another row
C & Ethylene & 0.45 mm $<$ MESG $\leq$ 0.75 mm & 0.40 $<$ MICR $\leq$ 0.80 \cr
%
\noalign{\hrule}
%
% Another row
D & Propane & 0.75 mm $<$ MESG & 0.80 $<$ MICR \cr
%
\noalign{\hrule}
} % End of \halign 
}$$ % End of \vbox

\vskip 10pt

Class II substances are grouped according to material type:

% No blank lines allowed between lines of an \halign structure!
% I use comments (%) instead, so that TeX doesn't choke.

$$\vbox{\offinterlineskip
\halign{\strut
\vrule \quad\hfil # \ \hfil & 
\vrule \quad\hfil # \ \hfil \vrule \cr
\noalign{\hrule}
%
% First row
{\bf Group} & {\bf Substances} \cr
%
\noalign{\hrule}
%
% Another row
E & Metal dusts \cr
%
\noalign{\hrule}
%
% Another row
F & Carbon-based dusts \cr
%
\noalign{\hrule}
%
% Another row
G & Other dusts (wood, grain, flour, plastic, etc.) \cr
%
\noalign{\hrule}
} % End of \halign 
}$$ % End of \vbox

\vskip 10pt

Zone 0, 1, and 2 hazards have their own unique groupings:

% No blank lines allowed between lines of an \halign structure!
% I use comments (%) instead, so that TeX doesn't choke.

$$\vbox{\offinterlineskip
\halign{\strut
\vrule \quad\hfil # \ \hfil & 
\vrule \quad\hfil # \ \hfil & 
\vrule \quad\hfil # \ \hfil & 
\vrule \quad\hfil # \ \hfil \vrule \cr
\noalign{\hrule}
%
% First row
{\bf Group} & {\bf Typical substance(s)} & {\bf Safe gap} & {\bf Ignition current} \cr
%
\noalign{\hrule}
%
% Another row
IIC & Acetylene, Hydrogen & MESG $\leq$ 0.50 mm & MICR $\leq$ 0.45 \cr
%
\noalign{\hrule}
%
% Another row
IIB & Ethylene & 0.50 mm $<$ MESG $\leq$ 0.90 mm & 0.45 $<$ MICR $\leq$ 0.80 \cr
%
\noalign{\hrule}
%
% Another row
IIA & Acetone, Propane & 0.90 mm $<$ MESG & 0.80 $<$ MICR \cr
%
\noalign{\hrule}
} % End of \halign 
}$$ % End of \vbox









\filbreak

\vskip 20pt \vbox{\hrule \hbox{\strut \vrule{} {\bf Suggestions for Socratic discussion} \vrule} \hrule}

\begin{itemize}
\item{} Explain what ``Class'' and ``Division'' ratings mean in the NEC.
\item{} Explain what ``Zone'' ratings mean in the NEC.
\item{} Explain what ``Group'' ratings mean for Class I and II areas in the NEC.
\item{} Zone 0 = Class ?, Div ?
\item{} Zone 1 = Class ?, Div ?
\item{} Zone 2 = Class ?, Div ?
\item{} Zone 20 = Class ?, Div ?
\item{} Zone 21 = Class ?, Div ?
\item{} Zone 22 = Class ?, Div ?
\item{} Class I, Div 1 = Zone ?
\item{} Class I, Div 2 = Zone ?
\item{} Class II, Div 1 = Zone ?
\item{} Class II, Div 2 = Zone ?
\item{} Class III, Div 1 = Zone ?
\item{} Class III, Div 2 = Zone ?
\item{} Explain how an MESG rating is determined for any particular explosive mixture.
\item{} Explain how an MICR rating is determined for any particular explosive mixture.
\item{} Explain why gasoline pumps are rated Division 1 at certain distances and Division 2 at others.
\item{} Identify any areas in a person's home or workshop which could legitimately be considered classified areas.
\item{} For those who have studied chemistry, identify where {\it activation energy} plays a role in classified area ratings.  Specifically, which rating directly corresponds to the amount of activation energy required to initiate combustion of a flammable mixture?
\end{itemize}

%INDEX% Reading assignment: Lessons In Industrial Instrumentation, process safety (classified areas, taxonomy)

%(END_NOTES)

