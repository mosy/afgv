
%(BEGIN_QUESTION)
% Copyright 2010, Tony R. Kuphaldt, released under the Creative Commons Attribution License (v 1.0)
% This means you may do almost anything with this work of mine, so long as you give me proper credit

Read and outline the ``Mathematical Probability'' subsection of the ``Concepts of Probability'' section of the ``Process Safety and Instrumentation'' chapter in your {\it Lessons In Industrial Instrumentation} textbook.  Note the page numbers where important illustrations, photographs, equations, tables, and other relevant details are found.  Prepare to thoughtfully discuss with your instructor and classmates the concepts and examples explored in this reading.

\underbar{file i04643}
%(END_QUESTION)





%(BEGIN_ANSWER)


%(END_ANSWER)





%(BEGIN_NOTES)

``Probability'' is defined as the ratio of specific outcomes to total possible outcomes.  An example is coin-flipping: the probability of the coin landing ``tails'' is 0.5 because tails is one specific possibility out of two total possibilities (heads + tails).  The probability of a 6-sided die landing on any one particular side is $1 \over 6$, once again because we are looking for 1 specific outcome out of 6 total possible outcomes.

Probability is always a value between 0 and 1, inclusive.  0 is impossible, while 1 is guaranteed.

\vskip 10pt

Calculations of probability have meaning only when significant samples are considered.  Probability is almost useless when applied to singular cases.  Prediction of weather events (e.g. rain) is a good example: a prediction of 65\% chance of rain doesn't mean much when applied to any one particular day, but it is very accurate when applied to hundreds of having the same forecast.

\vskip 10pt

Probability is applied to instrumented safety systems in order to calculate reliability over long periods of time, not to try to predict individual component failures.  Probability laws are applied in such a way that we may optimize the design of safety systems for maximum reliability.










\vskip 20pt \vbox{\hrule \hbox{\strut \vrule{} {\bf Suggestions for Socratic discussion} \vrule} \hrule}

\begin{itemize}
\item{} Define {\it probability} in the mathematical sense.
\item{} What does a probability value of 0 represent?
\item{} What does a probability value of 1 represent?
\item{} Explain why probability values are more meaningful for large samples than for small samples.
\item{} Suppose you meet a new friend for the first time.  Calculate the probability that the day of your meeting happens to be that person's birthday.  ({\it Answer: $1 \over 365$})
\item{} Suppose you meet a new friend for the first time, on February 29.  Calculate the probability that February 29 happens to be that person's birthday.  ({\it Answer: $1 \over {4 \times 365}$})
\item{} If a robot randomly dials one 7-digit telephone numbers, calculate the probability that the number it dials will be yours.  ({\it Answer: $n \over 10^7$ = $n 10^{-7}$}, assuming you own $n$ telephones) 
\item{} Suppose you pull an analog clock out of a storage box, and that clock's spring (or battery) has long gone dead.  Calculate the probability it will say the correct time (within $\pm$ 5 minutes) when you pull it out.  ({\it Answer: ${1 \over {6 \times 12}}$}, since a margin of $\pm$ 5 minutes means each hour is divided up into 6 discrete possibilities of 10 minutes' width each and there are 12 hours represented on the face of a typical analog clock.  The probability is one-half this value if you never sleep and may find this clock at any time in a 24-hour day!)
\end{itemize}










\vfil \eject

\noindent
{\bf Prep Quiz:}

The weather is recorded in a city over exactly three years (with no leap years in that time).  In all those days, the number of days when rain fell was recorded at 102.  Calculate the annual probability of rainy days in this city based on this information:

\begin{itemize}
\item{} $P$ = 0.279
\vskip 5pt 
\item{} $P$ = 0.140
\vskip 5pt 
\item{} $P$ = 0.093
\vskip 5pt 
\item{} $P$ = 3.578
\vskip 5pt 
\item{} $P$ = 1.02
\vskip 5pt 
\item{} $P$ = 0.721
\end{itemize}



\vfil \eject

\noindent
{\bf Prep Quiz:}

The weather is recorded in a city over exactly three years (with no leap years in that time).  In all those days, the number of days when rain fell was recorded at 83.  Calculate the annual probability of rainy days in this city based on this information:

\begin{itemize}
\item{} $P$ = 0.227
\vskip 5pt 
\item{} $P$ = 0.114
\vskip 5pt 
\item{} $P$ = 0.000
\vskip 5pt 
\item{} $P$ = 0.076
\vskip 5pt 
\item{} $P$ = 1.000
\vskip 5pt 
\item{} $P$ = 0.115
\end{itemize}



%INDEX% Reading assignment: Lessons In Industrial Instrumentation, process safety (mathematical probability)

%(END_NOTES)

