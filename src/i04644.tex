
%(BEGIN_QUESTION)
% Copyright 2010, Tony R. Kuphaldt, released under the Creative Commons Attribution License (v 1.0)
% This means you may do almost anything with this work of mine, so long as you give me proper credit

Read and outline the ``Laws of Probability'' subsection of the ``Concepts of Probability'' section of the ``Process Safety and Instrumentation'' chapter in your {\it Lessons In Industrial Instrumentation} textbook.  Note the page numbers where important illustrations, photographs, equations, tables, and other relevant details are found.  Prepare to thoughtfully discuss with your instructor and classmates the concepts and examples explored in this reading.

\underbar{file i04644}
%(END_QUESTION)





%(BEGIN_ANSWER)


%(END_ANSWER)





%(BEGIN_NOTES)

Like Boolean algebra, probability values range between 0 and 1.  Unlike Boolean values, however, probability values may range {\it between} 0 and 1.  However, we may still use Boolean concepts of {\tt AND}, {\tt OR}, and {\tt NOT} to work with probability values.

\vskip 10pt

The probability of an event {\it not} happening is equal to the complement of that event's probability.  For example, if the probability of a 6-sided die landing a ``four'' is $1 \over 6$, then the probability of that same die not landing on ``four'' is $5 \over 6$.  ``Probability of Failure on Demand'' (PFD) and ``Dependability'' are such complentary values.

\vskip 10pt

The probability of coincidental events (e.g. event A {\it and} event B) is the product of those events' individual probabilities.  For example, the probability of rolling a ``four'' and then a ``one'' on two successive throws of a 6-sided die is ${1 \over 6} \times {1 \over 6} = {1 \over 36}$.  This rule assumes the two events are completely independent of each other, and therefore are not related or contingent.

Lusser's Product Law of Reliabilities states that the reliability of any system comprised of several crucial components will be the product of those crucial components' individual reliabilities, and therefore will be less than the reliability of any one component. 

\vskip 10pt

The probability of redundant events (e.g. event A {\it or} event B) may be calculated by applying DeMorgan's Theorem to the above ``{\tt AND}'' rule.  Thus, the probability of event A {\it or} event B happening is equal to the improbability of both A {\it and} B NOT happening (i.e. the complement of the {\tt AND} of the complements).  The probability of rolling a ``four'' on a 6-sided die over two successive rolls is equal to ${1 \over 6} + {1 \over 6} - {1 \over 6} \times {1 \over 6} = {11 \over 36}$.

\vskip 10pt

The probability of exclusive events (e.g. event A {\it or} event B, but not both!) is equal to the simple sum of the individual probabilities.  For example, the probability of rolling a ``three'' or a ``four'' on one roll of a 6-sided die is ${1 \over 6} + {1 \over 6} = {2 \over 6}$.

\vskip 10pt

$$P(A) = 1 - \overline{P}(A) \hskip 30pt \hbox{Complementary events ({\tt NOT})}$$

$$P(\hbox{A {\it and} B}) = P(\hbox{A}) \times P(\hbox{B}) \hskip 30pt \hbox{Coincidental events ({\tt AND})}$$

$$P(\hbox{A {\it or} B}) = P(B) + P(A) - P(A) \times P(B) \hskip 30pt \hbox{Redundant events (inclusive-{\tt OR})}$$

$$P(\hbox{A {\it exclusively or} B {\it exclusively}}) = P(A) + P(B) \hskip 30pt \hbox{Exclusive events (exclusive-{\tt OR})}$$












\filbreak

\vskip 20pt \vbox{\hrule \hbox{\strut \vrule{} {\bf Suggestions for Socratic discussion} \vrule} \hrule}

\begin{itemize}
\item{} How are probability and Boolean values similar, and how are they different?
\item{} If a device has a PFD value of 0.003, calculate its dependability.
\item{} If a device has a PFD value of 0.002, calculate its dependability.
\item{} If a device has a PFD value of 0.001, calculate its dependability.
\item{} If a device has a dependability of 0.999, calculate its PFD.
\item{} If a device has a dependability of 0.998, calculate its PFD.
\item{} If a device has a dependability of 0.997, calculate its PFD.
\item{} Cite an example from everyday life to illustrate the Boolean {\tt NOT} probability function.
\item{} Cite an example from everyday life to illustrate the Boolean {\tt AND} probability function.
\item{} Cite an example from everyday life to illustrate the Boolean {\tt OR} probability function.
\item{} Cite an example from everyday life to illustrate the Boolean {\tt XOR} probability function.
\item{} Explain what a {\it double-block} valve assembly is used for.
\item{} Explain why it is important that the events be {\it coincidental} when calculating probability using the {\tt AND} function.
\item{} Give examples of {\it Lusser's Law} from everyday life (e.g. the probability of making it to school on time).
\item{} Explain how we could calculate the probability of three coincidental events (i.e. a three-input {\tt AND} probability function).
\item{} Explain how we could calculate the probability of three redundant events (i.e. a three-input Inclusive-{\tt OR} probability function).
\item{} Explain why it is only with {\it exclusive} events that the total probability is equal to the direct sum of the events' probabilities.
\end{itemize}













\vfil \eject

\noindent
{\bf Prep Quiz:}

Suppose the probability of a particular control valve failing over the course of one year's service is 0.000429.  Calculate this control valve's {\it reliability} (i.e. the probability of it NOT failing) over that same period of time.

\begin{itemize}
\item{} -0.00429
\item{} 0
\item{} 0.00429
\item{} 0.99144
\item{} 0.99429
\item{} 0.99571
\item{} 1
\item{} 1.00429
\end{itemize}




\vfil \eject

\noindent
{\bf Summary Quiz:}

Suppose a control loop contains a transmitter with a reliability ($R$) of 0.9992, a controller with a reliability of 0.9985, and a control valve with a reliability of 0.9910.  Ignoring the reliability values of all other components in this loop (wires, connections, power supplies, etc.), use Lusser's Law to calculate the reliability of the whole control loop:

\begin{itemize}
\item{} 0.9992
\vskip 5pt 
\item{} 0.991
\vskip 5pt 
\item{} 2.9887
\vskip 5pt 
\item{} 0.9962 
\vskip 5pt 
\item{} 0.9887
\vskip 5pt 
\item{} 0.9985 
\end{itemize}

%INDEX% Reading assignment: Lessons In Industrial Instrumentation, process safety (laws of probability)

%(END_NOTES)

