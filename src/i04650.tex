
%(BEGIN_QUESTION)
% Copyright 2010, Tony R. Kuphaldt, released under the Creative Commons Attribution License (v 1.0)
% This means you may do almost anything with this work of mine, so long as you give me proper credit

Read and outline the ``Safety Integrity Levels'' subsection of the ``Safety Instrumented Functions and Systems'' section of the ``Process Safety and Instrumentation'' chapter in your {\it Lessons In Industrial Instrumentation} textbook.  Note the page numbers where important illustrations, photographs, equations, tables, and other relevant details are found.  Prepare to thoughtfully discuss with your instructor and classmates the concepts and examples explored in this reading.

\underbar{file i04650}
%(END_QUESTION)





%(BEGIN_ANSWER)


%(END_ANSWER)





%(BEGIN_NOTES)

\noindent
{\bf SIL} = a number referring to the Required Safety Availability (RSA) or Probability of Failure on Demand (PFD).  The SIL number happens to match the minimum number of 9's in the dependability of the function.  For example, a function that is 99.993\% reliable would have a SIL rating of 4 (four 9's in the probability figure).  RSA is equivalent to {\it dependability} and PFD is equivalent to {\it undependability}.

\vskip 10pt

SIL figures apply to Safety Instrumented Functions (SIF's), {\it not} to entire processes and {\it not} to individual components.  SIL values must be calculated from the known reliabilities of components comprising a particular safety function.








\vskip 20pt \vbox{\hrule \hbox{\strut \vrule{} {\bf Suggestions for Socratic discussion} \vrule} \hrule}

\begin{itemize}
\item{} What does a SIL rating represent about a safety system?
\item{} Identify some ways that the SIL rating of a safety function may be increased.
\item{} In a particular SIF, which do you suppose is the ``weakest link'' in the chain: the sensor, the logic solver, or the final control element?
\end{itemize}


%INDEX% Reading assignment: Lessons In Industrial Instrumentation, process safety (SIL)

%(END_NOTES)

