
%(BEGIN_QUESTION)
% Copyright 2010, Tony R. Kuphaldt, released under the Creative Commons Attribution License (v 1.0)
% This means you may do almost anything with this work of mine, so long as you give me proper credit

Read selected portions of the US Chemical Safety and Hazard Investigation Board's analysis of the 2008 sugar dust explosion and fire at the Imperial Sugar Company in Port Wentworth, Georgia (Report number 2008-05-I-GA), and answer the following questions:

\vskip 10pt

Explain how the {\it dust deflagration index} ($K_{st}$) is calculated for any combustible powder or dust, and what it means in terms of explosion hazard.

\vskip 10pt

All other factors being equal, which presents the greatest explosion hazard: a coarse powder or a fine dust?

\vskip 10pt

The incident at this sugar refinery was characterized by a {\it primary} explosion followed by {\it secondary} explosions.  Define the words ``primary'' and ``secondary'' in this context.

\vskip 10pt

Identify the likely ignition source that led to this catastrophe, according to the Chemical Safety Board investigators.

\vskip 10pt

Pages 15 and 29 describe how the decision to enclose the conveyors likely contributed to the level of danger at this facility.  Explain how something as simple as the installation of enclosures around a conveyor could cause a safety problem, and elaborate on the need for {\it Management of Change} procedures to re-assess process safety after making such a modification.

\vskip 20pt \vbox{\hrule \hbox{\strut \vrule{} {\bf Suggestions for Socratic discussion} \vrule} \hrule}

\begin{itemize}
\item{} Most people do not recognize sugar as being a flammable substance, much less {\it explosive}.  Explain how you could successfully reason the flammability of sugar, based on what you know of its other properties and uses.
\end{itemize}

\underbar{file i04654}
%(END_QUESTION)





%(BEGIN_ANSWER)


%(END_ANSWER)





%(BEGIN_NOTES)

$$K_{st} = { {\root 3 \of {V}} \left({dP \over dt}\right)_{max} }$$

\noindent
Where,

$K_{st}$ = Dust deflagration index

$V$ = Explosion chamber volume

$dP \over dt$ = Rate of pressure rise during test explosion

\vskip 10pt

Fine dusts present a greater explosion hazard than coarse powders.

\vskip 10pt

The {\it primary} explosion was the initial event, while {\it secondary} explosions were those fueled by sugar dust shaken loose by the primary explosion.

\vskip 10pt

An overheated bearing in a steel belt conveyor was the likely trigger of the primary explosion. (pp. 2, 65)

\vskip 10pt

The enclosures trapped sugar dust, raising the concentrations to more dangerous levels than what were present when the conveyors were uncovered.  Management of Change (MOC) procedures would have helped identify the hazard this modification posed, and perhaps avoided this catastrophe.

%INDEX% Reading assignment: USCSB Accident Report, sugar dust explosion and fire at Imperial Sugar Company in Port Wentworth, Georgia (2008)

%(END_NOTES)


