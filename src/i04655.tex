
%(BEGIN_QUESTION)
% Copyright 2010, Tony R. Kuphaldt, released under the Creative Commons Attribution License (v 1.0)
% This means you may do almost anything with this work of mine, so long as you give me proper credit

Read selected portions of the US Chemical Safety and Hazard Investigation Board's analysis of the 2004 sterilizer explosion at the Sterigenics facility in Ontario, California (Report number 2004-11-I-CA), and answer the following questions:

\vskip 10pt

Identify the chemical agent used to sterilize medical components inside the sterilization chambers, and why this particular chemical posed so great an explosion hazard.

\vskip 10pt

Describe the proper operating sequence of the sterilizer units, explaining how the potential for explosion is mitigated by sequenced controls.

\vskip 10pt

Describe the devices used to neutralize the sterilizing gas as it is purged from the chambers after each sterilization cycle.  Two separate systems are used!

\vskip 10pt

Explain why the explosion occurred that day.

\vskip 10pt

Page 33 of the report details a fundamental mis-understanding of the process held by both the maintenance technicians and the maintenance supervisor that led to a poor decision being made.  Explain this misconception.

\vskip 20pt \vbox{\hrule \hbox{\strut \vrule{} {\bf Suggestions for Socratic discussion} \vrule} \hrule}

\begin{itemize}
\item{} An important safety policy at many industrial facilities is something called {\it stop-work authority}, which means any employee has the right to stop work they question as unsafe.  Explain how stop-work authority could have been applied to this particular incident.
\end{itemize}

\underbar{file i04655}
%(END_QUESTION)





%(BEGIN_ANSWER)


%(END_ANSWER)





%(BEGIN_NOTES)

The sterilizing agent was ethylene oxide, and it has a UEL value of 100\% (it needs no air to combust!).  Also, the concentrations necessary for effective sterilization (400,000 ppm, or 40\% by volume) are well within the combustible range of 2.6\% to 100\%.

\vskip 10pt

See page 15 for a timing diagram of chamber pressures with annotations for sequence steps.  Sterilization chambers are operated at a partial vacuum, pre-purged with steam, partially filled with ethylene oxide and nitrogen, and then purged by a series of ``gas washes'' where nitrogen and air are used to purge the chamber.  As a final purge step, the door is partially opened to let air in while the remaining ethylene oxide vapors leave the chamber through a ``back vent'' headed for the catalytic oxidizer where they are burnt.  See page 17 of the report for a diagram.

\vskip 10pt

An acid-wash scrubber is used to neutralize the ethylene oxide gas for high concentrations (initial gas washes), converting ethylene oxide to ethylene glycol by way of reaction with sulfuric acid.  To neutralize the remaining ethylene oxide during the low-concentration ``back vent'' phase, a catalytic oxidizer unit is used to burn the ethylene oxide with air as the oxidizer.  Both units work to prevent large amounts of ethylene oxide from escaping into the environment outside the facility.

Gases headed for the catalytic oxidizer unit are diluted with air to a concentration level approximately one-fourth the LEL of ethylene oxide, thus preventing the high temperatures of the catalytic oxidizer from igniting the source stream.

\vskip 10pt

Maintenance personnel were testing the ethylene oxide injection system, and bypassed the safety purge gas wash cycles in an effort to return the unit to service more quickly.  The supervisor gave one of the technicians a password to bypass the gas wash cycles and proceed directly to ``back venting'' as is normally done at the end of the sterilization cycle.  As a result of this poor decision, high concentrations of ethylene oxide reached the catalytic oxidizer, igniting and causing an explosion inside the sterilization chamber.

\vskip 10pt

Maintenance personnel believed the purpose of the gas wash cycles was to purge sterilized {\it product} inside the chambers of residual ethylene oxide gas.  Their test was done with no product inside the chamber, and so they mistakenly thought there was no need to perform the gas washes.  They mistakenly thought that the single evacuation of the chamber following the well period was sufficient to exhaust enough ethylene oxide gas to make it safe to proceed to the ``back vent'' step.

The only personnel interviewed who knew the truth were management and senior engineering staff: the initial evacuation step removed only about 60\% of the ethylene oxide gas from the chamber, and that the subsequent gas washes were necessary to ensure a safe condition prior to ``back venting'' with or without product inside the chamber.

\vfil \eject

\noindent
{\bf Prep Quiz:}

The root cause of the sterilizer explosion at the Sterigenics facility was:

\begin{itemize}
\item{} Poor maintenance on the catalytic oxidizer unit, causing a spark
\vskip 5pt 
\item{} Maintenance personnel bypassing the automatic safety purge system
\vskip 5pt 
\item{} Operators carrying cell phones into the manufacturing areas
\vskip 5pt 
\item{} A non-explosion-proof light fixture used inside the sterilizer
\vskip 5pt 
\item{} Malfunctioning LEL monitors worn by maintenance personnel
\vskip 5pt 
\item{} A leaky seal on the main door letting air inside the chamber
\end{itemize}


%INDEX% Reading assignment: USCSB Accident Report, Sterigenics sterilizer explosion in Ontario, California (2004)

%(END_NOTES)


