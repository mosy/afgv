
%(BEGIN_QUESTION)
% Copyright 2016, Tony R. Kuphaldt, released under the Creative Commons Attribution License (v 1.0)
% This means you may do almost anything with this work of mine, so long as you give me proper credit

Read selected portions of the US Chemical Safety and Hazard Investigation Board's analysis of the 1998 catastrophic vessel overpressurization at the Sonat Exploration facility in Pitkin, Louisiana (Report number 1998-002-I-LA), and answer the following questions:

\vskip 10pt

Pages 1 through 3 of the report outline the event and the Chemical Safety Board's key findings.  Describe what happened to the facility in your own words, based on what is reported in these pages.

\vskip 10pt

William Shakespeare wrote in {\it Romeo and Juliet}, ``What's in a name?  That which we call a rose by any other name would smell as sweet.''  If the Bard were alive in 1998 in Pitkin, Louisiana, he might have written, ``What's in a name?  A vapor recovery tower by any other name would blow up just as quickly.''  Explain how the decision to name this vessel a ``vapor recovery tower'' instead of a ``separator'' actually contributed to the danger at this facility.

\vskip 10pt

Figure 9 on page 22 of the report presents a pair of P\&ID schematics showing the planned versus as-found ``lineups'' of valves for the third-stage separator vessel.  Examine these diagrams and then explain why the vessel experienced an over-pressure incident because of the valve lineup.

%\vskip 10pt

%Would the inclusion of a pressure relief valve on the separator (vapor recovery tower) have improved the SIL rating of the overpressure protection function, or is this figure calculated based on instrument reliabilities alone?

\vskip 10pt

Examine the ``Causal Tree Analysis'' diagram shown on the last page of this report, and explain how the logic symbols are helpful in explaining the probability of the accident occurring.  

\vskip 20pt \vbox{\hrule \hbox{\strut \vrule{} {\bf Suggestions for Socratic discussion} \vrule} \hrule}

\begin{itemize}
\item{} In the context of this report, what does the word ``train'' refer to?
\item{} What is the purpose of a {\it pressure-relief valve}?  Can you think of any examples of pressure-relief valves in everyday life, such as applications in homes or automobiles?
\item{} According to the footnote on page 3, describe the distinction between pressure expressed in PSIG versus in PSIA.
\item{} What symbolic convention do the diagrams use to distinguish closed valves from open valves?
\item{} Given the type of logic ``gate'' symbols used in the Causal Tree diagram, how easy would it have been to prevent the final outcome (fatalities)?  
\item{} For those who have studied mathematical probability, identify which portions of the Causal Tree diagram have values of 1 (certainty), versus fractional (less than 1) probability values.
\item{} How enforceable (in a legal sense) are the American Petroleum Institute's rules for such things as relief valves?
\end{itemize}

\underbar{file i04656}
%(END_QUESTION)





%(BEGIN_ANSWER)


%(END_ANSWER)





%(BEGIN_NOTES)

On the 4th of March, 1998 a separation vessel on an oil/gas processing system violently ruptured due to overpressurization.  Four workers were killed, and two survived without injury.  The event was a ``purging'' operation to push air out of the pipes in preparation for processing.  The purging proceeded for about an hour before the vessel ruptured. (pp. 1-2)

Natural gas separated from well oil in three separator vessels, each one operating at a successively lower pressure.  The ruptured vessel, a third-stage separator, was rated for atmospheric pressure service and lacked isolation valves.  No pressure-relief valves were on this vessel.  Valves which should have been left open to allow purged vapors to go to atmosphere were closed, causing this separator vessel to pressurize. (page 3)

This vessel rated for 0 PSIG; well fluid can reach 800 PSIG; vessel probably failed around 135 PSIG, 208 PSIG, or 375 to 400 PSIG depending on who you ask (pp. 22-23).  This vessel had been tested by the manufacturer to only 21 PSI.  No pressure-relief devices were installed on this vessel!  No engineering design reviews or hazard analyses took place during the design or construction of this process; no written operating procedures were in place for operators to follow (other than generic procedures such as confined space entry: pg. 28); the facility didn't even have a documented P\&ID of its process (pg. 27)!

\vskip 10pt

The American Petroleum Institute (API) has very specific guidelines for the installation of pressure relief devices on {\it separator} vessels.  Sonat called the exploded vessel a ``vapor recovery tower'' and claimed it was classified as a ``storage tank'' like the others within the same berm, rather than as a separator (pp. 25-26; also footnote \#4 on page 2), which rationalized them to not place pressure-relief devices on it like other separator vessels in accordance with their own Pressure Relief Valve Standard.  The first-stage and second-stage separator vessels were in fact equipped with pressure relief valves, but not the third-stage separator (vapor recovery tower).  Page 26 of the report lists the CSB's multiple rationale for classifying the VRT as a separator.

Interestingly the API also recommends pressure relief devices on storage tanks (API standard 2000 ``Venting Atmospheric and Low-Pressure Storage Tanks''), so Sonat's reasoning doesn't pass the sniff test (pg. 26, footnote \#33).

Also noteworthy is the fact that OSHA's Process Safety Management standard contains information relevant to this incident (29 CFR 1910.119), but the OSAH PSM does not currently apply to oil and gas production facilities.

\vskip 10pt

According to figure 9 on page 22, two block valves (one upstream, one downstream, labeled \#11 and \#13 respectively) near the separator's level control valve (\#12) were left shut instead of open as they should have been.  This prevented the fluid from bypassing the third-stage separator as planned, causing that vessel to over-pressure.

\vskip 10pt

The Causal Tree Analysis Diagram on page 38 shows a set of ``AND'' logic gate symbols linking causes to effects, the last effect being operator fatalities.  Since every logic gate symbol is an AND function, it stands to reason that these fatalities could have been prevented if just {\it one} of the causal factors were not in place.  This holds a strong lesson for students: it doesn't take much to prevent serious accidents such as this, and that even the smallest positive action can avert catastrophe.

%\vskip 10pt

%Pressure relief valves and even piping design most definitely play parts in calculating the SIL of a function.  SIL means more than just instrumentation!  SIL refers to the probability of success or failure for any protective {\it function}, regardless of how that function is implemented.  Safety relief valves, containment areas (for spills), and other physical measures all contribute to this probability.

%INDEX% Reading assignment: USCSB Accident Report, Sterigenics sterilizer explosion in Ontario, California (2004)

%(END_NOTES)


