
%(BEGIN_QUESTION)
% Copyright 2010, Tony R. Kuphaldt, released under the Creative Commons Attribution License (v 1.0)
% This means you may do almost anything with this work of mine, so long as you give me proper credit

Read selected portions of the US Chemical Safety and Hazard Investigation Board's case study of the 2009 oleum release in Petrolia, Pennsylvania (Report number 2009-01-I-PA), and answer the following questions:

\vskip 10pt

What type of control system did operators use to monitor and control the transfer of oleum, and what process information did this system provide to the operators about the oleum?

\vskip 10pt

The fundamental problem leading to this accident was a dual power supply for the oleum transfer pumps, one ``Normal'' and the other ``Emergency.''  Describe how this dual-source configuration came to be, and explain how it ended up being a major contributing factor to the accident.

\vskip 10pt

Section 4.4 on page 9 titled ``Normalizing the Hazard'' holds an important lesson on how ``temporary'' changes to a system may become permanent.  Summarize this short section in your own words, and comment on the lessons it holds for you as future instrument technicians.

\vskip 10pt

A very important lesson we may draw from this incident is something called {\it Management of Change} (MOC).  Section 5.1 on page 12 (``Change Management'') discusses this briefly.  Summarize this brief section in your own words and be prepared to discuss its importance in industrial settings.

\underbar{file i04684}
%(END_QUESTION)





%(BEGIN_ANSWER)


%(END_ANSWER)





%(BEGIN_NOTES)

A DCS was the control system at this facility, and it displayed an analog level measurement of the oleum level in all five vessels.  Furthermore, all five vessels were equipped with high-high level switches (capacitive -- pg. 8) wired to shut off power from the ``Normal'' power source.

\vskip 10pt

The ``Emergency'' power supply was installed to provide electrical power for the transfer pumps in the event the ``Normal'' supply was unavailable due to equipment failures in the oleum building.  This ``temporary'' fix remained for 28 years, however, and became a routine source of electrical power for operators when they ran the transfer pumps.  Unlike the ``Normal'' power source, the ``Emergency'' power source was not connected to the high-high level shutdown switches on the oleum vessels, and therefore would not automatically shut off in the event of an over-fill condition (pg. 5).  Furthermore, the ``Emergency'' power source was not controlled by the DCS as the ``Normal'' power source was (pg. 8).  In other words, the ``Emergency'' power source is always on, and therefore the only guard against overfilling by a pump powered from this source was human intervention!

As the cherry on top of this mess, the ``Emergency'' power supply was wholly undocumented.  Its use was a matter of oral tradition, with senior operators telling new operators how to use it (incorrectly).

\vskip 10pt

This hazard was ``normalized'' when operators discovered they could use the ``Emergency'' supply in addition to the ``Normal'' supply to speed their work.  Even when a third operator was added to that area (thus relieving workload), the practice of using both power supplies to transfer oleum faster did not cease.

The normalization of danger went one more step when they re-wired the emergency power supply to be hard-wired in conduit rather than by flexible plug-in cord, following (ironically) a safety audit!

\vskip 10pt

To quote the Case Study, ``{\it Evaluate all changes -- even those considered temporary -- as permanent, including hazard analysis, procedures, training, and drawings.  Establish and enforce time limits for temporary changes.}'' (pg. 12)


%INDEX% Reading assignment: USCSB Case Study, uncontrolled oleum release in Petrolia, Pennsylvania (2009)

%(END_NOTES)


