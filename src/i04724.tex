
%(BEGIN_QUESTION)
% Copyright 2010, Tony R. Kuphaldt, released under the Creative Commons Attribution License (v 1.0)
% This means you may do almost anything with this work of mine, so long as you give me proper credit

\noindent
Fieldbus cable types and maximum lengths:

% No blank lines allowed between lines of an \halign structure!
% I use comments (%) instead, so that TeX doesn't choke.

$$\vbox{\offinterlineskip
\halign{\strut
\vrule \quad\hfil # \ \hfil & 
\vrule \quad\hfil # \ \hfil & 
\vrule \quad\hfil # \ \hfil & 
\vrule \quad\hfil # \ \hfil & 
\vrule \quad\hfil # \ \hfil \vrule \cr
\noalign{\hrule}
%
% First row
{\bf Cable Type} & Type A & Type B & Type C & Type D \cr
%
\noalign{\hrule}
%
% Another row
{\bf Wire size} & AWG 18 & AWG 22 & AWG 26 & AWG 16 \cr
%
\noalign{\hrule}
%
% Another row
{\bf Char. Impedance} & 100 $\Omega$ $\pm$ 20\% & 100 $\Omega$ $\pm$ 30\% & -- & -- \cr
%
\noalign{\hrule}
%
% Another row
{\bf Shielding} & 1 for each pair & 1 for entire cable & none & none \cr
%
\noalign{\hrule}
%
% Another row
{\bf Twisted pairs} & Yes & Yes & Yes & No \cr
%
\noalign{\hrule}
%
% Another row
{\bf Max. length} & 1900 m & 1200 m & 400 m & 200 m \cr
%
\noalign{\hrule}
} % End of \halign 
}$$ % End of \vbox

\vskip 10pt

\noindent
RF calculations:

$$\hbox{Wavelength vs. frequency: \hskip 30pt} \lambda = {c \over f} \hbox{\hskip 30pt Where } c = 2.9979 \times 10^8 \hbox{ m/s}$$

\vskip 10pt

$$\hbox{Basic decibel formula: \hskip 30pt decibels} = 10 \log \left({P \over P_{ref}}\right)$$

\vskip 10pt

$$\hbox{Decibels (milliwatt reference): \hskip 30pt dBm} = 10 \log \left({P \over 0.001 \hbox{ W}}\right)$$

\vskip 10pt

$$\hbox{Decibels (watt reference): \hskip 30pt dBW} = 10 \log \left({P \over 1 \hbox{ W}}\right)$$

\vskip 10pt

$$\hbox{RF link budget formulae: \hskip 30pt} P_{rx} = P_{tx} + G + L \hskip 30pt P_{tx} = P_{rx} - (G + L) $$

\vskip 10pt

$$\hbox{Path loss: \hskip 30pt} L_{path} = -20 \log \left({4 \pi D \over \lambda}\right)$$

\vskip 10pt

$$\hbox{Fresnel zone radius: \hskip 30pt} r = \sqrt{{n \lambda d_1 d_2} \over D}$$

\underbar{file i04724}
%(END_QUESTION)





%(BEGIN_ANSWER)


%(END_ANSWER)





%(BEGIN_NOTES)

{\bf This question is intended for exams only and not worksheets!}

%(END_NOTES)

