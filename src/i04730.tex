
%(BEGIN_QUESTION)
% Copyright 2010, Tony R. Kuphaldt, released under the Creative Commons Attribution License (v 1.0)
% This means you may do almost anything with this work of mine, so long as you give me proper credit

The {\it filter time} on a digital transmitter or controller is analogous to the {\it time constant} of an RC or LR circuit: the amount of time required for the variable to change 63.2\% of the way toward its final value.

\vskip 10pt

Suppose a ``smart'' pressure transmitter is configured with a filter time of 2 seconds, and that a technician suddenly vents it to atmosphere after it had been measuring a process fluid pressure of 83 PSI.  Calculate the pressure signal output by this transmitter at 6 seconds' time following the vent (when its sensed pressure instantly dropped to 0 PSI):

\vskip 10pt

Output value at $t$ = 6 seconds: \underbar{\hskip 50pt} PSI

\vskip 10pt

Furthermore, calculate the mA signal value of this transmitter at that same time, assuming its calibrated range is 0 to 95 PSI (with an output range of 4 to 20 mA):

\vskip 10pt

Output signal at $t$ = 6 seconds: \underbar{\hskip 50pt} mA

\underbar{file i04730}
%(END_QUESTION)





%(BEGIN_ANSWER)

Output value at $t$ = 6 seconds: \underbar{\bf 4.132} PSI

\vskip 10pt

Output value at $t$ = 6 seconds: \underbar{\bf 4.696} mA

%(END_ANSWER)





%(BEGIN_NOTES)

{\bf This question is intended for exams only and not worksheets!}

%(END_NOTES)


