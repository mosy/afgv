
%(BEGIN_QUESTION)
% Copyright 2010, Tony R. Kuphaldt, released under the Creative Commons Attribution License (v 1.0)
% This means you may do almost anything with this work of mine, so long as you give me proper credit

You task is to work in a team to remove the protective cover from an ``ice-cube'' control relay used for industrial control circuitry.  Identify the following components of the device once the cover is removed:

\begin{itemize}
\item{} Moving contacts
\vskip 10pt
\item{} Stationary contacts
\vskip 10pt
\item{} Armature (moving iron piece)
\vskip 10pt
\item{} Coil terminals
\vskip 10pt
\item{} Coil voltage rating
\vskip 10pt
\item{} Contact voltage, current, and/or horsepower ratings
\end{itemize}

Feel free to connect the coil of the uncovered relay to a DC voltage source to watch its operation.  Feel free also to photograph the disassembled contactor with a digital camera for your own future reference.  Reassemble the contactor (ensuring the armature still moves freely) when done.  Be sure to bring appropriate tools to class for this exercise (e.g. phillips and slotted screwdrivers, multimeter).

\vskip 20pt \vbox{\hrule \hbox{\strut \vrule{} {\bf Suggestions for Socratic discussion} \vrule} \hrule}

\begin{itemize}
\item{} Identify potential points of failure inside the relay you are examining.  For each proposed fault, identify the effect(s) of that fault on the relay's operation.
\item{} Describe a procedure you might use to clean the contact surfaces inside a relay if they were to become dirty with use and exposure.
\item{} Demonstrate how you could use a multimeter to identify pin assignments on an electromechanical relay if it did not have a transparent case, and if the pinout diagram were obscured from view.
\end{itemize}

\underbar{file i04733}
%(END_QUESTION)





%(BEGIN_ANSWER)


%(END_ANSWER)





%(BEGIN_NOTES)

%INDEX% Electronics review, ice-cube relay: disassembly and inspection

%(END_NOTES)

