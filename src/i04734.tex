
%(BEGIN_QUESTION)
% Copyright 2011, Tony R. Kuphaldt, released under the Creative Commons Attribution License (v 1.0)
% This means you may do almost anything with this work of mine, so long as you give me proper credit

Your instructor will choose one VFD or discrete sensing device and one brand/model of PLC from the lists shown below, for which you must sketch an accurate circuit diagram showing how the PLC would connect to the VFD or sensor control/receive its status.  If additional electrical components are required (e.g. DC power source, electromechanical relay, etc.), those must be incorporated into your diagram as well.  Instruction manuals for all devices listed are available on the electronic Instrumentation Reference for your convenience.  When your sketch is complete, you must show the relevant manual pages to your instructor for verification of correct connections.

This exercise tests your ability to locate appropriate information in technical manuals and sketch a correct discrete control circuit for a given PLC and sensor/VFD.  The electronic Instrumentation Reference will be available to you in order to answer this question.

\vskip 10pt

It is highly recommended that you approach this exercise the same as for the design of any other electrical circuit: carefully identify all {\it sources} and {\it loads} in the circuit, identify devices requiring DC versus devices requiring AC, trace directions of all currents (DC only), and mark the polarities of all voltages (DC only).  Most of the mistakes made in this type of circuit design challenge may be remedied by careful consideration of these specific circuit-analysis details.



%%%%%%%%%%%%%%%%%%%%%%%%%%%%%%%
\vskip 10pt
\filbreak
\hbox{ \vrule
\vbox{ \hrule \vskip 3pt
\hbox{ \hskip 3pt
\vbox{ \hsize=5in \raggedright

\noindent \centerline{\bf Switch options}
\item{} Proximity/limit switch
\begin{itemize}

\item{} Mechanical process switch with form-C contacts
\item{} Inductive proximity switch, sourcing
\item{} Inductive proximity switch, sinking
\vskip 2pt
\item{} Level
\begin{itemize}

\item{} Rosemount 2120 vibrating fork level switch with PNP output
\end{itemize}
\end{itemize}

} \hskip 3pt}%
\vskip 5pt \hrule}%
\vrule}
%%%%%%%%%%%%%%%%%%%%%%%%%%%%%%%



%%%%%%%%%%%%%%%%%%%%%%%%%%%%%%%
\vskip 10pt
\filbreak
\hbox{ \vrule
\vbox{ \hrule \vskip 3pt
\hbox{ \hskip 3pt
\vbox{ \hsize=5in \raggedright

\noindent \centerline{\bf PLC options}
\item{} Siemens S7-300 I/O cards
\begin{itemize}

\item{} Input module: DI 32 x DC 24 V (6ES7321-1BL00-0AA0)
\item{} Input module: DI 16 x DC 24 V (6ES7321-1BH50-0AA0)
\item{} Input module: DI 16 x DC 48-125 V (6ES7321-1CH20-0AA0)
\item{} Output module: DO 32 x AC 120/230 V/1A (6ES7322-1FL00-0AA0)
\item{} Output module: DO 16 x DC 24 V/0.5A (6ES7322-1BH01-0AA0)
\item{} Output module: DO 16 x Rel (6ES7322-1HH01-0AA0)
\vskip 2pt
\item{} Rockwell (Allen-Bradley) ControlLogix 5000 I/O cards
\begin{itemize}

\item{} Input module: 1756-IB16 (16-channel discrete 24 VDC)
\item{} Output module: 1756-OA8 (8-channel discrete 120 VAC)
\item{} Output module: 1756-OB8 (8-channel discrete 24 VDC)
\item{} Output module: 1756-OW16I (16-channel discrete relay)
\end{itemize}
\end{itemize}

} \hskip 3pt}%
\vskip 5pt \hrule}%
\vrule}
%%%%%%%%%%%%%%%%%%%%%%%%%%%%%%%




%%%%%%%%%%%%%%%%%%%%%%%%%%%%%%%
\vskip 10pt
\filbreak
\hbox{ \vrule
\vbox{ \hrule \vskip 3pt
\hbox{ \hskip 3pt
\vbox{ \hsize=5in \raggedright

\noindent \centerline{\bf VFD options}
\item{} Rockwell PowerFlex 4 (discrete ``run'' input in SNK mode)
\item{} Rockwell PowerFlex 4 (discrete ``run'' input in SRC mode)
\item{} Automation Direct GS1 (discrete ``forward'' input)
\end{itemize}

} \hskip 3pt}%
\vskip 5pt \hrule}%
\vrule}
%%%%%%%%%%%%%%%%%%%%%%%%%%%%%%%




\vfil

\underbar{file i04734}
\eject
%(END_QUESTION)





%(BEGIN_ANSWER)


%(END_ANSWER)





%(BEGIN_NOTES)


%INDEX% Mastery exam performance exercise (circuit), drive a VFD with a PLC output

%(END_NOTES)


