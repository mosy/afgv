
%(BEGIN_QUESTION)
% Copyright 2010, Tony R. Kuphaldt, released under the Creative Commons Attribution License (v 1.0)
% This means you may do almost anything with this work of mine, so long as you give me proper credit

Read and outline the ``Polyphase AC Power'' section of the ``AC'' chapter in your {\it Lessons In Industrial Instrumentation} textbook, \underbar{skipping the subsection on ``Symmetrical Components''}.  Note the page numbers where important illustrations, photographs, equations, tables, and other relevant details are found.  Prepare to thoughtfully discuss with your instructor and classmates the concepts and examples explored in this reading.

\vskip 30pt

A note-taking technique you will find far more productive in your academic reading than mere highlighting or underlining is to write your own {\it outline} of the text you read.  A section of your {\it Lessons In Industrial Instrumentation} textbook called ``Marking Versus Outlining a Text'' describes the technique and the learning benefits that come from practicing it.  This approach is especially useful when the text in question is dense with facts and/or challenging to grasp.  Ask your instructor for help if you would like assistance in applying this proven technique to your own reading.

\underbar{file i04759}
%(END_QUESTION)





%(BEGIN_ANSWER)


%(END_ANSWER)





%(BEGIN_NOTES)

``Polyphase'' means ``more than one phase'' -- multiple AC voltages out-of-phase with each other.

\vskip 10pt

AC generators (alternators) comprised of a spinning, magnetized rotor surrounded by stationary coils of wire called stator windings.  Three-phase alternators have three sets of stator windings physically offset from one another by 120 degrees of rotation, creating a 120$^{o}$ electrical phase shift between their outputs.  The rotor is typically an electromagnet, energized through slip rings and carbon brushes by an external DC source.  Altering the current through this rotor winding controls the AC output voltage of the alternator by strengthening or weakening the rotor's magnetic field.

\vskip 10pt

``Phase'' refers to each element in a wye or delta network.  ``Line'' refers to the power conductors connecting to the vertices of a wye or delta network.  

In a wye-connected network, line voltage is $\sqrt{3}$ greater than phase (winding) voltage, because two phases are in series with respect to each pair of lines.  Line current = phase current because each line is in series with one phase element.

In a delta-connected network, line voltage = phase voltage because each phase element is in parallel with each pair of lines.  Line current = $\sqrt{3}$ greater than phase current because two phase elements combine current at a line node.

$\sqrt{3}$ is always the ``magic'' multiplier from phase to line quantities in balanced three-phase circuits.  480/277 volt wye-connected systems are an example of this $\sqrt{3}$ ratio.

\vskip 10pt

Polyphase AC power systems are more efficient than single-phase AC power systems, because there is never any time when the current is zero: at least one phase is always doing work even when the other phase(s) are at their zero points in time.  This is akin to a multi-cylinder engine where the pistons are offset from each other, at least one of them in its power stroke while the others are not.  Polyphase AC power makes ``smoother'' DC power when rectified, too.

\vskip 10pt

Total power in any three-phase network is equal to the simple sum of all phase element powers, because power is a scalar quantity.  Total power may also be calculated from line quantities as follows:

$$P_{total} = (\sqrt{3}) (I_{line}) (V_{line})$$

\vskip 10pt

{\it Grounding} is important for the purpose of limiting common-mode voltages in a power system, especially during abnormal conditions like lightning strikes.  Wye-connected sources often grounded at the center point.  4-wire wye-connected systems provide two levels of voltages (e.g. 480 VAC / 277 VAC, or 208 VAC / 120 VAC) for different types of loads.  Delta-connected sources often grounded at a center-tap of one of the phases, providing three levels of voltages (240 VAC / 208 VAC / 120 VAC) for different types of loads.


\filbreak


\vskip 20pt \vbox{\hrule \hbox{\strut \vrule{} {\bf Suggestions for Socratic discussion} \vrule} \hrule}

\begin{itemize}
\item{} Explain how an {\it alternator} functions.
\item{} Identify what a ``rotor'' is in an electric motor or alternator.
\item{} Identify what a ``stator'' is in an electric motor or alternator.
\item{} Explain why the rotor in an alternator is an electromagnet rather than a permanent magnet.
\item{} Explain some of the advantages of three-phase power systems over single-phase power systems.
\item{} Distinguish a ``phase'' quantity from a ``line'' quantity in a three-phase power system.
\item{} Explain why line current is equal to phase current in a Wye-connected three-phase network.
\item{} Explain why line voltage is equal to phase voltage in a Delta-connected three-phase network.
\item{} Explain why line voltage is not equal to phase voltage in a Wye-connected three-phase network.
\item{} Explain why line current is not equal to phase current in a Delta-connected three-phase network.
\item{} Calculate power in a balanced Wye-connected network with a phase voltage of 110 volts and a phase current of 2 amps. ($P_{total}$ = 660 watts)
\item{} Calculate power in a balanced Delta-connected network with a phase voltage of 230 volts and a phase current of 1 amp. ($P_{total}$ = 690 watts)
\item{} Calculate power in a balanced Wye-connected network with a line voltage of 208 volts and a phase current of 2 amps. ($P_{total}$ = 720.53 watts)
\item{} Calculate power in a balanced Delta-connected network with a phase voltage of 480 volts and a line current of 3 amps. ($P_{total}$ = 2494.15 watts)
\item{} Explain why power systems are often {\it grounded} at some point, even though they are able to deliver power to loads without connections to earth ground.
\item{} Identify where to connect a 208 volt single-phase load in a 208/120 volt wye-connected power system.
\item{} Identify where to connect a 120 volt single-phase load in a 208/120 volt wye-connected power system.
\item{} Identify where to connect a 240 volt single-phase load in a 240 volt high-leg delta-connected power system.
\item{} Identify where to connect a 208 volt single-phase load in a 240 volt high-leg delta-connected power system.
\item{} Identify where to connect a 120 volt single-phase load in a 240 volt high-leg delta-connected power system.
\end{itemize}










\vfil \eject

\noindent
{\bf Prep Quiz:}

Explain why line current and phase current are equal in a wye-connected three-phase circuit:

\begin{itemize}
\item{} Because both windings have the same number of turns
\vskip 5pt 
\item{} Because series-connected elements share the same current
\vskip 5pt 
\item{} Because Kirchhoff's Current Law requires equal currents at a node
\vskip 5pt 
\item{} Because parallel-connected elements share the same current
\vskip 5pt 
\item{} Because the phases are shifted apart from each other by 120$^{o}$
\vskip 5pt 
\item{} Because Tony says so
\end{itemize}










\vfil \eject

\noindent
{\bf Prep Quiz:}

Explain why line voltage and phase voltage are equal in a delta-connected three-phase circuit:

\begin{itemize}
\item{} Because both windings have the same number of turns
\vskip 5pt 
\item{} Because series-connected elements share the same voltage
\vskip 5pt 
\item{} Because Kirchhoff's Voltage Law requires equal voltages at a node
\vskip 5pt 
\item{} Because parallel-connected elements share the same voltage
\vskip 5pt 
\item{} Because the phases are shifted apart from each other by 120$^{o}$
\vskip 5pt 
\item{} Because Tony says so
\end{itemize}












\vfil \eject

\noindent
{\bf Prep Quiz:}

An advantage of {\it three-phase} power systems over {\it single-phase} power systems is:

\begin{itemize}
\item{} Fewer conductors needed to convey the same amount of power
\vskip 5pt 
\item{} Significantly less risk of electric shock and arc flash
\vskip 5pt 
\item{} Reduced weight of transformers and electric motors
\vskip 5pt 
\item{} Fewer components needed to rectify AC into DC
\vskip 5pt 
\item{} More constant delivery of electrical power to the load
\vskip 5pt 
\item{} Far easier to understand and to troubleshoot faults
\end{itemize}








\vfil \eject

\noindent
{\bf Summary Quiz:}

Calculate the amount of power in a balanced three-phase AC system with a line voltage of 13.8 kV and a line current of 25.2 amps.

\begin{itemize}
\item{} 1.043 MW
\vskip 5pt 
\item{} 602.3 kW
\vskip 5pt 
\item{} 537.8 kW
\vskip 5pt 
\item{} 347.8 kW
\vskip 5pt 
\item{} 200.8 kW
\vskip 5pt 
\item{} 115.9 kW
\end{itemize}

%INDEX% Reading assignment: Lessons In Industrial Instrumentation, polyphase AC power

%(END_NOTES)

