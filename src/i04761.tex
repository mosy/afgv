
%(BEGIN_QUESTION)
% Copyright 2015, Tony R. Kuphaldt, released under the Creative Commons Attribution License (v 1.0)
% This means you may do almost anything with this work of mine, so long as you give me proper credit

Read and outline the ``AC Motor Braking'' section of the ``Variable-Speed Motor Controls'' chapter in your {\it Lessons In Industrial Instrumentation} textbook.  Note the page numbers where important illustrations, photographs, equations, tables, and other relevant details are found.  Prepare to thoughtfully discuss with your instructor and classmates the concepts and examples explored in this reading.

\underbar{file i04761}
%(END_QUESTION)





%(BEGIN_ANSWER)

If you have access to an AC induction motor and some batteries, feel free to experiment with DC injection braking in the class or lab!  All you need to do is connect a source of low-voltage DC to the stator winding(s) of an AC induction motor, and you will be able to feel the braking effect as you try to spin the motor's shaft with your fingers!

%(END_ANSWER)





%(BEGIN_NOTES)

VFDs may be used to brake AC induction motors.  Law of Energy Conservation states that the kinetic energy of the machine must go somewhere, so in each of these braking methods it is important to keep in mind where that energy will go.  Four methods used:

\begin{itemize}
\item{} {\bf DC injection:} DC is sent to stator windings instead of AC.  Induction between spinning rotor and stationary magnetic field causes reverse torque (Lenz's Law), braking the motor.  Heat energy dissipated in the rotor.  Braking force varies directly with rotor speed and also with DC injection current.
\vskip 10pt
\item{} {\bf Dynamic:} Motor run as generator by giving it a power frequency slower than its actual rotating speed.  When this happens, the DC bus voltage raises.  Power produced by motor will be dissipated in a load resistor connected to the VFD, current through that resistor controlled by an added transistor, turned on as needed to stabilize the DC bus voltage.  Less motor heating by this braking method than with DC injection.
\vskip 10pt
\item{} {\bf Regenerative:} Motor run as generator by giving it a power frequency slower than its actual rotating speed.  VFD equipped with an ``active front end'' whereby six more transistors synchronously switch to send excessive DC bus voltage back to AC power lines.  This sends braking energy to the AC power grid, rather than wasting it in the form of heat.  A simple form of regenerative braking not requiring an active front end in the VFD parallels the DC busses of two or more VFDs, so that excessive bus voltage produced by one braking motor may be put to use accelerating another motor.
\vskip 10pt
\item{} {\bf Plugging:} Reverse power sent to motor, with all braking energy dissipated as heat in the rotor.  Highly effective at braking, like putting a boat in reverse to slow down its forward motion.
\end{itemize}

















\vskip 20pt \vbox{\hrule \hbox{\strut \vrule{} {\bf Suggestions for Socratic discussion} \vrule} \hrule}

\begin{itemize}
\item{} Explain how and why {\it DC injection} works to brake an induction motor.  Where does the braking energy go?
\item{} Explain how and why {\it dynamic braking} works to brake an induction motor.  Where does the braking energy go?
\item{} Explain how and why {\it regenerative braking} works to brake an induction motor.  Where does the braking energy go?
\item{} Explain how and why {\it plugging} works to brake an induction motor.  Where does the braking energy go?
\item{} Identify which of the braking techniques described in this section may be implemented without a VFD.
\item{} Identify some advantages and disadvantages of paralleling the DC busses of two or more VFDs to exploit regenerative braking.
\item{} Explain why the {\it Law of Energy Conservation} is important to understand when studying motor braking.
\item{} Identify a reason why a mechanical brake might be preferred in some applications to VFD braking.
\item{} Explain how you might be able to demonstrate one of these braking techniques using an induction motor and some simple electrical/electronic components (no VFDs!).
\item{} Explain what would happen to a VFD with dynamic braking if the braking resistor failed open.
\item{} Explain what would happen to a VFD with dynamic braking if the braking resistor failed shorted.
\item{} Explain what would happen to a pair of VFDs with paralleled DC busses if the connection between them failed open while one of the VFDs was braking and the other was accelerating.
\end{itemize}


%INDEX% Reading assignment: Lessons In Industrial Instrumentation, AC motor braking

%(END_NOTES)

