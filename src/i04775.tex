
%(BEGIN_QUESTION)
% Copyright 2013, Tony R. Kuphaldt, released under the Creative Commons Attribution License (v 1.0)
% This means you may do almost anything with this work of mine, so long as you give me proper credit

The concepts of {\it sensible heat} and {\it latent heat} are often confusing to people first studying thermodynamics.  Do your best to differentiate between these two heat-transfer phenomena, and give practical examples of each.

\vskip 10pt

Additionally, express any mathematical formulae relevant to sensible heat and to latent heat.

\underbar{file i04775}
%(END_QUESTION)





%(BEGIN_ANSWER)

{\it Sensible heat} is any form of heat transfer resulting in a temperature change.  The relationship between heat transferred ($Q$) and temperature change ($\Delta T$) is proportional to both the mass of the sample and its {\it specific heat capacity} ($c$):

$$Q = mc \Delta T$$
 
\vskip 10pt

{\it Latent heat} is any form of heat transfer resulting in a phase change (e.g. solid to liquid, liquid to gas, or vice-versa).  The relationship between heat transferred ($Q$) and the amount of mass undergoing a phase change ($m$) it the {\it latent heat capacity} of the sample ($L$):

$$Q = mL$$

%(END_ANSWER)





%(BEGIN_NOTES)


%INDEX%: Physics, heat and temperature: latent heat versus specific heat

%(END_NOTES)


