
%(BEGIN_QUESTION)
% Copyright 2013, Tony R. Kuphaldt, released under the Creative Commons Attribution License (v 1.0)
% This means you may do almost anything with this work of mine, so long as you give me proper credit

The three major mechanisms of heat transfer are {\it conduction}, {\it convection}, and {\it radiation}.  Give practical examples for each of the following phenomena:

\begin{itemize}
\item{} {\it Conductive} transfer resulting in {\it sensible} heat
\vskip 10pt
\item{} {\it Convective} transfer resulting in {\it sensible} heat
\vskip 10pt
\item{} {\it Radiant} transfer resulting in {\it sensible} heat
\vskip 10pt
\item{} {\it Conductive} transfer resulting in {\it latent} heat
\vskip 10pt
\item{} {\it Convective} transfer resulting in {\it latent} heat
\vskip 10pt
\item{} {\it Radiant} transfer resulting in {\it latent} heat
\end{itemize}

\vskip 10pt

Additionally, express any mathematical formulae relevant to sensible versus latent heat.

\underbar{file i04776}
%(END_QUESTION)





%(BEGIN_ANSWER)

Some practical examples (by no means an exhaustive list) are given here:

\begin{itemize}
\item{} {\it Conductive} transfer resulting in {\it sensible} heat: {\it Placing a frying pan on a hot stove causes the pan's temperature to rise.}
\vskip 10pt
\item{} {\it Convective} transfer resulting in {\it sensible} heat: {\it Pointing the flame of a propane torch on to a metal surface causes that metal's temperature to rise.}
\vskip 10pt
\item{} {\it Radiant} transfer resulting in {\it sensible} heat: {\it Feeling the increased skin temperature resulting from standing several yards away from a large bonfire.}
\vskip 10pt
\item{} {\it Conductive} transfer resulting in {\it latent} heat: {\it An ice cube placed on a warm frying pan will melt into water.}
\vskip 10pt
\item{} {\it Convective} transfer resulting in {\it latent} heat: {\it A hair dryer causes liquid water in your hair to be forced into a vapor state.}
\vskip 10pt
\item{} {\it Radiant} transfer resulting in {\it latent} heat: {\it A frozen ice puddle melting under the sun's rays.}
\end{itemize}

\vskip 30pt

\noindent
{\bf Sensible heat} (resulting in a temperature change:

$$Q = m c \Delta T$$

\noindent
Where,

$Q$ = Heat gain or loss (metric calories or British BTU)

$m$ = Mass of sample (metric grams or British pounds)

$c$ = Specific heat of substance

$\Delta T$ = Temperature change (metric degrees Celsius or British degrees Fahrenheit)


\vskip 30pt


\noindent
{\bf Latent heat} (resulting in a phase change:

$$Q = m L$$

\noindent
Where,

$Q$ = Heat of transition required to completely change the phase of a sample (metric calories or British BTU)

$m$ = Mass of sample (metric grams or British pounds)

$L$ = Latent heat of substance

\vskip 10pt


%(END_ANSWER)





%(BEGIN_NOTES)


%INDEX%: Physics, heat and temperature: latent heat versus specific heat

%(END_NOTES)


