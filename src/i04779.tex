
%(BEGIN_QUESTION)
% Copyright 2013, Tony R. Kuphaldt, released under the Creative Commons Attribution License (v 1.0)
% This means you may do almost anything with this work of mine, so long as you give me proper credit

In 1804, an experimenter named Sir Benjamin Thompson (also known as Count Rumford) investigated the heat liberated from the mechanical boring of the hole in a newly manufactured cannon.  The cannon barrel was immersed in a box full of water for cooling, and by using a dull tool to bore the hole in the cannon Thompson was able to prolong the process of boring until he had generated enough heat to actually boil the water contained in the box.

\vskip 10pt

A popular theory of heat at this time was the so-called {\it caloric theory}, which maintained that heat was an invisible fluid called ``caloric'' that could be passed from one object to another by direct contact.  According to the caloric theory, the reason a new cannon barrel became hot during the process of boring a hole in it was because the cleaving of metal chips from the barrel released some of the caloric ``fluid'' stored within the metal.

\vskip 10pt

Albert Einstein was reported to have quipped, {\it No amount of experimentation can can ever prove me right; a single experiment can prove me wrong}.  The truth in this statement is that science does not validate some claims so much as it {\it invalidates} other claims -- the power of scientific experimentation being in the {\it disproof} rather than in the {\it proof}.  A theory may be overthrown if one or more of its testable predictions proves to be false.  

Apply this strategy to testing the caloric theory of heat (that heat is an invisible fluid released by cutting metal) in Thompson's cannon-boring experiment.  Does the caloric theory pass this test, or not?  Explain your answer in detail.

\underbar{file i04779}
%(END_QUESTION)





%(BEGIN_ANSWER)

If the caloric theory of heat were correct, then the cannon barrel would not be heated by a dull tool, but only heated when bored out with a sharp tool.  If heat is a fluid released by cutting, then only successful cutting of the metal (not unsuccessful grinding of a dull tool against the cannon barrel) would cause the water to heat up.  A dull tool should, according to the caloric theory, {\it liberate less heat than a sharp tool}.  What Thompson found instead was that a dull tool actually liberated more heat than a sharp tool because it allowed the fruitless grinding process to continue long after it would have taken a sharp tool to bore the hole.  In summary, Thompson's experiment demonstrated that heat was a function of mechanical work, not of cleaving metal.
 
%(END_ANSWER)





%(BEGIN_NOTES)


%INDEX% Physics, heat and temperature: equivalence of mechanical work with heat

%(END_NOTES)


