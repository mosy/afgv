
%(BEGIN_QUESTION)
% Copyright 2013, Tony R. Kuphaldt, released under the Creative Commons Attribution License (v 1.0)
% This means you may do almost anything with this work of mine, so long as you give me proper credit

Suppose a compressor is operating with the following suction and discharge parameters:

\vskip 10pt

\noindent
{\bf Suction:}
\begin{itemize}
\item{} Pressure = 45 PSIG
\item{} Volumetric flow = 1300 CFM
\item{} Temperature = 74 deg F
\end{itemize}

\vskip 10pt

\noindent
{\bf Discharge:}
\begin{itemize}
\item{} Pressure = 281 PSIG
\item{} Volumetric flow = 317 CFM
\item{} Temperature = 186 deg F
\end{itemize}

\vskip 10pt

From these figures, calculate the operating {\it compression ratio} of this gas compressor.


\underbar{file i04795}
%(END_QUESTION)





%(BEGIN_ANSWER)

The compression ratio is simply the reciprocal of the ratio of volumes between outlet and inlet (discharge and suction).  Taking the ratio of volumetric flow rates also works, since these are nothing more than volumes per unit measure of time.  

$$\hbox{Compression ratio} = \left(\hbox{Suction flow} \over \hbox{Discharge flow}\right) = \left(1300 \hbox{ CFM} \over 317 \hbox{ CFM}\right) = 4.101$$

Like all ratios, compression ratio is a unitless quantity because the units of measurement in the numerator and denominator are identical and therefore cancel each other.
 
%(END_ANSWER)





%(BEGIN_NOTES)


%INDEX% Process: compression ratio based on compressor flow rates

%(END_NOTES)


