
%(BEGIN_QUESTION)
% Copyright 2013, Tony R. Kuphaldt, released under the Creative Commons Attribution License (v 1.0)
% This means you may do almost anything with this work of mine, so long as you give me proper credit

Water flowing through an 8-inch pipe at an average velocity of 5 feet per second enters a narrower section of pipe (3 inches), changing its velocity accordingly.  Calculate the water's average velocity in this narrower section of pipe.

\underbar{file i04796}
%(END_QUESTION)





%(BEGIN_ANSWER)

The Continuity equation relates volumetric flow rate to pipe area and average velocity, assuming a constant fluid density:

$$Q = A_1 v_1 = A_2 v_2$$

Given the same flow rate in both sections of pipe, the relationship between pipe area and velocity is as such:

$${v_2 \over v_1} = {A_1 \over A_2}$$

Since area is proportional to the square of the diameter (or radius), we may express the ratio of velocities as a ratio of squared diameters:
 
$${v_2 \over v_1} = \left({d_1 \over d_2}\right)^2$$

This being the case, we may solve for the velocity in the narrower section of pipe:

$$v_2 = v_1 \left({d_1 \over d_2}\right)^2$$

$$v_2 = 5 \hbox{ ft/s} \left({8 \hbox{ in} \over 3 \hbox{ in}}\right)^2 = 35.56 \hbox{ ft/s}$$

%(END_ANSWER)





%(BEGIN_NOTES)


%INDEX% Physics, dynamic fluids: continuity equation

%(END_NOTES)


