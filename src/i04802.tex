
%(BEGIN_QUESTION)
% Copyright 2013, Tony R. Kuphaldt, released under the Creative Commons Attribution License (v 1.0)
% This means you may do almost anything with this work of mine, so long as you give me proper credit

Suppose a wrecking ball is suspended on the end of a 30 foot cable.  If the wrecking ball is drawn to the side until the cable is at a 20 degree angle (from vertical) and then released to swing, how fast will the wrecking ball be traveling at its maximum velocity?

\underbar{file i04802}
%(END_QUESTION)





%(BEGIN_ANSWER)

As the wrecking ball is pulled sideways until its cable forms a 20 degree angle from vertical, it will be lifted up 1.809 feet from the height it started (a right triangle with a hypotenuse of 30 feet and an adjacent angle of 20 degrees will have a side length of 28.19 feet, or 1.809 feet less than 30 feet).

\vskip 10pt

From this known amount of vertical lift, we may calculate the wrecking ball's maximum velocity by assuming all that potential energy gets converted into kinetic energy:

$${1 \over 2}mv^2 = mgh$$

$${1 \over 2}v^2 = gh$$

$$v^2 = 2gh$$

$$v = \sqrt{2 g h}$$

$$v = \sqrt{(2) (32.2 \hbox{ ft/s}^2)  (1.809 \hbox{ ft})}$$

$$v = 10.79 \hbox{ ft/s}$$

%(END_ANSWER)





%(BEGIN_NOTES)


%INDEX% Physics, energy, work, power

%(END_NOTES)


