
%(BEGIN_QUESTION)
% Copyright 2013, Tony R. Kuphaldt, released under the Creative Commons Attribution License (v 1.0)
% This means you may do almost anything with this work of mine, so long as you give me proper credit

A mixture of oxygen and helium gases is prepared for use by scuba divers, mixing the following quantities of these two gases together:

\begin{itemize}
\item{} 27 moles of pure oxygen (O$_{2}$) gas
\vskip 10pt
\item{} 108 moles of pure helium (He) gas
\end{itemize}

A chemist calculates the mass of this gas mixture in preparation to verify the mixture by means of a mass scale.  In calculating mass, the chemist writes the following mathematical expression:

$$\left(27 \hbox{ mol O}_2 \over 1 \right) \left(32 \hbox{ g} \over \hbox{mol O}_2 \right) + \left(108 \hbox{ mol He} \over 1 \right) \left(4 \hbox{ g} \over \hbox{mol He} \right)$$

\vskip 10pt

Complete this calculation for the total mass of the oxygen/helium gas mixture, and then explain the chemist's rationale for writing the calculation as she did.

\underbar{file i04805}
%(END_QUESTION)





%(BEGIN_ANSWER)

$$\left(27 \hbox{ mol O}_2 \over 1 \right) \left(32 \hbox{ g} \over \hbox{mol O}_2 \right) + \left(108 \hbox{ mol He} \over 1 \right) \left(4 \hbox{ g} \over \hbox{mol He} \right) = 1296 \hbox{ g} = 1.296 \hbox{ kg}$$

The fraction $32 \hbox{ g} \over 1 \hbox{ mol O}_2$ expresses the proportionality between grams of mass and moles of oxygen gas, as a ``unity fraction''.  The fraction $4 \hbox{ g} \over 1 \hbox{ mol He}$ does the same for helium gas.  This allows the chemist to proceed with calculations of mass as though it were nothing more than a simple unit conversion problem, with units of ``moles'' canceling to leave no unit left except for ``grams''.
 
%(END_ANSWER)





%(BEGIN_NOTES)



%INDEX% Chemistry, stoichiometry: moles

%(END_NOTES)


