
% Copyright 2012, Tony R. Kuphaldt, released under the Creative Commons Attribution License (v 1.0)
% This means you may do almost anything with this work of mine, so long as you give me proper credit

%(BEGIN_FRONTMATTER)

\centerline{\bf Methods of instruction}

\vskip 10pt

This course develops self-instructional and diagnostic skills by placing students in situations where they are required to research and think independently.  In all portions of the curriculum, the goal is to avoid a passive learning environment, favoring instead {\it active engagement} of the learner through reading, reflection, problem-solving, and experimental activities.  The curriculum may be roughly divided into two portions: {\it theory} and {\it practical}.

\vskip 30pt

\noindent
{\bf Theory}

In the theory portion of each course, students independently research subjects {\it prior} to entering the classroom for discussion.  This means working through all the day's assigned questions as completely as possible.  This usually requires a fair amount of technical reading, and may also require setting up and running simple experiments.  At the start of the classroom session, the instructor will check each student's preparation with a quiz.  Students then spend the rest of the classroom time working in groups and directly with the instructor to {\it thoroughly} answer all questions assigned for that day, articulate problem-solving strategies, and to approach the questions from multiple perspectives.  To put it simply: fact-gathering happens outside of class and is the individual responsibility of each student, so that class time may be devoted to the more complex tasks of critical thinking and problem solving where the instructor's attention is best applied.

Classroom theory sessions usually begin with either a brief Q\&A discussion or with a ``Virtual Troubleshooting'' session where the instructor shows one of the day's diagnostic question diagrams while students propose diagnostic tests and the instructor tells those students what the test results would be given some imagined (``virtual'') fault scenario, writing the test results on the board where all can see.  The students then attempt to identify the nature and location of the fault, based on the test results.

Each student is free to leave the classroom when they have completely worked through all problems and have answered a ``summary'' quiz designed to gauge their learning during the theory session.  If a student finishes ahead of time, they are free to leave, or may help tutor classmates who need extra help.

{\it The express goal of this ``inverted classroom'' teaching methodology is to help each student cultivate critical-thinking and problem-solving skills, and to sharpen their abilities as independent learners.  While this approach may be very new to you, it is more realistic and beneficial to the type of work done in instrumentation, where critical thinking, problem-solving, and independent learning are ``must-have'' skills.}

%INSTRUCTOR {\bf Quizzes are an effective tool for preparation assessment.  ``Prep'' quizzes should be simple and concept-related (not too many quantitative calculations).  The goal is to test whether or not students have spent significant time researching the material, not necessarily their mastery of it.  Each quiz should be designed so that any hard-working student should be able to get it right even if they have not yet mastered the concept.  Since each student is strongly encouraged to keep a notebook for reading annotation, answering of assigned questions, and classroom notes, each student may reference their notebook while taking the quizzes.}

%INSTRUCTOR {\bf Quizzes also work well at the end of each classroom session to assess student engagement during discussion.  I recommend making these ``Summary'' quizzes more challenging than the homework preparation quizzes.}




%INSTRUCTOR \vskip 10pt \filbreak





%INSTRUCTOR \noindent {\bf SOCRATIC DIALOGUE TIPS FOR THE INSTRUCTOR:}

%INSTRUCTOR \item {$\bullet$} {\bf Ask students to demonstrate how they applied specific tips listed in Question 0 to the subject and/or problems at hand.}

%INSTRUCTOR \item {$\bullet$} {\bf Ask students to articulate the \underbar{principles} applicable to the subject and/or problems at hand.  Most students exhibit the tendency to focus on procedures rather than principles, which is why they struggle at solving novel problems.  One of your main tasks as an instructor is to get them thinking in terms of general principles, asking ``why'' questions instead of ``how'' or ``what'' questions.}

%INSTRUCTOR \item {$\bullet$} {\bf Have students use the whiteboard to post questions they or their group has on specific problems.  This allows other students to see where their classmates need help, encouraging peer tutoring.}

%INSTRUCTOR \item {$\bullet$} {\bf Pose ``thought experiment'' problems, asking students to predict what will happen in a scenario if some variable is changed.}

%INSTRUCTOR \item {$\bullet$} {\bf Pose ``Virtual Troubleshooting'' problems, asking students to specify tests they would do on a faulted system to identify the problem.  The instructor's role during this type of exercise is to keep a fault scenario in mind while replying to students what the result(s) of each test would be.  Ask students what each test result tells them about the nature and location of the fault, and also ask them what it would mean if the test result(s) were different.}

%INSTRUCTOR \item {$\bullet$} {\bf If students get ``stuck'' during a large-group discussion, have all the students break into teams of 2 or 3 to share solution ideas to the problem (or to a similar problem posed Socratically by the instructor).  This almost never fails to resolve the difficulty and re-start the classroom dialogue.}

%INSTRUCTOR \item {$\bullet$} {\bf Another way to help students get ``un-stuck'' on a problem is to slowly and silently solve the problem yourself on the whiteboard (or on paper in a small group), pausing after each step to give students time to analyze your steps and explain to you the rationale behind each one.}

%INSTRUCTOR \item {$\bullet$} {\bf Ask student teams to write their explanation of a concept in their own words or to explain using their own diagrams, then ask other student teams to critique those explanations.  The goal here is to identify ways to improve each explanation, because there is always a way to improve something!}

\vskip 30pt







\filbreak








\noindent
{\bf Lab}

In the lab portion of each course, students work in teams to install, configure, document, calibrate, and troubleshoot working instrument loop systems.  Each lab exercise focuses on a different type of instrument, with a eight-day period typically allotted for completion.  An ordinary lab session might look like this:

\medskip
\item{$(1)$} Start of practical (lab) session: announcements and planning
\itemitem{(a)} The instructor makes general announcements to all students
\itemitem{(b)} The instructor works with team to plan that day's goals, making sure each team member has a clear idea of what they should accomplish
\item{$(2)$} Teams work on lab unit completion according to recommended schedule:
\item\item{(First day)} Select and bench-test instrument(s)
\item\item{(One day)} Connect instrument(s) into a complete loop
\item\item{(One day)} Each team member drafts their own loop documentation, inspection done as a team (with instructor)
\item\item{(One or two days)} Each team member calibrates/configures the instrument(s)
\item\item{(Remaining days, up to last)} Each team member troubleshoots the instrument loop
\item{$(3)$} End of practical (lab) session: debriefing where each team reports on their work to the whole class
\medskip

\vskip 10pt

\noindent
{\bf Troubleshooting assessments must meet the following guidelines:}

\medskip
\item{$\bullet$} Troubleshooting must be performed {\it on a system the student did not build themselves}.  This forces students to rely on another team's documentation rather than their own memory of how the system was built.
\item{$\bullet$} Each student must individually demonstrate proper troubleshooting technique.
\item{$\bullet$} Simply finding the fault is not good enough.  Each student must consistently demonstrate sound reasoning while troubleshooting.
\item{$\bullet$} If a student fails to properly diagnose the system fault, they must attempt (as many times as necessary) with different scenarios until they do, reviewing any mistakes with the instructor after each failed attempt.
\medskip

%INSTRUCTOR {\bf The same structure of having one student perform a task while the other members of the team observe may also be applied to calibration.  One student calibrates the instrument while the other members of the team observe the steps, take notes, and check instrument calibration by calculating and/or sketching error percentages.  This way, the observing students get to watch calibration technique(s), see how to operate the calibration equipment multiple times, and practice determining instrument calibration errors (e.g. zero shifts, span shifts, and nonlinearities) based on numerical data.}

%INSTRUCTOR {\bf I strongly recommend having students work as a team to inspect their loop and loop diagram accuracies.  This is a much more time-efficient way to check their construction and documentation work than to inspect their loop diagrams individually.  When all team members have their loop diagrams complete, take those diagrams and do a ``walk through'' of the loop, inspecting quality of assembly and diagram accuracy.  If any problems are noted, bring it to the attention of the whole group and have them correct the problems before doing another walk-through with you.}

%INSTRUCTOR {\bf Lab questions are answered by students individually, and may be asked at any point along the lab exercise schedule where appropriate.  The point of these questions is not just to measure student learning, but also to serve as a guide for students regarding the skills and concepts they should be mastering as they progress through the lab exercise.}



\vfil

\underbar{file {\tt instructional}}
\eject
%(END_FRONTMATTER)


