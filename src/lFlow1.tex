% !TEX root = /home/fred-olav/afgv/src/preamble.tex
\centerline{\bf Strømningsmålling 1}  \bigskip

Kompetansemål:
\begin{itemize}[noitemsep]

	\item montere, konfigurere, kalibrere og idriftsettelse digitale og analoge målesystemer
	\item måle fysiske størrelser i automatiserte anlegg
\end{itemize}
	Læringsmål
	\begin{itemize}[noitemsep]
		\item Kunne 
		\item Kunne 
	\end{itemize}

	Forkunnskaper

	\begin{itemize}[noitemsep]
		\item 

	\end{itemize}
\vfil \eject
\centerline{\bf Anbefalt fremdrift} 

\vskip 5pt

%%%%%%%%%%%%%%%
\filbreak
\hrule \vskip 5pt
\noindent \underbar{Leksjon 1}

\vskip 5pt

%INSTRUCTOR \noindent {\bf Problem-solving intro activity:} review INST241\_x1 exam.

\vskip 2pt \noindent {\bf Emne for lekesjonen:} Teknologier for flowmåling

\vskip 2pt \noindent Oppgave 1 - 20; \underbar{besvar oppgave 1-10} som forberedelse til leksjon %(remainder for practice)

\vskip 10pt



%%%%%%%%%%%%%%%
\filbreak
\hrule \vskip 5pt
\noindent \underbar{Leksjon 2}

\vskip 5pt

%INSTRUCTOR \noindent {\bf Problem-solving intro activity:} explore the ``curved arrow'' notation for voltage in DC circuits shown in the ``Electrical Sources and Loads'' section of the ``DC Electricity'' chapter of the LIII textbook, commenting on how this notation is analagous to force and displacement, helping to explain positive and negative quantities of mechanical work.

\vskip 2pt \noindent {\bf Emne for leksjonen:} Fluid dynamikk

\vskip 2pt \noindent Oppgave 21 - 40; \underbar{besvar oppgave 21-30} som forberedelse til leksjonen%(remainder for practice)

\vskip 10pt



%%%%%%%%%%%%%%%
\filbreak
\hrule \vskip 5pt
\noindent \underbar{Leskjon 3}

\vskip 5pt

%INSTRUCTOR \noindent {\bf Problem-solving intro activity:} apply the critical reading strategy suggested in Question 0 where readers are encouraged to work through mathematical exercises.  A specific example of this would be to verify the square-root scales of indicator gauges shown in the textbook using a calculator, correlating equivalent values shown on the linear versus square-root scales.

\vskip 2pt \noindent {\bf Emne for leksjonen:} Trykkbaserte strømningsmålere

\vskip 2pt \noindent Oppgave 41 - 60; \underbar{besvar oppgave 41-50} som forberedelse til leksjonen%(remainder for practice)

\vskip 10pt




%%%%%%%%%%%%%%%
\filbreak
\hrule \vskip 5pt

%\noindent \underbar{Leksjon 4}

%\vskip 5pt

%%%%%INSTRUCTOR \noindent {\bf Problem-solving intro activity:} Research equipment manuals to sketch a complete circuit connecting a loop controller to either a 4-20 mA transmitter or a 4-20 mA final control element ({\tt i03773})

%%%%%INSTRUCTOR \noindent {\bf Problem-solving intro activity:} identifying possible ways in which an orifice-based flowmeter can give false readings.  Refer to the P\&ID in {\tt i03490} for examples, such as nitrogen flowmeter FT-29 or steam flowmeter FT-28.

%\vskip 2pt \noindent {\bf Theory session topic:} High-accuracy pressure-based flow measurement

%\vskip 2pt \noindent Oppgave 61 - 80; \underbar{besvar oppgave 61-70} som forberedelse til leksjonen%(remainder for practice)

%\vskip 5pt

%\noindent Feedback questions {\it (81 through 90)} are optional and may be submitted for review at the end of the day
%
%\vskip 10pt

