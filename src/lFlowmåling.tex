% !TEX root = /home/fred-olav/afgv/src/preamble.tex
\centerline{\bf Strømningsmålling}  \bigskip

Kompetansemål:
\begin{itemize}[noitemsep]

	\item montere, konfigurere, kalibrere og idriftsettelse digitale og analoge målesystemer
	\item måle fysiske størrelser i automatiserte anlegg
\end{itemize}
	Læringsmål
	\begin{itemize}[noitemsep]
		\item Kunne 
		\item Kunne 
	\end{itemize}

	Forkunnskaper

	\begin{itemize}[noitemsep]
		\item 

	\end{itemize}
\vfil \eject
\centerline{\bf Anbefalt fremdrift} 

\vskip 5pt

%%%%%%%%%%%%%%%
\filbreak
\hrule \vskip 5pt
\noindent \underbar{Leksjon 1}

\vskip 5pt

%INSTRUCTOR \noindent {\bf Problem-solving intro activity:} review INST241\_x1 exam.

\vskip 2pt \noindent {\bf Emne for lekesjonen:} Teknologier for flowmåling

\vskip 2pt \noindent Oppgave 1 - 20; \underbar{besvar oppgave 1-10} som forberedelse til leksjon %(remainder for practice)

\vskip 10pt



%%%%%%%%%%%%%%%
\filbreak
\hrule \vskip 5pt
\noindent \underbar{Leksjon 2}

\vskip 5pt

%INSTRUCTOR \noindent {\bf Problem-solving intro activity:} explore the ``curved arrow'' notation for voltage in DC circuits shown in the ``Electrical Sources and Loads'' section of the ``DC Electricity'' chapter of the LIII textbook, commenting on how this notation is analagous to force and displacement, helping to explain positive and negative quantities of mechanical work.

\vskip 2pt \noindent {\bf Emne for leksjonen:} Fluid dynamikk

\vskip 2pt \noindent Oppgave 21 - 40; \underbar{besvar oppgave 21-30} som forberedelse til leksjonen%(remainder for practice)

\vskip 10pt



%%%%%%%%%%%%%%%
\filbreak
\hrule \vskip 5pt
\noindent \underbar{Leskjon 3}

\vskip 5pt

%INSTRUCTOR \noindent {\bf Problem-solving intro activity:} apply the critical reading strategy suggested in Question 0 where readers are encouraged to work through mathematical exercises.  A specific example of this would be to verify the square-root scales of indicator gauges shown in the textbook using a calculator, correlating equivalent values shown on the linear versus square-root scales.

\vskip 2pt \noindent {\bf Emne for leksjonen:} Trykkbaserte strømningsmålere

\vskip 2pt \noindent Oppgave 41 - 60; \underbar{besvar oppgave 41-50} som forberedelse til leksjonen%(remainder for practice)

\vskip 10pt




%%%%%%%%%%%%%%%
\filbreak
\hrule \vskip 5pt

\noindent \underbar{Leksjon 4}

\vskip 5pt

%%%%%INSTRUCTOR \noindent {\bf Problem-solving intro activity:} Research equipment manuals to sketch a complete circuit connecting a loop controller to either a 4-20 mA transmitter or a 4-20 mA final control element ({\tt i03773})

%%%%%INSTRUCTOR \noindent {\bf Problem-solving intro activity:} identifying possible ways in which an orifice-based flowmeter can give false readings.  Refer to the P\&ID in {\tt i03490} for examples, such as nitrogen flowmeter FT-29 or steam flowmeter FT-28.

\vskip 2pt \noindent {\bf Emne for leksjonen:} High-accuracy pressure-based flow measurement

\vskip 2pt \noindent Oppgave 61 - 80; \underbar{besvar oppgave 61-70} som forberedelse til leksjonen%(remainder for practice)

%\vskip 5pt

%\noindent Feedback questions {\it (81 through 90)} are optional and may be submitted for review at the end of the day
%
%\vskip 10pt

\noindent \underbar{Leksjon 5} 

\vskip 5pt

%INSTRUCTOR \noindent {\bf Problem-solving intro activity:} identifying possible/impossible faults in a flow measurement loop using either a turbine, vortex, or positive displacement flowmeter, referencing a P\&ID (e.g. from the {\it Realistic Instrumentation Diagrams} practice worksheet)

\vskip 2pt \noindent {\bf Emne for leksjonen:} Turbine, vortex, and positive-displacement flowmeters

\vskip 2pt \noindent Oppgave 1-20; \underbar{besvar oppgave 1-9} som forberedelse til laksjonen (remainder for practice)

\vskip 10pt



%%%%%%%%%%%%%%%
\filbreak
\hrule \vskip 5pt
\noindent \underbar{Leksjon 6}

\vskip 5pt

%INSTRUCTOR \noindent {\bf Problem-solving intro activity:} identifying valid/invalid diagnostic tests in a flow measurement system, referencing a loop diagram (e.g. from the {\it Realistic Instrumentation Diagrams} practice worksheet)

%INSTRUCTOR \noindent {\bf Problem-solving intro activity:} practice active reading and summarizing techniques referenced in Question 0 by summarizing the optical flowmeters subsection in the textbook.  Have students read this subsection of the textbook, then explain it to each other in their own terms, then share those explanations with the class at large.  After that, have students reflect on the similarities and differences between the different optical flowmeter technologies and other flowmeters they've studied thusfar.  Show your students some of the questions you may ask them (in the instructor's version of Question 0) in order to prompt them to apply these techniques to their study.

\vskip 2pt \noindent {\bf Emne for leksjonen:} Magnetiske og ultralyd strømningsmålere 

\vskip 2pt \noindent Oppgave 101-120 \underbar{besvar oppgave 101-108} in preparation for discussion (remainder for practice)

\vskip 10pt



%%%%%%%%%%%%%%%
\filbreak
\hrule \vskip 5pt
\noindent \underbar{Leksjon 7}

\vskip 5pt

%INSTRUCTOR \noindent {\bf Problem-solving intro activity:} brainstorm flow measusurement technologies suitable for refinery flare gas (file {\tt i00977})

%INSTRUCTOR \noindent {\bf Problem-solving intro activity:} brainstorm flow measusurement technologies suitable for natural gas (file {\tt i00978})

\vskip 2pt \noindent {\bf Emna for leksjonen:} True mass flowmeters, weirs and flumes

\vskip 2pt \noindent Questions 121-140; \underbar{answer questions 121-129} in preparation for discussion (remainder for practice)

\vskip 10pt




%%%%%%%%%%%%%%%
\filbreak
\hrule \vskip 5pt
\noindent \underbar{Leksjon 8}

\vskip 5pt

%INSTRUCTOR \noindent {\bf Problem-solving intro activity:} enthalpy calculation problem using a steam table (e.g. required steam mass flow rate for a heat exchanger)

\vskip 2pt \noindent {\bf Emne for leksjonen:} Guest presentation on Flow / Review for exam

\vskip 2pt \noindent Oppgave 141-160 ; \underbar{besvar oppgave 141-149} in preparation for discussion (remainder for practice)

\vskip 5pt

\noindent Feedback questions {\it (81 through 90)} are optional and may be submitted for review at the end of the day

\vskip 10pt
