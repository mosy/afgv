\centerline{PLS - Kombinatoriske oppgaver}  \bigskip

Kompetansemål:
\begin{itemize}[noitemsep]

	\item planlegge, programmere, montere og idriftsette programmerbare styresystemer
	\item endre og tilpasse skjermbilder for grensesnitt mellom menneske og maskin
	\item anvende ulike elektroniske kommunikasjonssystemer i automatiserte anlegg
\end{itemize}
	Læringsmål
	\begin{itemize}[noitemsep]
		\item Kunne løse kombinatoriske styringer med PLS programmering
	\end{itemize}

	Forkunnskaper

	\begin{itemize}[noitemsep]
		\item 

	\end{itemize}
\textbf{Teori}\\\\
Øvingsoppgaver til leksjon - følger neste side\\\\
Innlevering til leksjon - Det er ingen innlevering til leksjonen. 


Under disse oppgavene skal du ha fokus på følgende regler fra coding
guidelines:

4.2 All elements shall be commented

4.1 Comment shall describe the intention of the code

5.3 All variables shall be initialized before being used. 

5.13 Physical outputs shall be written once per PLC cycle

5.8. Floating point comparison shall not be equality or inequality

5.9. Time and physical measures comparison shall not be equality or
inequality

\begin{tabular}{|c|c|c|c|}
\hline 
Tilkoblet utstyr & IO på PLS & Variabel & Beskrivelse av tilkoblet utstyr\tabularnewline
\hline 
\hline 
S1 & DI1 & Start & Start av motor\tabularnewline
\hline 
S2 & DI2 & Stopp & Stopp av motor\tabularnewline
\hline 
F4 & DI3 & Motorvern & Inngang for motorvern\tabularnewline
\hline 
Q1 & DO1 & Motor & Utgang for styring av motor\tabularnewline
\hline 
H1 & DO2 & Varsellys & Varsel for motorvern utløst. \tabularnewline
\hline 
\end{tabular}
\bigskip 
\hrule
\vfil \eject

\bigskip 
 
\hrule

\vfil \eject

