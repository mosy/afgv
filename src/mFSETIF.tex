\documentclass{exam}
\usepackage[normalem]{ulem}
\usepackage{currfile}
\begin{document}
\title{FSE kurs -  Bestått/Ikke bestått}
\author{Faglærer: Fred-Olav Mosdal \\}
\maketitle
\begin{center}
Filename: \currfile
\end{center}

\vskip 5cm 
\Huge
Navn \uline{\hfill}
\vskip 2cm 
Underskrift lærer \uline{\hfill}
\vskip 0.5cm
\normalsize
\newpage
\begin{questions}
        
	\question Hvor ofte må  en ta FSE kurs? 
		\begin{oneparcheckboxes}
			\choice hver 6. mnd
			\choice hver 9. mnd
			\choice hver 12. mnd
			\choice hver 15. mnd
		\end{oneparcheckboxes}

	\question Hvem må ta FSE?
		\begin{oneparcheckboxes}
			\choice alle som bruker strøm
			\choice alle elektrikkere
			\choice alle som jobber på eller nær ved elektriske anlegg
			\choice bare mekanikkere som skal jobbe med motorer
		\end{oneparcheckboxes}

	\question Du arbeider med lavspenningsanlegg. Spenningsnivået er 12 V. Hva er største faren i et slikt anlegg?
	\begin{oneparcheckboxes}
		\choice Strømgjennomgan
		\choice Begge er farlige
		\choice Kortslutning
	\end{oneparcheckboxes}

	\question Hvor mange trykk/pust bør benyttes ved hjerte- og lungeredning?
		\begin{oneparcheckboxes}
			\choice 30 trykk og 1 innblåsning
			\choice 30 trykk og 2 innblåsninger
			\choice 15 trykk og 2 innblåsninger
		\end{oneparcheckboxes}


	\question Hvilken frekvens skal du trykke i ved hjertemassasje?
		\begin{oneparcheckboxes}
			\choice 100 ganger per minutt
			\choice 50 ganger per minutt
			\choice 80 ganger per minutt
		\end{oneparcheckboxes}
		
	\question Hvor mange sikkerhetsbarrierer skal etableres ved arbeid på lavspenningsanlgg?
		\begin{oneparcheckboxes}
			\choice Minst 3
			\choice Minst 2
			\choice Minst 1
		\end{oneparcheckboxes}
		
	\question Hva er hovedårsaken til ulykker på lavspenningsnett?
		\begin{oneparcheckboxes}
			\choice rutinearbeid
			\choice brudd på FSE
			\choice brudd på tekniske forskrifter
		\end{oneparcheckboxes}

	\question Hvem er ansvarlig for etterlevelse av forskriften?
		\begin{oneparcheckboxes}
			\choice Montøren
			\choice Brukeren av anlegget
			\choice Eier av virksomheten som utfører jobben
		\end{oneparcheckboxes}
		
	\question Hvilken risikoavstand anbefaler FSE for 690 VAC
		\begin{oneparcheckboxes}
			\choice 1,2 m
			\choice ingen bestemt avstand
			\choice 0,5 m
		\end{oneparcheckboxes}
		
	\question Hvilke ulykker skal meldes?
		\begin{oneparcheckboxes}
			\choice Personskader
			\choice Nesten ulykker
			\choice Alle ulykker, også tilløp til ulykker
		\end{oneparcheckboxes}
		
	\question Ved strømulykke og bevistløs person. Hva er det første du gjøre etter at du eventuelt løsnet en person fra nettet?
		\begin{oneparcheckboxes}
			\choice starte HLR
			\choice sjekke puls
			\choice lage støy/se om han reagerer
		\end{oneparcheckboxes}
		
	\question Hva menes med strømgjennomgang
		\begin{oneparcheckboxes}
			\choice En person som har fått strøm fra finger til finger
			\choice En person som har fått strøm fra hånd til hånd
			\choice En person som har fått strøm fra fot til fot
			\choice 
		\end{oneparcheckboxes}
		
	\question Hvor kaldt skal vann være når en skal kjøle ned en brannskade?
		\begin{oneparcheckboxes}
			\choice så kaldt som springvannet kan bli
			\choice kroppstemperert
			\choice 15-20°C
		\end{oneparcheckboxes}


\end{questions}
\end{document}
