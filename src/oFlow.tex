% !TEX root = /home/fred-olav/afgv/src/preamble.tex
\centerline{\bf Strømningsmålling}  \bigskip

Kompetansemål:
\begin{itemize}[noitemsep]

	\item montere, konfigurere, kalibrere og idriftsettelse digitale og analoge målesystemer
	\item måle fysiske størrelser i automatiserte anlegg
\end{itemize}
% til neste år må rangeability med
%	Forkunnskaper
\vskip 2cm
\begin{comment}
\noindent \underbar{Del 1}

\vskip 5pt

%INSTRUCTOR \noindent {\bf Problem-solving intro activity:} review INST241\_x1 exam.

\vskip 2pt \noindent {\bf Emne for lekesjonen:} Teknologier for flowmåling

\vskip 2pt \noindent Oppgave 1 - 20; \underbar{besvar oppgave 1-10} som forberedelse til leksjon %(remainder for practice)

\vskip 10pt

	Læringsmål:
	Målet for leksjonen er å gi en oversikt ovre ulike måleprinsipper som brukes for å måle strømning. 
	\begin{itemize}[noitemsep]
		\item Kunne navngi og beskrive virkemåte for ulike måleelementer som brukes i trykkbaserte strømningsmålere.% Venturirør, måleblende, pitotrør, annubar
		\item Kunne navngi og beskrive virkemåte for ulike måleprinsipper for hastighetsbaserte strømningsmålere.
		\item Kunne forklare måleprinsippet til en Coriolis strømningsmåler
		\item Kunne forklare måleprinsippet til fortrengingsmålere
		\item Kunne forklare måleprinsippet til termiske strømningsmålere. 
	\end{itemize}


%%%%%%%%%%%%%%%
\filbreak
\hrule \vskip 5pt
\noindent \underbar{Del 2}

\vskip 5pt

%INSTRUCTOR \noindent {\bf Problem-solving intro activity:} explore the ``curved arrow'' notation for voltage in DC circuits shown in the ``Electrical Sources and Loads'' section of the ``DC Electricity'' chapter of the LIII textbook, commenting on how this notation is analagous to force and displacement, helping to explain positive and negative quantities of mechanical work.

\vskip 2pt \noindent {\bf Emne for leksjonen:} Fluid dynamikk

\vskip 2pt \noindent Oppgave 21 - 40; \underbar{besvar oppgave 21-30} som forberedelse til leksjonen%(remainder for practice)

\vskip 10pt

	Målet for leksjonen er å bygge en intuisjon for hvordan fluider strømer i rør ved å regne på ulike eksempler. 

	Læringsmål:
	\begin{itemize}[noitemsep]
		\item Kunne forklar hva viskositet er og hvilken enhet absolutt viskositet måles i 
		\item Kunne forklare hva Reynolds nummer og regne ut dette for strømning i rør.
		\item Kunne forklare forskjellen på turbulent og laminær strømning
		\item Kunne forklare og bruke Law of Continuity 
		\item Kunne forklare og bruke Bernoullis formel
	\end{itemize}


%%%%%%%%%%%%%%%
\filbreak
\hrule \vskip 5pt
\noindent \underbar{Del 3}

\vskip 5pt

%INSTRUCTOR \noindent {\bf Problem-solving intro activity:} apply the critical reading strategy suggested in Question 0 where readers are encouraged to work through mathematical exercises.  A specific example of this would be to verify the square-root scales of indicator gauges shown in the textbook using a calculator, correlating equivalent values shown on the linear versus square-root scales.

\vskip 2pt \noindent {\bf Emne for leksjonen:} Trykkbaserte strømningsmålere

\vskip 2pt \noindent Oppgave 41 - 60; \underbar{besvar oppgave 41-50} som forberedelse til leksjonen%(remainder for practice)

\vskip 10pt


	Målet for leksjonen er å gi en oversikt over trykkbaserte strømningsmålere, hvordan disse installeres og at disse krever kvadratrotuttrekker. 

	Læringsmål:
	\begin{itemize}[noitemsep]
		\item Kunne forklare forskjellen på volumentrisk strømning og massestrømning.
		\item Kunne forklare hva som menes med kvadratrotuttrekker og regne med signalstyre inn og ut av en kvadratrotuttrekker.
		\item Kunne beskrive riktig installasjon av trykkbaserte stømningsmålere.
		\item Kunne regne på oppgaver med strømningn i trykkbaserte strømningsmålere. 
	\end{itemize}


%%%%%%%%%%%%%%%
\filbreak
\hrule \vskip 5pt

\noindent \underbar{Del 4}

\vskip 5pt

%%%%%INSTRUCTOR \noindent {\bf Problem-solving intro activity:} Research equipment manuals to sketch a complete circuit connecting a loop controller to either a 4-20 mA transmitter or a 4-20 mA final control element ({\tt i03773})

%%%%%INSTRUCTOR \noindent {\bf Problem-solving intro activity:} identifying possible ways in which an orifice-based flowmeter can give false readings.  Refer to the P\&ID in {\tt i03490} for examples, such as nitrogen flowmeter FT-29 or steam flowmeter FT-28.

\vskip 2pt \noindent {\bf Emne for leksjonen:} Hastighetsbasert strømningsmålere

\vskip 2pt \noindent Oppgave 61 - 90; \underbar{besvar oppgave 61-70} som forberedelse til leksjonen%(remainder for practice)

%\vskip 5pt

Målet for leksjonen er å gi en oversikt over måleprinsippene for hastighetsbaserte strømningsmålere.

	Læringsmål:
	\begin{itemize}[noitemsep]
		\item Kunne installere og regne ut k-verdi for turbin strømningsmåler
		\item Kunne installere og renge ut k-verdi for vortex strømningsmålere. 
		\item Kunne installere magnetiske strømningsmålere
		\item Kunne installere ultralyd strømningsmålere. 
	\end{itemize}

%\noindent Feedback questions {\it (81 through 90)} are optional and may be submitted for review at the end of the day
%
%\vskip 10pt

\filbreak
\hrule \vskip 5pt
\noindent \underbar{Del 5} 

\vskip 5pt

%INSTRUCTOR \noindent {\bf Problem-solving intro activity:} identifying possible/impossible faults in a flow measurement loop using either a turbine, vortex, or positive displacement flowmeter, referencing a P\&ID (e.g. from the {\it Realistic Instrumentation Diagrams} practice worksheet)

\vskip 2pt \noindent {\bf Emne for leksjonen:} Andre strømningsmålere. 

\vskip 2pt \noindent Oppgave 91-120; \underbar{besvar oppgave 91-99} som forberedelse til laksjonen (remainder for practice)

\vskip 10pt


Målet for leksjonen er å gi en oversikt over måleprinsippene for ulike strømningsmålere som ikke kommer inn under de andre kateboriene.

	Læringsmål:
	\begin{itemize}[noitemsep]
		\item  Kunne installere coriolis strømningsmålere. 
	\end{itemize}
	\end{comment}

%%%%%%%%%%%%%%%
