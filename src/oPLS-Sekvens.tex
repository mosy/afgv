% Start preamble
\documentclass[12pt,a4paper]{article}
\usepackage{geometry}
 \geometry{
 a4paper,
 total={170mm,257mm},
 left=20mm,
 top=20mm,
 }
\usepackage{currfile}
\usepackage[utf8]{inputenc}
\usepackage[T1]{fontenc}
\usepackage[pdftex]{graphicx}
\graphicspath{{./}}
\usepackage{enumitem}
\usepackage{pdfpages}
\usepackage{hyperref}
\usepackage{tikz}
\usepackage{attachfile}
\usepackage{epstopdf}
\usepackage{array}
\usepackage{multirow}
\usepackage{multicol}
\usepackage{float}
%\usepackage[table]{xcolor,colorbl}
\setlength{\textwidth}{16cm}
\setlength{\oddsidemargin}{-0.5cm}
\setlength{\evensidemargin}{-0.5cm}
%\setlenght{\headsep}{0cm}
\setlength\parindent{0pt}
%\setlength{\extrarowheight}{3pt}
\usepackage{listings}
%\usepackage{xcolor}

\input{arduinoLanguage.tex}
%%%%%% Counting oppgaves %%%%%%
 \newcount\questnum \questnum=0
 \def\oppgave{
            \advance\questnum by 1
            \ifnum \questnum > 0
	    \vskip 1cm
                 \vskip 10pt
                 \leftline{Oppgave \the\questnum}
                 \hrule
                 \vskip 3pt \fi}


% End preamble

\begin{document}
\title{PLS programmering - Sekvensoppgaver}
\author{Faglærer: Fred-Olav Mosdal 90507684\\}
\maketitle



\oppgave{}%1
\vskip 2.5pt 

Hva er en prosess? Forklar med et praktisk eksempel. 

\vskip 2.5pt 



\oppgave{}%1
\vskip 2.5pt 
Start/Stopp sekvens med forsinkelse:
Design en SFC for å starte en motor når en startknapp trykkes, vent i 5 sekunder og deretter stoppe motoren når en stoppknapp trykkes.

\oppgave{}%1
\vskip 2.5pt 
Fotgjengerovergang med trafikklys og fotgjengerknapp:
Lag en SFC for å styre et trafikklys og en fotgjengerovergang. Når fotgjengerknappen trykkes, skal trafikklyset endre seg fra grønt til gult, deretter rødt, og til slutt, gi fotgjengere grønt lys for å krysse. Etter en viss tid, gå tilbake til normal drift.


\oppgave{}%1
\vskip 2.5pt 

Start/Stopp sekvens med forsinkelse:
Design en SFC for å starte en motor når en startknapp trykkes, vent i 5 sekunder og deretter stoppe motoren når en stoppknapp trykkes.

\oppgave{}%1
\vskip 2.5pt 
Fotgjengerovergang med trafikklys og fotgjengerknapp:
Lag en SFC for å styre et trafikklys og en fotgjengerovergang. Når fotgjengerknappen trykkes, skal trafikklyset endre seg fra grønt til gult, deretter rødt, og til slutt, gi fotgjengere grønt lys for å krysse. Etter en viss tid, gå tilbake til normal drift.

\vskip 2.5pt 
Så i normal drift, vil trafikklysene for biler vise grønt lys, og fotgjengerlysene vil vise rødt lys. Når fotgjengerknappen trykkes, vil trafikklysene for biler endre seg fra grønt til gult, og deretter til rødt. Samtidig vil fotgjengerlysene endre seg fra rødt til grønt, slik at fotgjengerne kan krysse. Etter en viss tid med grønt lys for fotgjengerne, vil fotgjengerlyset skifte tilbake til rødt, og trafikklysene for biler vil gå tilbake til grønt, gjenopprette normal drift.

\oppgave{}%1
\vskip 2.5pt 



Kjede av transportbånd med lastesensorer:
Design en SFC for å styre en kjede av tre transportbånd. Hvert transportbånd skal starte når lastesensoren på det forrige transportbåndet detekterer at objektet har forlatt transportbåndet. Implementer en tidsforsinkelse mellom transportbåndene for å forhindre kollisjoner.

\oppgave{}%1
\vskip 2.5pt 

Flervalgs sekvens med tre valgmuligheter:
Design en SFC som lar brukeren velge mellom tre forskjellige sekvenser ved å trykke på knapp A, B eller C. Sekvens A styrer en motor i 10 sekunder, sekvens B styrer en ventil i 20 sekunder og sekvens C styrer både motoren og ventilen i 30 sekunder.

\oppgave{}%1
\vskip 2.5pt 
Batchprosess med tre trinn og rapportering:
Lag en SFC for å styre en batchprosess som innebærer oppvarming av ingredienser til 80 grader Celsius, blanding i 5 minutter og kjøling til 25 grader Celsius. Implementer også en rapporteringsfunksjon som sender en melding eller lyser en indikatorlampe når prosessen er fullført.

\oppgave{}%1
\vskip 2.5pt 
Pumpestyring med overlappende drift:
Design en SFC for å kontrollere to pumper som arbeider i tandem for å holde en væskenivå innenfor et akseptabelt område. Pumpene skal jobbe sammen med en overlapp på 5 sekunder ved oppstart og avslutning for å unngå trykkfall.

\oppgave{}%1
\vskip 2.5pt 
Sikkerhetsprosedyrer med nødstopp og feilkoder:
Design en SFC som integrerer sikkerhetsprosedyrer som nødstopp og feilhåndtering. Ved å trykke på nødstoppknappen, skal systemet stoppe umiddelbart og vise en feilkode på en skjerm eller en indikatorlampe.

\oppgave{}%1
\vskip 2.5pt 
Temperaturkontroll med PID og alarmfunksjon:
Lag en SFC for å styre temperaturen i et system, ved å bruke PID-regulering for å kontrollere et varmeelement. Implementer også en alarmfunksjon som varsler operatøren når temperaturen går utenfor det akseptable området.

\oppgave{}%1
\vskip 2.5pt 
Oppskaleringsprosess med sanntidskontroll:
Design en SFC som oppskalerer en prosess fra manuell til automat

\oppgave{}%1
\vskip 2.5pt 
Kjede av konveier med lastesensorer:
Design en SFC for å styre en kjede av tre konveier. Hver konveier skal starte når lastesensoren på den forrige konveier detekterer at objektet har forlatt konveieren. Implementer en tidsforsinkelse mellom konveier for å forhindre kollisjoner.

\oppgave{}%1
\vskip 2.5pt 
Flervalgs sekvens med tre valgmuligheter:
Design en SFC som lar brukeren velge mellom tre forskjellige sekvenser ved å trykke på knapp A, B eller C. Sekvens A styrer en motor i 10 sekunder, sekvens B styrer en ventil i 20 sekunder og sekvens C styrer både motoren og ventilen i 30 sekunder.

\oppgave{}%1
\vskip 2.5pt 
Batchprosess med tre trinn og rapportering:
Lag en SFC for å styre en batchprosess som innebærer oppvarming av ingredienser til 80 grader Celsius, blanding i 5 minutter og kjøling til 25 grader Celsius. Implementer også en rapporteringsfunksjon som sender en melding eller lyser en indikatorlampe når prosessen er fullført.

\oppgave{}%1
\vskip 2.5pt 
Pumpestyring med overlappende drift:
Design en SFC for å kontrollere to pumper som arbeider i tandem for å holde en væskenivå innenfor et akseptabelt område. Pumpene skal jobbe sammen med en overlapp på 5 sekunder ved oppstart og avslutning for å unngå trykkfall.

\oppgave{}%1
\vskip 2.5pt 
Sikkerhetsprosedyrer med nødstopp og feilkoder:
Design en SFC som integrerer sikkerhetsprosedyrer som nødstopp og feilhåndtering. Ved å trykke på nødstoppknappen, skal systemet stoppe umiddelbart og vise en feilkode på en skjerm eller en indikatorlampe.

\oppgave{}%1
\vskip 2.5pt 
Temperaturkontroll med PID og alarmfunksjon:
Lag en SFC for å styre temperaturen i et system, ved å bruke PID-regulering for å kontrollere et varmeelement. Implementer også en alarmfunksjon som varsler operatøren når temperaturen går utenfor det akseptable området.

\oppgave{}%1
\vskip 2.5pt 
Oppskaleringsprosess med sanntidskontroll:
Design en SFC som oppskalerer en prosess fra manuell til automat

\oppgave{}%1
\vskip 2.5pt
Parkeringsplass styringssystem

\vskip 2.5pt
Lag en SFC for å styre en parkeringsplass med kapasitet for 20 biler. Systemet skal holde oversikt over antall biler som kommer inn og går ut av parkeringsplassen og styre tilgangen basert på tilgjengelige parkeringsplasser. Systemet skal inneholde følgende funksjoner:

\vskip 2.5pt
Inngangsport: En bil detekteres av en sensor ved inngangsporten. Hvis det er ledige parkeringsplasser, åpnes porten og en bil kan kjøre inn. Hvis parkeringsplassen er full, skal porten forbli lukket, og en "Full" indikatorlampe skal lyse.

\vskip 2.5pt
Utgangsport: En bil detekteres av en sensor ved utgangsporten. Når en bil forlater parkeringsplassen, skal porten åpnes, og antall ledige parkeringsplasser skal øke med én. Når bilen har forlatt parkeringsområdet, skal porten lukkes igjen.

\vskip 2.5pt
Tellefunksjon: Systemet skal holde oversikt over antall biler inne på parkeringsplassen ved å inkrementere eller dekrementere en teller hver gang en bil kjører inn eller ut.

\vskip 2.5pt
Skilt med antall ledige plasser: Et skilt ved inngangen til parkeringsplassen skal vise antall ledige parkeringsplasser i sanntid.

\vskip 2.5pt
Feilhåndtering: Implementer feilhåndtering for tilfeller der sensorene ikke fungerer som forventet eller hvis en bil blokkerer porten.

\vskip 2.5pt
Design SFC-diagrammet og beskriv trinnene, overgangene og aksjonene som kreves for å implementere dette parkeringsplass styringssystemet.

\vskip 2.5pt

\end{document}

