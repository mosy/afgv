% !TEX root = /home/fred-olav/afgv/src/preamble.tex
\centerline{\bf Grunnleggende PLS for 3AUA}  \bigskip

Kompetansemål:
\begin{itemize}[noitemsep]

	\item planlegge, utføre, vurdere kvalitet, sluttkontrollere og dokumentere arbeidet
	\item planlegge, programmere, montere og idriftsette programmerbare styresystemer
	\item endre og tilpasse skjermbilder for grensesnitt mellom menneske og maskin
	\item anvende ulike elektroniske kommunikasjonssystemer i automatiserte anlegg
	\item vurdere datasikkerhet i automatiserte anlegg
	\item feilsøke og rette feil i automatiserte anlegg
	\item bruke gjeldende regelverk og normer for elektriske installasjoner på maskiner
	\item bruke gjeldende regelverk og normer for installasjon av elektroniske kommunikasjonssystemer
	\item dokumentere egen opplæring i automatiseringssystemer
\end{itemize}
	Læringsmål
	\begin{itemize}[noitemsep]
		\item Kunne forklare hvordan inngangskretsen for en digital inngang (DI) virker. 
		\item Kunne forklare hvordan utgangskretsen for en digital utgang (DO) virker. 
	\end{itemize}

	Forkunnskaper

	\begin{itemize}[noitemsep]
		\item Elektroteknikk seriekoblinger
		\item Vite hvordan en optokobler virker
		\item Vite hvordan nærhetsbrytere virker
		\item Vite hvordan en transistor virker
		\item Vite hvordan en diode virker

	\end{itemize}
\textbf{Teori}\\\\
afgv.pdf - Programmerbare Logiske Styringer - Inngangs- og utgangs tilkoblinger (IO-er)\\\\
Øvingsoppgaver til leksjon - følger neste side\\\\
Innlevering til leksjon - Det er ingen innlevering til leksjonen. 
\bigskip 
\hrule
\vfil \eject
