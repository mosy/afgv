% Start preamble
\documentclass[12pt,a4paper]{article}
\usepackage{geometry}
 \geometry{
 a4paper,
 total={170mm,257mm},
 left=20mm,
 top=20mm,
 }
\usepackage{currfile}
\usepackage[utf8]{inputenc}
\usepackage[T1]{fontenc}
\usepackage[pdftex]{graphicx}
\graphicspath{{./}}
\usepackage{enumitem}
\usepackage{pdfpages}
\usepackage{hyperref}
\usepackage{tikz}
\usepackage{attachfile}
\usepackage{epstopdf}
\usepackage{array}
\usepackage{multirow}
\usepackage{multicol}
\usepackage{float}
%\usepackage[table]{xcolor,colorbl}
\setlength{\textwidth}{16cm}
\setlength{\oddsidemargin}{-0.5cm}
\setlength{\evensidemargin}{-0.5cm}
%\setlenght{\headsep}{0cm}
\setlength\parindent{0pt}
%\setlength{\extrarowheight}{3pt}
\usepackage{listings}
%\usepackage{xcolor}
\input{arduinoLanguage.tex}
%%%%%% Counting oppgaves %%%%%%
 \newcount\questnum \questnum=0
 \def\oppgave{
		}
% End preamble
%$$\includegraphics[width=0.4\textwidth]{../output/nogpl/trukkitank.png}$$
\begin{document}
\title{3AUA Person og Maksinsikkerhet}
\author{Faglærer: Fred-Olav Mosdal 90507684\\}
\maketitle
\begin{enumerate}
	\item Hvilken forskrift pålegger bedrifter å kartlegge og dokumentere farer?
	\item Hva menes med en sikkerhetsbariære?
	\item Hva er en SJA og hva menes med SJA1 og SJA2
	\item Hvilke kostnader har en ulykke?
	\item Hva er en HMS-plan?
	\item Hva er et løsmiddel og hvordan kan det skade kroppen?
	\item ved hvilken høyde over bakken, er det aktuelt å gjøre tiltak?
	\item Hvordan sikres verktøy verktøy i ved arbeid i høyden?
	\item Hvordan skal maskiner og sikkerhetkomponenter konstrueres?
	\item Hvilke lover, forskrifter og normer gjelder for maskinsikkerhet?
	\item Forklar hvordan prosessen for risikoverdering av en maskin er. 
	\item Hva er en sikkerhetsbryter med interlock og hva er den byddet opp av?
\end{enumerate}		
Dette er oppgaver fra chatGPT gjøres på egen risiko. 
\begin{enumerate}
	\item Gitt et diagram over en nødstoppkrets for en maskin, identifiser og merk de viktigste komponentene, som nødstoppbryter, kontaktorer, releer og styreenhet.
	\item En maskin slutter plutselig å virke, og det er mistanke om at det kan være et problem med nødstoppkretsen. Beskriv en systematisk tilnærming til feilsøking for å identifisere og reparere problemet.
	\item To maskiner er plassert ved siden av hverandre og skal kobles sammen med en felles nødstoppkrets. Beskriv hvordan dette kan gjøres og hvilke hensyn som må tas for å sikre at begge maskinene stopper ved aktivering av nødstopp.
	\item Når man skal velge en nødstoppbryter til en maskin, er det flere faktorer man må ta hensyn til. Diskuter hvilke faktorer som er viktige å vurdere, og hvordan disse påvirker valget av nødstoppbryter.
	\item Forklar hovedprinsippene og formålet med EN ISO 13849-1 standarden. Beskriv hvordan denne standarden bidrar til å forbedre sikkerheten for maskiner og deres brukere.
	\item Beskriv de forskjellige kategoriene (B, 1, 2, 3, og 4) og ytelsesnivåene (PL a-e) i EN ISO 13849-1 standarden. Forklar hvordan de ulike kategoriene og ytelsesnivåene relaterer seg til hverandre og hvilke faktorer som påvirker valget av kategori og ytelsesnivå for et gitt maskinsikkerhetssystem.
	\item Gitt et eksempel på en maskinsikkerhetskrets, beregn ytelsesnivået (PL) for kretsen i henhold til EN ISO 13849-1 standarden. Forklar hvordan ulike faktorer, som komponentenes levetid og sannsynligheten for farlige feil, påvirker beregningen av PL.
	\item Beskriv prosessen for validering og dokumentasjon av maskinsikkerhetssystemer i henhold til EN ISO 13849-1 standarden. Forklar hvilke tester og verifikasjoner som må utføres, samt hvilken type dokumentasjon som kreves for å bevise at sikkerhetssystemet oppfyller standardens krav.
	\item Design og koble opp en sikkerhetskrets som inkluderer en nødstoppbryter, en sikkerhetsdørlås, et rele for sikkerhetsstopp og en manuell reset-knapp for å kontrollere en maskin. Beskriv hvordan kretsen fungerer og hvordan den oppfyller kravene for maskinsikkerhet.
	\item Mål:
	\item Studentene skal kunne designe og koble opp en sikkerhetskrets som inkluderer nødstopp, sikkerhetsdørlås, sikkerhetsstopp-rele og en manuell reset-knapp. De skal også kunne forklare hvordan kretsen fungerer og hvordan den oppfyller kravene for maskinsikkerhet.
	\item 
	\item Trinn:
	\item 
	\item Identifiser hovedkomponentene i sikkerhetskretsen: nødstoppbryter, sikkerhetsdørlås, sikkerhetsstopp-rele og manuell reset-knapp.
	\item 
	\item Koble nødstoppbryteren og sikkerhetsdørlåsen i serie, slik at begge komponentene må være aktivert for å tillate maskinen å starte eller fortsette å kjøre. Når enten nødstoppbryteren er aktivert eller sikkerhetsdøren er åpen, skal kretsen brytes og maskinen stoppe.
	\item 
	\item Koble den seriekoblede nødstoppbryteren og sikkerhetsdørlåsen til inngangen på sikkerhetsstopp-releet. Releet skal ha tvungen veiledning og innebygde overvåkingsfunksjoner for å sikre at maskinen stopper pålitelig når det kreves.
	\item 
	\item Koble utgangen fra sikkerhetsstopp-releet til maskinens styresystem eller motorkontroller for å starte og stoppe maskinen i henhold til sikkerhetskretsens status.
	\item 
	\item Koble den manuelle reset-knappen til sikkerhetsstopp-releet. Når både nødstoppbryteren er tilbakestilt (deaktivert) og sikkerhetsdøren er lukket og låst, skal operatøren kunne trykke på reset-knappen for å reaktivere sikkerhetskretsen og tillate maskinen å starte.
	\item 
	\item Beskriv hvordan kretsen fungerer og hvordan den oppfyller kravene for maskinsikkerhet i henhold til relevante standarder, som EN ISO 13849-1.
	\item 
	\item Ved å utføre disse trinnene, skal studentene kunne designe og koble opp en sikkerhetskrets med manuell reset og forklare hvordan den fungerer og oppfyller kravene for maskinsikkerhet.
\end{enumerate}





































%I alle svarene må dere i tillegg til deres forklaring ha en link til
%paragraf som dere har funnet svaret i
%\begin{enumerate}
%\item Hva er en maskin?
%	\vskip 5pt
%\begin{tikzpicture}
%	\draw[step=0.5cm,gray!20,very thin]  grid (16,7) ;
%\end{tikzpicture}
%\item Når en en samling maskiner ansett for å være en maskin?
%\item Hva må produsentan av en maskin sikre før en settes i omsettning?
%\item På hvilken måte skal en maskin som ikke omfattes av vedlegg IV samsvarsvurderes. 
%\item Hvilke forpliktelser påhvilder en som monteres maskiner?
%\item I den gjentagende prosessen med risikovurdering og valg av risikoreduserende
%tiltak skal produsenten?
%\item Hva menes med?
%\begin{enumerate}
%\item Fare
%\item Risiko
%\item Vern
%\item feil bruk som med rimelighet kan forventes. 
%\end{enumerate}
%\item Ved integrering av sikkerhet skal følgende prinsipper brukes ( svar
%i prioritert rekkefølge)
%\item Hvilke stopptyper beskrives av forskriften?
%\item Hvilke krav gjelder for ulike driftsmåter?
%\item Hvilke krav gjelder for styresystemet ved svikt i energitilførsel
%for en maskin?
%\item Hvilke krav gjelder for nødstopp
%\item Hva er formålet med maskinforskriften, og hvilke typer maskiner er omfattet av den?
%\item Forklar hva en risikovurdering innebærer i henhold til maskinforskriften, og hvorfor det er viktig.
%\item Hvilke krav stilles til dokumentasjonen som produsenten må levere sammen med maskinen i henhold til maskinforskriften?
%\item Hvordan skal CE-merkingen plasseres på maskinen, og hva er hensikten med merkingen?
%\item Hva er forskjellen mellom "produsent" og "importør" i henhold til maskinforskriften, og hvilke forpliktelser har de?
%\item Forklar hva som menes med "tilpasning til formål" i henhold til maskinforskriften, og hva som kreves for å oppfylle dette kravet.
%\item Hvordan skal det utføres en konformitetsvurdering i henhold til maskinforskriften, og hva er formålet med denne vurderingen?
%\item Hvilke krav stilles til maskiner som allerede er i bruk, men som ikke er i samsvar med maskinforskriften?
%\item Hva er forskjellen mellom "krav til teknisk dokumentasjon" og "krav til bruksanvisning" i henhold til maskinforskriften?
%\item Hva er strafferammen for å bryte maskinforskriften, og hvilke sanksjoner kan ilegges?
%\item <LeftMouse>Hva er forskjellen mellom "vesentlige krav" og "ytelseskrav" i henhold til maskinforskriften, og hvordan påvirker dette produsentens ansvar?
%\item Hva er kravene til risikoreduserende tiltak i henhold til maskinforskriften, og hvordan skal disse implementeres?
%\item Hvordan kan en produsent vurdere om en maskin er i samsvar med maskinforskriften før den markedsføres?
%\item Hva er kravene til bruk av personlig verneutstyr (PVU) i henhold til maskinforskriften, og hvilken rolle spiller PVU i risikovurderingen?
%\item Hva er kravene til dokumentasjon av endringer som gjøres på en maskin etter at den er satt i drift?
%\item Hva er kravene til utdanning og opplæring av ansatte som jobber med maskiner i henhold til maskinforskriften?
%\item Hva er kravene til teknisk dokumentasjon som produsenten må utarbeide for en maskin i henhold til maskinforskriften?
%\item Hva er kravene til bruk av maskiner som er tatt i bruk før maskinforskriften trådte i kraft?
%\item Hva er kravene til importører av maskiner i henhold til maskinforskriften, og hva er deres ansvar?
%\item Hva er kravene til revisjon og oppdatering av risikovurderinger for maskiner i henhold til maskinforskriften?
%\end{enumerate}
%Forklar hvordan et generelt måleinstrument er bygd opp.
%\oppgave{}%1
%\vskip 2.5pt 
\vskip 1cm
\oppgave{}%1
\vskip 5pt 
\vskip 1cm
\end{document}


