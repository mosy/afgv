% Start preamble
\documentclass[12pt,a4paper]{article}
\usepackage{geometry}
 \geometry{
 a4paper,
 total={170mm,257mm},
 left=20mm,
 top=20mm,
 }
\usepackage[utf8]{inputenc}
\usepackage[T1]{fontenc}
\usepackage[pdftex]{graphicx}
\graphicspath{{./}}
\usepackage{enumitem}
\usepackage{pdfpages}
\usepackage{hyperref}
\usepackage{tikz}
\usepackage{attachfile}
\usepackage{epstopdf}
\usepackage{array}
%\usepackage[table]{xcolor,colorbl}
\setlength{\textwidth}{16cm}
\setlength{\oddsidemargin}{-0.5cm}
\setlength{\evensidemargin}{-0.5cm}
%\setlenght{\headsep}{0cm}
\setlength\parindent{0pt}
%\setlength{\extrarowheight}{3pt}
\usepackage{listings}
%\usepackage{xcolor}

\input{arduinoLanguage.tex}
%%%%%% Counting oppgaves %%%%%%
 \newcount\questnum \questnum=0
 \def\oppgave{
            \advance\questnum by 1
            \ifnum \questnum > 0
                 \hrule
                 \vskip 3pt
                 \leftline{Oppgave \the\questnum}
                 \vskip 3pt \fi}
 %%%%%%%%%%%%%%%%%%%%%%
%%%%%%%%%%%%%%%%%%%%%%


%%%%%% Counting answers %%%%%%
\newcount\answnum \answnum=0
\def\svar{
           \advance\answnum by 1
           \ifnum \answnum > 0
                \hrule
                \vskip 3pt
                \leftline{Svar \the\answnum}
                \vskip 3pt \fi}
%%%%%%%%%%%%%%%%%%%%%%


%%%%%% Counting notes %%%%%%
\newcount\explnum \explnum=0
\def\notes{
           \advance\explnum by 1
           \ifnum \explnum > 0
                \hrule
                \vskip 3pt
                \leftline{Notes \the\explnum}
                \vskip 3pt \fi}
%%%%%%%%%%%%%%%%%%%%%%

% End preamble

\begin{document}
\title {YFF Prøve i Sekvensprogrammering 2020}
\author {Fred-Olav Mosdal \\\\ Leveres på mail til fred-olav.mosdal@skole.rogfk.no }
\date {03.06.2020}
\maketitle
\vfil \eject
\vfil \eject
\centerline{\bf Oppgaver}
\vskip 5pt
\oppgave{} 
% Copyright 2012, Tony R. Kuphaldt, released under the Creative Commons Attribution License (v 1.0)
% This means you may do almost anything with this work of mine, so long as you give me proper credit

\attachfile [icon=Paperclip,color=1 0 0]{sg002.smk}{     Vedlegg til å bruke i Simumatik3d}\\[0.8cm]
System beskrivelse:\\[0.051cm]

Systemet består av to transportbånd og to pneumatiske sylindere. Disse styres av h.h.v. Conv1, Conv2, Sylinder1 og Sylinder2. \\[0.3cm]

På transportbånd1 er det to sensorer NyEkse registrerer at det kommer en eske inn på båndet. EndeConv1 forteller at esken er kommet til enden av båndet. Her skal den dyttes over mot transportbånd2. Sylinder1Ute forteller at esken er kommet over. \\[0.3cm]

Før esken kommer inn på transportbånd2 må den dyttes av Sylinder2. Sylinder2Ute forteller at denne sylinderen er ute. StartConv2 forteller at det er kommet en eske på transportbånd2, som da kan starte og flytte esken. Når esken treffer EndeConv2 skal transportbåndet stoppe og esken vil fjernes etter en stund. \\[0.3cm]





Oppgave:\\
Din oppgave er å lage et program med tilhørende sekvensielt funksjonskart. For beskrevet funksjon. Hvor langt mot enden du får esken, vil påvirke karakteren. \\[0.5cm]
Vurderingskriterier:
\begin{itemize}[noitemsep]
	\item Karakteren 2: Kan opprette kommunikasjon med Simumatik3d
	\item Karakteren 3: Kan registrere en eske og forflytte denne til enden av transportbånd1
	\item Karakteren 4: Kan skubbe esken over på plattformen
	\item Karakteren 5: Kan skubbe esken over på transportbånd2 og videre til kjeglen som fjerner esken. 
	\item Karakteren 6: Programmet er oversiktlig å lettlest. Sekvensen gjentar seg gang etter gang. Helt til en trykker stopp (Denne knappen finnes ikke du må lage den selv.)
\end{itemize}

\vskip 10pt \filbreak 
\vfil \eject
\centerline{\bf Svar}
\vskip 5pt
\svar{} 
\vskip 10pt \filbreak 
\end{document}
