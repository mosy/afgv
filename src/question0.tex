
% Copyright 2015, Tony R. Kuphaldt, released under the Creative Commons Attribution License (v 1.0)
% This means you may do almost anything with this work of mine, so long as you give me proper credit

%(BEGIN_FRONTMATTER)

\hrule

\vskip 3pt

\noindent
Question 0

\vskip 10pt

\noindent
{\bf How to get the most out of academic reading:}

\item{$\bullet$} \underbar{Articulate your thoughts} as you read (i.e. ``have a conversation'' with the author).  This will develop {\it metacognition}: active supervision of your own thoughts.  Write your thoughts as you read, noting points of agreement, disagreement, confusion, epiphanies, and connections between different concepts or applications.  These notes should also document important math formulae, explaining in your own words what each formula means and the proper units of measurement used.
\vskip 5pt
\item{$\bullet$} \underbar{Outline, don't highlight!}  Writing your own summary or outline is a far more effective way to comprehend a text than simply underlining and highlighting key words.  A suggested ratio is one sentence of your own thoughts per paragraph of text read.  Note points of disagreement or confusion to explore later.
\vskip 5pt
\item{$\bullet$} \underbar{Work through all mathematical exercises} shown within the text, to ensure you understand all the steps.
\vskip 5pt
\item{$\bullet$} \underbar{Imagine explaining concepts you've just learned to someone else.}  Teaching forces you to distill concepts to their essence, thereby clarifying those concepts, revealing assumptions, and exposing misconceptions.  Your goal is to create the simplest explanation that is still technically accurate.
\vskip 5pt
\item{$\bullet$} \underbar{Write your own questions} based on what you read, as though you are a teacher preparing to test students' comprehension of the subject matter.
\medskip

\vskip 10pt

\noindent
{\bf How to effectively problem-solve and troubleshoot:}

\item{$\bullet$} \underbar{Study principles, not procedures.}  Don't be satisfied with merely knowing how to compute solutions -- learn {\it why} those solutions work.  In mathematical problem-solving this means being able to identify the practical meaning (and units of measurement) of every intermediate calculation.  In other words, {\it every step of your solution should make logical sense}.
\vskip 5pt
\item{$\bullet$} \underbar{Sketch a diagram} to help visualize the problem.  When building a real system, always prototype it on paper and analyze its function {\it before} constructing it.
\vskip 5pt
\item{$\bullet$} \underbar{Identify} what it is you need to solve, \underbar{identify} all relevant data, \underbar{identify} all units of measurement, \underbar{identify} any general principles or formulae linking the given information to the solution, and then \underbar{identify} any ``missing pieces'' to a solution.  \underbar{Annotate} all diagrams with this data.
\vskip 5pt
\item{$\bullet$} \underbar{Perform ``thought experiments''} to explore the effects of different conditions for theoretical problems.  When troubleshooting real systems, perform {\it diagnostic tests} rather than visually inspecting for faults.
\vskip 5pt
\item{$\bullet$} \underbar{Simplify the problem} and solve that simplified problem to identify strategies applicable to the original problem (e.g. change quantitative to qualitative, or visa-versa; substitute easier numerical values; eliminate confusing details; add details to eliminate unknowns; consider simple limiting cases; apply an analogy).  Often you can add or remove components in a malfunctioning system to simplify it as well and better identify the nature and location of the problem.
\vskip 5pt
\item{$\bullet$} \underbar{Work ``backward''} from a hypothetical solution to a new set of given conditions.
\medskip

\vskip 10pt

\noindent
{\bf How to create more time for study:}

\item{$\bullet$} \underbar{Kill your television and video games.}  Seriously -- these are incredible wastes of time.  Eliminate distractions (e.g. cell phone, internet, socializing) in your place and time of study.
\vskip 5pt
\item{$\bullet$} \underbar{Use your ``in between'' time productively.}  Don't leave campus for lunch.  Arrive to school early.  If you finish your assigned work early, begin studying the next day's material.  
\medskip


\vskip 10pt

\noindent
{\bf Above all, cultivate \underbar{persistence}.}  Persistent effort is necessary to master anything non-trivial.  The keys to persistence are (1) having the desire to achieve that mastery, and (2) realizing challenges are normal and not an indication of something gone wrong.  A common error is to equate {\it easy} with {\it effective}: students often believe learning should be easy if everything is done right.  The truth is that mastery never comes easy!

\vfil 

\underbar{file {\tt question0}}
\eject





%INSTRUCTOR \vskip 20pt \vbox{\hrule \hbox{\strut \vrule{} {\bf Suggestions for Socratic discussion following a reading assignment} \vrule} \hrule}

%INSTRUCTOR \medskip
%INSTRUCTOR \item{$\bullet$} Show your outline of today's reading (i.e. a summary written \underbar{in your own words} of what the text says).  ({\it Instructor: probe the students' understanding in any areas you see under-represented in their outlines.}) 
%INSTRUCTOR \item{$\bullet$} Identify which part(s) of the reading assignment make perfect sense to you.
%INSTRUCTOR \item{$\bullet$} Identify the most challenging part(s) of the reading assignment, and explain why.
%INSTRUCTOR \item{$\bullet$} Identify any part(s) of the reading assignment you found surprising, and why you found them so.
%
%INSTRUCTOR \vskip 10pt
%
%INSTRUCTOR \item{$\bullet$} Students and instructor both practice the ``Think Aloud'' active reading technique on a section of particularly challenging text.  {\it Instructor: this is where the reader reads out loud, verbalizing their interpretation of the text as they go.  Model this for your students, demonstrating your own approach.}
%INSTRUCTOR \item{$\bullet$} Summarize the essence of today's reading in one succinct paragraph.
%INSTRUCTOR \item{$\bullet$} Identify concepts in today's reading that seemed non-intuitive, and explain why that was so.
%INSTRUCTOR \item{$\bullet$} Identify over-arching themes in today's reading, especially where you see connections between what you've learned today and what you've learned previously (e.g. concepts and facts from previous courses).
%INSTRUCTOR \item{$\bullet$} Imagine explaining what you have learned today to a precocious child (i.e. a person sufficiently mature in their thinking but lacking the background knowledge you possess).
%
%INSTRUCTOR \vskip 10pt
%
%INSTRUCTOR \item{$\bullet$} Formulate your own question based on the reading, where the answer is found verbatim in the text.  Explain what makes this question a good one for students to ponder.
%INSTRUCTOR \item{$\bullet$} Formulate your own question based on the reading, where the answer requires you to combine multiple statements or concepts found in different portions of today's reading.  Explain what makes this question a good one for students to ponder (i.e. why it prompts critical thinking).
%INSTRUCTOR \item{$\bullet$} Formulate your own question based on the reading, where the answer requires you to combine facts or concepts found in today's reading with information learned at some other time.  Explain what makes this question a good one for students to ponder (i.e. why it prompts critical thinking).
%INSTRUCTOR \item{$\bullet$} Formulate your own question based on the reading, where the answer is open-ended (i.e. where there may be multiple different yet correct answers).
%INSTRUCTOR \item{$\bullet$} Devise a scientific experiment to test a principle articulated in today's reading.  {\it Challenge students to identify what result(s) would constitute \underbar{disproof} of the principle.}
%INSTRUCTOR \medskip

%
%INSTRUCTOR \vskip 20pt \vbox{\hrule \hbox{\strut \vrule{} {\bf Suggestions for Socratic discussion following a problem-solving assignment} \vrule} \hrule}

%INSTRUCTOR \medskip
%INSTRUCTOR \item{$\bullet$} Apply any one of the problem-solving strategies listed in Question 0 to one of today's assigned problems.
%INSTRUCTOR \item{$\bullet$} Identify any {\it first principles} applicable to this problem, such as Conservation Laws of physics, general principles of electric circuits, principles of algebra or calculus, etc.  ({\it Instructor note: every chapter in the LIII textbook ends with a section listing some of the fundamental principles applied in that section.})
%INSTRUCTOR \item{$\bullet$} Explain how you were able to identify relevant data to solve the problem, and show how to annotate any illustrations or schematics provided in the problem with this relevant data.
%INSTRUCTOR \item{$\bullet$} Identify the most challenging aspect of the problem, and then remove that challenging aspect from the problem so that it becomes simpler to solve.  {\it Instructor: this is often a very useful technique when someone is ``stuck'' on a problem and doesn't know how to tackle it.}
%INSTRUCTOR \item{$\bullet$} Devise your own ``thought experiment'' useful for understanding this problem.
%INSTRUCTOR \item{$\bullet$} Identify any ``limiting cases'' (i.e. altering conditions to extremes) that simplify the problem.
%INSTRUCTOR \item{$\bullet$} Identify any misconceptions you discovered while solving this problem, and explain how you were able to identify them as such.
%INSTRUCTOR \item{$\bullet$} Sketch a diagram that makes the problem easier to grasp.
%INSTRUCTOR \item{$\bullet$} Show how different mathematical formulae link together to give answers to this problem.
%INSTRUCTOR \item{$\bullet$} Identify the meaning of all intermediate calculations in your mathematical solution.  Each and every such calculated result should be identifiable in terms of both units of measurement and application to the problem at hand.
%INSTRUCTOR \item{$\bullet$} {\it Instructor: suggest a ``thought experiment'' for students to perform on the problem at hand.}
%INSTRUCTOR \medskip

%(END_FRONTMATTER)





%INDEX% Reading Apprenticeship technique, Think Aloud





