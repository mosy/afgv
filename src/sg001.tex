
%(BEGIN_QUESTION)
% Copyright 2012, Tony R. Kuphaldt, released under the Creative Commons Attribution License (v 1.0)
% This means you may do almost anything with this work of mine, so long as you give me proper credit

\attachfile [icon=Paperclip,color=1 0 0]{sg001.smk}{     Vedlegg til å bruke i Simumatik3d}\\[0.2cm]
System beskrivelse

Det er 3 transportbånd, to vanlige (1 og 3) og et som kan snu (2). Transportbånd 2 snur rundt sin vertikale akse. Hvert transportbånd styres av en DOL startet motor. Startsignalene for motorene er hhv. Motor1Fwd, Motor2Fwd og Motor3Fwd. For å slippe esker på båndet aktiveres xDropBox. Disse eskene kan slippes så for det er plass på båndet. 
På transportbånd 1 er det to sensorer. SensorDrop registrerer at det er ny eske på båndet og Sensor1 registrerer at det er en eske på enden av båndet.
\vskip 2.5pt 
 
På transportbånd 2 er 4 sensorer. Sensor2 og Sensor3 skal plassere eskene midt på og i enden av båndet. Sensor 4 markerer at transportbånd 2 er på linje med transportbånd 1 og Sensor 5 marker at transportbånd 2 er på linje med transportbånd 3. Transportbånd 2 har en motor med dreieretningsvender for snu båndet 90 grader, på denne måten kan det ta imot esker fra transportbånd 1 og levere til transportbånd 3.
\vskip 2.5pt 
  
På transportbånd 3 er det to sensorer (Sensor6 og Sensor7), disse brukes til å registrere eske på bånd og eske i enden av båndet. Når esker står i ro ved Sensor7 vil de tas vekk. 
\vskip 2.5pt 

Det er på knapper på panelet for Start, som skal starte prosessen og SFCReset som restarter sekvensen \\

Oppgave:
Din oppgave er å transportere esker igjennom systemet. Alt etter løsningen du velger å programmere, vil mengde esker igjennom systemet variere. En del av oppgaven er å få systemet så effektivt som mulig. 
Kriterier som er en del av vurderingen:
\begin{itemize}[noitemsep]
	\item Kan sette opp kommunikasjon mellom programmene. 
	\item Har med et fungerende program, transportere esker gjennom systemet
	\item Lager programmet på en slik måte at det er lett å lese for andre. (oversiktlig og gode navn på variabler). 
	\item Hvor effektivt programmet er på å få esker igjennom systemet. 
	\item Hvor energieffektivt programmet er. (går motorer lenger en de trenger?)
\end{itemize}

%\href {https://raw.githubusercontent.com/mosy/afgv/master/src/sg001.smk}{Simumatik Prosjektfil}
$$\includegraphics[width=16cm]{sg001.eps}$$
%(END_QUESTION)
%(BEGIN_ANSWER)
%(END_ANSWER)
%(BEGIN_NOTES)
Vurdering:
\begin{itemize}[noitemsep]
	\item Kan sette opp kommunikasjon med simumatik (2)
	\item Kan forflytte en eske (2)
	\item Kan forflytte en eske over (3)
	\item Kan forflytte to esker over (4)
	\item Kan få to operasjoner til å gå parallelt. (5)
	\item Kan foå tre operasjoner til å gå paralellt.(6)
\end{itemize}


%(END_NOTES)


