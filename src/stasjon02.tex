\input preamble.tex
\noindent
\section*{Stasjon 2 - Styreskap for flere maskiner}

\vskip 5pt
%beskrivelse av oppgaven
Dette arbeidsoppdraget handler om dokumentasjon. Du skal lære å kunne orientere deg i et system bestående at et styreskap med en prosess koblet til h.h.a. kabler. I oppdraget må du kunne kunne lese skjema opp mot koblet anlegg kunne identifisere feil og oppdatere dokumentasjonen med rettelser. Du må også kunne følge dokumentasjon for oppkobling mot ukjent RIO og bruke denne til å sjekke signaler (IO-sjekk)\\

\vskip 5pt 
%kompetansemål som oppgaven dekker
Kompetansemål:
\begin{itemize}[noitemsep]

	\item tegne, lese og forklare instrumenterte prosessflytskjemaer og bruke annen relevant dokumentasjon for automatiserte anlegg
\end{itemize}

%anbefalt lesning til arbeidsoppdragene
Anbefalt lesning:

\begin{enumerate}
	\item afgv.pdf/ 
\end{enumerate}

%Liste over oppdrag som skal gjøres med ruter for godkjennening

\begin{center}
\begin{tabular}{ | m{10cm} | m{1cm}| m{2cm} | } 
\hline
\multicolumn{3}{|c|}{Liste over oppgaver som skal utføres} \\
	\hline
	Oppgave	& Utført & Signatur \\ 
	\hline
	\cellcolor{green!60}(Nivå 1) Oppkobling til anlegget ved hjelp av dokumentasjon	& & \\ 
	\hline
	\cellcolor{yellow!60}(Nivå 2) Bruk av skjema til å gjøre IO-sjekk på 3 signaler (spesifiseres av lærer)
	& & \\ 
	\hline
	\cellcolor{orange!60}(Nivå 3) Oppdatering av dokumentasjon for tre IO-er
	& & \\ 
	\hline
	\cellcolor{red!60}(Nivå 4) Oppkobling med TIA portal til RIO. 
	& & \\ 
	\hline
\end{tabular}
\end{center}

%--------------------------------------
Arbeidsoppdrag kan deles inn i planlegging, gjennomføring og dokumentasjon.\\
For arbeidsoppdragene er det tenkt at dere skal bruke dette slik:\\\\
\textbf{Planlegging}\\
Gjøres før arbeidsuke starter. Da leser du gjennom anbefalt lesning, finner frem manualer og for  utstyret på stasjonen og setter deg inn i dette.\\ \\
\textbf{Gjennomføring}\\
Gjennomføres i aktuell arbeidsuke\\\\

\textbf{Dokumentasjon}\\
Gjennomføres i aktuell arbeidsuke, det vil være ulike dokumentasjonskrav til de forskjellige arbeidsoppdragene. Generelt vil det være en beskrivelse av arbeidet til lærer når oppdrag skal godkjennes. \\


% Detaljert beskrivelse av hvert arbeidsoppdrag
\newpage

\subsection*{Arbeidsoppdrag 1 -  Oppkobling til anlegget ved hjelp av dokumentasjon(nivå 1)}

I dette oppdraget skal du følge Profinet tutorial.docx som ligger i manual mappen  for stasjonen. Ved hjelp av den skal du koble det til RIO nummer 22 i skapet. 

\begin{center} \begin{tabular}{ | m{12cm} | m{1cm}| m{2cm} | } 
\hline
\multicolumn{3}{|c|}{Punkter som skal godkjennes før en går videre på neste nivå} \\
	\hline
	Oppgave	& Utført & Signatur \\ 
	\hline
Eleven kan demostrere at en utgang kan aktiveres, dette signaliseres på DO modulen med et lys& & \\ 
	\hline
\end{tabular}
\end{center}

\textbf{Vanlige feil:}
\begin{itemize}[noitemsep]
	\item 
\end{itemize}
\newpage
\subsection*{Arbeidsoppdrag 2 - Bruk av skjema til å gjøre IO-sjekk på 3 signaler (nivå 2)}

På dette oppdraget vil du få utlevert skjema for tre signaler som du skal sjekke at virker. 

\begin{center}
\begin{tabular}{ | m{12cm} | m{1cm}| m{2cm} | } 
\hline
\multicolumn{3}{|c|}{Punkter som skal godkjennes før en går videre på neste nivå} \\
	\hline
	Oppgave	& Utført & Signatur \\ 
	\hline
Eleven demonstrer at skjema stemmer for alle IO-ene& & \\ 
	\hline
Eleven demonstrer ved å vise i codesys at alle IO-ene virker. & & \\ 
	\hline
\end{tabular}
\end{center}
\textbf{Vanlige feil:}
\begin{itemize}[noitemsep]
	\item 
\end{itemize}
\newpage
\subsection*{Arbeidsoppdrag 3 - Oppdatering av dokumentasjon for tre IO-er (nivå 3)}

På dette oppdraget vil du få beskjed om å dokumentere tre nye IO-er. Det kan være at du må gjøre omkoblinger for å de skal virke. Dette er en del av arbeidet. 
\begin{center}
\begin{tabular}{ | m{8cm} | m{1cm}| m{2cm} | } 
\hline
\multicolumn{3}{|c|}{Punkter som skal godkjennes før en går videre på neste nivå} \\
	\hline
	Oppgave	& Utført & Signatur \\ 
	\hline
Eleven demonstrer at skjema stemmer for alle IO-ene& & \\ 
	\hline
Om skjema stemmer skal det sendes på mail til Fred-Olav & & \\ 
	\hline
Eleven demonstrer ved å vise i codesys at alle IO-ene virker. & & \\ 
	\hline
\end{tabular}
\end{center}
\textbf{Vanlige feil:}
\begin{itemize}[noitemsep]
	\item 
\end{itemize}
\newpage

\subsection*{Arbeidsoppdrag 4 - Oppkobling av TIA-portale mot RIO (nivå 4)}

Dette oppdraget går ut på å kunne sette seg inn i dokumentasjon på PLS software. Du må selv finne relevant dokumentasjon for å kunne koble opp en PC med TIA portal innstallert. Denne skal du konfigurere til å kommunisere med RIO-en fra tidligere oppgaver. Du skal også demonstrere at de samme IO-ene virker i dette programmet (6 stk).


\begin{center}
\begin{tabular}{ | m{8cm} | m{1cm}| m{2cm} | } 
\hline
\multicolumn{3}{|c|}{Punkter som skal godkjennes før en går videre på neste nivå} \\
	\hline
	Oppgave	& Utført & Signatur \\ 
	\hline
Eleven demonsterer i TIA portal at alle 6 IO-er virker. & & \\ 
	\hline
\end{tabular}
\end{center}
\textbf{Vanlige feil:}
\begin{itemize}[noitemsep]
	\item 
\end{itemize}
\newpage

\underbar{file stasjon02.tex}

\end{document}

