\input preamble.tex
\noindent
\section*{Stasjon 3 - Planlegging, gjennomføring  og dokumentasjon  av arbeid}

\vskip 5pt
%beskrivelse av oppgaven
I dette arbeidsoppdraget skal du planlegge, gjennomføre og dokumentere montering av et reguleringsanlegg. Da tiden til oppdraget er begrenset til 15  timer, vil det være utvalgte deler av jobben du skal gjøre. 

\vskip 5pt 
%kompetansemål som oppgaven dekker
Kompetansemål:
\begin{itemize}[noitemsep]

	\item utføre arbeid på automatiserte anlegg fagmessig, nøyaktig og i overensstemmelse med krav til helse, miljø og sikkerhet og rutiner for kvalitetssikring og internkontroll
	\item utføre risikovurdering og vurdere tiltak for ivaretakelse av person- og maskinsikkerhet
	\item vurdere hvilke regelverk og normer som gjelder for arbeidet som skal utføres og anvende dette
\end{itemize}

%anbefalt lesning til arbeidsoppdragene
Anbefalt lesning:

\begin{enumerate}
	\item ingen       
\end{enumerate}

%Liste over oppdrag som skal gjøres med ruter for godkjennening

\begin{center}
\begin{tabular}{ | m{10cm} | m{1cm}| m{2cm} | } 
\hline
\multicolumn{3}{|c|}{Liste over oppgaver som skal utføres} \\
	\hline
	Oppgave	& Utført & Signatur \\ 
	\hline
	\hline
	\cellcolor{green!60}(Nivå 1)Kobling med bakplate ute av skapet.	& & \\ 
	\hline
	\cellcolor{yellow!60}(Nivå 2)Koble reseten av anlegget. 	& & \\ 
	\hline
	\cellcolor{orange!60}(Nivå 3)Sluttkontroll	& & \\ 
	\hline
	\cellcolor{red!60}(Nivå 4)Dokumentasjon	& & \\ 
	\hline
\end{tabular}
\end{center}

%--------------------------------------
Arbeidsoppdrag kan deles inn i planlegging, gjennomføring og dokumentasjon.\\
For arbeidsoppdragene er det tenkt at dere skal bruke dette slik:\\\\
\textbf{Planlegging}\\
Gjøres før arbeidsuke starter. Da leser du gjennom anbefalt lesning, finner frem manualer og for  utstyret på stasjonen og setter deg inn i dette.\\ \\
\textbf{Gjennomføring}\\
Gjennomføres i aktuell arbeidsuke\\\\

\textbf{Dokumentasjon}\\
Gjennomføres i aktuell arbeidsuke, det vil være ulike dokumentasjonskrav til de forskjellige arbeidsoppdragene. Generelt vil det være en beskrivelse av arbeidet til lærer når oppdrag skal godkjennes. \\


% Detaljert beskrivelse av hvert arbeidsoppdrag
\newpage

\subsection*{Arbeidsoppdrag 1 -  Kobling med bakplate ute av skapet (nivå 1)}

I dette oppdraget skal du koble fra alle ledninger som er nødvendige for å få bakplaten ut av skapet. Så tar du bakplaten med til koblingsstasjonene som du har fått henvist. Her fjerner du alle ledninger utenom:
\begin{itemize}[noitemsep]
	\item Kommunikasjonskabel til frekvensomformer
\end{itemize}


\vskip 5pt 
\begin{center} \begin{tabular}{ | m{8cm} | m{1cm}| m{2cm} | } 
\hline
\multicolumn{3}{|c|}{Punkter som skal godkjennes før en går videre på neste nivå} \\
	\hline
	Oppgave	& Utført & Signatur \\ 
	\hline
Eleven viser at alle kabler og ledere er fjernet& & \\ 
	\hline
Når eleven er ferdig å koble sjekker lærer tre tilfeldige ledere& & \\ 
	\hline
\end{tabular}
\end{center}

\textbf{Vanlige feil:}
\begin{itemize}[noitemsep]
	\item 
\end{itemize}
\newpage
\subsection*{Arbeidsoppdrag 2 - Koblling av resten av anlegget (Nivå 2)}

Nå skal du montere bakplaten tilbake i skapet og koble til resten av kabler og ledninger. 

\begin{center}
\begin{tabular}{ | m{8cm} | m{1cm}| m{2cm} | } 
\hline
\multicolumn{3}{|c|}{Punkter som skal godkjennes før en går videre på neste nivå} \\
	\hline
	Oppgave	& Utført & Signatur \\ 
	\hline
Lærer sjkker at det ikke er kortslutning på 24V forsyning og tre tilfeldige ledere & & \\ 
	\hline
\end{tabular}
\end{center}
\textbf{Vanlige feil:}
\begin{itemize}[noitemsep]
	\item 
\end{itemize}
\newpage
\subsection*{Arbeidsoppdrag 3 - Sluttkontroll (Nivå 3)}

I dette oppdraget skal du utføre en sluttkontroll at den elektriske installasjonen på maskinen. I mappe 05 sjekkliste for normer finner du sluttkontrollskjema for elektrisk utstyr på maskiner. Du må fylle ut dette skemaet (NB. du skal ikke utføre isolasjonsmåling. ). Under punktet funksjonsprøving må du utvide listen over alle funksner du mener skal testes. 

\begin{center}
\begin{tabular}{ | m{8cm} | m{1cm}| m{2cm} | } 
\hline
\multicolumn{3}{|c|}{Punkter som skal godkjennes før en går videre på neste nivå} \\
	\hline
	Oppgave	& Utført & Signatur \\ 
	\hline
Lærer går igjennom skjemaet og stiller eleven sprøsmål for å bekrefte at eleven skønner hva som er gjort. & & \\ 
	\hline
\end{tabular}
\end{center}
\textbf{Vanlige feil:}
\begin{itemize}[noitemsep]
	\item 
\end{itemize}
\newpage

\subsection*{Arbeidsoppdrag 4 - Tegning av skjema (Nivå 4)}

I dette oppdraget skal du tegne av skjema som du har bruk under montering. Skjemaet må leveres før arbeidsuken er ferdig (normalt fredag kl.14.15). 
\begin{center}
\begin{tabular}{ | m{8cm} | m{1cm}| m{2cm} | } 
\hline
\multicolumn{3}{|c|}{Punkter som skal godkjennes før en går videre på neste nivå} \\
	\hline
	Oppgave	& Utført & Signatur \\ 
	\hline
Ingen punkt& & \\ 
	\hline
\end{tabular}
\end{center}
\textbf{Vanlige feil:}
\begin{itemize}[noitemsep]
	\item 
\end{itemize}
\newpage

\underbar{file stasjonMal.tex}

\end{document}

\noindent
\underbar{file stasjon03}
\end{document}

