\input preamble.tex
\noindent
\section*{Stasjon 3 - Planlegging, gjennomføring  og dokumentasjon  av arbeid}

\vskip 5pt
%beskrivelse av oppgaven
I dette arbeidsoppdraget skal du planlegge, gjennomføre og dokumentere montering av et reguleringsanlegg. Da tiden til oppdraget er begrenset til 15  timer, vil det være utvalgte deler av jobben du skal gjøre. 

%kompetansemål som oppgaven dekker
Kompetansemål:
\begin{itemize}[noitemsep]

	\item utføre arbeid på automatiserte anlegg fagmessig, nøyaktig og i overensstemmelse med krav til helse, miljø og sikkerhet og rutiner for kvalitetssikring og internkontroll
	\item utføre risikovurdering og vurdere tiltak for ivaretakelse av person- og maskinsikkerhet
	\item vurdere hvilke regelverk og normer som gjelder for arbeidet som skal utføres og anvende dette
\end{itemize}

%anbefalt lesning til arbeidsoppdragene
Anbefalt lesning:

\begin{enumerate}
	\item ingen       
\end{enumerate}

%Liste over oppdrag som skal gjøres med ruter for godkjennening

\begin{center}
\begin{tabular}{ | m{10cm} | m{1cm}| m{2cm} | } 
\hline
\multicolumn{3}{|c|}{Liste over oppgaver som skal utføres} \\
	\hline
	Oppgave	& Utført & Signatur \\ 
	\hline
	\hline
	\cellcolor{green!60}(Nivå 1)Kobling med bakplate ute av skapet.	& & \\ 
	\hline
	\cellcolor{yellow!60}(Nivå 2)Koble reseten av anlegget. 	& & \\ 
	\hline
	\cellcolor{orange!60}(Nivå 3)Sluttkontroll	& & \\ 
	\hline
	\cellcolor{red!60}(Nivå 4)Dokumentasjon	& & \\ 
	\hline
\end{tabular}
\end{center}

%--------------------------------------
Arbeidsoppdrag kan deles inn i planlegging, gjennomføring og dokumentasjon.\\
For arbeidsoppdragene er det tenkt at dere skal bruke dette slik:\\\\
\textbf{Planlegging}\\
Gjøres før arbeidsuke starter. Da leser du gjennom anbefalt lesning, finner frem manualer og for  utstyret på stasjonen og setter deg inn i dette.\\ \\
\textbf{Gjennomføring}\\
Gjennomføres i aktuell arbeidsuke\\\\

\textbf{Dokumentasjon}\\
Gjennomføres i aktuell arbeidsuke, det vil være ulike dokumentasjonskrav til de forskjellige arbeidsoppdragene. Generelt vil det være en beskrivelse av arbeidet til lærer når oppdrag skal godkjennes. \\


% Detaljert beskrivelse av hvert arbeidsoppdrag
\newpage
\subsection*{Arbeidsoppdrag på Stasjon x}

\subsubsection*{Arbeidsoppdrag 1 -  emne (nivå 1)}

\begin{center} \begin{tabular}{ | m{8cm} | m{1cm}| m{2cm} | } 
\hline
\multicolumn{3}{|c|}{Punkter som skal godkjennes før en går videre på neste nivå} \\
	\hline
	Oppgave	& Utført & Signatur \\ 
	\hline
& & \\ 
	\hline
\end{tabular}
\end{center}

\textbf{Vanlige feil:}
\begin{itemize}[noitemsep]
	\item 
\end{itemize}
\newpage
\subsection*{Arbeidsoppdrag 2 - Fikse feil i PLC program}

\begin{center}
\begin{tabular}{ | m{8cm} | m{1cm}| m{2cm} | } 
\hline
\multicolumn{3}{|c|}{Punkter som skal godkjennes før en går videre på neste nivå} \\
	\hline
	Oppgave	& Utført & Signatur \\ 
	\hline
& & \\ 
	\hline
\end{tabular}
\end{center}
\textbf{Vanlige feil:}
\begin{itemize}[noitemsep]
	\item 
\end{itemize}
\newpage
\subsection*{Arbeidsoppdrag 3 - programmering av program for oppstart}

\begin{center}
\begin{tabular}{ | m{8cm} | m{1cm}| m{2cm} | } 
\hline
\multicolumn{3}{|c|}{Punkter som skal godkjennes før en går videre på neste nivå} \\
	\hline
	Oppgave	& Utført & Signatur \\ 
	\hline
& & \\ 
	\hline
\end{tabular}
\end{center}
\textbf{Vanlige feil:}
\begin{itemize}[noitemsep]
	\item 
\end{itemize}
\newpage

\subsection*{Arbeidsoppdrag 4 - Lage HMI bilde for styring}
\begin{center}
\begin{tabular}{ | m{8cm} | m{1cm}| m{2cm} | } 
\hline
\multicolumn{3}{|c|}{Punkter som skal godkjennes før en går videre på neste nivå} \\
	\hline
	Oppgave	& Utført & Signatur \\ 
	\hline
& & \\ 
	\hline
\end{tabular}
\end{center}
\textbf{Vanlige feil:}
\begin{itemize}[noitemsep]
	\item 
\end{itemize}
\newpage

\underbar{file stasjonMal.tex}

\end{document}

\noindent
\underbar{file stasjon03}
\end{document}

