\input preamble.tex
\noindent
\section*{Stasjon 04 - Gand reguleringsstasjon}

I dette arbeidsoppdraget skal du jobbe med optimalisering av ulike reguleringstrukturer. I de to første oppdragene skal du jobbe med stasjonen i tilbakekoblet modus for så å gå over til kaskadekoblet- og foroverkoblet regulering. 

\vskip 5pt
Kompetansemål:
idriftsette og optimalisere regulatorer basert på prosessbehov

Leselekse til uten du har stasjonen:

afgv.pdf (Closed-loop Control,Process dynamics and PID controller tuning, Basic process control strategies)

Dette arbeidsoppdraget består at følgende oppdrag:

\begin{center}
\begin{tabular}{ | m{10cm} | m{1cm}| m{2cm} | } 
\hline
\multicolumn{3}{|c|}{Liste over oppgaver som skal utføres} \\
	\hline
	Oppgave	& Utført & Signatur \\ 
	\hline
	\hline
	\cellcolor{green!60}(Nivå 1)Optimalisering med Tims og Z\&N (integrerende- og selvregulerende prosess)	& & \\ 
	\hline
	\cellcolor{yellow!60}(Nivå 2)Optimalisering med Skogestad (integrerende- og selvregulerende prosess)	& & \\ 
	\hline
	\cellcolor{orange!60}(Nivå 3)Optimalisering av kaskadekoblet regulering	& & \\ 
	\hline
	\cellcolor{red!60}(Nivå 4)Idriftsettelse av Foroverkoblet regulering	& & \\ 
	\hline
\end{tabular}
\end{center}

\subsection*{Arbeidsoppdrag på Stasjon 4}

\subsubsection*{Arbeidsoppdrag 1 - Optimalisering med Tims og Z\&N (nivå 1)}

I dette arbeidsoppdraget skal du optimalisere stasjonen i tilbakekoblet modus (TB på modusvelger). Ved hjelp av Tims roule og Z\&N skal du optimalisere prosessen med instillingene selvregulerende og integrerende

\subsubsection*{Arbeidsoppdrag 2 - Optimaliseringa med Skogestad (Nivå 2)}


I dette arbeidsoppdraget skal du optimalisere stasjonen i tilbakekoblet modus (TB på modusvelger). Ved hjelp av Skogestads metode skal du optimalisere prosessen med instillingene selvregulerende og integrerende

\underbar{file arb04}
\end{document}

