\input preamble.tex
\noindent
\section*{Stasjon 04 - Gand reguleringsstasjon}

I dette arbeidsoppdraget skal du jobbe med optimalisering av ulike reguleringstrukturer. I de to første oppdragene skal du jobbe med stasjonen i tilbakekoblet modus for så å gå over til kaskadekoblet- og foroverkoblet regulering. 

\vskip 5pt
Kompetansemål:
\begin{itemize}[noitemsep]
	\item idriftsette og optimalisere regulatorer basert på prosessbehov
\end{itemize}




Leselekse til uten du har stasjonen:

afgv.pdf (Closed-loop Control,Process dynamics and PID controller tuning, Basic process control strategies)

\begin{center}
\begin{tabular}{ | m{10cm} | m{1cm}| m{2cm} | } 
\hline
\multicolumn{3}{|c|}{Liste over oppgaver som skal utføres} \\
	\hline
	Oppgave	& Utført & Signatur \\ 
	\hline
	\hline
	\cellcolor{green!60}(Nivå 1)Optimalisering med Tims og Z\&N (integrerende- og selvregulerende prosess)	& & \\ 
	\hline
	\cellcolor{yellow!60}(Nivå 2)Optimalisering med Skogestad (integrerende- og selvregulerende prosess)	& & \\ 
	\hline
	\cellcolor{orange!60}(Nivå 3)Optimalisering av kaskadekoblet regulering	& & \\ 
	\hline
	\cellcolor{red!60}(Nivå 4)Idriftsettelse av Foroverkoblet regulering	& & \\ 
	\hline
\end{tabular}
\end{center}
\newpage
\subsection*{Arbeidsoppdrag 1 - Optimalisering med Tims og Z\&N (nivå 1)}

I dette arbeidsoppdraget skal du optimalisere stasjonen i tilbakekoblet modus (TB på modusvelger). Ved hjelp av Tims roule og Z\&N skal du optimalisere prosessen med instillingene selvregulerende og integrerende

\begin{center} \begin{tabular}{ | m{8cm} | m{1cm}| m{2cm} | } 
\hline
\multicolumn{3}{|c|}{Punkter som skal godkjennes før en går videre på neste nivå} \\
	\hline
	Oppgave	& Utført & Signatur \\ 
	\hline
Eleven beskriver hvordan han har gjort for å optimalisere i et word dokument. Dette dokumentet skal ligge i mappen som eleven personlig deler med kontaktlærer. Lærer ser igjennom og stiller spørsmål ut fra beskrivelsen.& & \\ 
	\hline
\end{tabular}
\end{center}

\textbf{Vanlige feil:}
\begin{itemize}[noitemsep]
	\item 
\end{itemize}
\newpage
\subsection*{Arbeidsoppdrag 2 - Optimaliseringa med Skogestad (Nivå 2)}

I dette arbeidsoppdraget skal du optimalisere stasjonen i tilbakekoblet modus (TB på modusvelger). Ved hjelp av Skogestads metode skal du optimalisere prosessen med instillingene selvregulerende og integrerende

\begin{center}
\begin{tabular}{ | m{8cm} | m{1cm}| m{2cm} | } 
\hline
\multicolumn{3}{|c|}{Punkter som skal godkjennes før en går videre på neste nivå} \\
	\hline
	Oppgave	& Utført & Signatur \\ 
	\hline
Eleven beskriver hvordan han har gjort for å optimalisere i et word dokument. Dette dokumentet skal ligge i mappen som eleven personlig deler med kontaktlærer. Lærer ser igjennom og stiller spørsmål ut fra beskrivelsen.& & \\ 
	\hline
\end{tabular}
\end{center}
\textbf{Vanlige feil:}
\begin{itemize}[noitemsep]
	\item 
\end{itemize}
\newpage
\subsection*{Arbeidsoppdrag 3 - emne (nivå 3)}

I dette oppdraget skal du sette stasjonen i kaskadekoblet modus (Midtstilling på velgerbryteren). Husk at når en optimaliserer en kaskadekoblet sløyfe må den indre sløyfen optimaliseres først. 

\begin{center}
\begin{tabular}{ | m{8cm} | m{1cm}| m{2cm} | } 
\hline
\multicolumn{3}{|c|}{Punkter som skal godkjennes før en går videre på neste nivå} \\
	\hline
	Oppgave	& Utført & Signatur \\ 
	\hline
Eleven beskriver hvordan han har gjort for å optimalisere i et word dokument. Dette dokumentet skal ligge i mappen som eleven personlig deler med kontaktlærer. Lærer ser igjennom og stiller spørsmål ut fra beskrivelsen.& & \\ 
	\hline
\end{tabular}
\end{center}
\textbf{Vanlige feil:}
\begin{itemize}[noitemsep]
	\item 
\end{itemize}
\newpage

\subsection*{Arbeidsoppdrag 4 - emne (nivå 4)}

I dette oppdraget skal du sette stasjonen i foroverkoblet modus (velgerbryteren til høyre). Du skal sette opp en foroverkoblingsfunksjon for et settpunkt på 50\%. Sett opp stasjonen uten foroverkobling først og ta opp en kurve over pådrag (MV) som er nødvendig med ulike typer belastning (sjekk flowmåler). Denne kurven skal du legge inn i geogebra og generere en formel for. Denne formelen skal du legge inni PLS programmet. 
\begin{center}
\begin{tabular}{ | m{8cm} | m{1cm}| m{2cm} | } 
\hline
\multicolumn{3}{|c|}{Punkter som skal godkjennes før en går videre på neste nivå} \\
	\hline
	Oppgave	& Utført & Signatur \\ 
	\hline
Eleven beskriver hvordan han har gjort for å optimalisere i et word dokument. Dette dokumentet skal ligge i mappen som eleven personlig deler med kontaktlærer. Lærer ser igjennom og stiller spørsmål ut fra beskrivelsen.& & \\ 
	\hline
\end{tabular}
\end{center}
\textbf{Vanlige feil:}
\begin{itemize}[noitemsep]
	\item 
\end{itemize}
\newpage

\underbar{file stasjon04}
\end{document}

