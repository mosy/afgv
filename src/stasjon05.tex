\input preamble.tex
\noindent
\section*{Stasjon x - Navn på stasjon}

\vskip 5pt
%beskrivelse av oppgaven
I dette arbeidsoppgaver skal du koble opp ulike sikkerhetsfunksjoner som et sikkerhetsrelatert styresystem normålt utfører. I det to første oppgavenen skal du bruke et sikkerhetsrele og i de to siste skal du bruke en sikkerhets PLS. 

%kompetansemål som oppgaven dekker
Kompetansemål:
\begin{itemize}[noitemsep]

	\item utføre risikovurdering og vurdere tiltak for ivaretakelse av person- og maskinsikkerhet
	\item vurdere hvilke regelverk og normer som gjelder for arbeidet som skal utføres og anvende dette
	\item planlegge, programmere, montere og idriftsette programmerbare styresystemer
	\item bruke gjeldende regelverk og normer for elektriske installasjoner på maskiner
\end{itemize}

%anbefalt lesning til arbeidsoppdragene
Anbefalt lesning:

\begin{enumerate}
	\item afgv.pdf/ 
\end{enumerate}

%Liste over oppdrag som skal gjøres med ruter for godkjennening

\begin{center}
\begin{tabular}{ | m{10cm} | m{1cm}| m{2cm} | } 
\hline
\multicolumn{3}{|c|}{Liste over oppgaver som skal utføres} \\
	\hline
	Oppgave	& Utført & Signatur \\ 
	\hline
	\cellcolor{green!60}(Nivå 1) Start/Stopp med dobbelstløyfe nødstopp &&\\
	\hline
	\cellcolor{yellow!60}(Nivå 2) Dørbryter med auto/manuell reset & & \\ 
	\hline
	\cellcolor{orange!60}(Nivå 3) Sikkerhets-PLS: utføre tutorial for tohåndsstart	& & \\ 
	\hline
	\cellcolor{red!60}(Nivå 4) Sikkerhets-PLS med auto/manuell reset og nødstopp. (dobbelsllyfe)	& & \\ 
	\hline
\end{tabular}
\end{center}

%--------------------------------------
Arbeidsoppdrag kan deles inn i planlegging, gjennomføring og dokumentasjon.\\
For arbeidsoppdragene er det tenkt at dere skal bruke dette slik:\\\\
\textbf{Planlegging}\\
Gjøres før arbeidsuke starter. Da leser du gjennom anbefalt lesning, finner frem manualer og for  utstyret på stasjonen og setter deg inn i dette.\\ \\
\textbf{Gjennomføring}\\
Gjennomføres i aktuell arbeidsuke\\\\

\textbf{Dokumentasjon}\\
Gjennomføres i aktuell arbeidsuke, det vil være ulike dokumentasjonskrav til de forskjellige arbeidsoppdragene. Generelt vil det være en beskrivelse av arbeidet til lærer når oppdrag skal godkjennes. \\


% Detaljert beskrivelse av hvert arbeidsoppdrag
\newpage
\subsection*{Arbeidsoppdrag på Stasjon x}

\subsubsection*{Arbeidsoppdrag 1 -  emne (nivå 1)}

\begin{center} \begin{tabular}{ | m{8cm} | m{1cm}| m{2cm} | } 
\hline
\multicolumn{3}{|c|}{Punkter som skal godkjennes før en går videre på neste nivå} \\
	\hline
	Oppgave	& Utført & Signatur \\ 
	\hline
& & \\ 
	\hline
\end{tabular}
\end{center}

\textbf{Vanlige feil:}
\begin{itemize}[noitemsep]
	\item 
\end{itemize}
\newpage
\subsection*{Arbeidsoppdrag 2 - emne (nivå 2)}

\begin{center}
\begin{tabular}{ | m{8cm} | m{1cm}| m{2cm} | } 
\hline
\multicolumn{3}{|c|}{Punkter som skal godkjennes før en går videre på neste nivå} \\
	\hline
	Oppgave	& Utført & Signatur \\ 
	\hline
& & \\ 
	\hline
\end{tabular}
\end{center}
\textbf{Vanlige feil:}
\begin{itemize}[noitemsep]
	\item 
\end{itemize}
\newpage
\subsection*{Arbeidsoppdrag 3 - emne (nivå 3)}

\begin{center}
\begin{tabular}{ | m{8cm} | m{1cm}| m{2cm} | } 
\hline
\multicolumn{3}{|c|}{Punkter som skal godkjennes før en går videre på neste nivå} \\
	\hline
	Oppgave	& Utført & Signatur \\ 
	\hline
& & \\ 
	\hline
\end{tabular}
\end{center}
\textbf{Vanlige feil:}
\begin{itemize}[noitemsep]
	\item 
\end{itemize}
\newpage

\subsection*{Arbeidsoppdrag 4 - emne (nivå 4)}
\begin{center}
\begin{tabular}{ | m{8cm} | m{1cm}| m{2cm} | } 
\hline
\multicolumn{3}{|c|}{Punkter som skal godkjennes før en går videre på neste nivå} \\
	\hline
	Oppgave	& Utført & Signatur \\ 
	\hline
& & \\ 
	\hline
\end{tabular}
\end{center}
\textbf{Vanlige feil:}
\begin{itemize}[noitemsep]
	\item 
\end{itemize}
\newpage

\underbar{file stasjon05.tex}

\end{document}

