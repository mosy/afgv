\input preamble.tex
\noindent
\section*{Stasjon 5 - Sikkerhetsrelatierte deler av styresystemer}

\vskip 5pt
%beskrivelse av oppgaven
I dette arbeidsoppgaver skal du koble opp ulike sikkerhetsfunksjoner som et sikkerhetsrelatert styresystem normålt utfører. I det to første oppgavenen skal du bruke et sikkerhetsrele og i de to siste skal du bruke en sikkerhets PLS. 

\vskip 5pt 
%kompetansemål som oppgaven dekker
Kompetansemål:
\begin{itemize}[noitemsep]

	\item utføre risikovurdering og vurdere tiltak for ivaretakelse av person- og maskinsikkerhet
	\item planlegge, programmere, montere og idriftsette programmerbare styresystemer
	\item bruke gjeldende regelverk og normer for elektriske installasjoner på maskiner
\end{itemize}

%anbefalt lesning til arbeidsoppdragene
Anbefalt lesning:

\begin{enumerate}
	\item afgv.pdf/ 
\end{enumerate}

%Liste over oppdrag som skal gjøres med ruter for godkjennening

\begin{center}
\begin{tabular}{ | m{10cm} | m{1cm}| m{2cm} | } 
\hline
\multicolumn{3}{|c|}{Liste over oppgaver som skal utføres} \\
	\hline
	Oppgave	& Utført & Signatur \\ 
	\hline
	\cellcolor{green!60}(Nivå 1) Start/Stopp med dobbelstløyfe nødstopp &&\\
	\hline
	\cellcolor{yellow!60}(Nivå 2) Dørbryter med auto/manuell reset & & \\ 
	\hline
	\cellcolor{orange!60}(Nivå 3) Sikkerhets-PLS: utføre tutorial for tohåndsstart	& & \\ 
	\hline
	\cellcolor{red!60}(Nivå 4) Dørbryter med pre-reset & & \\ 
	\hline
\end{tabular}
\end{center}

%--------------------------------------
Arbeidsoppdrag kan deles inn i planlegging, gjennomføring og dokumentasjon.\\
For arbeidsoppdragene er det tenkt at dere skal bruke dette slik:\\\\
\textbf{Planlegging}\\
Gjøres før arbeidsuke starter. Da leser du gjennom anbefalt lesning, finner frem manualer og for  utstyret på stasjonen og setter deg inn i dette.\\ \\
\textbf{Gjennomføring}\\
Gjennomføres i aktuell arbeidsuke\\\\

\textbf{Dokumentasjon}\\
Gjennomføres i aktuell arbeidsuke, det vil være ulike dokumentasjonskrav til de forskjellige arbeidsoppdragene. Generelt vil det være en beskrivelse av arbeidet til lærer når oppdrag skal godkjennes. \\


% Detaljert beskrivelse av hvert arbeidsoppdrag
\newpage

\subsection*{Arbeidsoppdrag 1 -  Start/stopp med dobbeltsløyfe nødstopp(nivå 1)}
I dette oppdraget skal du koble opp start og stopp av en direkte startet motor. Nødstoppen skal bruke dobbeltsløyfe (two channel) og ha automatisk tilbakestilling (Automatic reset)
\begin{center} \begin{tabular}{ | m{8cm} | m{1cm}| m{2cm} | } 
\hline
\multicolumn{3}{|c|}{Punkter som skal godkjennes før en går videre på neste nivå} \\
	\hline
	Oppgave	& Utført & Signatur \\ 
	\hline
Lærer sjkker at nødstopp motoren og at dobbeltsløyfe virker som den skal& & \\ 
	\hline
\end{tabular}
\end{center}

\textbf{Vanlige feil:}
\begin{itemize}[noitemsep]
	\item 
\end{itemize}
\newpage
\subsection*{Arbeidsoppdrag 2 - emne (nivå 2)}
I dette oppdraget skal du utvide koblingen din med en dørbryter(grindbryter). I dette oppdraget skal anegget ha manuell reset. 
\begin{center}
\begin{tabular}{ | m{8cm} | m{1cm}| m{2cm} | } 
\hline
\multicolumn{3}{|c|}{Punkter som skal godkjennes før en går videre på neste nivå} \\
	\hline
	Oppgave	& Utført & Signatur \\ 
	\hline
Lærer sjkker at nødstopp og dørbryter stopper motoren og at dobbeltsløyfe virker som den skal& & \\ 
	\hline
\end{tabular}
\end{center}
\textbf{Vanlige feil:}
\begin{itemize}[noitemsep]
	\item 
\end{itemize}
\newpage
\subsection*{Arbeidsoppdrag 3 - Tohåndsstart(nivå 3)}
I dette oppdraget skal du følge tutorialen som ligger i manual mappern for stasjonen. 
\begin{center}
\begin{tabular}{ | m{8cm} | m{1cm}| m{2cm} | } 
\hline
\multicolumn{3}{|c|}{Punkter som skal godkjennes før en går videre på neste nivå} \\
	\hline
	Oppgave	& Utført & Signatur \\ 
	\hline
Eleven viser PLS program for sikkerhet PLS-en og demonstrerer at tohåndsstart virker. & & \\ 
	\hline
\end{tabular}
\end{center}
\textbf{Vanlige feil:}
\begin{itemize}[noitemsep]
	\item 
\end{itemize}
\newpage

\subsection*{Arbeidsoppdrag 4 - Dørbryter med pre-reset i sikkerhets PLS(nivå 4)}

I dette oppdraget skal du koble opp en dørbryter med pre-reset. Det vil si at det skal være to resetbrytere, en som er innenfor grinden og som gir oversikt for farlig område. Den andre reseten skal stå utenfor grinden. \\
Anlegget skal virke slik at når en trykke på pre-reset har en valgbar tid (eks. 20s) på seg for å lukke grinden og trykke reset. 
\begin{center}
\begin{tabular}{ | m{8cm} | m{1cm}| m{2cm} | } 
\hline
\multicolumn{3}{|c|}{Punkter som skal godkjennes før en går videre på neste nivå} \\
	\hline
	Oppgave	& Utført & Signatur \\ 
	\hline
Lærer sjekker program i sikkerhets-PLS og funksjons på anlegLærer sjekker program i sikkerhets-PLS og funksjons på anleggg& & \\ 
	\hline
\end{tabular}
\end{center}
\textbf{Vanlige feil:}
\begin{itemize}[noitemsep]
	\item 
\end{itemize}
\newpage

\underbar{file stasjon05.tex}

\end{document}

