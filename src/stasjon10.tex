\input preamble.tex
\noindent
\section*{Stasjon 10 - Trykkbenk}

\vskip 5pt
%beskrivelse av stasjonen 
%kompetansemål som oppgaven dekker
Anbefalt lesning:

\begin{enumerate}
	\item Ingen
\end{enumerate}



%Liste over oppdrag som skal gjøres med ruter for godkjennening

% Detaljert beskrivelse av hvert arbeidsoppdrag
\section{Arbeidsoppdrag på Stasjon 10}

\subsection{Arbeidsoppdrag - Justering av Danfoss Trykkbryter}

Målet med oppgaven er å lære hvordan vi justerer en pressostat

Utstyr:
\begin{itemize}[noitemsep]
	\item Trykkbryter danfoss
	\item trykkregulator og manometer
	\item multimeter
\end{itemize}


Oppgaver \begin{enumerate}
	\item Utført en As found kalibrering 
	\item Les vedlagt bruksanvisning for å sette dere inn i virkemåten til utstyret.
	\item Vi ønsker å bruke pressostaten til å styre tanktrykket i et hydroforanlegg. Maksimalt pumpetrykk skal være 6 barg og laveste trykke 5 barg. 
	\item Bruk instruksjons arket til å justere pressostaten.
	\item Utfør en As left kalibrering 
	\item Beskriv hvordan du planla, gjennomførte og dokumentere jobben. Forklar hvordan en slik pressostat kan nyttes til en pumpe eller kompressor. Forklar eventuelle avvik dere måtte observere under forsøket. 
\end{enumerate}

\subsection{Arbeidsoppdrag - Justering av 4-20mA DP-Celle}

Målet med oppgaven er å lære hvordan vi justerer en transmitter for trykk.


Utstyr:
\begin{itemize}[noitemsep]
	\item analog DP-celle 
	\item Utstyr for å trykksette og å måle trykk 
	\item +24VDC strøforsyning 
	\item multimeter/sløyfekalibrator
\end{itemize}

Oppgaver \begin{enumerate}
	\item Utfør en As found kalibrering. Tegn opp en skisse av oppkoblingen med alle komponentene med.
	\item Måleområdets laveste trykk inn velges til 0 barg og høyeste trykk velges til 7 barg.
	\item Området deles opp i 4-5 punkter og kalibreringssertefikat fylles ut med stigende og fallende verdier for transmitteren før dere justerer på den. Det er viktig å justere riktig veg når dere skal lese av, (dvs ved stigende verdier må du ikke stille for høyt og så gå tilbake). 
	\item Etter verdiene før kalibrering er fylt ut justerer dere transmitteren slik at den viser riktig på laveste og høyeste trykk. (4 mA ved laveste og 20 mA ved høyeste verdi).
	\item Deretter fylles kolonnene for verdier etter kalibrering ut.
	\item Ut fra dataene i kalibreringssertefikatet tegner dere en graf som viser sammenhengen mellom inngangssignal og utgangssignal.
	\item Lag en kort beskrivelse av hvordan måleverdiomformeren er bygget opp og hvilket måleprinsipp som er benyttet i denne omformeren.
	\item Utfør en As left kalibrering 
\end{enumerate}



\subsection{Arbeidsoppdrag - Justering av av/på og range for trykkbryter}

Utstyr:
\begin{itemize}[noitemsep]
	\item Trykktransmitter med HART kommunikasjon 
	\item Utstyr for å trykksette og å måle trykk 
	\item HART-communicator 
	\item +24VDC strømforsyning 
\end{itemize}

Oppgaver \begin{enumerate}
	\item		Utført en As found kalibrering 
	\item		Bruk HART-communitatoren til å justere transmitteren slik at den gir 4 mA ved 1 barg og 20mA ved 6 barg 
	\item		Utfør en As left kalibrering 
\end{enumerate}

\underbar{file stasjon10.tex}

\end{document}

