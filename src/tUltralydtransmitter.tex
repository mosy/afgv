\input preamble.tex
\noindent
\section*{Stasjon x - Navn på stasjon}

\vskip 5pt
%beskrivelse av oppgaven
Beskrivelse av stasjonen

%kompetansemål som oppgaven dekker
Kompetansemål:
\begin{itemize}[noitemsep]

	\item utføre arbeid på automatiserte anlegg fagmessig, nøyaktig og i overensstemmelse med krav til helse, miljø og sikkerhet og rutiner for kvalitetssikring og internkontroll
	\item utføre risikovurdering og vurdere tiltak for ivaretakelse av person- og maskinsikkerhet
	\item vurdere hvilke regelverk og normer som gjelder for arbeidet som skal utføres og anvende dette
	\item planlegge, utføre, vurdere kvalitet, sluttkontrollere og dokumentere arbeidet
	\item planlegge, programmere, montere og idriftsette programmerbare styresystemer
	\item endre og tilpasse skjermbilder for grensesnitt mellom menneske og maskin
	\item anvende ulike elektroniske kommunikasjonssystemer i automatiserte anlegg
	\item vurdere datasikkerhet i automatiserte anlegg
	\item tegne, lese og forklare instrumenterte prosessflytskjemaer og bruke annen relevant dokumentasjon for automatiserte anlegg
	\item montere, konfigurere, kalibrere og idriftsettelse digitale og analoge målesystemer
	\item idriftsette og optimalisere regulatorer basert på prosessbehov
	\item montere og idriftsette ulike typer pådragsorganer med tilhørende forstillingselementer og hjelpeutstyr
	\item programmere, idriftsette samt gjøre rede for roboters funksjon og anvendelse i produksjonsanlegg
	\item måle fysiske størrelser i automatiserte anlegg
	\item feilsøke og rette feil i automatiserte anlegg
	\item bruke gjeldende regelverk og normer for elektriske installasjoner på maskiner
	\item bruke gjeldende regelverk og normer for installasjon av elektroniske kommunikasjonssystemer
	\item beskrive ulike vedlikeholdssystemer og -rutiner knyttet til automatiserte anlegg, og anvende et av disse
	\item redegjøre for bedriftens organisasjonsoppbygging og bedriftens verdiskapning i et samfunnsperspektiv
	\item dokumentere egen opplæring i automatiseringssystemer
\end{itemize}

%anbefalt lesning til arbeidsoppdragene
Anbefalt lesning:

\begin{enumerate}
	\item afgv.pdf/ 
\end{enumerate}

%Liste over oppdrag som skal gjøres med ruter for godkjennening

\begin{center}
\begin{tabular}{ | m{10cm} | m{1cm}| m{2cm} | } 
\hline
\multicolumn{3}{|c|}{Liste over oppgaver som skal utføres} \\
	\hline
	Oppgave	& Utført & Signatur \\ 
	\hline
	\hline
	\cellcolor{green!60}(Nivå 1)	& & \\ 
	\hline
	\cellcolor{yellow!60}(Nivå 2)	& & \\ 
	\hline
	\cellcolor{orange!60}(Nivå 3)	& & \\ 
	\hline
	\cellcolor{red!60}(Nivå 4)	& & \\ 
	\hline
\end{tabular}
\end{center}

%--------------------------------------
Arbeidsoppdrag kan deles inn i planlegging, gjennomføring og dokumentasjon.\\
For arbeidsoppdragene er det tenkt at dere skal bruke dette slik:\\\\
\textbf{Planlegging}\\
Gjøres før arbeidsuke starter. Da leser du gjennom anbefalt lesning, finner frem manualer og for  utstyret på stasjonen og setter deg inn i dette.\\ \\
\textbf{Gjennomføring}\\
Gjennomføres i aktuell arbeidsuke\\\\

\textbf{Dokumentasjon}\\
Gjennomføres i aktuell arbeidsuke, det vil være ulike dokumentasjonskrav til de forskjellige arbeidsoppdragene. Generelt vil det være en beskrivelse av arbeidet til lærer når oppdrag skal godkjennes. \\


% Detaljert beskrivelse av hvert arbeidsoppdrag
\newpage

\subsection*{Arbeidsoppdrag 1 -  emne (nivå 1)}

\begin{center} \begin{tabular}{ | m{8cm} | m{1cm}| m{2cm} | } 
\hline
\multicolumn{3}{|c|}{Punkter som skal godkjennes før en går videre på neste nivå} \\
	\hline
	Oppgave	& Utført & Signatur \\ 
	\hline
& & \\ 
	\hline
\end{tabular}
\end{center}

\textbf{Vanlige feil:}
\begin{itemize}[noitemsep]
	\item 
\end{itemize}
\newpage
\subsection*{Arbeidsoppdrag 2 - emne (nivå 2)}

\begin{center}
\begin{tabular}{ | m{8cm} | m{1cm}| m{2cm} | } 
\hline
\multicolumn{3}{|c|}{Punkter som skal godkjennes før en går videre på neste nivå} \\
	\hline
	Oppgave	& Utført & Signatur \\ 
	\hline
& & \\ 
	\hline
\end{tabular}
\end{center}
\textbf{Vanlige feil:}
\begin{itemize}[noitemsep]
	\item 
\end{itemize}
\newpage
\subsection*{Arbeidsoppdrag 3 - emne (nivå 3)}

\begin{center}
\begin{tabular}{ | m{8cm} | m{1cm}| m{2cm} | } 
\hline
\multicolumn{3}{|c|}{Punkter som skal godkjennes før en går videre på neste nivå} \\
	\hline
	Oppgave	& Utført & Signatur \\ 
	\hline
& & \\ 
	\hline
\end{tabular}
\end{center}
\textbf{Vanlige feil:}
\begin{itemize}[noitemsep]
	\item 
\end{itemize}
\newpage

\subsection*{Arbeidsoppdrag 4 - emne (nivå 4)}
\begin{center}
\begin{tabular}{ | m{8cm} | m{1cm}| m{2cm} | } 
\hline
\multicolumn{3}{|c|}{Punkter som skal godkjennes før en går videre på neste nivå} \\
	\hline
	Oppgave	& Utført & Signatur \\ 
	\hline
& & \\ 
	\hline
\end{tabular}
\end{center}
\textbf{Vanlige feil:}
\begin{itemize}[noitemsep]
	\item 
\end{itemize}
\newpage

\underbar{file stasjonMal.tex}

\end{document}

